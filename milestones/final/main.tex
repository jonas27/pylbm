\documentclass[a4paper,11pt]{report}
\usepackage[T1]{fontenc}
\usepackage[utf8]{inputenc}
\usepackage{lmodern}

\usepackage{hyperref}
\usepackage{graphicx}
\usepackage[english]{babel}

\usepackage{graphicx}
\usepackage{amsmath}

\usepackage{listings} % package for listing parts of code

\renewcommand*\footnoterule{}

\makeatletter
\renewcommand{\@chapapp}{}% Not necessary...
\newenvironment{chapquote}[2][2em]
  {\setlength{\@tempdima}{#1}%
   \def\chapquote@author{#2}%
   \parshape 1 \@tempdima \dimexpr\textwidth-2\@tempdima\relax%
   \itshape}
  {\par\normalfont\hfill--\ \chapquote@author\hspace*{\@tempdima}\par\bigskip}
\makeatother




% Book's title and subtitle
\title{\Huge \textbf{High Performance Computing with Python} \vspace{4mm} \\ \huge Final Report}
% Author
% \author{\textsc{First-name Last-name}\footnote{email address}}
\author{\textsc{Jonas Manser} \\ \vspace{3mm}\text{matriculation number}  \\
\vspace{3mm}\text{jonas.burster@gmail.com}}


\begin{document}

\makeatletter
\begin{titlepage}
    \begin{center}
        \includegraphics[width=0.5\linewidth]{logos/Uni_Logo-Grundversion_E1_A4_CMYK.eps}\\[4ex]
        {\huge \bfseries  \@title }\\[2ex]
        {\LARGE  \@author}\\[30ex]
        {\large \@date}
    \end{center}
\end{titlepage}
\makeatother
\thispagestyle{empty}
\newpage



\tableofcontents


\section{Introduction}
The Lattice Boltzmann Method (LBM) is a numerical solution of (nonlinear) partial differential equations.
It is used to simulate fluid flows in a closed system and is based on the core assumption that fluids can be approximated to particles on a lattice.

It originated from the lattice gas automata (LGA) pioneered by Hardy, Pomeau and de Pazzis in the 1970s with the HPP-model \cite{hardy1973timeHPP}.
This model could be used to simluate both gas and fluid, but did not not, as initially hoped by by the authors, lead to the Navier-Stokes equation in the macroscopic limit.
Later lattice gas automata models like the FPH-model \cite{PhysRevLett.56.1505-fhp} were able to satisfy the Navier-Stokes equation but were still plagued by many problens, like the lack of Galilean invariance\cite{nie2008galileanInvariance} or the strong assumption that at each grid point, each discrete velocity $c_i$ may be taken by just one or no particle\cite{IIISELEC71:online-univie}, which lead to massive compyting requirements.
In 1988 McNamara and Zanetti introduce the LBM as a direct alternative to the LGA \cite{mcnamara1988boltzmann-method}.
Their new method "is based on the simulation of a very simple microscopic system, rather than on the direct integration of partial differential equations" \cite{mcnamara1988boltzmann-method}.
Because of their close similarity the LBM shares many features with the LGA, like the lack of Galilean invariance.
But whereas in LGA statistical averaging to compute the velocity in LGA, in LBM "we directly study the time evolution of the mean values"\cite{mcnamara1988boltzmann-method}.

The key points of the LBM success is it's simplicity, relatively low consumption of computing resources and easy parallelization of the algorithm.
Unlike other computational fluid dynamics (CFD) method it does not numerically solve the macroscopic properties of a fluied, i.e. the mass, momentum, and energy.
But rather, approximates the fluid to particles on a grid and uses the streaming step and the collision step to simulate the particles behaviour over time.
Using particles also makes incorperating boundries and microscopic interactions easier than in most other CFD models.

\section{Methods}


\section{Boltzmann Transport Equation}
At core of the LBM stands the Boltzmann transport equation (BTE) published by Ludwig Boltzmann in 1872, which describes the statistical behaviour of particles in a closed system.

% The equation arises not by analyzing the individual positions and momenta of each particle in the fluid but rather by considering a probability distribution for the position and momentum of a typical particle—that is, the probability that the particle occupies a given very small region of space (mathematically the volume element {\displaystyle \mathrm {d} ^{3}{\bf {r}}}{\displaystyle \mathrm {d} ^{3}{\bf {r}}}) centered at the position {\displaystyle {\bf {r}}}{\displaystyle {\bf {r}}}, and has momentum nearly equal to a given momentum vector {\displaystyle {\bf {p}}}{\displaystyle {\bf {p}}} (thus occupying a very small region of momentum space {\displaystyle \mathrm {d} ^{3}{\bf {p}}}{\displaystyle \mathrm {d} ^{3}{\bf {p}}}), at an instant of time.

The


\section{Model Properties}
https://de.wikipedia.org/wiki/FHP-Modell
Teilchen existieren immer nur auf den Gitterpunkten, d. h. nie auf den Kanten und nie auf den Flächen.
Jedem Teilchen ist eine Richtung (von einem Gitterpunkt zu einem anderen unmittelbar benachbarten) zugeordnet.
Ein Gitterpunkt kann für jede Richtung maximal ein Teilchen, d. h. insgesamt zwischen null und sechs Teilchen enthalten.
Die Teilchen werden rundenweise vorwärts bewegt. Zwischen den Zügen wird jeweils überprüft, ob es an einem Gitterpunkt zu einer Streuung kommt.



\bibliographystyle{unsrt}
\bibliography{biblio}
% \bibliography{"/home/joe/repos/pylbm/milestones/final/biblio.bib"}

\end{document}
