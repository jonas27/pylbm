\documentclass[a4paper,11pt]{report}
\usepackage[T1]{fontenc}
\usepackage[utf8]{inputenc}
\usepackage{lmodern}

\usepackage{hyperref}
\usepackage{graphicx}
\usepackage[english]{babel}

\usepackage{graphicx}
\usepackage{amsmath}

\usepackage{listings} % package for listing parts of code

\renewcommand*\footnoterule{}

\makeatletter
\renewcommand{\@chapapp}{}% Not necessary...
\newenvironment{chapquote}[2][2em]
  {\setlength{\@tempdima}{#1}%
   \def\chapquote@author{#2}%
   \parshape 1 \@tempdima \dimexpr\textwidth-2\@tempdima\relax%
   \itshape}
  {\par\normalfont\hfill--\ \chapquote@author\hspace*{\@tempdima}\par\bigskip}
\makeatother




% Book's title and subtitle
\title{\Huge \textbf{High Performance Computing with Python} \vspace{4mm} \\ \huge Final Report}
% Author
% \author{\textsc{First-name Last-name}\footnote{email address}}
\author{\textsc{First-name Last-name} \\ \vspace{3mm}\text{matriculation number}  \\
\vspace{3mm}\text{email-address}}


\begin{document}

\makeatletter
\begin{titlepage}
    \begin{center}
        \includegraphics[width=0.5\linewidth]{logos/Uni_Logo-Grundversion_E1_A4_CMYK.eps}\\[4ex]
        {\huge \bfseries  \@title }\\[2ex]
        {\LARGE  \@author}\\[30ex]
        {\large \@date}
    \end{center}
\end{titlepage}
\makeatother
\thispagestyle{empty}
\newpage



\tableofcontents

\section{Introduction}
The Lattice Boltzmann Method (LBM) is used to simulate fluid flows in a closed system and is based on the core assumptions that fluids can be approximated to particles on a lattice.
It originated from the lattice gas automata (LGCA) pioneered by Hardy–Pomeau–de Pazzis and Frisch–Hasslacher–Pomeau.

a fluid density on a lattice is simulated with streaming and collision (relaxation) processes
core it utalizes the Boltzmann transport equation (BTE) published by Ludwig Boltzmann in 1872 describes the statistical behaviour of particles in a closed system.

The


\section{Model Properties}
https://de.wikipedia.org/wiki/FHP-Modell
Teilchen existieren immer nur auf den Gitterpunkten, d. h. nie auf den Kanten und nie auf den Flächen.
Jedem Teilchen ist eine Richtung (von einem Gitterpunkt zu einem anderen unmittelbar benachbarten) zugeordnet.
Ein Gitterpunkt kann für jede Richtung maximal ein Teilchen, d. h. insgesamt zwischen null und sechs Teilchen enthalten.
Die Teilchen werden rundenweise vorwärts bewegt. Zwischen den Zügen wird jeweils überprüft, ob es an einem Gitterpunkt zu einer Streuung kommt.









\bibliographystyle{unsrt}
\bibliography{biblio}

\end{document}
