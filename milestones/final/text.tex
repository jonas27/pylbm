
\chapter{Introduction}
The Lattice Boltzmann Method (LBM) is a numerical solution of (nonlinear) partial differential equations of the original BLT introduced in 1988 by McNamara and Zanetti~\cite{mcnamara1988boltzmann-method}.
It is used to simulate liquids\footnote{I will focus exclusively on liquids, but most aspects in the report hold for some gases as well.} in a closed system and is based on the core assumption that their flows can be approximated to particles on a lattice.
To simulate the correct flow of liquids a model must lead to the the Navier–Stokes equations in the macoscopic limit.
The Navier–Stokes equations are partial differential equations to describe the motion of viscous fluid substances, named after the physicists Claude-Louis Navier and George Gabriel Stokes.
It has been shown that the LBM does indeed fulfill these equations for incompressible subsonic flows of fluids.
Today, the LBM is used in a wide variety of fields from car aerodynamics to ocean current flows.

The LBM originates from the lattice gas automata (LGA) pioneered by Hardy, Pomeau and de Pazzis in the 1970s with the HPP-model~\cite{hardy1973timeHPP}.
This model could be used to simluate both gas and fluid flows, but did not not, as initially hoped by the authors, lead to the Navier-Stokes equation in the macroscopic limit.
Later lattice gas automata models like the FPH-model~\cite{PhysRevLett.56.1505-fhp} were able to satisfy the Navier-Stokes equation but were still plagued by many problens, like the lack of Galilean invariance~\cite{nie2008galileanInvariance} or the strong assumption
that each node is surrounded by discrete particle cells, which resulted in massive computing requirements.
It also assumed that streaming and collision happened synchronously for all nodes and thus the collision was non-deterministic.

In 1988 McNamara and Zanetti introduced the LBM as a direct alternative to the LGA~\cite{mcnamara1988boltzmann-method}.
Their new method "is based on the simulation of a very simple microscopic system, rather than on the direct integration of partial differential equations"~\cite{mcnamara1988boltzmann-method}.
Because of their close similarity, the LBM shares many features with the LGA, like the lack of Galilean invariance and both satisfy the Navier-Stokes equation in the macroscopic limit.
It, crucially, "directly stud[ies] the time evolution of the mean values"~\cite{mcnamara1988boltzmann-method} and thus does not need statistical averaging to compute the velocity, as is the case in LGA, leading to lower computing requirements.

The key points of the LBM success is its simplicity, relatively low consumption of computing requirements and easy parallelization of the algorithm.
This is achieved by approximating the fluid to particles on a grid and using a separate streaming and collision step to simulate the particles behaviour over time.
This is unlike other computational fluid dynamics (CFD) methods which directly solve the numerically macroscopic properties of a fluied, i.e. the mass, momentum, and energy.
Simulating the particles themselves also makes incorperating boundries and microscopic interactions easier than in most other CFD models.

The reminder of the report is structed in the follwing way: The second chapter introduce the theoretical background of the LBM. Chapter three introduces the most important implementations and chapter four shows the results of the milestones.
Afterwards I will conclude and give a future outlook.

\chapter{Theoretical Background}
The probability density function is used to describe the state of particles in a closed system and the Boltzmann transport equation tracks the time evolution of this function.
Combining these methods and discretizing them over a lattice leads to the LBM.
In this section I will discuss each individual component in greater detail.

\section{Probability Density Function}
The probability density function (PDF) describes the statistical probability of finding particles in a closed system not in equilibrium and is denoted by $f$ in the implementation.
It thereby replaces tagging each particle, as in molecular dynamic simulations.
The PDF is given by $f(r_i,v_i,t)$ where $r$ are the physical positions, $v$ the velocities and $t$ the time.
The probability for finding a particle in a certain part of the phase space is then given by equation~\ref{eq:pdf}.
\begin{equation}
  \label{eq:pdf}
  \begin{aligned}
    dP = f(r,v,t) d^{3}r d^{3}v
  \end{aligned}
\end{equation}
The phase space for equation \ref{eq:pdf} is given by $[r, r+dr, v, v+dv]$.
Thus, the probability for finding a particle in the phase space at position $r_i$ is only depended on the velocity $v_i$ and time $t$.

The probability of finding a single particle with an arbitraty place $r$ and an arbitrary velocity $v$ in the entire phase space is given by equation \ref{eq:prob-single}.
\begin{equation}
  \label{eq:prob-single}
  \begin{aligned}
    P = \int_{\Omega_{r}} \int_{\Omega_{\vec{v}}}  f(r,v,t) d^{3}r d^{3}v \quad \overset{!}{=} 1
  \end{aligned}
\end{equation}
Assuming the fluid is in a closed system where no particles are added or removed, it must hold that equation \href{eq:prob-single} is equal to one, i.e. normalized to unity. 
This means the chance to find an arbitrary particle in an arbitrary location is equal to one.

This then leads to the general form of the $i$-th moments of $ f(r,v,t) $ w.r.t. $ v $ shown in equation \ref{eq:general-moments}.
\begin{equation}
  \label{eq:general-moments}
  \begin{aligned}
    \mu_{i}(r) =\int_{\Omega_{v}}v^{i} f(r,v,t)d^{3}v
  \end{aligned}
\end{equation}
The first and second order moments are the main subjects of this project and can be readily interpreted.
The first order moment is the velocity in the system, which for liquids is the flow field.
In a 2D lattice a two dimensional vector for each field is needed to describe the average local velocity of each particle.
The second order moment is the kenetic energy density in the system which can be readily interpreted as the temperature of a fluid.
Different fluids can have different translation ratios between velocities and temperature and thus an increase in temperature in a closed system would lead to higher velocities but the exact increase is depended on the fluid.


\section{Boltzmann Transport Equation}
The Boltzmann transport equation (BTE) tracks the time evolution of the probability distribution function and was published by Ludwig Boltzmann in 1872. 
Ludwig Eduard Boltzmann (1844–1906) was an Austrian physicist whose greatest achievement was in the development of statistical mechanics~\cite{book}. 

To derive it we take the first order derivative of the PDF in respect to time as shown in
equation \ref{eq:BTE}.
\begin{equation}
  \label{eq:BTE}
  \begin{aligned}
    \frac{df}{dt} =\frac{\partial f}{\partial t} + \frac{dr(t)}{dt} \nabla_{r}f + \frac{dv(t)}{dt} \nabla_{v}f = \left( \frac{\partial f}{\partial t} \right)_{col}
  \end{aligned}
\end{equation}
Where the term $f$ is the PDF, $\frac{dr(t)}{dt}$ is called the velocity and $\frac{dv(t)}{dt}$ is called the acceleration and by Newtons law is the force acting on the particles devivded by their respective mass.
The \textit{l.h.s.} of equation \ref{eq:BTE} is called the streaming term and the \textit{r.h.s} is the collision term.
To implement the streaming and collision terms the BTE first needs to be discretized.
The implementation of the streaming and collision terms are discussed in more detail in chapter 3.


\section{Lattice-Boltzmann Method} \label{sec:lbm}
The BTE is defined in a continous phase space which is not readily implementable in computer code.
This can be overcome by approximating the continous phase space to a discrete phase space, called the Lattice-Boltzmann method (LBM).
The idea is to discretize over a lattice and use the partial derivatives to solve the BTE.
In this report the spatial dimension is discretized to a 2D lattice and the velocities are descritized to 9 directions also called a \textit{D2Q9}-model.
This scheme is illustarate in \ref{fig:d2q9-scheme} where $a$ shows the discretazation of the velocity space and $b$ the discretized of the physical space on a 2D grid with the velocity layered on top. Thus, each particle has a 2D positional vector and a 9D velocity vector.
\begin{figure}[h]
  \centering
  \includegraphics[width=0.8\columnwidth]{d2q9_scheme.png}\\
  \small{(Material from lecture)}
  \caption[Discretization of the BTE]{Discretization of the BTE. a) is the discretization on the velocity space according to D2Q9 and b) the discretization of the physical space on a  2D lattice or grid.\\
  Source: Lecture}
  \label{fig:d2q9-scheme}
\end{figure}



\chapter{Implementation}
This chapter shows the most important code implementations, mainly the streaming, the collision, the boundaries and the moving top lid.
Throughout the report I will use the code notation $\_cxy$ where $c$ is the rolling dimension, here the velocity directions, and $x$ and $y$ are the dimensions of the physical space. The same logic applies for the local average velocities, notated as $\_axy$, where $a=2$, and the local densities $\_xy$.
The git repository for this report is under \href{https://github.com/jonas27/pylbm}{https://github.com/jonas27/pylbm} and can be installed with `\textit{pip install -e .}` from inside the root directory of the repo.

\section{Streaming}
The main part of milestone 1 is the streaming operator. It is explained here for reference and used throughout the report. 

From equation~\ref{eq:pdf} it follows that streaming and collision are two separate terms and collision can be ignored by setting it to zero, i.e. $\left( \frac{\partial f}{\partial t} \right)_{col} \overset{!}{=} 0$.
This simplifies the BTE to equation \ref{eq:streaming} 
\begin{equation}
  \label{eq:streaming}
  \begin{aligned}
    f_{i}(r+c_{i} \nabla t,t+\nabla t)=f_{i}(r,t)
  \end{aligned}
\end{equation}
and implies the movement of particles in the vacuum with no mutual interaction between particles.
The code implementation of a streaming function on a 2D lattice is shown in listing \ref{lst:streaming}
\begin{center}
\begin{lstlisting}[caption=Implementation of the streaming operator,label=lst:streaming, basicstyle=\small]
def stream(f_cxy: np.array) -> np.array:
    for i in range(1, 9):
        f_cxy[i, :, :] = np.roll(f_cxy[i, :, :], 
                      shift=C_CA[i], 
                      axis=(0, 1))
    return f_cxy
  \end{lstlisting}
\end{center}
$f_{cxy}$ is the PDF and $C \_ CA$ are the discretized velocity directions.
For each velocity direction the streaming shifts the velocities values in the respective direction.
Testing the streaming operator can be easily done by visualizing the different shift on a 2D grid for each velocity as shown in figure \ref{fig:m1-shifting} for the velocities $v=2$ $v=3$.
\begin{figure}[ht]
\centering
\resizebox{\columnwidth}{!}{\large%% Creator: Matplotlib, PGF backend
%%
%% To include the figure in your LaTeX document, write
%%   \input{<filename>.pgf}
%%
%% Make sure the required packages are loaded in your preamble
%%   \usepackage{pgf}
%%
%% Also ensure that all the required font packages are loaded; for instance,
%% the lmodern package is sometimes necessary when using math font.
%%   \usepackage{lmodern}
%%
%% Figures using additional raster images can only be included by \input if
%% they are in the same directory as the main LaTeX file. For loading figures
%% from other directories you can use the `import` package
%%   \usepackage{import}
%%
%% and then include the figures with
%%   \import{<path to file>}{<filename>.pgf}
%%
%% Matplotlib used the following preamble
%%   \usepackage{fontspec}
%%   \setmainfont{DejaVuSerif.ttf}[Path=\detokenize{/home/joe/miniconda3/envs/high/lib/python3.9/site-packages/matplotlib/mpl-data/fonts/ttf/}]
%%   \setsansfont{DejaVuSans.ttf}[Path=\detokenize{/home/joe/miniconda3/envs/high/lib/python3.9/site-packages/matplotlib/mpl-data/fonts/ttf/}]
%%   \setmonofont{DejaVuSansMono.ttf}[Path=\detokenize{/home/joe/miniconda3/envs/high/lib/python3.9/site-packages/matplotlib/mpl-data/fonts/ttf/}]
%%
\begingroup%
\makeatletter%
\begin{pgfpicture}%
\pgfpathrectangle{\pgfpointorigin}{\pgfqpoint{15.000000in}{5.000000in}}%
\pgfusepath{use as bounding box, clip}%
\begin{pgfscope}%
\pgfsetbuttcap%
\pgfsetmiterjoin%
\pgfsetlinewidth{0.000000pt}%
\definecolor{currentstroke}{rgb}{1.000000,1.000000,1.000000}%
\pgfsetstrokecolor{currentstroke}%
\pgfsetstrokeopacity{0.000000}%
\pgfsetdash{}{0pt}%
\pgfpathmoveto{\pgfqpoint{0.000000in}{0.000000in}}%
\pgfpathlineto{\pgfqpoint{15.000000in}{0.000000in}}%
\pgfpathlineto{\pgfqpoint{15.000000in}{5.000000in}}%
\pgfpathlineto{\pgfqpoint{0.000000in}{5.000000in}}%
\pgfpathlineto{\pgfqpoint{0.000000in}{0.000000in}}%
\pgfpathclose%
\pgfusepath{}%
\end{pgfscope}%
\begin{pgfscope}%
\pgfsetbuttcap%
\pgfsetmiterjoin%
\definecolor{currentfill}{rgb}{1.000000,1.000000,1.000000}%
\pgfsetfillcolor{currentfill}%
\pgfsetlinewidth{0.000000pt}%
\definecolor{currentstroke}{rgb}{0.000000,0.000000,0.000000}%
\pgfsetstrokecolor{currentstroke}%
\pgfsetstrokeopacity{0.000000}%
\pgfsetdash{}{0pt}%
\pgfpathmoveto{\pgfqpoint{0.794097in}{3.269444in}}%
\pgfpathlineto{\pgfqpoint{3.712500in}{3.269444in}}%
\pgfpathlineto{\pgfqpoint{3.712500in}{4.850000in}}%
\pgfpathlineto{\pgfqpoint{0.794097in}{4.850000in}}%
\pgfpathlineto{\pgfqpoint{0.794097in}{3.269444in}}%
\pgfpathclose%
\pgfusepath{fill}%
\end{pgfscope}%
\begin{pgfscope}%
\pgfpathrectangle{\pgfqpoint{0.794097in}{3.269444in}}{\pgfqpoint{2.918403in}{1.580556in}}%
\pgfusepath{clip}%
\pgfsetbuttcap%
\pgfsetroundjoin%
\definecolor{currentfill}{rgb}{0.984744,0.432910,0.307420}%
\pgfsetfillcolor{currentfill}%
\pgfsetlinewidth{0.000000pt}%
\definecolor{currentstroke}{rgb}{0.000000,0.000000,0.000000}%
\pgfsetstrokecolor{currentstroke}%
\pgfsetdash{}{0pt}%
\pgfpathmoveto{\pgfqpoint{0.794097in}{3.269444in}}%
\pgfpathlineto{\pgfqpoint{0.794097in}{3.664583in}}%
\pgfpathlineto{\pgfqpoint{1.377778in}{3.664583in}}%
\pgfpathlineto{\pgfqpoint{1.377778in}{3.269444in}}%
\pgfpathlineto{\pgfqpoint{0.794097in}{3.269444in}}%
\pgfpathlineto{\pgfqpoint{0.794097in}{3.269444in}}%
\pgfpathclose%
\pgfusepath{fill}%
\end{pgfscope}%
\begin{pgfscope}%
\pgfpathrectangle{\pgfqpoint{0.794097in}{3.269444in}}{\pgfqpoint{2.918403in}{1.580556in}}%
\pgfusepath{clip}%
\pgfsetbuttcap%
\pgfsetroundjoin%
\definecolor{currentfill}{rgb}{0.974717,0.378101,0.266205}%
\pgfsetfillcolor{currentfill}%
\pgfsetlinewidth{0.000000pt}%
\definecolor{currentstroke}{rgb}{0.000000,0.000000,0.000000}%
\pgfsetstrokecolor{currentstroke}%
\pgfsetdash{}{0pt}%
\pgfpathmoveto{\pgfqpoint{1.377778in}{3.269444in}}%
\pgfpathlineto{\pgfqpoint{1.377778in}{3.664583in}}%
\pgfpathlineto{\pgfqpoint{1.961458in}{3.664583in}}%
\pgfpathlineto{\pgfqpoint{1.961458in}{3.269444in}}%
\pgfpathlineto{\pgfqpoint{1.377778in}{3.269444in}}%
\pgfpathlineto{\pgfqpoint{1.377778in}{3.269444in}}%
\pgfpathclose%
\pgfusepath{fill}%
\end{pgfscope}%
\begin{pgfscope}%
\pgfpathrectangle{\pgfqpoint{0.794097in}{3.269444in}}{\pgfqpoint{2.918403in}{1.580556in}}%
\pgfusepath{clip}%
\pgfsetbuttcap%
\pgfsetroundjoin%
\definecolor{currentfill}{rgb}{0.988235,0.681630,0.572103}%
\pgfsetfillcolor{currentfill}%
\pgfsetlinewidth{0.000000pt}%
\definecolor{currentstroke}{rgb}{0.000000,0.000000,0.000000}%
\pgfsetstrokecolor{currentstroke}%
\pgfsetdash{}{0pt}%
\pgfpathmoveto{\pgfqpoint{1.961458in}{3.269444in}}%
\pgfpathlineto{\pgfqpoint{1.961458in}{3.664583in}}%
\pgfpathlineto{\pgfqpoint{2.545139in}{3.664583in}}%
\pgfpathlineto{\pgfqpoint{2.545139in}{3.269444in}}%
\pgfpathlineto{\pgfqpoint{1.961458in}{3.269444in}}%
\pgfpathlineto{\pgfqpoint{1.961458in}{3.269444in}}%
\pgfpathclose%
\pgfusepath{fill}%
\end{pgfscope}%
\begin{pgfscope}%
\pgfpathrectangle{\pgfqpoint{0.794097in}{3.269444in}}{\pgfqpoint{2.918403in}{1.580556in}}%
\pgfusepath{clip}%
\pgfsetbuttcap%
\pgfsetroundjoin%
\definecolor{currentfill}{rgb}{0.525967,0.029527,0.066728}%
\pgfsetfillcolor{currentfill}%
\pgfsetlinewidth{0.000000pt}%
\definecolor{currentstroke}{rgb}{0.000000,0.000000,0.000000}%
\pgfsetstrokecolor{currentstroke}%
\pgfsetdash{}{0pt}%
\pgfpathmoveto{\pgfqpoint{2.545139in}{3.269444in}}%
\pgfpathlineto{\pgfqpoint{2.545139in}{3.664583in}}%
\pgfpathlineto{\pgfqpoint{3.128819in}{3.664583in}}%
\pgfpathlineto{\pgfqpoint{3.128819in}{3.269444in}}%
\pgfpathlineto{\pgfqpoint{2.545139in}{3.269444in}}%
\pgfpathlineto{\pgfqpoint{2.545139in}{3.269444in}}%
\pgfpathclose%
\pgfusepath{fill}%
\end{pgfscope}%
\begin{pgfscope}%
\pgfpathrectangle{\pgfqpoint{0.794097in}{3.269444in}}{\pgfqpoint{2.918403in}{1.580556in}}%
\pgfusepath{clip}%
\pgfsetbuttcap%
\pgfsetroundjoin%
\definecolor{currentfill}{rgb}{0.525967,0.029527,0.066728}%
\pgfsetfillcolor{currentfill}%
\pgfsetlinewidth{0.000000pt}%
\definecolor{currentstroke}{rgb}{0.000000,0.000000,0.000000}%
\pgfsetstrokecolor{currentstroke}%
\pgfsetdash{}{0pt}%
\pgfpathmoveto{\pgfqpoint{3.128819in}{3.269444in}}%
\pgfpathlineto{\pgfqpoint{3.128819in}{3.664583in}}%
\pgfpathlineto{\pgfqpoint{3.712500in}{3.664583in}}%
\pgfpathlineto{\pgfqpoint{3.712500in}{3.269444in}}%
\pgfpathlineto{\pgfqpoint{3.128819in}{3.269444in}}%
\pgfpathlineto{\pgfqpoint{3.128819in}{3.269444in}}%
\pgfpathclose%
\pgfusepath{fill}%
\end{pgfscope}%
\begin{pgfscope}%
\pgfpathrectangle{\pgfqpoint{0.794097in}{3.269444in}}{\pgfqpoint{2.918403in}{1.580556in}}%
\pgfusepath{clip}%
\pgfsetbuttcap%
\pgfsetroundjoin%
\definecolor{currentfill}{rgb}{0.974717,0.378101,0.266205}%
\pgfsetfillcolor{currentfill}%
\pgfsetlinewidth{0.000000pt}%
\definecolor{currentstroke}{rgb}{0.000000,0.000000,0.000000}%
\pgfsetstrokecolor{currentstroke}%
\pgfsetdash{}{0pt}%
\pgfpathmoveto{\pgfqpoint{0.794097in}{3.664583in}}%
\pgfpathlineto{\pgfqpoint{0.794097in}{4.059722in}}%
\pgfpathlineto{\pgfqpoint{1.377778in}{4.059722in}}%
\pgfpathlineto{\pgfqpoint{1.377778in}{3.664583in}}%
\pgfpathlineto{\pgfqpoint{0.794097in}{3.664583in}}%
\pgfpathlineto{\pgfqpoint{0.794097in}{3.664583in}}%
\pgfpathclose%
\pgfusepath{fill}%
\end{pgfscope}%
\begin{pgfscope}%
\pgfpathrectangle{\pgfqpoint{0.794097in}{3.269444in}}{\pgfqpoint{2.918403in}{1.580556in}}%
\pgfusepath{clip}%
\pgfsetbuttcap%
\pgfsetroundjoin%
\definecolor{currentfill}{rgb}{0.640384,0.057209,0.081492}%
\pgfsetfillcolor{currentfill}%
\pgfsetlinewidth{0.000000pt}%
\definecolor{currentstroke}{rgb}{0.000000,0.000000,0.000000}%
\pgfsetstrokecolor{currentstroke}%
\pgfsetdash{}{0pt}%
\pgfpathmoveto{\pgfqpoint{1.377778in}{3.664583in}}%
\pgfpathlineto{\pgfqpoint{1.377778in}{4.059722in}}%
\pgfpathlineto{\pgfqpoint{1.961458in}{4.059722in}}%
\pgfpathlineto{\pgfqpoint{1.961458in}{3.664583in}}%
\pgfpathlineto{\pgfqpoint{1.377778in}{3.664583in}}%
\pgfpathlineto{\pgfqpoint{1.377778in}{3.664583in}}%
\pgfpathclose%
\pgfusepath{fill}%
\end{pgfscope}%
\begin{pgfscope}%
\pgfpathrectangle{\pgfqpoint{0.794097in}{3.269444in}}{\pgfqpoint{2.918403in}{1.580556in}}%
\pgfusepath{clip}%
\pgfsetbuttcap%
\pgfsetroundjoin%
\definecolor{currentfill}{rgb}{0.948143,0.274018,0.199769}%
\pgfsetfillcolor{currentfill}%
\pgfsetlinewidth{0.000000pt}%
\definecolor{currentstroke}{rgb}{0.000000,0.000000,0.000000}%
\pgfsetstrokecolor{currentstroke}%
\pgfsetdash{}{0pt}%
\pgfpathmoveto{\pgfqpoint{1.961458in}{3.664583in}}%
\pgfpathlineto{\pgfqpoint{1.961458in}{4.059722in}}%
\pgfpathlineto{\pgfqpoint{2.545139in}{4.059722in}}%
\pgfpathlineto{\pgfqpoint{2.545139in}{3.664583in}}%
\pgfpathlineto{\pgfqpoint{1.961458in}{3.664583in}}%
\pgfpathlineto{\pgfqpoint{1.961458in}{3.664583in}}%
\pgfpathclose%
\pgfusepath{fill}%
\end{pgfscope}%
\begin{pgfscope}%
\pgfpathrectangle{\pgfqpoint{0.794097in}{3.269444in}}{\pgfqpoint{2.918403in}{1.580556in}}%
\pgfusepath{clip}%
\pgfsetbuttcap%
\pgfsetroundjoin%
\definecolor{currentfill}{rgb}{0.666344,0.063391,0.086413}%
\pgfsetfillcolor{currentfill}%
\pgfsetlinewidth{0.000000pt}%
\definecolor{currentstroke}{rgb}{0.000000,0.000000,0.000000}%
\pgfsetstrokecolor{currentstroke}%
\pgfsetdash{}{0pt}%
\pgfpathmoveto{\pgfqpoint{2.545139in}{3.664583in}}%
\pgfpathlineto{\pgfqpoint{2.545139in}{4.059722in}}%
\pgfpathlineto{\pgfqpoint{3.128819in}{4.059722in}}%
\pgfpathlineto{\pgfqpoint{3.128819in}{3.664583in}}%
\pgfpathlineto{\pgfqpoint{2.545139in}{3.664583in}}%
\pgfpathlineto{\pgfqpoint{2.545139in}{3.664583in}}%
\pgfpathclose%
\pgfusepath{fill}%
\end{pgfscope}%
\begin{pgfscope}%
\pgfpathrectangle{\pgfqpoint{0.794097in}{3.269444in}}{\pgfqpoint{2.918403in}{1.580556in}}%
\pgfusepath{clip}%
\pgfsetbuttcap%
\pgfsetroundjoin%
\definecolor{currentfill}{rgb}{0.988235,0.666498,0.554756}%
\pgfsetfillcolor{currentfill}%
\pgfsetlinewidth{0.000000pt}%
\definecolor{currentstroke}{rgb}{0.000000,0.000000,0.000000}%
\pgfsetstrokecolor{currentstroke}%
\pgfsetdash{}{0pt}%
\pgfpathmoveto{\pgfqpoint{3.128819in}{3.664583in}}%
\pgfpathlineto{\pgfqpoint{3.128819in}{4.059722in}}%
\pgfpathlineto{\pgfqpoint{3.712500in}{4.059722in}}%
\pgfpathlineto{\pgfqpoint{3.712500in}{3.664583in}}%
\pgfpathlineto{\pgfqpoint{3.128819in}{3.664583in}}%
\pgfpathlineto{\pgfqpoint{3.128819in}{3.664583in}}%
\pgfpathclose%
\pgfusepath{fill}%
\end{pgfscope}%
\begin{pgfscope}%
\pgfpathrectangle{\pgfqpoint{0.794097in}{3.269444in}}{\pgfqpoint{2.918403in}{1.580556in}}%
\pgfusepath{clip}%
\pgfsetbuttcap%
\pgfsetroundjoin%
\definecolor{currentfill}{rgb}{0.954048,0.297147,0.214533}%
\pgfsetfillcolor{currentfill}%
\pgfsetlinewidth{0.000000pt}%
\definecolor{currentstroke}{rgb}{0.000000,0.000000,0.000000}%
\pgfsetstrokecolor{currentstroke}%
\pgfsetdash{}{0pt}%
\pgfpathmoveto{\pgfqpoint{0.794097in}{4.059722in}}%
\pgfpathlineto{\pgfqpoint{0.794097in}{4.454861in}}%
\pgfpathlineto{\pgfqpoint{1.377778in}{4.454861in}}%
\pgfpathlineto{\pgfqpoint{1.377778in}{4.059722in}}%
\pgfpathlineto{\pgfqpoint{0.794097in}{4.059722in}}%
\pgfpathlineto{\pgfqpoint{0.794097in}{4.059722in}}%
\pgfpathclose%
\pgfusepath{fill}%
\end{pgfscope}%
\begin{pgfscope}%
\pgfpathrectangle{\pgfqpoint{0.794097in}{3.269444in}}{\pgfqpoint{2.918403in}{1.580556in}}%
\pgfusepath{clip}%
\pgfsetbuttcap%
\pgfsetroundjoin%
\definecolor{currentfill}{rgb}{0.988189,0.570704,0.445213}%
\pgfsetfillcolor{currentfill}%
\pgfsetlinewidth{0.000000pt}%
\definecolor{currentstroke}{rgb}{0.000000,0.000000,0.000000}%
\pgfsetstrokecolor{currentstroke}%
\pgfsetdash{}{0pt}%
\pgfpathmoveto{\pgfqpoint{1.377778in}{4.059722in}}%
\pgfpathlineto{\pgfqpoint{1.377778in}{4.454861in}}%
\pgfpathlineto{\pgfqpoint{1.961458in}{4.454861in}}%
\pgfpathlineto{\pgfqpoint{1.961458in}{4.059722in}}%
\pgfpathlineto{\pgfqpoint{1.377778in}{4.059722in}}%
\pgfpathlineto{\pgfqpoint{1.377778in}{4.059722in}}%
\pgfpathclose%
\pgfusepath{fill}%
\end{pgfscope}%
\begin{pgfscope}%
\pgfpathrectangle{\pgfqpoint{0.794097in}{3.269444in}}{\pgfqpoint{2.918403in}{1.580556in}}%
\pgfusepath{clip}%
\pgfsetbuttcap%
\pgfsetroundjoin%
\definecolor{currentfill}{rgb}{0.961430,0.326059,0.232987}%
\pgfsetfillcolor{currentfill}%
\pgfsetlinewidth{0.000000pt}%
\definecolor{currentstroke}{rgb}{0.000000,0.000000,0.000000}%
\pgfsetstrokecolor{currentstroke}%
\pgfsetdash{}{0pt}%
\pgfpathmoveto{\pgfqpoint{1.961458in}{4.059722in}}%
\pgfpathlineto{\pgfqpoint{1.961458in}{4.454861in}}%
\pgfpathlineto{\pgfqpoint{2.545139in}{4.454861in}}%
\pgfpathlineto{\pgfqpoint{2.545139in}{4.059722in}}%
\pgfpathlineto{\pgfqpoint{1.961458in}{4.059722in}}%
\pgfpathlineto{\pgfqpoint{1.961458in}{4.059722in}}%
\pgfpathclose%
\pgfusepath{fill}%
\end{pgfscope}%
\begin{pgfscope}%
\pgfpathrectangle{\pgfqpoint{0.794097in}{3.269444in}}{\pgfqpoint{2.918403in}{1.580556in}}%
\pgfusepath{clip}%
\pgfsetbuttcap%
\pgfsetroundjoin%
\definecolor{currentfill}{rgb}{0.990388,0.773164,0.684121}%
\pgfsetfillcolor{currentfill}%
\pgfsetlinewidth{0.000000pt}%
\definecolor{currentstroke}{rgb}{0.000000,0.000000,0.000000}%
\pgfsetstrokecolor{currentstroke}%
\pgfsetdash{}{0pt}%
\pgfpathmoveto{\pgfqpoint{2.545139in}{4.059722in}}%
\pgfpathlineto{\pgfqpoint{2.545139in}{4.454861in}}%
\pgfpathlineto{\pgfqpoint{3.128819in}{4.454861in}}%
\pgfpathlineto{\pgfqpoint{3.128819in}{4.059722in}}%
\pgfpathlineto{\pgfqpoint{2.545139in}{4.059722in}}%
\pgfpathlineto{\pgfqpoint{2.545139in}{4.059722in}}%
\pgfpathclose%
\pgfusepath{fill}%
\end{pgfscope}%
\begin{pgfscope}%
\pgfpathrectangle{\pgfqpoint{0.794097in}{3.269444in}}{\pgfqpoint{2.918403in}{1.580556in}}%
\pgfusepath{clip}%
\pgfsetbuttcap%
\pgfsetroundjoin%
\definecolor{currentfill}{rgb}{0.988235,0.575702,0.450673}%
\pgfsetfillcolor{currentfill}%
\pgfsetlinewidth{0.000000pt}%
\definecolor{currentstroke}{rgb}{0.000000,0.000000,0.000000}%
\pgfsetstrokecolor{currentstroke}%
\pgfsetdash{}{0pt}%
\pgfpathmoveto{\pgfqpoint{3.128819in}{4.059722in}}%
\pgfpathlineto{\pgfqpoint{3.128819in}{4.454861in}}%
\pgfpathlineto{\pgfqpoint{3.712500in}{4.454861in}}%
\pgfpathlineto{\pgfqpoint{3.712500in}{4.059722in}}%
\pgfpathlineto{\pgfqpoint{3.128819in}{4.059722in}}%
\pgfpathlineto{\pgfqpoint{3.128819in}{4.059722in}}%
\pgfpathclose%
\pgfusepath{fill}%
\end{pgfscope}%
\begin{pgfscope}%
\pgfpathrectangle{\pgfqpoint{0.794097in}{3.269444in}}{\pgfqpoint{2.918403in}{1.580556in}}%
\pgfusepath{clip}%
\pgfsetbuttcap%
\pgfsetroundjoin%
\definecolor{currentfill}{rgb}{0.991373,0.791373,0.708235}%
\pgfsetfillcolor{currentfill}%
\pgfsetlinewidth{0.000000pt}%
\definecolor{currentstroke}{rgb}{0.000000,0.000000,0.000000}%
\pgfsetstrokecolor{currentstroke}%
\pgfsetdash{}{0pt}%
\pgfpathmoveto{\pgfqpoint{0.794097in}{4.454861in}}%
\pgfpathlineto{\pgfqpoint{0.794097in}{4.850000in}}%
\pgfpathlineto{\pgfqpoint{1.377778in}{4.850000in}}%
\pgfpathlineto{\pgfqpoint{1.377778in}{4.454861in}}%
\pgfpathlineto{\pgfqpoint{0.794097in}{4.454861in}}%
\pgfpathlineto{\pgfqpoint{0.794097in}{4.454861in}}%
\pgfpathclose%
\pgfusepath{fill}%
\end{pgfscope}%
\begin{pgfscope}%
\pgfpathrectangle{\pgfqpoint{0.794097in}{3.269444in}}{\pgfqpoint{2.918403in}{1.580556in}}%
\pgfusepath{clip}%
\pgfsetbuttcap%
\pgfsetroundjoin%
\definecolor{currentfill}{rgb}{0.985729,0.472280,0.346790}%
\pgfsetfillcolor{currentfill}%
\pgfsetlinewidth{0.000000pt}%
\definecolor{currentstroke}{rgb}{0.000000,0.000000,0.000000}%
\pgfsetstrokecolor{currentstroke}%
\pgfsetdash{}{0pt}%
\pgfpathmoveto{\pgfqpoint{1.377778in}{4.454861in}}%
\pgfpathlineto{\pgfqpoint{1.377778in}{4.850000in}}%
\pgfpathlineto{\pgfqpoint{1.961458in}{4.850000in}}%
\pgfpathlineto{\pgfqpoint{1.961458in}{4.454861in}}%
\pgfpathlineto{\pgfqpoint{1.377778in}{4.454861in}}%
\pgfpathlineto{\pgfqpoint{1.377778in}{4.454861in}}%
\pgfpathclose%
\pgfusepath{fill}%
\end{pgfscope}%
\begin{pgfscope}%
\pgfpathrectangle{\pgfqpoint{0.794097in}{3.269444in}}{\pgfqpoint{2.918403in}{1.580556in}}%
\pgfusepath{clip}%
\pgfsetbuttcap%
\pgfsetroundjoin%
\definecolor{currentfill}{rgb}{0.868051,0.164091,0.143714}%
\pgfsetfillcolor{currentfill}%
\pgfsetlinewidth{0.000000pt}%
\definecolor{currentstroke}{rgb}{0.000000,0.000000,0.000000}%
\pgfsetstrokecolor{currentstroke}%
\pgfsetdash{}{0pt}%
\pgfpathmoveto{\pgfqpoint{1.961458in}{4.454861in}}%
\pgfpathlineto{\pgfqpoint{1.961458in}{4.850000in}}%
\pgfpathlineto{\pgfqpoint{2.545139in}{4.850000in}}%
\pgfpathlineto{\pgfqpoint{2.545139in}{4.454861in}}%
\pgfpathlineto{\pgfqpoint{1.961458in}{4.454861in}}%
\pgfpathlineto{\pgfqpoint{1.961458in}{4.454861in}}%
\pgfpathclose%
\pgfusepath{fill}%
\end{pgfscope}%
\begin{pgfscope}%
\pgfpathrectangle{\pgfqpoint{0.794097in}{3.269444in}}{\pgfqpoint{2.918403in}{1.580556in}}%
\pgfusepath{clip}%
\pgfsetbuttcap%
\pgfsetroundjoin%
\definecolor{currentfill}{rgb}{1.000000,0.960784,0.941176}%
\pgfsetfillcolor{currentfill}%
\pgfsetlinewidth{0.000000pt}%
\definecolor{currentstroke}{rgb}{0.000000,0.000000,0.000000}%
\pgfsetstrokecolor{currentstroke}%
\pgfsetdash{}{0pt}%
\pgfpathmoveto{\pgfqpoint{2.545139in}{4.454861in}}%
\pgfpathlineto{\pgfqpoint{2.545139in}{4.850000in}}%
\pgfpathlineto{\pgfqpoint{3.128819in}{4.850000in}}%
\pgfpathlineto{\pgfqpoint{3.128819in}{4.454861in}}%
\pgfpathlineto{\pgfqpoint{2.545139in}{4.454861in}}%
\pgfpathlineto{\pgfqpoint{2.545139in}{4.454861in}}%
\pgfpathclose%
\pgfusepath{fill}%
\end{pgfscope}%
\begin{pgfscope}%
\pgfpathrectangle{\pgfqpoint{0.794097in}{3.269444in}}{\pgfqpoint{2.918403in}{1.580556in}}%
\pgfusepath{clip}%
\pgfsetbuttcap%
\pgfsetroundjoin%
\definecolor{currentfill}{rgb}{0.403922,0.000000,0.050980}%
\pgfsetfillcolor{currentfill}%
\pgfsetlinewidth{0.000000pt}%
\definecolor{currentstroke}{rgb}{0.000000,0.000000,0.000000}%
\pgfsetstrokecolor{currentstroke}%
\pgfsetdash{}{0pt}%
\pgfpathmoveto{\pgfqpoint{3.128819in}{4.454861in}}%
\pgfpathlineto{\pgfqpoint{3.128819in}{4.850000in}}%
\pgfpathlineto{\pgfqpoint{3.712500in}{4.850000in}}%
\pgfpathlineto{\pgfqpoint{3.712500in}{4.454861in}}%
\pgfpathlineto{\pgfqpoint{3.128819in}{4.454861in}}%
\pgfpathlineto{\pgfqpoint{3.128819in}{4.454861in}}%
\pgfpathclose%
\pgfusepath{fill}%
\end{pgfscope}%
\begin{pgfscope}%
\pgfsetbuttcap%
\pgfsetroundjoin%
\definecolor{currentfill}{rgb}{0.000000,0.000000,0.000000}%
\pgfsetfillcolor{currentfill}%
\pgfsetlinewidth{0.803000pt}%
\definecolor{currentstroke}{rgb}{0.000000,0.000000,0.000000}%
\pgfsetstrokecolor{currentstroke}%
\pgfsetdash{}{0pt}%
\pgfsys@defobject{currentmarker}{\pgfqpoint{0.000000in}{-0.048611in}}{\pgfqpoint{0.000000in}{0.000000in}}{%
\pgfpathmoveto{\pgfqpoint{0.000000in}{0.000000in}}%
\pgfpathlineto{\pgfqpoint{0.000000in}{-0.048611in}}%
\pgfusepath{stroke,fill}%
}%
\begin{pgfscope}%
\pgfsys@transformshift{0.794097in}{3.269444in}%
\pgfsys@useobject{currentmarker}{}%
\end{pgfscope}%
\end{pgfscope}%
\begin{pgfscope}%
\definecolor{textcolor}{rgb}{0.000000,0.000000,0.000000}%
\pgfsetstrokecolor{textcolor}%
\pgfsetfillcolor{textcolor}%
\pgftext[x=0.794097in,y=3.172222in,,top]{\color{textcolor}\sffamily\fontsize{10.000000}{12.000000}\selectfont 0}%
\end{pgfscope}%
\begin{pgfscope}%
\pgfsetbuttcap%
\pgfsetroundjoin%
\definecolor{currentfill}{rgb}{0.000000,0.000000,0.000000}%
\pgfsetfillcolor{currentfill}%
\pgfsetlinewidth{0.803000pt}%
\definecolor{currentstroke}{rgb}{0.000000,0.000000,0.000000}%
\pgfsetstrokecolor{currentstroke}%
\pgfsetdash{}{0pt}%
\pgfsys@defobject{currentmarker}{\pgfqpoint{0.000000in}{-0.048611in}}{\pgfqpoint{0.000000in}{0.000000in}}{%
\pgfpathmoveto{\pgfqpoint{0.000000in}{0.000000in}}%
\pgfpathlineto{\pgfqpoint{0.000000in}{-0.048611in}}%
\pgfusepath{stroke,fill}%
}%
\begin{pgfscope}%
\pgfsys@transformshift{1.377778in}{3.269444in}%
\pgfsys@useobject{currentmarker}{}%
\end{pgfscope}%
\end{pgfscope}%
\begin{pgfscope}%
\definecolor{textcolor}{rgb}{0.000000,0.000000,0.000000}%
\pgfsetstrokecolor{textcolor}%
\pgfsetfillcolor{textcolor}%
\pgftext[x=1.377778in,y=3.172222in,,top]{\color{textcolor}\sffamily\fontsize{10.000000}{12.000000}\selectfont 1}%
\end{pgfscope}%
\begin{pgfscope}%
\pgfsetbuttcap%
\pgfsetroundjoin%
\definecolor{currentfill}{rgb}{0.000000,0.000000,0.000000}%
\pgfsetfillcolor{currentfill}%
\pgfsetlinewidth{0.803000pt}%
\definecolor{currentstroke}{rgb}{0.000000,0.000000,0.000000}%
\pgfsetstrokecolor{currentstroke}%
\pgfsetdash{}{0pt}%
\pgfsys@defobject{currentmarker}{\pgfqpoint{0.000000in}{-0.048611in}}{\pgfqpoint{0.000000in}{0.000000in}}{%
\pgfpathmoveto{\pgfqpoint{0.000000in}{0.000000in}}%
\pgfpathlineto{\pgfqpoint{0.000000in}{-0.048611in}}%
\pgfusepath{stroke,fill}%
}%
\begin{pgfscope}%
\pgfsys@transformshift{1.961458in}{3.269444in}%
\pgfsys@useobject{currentmarker}{}%
\end{pgfscope}%
\end{pgfscope}%
\begin{pgfscope}%
\definecolor{textcolor}{rgb}{0.000000,0.000000,0.000000}%
\pgfsetstrokecolor{textcolor}%
\pgfsetfillcolor{textcolor}%
\pgftext[x=1.961458in,y=3.172222in,,top]{\color{textcolor}\sffamily\fontsize{10.000000}{12.000000}\selectfont 2}%
\end{pgfscope}%
\begin{pgfscope}%
\pgfsetbuttcap%
\pgfsetroundjoin%
\definecolor{currentfill}{rgb}{0.000000,0.000000,0.000000}%
\pgfsetfillcolor{currentfill}%
\pgfsetlinewidth{0.803000pt}%
\definecolor{currentstroke}{rgb}{0.000000,0.000000,0.000000}%
\pgfsetstrokecolor{currentstroke}%
\pgfsetdash{}{0pt}%
\pgfsys@defobject{currentmarker}{\pgfqpoint{0.000000in}{-0.048611in}}{\pgfqpoint{0.000000in}{0.000000in}}{%
\pgfpathmoveto{\pgfqpoint{0.000000in}{0.000000in}}%
\pgfpathlineto{\pgfqpoint{0.000000in}{-0.048611in}}%
\pgfusepath{stroke,fill}%
}%
\begin{pgfscope}%
\pgfsys@transformshift{2.545139in}{3.269444in}%
\pgfsys@useobject{currentmarker}{}%
\end{pgfscope}%
\end{pgfscope}%
\begin{pgfscope}%
\definecolor{textcolor}{rgb}{0.000000,0.000000,0.000000}%
\pgfsetstrokecolor{textcolor}%
\pgfsetfillcolor{textcolor}%
\pgftext[x=2.545139in,y=3.172222in,,top]{\color{textcolor}\sffamily\fontsize{10.000000}{12.000000}\selectfont 3}%
\end{pgfscope}%
\begin{pgfscope}%
\pgfsetbuttcap%
\pgfsetroundjoin%
\definecolor{currentfill}{rgb}{0.000000,0.000000,0.000000}%
\pgfsetfillcolor{currentfill}%
\pgfsetlinewidth{0.803000pt}%
\definecolor{currentstroke}{rgb}{0.000000,0.000000,0.000000}%
\pgfsetstrokecolor{currentstroke}%
\pgfsetdash{}{0pt}%
\pgfsys@defobject{currentmarker}{\pgfqpoint{0.000000in}{-0.048611in}}{\pgfqpoint{0.000000in}{0.000000in}}{%
\pgfpathmoveto{\pgfqpoint{0.000000in}{0.000000in}}%
\pgfpathlineto{\pgfqpoint{0.000000in}{-0.048611in}}%
\pgfusepath{stroke,fill}%
}%
\begin{pgfscope}%
\pgfsys@transformshift{3.128819in}{3.269444in}%
\pgfsys@useobject{currentmarker}{}%
\end{pgfscope}%
\end{pgfscope}%
\begin{pgfscope}%
\definecolor{textcolor}{rgb}{0.000000,0.000000,0.000000}%
\pgfsetstrokecolor{textcolor}%
\pgfsetfillcolor{textcolor}%
\pgftext[x=3.128819in,y=3.172222in,,top]{\color{textcolor}\sffamily\fontsize{10.000000}{12.000000}\selectfont 4}%
\end{pgfscope}%
\begin{pgfscope}%
\definecolor{textcolor}{rgb}{0.000000,0.000000,0.000000}%
\pgfsetstrokecolor{textcolor}%
\pgfsetfillcolor{textcolor}%
\pgftext[x=2.253299in,y=2.982254in,,top]{\color{textcolor}\sffamily\fontsize{30.000000}{36.000000}\selectfont x axis}%
\end{pgfscope}%
\begin{pgfscope}%
\pgfsetbuttcap%
\pgfsetroundjoin%
\definecolor{currentfill}{rgb}{0.000000,0.000000,0.000000}%
\pgfsetfillcolor{currentfill}%
\pgfsetlinewidth{0.803000pt}%
\definecolor{currentstroke}{rgb}{0.000000,0.000000,0.000000}%
\pgfsetstrokecolor{currentstroke}%
\pgfsetdash{}{0pt}%
\pgfsys@defobject{currentmarker}{\pgfqpoint{-0.048611in}{0.000000in}}{\pgfqpoint{-0.000000in}{0.000000in}}{%
\pgfpathmoveto{\pgfqpoint{-0.000000in}{0.000000in}}%
\pgfpathlineto{\pgfqpoint{-0.048611in}{0.000000in}}%
\pgfusepath{stroke,fill}%
}%
\begin{pgfscope}%
\pgfsys@transformshift{0.794097in}{3.269444in}%
\pgfsys@useobject{currentmarker}{}%
\end{pgfscope}%
\end{pgfscope}%
\begin{pgfscope}%
\definecolor{textcolor}{rgb}{0.000000,0.000000,0.000000}%
\pgfsetstrokecolor{textcolor}%
\pgfsetfillcolor{textcolor}%
\pgftext[x=0.608510in, y=3.216683in, left, base]{\color{textcolor}\sffamily\fontsize{10.000000}{12.000000}\selectfont 0}%
\end{pgfscope}%
\begin{pgfscope}%
\pgfsetbuttcap%
\pgfsetroundjoin%
\definecolor{currentfill}{rgb}{0.000000,0.000000,0.000000}%
\pgfsetfillcolor{currentfill}%
\pgfsetlinewidth{0.803000pt}%
\definecolor{currentstroke}{rgb}{0.000000,0.000000,0.000000}%
\pgfsetstrokecolor{currentstroke}%
\pgfsetdash{}{0pt}%
\pgfsys@defobject{currentmarker}{\pgfqpoint{-0.048611in}{0.000000in}}{\pgfqpoint{-0.000000in}{0.000000in}}{%
\pgfpathmoveto{\pgfqpoint{-0.000000in}{0.000000in}}%
\pgfpathlineto{\pgfqpoint{-0.048611in}{0.000000in}}%
\pgfusepath{stroke,fill}%
}%
\begin{pgfscope}%
\pgfsys@transformshift{0.794097in}{3.664583in}%
\pgfsys@useobject{currentmarker}{}%
\end{pgfscope}%
\end{pgfscope}%
\begin{pgfscope}%
\definecolor{textcolor}{rgb}{0.000000,0.000000,0.000000}%
\pgfsetstrokecolor{textcolor}%
\pgfsetfillcolor{textcolor}%
\pgftext[x=0.608510in, y=3.611822in, left, base]{\color{textcolor}\sffamily\fontsize{10.000000}{12.000000}\selectfont 1}%
\end{pgfscope}%
\begin{pgfscope}%
\pgfsetbuttcap%
\pgfsetroundjoin%
\definecolor{currentfill}{rgb}{0.000000,0.000000,0.000000}%
\pgfsetfillcolor{currentfill}%
\pgfsetlinewidth{0.803000pt}%
\definecolor{currentstroke}{rgb}{0.000000,0.000000,0.000000}%
\pgfsetstrokecolor{currentstroke}%
\pgfsetdash{}{0pt}%
\pgfsys@defobject{currentmarker}{\pgfqpoint{-0.048611in}{0.000000in}}{\pgfqpoint{-0.000000in}{0.000000in}}{%
\pgfpathmoveto{\pgfqpoint{-0.000000in}{0.000000in}}%
\pgfpathlineto{\pgfqpoint{-0.048611in}{0.000000in}}%
\pgfusepath{stroke,fill}%
}%
\begin{pgfscope}%
\pgfsys@transformshift{0.794097in}{4.059722in}%
\pgfsys@useobject{currentmarker}{}%
\end{pgfscope}%
\end{pgfscope}%
\begin{pgfscope}%
\definecolor{textcolor}{rgb}{0.000000,0.000000,0.000000}%
\pgfsetstrokecolor{textcolor}%
\pgfsetfillcolor{textcolor}%
\pgftext[x=0.608510in, y=4.006961in, left, base]{\color{textcolor}\sffamily\fontsize{10.000000}{12.000000}\selectfont 2}%
\end{pgfscope}%
\begin{pgfscope}%
\pgfsetbuttcap%
\pgfsetroundjoin%
\definecolor{currentfill}{rgb}{0.000000,0.000000,0.000000}%
\pgfsetfillcolor{currentfill}%
\pgfsetlinewidth{0.803000pt}%
\definecolor{currentstroke}{rgb}{0.000000,0.000000,0.000000}%
\pgfsetstrokecolor{currentstroke}%
\pgfsetdash{}{0pt}%
\pgfsys@defobject{currentmarker}{\pgfqpoint{-0.048611in}{0.000000in}}{\pgfqpoint{-0.000000in}{0.000000in}}{%
\pgfpathmoveto{\pgfqpoint{-0.000000in}{0.000000in}}%
\pgfpathlineto{\pgfqpoint{-0.048611in}{0.000000in}}%
\pgfusepath{stroke,fill}%
}%
\begin{pgfscope}%
\pgfsys@transformshift{0.794097in}{4.454861in}%
\pgfsys@useobject{currentmarker}{}%
\end{pgfscope}%
\end{pgfscope}%
\begin{pgfscope}%
\definecolor{textcolor}{rgb}{0.000000,0.000000,0.000000}%
\pgfsetstrokecolor{textcolor}%
\pgfsetfillcolor{textcolor}%
\pgftext[x=0.608510in, y=4.402100in, left, base]{\color{textcolor}\sffamily\fontsize{10.000000}{12.000000}\selectfont 3}%
\end{pgfscope}%
\begin{pgfscope}%
\definecolor{textcolor}{rgb}{0.000000,0.000000,0.000000}%
\pgfsetstrokecolor{textcolor}%
\pgfsetfillcolor{textcolor}%
\pgftext[x=0.552954in,y=4.059722in,,bottom,rotate=90.000000]{\color{textcolor}\sffamily\fontsize{30.000000}{36.000000}\selectfont y axis}%
\end{pgfscope}%
\begin{pgfscope}%
\pgfsetrectcap%
\pgfsetmiterjoin%
\pgfsetlinewidth{0.803000pt}%
\definecolor{currentstroke}{rgb}{0.000000,0.000000,0.000000}%
\pgfsetstrokecolor{currentstroke}%
\pgfsetdash{}{0pt}%
\pgfpathmoveto{\pgfqpoint{0.794097in}{3.269444in}}%
\pgfpathlineto{\pgfqpoint{0.794097in}{4.850000in}}%
\pgfusepath{stroke}%
\end{pgfscope}%
\begin{pgfscope}%
\pgfsetrectcap%
\pgfsetmiterjoin%
\pgfsetlinewidth{0.803000pt}%
\definecolor{currentstroke}{rgb}{0.000000,0.000000,0.000000}%
\pgfsetstrokecolor{currentstroke}%
\pgfsetdash{}{0pt}%
\pgfpathmoveto{\pgfqpoint{3.712500in}{3.269444in}}%
\pgfpathlineto{\pgfqpoint{3.712500in}{4.850000in}}%
\pgfusepath{stroke}%
\end{pgfscope}%
\begin{pgfscope}%
\pgfsetrectcap%
\pgfsetmiterjoin%
\pgfsetlinewidth{0.803000pt}%
\definecolor{currentstroke}{rgb}{0.000000,0.000000,0.000000}%
\pgfsetstrokecolor{currentstroke}%
\pgfsetdash{}{0pt}%
\pgfpathmoveto{\pgfqpoint{0.794097in}{3.269444in}}%
\pgfpathlineto{\pgfqpoint{3.712500in}{3.269444in}}%
\pgfusepath{stroke}%
\end{pgfscope}%
\begin{pgfscope}%
\pgfsetrectcap%
\pgfsetmiterjoin%
\pgfsetlinewidth{0.803000pt}%
\definecolor{currentstroke}{rgb}{0.000000,0.000000,0.000000}%
\pgfsetstrokecolor{currentstroke}%
\pgfsetdash{}{0pt}%
\pgfpathmoveto{\pgfqpoint{0.794097in}{4.850000in}}%
\pgfpathlineto{\pgfqpoint{3.712500in}{4.850000in}}%
\pgfusepath{stroke}%
\end{pgfscope}%
\begin{pgfscope}%
\pgfsetbuttcap%
\pgfsetmiterjoin%
\definecolor{currentfill}{rgb}{1.000000,1.000000,1.000000}%
\pgfsetfillcolor{currentfill}%
\pgfsetlinewidth{0.000000pt}%
\definecolor{currentstroke}{rgb}{0.000000,0.000000,0.000000}%
\pgfsetstrokecolor{currentstroke}%
\pgfsetstrokeopacity{0.000000}%
\pgfsetdash{}{0pt}%
\pgfpathmoveto{\pgfqpoint{4.506597in}{3.269444in}}%
\pgfpathlineto{\pgfqpoint{7.425000in}{3.269444in}}%
\pgfpathlineto{\pgfqpoint{7.425000in}{4.850000in}}%
\pgfpathlineto{\pgfqpoint{4.506597in}{4.850000in}}%
\pgfpathlineto{\pgfqpoint{4.506597in}{3.269444in}}%
\pgfpathclose%
\pgfusepath{fill}%
\end{pgfscope}%
\begin{pgfscope}%
\pgfpathrectangle{\pgfqpoint{4.506597in}{3.269444in}}{\pgfqpoint{2.918403in}{1.580556in}}%
\pgfusepath{clip}%
\pgfsetbuttcap%
\pgfsetroundjoin%
\definecolor{currentfill}{rgb}{0.991373,0.791373,0.708235}%
\pgfsetfillcolor{currentfill}%
\pgfsetlinewidth{0.000000pt}%
\definecolor{currentstroke}{rgb}{0.000000,0.000000,0.000000}%
\pgfsetstrokecolor{currentstroke}%
\pgfsetdash{}{0pt}%
\pgfpathmoveto{\pgfqpoint{4.506597in}{3.269444in}}%
\pgfpathlineto{\pgfqpoint{4.506597in}{3.664583in}}%
\pgfpathlineto{\pgfqpoint{5.090278in}{3.664583in}}%
\pgfpathlineto{\pgfqpoint{5.090278in}{3.269444in}}%
\pgfpathlineto{\pgfqpoint{4.506597in}{3.269444in}}%
\pgfpathlineto{\pgfqpoint{4.506597in}{3.269444in}}%
\pgfpathclose%
\pgfusepath{fill}%
\end{pgfscope}%
\begin{pgfscope}%
\pgfpathrectangle{\pgfqpoint{4.506597in}{3.269444in}}{\pgfqpoint{2.918403in}{1.580556in}}%
\pgfusepath{clip}%
\pgfsetbuttcap%
\pgfsetroundjoin%
\definecolor{currentfill}{rgb}{0.985729,0.472280,0.346790}%
\pgfsetfillcolor{currentfill}%
\pgfsetlinewidth{0.000000pt}%
\definecolor{currentstroke}{rgb}{0.000000,0.000000,0.000000}%
\pgfsetstrokecolor{currentstroke}%
\pgfsetdash{}{0pt}%
\pgfpathmoveto{\pgfqpoint{5.090278in}{3.269444in}}%
\pgfpathlineto{\pgfqpoint{5.090278in}{3.664583in}}%
\pgfpathlineto{\pgfqpoint{5.673958in}{3.664583in}}%
\pgfpathlineto{\pgfqpoint{5.673958in}{3.269444in}}%
\pgfpathlineto{\pgfqpoint{5.090278in}{3.269444in}}%
\pgfpathlineto{\pgfqpoint{5.090278in}{3.269444in}}%
\pgfpathclose%
\pgfusepath{fill}%
\end{pgfscope}%
\begin{pgfscope}%
\pgfpathrectangle{\pgfqpoint{4.506597in}{3.269444in}}{\pgfqpoint{2.918403in}{1.580556in}}%
\pgfusepath{clip}%
\pgfsetbuttcap%
\pgfsetroundjoin%
\definecolor{currentfill}{rgb}{0.868051,0.164091,0.143714}%
\pgfsetfillcolor{currentfill}%
\pgfsetlinewidth{0.000000pt}%
\definecolor{currentstroke}{rgb}{0.000000,0.000000,0.000000}%
\pgfsetstrokecolor{currentstroke}%
\pgfsetdash{}{0pt}%
\pgfpathmoveto{\pgfqpoint{5.673958in}{3.269444in}}%
\pgfpathlineto{\pgfqpoint{5.673958in}{3.664583in}}%
\pgfpathlineto{\pgfqpoint{6.257639in}{3.664583in}}%
\pgfpathlineto{\pgfqpoint{6.257639in}{3.269444in}}%
\pgfpathlineto{\pgfqpoint{5.673958in}{3.269444in}}%
\pgfpathlineto{\pgfqpoint{5.673958in}{3.269444in}}%
\pgfpathclose%
\pgfusepath{fill}%
\end{pgfscope}%
\begin{pgfscope}%
\pgfpathrectangle{\pgfqpoint{4.506597in}{3.269444in}}{\pgfqpoint{2.918403in}{1.580556in}}%
\pgfusepath{clip}%
\pgfsetbuttcap%
\pgfsetroundjoin%
\definecolor{currentfill}{rgb}{1.000000,0.960784,0.941176}%
\pgfsetfillcolor{currentfill}%
\pgfsetlinewidth{0.000000pt}%
\definecolor{currentstroke}{rgb}{0.000000,0.000000,0.000000}%
\pgfsetstrokecolor{currentstroke}%
\pgfsetdash{}{0pt}%
\pgfpathmoveto{\pgfqpoint{6.257639in}{3.269444in}}%
\pgfpathlineto{\pgfqpoint{6.257639in}{3.664583in}}%
\pgfpathlineto{\pgfqpoint{6.841319in}{3.664583in}}%
\pgfpathlineto{\pgfqpoint{6.841319in}{3.269444in}}%
\pgfpathlineto{\pgfqpoint{6.257639in}{3.269444in}}%
\pgfpathlineto{\pgfqpoint{6.257639in}{3.269444in}}%
\pgfpathclose%
\pgfusepath{fill}%
\end{pgfscope}%
\begin{pgfscope}%
\pgfpathrectangle{\pgfqpoint{4.506597in}{3.269444in}}{\pgfqpoint{2.918403in}{1.580556in}}%
\pgfusepath{clip}%
\pgfsetbuttcap%
\pgfsetroundjoin%
\definecolor{currentfill}{rgb}{0.403922,0.000000,0.050980}%
\pgfsetfillcolor{currentfill}%
\pgfsetlinewidth{0.000000pt}%
\definecolor{currentstroke}{rgb}{0.000000,0.000000,0.000000}%
\pgfsetstrokecolor{currentstroke}%
\pgfsetdash{}{0pt}%
\pgfpathmoveto{\pgfqpoint{6.841319in}{3.269444in}}%
\pgfpathlineto{\pgfqpoint{6.841319in}{3.664583in}}%
\pgfpathlineto{\pgfqpoint{7.425000in}{3.664583in}}%
\pgfpathlineto{\pgfqpoint{7.425000in}{3.269444in}}%
\pgfpathlineto{\pgfqpoint{6.841319in}{3.269444in}}%
\pgfpathlineto{\pgfqpoint{6.841319in}{3.269444in}}%
\pgfpathclose%
\pgfusepath{fill}%
\end{pgfscope}%
\begin{pgfscope}%
\pgfpathrectangle{\pgfqpoint{4.506597in}{3.269444in}}{\pgfqpoint{2.918403in}{1.580556in}}%
\pgfusepath{clip}%
\pgfsetbuttcap%
\pgfsetroundjoin%
\definecolor{currentfill}{rgb}{0.984744,0.432910,0.307420}%
\pgfsetfillcolor{currentfill}%
\pgfsetlinewidth{0.000000pt}%
\definecolor{currentstroke}{rgb}{0.000000,0.000000,0.000000}%
\pgfsetstrokecolor{currentstroke}%
\pgfsetdash{}{0pt}%
\pgfpathmoveto{\pgfqpoint{4.506597in}{3.664583in}}%
\pgfpathlineto{\pgfqpoint{4.506597in}{4.059722in}}%
\pgfpathlineto{\pgfqpoint{5.090278in}{4.059722in}}%
\pgfpathlineto{\pgfqpoint{5.090278in}{3.664583in}}%
\pgfpathlineto{\pgfqpoint{4.506597in}{3.664583in}}%
\pgfpathlineto{\pgfqpoint{4.506597in}{3.664583in}}%
\pgfpathclose%
\pgfusepath{fill}%
\end{pgfscope}%
\begin{pgfscope}%
\pgfpathrectangle{\pgfqpoint{4.506597in}{3.269444in}}{\pgfqpoint{2.918403in}{1.580556in}}%
\pgfusepath{clip}%
\pgfsetbuttcap%
\pgfsetroundjoin%
\definecolor{currentfill}{rgb}{0.974717,0.378101,0.266205}%
\pgfsetfillcolor{currentfill}%
\pgfsetlinewidth{0.000000pt}%
\definecolor{currentstroke}{rgb}{0.000000,0.000000,0.000000}%
\pgfsetstrokecolor{currentstroke}%
\pgfsetdash{}{0pt}%
\pgfpathmoveto{\pgfqpoint{5.090278in}{3.664583in}}%
\pgfpathlineto{\pgfqpoint{5.090278in}{4.059722in}}%
\pgfpathlineto{\pgfqpoint{5.673958in}{4.059722in}}%
\pgfpathlineto{\pgfqpoint{5.673958in}{3.664583in}}%
\pgfpathlineto{\pgfqpoint{5.090278in}{3.664583in}}%
\pgfpathlineto{\pgfqpoint{5.090278in}{3.664583in}}%
\pgfpathclose%
\pgfusepath{fill}%
\end{pgfscope}%
\begin{pgfscope}%
\pgfpathrectangle{\pgfqpoint{4.506597in}{3.269444in}}{\pgfqpoint{2.918403in}{1.580556in}}%
\pgfusepath{clip}%
\pgfsetbuttcap%
\pgfsetroundjoin%
\definecolor{currentfill}{rgb}{0.988235,0.681630,0.572103}%
\pgfsetfillcolor{currentfill}%
\pgfsetlinewidth{0.000000pt}%
\definecolor{currentstroke}{rgb}{0.000000,0.000000,0.000000}%
\pgfsetstrokecolor{currentstroke}%
\pgfsetdash{}{0pt}%
\pgfpathmoveto{\pgfqpoint{5.673958in}{3.664583in}}%
\pgfpathlineto{\pgfqpoint{5.673958in}{4.059722in}}%
\pgfpathlineto{\pgfqpoint{6.257639in}{4.059722in}}%
\pgfpathlineto{\pgfqpoint{6.257639in}{3.664583in}}%
\pgfpathlineto{\pgfqpoint{5.673958in}{3.664583in}}%
\pgfpathlineto{\pgfqpoint{5.673958in}{3.664583in}}%
\pgfpathclose%
\pgfusepath{fill}%
\end{pgfscope}%
\begin{pgfscope}%
\pgfpathrectangle{\pgfqpoint{4.506597in}{3.269444in}}{\pgfqpoint{2.918403in}{1.580556in}}%
\pgfusepath{clip}%
\pgfsetbuttcap%
\pgfsetroundjoin%
\definecolor{currentfill}{rgb}{0.525967,0.029527,0.066728}%
\pgfsetfillcolor{currentfill}%
\pgfsetlinewidth{0.000000pt}%
\definecolor{currentstroke}{rgb}{0.000000,0.000000,0.000000}%
\pgfsetstrokecolor{currentstroke}%
\pgfsetdash{}{0pt}%
\pgfpathmoveto{\pgfqpoint{6.257639in}{3.664583in}}%
\pgfpathlineto{\pgfqpoint{6.257639in}{4.059722in}}%
\pgfpathlineto{\pgfqpoint{6.841319in}{4.059722in}}%
\pgfpathlineto{\pgfqpoint{6.841319in}{3.664583in}}%
\pgfpathlineto{\pgfqpoint{6.257639in}{3.664583in}}%
\pgfpathlineto{\pgfqpoint{6.257639in}{3.664583in}}%
\pgfpathclose%
\pgfusepath{fill}%
\end{pgfscope}%
\begin{pgfscope}%
\pgfpathrectangle{\pgfqpoint{4.506597in}{3.269444in}}{\pgfqpoint{2.918403in}{1.580556in}}%
\pgfusepath{clip}%
\pgfsetbuttcap%
\pgfsetroundjoin%
\definecolor{currentfill}{rgb}{0.525967,0.029527,0.066728}%
\pgfsetfillcolor{currentfill}%
\pgfsetlinewidth{0.000000pt}%
\definecolor{currentstroke}{rgb}{0.000000,0.000000,0.000000}%
\pgfsetstrokecolor{currentstroke}%
\pgfsetdash{}{0pt}%
\pgfpathmoveto{\pgfqpoint{6.841319in}{3.664583in}}%
\pgfpathlineto{\pgfqpoint{6.841319in}{4.059722in}}%
\pgfpathlineto{\pgfqpoint{7.425000in}{4.059722in}}%
\pgfpathlineto{\pgfqpoint{7.425000in}{3.664583in}}%
\pgfpathlineto{\pgfqpoint{6.841319in}{3.664583in}}%
\pgfpathlineto{\pgfqpoint{6.841319in}{3.664583in}}%
\pgfpathclose%
\pgfusepath{fill}%
\end{pgfscope}%
\begin{pgfscope}%
\pgfpathrectangle{\pgfqpoint{4.506597in}{3.269444in}}{\pgfqpoint{2.918403in}{1.580556in}}%
\pgfusepath{clip}%
\pgfsetbuttcap%
\pgfsetroundjoin%
\definecolor{currentfill}{rgb}{0.974717,0.378101,0.266205}%
\pgfsetfillcolor{currentfill}%
\pgfsetlinewidth{0.000000pt}%
\definecolor{currentstroke}{rgb}{0.000000,0.000000,0.000000}%
\pgfsetstrokecolor{currentstroke}%
\pgfsetdash{}{0pt}%
\pgfpathmoveto{\pgfqpoint{4.506597in}{4.059722in}}%
\pgfpathlineto{\pgfqpoint{4.506597in}{4.454861in}}%
\pgfpathlineto{\pgfqpoint{5.090278in}{4.454861in}}%
\pgfpathlineto{\pgfqpoint{5.090278in}{4.059722in}}%
\pgfpathlineto{\pgfqpoint{4.506597in}{4.059722in}}%
\pgfpathlineto{\pgfqpoint{4.506597in}{4.059722in}}%
\pgfpathclose%
\pgfusepath{fill}%
\end{pgfscope}%
\begin{pgfscope}%
\pgfpathrectangle{\pgfqpoint{4.506597in}{3.269444in}}{\pgfqpoint{2.918403in}{1.580556in}}%
\pgfusepath{clip}%
\pgfsetbuttcap%
\pgfsetroundjoin%
\definecolor{currentfill}{rgb}{0.640384,0.057209,0.081492}%
\pgfsetfillcolor{currentfill}%
\pgfsetlinewidth{0.000000pt}%
\definecolor{currentstroke}{rgb}{0.000000,0.000000,0.000000}%
\pgfsetstrokecolor{currentstroke}%
\pgfsetdash{}{0pt}%
\pgfpathmoveto{\pgfqpoint{5.090278in}{4.059722in}}%
\pgfpathlineto{\pgfqpoint{5.090278in}{4.454861in}}%
\pgfpathlineto{\pgfqpoint{5.673958in}{4.454861in}}%
\pgfpathlineto{\pgfqpoint{5.673958in}{4.059722in}}%
\pgfpathlineto{\pgfqpoint{5.090278in}{4.059722in}}%
\pgfpathlineto{\pgfqpoint{5.090278in}{4.059722in}}%
\pgfpathclose%
\pgfusepath{fill}%
\end{pgfscope}%
\begin{pgfscope}%
\pgfpathrectangle{\pgfqpoint{4.506597in}{3.269444in}}{\pgfqpoint{2.918403in}{1.580556in}}%
\pgfusepath{clip}%
\pgfsetbuttcap%
\pgfsetroundjoin%
\definecolor{currentfill}{rgb}{0.948143,0.274018,0.199769}%
\pgfsetfillcolor{currentfill}%
\pgfsetlinewidth{0.000000pt}%
\definecolor{currentstroke}{rgb}{0.000000,0.000000,0.000000}%
\pgfsetstrokecolor{currentstroke}%
\pgfsetdash{}{0pt}%
\pgfpathmoveto{\pgfqpoint{5.673958in}{4.059722in}}%
\pgfpathlineto{\pgfqpoint{5.673958in}{4.454861in}}%
\pgfpathlineto{\pgfqpoint{6.257639in}{4.454861in}}%
\pgfpathlineto{\pgfqpoint{6.257639in}{4.059722in}}%
\pgfpathlineto{\pgfqpoint{5.673958in}{4.059722in}}%
\pgfpathlineto{\pgfqpoint{5.673958in}{4.059722in}}%
\pgfpathclose%
\pgfusepath{fill}%
\end{pgfscope}%
\begin{pgfscope}%
\pgfpathrectangle{\pgfqpoint{4.506597in}{3.269444in}}{\pgfqpoint{2.918403in}{1.580556in}}%
\pgfusepath{clip}%
\pgfsetbuttcap%
\pgfsetroundjoin%
\definecolor{currentfill}{rgb}{0.666344,0.063391,0.086413}%
\pgfsetfillcolor{currentfill}%
\pgfsetlinewidth{0.000000pt}%
\definecolor{currentstroke}{rgb}{0.000000,0.000000,0.000000}%
\pgfsetstrokecolor{currentstroke}%
\pgfsetdash{}{0pt}%
\pgfpathmoveto{\pgfqpoint{6.257639in}{4.059722in}}%
\pgfpathlineto{\pgfqpoint{6.257639in}{4.454861in}}%
\pgfpathlineto{\pgfqpoint{6.841319in}{4.454861in}}%
\pgfpathlineto{\pgfqpoint{6.841319in}{4.059722in}}%
\pgfpathlineto{\pgfqpoint{6.257639in}{4.059722in}}%
\pgfpathlineto{\pgfqpoint{6.257639in}{4.059722in}}%
\pgfpathclose%
\pgfusepath{fill}%
\end{pgfscope}%
\begin{pgfscope}%
\pgfpathrectangle{\pgfqpoint{4.506597in}{3.269444in}}{\pgfqpoint{2.918403in}{1.580556in}}%
\pgfusepath{clip}%
\pgfsetbuttcap%
\pgfsetroundjoin%
\definecolor{currentfill}{rgb}{0.988235,0.666498,0.554756}%
\pgfsetfillcolor{currentfill}%
\pgfsetlinewidth{0.000000pt}%
\definecolor{currentstroke}{rgb}{0.000000,0.000000,0.000000}%
\pgfsetstrokecolor{currentstroke}%
\pgfsetdash{}{0pt}%
\pgfpathmoveto{\pgfqpoint{6.841319in}{4.059722in}}%
\pgfpathlineto{\pgfqpoint{6.841319in}{4.454861in}}%
\pgfpathlineto{\pgfqpoint{7.425000in}{4.454861in}}%
\pgfpathlineto{\pgfqpoint{7.425000in}{4.059722in}}%
\pgfpathlineto{\pgfqpoint{6.841319in}{4.059722in}}%
\pgfpathlineto{\pgfqpoint{6.841319in}{4.059722in}}%
\pgfpathclose%
\pgfusepath{fill}%
\end{pgfscope}%
\begin{pgfscope}%
\pgfpathrectangle{\pgfqpoint{4.506597in}{3.269444in}}{\pgfqpoint{2.918403in}{1.580556in}}%
\pgfusepath{clip}%
\pgfsetbuttcap%
\pgfsetroundjoin%
\definecolor{currentfill}{rgb}{0.954048,0.297147,0.214533}%
\pgfsetfillcolor{currentfill}%
\pgfsetlinewidth{0.000000pt}%
\definecolor{currentstroke}{rgb}{0.000000,0.000000,0.000000}%
\pgfsetstrokecolor{currentstroke}%
\pgfsetdash{}{0pt}%
\pgfpathmoveto{\pgfqpoint{4.506597in}{4.454861in}}%
\pgfpathlineto{\pgfqpoint{4.506597in}{4.850000in}}%
\pgfpathlineto{\pgfqpoint{5.090278in}{4.850000in}}%
\pgfpathlineto{\pgfqpoint{5.090278in}{4.454861in}}%
\pgfpathlineto{\pgfqpoint{4.506597in}{4.454861in}}%
\pgfpathlineto{\pgfqpoint{4.506597in}{4.454861in}}%
\pgfpathclose%
\pgfusepath{fill}%
\end{pgfscope}%
\begin{pgfscope}%
\pgfpathrectangle{\pgfqpoint{4.506597in}{3.269444in}}{\pgfqpoint{2.918403in}{1.580556in}}%
\pgfusepath{clip}%
\pgfsetbuttcap%
\pgfsetroundjoin%
\definecolor{currentfill}{rgb}{0.988189,0.570704,0.445213}%
\pgfsetfillcolor{currentfill}%
\pgfsetlinewidth{0.000000pt}%
\definecolor{currentstroke}{rgb}{0.000000,0.000000,0.000000}%
\pgfsetstrokecolor{currentstroke}%
\pgfsetdash{}{0pt}%
\pgfpathmoveto{\pgfqpoint{5.090278in}{4.454861in}}%
\pgfpathlineto{\pgfqpoint{5.090278in}{4.850000in}}%
\pgfpathlineto{\pgfqpoint{5.673958in}{4.850000in}}%
\pgfpathlineto{\pgfqpoint{5.673958in}{4.454861in}}%
\pgfpathlineto{\pgfqpoint{5.090278in}{4.454861in}}%
\pgfpathlineto{\pgfqpoint{5.090278in}{4.454861in}}%
\pgfpathclose%
\pgfusepath{fill}%
\end{pgfscope}%
\begin{pgfscope}%
\pgfpathrectangle{\pgfqpoint{4.506597in}{3.269444in}}{\pgfqpoint{2.918403in}{1.580556in}}%
\pgfusepath{clip}%
\pgfsetbuttcap%
\pgfsetroundjoin%
\definecolor{currentfill}{rgb}{0.961430,0.326059,0.232987}%
\pgfsetfillcolor{currentfill}%
\pgfsetlinewidth{0.000000pt}%
\definecolor{currentstroke}{rgb}{0.000000,0.000000,0.000000}%
\pgfsetstrokecolor{currentstroke}%
\pgfsetdash{}{0pt}%
\pgfpathmoveto{\pgfqpoint{5.673958in}{4.454861in}}%
\pgfpathlineto{\pgfqpoint{5.673958in}{4.850000in}}%
\pgfpathlineto{\pgfqpoint{6.257639in}{4.850000in}}%
\pgfpathlineto{\pgfqpoint{6.257639in}{4.454861in}}%
\pgfpathlineto{\pgfqpoint{5.673958in}{4.454861in}}%
\pgfpathlineto{\pgfqpoint{5.673958in}{4.454861in}}%
\pgfpathclose%
\pgfusepath{fill}%
\end{pgfscope}%
\begin{pgfscope}%
\pgfpathrectangle{\pgfqpoint{4.506597in}{3.269444in}}{\pgfqpoint{2.918403in}{1.580556in}}%
\pgfusepath{clip}%
\pgfsetbuttcap%
\pgfsetroundjoin%
\definecolor{currentfill}{rgb}{0.990388,0.773164,0.684121}%
\pgfsetfillcolor{currentfill}%
\pgfsetlinewidth{0.000000pt}%
\definecolor{currentstroke}{rgb}{0.000000,0.000000,0.000000}%
\pgfsetstrokecolor{currentstroke}%
\pgfsetdash{}{0pt}%
\pgfpathmoveto{\pgfqpoint{6.257639in}{4.454861in}}%
\pgfpathlineto{\pgfqpoint{6.257639in}{4.850000in}}%
\pgfpathlineto{\pgfqpoint{6.841319in}{4.850000in}}%
\pgfpathlineto{\pgfqpoint{6.841319in}{4.454861in}}%
\pgfpathlineto{\pgfqpoint{6.257639in}{4.454861in}}%
\pgfpathlineto{\pgfqpoint{6.257639in}{4.454861in}}%
\pgfpathclose%
\pgfusepath{fill}%
\end{pgfscope}%
\begin{pgfscope}%
\pgfpathrectangle{\pgfqpoint{4.506597in}{3.269444in}}{\pgfqpoint{2.918403in}{1.580556in}}%
\pgfusepath{clip}%
\pgfsetbuttcap%
\pgfsetroundjoin%
\definecolor{currentfill}{rgb}{0.988235,0.575702,0.450673}%
\pgfsetfillcolor{currentfill}%
\pgfsetlinewidth{0.000000pt}%
\definecolor{currentstroke}{rgb}{0.000000,0.000000,0.000000}%
\pgfsetstrokecolor{currentstroke}%
\pgfsetdash{}{0pt}%
\pgfpathmoveto{\pgfqpoint{6.841319in}{4.454861in}}%
\pgfpathlineto{\pgfqpoint{6.841319in}{4.850000in}}%
\pgfpathlineto{\pgfqpoint{7.425000in}{4.850000in}}%
\pgfpathlineto{\pgfqpoint{7.425000in}{4.454861in}}%
\pgfpathlineto{\pgfqpoint{6.841319in}{4.454861in}}%
\pgfpathlineto{\pgfqpoint{6.841319in}{4.454861in}}%
\pgfpathclose%
\pgfusepath{fill}%
\end{pgfscope}%
\begin{pgfscope}%
\pgfsetbuttcap%
\pgfsetroundjoin%
\definecolor{currentfill}{rgb}{0.000000,0.000000,0.000000}%
\pgfsetfillcolor{currentfill}%
\pgfsetlinewidth{0.803000pt}%
\definecolor{currentstroke}{rgb}{0.000000,0.000000,0.000000}%
\pgfsetstrokecolor{currentstroke}%
\pgfsetdash{}{0pt}%
\pgfsys@defobject{currentmarker}{\pgfqpoint{0.000000in}{-0.048611in}}{\pgfqpoint{0.000000in}{0.000000in}}{%
\pgfpathmoveto{\pgfqpoint{0.000000in}{0.000000in}}%
\pgfpathlineto{\pgfqpoint{0.000000in}{-0.048611in}}%
\pgfusepath{stroke,fill}%
}%
\begin{pgfscope}%
\pgfsys@transformshift{4.506597in}{3.269444in}%
\pgfsys@useobject{currentmarker}{}%
\end{pgfscope}%
\end{pgfscope}%
\begin{pgfscope}%
\definecolor{textcolor}{rgb}{0.000000,0.000000,0.000000}%
\pgfsetstrokecolor{textcolor}%
\pgfsetfillcolor{textcolor}%
\pgftext[x=4.506597in,y=3.172222in,,top]{\color{textcolor}\sffamily\fontsize{10.000000}{12.000000}\selectfont 0}%
\end{pgfscope}%
\begin{pgfscope}%
\pgfsetbuttcap%
\pgfsetroundjoin%
\definecolor{currentfill}{rgb}{0.000000,0.000000,0.000000}%
\pgfsetfillcolor{currentfill}%
\pgfsetlinewidth{0.803000pt}%
\definecolor{currentstroke}{rgb}{0.000000,0.000000,0.000000}%
\pgfsetstrokecolor{currentstroke}%
\pgfsetdash{}{0pt}%
\pgfsys@defobject{currentmarker}{\pgfqpoint{0.000000in}{-0.048611in}}{\pgfqpoint{0.000000in}{0.000000in}}{%
\pgfpathmoveto{\pgfqpoint{0.000000in}{0.000000in}}%
\pgfpathlineto{\pgfqpoint{0.000000in}{-0.048611in}}%
\pgfusepath{stroke,fill}%
}%
\begin{pgfscope}%
\pgfsys@transformshift{5.090278in}{3.269444in}%
\pgfsys@useobject{currentmarker}{}%
\end{pgfscope}%
\end{pgfscope}%
\begin{pgfscope}%
\definecolor{textcolor}{rgb}{0.000000,0.000000,0.000000}%
\pgfsetstrokecolor{textcolor}%
\pgfsetfillcolor{textcolor}%
\pgftext[x=5.090278in,y=3.172222in,,top]{\color{textcolor}\sffamily\fontsize{10.000000}{12.000000}\selectfont 1}%
\end{pgfscope}%
\begin{pgfscope}%
\pgfsetbuttcap%
\pgfsetroundjoin%
\definecolor{currentfill}{rgb}{0.000000,0.000000,0.000000}%
\pgfsetfillcolor{currentfill}%
\pgfsetlinewidth{0.803000pt}%
\definecolor{currentstroke}{rgb}{0.000000,0.000000,0.000000}%
\pgfsetstrokecolor{currentstroke}%
\pgfsetdash{}{0pt}%
\pgfsys@defobject{currentmarker}{\pgfqpoint{0.000000in}{-0.048611in}}{\pgfqpoint{0.000000in}{0.000000in}}{%
\pgfpathmoveto{\pgfqpoint{0.000000in}{0.000000in}}%
\pgfpathlineto{\pgfqpoint{0.000000in}{-0.048611in}}%
\pgfusepath{stroke,fill}%
}%
\begin{pgfscope}%
\pgfsys@transformshift{5.673958in}{3.269444in}%
\pgfsys@useobject{currentmarker}{}%
\end{pgfscope}%
\end{pgfscope}%
\begin{pgfscope}%
\definecolor{textcolor}{rgb}{0.000000,0.000000,0.000000}%
\pgfsetstrokecolor{textcolor}%
\pgfsetfillcolor{textcolor}%
\pgftext[x=5.673958in,y=3.172222in,,top]{\color{textcolor}\sffamily\fontsize{10.000000}{12.000000}\selectfont 2}%
\end{pgfscope}%
\begin{pgfscope}%
\pgfsetbuttcap%
\pgfsetroundjoin%
\definecolor{currentfill}{rgb}{0.000000,0.000000,0.000000}%
\pgfsetfillcolor{currentfill}%
\pgfsetlinewidth{0.803000pt}%
\definecolor{currentstroke}{rgb}{0.000000,0.000000,0.000000}%
\pgfsetstrokecolor{currentstroke}%
\pgfsetdash{}{0pt}%
\pgfsys@defobject{currentmarker}{\pgfqpoint{0.000000in}{-0.048611in}}{\pgfqpoint{0.000000in}{0.000000in}}{%
\pgfpathmoveto{\pgfqpoint{0.000000in}{0.000000in}}%
\pgfpathlineto{\pgfqpoint{0.000000in}{-0.048611in}}%
\pgfusepath{stroke,fill}%
}%
\begin{pgfscope}%
\pgfsys@transformshift{6.257639in}{3.269444in}%
\pgfsys@useobject{currentmarker}{}%
\end{pgfscope}%
\end{pgfscope}%
\begin{pgfscope}%
\definecolor{textcolor}{rgb}{0.000000,0.000000,0.000000}%
\pgfsetstrokecolor{textcolor}%
\pgfsetfillcolor{textcolor}%
\pgftext[x=6.257639in,y=3.172222in,,top]{\color{textcolor}\sffamily\fontsize{10.000000}{12.000000}\selectfont 3}%
\end{pgfscope}%
\begin{pgfscope}%
\pgfsetbuttcap%
\pgfsetroundjoin%
\definecolor{currentfill}{rgb}{0.000000,0.000000,0.000000}%
\pgfsetfillcolor{currentfill}%
\pgfsetlinewidth{0.803000pt}%
\definecolor{currentstroke}{rgb}{0.000000,0.000000,0.000000}%
\pgfsetstrokecolor{currentstroke}%
\pgfsetdash{}{0pt}%
\pgfsys@defobject{currentmarker}{\pgfqpoint{0.000000in}{-0.048611in}}{\pgfqpoint{0.000000in}{0.000000in}}{%
\pgfpathmoveto{\pgfqpoint{0.000000in}{0.000000in}}%
\pgfpathlineto{\pgfqpoint{0.000000in}{-0.048611in}}%
\pgfusepath{stroke,fill}%
}%
\begin{pgfscope}%
\pgfsys@transformshift{6.841319in}{3.269444in}%
\pgfsys@useobject{currentmarker}{}%
\end{pgfscope}%
\end{pgfscope}%
\begin{pgfscope}%
\definecolor{textcolor}{rgb}{0.000000,0.000000,0.000000}%
\pgfsetstrokecolor{textcolor}%
\pgfsetfillcolor{textcolor}%
\pgftext[x=6.841319in,y=3.172222in,,top]{\color{textcolor}\sffamily\fontsize{10.000000}{12.000000}\selectfont 4}%
\end{pgfscope}%
\begin{pgfscope}%
\definecolor{textcolor}{rgb}{0.000000,0.000000,0.000000}%
\pgfsetstrokecolor{textcolor}%
\pgfsetfillcolor{textcolor}%
\pgftext[x=5.965799in,y=2.982254in,,top]{\color{textcolor}\sffamily\fontsize{30.000000}{36.000000}\selectfont x axis}%
\end{pgfscope}%
\begin{pgfscope}%
\pgfsetbuttcap%
\pgfsetroundjoin%
\definecolor{currentfill}{rgb}{0.000000,0.000000,0.000000}%
\pgfsetfillcolor{currentfill}%
\pgfsetlinewidth{0.803000pt}%
\definecolor{currentstroke}{rgb}{0.000000,0.000000,0.000000}%
\pgfsetstrokecolor{currentstroke}%
\pgfsetdash{}{0pt}%
\pgfsys@defobject{currentmarker}{\pgfqpoint{-0.048611in}{0.000000in}}{\pgfqpoint{-0.000000in}{0.000000in}}{%
\pgfpathmoveto{\pgfqpoint{-0.000000in}{0.000000in}}%
\pgfpathlineto{\pgfqpoint{-0.048611in}{0.000000in}}%
\pgfusepath{stroke,fill}%
}%
\begin{pgfscope}%
\pgfsys@transformshift{4.506597in}{3.269444in}%
\pgfsys@useobject{currentmarker}{}%
\end{pgfscope}%
\end{pgfscope}%
\begin{pgfscope}%
\definecolor{textcolor}{rgb}{0.000000,0.000000,0.000000}%
\pgfsetstrokecolor{textcolor}%
\pgfsetfillcolor{textcolor}%
\pgftext[x=4.321010in, y=3.216683in, left, base]{\color{textcolor}\sffamily\fontsize{10.000000}{12.000000}\selectfont 0}%
\end{pgfscope}%
\begin{pgfscope}%
\pgfsetbuttcap%
\pgfsetroundjoin%
\definecolor{currentfill}{rgb}{0.000000,0.000000,0.000000}%
\pgfsetfillcolor{currentfill}%
\pgfsetlinewidth{0.803000pt}%
\definecolor{currentstroke}{rgb}{0.000000,0.000000,0.000000}%
\pgfsetstrokecolor{currentstroke}%
\pgfsetdash{}{0pt}%
\pgfsys@defobject{currentmarker}{\pgfqpoint{-0.048611in}{0.000000in}}{\pgfqpoint{-0.000000in}{0.000000in}}{%
\pgfpathmoveto{\pgfqpoint{-0.000000in}{0.000000in}}%
\pgfpathlineto{\pgfqpoint{-0.048611in}{0.000000in}}%
\pgfusepath{stroke,fill}%
}%
\begin{pgfscope}%
\pgfsys@transformshift{4.506597in}{3.664583in}%
\pgfsys@useobject{currentmarker}{}%
\end{pgfscope}%
\end{pgfscope}%
\begin{pgfscope}%
\definecolor{textcolor}{rgb}{0.000000,0.000000,0.000000}%
\pgfsetstrokecolor{textcolor}%
\pgfsetfillcolor{textcolor}%
\pgftext[x=4.321010in, y=3.611822in, left, base]{\color{textcolor}\sffamily\fontsize{10.000000}{12.000000}\selectfont 1}%
\end{pgfscope}%
\begin{pgfscope}%
\pgfsetbuttcap%
\pgfsetroundjoin%
\definecolor{currentfill}{rgb}{0.000000,0.000000,0.000000}%
\pgfsetfillcolor{currentfill}%
\pgfsetlinewidth{0.803000pt}%
\definecolor{currentstroke}{rgb}{0.000000,0.000000,0.000000}%
\pgfsetstrokecolor{currentstroke}%
\pgfsetdash{}{0pt}%
\pgfsys@defobject{currentmarker}{\pgfqpoint{-0.048611in}{0.000000in}}{\pgfqpoint{-0.000000in}{0.000000in}}{%
\pgfpathmoveto{\pgfqpoint{-0.000000in}{0.000000in}}%
\pgfpathlineto{\pgfqpoint{-0.048611in}{0.000000in}}%
\pgfusepath{stroke,fill}%
}%
\begin{pgfscope}%
\pgfsys@transformshift{4.506597in}{4.059722in}%
\pgfsys@useobject{currentmarker}{}%
\end{pgfscope}%
\end{pgfscope}%
\begin{pgfscope}%
\definecolor{textcolor}{rgb}{0.000000,0.000000,0.000000}%
\pgfsetstrokecolor{textcolor}%
\pgfsetfillcolor{textcolor}%
\pgftext[x=4.321010in, y=4.006961in, left, base]{\color{textcolor}\sffamily\fontsize{10.000000}{12.000000}\selectfont 2}%
\end{pgfscope}%
\begin{pgfscope}%
\pgfsetbuttcap%
\pgfsetroundjoin%
\definecolor{currentfill}{rgb}{0.000000,0.000000,0.000000}%
\pgfsetfillcolor{currentfill}%
\pgfsetlinewidth{0.803000pt}%
\definecolor{currentstroke}{rgb}{0.000000,0.000000,0.000000}%
\pgfsetstrokecolor{currentstroke}%
\pgfsetdash{}{0pt}%
\pgfsys@defobject{currentmarker}{\pgfqpoint{-0.048611in}{0.000000in}}{\pgfqpoint{-0.000000in}{0.000000in}}{%
\pgfpathmoveto{\pgfqpoint{-0.000000in}{0.000000in}}%
\pgfpathlineto{\pgfqpoint{-0.048611in}{0.000000in}}%
\pgfusepath{stroke,fill}%
}%
\begin{pgfscope}%
\pgfsys@transformshift{4.506597in}{4.454861in}%
\pgfsys@useobject{currentmarker}{}%
\end{pgfscope}%
\end{pgfscope}%
\begin{pgfscope}%
\definecolor{textcolor}{rgb}{0.000000,0.000000,0.000000}%
\pgfsetstrokecolor{textcolor}%
\pgfsetfillcolor{textcolor}%
\pgftext[x=4.321010in, y=4.402100in, left, base]{\color{textcolor}\sffamily\fontsize{10.000000}{12.000000}\selectfont 3}%
\end{pgfscope}%
\begin{pgfscope}%
\definecolor{textcolor}{rgb}{0.000000,0.000000,0.000000}%
\pgfsetstrokecolor{textcolor}%
\pgfsetfillcolor{textcolor}%
\pgftext[x=4.265454in,y=4.059722in,,bottom,rotate=90.000000]{\color{textcolor}\sffamily\fontsize{30.000000}{36.000000}\selectfont y axis}%
\end{pgfscope}%
\begin{pgfscope}%
\pgfsetrectcap%
\pgfsetmiterjoin%
\pgfsetlinewidth{0.803000pt}%
\definecolor{currentstroke}{rgb}{0.000000,0.000000,0.000000}%
\pgfsetstrokecolor{currentstroke}%
\pgfsetdash{}{0pt}%
\pgfpathmoveto{\pgfqpoint{4.506597in}{3.269444in}}%
\pgfpathlineto{\pgfqpoint{4.506597in}{4.850000in}}%
\pgfusepath{stroke}%
\end{pgfscope}%
\begin{pgfscope}%
\pgfsetrectcap%
\pgfsetmiterjoin%
\pgfsetlinewidth{0.803000pt}%
\definecolor{currentstroke}{rgb}{0.000000,0.000000,0.000000}%
\pgfsetstrokecolor{currentstroke}%
\pgfsetdash{}{0pt}%
\pgfpathmoveto{\pgfqpoint{7.425000in}{3.269444in}}%
\pgfpathlineto{\pgfqpoint{7.425000in}{4.850000in}}%
\pgfusepath{stroke}%
\end{pgfscope}%
\begin{pgfscope}%
\pgfsetrectcap%
\pgfsetmiterjoin%
\pgfsetlinewidth{0.803000pt}%
\definecolor{currentstroke}{rgb}{0.000000,0.000000,0.000000}%
\pgfsetstrokecolor{currentstroke}%
\pgfsetdash{}{0pt}%
\pgfpathmoveto{\pgfqpoint{4.506597in}{3.269444in}}%
\pgfpathlineto{\pgfqpoint{7.425000in}{3.269444in}}%
\pgfusepath{stroke}%
\end{pgfscope}%
\begin{pgfscope}%
\pgfsetrectcap%
\pgfsetmiterjoin%
\pgfsetlinewidth{0.803000pt}%
\definecolor{currentstroke}{rgb}{0.000000,0.000000,0.000000}%
\pgfsetstrokecolor{currentstroke}%
\pgfsetdash{}{0pt}%
\pgfpathmoveto{\pgfqpoint{4.506597in}{4.850000in}}%
\pgfpathlineto{\pgfqpoint{7.425000in}{4.850000in}}%
\pgfusepath{stroke}%
\end{pgfscope}%
\begin{pgfscope}%
\pgfsetbuttcap%
\pgfsetmiterjoin%
\definecolor{currentfill}{rgb}{1.000000,1.000000,1.000000}%
\pgfsetfillcolor{currentfill}%
\pgfsetlinewidth{0.000000pt}%
\definecolor{currentstroke}{rgb}{0.000000,0.000000,0.000000}%
\pgfsetstrokecolor{currentstroke}%
\pgfsetstrokeopacity{0.000000}%
\pgfsetdash{}{0pt}%
\pgfpathmoveto{\pgfqpoint{8.219097in}{3.269444in}}%
\pgfpathlineto{\pgfqpoint{11.137500in}{3.269444in}}%
\pgfpathlineto{\pgfqpoint{11.137500in}{4.850000in}}%
\pgfpathlineto{\pgfqpoint{8.219097in}{4.850000in}}%
\pgfpathlineto{\pgfqpoint{8.219097in}{3.269444in}}%
\pgfpathclose%
\pgfusepath{fill}%
\end{pgfscope}%
\begin{pgfscope}%
\pgfpathrectangle{\pgfqpoint{8.219097in}{3.269444in}}{\pgfqpoint{2.918403in}{1.580556in}}%
\pgfusepath{clip}%
\pgfsetbuttcap%
\pgfsetroundjoin%
\definecolor{currentfill}{rgb}{0.954048,0.297147,0.214533}%
\pgfsetfillcolor{currentfill}%
\pgfsetlinewidth{0.000000pt}%
\definecolor{currentstroke}{rgb}{0.000000,0.000000,0.000000}%
\pgfsetstrokecolor{currentstroke}%
\pgfsetdash{}{0pt}%
\pgfpathmoveto{\pgfqpoint{8.219097in}{3.269444in}}%
\pgfpathlineto{\pgfqpoint{8.219097in}{3.664583in}}%
\pgfpathlineto{\pgfqpoint{8.802778in}{3.664583in}}%
\pgfpathlineto{\pgfqpoint{8.802778in}{3.269444in}}%
\pgfpathlineto{\pgfqpoint{8.219097in}{3.269444in}}%
\pgfpathlineto{\pgfqpoint{8.219097in}{3.269444in}}%
\pgfpathclose%
\pgfusepath{fill}%
\end{pgfscope}%
\begin{pgfscope}%
\pgfpathrectangle{\pgfqpoint{8.219097in}{3.269444in}}{\pgfqpoint{2.918403in}{1.580556in}}%
\pgfusepath{clip}%
\pgfsetbuttcap%
\pgfsetroundjoin%
\definecolor{currentfill}{rgb}{0.988189,0.570704,0.445213}%
\pgfsetfillcolor{currentfill}%
\pgfsetlinewidth{0.000000pt}%
\definecolor{currentstroke}{rgb}{0.000000,0.000000,0.000000}%
\pgfsetstrokecolor{currentstroke}%
\pgfsetdash{}{0pt}%
\pgfpathmoveto{\pgfqpoint{8.802778in}{3.269444in}}%
\pgfpathlineto{\pgfqpoint{8.802778in}{3.664583in}}%
\pgfpathlineto{\pgfqpoint{9.386458in}{3.664583in}}%
\pgfpathlineto{\pgfqpoint{9.386458in}{3.269444in}}%
\pgfpathlineto{\pgfqpoint{8.802778in}{3.269444in}}%
\pgfpathlineto{\pgfqpoint{8.802778in}{3.269444in}}%
\pgfpathclose%
\pgfusepath{fill}%
\end{pgfscope}%
\begin{pgfscope}%
\pgfpathrectangle{\pgfqpoint{8.219097in}{3.269444in}}{\pgfqpoint{2.918403in}{1.580556in}}%
\pgfusepath{clip}%
\pgfsetbuttcap%
\pgfsetroundjoin%
\definecolor{currentfill}{rgb}{0.961430,0.326059,0.232987}%
\pgfsetfillcolor{currentfill}%
\pgfsetlinewidth{0.000000pt}%
\definecolor{currentstroke}{rgb}{0.000000,0.000000,0.000000}%
\pgfsetstrokecolor{currentstroke}%
\pgfsetdash{}{0pt}%
\pgfpathmoveto{\pgfqpoint{9.386458in}{3.269444in}}%
\pgfpathlineto{\pgfqpoint{9.386458in}{3.664583in}}%
\pgfpathlineto{\pgfqpoint{9.970139in}{3.664583in}}%
\pgfpathlineto{\pgfqpoint{9.970139in}{3.269444in}}%
\pgfpathlineto{\pgfqpoint{9.386458in}{3.269444in}}%
\pgfpathlineto{\pgfqpoint{9.386458in}{3.269444in}}%
\pgfpathclose%
\pgfusepath{fill}%
\end{pgfscope}%
\begin{pgfscope}%
\pgfpathrectangle{\pgfqpoint{8.219097in}{3.269444in}}{\pgfqpoint{2.918403in}{1.580556in}}%
\pgfusepath{clip}%
\pgfsetbuttcap%
\pgfsetroundjoin%
\definecolor{currentfill}{rgb}{0.990388,0.773164,0.684121}%
\pgfsetfillcolor{currentfill}%
\pgfsetlinewidth{0.000000pt}%
\definecolor{currentstroke}{rgb}{0.000000,0.000000,0.000000}%
\pgfsetstrokecolor{currentstroke}%
\pgfsetdash{}{0pt}%
\pgfpathmoveto{\pgfqpoint{9.970139in}{3.269444in}}%
\pgfpathlineto{\pgfqpoint{9.970139in}{3.664583in}}%
\pgfpathlineto{\pgfqpoint{10.553819in}{3.664583in}}%
\pgfpathlineto{\pgfqpoint{10.553819in}{3.269444in}}%
\pgfpathlineto{\pgfqpoint{9.970139in}{3.269444in}}%
\pgfpathlineto{\pgfqpoint{9.970139in}{3.269444in}}%
\pgfpathclose%
\pgfusepath{fill}%
\end{pgfscope}%
\begin{pgfscope}%
\pgfpathrectangle{\pgfqpoint{8.219097in}{3.269444in}}{\pgfqpoint{2.918403in}{1.580556in}}%
\pgfusepath{clip}%
\pgfsetbuttcap%
\pgfsetroundjoin%
\definecolor{currentfill}{rgb}{0.988235,0.575702,0.450673}%
\pgfsetfillcolor{currentfill}%
\pgfsetlinewidth{0.000000pt}%
\definecolor{currentstroke}{rgb}{0.000000,0.000000,0.000000}%
\pgfsetstrokecolor{currentstroke}%
\pgfsetdash{}{0pt}%
\pgfpathmoveto{\pgfqpoint{10.553819in}{3.269444in}}%
\pgfpathlineto{\pgfqpoint{10.553819in}{3.664583in}}%
\pgfpathlineto{\pgfqpoint{11.137500in}{3.664583in}}%
\pgfpathlineto{\pgfqpoint{11.137500in}{3.269444in}}%
\pgfpathlineto{\pgfqpoint{10.553819in}{3.269444in}}%
\pgfpathlineto{\pgfqpoint{10.553819in}{3.269444in}}%
\pgfpathclose%
\pgfusepath{fill}%
\end{pgfscope}%
\begin{pgfscope}%
\pgfpathrectangle{\pgfqpoint{8.219097in}{3.269444in}}{\pgfqpoint{2.918403in}{1.580556in}}%
\pgfusepath{clip}%
\pgfsetbuttcap%
\pgfsetroundjoin%
\definecolor{currentfill}{rgb}{0.991373,0.791373,0.708235}%
\pgfsetfillcolor{currentfill}%
\pgfsetlinewidth{0.000000pt}%
\definecolor{currentstroke}{rgb}{0.000000,0.000000,0.000000}%
\pgfsetstrokecolor{currentstroke}%
\pgfsetdash{}{0pt}%
\pgfpathmoveto{\pgfqpoint{8.219097in}{3.664583in}}%
\pgfpathlineto{\pgfqpoint{8.219097in}{4.059722in}}%
\pgfpathlineto{\pgfqpoint{8.802778in}{4.059722in}}%
\pgfpathlineto{\pgfqpoint{8.802778in}{3.664583in}}%
\pgfpathlineto{\pgfqpoint{8.219097in}{3.664583in}}%
\pgfpathlineto{\pgfqpoint{8.219097in}{3.664583in}}%
\pgfpathclose%
\pgfusepath{fill}%
\end{pgfscope}%
\begin{pgfscope}%
\pgfpathrectangle{\pgfqpoint{8.219097in}{3.269444in}}{\pgfqpoint{2.918403in}{1.580556in}}%
\pgfusepath{clip}%
\pgfsetbuttcap%
\pgfsetroundjoin%
\definecolor{currentfill}{rgb}{0.985729,0.472280,0.346790}%
\pgfsetfillcolor{currentfill}%
\pgfsetlinewidth{0.000000pt}%
\definecolor{currentstroke}{rgb}{0.000000,0.000000,0.000000}%
\pgfsetstrokecolor{currentstroke}%
\pgfsetdash{}{0pt}%
\pgfpathmoveto{\pgfqpoint{8.802778in}{3.664583in}}%
\pgfpathlineto{\pgfqpoint{8.802778in}{4.059722in}}%
\pgfpathlineto{\pgfqpoint{9.386458in}{4.059722in}}%
\pgfpathlineto{\pgfqpoint{9.386458in}{3.664583in}}%
\pgfpathlineto{\pgfqpoint{8.802778in}{3.664583in}}%
\pgfpathlineto{\pgfqpoint{8.802778in}{3.664583in}}%
\pgfpathclose%
\pgfusepath{fill}%
\end{pgfscope}%
\begin{pgfscope}%
\pgfpathrectangle{\pgfqpoint{8.219097in}{3.269444in}}{\pgfqpoint{2.918403in}{1.580556in}}%
\pgfusepath{clip}%
\pgfsetbuttcap%
\pgfsetroundjoin%
\definecolor{currentfill}{rgb}{0.868051,0.164091,0.143714}%
\pgfsetfillcolor{currentfill}%
\pgfsetlinewidth{0.000000pt}%
\definecolor{currentstroke}{rgb}{0.000000,0.000000,0.000000}%
\pgfsetstrokecolor{currentstroke}%
\pgfsetdash{}{0pt}%
\pgfpathmoveto{\pgfqpoint{9.386458in}{3.664583in}}%
\pgfpathlineto{\pgfqpoint{9.386458in}{4.059722in}}%
\pgfpathlineto{\pgfqpoint{9.970139in}{4.059722in}}%
\pgfpathlineto{\pgfqpoint{9.970139in}{3.664583in}}%
\pgfpathlineto{\pgfqpoint{9.386458in}{3.664583in}}%
\pgfpathlineto{\pgfqpoint{9.386458in}{3.664583in}}%
\pgfpathclose%
\pgfusepath{fill}%
\end{pgfscope}%
\begin{pgfscope}%
\pgfpathrectangle{\pgfqpoint{8.219097in}{3.269444in}}{\pgfqpoint{2.918403in}{1.580556in}}%
\pgfusepath{clip}%
\pgfsetbuttcap%
\pgfsetroundjoin%
\definecolor{currentfill}{rgb}{1.000000,0.960784,0.941176}%
\pgfsetfillcolor{currentfill}%
\pgfsetlinewidth{0.000000pt}%
\definecolor{currentstroke}{rgb}{0.000000,0.000000,0.000000}%
\pgfsetstrokecolor{currentstroke}%
\pgfsetdash{}{0pt}%
\pgfpathmoveto{\pgfqpoint{9.970139in}{3.664583in}}%
\pgfpathlineto{\pgfqpoint{9.970139in}{4.059722in}}%
\pgfpathlineto{\pgfqpoint{10.553819in}{4.059722in}}%
\pgfpathlineto{\pgfqpoint{10.553819in}{3.664583in}}%
\pgfpathlineto{\pgfqpoint{9.970139in}{3.664583in}}%
\pgfpathlineto{\pgfqpoint{9.970139in}{3.664583in}}%
\pgfpathclose%
\pgfusepath{fill}%
\end{pgfscope}%
\begin{pgfscope}%
\pgfpathrectangle{\pgfqpoint{8.219097in}{3.269444in}}{\pgfqpoint{2.918403in}{1.580556in}}%
\pgfusepath{clip}%
\pgfsetbuttcap%
\pgfsetroundjoin%
\definecolor{currentfill}{rgb}{0.403922,0.000000,0.050980}%
\pgfsetfillcolor{currentfill}%
\pgfsetlinewidth{0.000000pt}%
\definecolor{currentstroke}{rgb}{0.000000,0.000000,0.000000}%
\pgfsetstrokecolor{currentstroke}%
\pgfsetdash{}{0pt}%
\pgfpathmoveto{\pgfqpoint{10.553819in}{3.664583in}}%
\pgfpathlineto{\pgfqpoint{10.553819in}{4.059722in}}%
\pgfpathlineto{\pgfqpoint{11.137500in}{4.059722in}}%
\pgfpathlineto{\pgfqpoint{11.137500in}{3.664583in}}%
\pgfpathlineto{\pgfqpoint{10.553819in}{3.664583in}}%
\pgfpathlineto{\pgfqpoint{10.553819in}{3.664583in}}%
\pgfpathclose%
\pgfusepath{fill}%
\end{pgfscope}%
\begin{pgfscope}%
\pgfpathrectangle{\pgfqpoint{8.219097in}{3.269444in}}{\pgfqpoint{2.918403in}{1.580556in}}%
\pgfusepath{clip}%
\pgfsetbuttcap%
\pgfsetroundjoin%
\definecolor{currentfill}{rgb}{0.984744,0.432910,0.307420}%
\pgfsetfillcolor{currentfill}%
\pgfsetlinewidth{0.000000pt}%
\definecolor{currentstroke}{rgb}{0.000000,0.000000,0.000000}%
\pgfsetstrokecolor{currentstroke}%
\pgfsetdash{}{0pt}%
\pgfpathmoveto{\pgfqpoint{8.219097in}{4.059722in}}%
\pgfpathlineto{\pgfqpoint{8.219097in}{4.454861in}}%
\pgfpathlineto{\pgfqpoint{8.802778in}{4.454861in}}%
\pgfpathlineto{\pgfqpoint{8.802778in}{4.059722in}}%
\pgfpathlineto{\pgfqpoint{8.219097in}{4.059722in}}%
\pgfpathlineto{\pgfqpoint{8.219097in}{4.059722in}}%
\pgfpathclose%
\pgfusepath{fill}%
\end{pgfscope}%
\begin{pgfscope}%
\pgfpathrectangle{\pgfqpoint{8.219097in}{3.269444in}}{\pgfqpoint{2.918403in}{1.580556in}}%
\pgfusepath{clip}%
\pgfsetbuttcap%
\pgfsetroundjoin%
\definecolor{currentfill}{rgb}{0.974717,0.378101,0.266205}%
\pgfsetfillcolor{currentfill}%
\pgfsetlinewidth{0.000000pt}%
\definecolor{currentstroke}{rgb}{0.000000,0.000000,0.000000}%
\pgfsetstrokecolor{currentstroke}%
\pgfsetdash{}{0pt}%
\pgfpathmoveto{\pgfqpoint{8.802778in}{4.059722in}}%
\pgfpathlineto{\pgfqpoint{8.802778in}{4.454861in}}%
\pgfpathlineto{\pgfqpoint{9.386458in}{4.454861in}}%
\pgfpathlineto{\pgfqpoint{9.386458in}{4.059722in}}%
\pgfpathlineto{\pgfqpoint{8.802778in}{4.059722in}}%
\pgfpathlineto{\pgfqpoint{8.802778in}{4.059722in}}%
\pgfpathclose%
\pgfusepath{fill}%
\end{pgfscope}%
\begin{pgfscope}%
\pgfpathrectangle{\pgfqpoint{8.219097in}{3.269444in}}{\pgfqpoint{2.918403in}{1.580556in}}%
\pgfusepath{clip}%
\pgfsetbuttcap%
\pgfsetroundjoin%
\definecolor{currentfill}{rgb}{0.988235,0.681630,0.572103}%
\pgfsetfillcolor{currentfill}%
\pgfsetlinewidth{0.000000pt}%
\definecolor{currentstroke}{rgb}{0.000000,0.000000,0.000000}%
\pgfsetstrokecolor{currentstroke}%
\pgfsetdash{}{0pt}%
\pgfpathmoveto{\pgfqpoint{9.386458in}{4.059722in}}%
\pgfpathlineto{\pgfqpoint{9.386458in}{4.454861in}}%
\pgfpathlineto{\pgfqpoint{9.970139in}{4.454861in}}%
\pgfpathlineto{\pgfqpoint{9.970139in}{4.059722in}}%
\pgfpathlineto{\pgfqpoint{9.386458in}{4.059722in}}%
\pgfpathlineto{\pgfqpoint{9.386458in}{4.059722in}}%
\pgfpathclose%
\pgfusepath{fill}%
\end{pgfscope}%
\begin{pgfscope}%
\pgfpathrectangle{\pgfqpoint{8.219097in}{3.269444in}}{\pgfqpoint{2.918403in}{1.580556in}}%
\pgfusepath{clip}%
\pgfsetbuttcap%
\pgfsetroundjoin%
\definecolor{currentfill}{rgb}{0.525967,0.029527,0.066728}%
\pgfsetfillcolor{currentfill}%
\pgfsetlinewidth{0.000000pt}%
\definecolor{currentstroke}{rgb}{0.000000,0.000000,0.000000}%
\pgfsetstrokecolor{currentstroke}%
\pgfsetdash{}{0pt}%
\pgfpathmoveto{\pgfqpoint{9.970139in}{4.059722in}}%
\pgfpathlineto{\pgfqpoint{9.970139in}{4.454861in}}%
\pgfpathlineto{\pgfqpoint{10.553819in}{4.454861in}}%
\pgfpathlineto{\pgfqpoint{10.553819in}{4.059722in}}%
\pgfpathlineto{\pgfqpoint{9.970139in}{4.059722in}}%
\pgfpathlineto{\pgfqpoint{9.970139in}{4.059722in}}%
\pgfpathclose%
\pgfusepath{fill}%
\end{pgfscope}%
\begin{pgfscope}%
\pgfpathrectangle{\pgfqpoint{8.219097in}{3.269444in}}{\pgfqpoint{2.918403in}{1.580556in}}%
\pgfusepath{clip}%
\pgfsetbuttcap%
\pgfsetroundjoin%
\definecolor{currentfill}{rgb}{0.525967,0.029527,0.066728}%
\pgfsetfillcolor{currentfill}%
\pgfsetlinewidth{0.000000pt}%
\definecolor{currentstroke}{rgb}{0.000000,0.000000,0.000000}%
\pgfsetstrokecolor{currentstroke}%
\pgfsetdash{}{0pt}%
\pgfpathmoveto{\pgfqpoint{10.553819in}{4.059722in}}%
\pgfpathlineto{\pgfqpoint{10.553819in}{4.454861in}}%
\pgfpathlineto{\pgfqpoint{11.137500in}{4.454861in}}%
\pgfpathlineto{\pgfqpoint{11.137500in}{4.059722in}}%
\pgfpathlineto{\pgfqpoint{10.553819in}{4.059722in}}%
\pgfpathlineto{\pgfqpoint{10.553819in}{4.059722in}}%
\pgfpathclose%
\pgfusepath{fill}%
\end{pgfscope}%
\begin{pgfscope}%
\pgfpathrectangle{\pgfqpoint{8.219097in}{3.269444in}}{\pgfqpoint{2.918403in}{1.580556in}}%
\pgfusepath{clip}%
\pgfsetbuttcap%
\pgfsetroundjoin%
\definecolor{currentfill}{rgb}{0.974717,0.378101,0.266205}%
\pgfsetfillcolor{currentfill}%
\pgfsetlinewidth{0.000000pt}%
\definecolor{currentstroke}{rgb}{0.000000,0.000000,0.000000}%
\pgfsetstrokecolor{currentstroke}%
\pgfsetdash{}{0pt}%
\pgfpathmoveto{\pgfqpoint{8.219097in}{4.454861in}}%
\pgfpathlineto{\pgfqpoint{8.219097in}{4.850000in}}%
\pgfpathlineto{\pgfqpoint{8.802778in}{4.850000in}}%
\pgfpathlineto{\pgfqpoint{8.802778in}{4.454861in}}%
\pgfpathlineto{\pgfqpoint{8.219097in}{4.454861in}}%
\pgfpathlineto{\pgfqpoint{8.219097in}{4.454861in}}%
\pgfpathclose%
\pgfusepath{fill}%
\end{pgfscope}%
\begin{pgfscope}%
\pgfpathrectangle{\pgfqpoint{8.219097in}{3.269444in}}{\pgfqpoint{2.918403in}{1.580556in}}%
\pgfusepath{clip}%
\pgfsetbuttcap%
\pgfsetroundjoin%
\definecolor{currentfill}{rgb}{0.640384,0.057209,0.081492}%
\pgfsetfillcolor{currentfill}%
\pgfsetlinewidth{0.000000pt}%
\definecolor{currentstroke}{rgb}{0.000000,0.000000,0.000000}%
\pgfsetstrokecolor{currentstroke}%
\pgfsetdash{}{0pt}%
\pgfpathmoveto{\pgfqpoint{8.802778in}{4.454861in}}%
\pgfpathlineto{\pgfqpoint{8.802778in}{4.850000in}}%
\pgfpathlineto{\pgfqpoint{9.386458in}{4.850000in}}%
\pgfpathlineto{\pgfqpoint{9.386458in}{4.454861in}}%
\pgfpathlineto{\pgfqpoint{8.802778in}{4.454861in}}%
\pgfpathlineto{\pgfqpoint{8.802778in}{4.454861in}}%
\pgfpathclose%
\pgfusepath{fill}%
\end{pgfscope}%
\begin{pgfscope}%
\pgfpathrectangle{\pgfqpoint{8.219097in}{3.269444in}}{\pgfqpoint{2.918403in}{1.580556in}}%
\pgfusepath{clip}%
\pgfsetbuttcap%
\pgfsetroundjoin%
\definecolor{currentfill}{rgb}{0.948143,0.274018,0.199769}%
\pgfsetfillcolor{currentfill}%
\pgfsetlinewidth{0.000000pt}%
\definecolor{currentstroke}{rgb}{0.000000,0.000000,0.000000}%
\pgfsetstrokecolor{currentstroke}%
\pgfsetdash{}{0pt}%
\pgfpathmoveto{\pgfqpoint{9.386458in}{4.454861in}}%
\pgfpathlineto{\pgfqpoint{9.386458in}{4.850000in}}%
\pgfpathlineto{\pgfqpoint{9.970139in}{4.850000in}}%
\pgfpathlineto{\pgfqpoint{9.970139in}{4.454861in}}%
\pgfpathlineto{\pgfqpoint{9.386458in}{4.454861in}}%
\pgfpathlineto{\pgfqpoint{9.386458in}{4.454861in}}%
\pgfpathclose%
\pgfusepath{fill}%
\end{pgfscope}%
\begin{pgfscope}%
\pgfpathrectangle{\pgfqpoint{8.219097in}{3.269444in}}{\pgfqpoint{2.918403in}{1.580556in}}%
\pgfusepath{clip}%
\pgfsetbuttcap%
\pgfsetroundjoin%
\definecolor{currentfill}{rgb}{0.666344,0.063391,0.086413}%
\pgfsetfillcolor{currentfill}%
\pgfsetlinewidth{0.000000pt}%
\definecolor{currentstroke}{rgb}{0.000000,0.000000,0.000000}%
\pgfsetstrokecolor{currentstroke}%
\pgfsetdash{}{0pt}%
\pgfpathmoveto{\pgfqpoint{9.970139in}{4.454861in}}%
\pgfpathlineto{\pgfqpoint{9.970139in}{4.850000in}}%
\pgfpathlineto{\pgfqpoint{10.553819in}{4.850000in}}%
\pgfpathlineto{\pgfqpoint{10.553819in}{4.454861in}}%
\pgfpathlineto{\pgfqpoint{9.970139in}{4.454861in}}%
\pgfpathlineto{\pgfqpoint{9.970139in}{4.454861in}}%
\pgfpathclose%
\pgfusepath{fill}%
\end{pgfscope}%
\begin{pgfscope}%
\pgfpathrectangle{\pgfqpoint{8.219097in}{3.269444in}}{\pgfqpoint{2.918403in}{1.580556in}}%
\pgfusepath{clip}%
\pgfsetbuttcap%
\pgfsetroundjoin%
\definecolor{currentfill}{rgb}{0.988235,0.666498,0.554756}%
\pgfsetfillcolor{currentfill}%
\pgfsetlinewidth{0.000000pt}%
\definecolor{currentstroke}{rgb}{0.000000,0.000000,0.000000}%
\pgfsetstrokecolor{currentstroke}%
\pgfsetdash{}{0pt}%
\pgfpathmoveto{\pgfqpoint{10.553819in}{4.454861in}}%
\pgfpathlineto{\pgfqpoint{10.553819in}{4.850000in}}%
\pgfpathlineto{\pgfqpoint{11.137500in}{4.850000in}}%
\pgfpathlineto{\pgfqpoint{11.137500in}{4.454861in}}%
\pgfpathlineto{\pgfqpoint{10.553819in}{4.454861in}}%
\pgfpathlineto{\pgfqpoint{10.553819in}{4.454861in}}%
\pgfpathclose%
\pgfusepath{fill}%
\end{pgfscope}%
\begin{pgfscope}%
\pgfsetbuttcap%
\pgfsetroundjoin%
\definecolor{currentfill}{rgb}{0.000000,0.000000,0.000000}%
\pgfsetfillcolor{currentfill}%
\pgfsetlinewidth{0.803000pt}%
\definecolor{currentstroke}{rgb}{0.000000,0.000000,0.000000}%
\pgfsetstrokecolor{currentstroke}%
\pgfsetdash{}{0pt}%
\pgfsys@defobject{currentmarker}{\pgfqpoint{0.000000in}{-0.048611in}}{\pgfqpoint{0.000000in}{0.000000in}}{%
\pgfpathmoveto{\pgfqpoint{0.000000in}{0.000000in}}%
\pgfpathlineto{\pgfqpoint{0.000000in}{-0.048611in}}%
\pgfusepath{stroke,fill}%
}%
\begin{pgfscope}%
\pgfsys@transformshift{8.219097in}{3.269444in}%
\pgfsys@useobject{currentmarker}{}%
\end{pgfscope}%
\end{pgfscope}%
\begin{pgfscope}%
\definecolor{textcolor}{rgb}{0.000000,0.000000,0.000000}%
\pgfsetstrokecolor{textcolor}%
\pgfsetfillcolor{textcolor}%
\pgftext[x=8.219097in,y=3.172222in,,top]{\color{textcolor}\sffamily\fontsize{10.000000}{12.000000}\selectfont 0}%
\end{pgfscope}%
\begin{pgfscope}%
\pgfsetbuttcap%
\pgfsetroundjoin%
\definecolor{currentfill}{rgb}{0.000000,0.000000,0.000000}%
\pgfsetfillcolor{currentfill}%
\pgfsetlinewidth{0.803000pt}%
\definecolor{currentstroke}{rgb}{0.000000,0.000000,0.000000}%
\pgfsetstrokecolor{currentstroke}%
\pgfsetdash{}{0pt}%
\pgfsys@defobject{currentmarker}{\pgfqpoint{0.000000in}{-0.048611in}}{\pgfqpoint{0.000000in}{0.000000in}}{%
\pgfpathmoveto{\pgfqpoint{0.000000in}{0.000000in}}%
\pgfpathlineto{\pgfqpoint{0.000000in}{-0.048611in}}%
\pgfusepath{stroke,fill}%
}%
\begin{pgfscope}%
\pgfsys@transformshift{8.802778in}{3.269444in}%
\pgfsys@useobject{currentmarker}{}%
\end{pgfscope}%
\end{pgfscope}%
\begin{pgfscope}%
\definecolor{textcolor}{rgb}{0.000000,0.000000,0.000000}%
\pgfsetstrokecolor{textcolor}%
\pgfsetfillcolor{textcolor}%
\pgftext[x=8.802778in,y=3.172222in,,top]{\color{textcolor}\sffamily\fontsize{10.000000}{12.000000}\selectfont 1}%
\end{pgfscope}%
\begin{pgfscope}%
\pgfsetbuttcap%
\pgfsetroundjoin%
\definecolor{currentfill}{rgb}{0.000000,0.000000,0.000000}%
\pgfsetfillcolor{currentfill}%
\pgfsetlinewidth{0.803000pt}%
\definecolor{currentstroke}{rgb}{0.000000,0.000000,0.000000}%
\pgfsetstrokecolor{currentstroke}%
\pgfsetdash{}{0pt}%
\pgfsys@defobject{currentmarker}{\pgfqpoint{0.000000in}{-0.048611in}}{\pgfqpoint{0.000000in}{0.000000in}}{%
\pgfpathmoveto{\pgfqpoint{0.000000in}{0.000000in}}%
\pgfpathlineto{\pgfqpoint{0.000000in}{-0.048611in}}%
\pgfusepath{stroke,fill}%
}%
\begin{pgfscope}%
\pgfsys@transformshift{9.386458in}{3.269444in}%
\pgfsys@useobject{currentmarker}{}%
\end{pgfscope}%
\end{pgfscope}%
\begin{pgfscope}%
\definecolor{textcolor}{rgb}{0.000000,0.000000,0.000000}%
\pgfsetstrokecolor{textcolor}%
\pgfsetfillcolor{textcolor}%
\pgftext[x=9.386458in,y=3.172222in,,top]{\color{textcolor}\sffamily\fontsize{10.000000}{12.000000}\selectfont 2}%
\end{pgfscope}%
\begin{pgfscope}%
\pgfsetbuttcap%
\pgfsetroundjoin%
\definecolor{currentfill}{rgb}{0.000000,0.000000,0.000000}%
\pgfsetfillcolor{currentfill}%
\pgfsetlinewidth{0.803000pt}%
\definecolor{currentstroke}{rgb}{0.000000,0.000000,0.000000}%
\pgfsetstrokecolor{currentstroke}%
\pgfsetdash{}{0pt}%
\pgfsys@defobject{currentmarker}{\pgfqpoint{0.000000in}{-0.048611in}}{\pgfqpoint{0.000000in}{0.000000in}}{%
\pgfpathmoveto{\pgfqpoint{0.000000in}{0.000000in}}%
\pgfpathlineto{\pgfqpoint{0.000000in}{-0.048611in}}%
\pgfusepath{stroke,fill}%
}%
\begin{pgfscope}%
\pgfsys@transformshift{9.970139in}{3.269444in}%
\pgfsys@useobject{currentmarker}{}%
\end{pgfscope}%
\end{pgfscope}%
\begin{pgfscope}%
\definecolor{textcolor}{rgb}{0.000000,0.000000,0.000000}%
\pgfsetstrokecolor{textcolor}%
\pgfsetfillcolor{textcolor}%
\pgftext[x=9.970139in,y=3.172222in,,top]{\color{textcolor}\sffamily\fontsize{10.000000}{12.000000}\selectfont 3}%
\end{pgfscope}%
\begin{pgfscope}%
\pgfsetbuttcap%
\pgfsetroundjoin%
\definecolor{currentfill}{rgb}{0.000000,0.000000,0.000000}%
\pgfsetfillcolor{currentfill}%
\pgfsetlinewidth{0.803000pt}%
\definecolor{currentstroke}{rgb}{0.000000,0.000000,0.000000}%
\pgfsetstrokecolor{currentstroke}%
\pgfsetdash{}{0pt}%
\pgfsys@defobject{currentmarker}{\pgfqpoint{0.000000in}{-0.048611in}}{\pgfqpoint{0.000000in}{0.000000in}}{%
\pgfpathmoveto{\pgfqpoint{0.000000in}{0.000000in}}%
\pgfpathlineto{\pgfqpoint{0.000000in}{-0.048611in}}%
\pgfusepath{stroke,fill}%
}%
\begin{pgfscope}%
\pgfsys@transformshift{10.553819in}{3.269444in}%
\pgfsys@useobject{currentmarker}{}%
\end{pgfscope}%
\end{pgfscope}%
\begin{pgfscope}%
\definecolor{textcolor}{rgb}{0.000000,0.000000,0.000000}%
\pgfsetstrokecolor{textcolor}%
\pgfsetfillcolor{textcolor}%
\pgftext[x=10.553819in,y=3.172222in,,top]{\color{textcolor}\sffamily\fontsize{10.000000}{12.000000}\selectfont 4}%
\end{pgfscope}%
\begin{pgfscope}%
\definecolor{textcolor}{rgb}{0.000000,0.000000,0.000000}%
\pgfsetstrokecolor{textcolor}%
\pgfsetfillcolor{textcolor}%
\pgftext[x=9.678299in,y=2.982254in,,top]{\color{textcolor}\sffamily\fontsize{30.000000}{36.000000}\selectfont x axis}%
\end{pgfscope}%
\begin{pgfscope}%
\pgfsetbuttcap%
\pgfsetroundjoin%
\definecolor{currentfill}{rgb}{0.000000,0.000000,0.000000}%
\pgfsetfillcolor{currentfill}%
\pgfsetlinewidth{0.803000pt}%
\definecolor{currentstroke}{rgb}{0.000000,0.000000,0.000000}%
\pgfsetstrokecolor{currentstroke}%
\pgfsetdash{}{0pt}%
\pgfsys@defobject{currentmarker}{\pgfqpoint{-0.048611in}{0.000000in}}{\pgfqpoint{-0.000000in}{0.000000in}}{%
\pgfpathmoveto{\pgfqpoint{-0.000000in}{0.000000in}}%
\pgfpathlineto{\pgfqpoint{-0.048611in}{0.000000in}}%
\pgfusepath{stroke,fill}%
}%
\begin{pgfscope}%
\pgfsys@transformshift{8.219097in}{3.269444in}%
\pgfsys@useobject{currentmarker}{}%
\end{pgfscope}%
\end{pgfscope}%
\begin{pgfscope}%
\definecolor{textcolor}{rgb}{0.000000,0.000000,0.000000}%
\pgfsetstrokecolor{textcolor}%
\pgfsetfillcolor{textcolor}%
\pgftext[x=8.033510in, y=3.216683in, left, base]{\color{textcolor}\sffamily\fontsize{10.000000}{12.000000}\selectfont 0}%
\end{pgfscope}%
\begin{pgfscope}%
\pgfsetbuttcap%
\pgfsetroundjoin%
\definecolor{currentfill}{rgb}{0.000000,0.000000,0.000000}%
\pgfsetfillcolor{currentfill}%
\pgfsetlinewidth{0.803000pt}%
\definecolor{currentstroke}{rgb}{0.000000,0.000000,0.000000}%
\pgfsetstrokecolor{currentstroke}%
\pgfsetdash{}{0pt}%
\pgfsys@defobject{currentmarker}{\pgfqpoint{-0.048611in}{0.000000in}}{\pgfqpoint{-0.000000in}{0.000000in}}{%
\pgfpathmoveto{\pgfqpoint{-0.000000in}{0.000000in}}%
\pgfpathlineto{\pgfqpoint{-0.048611in}{0.000000in}}%
\pgfusepath{stroke,fill}%
}%
\begin{pgfscope}%
\pgfsys@transformshift{8.219097in}{3.664583in}%
\pgfsys@useobject{currentmarker}{}%
\end{pgfscope}%
\end{pgfscope}%
\begin{pgfscope}%
\definecolor{textcolor}{rgb}{0.000000,0.000000,0.000000}%
\pgfsetstrokecolor{textcolor}%
\pgfsetfillcolor{textcolor}%
\pgftext[x=8.033510in, y=3.611822in, left, base]{\color{textcolor}\sffamily\fontsize{10.000000}{12.000000}\selectfont 1}%
\end{pgfscope}%
\begin{pgfscope}%
\pgfsetbuttcap%
\pgfsetroundjoin%
\definecolor{currentfill}{rgb}{0.000000,0.000000,0.000000}%
\pgfsetfillcolor{currentfill}%
\pgfsetlinewidth{0.803000pt}%
\definecolor{currentstroke}{rgb}{0.000000,0.000000,0.000000}%
\pgfsetstrokecolor{currentstroke}%
\pgfsetdash{}{0pt}%
\pgfsys@defobject{currentmarker}{\pgfqpoint{-0.048611in}{0.000000in}}{\pgfqpoint{-0.000000in}{0.000000in}}{%
\pgfpathmoveto{\pgfqpoint{-0.000000in}{0.000000in}}%
\pgfpathlineto{\pgfqpoint{-0.048611in}{0.000000in}}%
\pgfusepath{stroke,fill}%
}%
\begin{pgfscope}%
\pgfsys@transformshift{8.219097in}{4.059722in}%
\pgfsys@useobject{currentmarker}{}%
\end{pgfscope}%
\end{pgfscope}%
\begin{pgfscope}%
\definecolor{textcolor}{rgb}{0.000000,0.000000,0.000000}%
\pgfsetstrokecolor{textcolor}%
\pgfsetfillcolor{textcolor}%
\pgftext[x=8.033510in, y=4.006961in, left, base]{\color{textcolor}\sffamily\fontsize{10.000000}{12.000000}\selectfont 2}%
\end{pgfscope}%
\begin{pgfscope}%
\pgfsetbuttcap%
\pgfsetroundjoin%
\definecolor{currentfill}{rgb}{0.000000,0.000000,0.000000}%
\pgfsetfillcolor{currentfill}%
\pgfsetlinewidth{0.803000pt}%
\definecolor{currentstroke}{rgb}{0.000000,0.000000,0.000000}%
\pgfsetstrokecolor{currentstroke}%
\pgfsetdash{}{0pt}%
\pgfsys@defobject{currentmarker}{\pgfqpoint{-0.048611in}{0.000000in}}{\pgfqpoint{-0.000000in}{0.000000in}}{%
\pgfpathmoveto{\pgfqpoint{-0.000000in}{0.000000in}}%
\pgfpathlineto{\pgfqpoint{-0.048611in}{0.000000in}}%
\pgfusepath{stroke,fill}%
}%
\begin{pgfscope}%
\pgfsys@transformshift{8.219097in}{4.454861in}%
\pgfsys@useobject{currentmarker}{}%
\end{pgfscope}%
\end{pgfscope}%
\begin{pgfscope}%
\definecolor{textcolor}{rgb}{0.000000,0.000000,0.000000}%
\pgfsetstrokecolor{textcolor}%
\pgfsetfillcolor{textcolor}%
\pgftext[x=8.033510in, y=4.402100in, left, base]{\color{textcolor}\sffamily\fontsize{10.000000}{12.000000}\selectfont 3}%
\end{pgfscope}%
\begin{pgfscope}%
\definecolor{textcolor}{rgb}{0.000000,0.000000,0.000000}%
\pgfsetstrokecolor{textcolor}%
\pgfsetfillcolor{textcolor}%
\pgftext[x=7.977954in,y=4.059722in,,bottom,rotate=90.000000]{\color{textcolor}\sffamily\fontsize{30.000000}{36.000000}\selectfont y axis}%
\end{pgfscope}%
\begin{pgfscope}%
\pgfsetrectcap%
\pgfsetmiterjoin%
\pgfsetlinewidth{0.803000pt}%
\definecolor{currentstroke}{rgb}{0.000000,0.000000,0.000000}%
\pgfsetstrokecolor{currentstroke}%
\pgfsetdash{}{0pt}%
\pgfpathmoveto{\pgfqpoint{8.219097in}{3.269444in}}%
\pgfpathlineto{\pgfqpoint{8.219097in}{4.850000in}}%
\pgfusepath{stroke}%
\end{pgfscope}%
\begin{pgfscope}%
\pgfsetrectcap%
\pgfsetmiterjoin%
\pgfsetlinewidth{0.803000pt}%
\definecolor{currentstroke}{rgb}{0.000000,0.000000,0.000000}%
\pgfsetstrokecolor{currentstroke}%
\pgfsetdash{}{0pt}%
\pgfpathmoveto{\pgfqpoint{11.137500in}{3.269444in}}%
\pgfpathlineto{\pgfqpoint{11.137500in}{4.850000in}}%
\pgfusepath{stroke}%
\end{pgfscope}%
\begin{pgfscope}%
\pgfsetrectcap%
\pgfsetmiterjoin%
\pgfsetlinewidth{0.803000pt}%
\definecolor{currentstroke}{rgb}{0.000000,0.000000,0.000000}%
\pgfsetstrokecolor{currentstroke}%
\pgfsetdash{}{0pt}%
\pgfpathmoveto{\pgfqpoint{8.219097in}{3.269444in}}%
\pgfpathlineto{\pgfqpoint{11.137500in}{3.269444in}}%
\pgfusepath{stroke}%
\end{pgfscope}%
\begin{pgfscope}%
\pgfsetrectcap%
\pgfsetmiterjoin%
\pgfsetlinewidth{0.803000pt}%
\definecolor{currentstroke}{rgb}{0.000000,0.000000,0.000000}%
\pgfsetstrokecolor{currentstroke}%
\pgfsetdash{}{0pt}%
\pgfpathmoveto{\pgfqpoint{8.219097in}{4.850000in}}%
\pgfpathlineto{\pgfqpoint{11.137500in}{4.850000in}}%
\pgfusepath{stroke}%
\end{pgfscope}%
\begin{pgfscope}%
\pgfsetbuttcap%
\pgfsetmiterjoin%
\definecolor{currentfill}{rgb}{1.000000,1.000000,1.000000}%
\pgfsetfillcolor{currentfill}%
\pgfsetlinewidth{0.000000pt}%
\definecolor{currentstroke}{rgb}{0.000000,0.000000,0.000000}%
\pgfsetstrokecolor{currentstroke}%
\pgfsetstrokeopacity{0.000000}%
\pgfsetdash{}{0pt}%
\pgfpathmoveto{\pgfqpoint{11.931597in}{3.269444in}}%
\pgfpathlineto{\pgfqpoint{14.850000in}{3.269444in}}%
\pgfpathlineto{\pgfqpoint{14.850000in}{4.850000in}}%
\pgfpathlineto{\pgfqpoint{11.931597in}{4.850000in}}%
\pgfpathlineto{\pgfqpoint{11.931597in}{3.269444in}}%
\pgfpathclose%
\pgfusepath{fill}%
\end{pgfscope}%
\begin{pgfscope}%
\pgfpathrectangle{\pgfqpoint{11.931597in}{3.269444in}}{\pgfqpoint{2.918403in}{1.580556in}}%
\pgfusepath{clip}%
\pgfsetbuttcap%
\pgfsetroundjoin%
\definecolor{currentfill}{rgb}{0.974717,0.378101,0.266205}%
\pgfsetfillcolor{currentfill}%
\pgfsetlinewidth{0.000000pt}%
\definecolor{currentstroke}{rgb}{0.000000,0.000000,0.000000}%
\pgfsetstrokecolor{currentstroke}%
\pgfsetdash{}{0pt}%
\pgfpathmoveto{\pgfqpoint{11.931597in}{3.269444in}}%
\pgfpathlineto{\pgfqpoint{11.931597in}{3.664583in}}%
\pgfpathlineto{\pgfqpoint{12.515278in}{3.664583in}}%
\pgfpathlineto{\pgfqpoint{12.515278in}{3.269444in}}%
\pgfpathlineto{\pgfqpoint{11.931597in}{3.269444in}}%
\pgfpathlineto{\pgfqpoint{11.931597in}{3.269444in}}%
\pgfpathclose%
\pgfusepath{fill}%
\end{pgfscope}%
\begin{pgfscope}%
\pgfpathrectangle{\pgfqpoint{11.931597in}{3.269444in}}{\pgfqpoint{2.918403in}{1.580556in}}%
\pgfusepath{clip}%
\pgfsetbuttcap%
\pgfsetroundjoin%
\definecolor{currentfill}{rgb}{0.640384,0.057209,0.081492}%
\pgfsetfillcolor{currentfill}%
\pgfsetlinewidth{0.000000pt}%
\definecolor{currentstroke}{rgb}{0.000000,0.000000,0.000000}%
\pgfsetstrokecolor{currentstroke}%
\pgfsetdash{}{0pt}%
\pgfpathmoveto{\pgfqpoint{12.515278in}{3.269444in}}%
\pgfpathlineto{\pgfqpoint{12.515278in}{3.664583in}}%
\pgfpathlineto{\pgfqpoint{13.098958in}{3.664583in}}%
\pgfpathlineto{\pgfqpoint{13.098958in}{3.269444in}}%
\pgfpathlineto{\pgfqpoint{12.515278in}{3.269444in}}%
\pgfpathlineto{\pgfqpoint{12.515278in}{3.269444in}}%
\pgfpathclose%
\pgfusepath{fill}%
\end{pgfscope}%
\begin{pgfscope}%
\pgfpathrectangle{\pgfqpoint{11.931597in}{3.269444in}}{\pgfqpoint{2.918403in}{1.580556in}}%
\pgfusepath{clip}%
\pgfsetbuttcap%
\pgfsetroundjoin%
\definecolor{currentfill}{rgb}{0.948143,0.274018,0.199769}%
\pgfsetfillcolor{currentfill}%
\pgfsetlinewidth{0.000000pt}%
\definecolor{currentstroke}{rgb}{0.000000,0.000000,0.000000}%
\pgfsetstrokecolor{currentstroke}%
\pgfsetdash{}{0pt}%
\pgfpathmoveto{\pgfqpoint{13.098958in}{3.269444in}}%
\pgfpathlineto{\pgfqpoint{13.098958in}{3.664583in}}%
\pgfpathlineto{\pgfqpoint{13.682639in}{3.664583in}}%
\pgfpathlineto{\pgfqpoint{13.682639in}{3.269444in}}%
\pgfpathlineto{\pgfqpoint{13.098958in}{3.269444in}}%
\pgfpathlineto{\pgfqpoint{13.098958in}{3.269444in}}%
\pgfpathclose%
\pgfusepath{fill}%
\end{pgfscope}%
\begin{pgfscope}%
\pgfpathrectangle{\pgfqpoint{11.931597in}{3.269444in}}{\pgfqpoint{2.918403in}{1.580556in}}%
\pgfusepath{clip}%
\pgfsetbuttcap%
\pgfsetroundjoin%
\definecolor{currentfill}{rgb}{0.666344,0.063391,0.086413}%
\pgfsetfillcolor{currentfill}%
\pgfsetlinewidth{0.000000pt}%
\definecolor{currentstroke}{rgb}{0.000000,0.000000,0.000000}%
\pgfsetstrokecolor{currentstroke}%
\pgfsetdash{}{0pt}%
\pgfpathmoveto{\pgfqpoint{13.682639in}{3.269444in}}%
\pgfpathlineto{\pgfqpoint{13.682639in}{3.664583in}}%
\pgfpathlineto{\pgfqpoint{14.266319in}{3.664583in}}%
\pgfpathlineto{\pgfqpoint{14.266319in}{3.269444in}}%
\pgfpathlineto{\pgfqpoint{13.682639in}{3.269444in}}%
\pgfpathlineto{\pgfqpoint{13.682639in}{3.269444in}}%
\pgfpathclose%
\pgfusepath{fill}%
\end{pgfscope}%
\begin{pgfscope}%
\pgfpathrectangle{\pgfqpoint{11.931597in}{3.269444in}}{\pgfqpoint{2.918403in}{1.580556in}}%
\pgfusepath{clip}%
\pgfsetbuttcap%
\pgfsetroundjoin%
\definecolor{currentfill}{rgb}{0.988235,0.666498,0.554756}%
\pgfsetfillcolor{currentfill}%
\pgfsetlinewidth{0.000000pt}%
\definecolor{currentstroke}{rgb}{0.000000,0.000000,0.000000}%
\pgfsetstrokecolor{currentstroke}%
\pgfsetdash{}{0pt}%
\pgfpathmoveto{\pgfqpoint{14.266319in}{3.269444in}}%
\pgfpathlineto{\pgfqpoint{14.266319in}{3.664583in}}%
\pgfpathlineto{\pgfqpoint{14.850000in}{3.664583in}}%
\pgfpathlineto{\pgfqpoint{14.850000in}{3.269444in}}%
\pgfpathlineto{\pgfqpoint{14.266319in}{3.269444in}}%
\pgfpathlineto{\pgfqpoint{14.266319in}{3.269444in}}%
\pgfpathclose%
\pgfusepath{fill}%
\end{pgfscope}%
\begin{pgfscope}%
\pgfpathrectangle{\pgfqpoint{11.931597in}{3.269444in}}{\pgfqpoint{2.918403in}{1.580556in}}%
\pgfusepath{clip}%
\pgfsetbuttcap%
\pgfsetroundjoin%
\definecolor{currentfill}{rgb}{0.954048,0.297147,0.214533}%
\pgfsetfillcolor{currentfill}%
\pgfsetlinewidth{0.000000pt}%
\definecolor{currentstroke}{rgb}{0.000000,0.000000,0.000000}%
\pgfsetstrokecolor{currentstroke}%
\pgfsetdash{}{0pt}%
\pgfpathmoveto{\pgfqpoint{11.931597in}{3.664583in}}%
\pgfpathlineto{\pgfqpoint{11.931597in}{4.059722in}}%
\pgfpathlineto{\pgfqpoint{12.515278in}{4.059722in}}%
\pgfpathlineto{\pgfqpoint{12.515278in}{3.664583in}}%
\pgfpathlineto{\pgfqpoint{11.931597in}{3.664583in}}%
\pgfpathlineto{\pgfqpoint{11.931597in}{3.664583in}}%
\pgfpathclose%
\pgfusepath{fill}%
\end{pgfscope}%
\begin{pgfscope}%
\pgfpathrectangle{\pgfqpoint{11.931597in}{3.269444in}}{\pgfqpoint{2.918403in}{1.580556in}}%
\pgfusepath{clip}%
\pgfsetbuttcap%
\pgfsetroundjoin%
\definecolor{currentfill}{rgb}{0.988189,0.570704,0.445213}%
\pgfsetfillcolor{currentfill}%
\pgfsetlinewidth{0.000000pt}%
\definecolor{currentstroke}{rgb}{0.000000,0.000000,0.000000}%
\pgfsetstrokecolor{currentstroke}%
\pgfsetdash{}{0pt}%
\pgfpathmoveto{\pgfqpoint{12.515278in}{3.664583in}}%
\pgfpathlineto{\pgfqpoint{12.515278in}{4.059722in}}%
\pgfpathlineto{\pgfqpoint{13.098958in}{4.059722in}}%
\pgfpathlineto{\pgfqpoint{13.098958in}{3.664583in}}%
\pgfpathlineto{\pgfqpoint{12.515278in}{3.664583in}}%
\pgfpathlineto{\pgfqpoint{12.515278in}{3.664583in}}%
\pgfpathclose%
\pgfusepath{fill}%
\end{pgfscope}%
\begin{pgfscope}%
\pgfpathrectangle{\pgfqpoint{11.931597in}{3.269444in}}{\pgfqpoint{2.918403in}{1.580556in}}%
\pgfusepath{clip}%
\pgfsetbuttcap%
\pgfsetroundjoin%
\definecolor{currentfill}{rgb}{0.961430,0.326059,0.232987}%
\pgfsetfillcolor{currentfill}%
\pgfsetlinewidth{0.000000pt}%
\definecolor{currentstroke}{rgb}{0.000000,0.000000,0.000000}%
\pgfsetstrokecolor{currentstroke}%
\pgfsetdash{}{0pt}%
\pgfpathmoveto{\pgfqpoint{13.098958in}{3.664583in}}%
\pgfpathlineto{\pgfqpoint{13.098958in}{4.059722in}}%
\pgfpathlineto{\pgfqpoint{13.682639in}{4.059722in}}%
\pgfpathlineto{\pgfqpoint{13.682639in}{3.664583in}}%
\pgfpathlineto{\pgfqpoint{13.098958in}{3.664583in}}%
\pgfpathlineto{\pgfqpoint{13.098958in}{3.664583in}}%
\pgfpathclose%
\pgfusepath{fill}%
\end{pgfscope}%
\begin{pgfscope}%
\pgfpathrectangle{\pgfqpoint{11.931597in}{3.269444in}}{\pgfqpoint{2.918403in}{1.580556in}}%
\pgfusepath{clip}%
\pgfsetbuttcap%
\pgfsetroundjoin%
\definecolor{currentfill}{rgb}{0.990388,0.773164,0.684121}%
\pgfsetfillcolor{currentfill}%
\pgfsetlinewidth{0.000000pt}%
\definecolor{currentstroke}{rgb}{0.000000,0.000000,0.000000}%
\pgfsetstrokecolor{currentstroke}%
\pgfsetdash{}{0pt}%
\pgfpathmoveto{\pgfqpoint{13.682639in}{3.664583in}}%
\pgfpathlineto{\pgfqpoint{13.682639in}{4.059722in}}%
\pgfpathlineto{\pgfqpoint{14.266319in}{4.059722in}}%
\pgfpathlineto{\pgfqpoint{14.266319in}{3.664583in}}%
\pgfpathlineto{\pgfqpoint{13.682639in}{3.664583in}}%
\pgfpathlineto{\pgfqpoint{13.682639in}{3.664583in}}%
\pgfpathclose%
\pgfusepath{fill}%
\end{pgfscope}%
\begin{pgfscope}%
\pgfpathrectangle{\pgfqpoint{11.931597in}{3.269444in}}{\pgfqpoint{2.918403in}{1.580556in}}%
\pgfusepath{clip}%
\pgfsetbuttcap%
\pgfsetroundjoin%
\definecolor{currentfill}{rgb}{0.988235,0.575702,0.450673}%
\pgfsetfillcolor{currentfill}%
\pgfsetlinewidth{0.000000pt}%
\definecolor{currentstroke}{rgb}{0.000000,0.000000,0.000000}%
\pgfsetstrokecolor{currentstroke}%
\pgfsetdash{}{0pt}%
\pgfpathmoveto{\pgfqpoint{14.266319in}{3.664583in}}%
\pgfpathlineto{\pgfqpoint{14.266319in}{4.059722in}}%
\pgfpathlineto{\pgfqpoint{14.850000in}{4.059722in}}%
\pgfpathlineto{\pgfqpoint{14.850000in}{3.664583in}}%
\pgfpathlineto{\pgfqpoint{14.266319in}{3.664583in}}%
\pgfpathlineto{\pgfqpoint{14.266319in}{3.664583in}}%
\pgfpathclose%
\pgfusepath{fill}%
\end{pgfscope}%
\begin{pgfscope}%
\pgfpathrectangle{\pgfqpoint{11.931597in}{3.269444in}}{\pgfqpoint{2.918403in}{1.580556in}}%
\pgfusepath{clip}%
\pgfsetbuttcap%
\pgfsetroundjoin%
\definecolor{currentfill}{rgb}{0.991373,0.791373,0.708235}%
\pgfsetfillcolor{currentfill}%
\pgfsetlinewidth{0.000000pt}%
\definecolor{currentstroke}{rgb}{0.000000,0.000000,0.000000}%
\pgfsetstrokecolor{currentstroke}%
\pgfsetdash{}{0pt}%
\pgfpathmoveto{\pgfqpoint{11.931597in}{4.059722in}}%
\pgfpathlineto{\pgfqpoint{11.931597in}{4.454861in}}%
\pgfpathlineto{\pgfqpoint{12.515278in}{4.454861in}}%
\pgfpathlineto{\pgfqpoint{12.515278in}{4.059722in}}%
\pgfpathlineto{\pgfqpoint{11.931597in}{4.059722in}}%
\pgfpathlineto{\pgfqpoint{11.931597in}{4.059722in}}%
\pgfpathclose%
\pgfusepath{fill}%
\end{pgfscope}%
\begin{pgfscope}%
\pgfpathrectangle{\pgfqpoint{11.931597in}{3.269444in}}{\pgfqpoint{2.918403in}{1.580556in}}%
\pgfusepath{clip}%
\pgfsetbuttcap%
\pgfsetroundjoin%
\definecolor{currentfill}{rgb}{0.985729,0.472280,0.346790}%
\pgfsetfillcolor{currentfill}%
\pgfsetlinewidth{0.000000pt}%
\definecolor{currentstroke}{rgb}{0.000000,0.000000,0.000000}%
\pgfsetstrokecolor{currentstroke}%
\pgfsetdash{}{0pt}%
\pgfpathmoveto{\pgfqpoint{12.515278in}{4.059722in}}%
\pgfpathlineto{\pgfqpoint{12.515278in}{4.454861in}}%
\pgfpathlineto{\pgfqpoint{13.098958in}{4.454861in}}%
\pgfpathlineto{\pgfqpoint{13.098958in}{4.059722in}}%
\pgfpathlineto{\pgfqpoint{12.515278in}{4.059722in}}%
\pgfpathlineto{\pgfqpoint{12.515278in}{4.059722in}}%
\pgfpathclose%
\pgfusepath{fill}%
\end{pgfscope}%
\begin{pgfscope}%
\pgfpathrectangle{\pgfqpoint{11.931597in}{3.269444in}}{\pgfqpoint{2.918403in}{1.580556in}}%
\pgfusepath{clip}%
\pgfsetbuttcap%
\pgfsetroundjoin%
\definecolor{currentfill}{rgb}{0.868051,0.164091,0.143714}%
\pgfsetfillcolor{currentfill}%
\pgfsetlinewidth{0.000000pt}%
\definecolor{currentstroke}{rgb}{0.000000,0.000000,0.000000}%
\pgfsetstrokecolor{currentstroke}%
\pgfsetdash{}{0pt}%
\pgfpathmoveto{\pgfqpoint{13.098958in}{4.059722in}}%
\pgfpathlineto{\pgfqpoint{13.098958in}{4.454861in}}%
\pgfpathlineto{\pgfqpoint{13.682639in}{4.454861in}}%
\pgfpathlineto{\pgfqpoint{13.682639in}{4.059722in}}%
\pgfpathlineto{\pgfqpoint{13.098958in}{4.059722in}}%
\pgfpathlineto{\pgfqpoint{13.098958in}{4.059722in}}%
\pgfpathclose%
\pgfusepath{fill}%
\end{pgfscope}%
\begin{pgfscope}%
\pgfpathrectangle{\pgfqpoint{11.931597in}{3.269444in}}{\pgfqpoint{2.918403in}{1.580556in}}%
\pgfusepath{clip}%
\pgfsetbuttcap%
\pgfsetroundjoin%
\definecolor{currentfill}{rgb}{1.000000,0.960784,0.941176}%
\pgfsetfillcolor{currentfill}%
\pgfsetlinewidth{0.000000pt}%
\definecolor{currentstroke}{rgb}{0.000000,0.000000,0.000000}%
\pgfsetstrokecolor{currentstroke}%
\pgfsetdash{}{0pt}%
\pgfpathmoveto{\pgfqpoint{13.682639in}{4.059722in}}%
\pgfpathlineto{\pgfqpoint{13.682639in}{4.454861in}}%
\pgfpathlineto{\pgfqpoint{14.266319in}{4.454861in}}%
\pgfpathlineto{\pgfqpoint{14.266319in}{4.059722in}}%
\pgfpathlineto{\pgfqpoint{13.682639in}{4.059722in}}%
\pgfpathlineto{\pgfqpoint{13.682639in}{4.059722in}}%
\pgfpathclose%
\pgfusepath{fill}%
\end{pgfscope}%
\begin{pgfscope}%
\pgfpathrectangle{\pgfqpoint{11.931597in}{3.269444in}}{\pgfqpoint{2.918403in}{1.580556in}}%
\pgfusepath{clip}%
\pgfsetbuttcap%
\pgfsetroundjoin%
\definecolor{currentfill}{rgb}{0.403922,0.000000,0.050980}%
\pgfsetfillcolor{currentfill}%
\pgfsetlinewidth{0.000000pt}%
\definecolor{currentstroke}{rgb}{0.000000,0.000000,0.000000}%
\pgfsetstrokecolor{currentstroke}%
\pgfsetdash{}{0pt}%
\pgfpathmoveto{\pgfqpoint{14.266319in}{4.059722in}}%
\pgfpathlineto{\pgfqpoint{14.266319in}{4.454861in}}%
\pgfpathlineto{\pgfqpoint{14.850000in}{4.454861in}}%
\pgfpathlineto{\pgfqpoint{14.850000in}{4.059722in}}%
\pgfpathlineto{\pgfqpoint{14.266319in}{4.059722in}}%
\pgfpathlineto{\pgfqpoint{14.266319in}{4.059722in}}%
\pgfpathclose%
\pgfusepath{fill}%
\end{pgfscope}%
\begin{pgfscope}%
\pgfpathrectangle{\pgfqpoint{11.931597in}{3.269444in}}{\pgfqpoint{2.918403in}{1.580556in}}%
\pgfusepath{clip}%
\pgfsetbuttcap%
\pgfsetroundjoin%
\definecolor{currentfill}{rgb}{0.984744,0.432910,0.307420}%
\pgfsetfillcolor{currentfill}%
\pgfsetlinewidth{0.000000pt}%
\definecolor{currentstroke}{rgb}{0.000000,0.000000,0.000000}%
\pgfsetstrokecolor{currentstroke}%
\pgfsetdash{}{0pt}%
\pgfpathmoveto{\pgfqpoint{11.931597in}{4.454861in}}%
\pgfpathlineto{\pgfqpoint{11.931597in}{4.850000in}}%
\pgfpathlineto{\pgfqpoint{12.515278in}{4.850000in}}%
\pgfpathlineto{\pgfqpoint{12.515278in}{4.454861in}}%
\pgfpathlineto{\pgfqpoint{11.931597in}{4.454861in}}%
\pgfpathlineto{\pgfqpoint{11.931597in}{4.454861in}}%
\pgfpathclose%
\pgfusepath{fill}%
\end{pgfscope}%
\begin{pgfscope}%
\pgfpathrectangle{\pgfqpoint{11.931597in}{3.269444in}}{\pgfqpoint{2.918403in}{1.580556in}}%
\pgfusepath{clip}%
\pgfsetbuttcap%
\pgfsetroundjoin%
\definecolor{currentfill}{rgb}{0.974717,0.378101,0.266205}%
\pgfsetfillcolor{currentfill}%
\pgfsetlinewidth{0.000000pt}%
\definecolor{currentstroke}{rgb}{0.000000,0.000000,0.000000}%
\pgfsetstrokecolor{currentstroke}%
\pgfsetdash{}{0pt}%
\pgfpathmoveto{\pgfqpoint{12.515278in}{4.454861in}}%
\pgfpathlineto{\pgfqpoint{12.515278in}{4.850000in}}%
\pgfpathlineto{\pgfqpoint{13.098958in}{4.850000in}}%
\pgfpathlineto{\pgfqpoint{13.098958in}{4.454861in}}%
\pgfpathlineto{\pgfqpoint{12.515278in}{4.454861in}}%
\pgfpathlineto{\pgfqpoint{12.515278in}{4.454861in}}%
\pgfpathclose%
\pgfusepath{fill}%
\end{pgfscope}%
\begin{pgfscope}%
\pgfpathrectangle{\pgfqpoint{11.931597in}{3.269444in}}{\pgfqpoint{2.918403in}{1.580556in}}%
\pgfusepath{clip}%
\pgfsetbuttcap%
\pgfsetroundjoin%
\definecolor{currentfill}{rgb}{0.988235,0.681630,0.572103}%
\pgfsetfillcolor{currentfill}%
\pgfsetlinewidth{0.000000pt}%
\definecolor{currentstroke}{rgb}{0.000000,0.000000,0.000000}%
\pgfsetstrokecolor{currentstroke}%
\pgfsetdash{}{0pt}%
\pgfpathmoveto{\pgfqpoint{13.098958in}{4.454861in}}%
\pgfpathlineto{\pgfqpoint{13.098958in}{4.850000in}}%
\pgfpathlineto{\pgfqpoint{13.682639in}{4.850000in}}%
\pgfpathlineto{\pgfqpoint{13.682639in}{4.454861in}}%
\pgfpathlineto{\pgfqpoint{13.098958in}{4.454861in}}%
\pgfpathlineto{\pgfqpoint{13.098958in}{4.454861in}}%
\pgfpathclose%
\pgfusepath{fill}%
\end{pgfscope}%
\begin{pgfscope}%
\pgfpathrectangle{\pgfqpoint{11.931597in}{3.269444in}}{\pgfqpoint{2.918403in}{1.580556in}}%
\pgfusepath{clip}%
\pgfsetbuttcap%
\pgfsetroundjoin%
\definecolor{currentfill}{rgb}{0.525967,0.029527,0.066728}%
\pgfsetfillcolor{currentfill}%
\pgfsetlinewidth{0.000000pt}%
\definecolor{currentstroke}{rgb}{0.000000,0.000000,0.000000}%
\pgfsetstrokecolor{currentstroke}%
\pgfsetdash{}{0pt}%
\pgfpathmoveto{\pgfqpoint{13.682639in}{4.454861in}}%
\pgfpathlineto{\pgfqpoint{13.682639in}{4.850000in}}%
\pgfpathlineto{\pgfqpoint{14.266319in}{4.850000in}}%
\pgfpathlineto{\pgfqpoint{14.266319in}{4.454861in}}%
\pgfpathlineto{\pgfqpoint{13.682639in}{4.454861in}}%
\pgfpathlineto{\pgfqpoint{13.682639in}{4.454861in}}%
\pgfpathclose%
\pgfusepath{fill}%
\end{pgfscope}%
\begin{pgfscope}%
\pgfpathrectangle{\pgfqpoint{11.931597in}{3.269444in}}{\pgfqpoint{2.918403in}{1.580556in}}%
\pgfusepath{clip}%
\pgfsetbuttcap%
\pgfsetroundjoin%
\definecolor{currentfill}{rgb}{0.525967,0.029527,0.066728}%
\pgfsetfillcolor{currentfill}%
\pgfsetlinewidth{0.000000pt}%
\definecolor{currentstroke}{rgb}{0.000000,0.000000,0.000000}%
\pgfsetstrokecolor{currentstroke}%
\pgfsetdash{}{0pt}%
\pgfpathmoveto{\pgfqpoint{14.266319in}{4.454861in}}%
\pgfpathlineto{\pgfqpoint{14.266319in}{4.850000in}}%
\pgfpathlineto{\pgfqpoint{14.850000in}{4.850000in}}%
\pgfpathlineto{\pgfqpoint{14.850000in}{4.454861in}}%
\pgfpathlineto{\pgfqpoint{14.266319in}{4.454861in}}%
\pgfpathlineto{\pgfqpoint{14.266319in}{4.454861in}}%
\pgfpathclose%
\pgfusepath{fill}%
\end{pgfscope}%
\begin{pgfscope}%
\pgfsetbuttcap%
\pgfsetroundjoin%
\definecolor{currentfill}{rgb}{0.000000,0.000000,0.000000}%
\pgfsetfillcolor{currentfill}%
\pgfsetlinewidth{0.803000pt}%
\definecolor{currentstroke}{rgb}{0.000000,0.000000,0.000000}%
\pgfsetstrokecolor{currentstroke}%
\pgfsetdash{}{0pt}%
\pgfsys@defobject{currentmarker}{\pgfqpoint{0.000000in}{-0.048611in}}{\pgfqpoint{0.000000in}{0.000000in}}{%
\pgfpathmoveto{\pgfqpoint{0.000000in}{0.000000in}}%
\pgfpathlineto{\pgfqpoint{0.000000in}{-0.048611in}}%
\pgfusepath{stroke,fill}%
}%
\begin{pgfscope}%
\pgfsys@transformshift{11.931597in}{3.269444in}%
\pgfsys@useobject{currentmarker}{}%
\end{pgfscope}%
\end{pgfscope}%
\begin{pgfscope}%
\definecolor{textcolor}{rgb}{0.000000,0.000000,0.000000}%
\pgfsetstrokecolor{textcolor}%
\pgfsetfillcolor{textcolor}%
\pgftext[x=11.931597in,y=3.172222in,,top]{\color{textcolor}\sffamily\fontsize{10.000000}{12.000000}\selectfont 0}%
\end{pgfscope}%
\begin{pgfscope}%
\pgfsetbuttcap%
\pgfsetroundjoin%
\definecolor{currentfill}{rgb}{0.000000,0.000000,0.000000}%
\pgfsetfillcolor{currentfill}%
\pgfsetlinewidth{0.803000pt}%
\definecolor{currentstroke}{rgb}{0.000000,0.000000,0.000000}%
\pgfsetstrokecolor{currentstroke}%
\pgfsetdash{}{0pt}%
\pgfsys@defobject{currentmarker}{\pgfqpoint{0.000000in}{-0.048611in}}{\pgfqpoint{0.000000in}{0.000000in}}{%
\pgfpathmoveto{\pgfqpoint{0.000000in}{0.000000in}}%
\pgfpathlineto{\pgfqpoint{0.000000in}{-0.048611in}}%
\pgfusepath{stroke,fill}%
}%
\begin{pgfscope}%
\pgfsys@transformshift{12.515278in}{3.269444in}%
\pgfsys@useobject{currentmarker}{}%
\end{pgfscope}%
\end{pgfscope}%
\begin{pgfscope}%
\definecolor{textcolor}{rgb}{0.000000,0.000000,0.000000}%
\pgfsetstrokecolor{textcolor}%
\pgfsetfillcolor{textcolor}%
\pgftext[x=12.515278in,y=3.172222in,,top]{\color{textcolor}\sffamily\fontsize{10.000000}{12.000000}\selectfont 1}%
\end{pgfscope}%
\begin{pgfscope}%
\pgfsetbuttcap%
\pgfsetroundjoin%
\definecolor{currentfill}{rgb}{0.000000,0.000000,0.000000}%
\pgfsetfillcolor{currentfill}%
\pgfsetlinewidth{0.803000pt}%
\definecolor{currentstroke}{rgb}{0.000000,0.000000,0.000000}%
\pgfsetstrokecolor{currentstroke}%
\pgfsetdash{}{0pt}%
\pgfsys@defobject{currentmarker}{\pgfqpoint{0.000000in}{-0.048611in}}{\pgfqpoint{0.000000in}{0.000000in}}{%
\pgfpathmoveto{\pgfqpoint{0.000000in}{0.000000in}}%
\pgfpathlineto{\pgfqpoint{0.000000in}{-0.048611in}}%
\pgfusepath{stroke,fill}%
}%
\begin{pgfscope}%
\pgfsys@transformshift{13.098958in}{3.269444in}%
\pgfsys@useobject{currentmarker}{}%
\end{pgfscope}%
\end{pgfscope}%
\begin{pgfscope}%
\definecolor{textcolor}{rgb}{0.000000,0.000000,0.000000}%
\pgfsetstrokecolor{textcolor}%
\pgfsetfillcolor{textcolor}%
\pgftext[x=13.098958in,y=3.172222in,,top]{\color{textcolor}\sffamily\fontsize{10.000000}{12.000000}\selectfont 2}%
\end{pgfscope}%
\begin{pgfscope}%
\pgfsetbuttcap%
\pgfsetroundjoin%
\definecolor{currentfill}{rgb}{0.000000,0.000000,0.000000}%
\pgfsetfillcolor{currentfill}%
\pgfsetlinewidth{0.803000pt}%
\definecolor{currentstroke}{rgb}{0.000000,0.000000,0.000000}%
\pgfsetstrokecolor{currentstroke}%
\pgfsetdash{}{0pt}%
\pgfsys@defobject{currentmarker}{\pgfqpoint{0.000000in}{-0.048611in}}{\pgfqpoint{0.000000in}{0.000000in}}{%
\pgfpathmoveto{\pgfqpoint{0.000000in}{0.000000in}}%
\pgfpathlineto{\pgfqpoint{0.000000in}{-0.048611in}}%
\pgfusepath{stroke,fill}%
}%
\begin{pgfscope}%
\pgfsys@transformshift{13.682639in}{3.269444in}%
\pgfsys@useobject{currentmarker}{}%
\end{pgfscope}%
\end{pgfscope}%
\begin{pgfscope}%
\definecolor{textcolor}{rgb}{0.000000,0.000000,0.000000}%
\pgfsetstrokecolor{textcolor}%
\pgfsetfillcolor{textcolor}%
\pgftext[x=13.682639in,y=3.172222in,,top]{\color{textcolor}\sffamily\fontsize{10.000000}{12.000000}\selectfont 3}%
\end{pgfscope}%
\begin{pgfscope}%
\pgfsetbuttcap%
\pgfsetroundjoin%
\definecolor{currentfill}{rgb}{0.000000,0.000000,0.000000}%
\pgfsetfillcolor{currentfill}%
\pgfsetlinewidth{0.803000pt}%
\definecolor{currentstroke}{rgb}{0.000000,0.000000,0.000000}%
\pgfsetstrokecolor{currentstroke}%
\pgfsetdash{}{0pt}%
\pgfsys@defobject{currentmarker}{\pgfqpoint{0.000000in}{-0.048611in}}{\pgfqpoint{0.000000in}{0.000000in}}{%
\pgfpathmoveto{\pgfqpoint{0.000000in}{0.000000in}}%
\pgfpathlineto{\pgfqpoint{0.000000in}{-0.048611in}}%
\pgfusepath{stroke,fill}%
}%
\begin{pgfscope}%
\pgfsys@transformshift{14.266319in}{3.269444in}%
\pgfsys@useobject{currentmarker}{}%
\end{pgfscope}%
\end{pgfscope}%
\begin{pgfscope}%
\definecolor{textcolor}{rgb}{0.000000,0.000000,0.000000}%
\pgfsetstrokecolor{textcolor}%
\pgfsetfillcolor{textcolor}%
\pgftext[x=14.266319in,y=3.172222in,,top]{\color{textcolor}\sffamily\fontsize{10.000000}{12.000000}\selectfont 4}%
\end{pgfscope}%
\begin{pgfscope}%
\definecolor{textcolor}{rgb}{0.000000,0.000000,0.000000}%
\pgfsetstrokecolor{textcolor}%
\pgfsetfillcolor{textcolor}%
\pgftext[x=13.390799in,y=2.982254in,,top]{\color{textcolor}\sffamily\fontsize{30.000000}{36.000000}\selectfont x axis}%
\end{pgfscope}%
\begin{pgfscope}%
\pgfsetbuttcap%
\pgfsetroundjoin%
\definecolor{currentfill}{rgb}{0.000000,0.000000,0.000000}%
\pgfsetfillcolor{currentfill}%
\pgfsetlinewidth{0.803000pt}%
\definecolor{currentstroke}{rgb}{0.000000,0.000000,0.000000}%
\pgfsetstrokecolor{currentstroke}%
\pgfsetdash{}{0pt}%
\pgfsys@defobject{currentmarker}{\pgfqpoint{-0.048611in}{0.000000in}}{\pgfqpoint{-0.000000in}{0.000000in}}{%
\pgfpathmoveto{\pgfqpoint{-0.000000in}{0.000000in}}%
\pgfpathlineto{\pgfqpoint{-0.048611in}{0.000000in}}%
\pgfusepath{stroke,fill}%
}%
\begin{pgfscope}%
\pgfsys@transformshift{11.931597in}{3.269444in}%
\pgfsys@useobject{currentmarker}{}%
\end{pgfscope}%
\end{pgfscope}%
\begin{pgfscope}%
\definecolor{textcolor}{rgb}{0.000000,0.000000,0.000000}%
\pgfsetstrokecolor{textcolor}%
\pgfsetfillcolor{textcolor}%
\pgftext[x=11.746010in, y=3.216683in, left, base]{\color{textcolor}\sffamily\fontsize{10.000000}{12.000000}\selectfont 0}%
\end{pgfscope}%
\begin{pgfscope}%
\pgfsetbuttcap%
\pgfsetroundjoin%
\definecolor{currentfill}{rgb}{0.000000,0.000000,0.000000}%
\pgfsetfillcolor{currentfill}%
\pgfsetlinewidth{0.803000pt}%
\definecolor{currentstroke}{rgb}{0.000000,0.000000,0.000000}%
\pgfsetstrokecolor{currentstroke}%
\pgfsetdash{}{0pt}%
\pgfsys@defobject{currentmarker}{\pgfqpoint{-0.048611in}{0.000000in}}{\pgfqpoint{-0.000000in}{0.000000in}}{%
\pgfpathmoveto{\pgfqpoint{-0.000000in}{0.000000in}}%
\pgfpathlineto{\pgfqpoint{-0.048611in}{0.000000in}}%
\pgfusepath{stroke,fill}%
}%
\begin{pgfscope}%
\pgfsys@transformshift{11.931597in}{3.664583in}%
\pgfsys@useobject{currentmarker}{}%
\end{pgfscope}%
\end{pgfscope}%
\begin{pgfscope}%
\definecolor{textcolor}{rgb}{0.000000,0.000000,0.000000}%
\pgfsetstrokecolor{textcolor}%
\pgfsetfillcolor{textcolor}%
\pgftext[x=11.746010in, y=3.611822in, left, base]{\color{textcolor}\sffamily\fontsize{10.000000}{12.000000}\selectfont 1}%
\end{pgfscope}%
\begin{pgfscope}%
\pgfsetbuttcap%
\pgfsetroundjoin%
\definecolor{currentfill}{rgb}{0.000000,0.000000,0.000000}%
\pgfsetfillcolor{currentfill}%
\pgfsetlinewidth{0.803000pt}%
\definecolor{currentstroke}{rgb}{0.000000,0.000000,0.000000}%
\pgfsetstrokecolor{currentstroke}%
\pgfsetdash{}{0pt}%
\pgfsys@defobject{currentmarker}{\pgfqpoint{-0.048611in}{0.000000in}}{\pgfqpoint{-0.000000in}{0.000000in}}{%
\pgfpathmoveto{\pgfqpoint{-0.000000in}{0.000000in}}%
\pgfpathlineto{\pgfqpoint{-0.048611in}{0.000000in}}%
\pgfusepath{stroke,fill}%
}%
\begin{pgfscope}%
\pgfsys@transformshift{11.931597in}{4.059722in}%
\pgfsys@useobject{currentmarker}{}%
\end{pgfscope}%
\end{pgfscope}%
\begin{pgfscope}%
\definecolor{textcolor}{rgb}{0.000000,0.000000,0.000000}%
\pgfsetstrokecolor{textcolor}%
\pgfsetfillcolor{textcolor}%
\pgftext[x=11.746010in, y=4.006961in, left, base]{\color{textcolor}\sffamily\fontsize{10.000000}{12.000000}\selectfont 2}%
\end{pgfscope}%
\begin{pgfscope}%
\pgfsetbuttcap%
\pgfsetroundjoin%
\definecolor{currentfill}{rgb}{0.000000,0.000000,0.000000}%
\pgfsetfillcolor{currentfill}%
\pgfsetlinewidth{0.803000pt}%
\definecolor{currentstroke}{rgb}{0.000000,0.000000,0.000000}%
\pgfsetstrokecolor{currentstroke}%
\pgfsetdash{}{0pt}%
\pgfsys@defobject{currentmarker}{\pgfqpoint{-0.048611in}{0.000000in}}{\pgfqpoint{-0.000000in}{0.000000in}}{%
\pgfpathmoveto{\pgfqpoint{-0.000000in}{0.000000in}}%
\pgfpathlineto{\pgfqpoint{-0.048611in}{0.000000in}}%
\pgfusepath{stroke,fill}%
}%
\begin{pgfscope}%
\pgfsys@transformshift{11.931597in}{4.454861in}%
\pgfsys@useobject{currentmarker}{}%
\end{pgfscope}%
\end{pgfscope}%
\begin{pgfscope}%
\definecolor{textcolor}{rgb}{0.000000,0.000000,0.000000}%
\pgfsetstrokecolor{textcolor}%
\pgfsetfillcolor{textcolor}%
\pgftext[x=11.746010in, y=4.402100in, left, base]{\color{textcolor}\sffamily\fontsize{10.000000}{12.000000}\selectfont 3}%
\end{pgfscope}%
\begin{pgfscope}%
\definecolor{textcolor}{rgb}{0.000000,0.000000,0.000000}%
\pgfsetstrokecolor{textcolor}%
\pgfsetfillcolor{textcolor}%
\pgftext[x=11.690454in,y=4.059722in,,bottom,rotate=90.000000]{\color{textcolor}\sffamily\fontsize{30.000000}{36.000000}\selectfont y axis}%
\end{pgfscope}%
\begin{pgfscope}%
\pgfsetrectcap%
\pgfsetmiterjoin%
\pgfsetlinewidth{0.803000pt}%
\definecolor{currentstroke}{rgb}{0.000000,0.000000,0.000000}%
\pgfsetstrokecolor{currentstroke}%
\pgfsetdash{}{0pt}%
\pgfpathmoveto{\pgfqpoint{11.931597in}{3.269444in}}%
\pgfpathlineto{\pgfqpoint{11.931597in}{4.850000in}}%
\pgfusepath{stroke}%
\end{pgfscope}%
\begin{pgfscope}%
\pgfsetrectcap%
\pgfsetmiterjoin%
\pgfsetlinewidth{0.803000pt}%
\definecolor{currentstroke}{rgb}{0.000000,0.000000,0.000000}%
\pgfsetstrokecolor{currentstroke}%
\pgfsetdash{}{0pt}%
\pgfpathmoveto{\pgfqpoint{14.850000in}{3.269444in}}%
\pgfpathlineto{\pgfqpoint{14.850000in}{4.850000in}}%
\pgfusepath{stroke}%
\end{pgfscope}%
\begin{pgfscope}%
\pgfsetrectcap%
\pgfsetmiterjoin%
\pgfsetlinewidth{0.803000pt}%
\definecolor{currentstroke}{rgb}{0.000000,0.000000,0.000000}%
\pgfsetstrokecolor{currentstroke}%
\pgfsetdash{}{0pt}%
\pgfpathmoveto{\pgfqpoint{11.931597in}{3.269444in}}%
\pgfpathlineto{\pgfqpoint{14.850000in}{3.269444in}}%
\pgfusepath{stroke}%
\end{pgfscope}%
\begin{pgfscope}%
\pgfsetrectcap%
\pgfsetmiterjoin%
\pgfsetlinewidth{0.803000pt}%
\definecolor{currentstroke}{rgb}{0.000000,0.000000,0.000000}%
\pgfsetstrokecolor{currentstroke}%
\pgfsetdash{}{0pt}%
\pgfpathmoveto{\pgfqpoint{11.931597in}{4.850000in}}%
\pgfpathlineto{\pgfqpoint{14.850000in}{4.850000in}}%
\pgfusepath{stroke}%
\end{pgfscope}%
\begin{pgfscope}%
\pgfsetbuttcap%
\pgfsetmiterjoin%
\definecolor{currentfill}{rgb}{1.000000,1.000000,1.000000}%
\pgfsetfillcolor{currentfill}%
\pgfsetlinewidth{0.000000pt}%
\definecolor{currentstroke}{rgb}{0.000000,0.000000,0.000000}%
\pgfsetstrokecolor{currentstroke}%
\pgfsetstrokeopacity{0.000000}%
\pgfsetdash{}{0pt}%
\pgfpathmoveto{\pgfqpoint{0.794097in}{0.844444in}}%
\pgfpathlineto{\pgfqpoint{3.712500in}{0.844444in}}%
\pgfpathlineto{\pgfqpoint{3.712500in}{2.425000in}}%
\pgfpathlineto{\pgfqpoint{0.794097in}{2.425000in}}%
\pgfpathlineto{\pgfqpoint{0.794097in}{0.844444in}}%
\pgfpathclose%
\pgfusepath{fill}%
\end{pgfscope}%
\begin{pgfscope}%
\pgfpathrectangle{\pgfqpoint{0.794097in}{0.844444in}}{\pgfqpoint{2.918403in}{1.580556in}}%
\pgfusepath{clip}%
\pgfsetbuttcap%
\pgfsetroundjoin%
\definecolor{currentfill}{rgb}{0.984744,0.432910,0.307420}%
\pgfsetfillcolor{currentfill}%
\pgfsetlinewidth{0.000000pt}%
\definecolor{currentstroke}{rgb}{0.000000,0.000000,0.000000}%
\pgfsetstrokecolor{currentstroke}%
\pgfsetdash{}{0pt}%
\pgfpathmoveto{\pgfqpoint{0.794097in}{0.844444in}}%
\pgfpathlineto{\pgfqpoint{0.794097in}{1.239583in}}%
\pgfpathlineto{\pgfqpoint{1.377778in}{1.239583in}}%
\pgfpathlineto{\pgfqpoint{1.377778in}{0.844444in}}%
\pgfpathlineto{\pgfqpoint{0.794097in}{0.844444in}}%
\pgfpathlineto{\pgfqpoint{0.794097in}{0.844444in}}%
\pgfpathclose%
\pgfusepath{fill}%
\end{pgfscope}%
\begin{pgfscope}%
\pgfpathrectangle{\pgfqpoint{0.794097in}{0.844444in}}{\pgfqpoint{2.918403in}{1.580556in}}%
\pgfusepath{clip}%
\pgfsetbuttcap%
\pgfsetroundjoin%
\definecolor{currentfill}{rgb}{0.988235,0.661453,0.548973}%
\pgfsetfillcolor{currentfill}%
\pgfsetlinewidth{0.000000pt}%
\definecolor{currentstroke}{rgb}{0.000000,0.000000,0.000000}%
\pgfsetstrokecolor{currentstroke}%
\pgfsetdash{}{0pt}%
\pgfpathmoveto{\pgfqpoint{1.377778in}{0.844444in}}%
\pgfpathlineto{\pgfqpoint{1.377778in}{1.239583in}}%
\pgfpathlineto{\pgfqpoint{1.961458in}{1.239583in}}%
\pgfpathlineto{\pgfqpoint{1.961458in}{0.844444in}}%
\pgfpathlineto{\pgfqpoint{1.377778in}{0.844444in}}%
\pgfpathlineto{\pgfqpoint{1.377778in}{0.844444in}}%
\pgfpathclose%
\pgfusepath{fill}%
\end{pgfscope}%
\begin{pgfscope}%
\pgfpathrectangle{\pgfqpoint{0.794097in}{0.844444in}}{\pgfqpoint{2.918403in}{1.580556in}}%
\pgfusepath{clip}%
\pgfsetbuttcap%
\pgfsetroundjoin%
\definecolor{currentfill}{rgb}{0.988235,0.590834,0.468020}%
\pgfsetfillcolor{currentfill}%
\pgfsetlinewidth{0.000000pt}%
\definecolor{currentstroke}{rgb}{0.000000,0.000000,0.000000}%
\pgfsetstrokecolor{currentstroke}%
\pgfsetdash{}{0pt}%
\pgfpathmoveto{\pgfqpoint{1.961458in}{0.844444in}}%
\pgfpathlineto{\pgfqpoint{1.961458in}{1.239583in}}%
\pgfpathlineto{\pgfqpoint{2.545139in}{1.239583in}}%
\pgfpathlineto{\pgfqpoint{2.545139in}{0.844444in}}%
\pgfpathlineto{\pgfqpoint{1.961458in}{0.844444in}}%
\pgfpathlineto{\pgfqpoint{1.961458in}{0.844444in}}%
\pgfpathclose%
\pgfusepath{fill}%
\end{pgfscope}%
\begin{pgfscope}%
\pgfpathrectangle{\pgfqpoint{0.794097in}{0.844444in}}{\pgfqpoint{2.918403in}{1.580556in}}%
\pgfusepath{clip}%
\pgfsetbuttcap%
\pgfsetroundjoin%
\definecolor{currentfill}{rgb}{0.994079,0.841446,0.774548}%
\pgfsetfillcolor{currentfill}%
\pgfsetlinewidth{0.000000pt}%
\definecolor{currentstroke}{rgb}{0.000000,0.000000,0.000000}%
\pgfsetstrokecolor{currentstroke}%
\pgfsetdash{}{0pt}%
\pgfpathmoveto{\pgfqpoint{2.545139in}{0.844444in}}%
\pgfpathlineto{\pgfqpoint{2.545139in}{1.239583in}}%
\pgfpathlineto{\pgfqpoint{3.128819in}{1.239583in}}%
\pgfpathlineto{\pgfqpoint{3.128819in}{0.844444in}}%
\pgfpathlineto{\pgfqpoint{2.545139in}{0.844444in}}%
\pgfpathlineto{\pgfqpoint{2.545139in}{0.844444in}}%
\pgfpathclose%
\pgfusepath{fill}%
\end{pgfscope}%
\begin{pgfscope}%
\pgfpathrectangle{\pgfqpoint{0.794097in}{0.844444in}}{\pgfqpoint{2.918403in}{1.580556in}}%
\pgfusepath{clip}%
\pgfsetbuttcap%
\pgfsetroundjoin%
\definecolor{currentfill}{rgb}{1.000000,0.960784,0.941176}%
\pgfsetfillcolor{currentfill}%
\pgfsetlinewidth{0.000000pt}%
\definecolor{currentstroke}{rgb}{0.000000,0.000000,0.000000}%
\pgfsetstrokecolor{currentstroke}%
\pgfsetdash{}{0pt}%
\pgfpathmoveto{\pgfqpoint{3.128819in}{0.844444in}}%
\pgfpathlineto{\pgfqpoint{3.128819in}{1.239583in}}%
\pgfpathlineto{\pgfqpoint{3.712500in}{1.239583in}}%
\pgfpathlineto{\pgfqpoint{3.712500in}{0.844444in}}%
\pgfpathlineto{\pgfqpoint{3.128819in}{0.844444in}}%
\pgfpathlineto{\pgfqpoint{3.128819in}{0.844444in}}%
\pgfpathclose%
\pgfusepath{fill}%
\end{pgfscope}%
\begin{pgfscope}%
\pgfpathrectangle{\pgfqpoint{0.794097in}{0.844444in}}{\pgfqpoint{2.918403in}{1.580556in}}%
\pgfusepath{clip}%
\pgfsetbuttcap%
\pgfsetroundjoin%
\definecolor{currentfill}{rgb}{0.988235,0.676586,0.566321}%
\pgfsetfillcolor{currentfill}%
\pgfsetlinewidth{0.000000pt}%
\definecolor{currentstroke}{rgb}{0.000000,0.000000,0.000000}%
\pgfsetstrokecolor{currentstroke}%
\pgfsetdash{}{0pt}%
\pgfpathmoveto{\pgfqpoint{0.794097in}{1.239583in}}%
\pgfpathlineto{\pgfqpoint{0.794097in}{1.634722in}}%
\pgfpathlineto{\pgfqpoint{1.377778in}{1.634722in}}%
\pgfpathlineto{\pgfqpoint{1.377778in}{1.239583in}}%
\pgfpathlineto{\pgfqpoint{0.794097in}{1.239583in}}%
\pgfpathlineto{\pgfqpoint{0.794097in}{1.239583in}}%
\pgfpathclose%
\pgfusepath{fill}%
\end{pgfscope}%
\begin{pgfscope}%
\pgfpathrectangle{\pgfqpoint{0.794097in}{0.844444in}}{\pgfqpoint{2.918403in}{1.580556in}}%
\pgfusepath{clip}%
\pgfsetbuttcap%
\pgfsetroundjoin%
\definecolor{currentfill}{rgb}{0.988235,0.681630,0.572103}%
\pgfsetfillcolor{currentfill}%
\pgfsetlinewidth{0.000000pt}%
\definecolor{currentstroke}{rgb}{0.000000,0.000000,0.000000}%
\pgfsetstrokecolor{currentstroke}%
\pgfsetdash{}{0pt}%
\pgfpathmoveto{\pgfqpoint{1.377778in}{1.239583in}}%
\pgfpathlineto{\pgfqpoint{1.377778in}{1.634722in}}%
\pgfpathlineto{\pgfqpoint{1.961458in}{1.634722in}}%
\pgfpathlineto{\pgfqpoint{1.961458in}{1.239583in}}%
\pgfpathlineto{\pgfqpoint{1.377778in}{1.239583in}}%
\pgfpathlineto{\pgfqpoint{1.377778in}{1.239583in}}%
\pgfpathclose%
\pgfusepath{fill}%
\end{pgfscope}%
\begin{pgfscope}%
\pgfpathrectangle{\pgfqpoint{0.794097in}{0.844444in}}{\pgfqpoint{2.918403in}{1.580556in}}%
\pgfusepath{clip}%
\pgfsetbuttcap%
\pgfsetroundjoin%
\definecolor{currentfill}{rgb}{0.986098,0.487043,0.361553}%
\pgfsetfillcolor{currentfill}%
\pgfsetlinewidth{0.000000pt}%
\definecolor{currentstroke}{rgb}{0.000000,0.000000,0.000000}%
\pgfsetstrokecolor{currentstroke}%
\pgfsetdash{}{0pt}%
\pgfpathmoveto{\pgfqpoint{1.961458in}{1.239583in}}%
\pgfpathlineto{\pgfqpoint{1.961458in}{1.634722in}}%
\pgfpathlineto{\pgfqpoint{2.545139in}{1.634722in}}%
\pgfpathlineto{\pgfqpoint{2.545139in}{1.239583in}}%
\pgfpathlineto{\pgfqpoint{1.961458in}{1.239583in}}%
\pgfpathlineto{\pgfqpoint{1.961458in}{1.239583in}}%
\pgfpathclose%
\pgfusepath{fill}%
\end{pgfscope}%
\begin{pgfscope}%
\pgfpathrectangle{\pgfqpoint{0.794097in}{0.844444in}}{\pgfqpoint{2.918403in}{1.580556in}}%
\pgfusepath{clip}%
\pgfsetbuttcap%
\pgfsetroundjoin%
\definecolor{currentfill}{rgb}{0.986959,0.521492,0.396002}%
\pgfsetfillcolor{currentfill}%
\pgfsetlinewidth{0.000000pt}%
\definecolor{currentstroke}{rgb}{0.000000,0.000000,0.000000}%
\pgfsetstrokecolor{currentstroke}%
\pgfsetdash{}{0pt}%
\pgfpathmoveto{\pgfqpoint{2.545139in}{1.239583in}}%
\pgfpathlineto{\pgfqpoint{2.545139in}{1.634722in}}%
\pgfpathlineto{\pgfqpoint{3.128819in}{1.634722in}}%
\pgfpathlineto{\pgfqpoint{3.128819in}{1.239583in}}%
\pgfpathlineto{\pgfqpoint{2.545139in}{1.239583in}}%
\pgfpathlineto{\pgfqpoint{2.545139in}{1.239583in}}%
\pgfpathclose%
\pgfusepath{fill}%
\end{pgfscope}%
\begin{pgfscope}%
\pgfpathrectangle{\pgfqpoint{0.794097in}{0.844444in}}{\pgfqpoint{2.918403in}{1.580556in}}%
\pgfusepath{clip}%
\pgfsetbuttcap%
\pgfsetroundjoin%
\definecolor{currentfill}{rgb}{0.988066,0.565782,0.440292}%
\pgfsetfillcolor{currentfill}%
\pgfsetlinewidth{0.000000pt}%
\definecolor{currentstroke}{rgb}{0.000000,0.000000,0.000000}%
\pgfsetstrokecolor{currentstroke}%
\pgfsetdash{}{0pt}%
\pgfpathmoveto{\pgfqpoint{3.128819in}{1.239583in}}%
\pgfpathlineto{\pgfqpoint{3.128819in}{1.634722in}}%
\pgfpathlineto{\pgfqpoint{3.712500in}{1.634722in}}%
\pgfpathlineto{\pgfqpoint{3.712500in}{1.239583in}}%
\pgfpathlineto{\pgfqpoint{3.128819in}{1.239583in}}%
\pgfpathlineto{\pgfqpoint{3.128819in}{1.239583in}}%
\pgfpathclose%
\pgfusepath{fill}%
\end{pgfscope}%
\begin{pgfscope}%
\pgfpathrectangle{\pgfqpoint{0.794097in}{0.844444in}}{\pgfqpoint{2.918403in}{1.580556in}}%
\pgfusepath{clip}%
\pgfsetbuttcap%
\pgfsetroundjoin%
\definecolor{currentfill}{rgb}{0.403922,0.000000,0.050980}%
\pgfsetfillcolor{currentfill}%
\pgfsetlinewidth{0.000000pt}%
\definecolor{currentstroke}{rgb}{0.000000,0.000000,0.000000}%
\pgfsetstrokecolor{currentstroke}%
\pgfsetdash{}{0pt}%
\pgfpathmoveto{\pgfqpoint{0.794097in}{1.634722in}}%
\pgfpathlineto{\pgfqpoint{0.794097in}{2.029861in}}%
\pgfpathlineto{\pgfqpoint{1.377778in}{2.029861in}}%
\pgfpathlineto{\pgfqpoint{1.377778in}{1.634722in}}%
\pgfpathlineto{\pgfqpoint{0.794097in}{1.634722in}}%
\pgfpathlineto{\pgfqpoint{0.794097in}{1.634722in}}%
\pgfpathclose%
\pgfusepath{fill}%
\end{pgfscope}%
\begin{pgfscope}%
\pgfpathrectangle{\pgfqpoint{0.794097in}{0.844444in}}{\pgfqpoint{2.918403in}{1.580556in}}%
\pgfusepath{clip}%
\pgfsetbuttcap%
\pgfsetroundjoin%
\definecolor{currentfill}{rgb}{0.988235,0.580746,0.456455}%
\pgfsetfillcolor{currentfill}%
\pgfsetlinewidth{0.000000pt}%
\definecolor{currentstroke}{rgb}{0.000000,0.000000,0.000000}%
\pgfsetstrokecolor{currentstroke}%
\pgfsetdash{}{0pt}%
\pgfpathmoveto{\pgfqpoint{1.377778in}{1.634722in}}%
\pgfpathlineto{\pgfqpoint{1.377778in}{2.029861in}}%
\pgfpathlineto{\pgfqpoint{1.961458in}{2.029861in}}%
\pgfpathlineto{\pgfqpoint{1.961458in}{1.634722in}}%
\pgfpathlineto{\pgfqpoint{1.377778in}{1.634722in}}%
\pgfpathlineto{\pgfqpoint{1.377778in}{1.634722in}}%
\pgfpathclose%
\pgfusepath{fill}%
\end{pgfscope}%
\begin{pgfscope}%
\pgfpathrectangle{\pgfqpoint{0.794097in}{0.844444in}}{\pgfqpoint{2.918403in}{1.580556in}}%
\pgfusepath{clip}%
\pgfsetbuttcap%
\pgfsetroundjoin%
\definecolor{currentfill}{rgb}{0.984498,0.423068,0.297578}%
\pgfsetfillcolor{currentfill}%
\pgfsetlinewidth{0.000000pt}%
\definecolor{currentstroke}{rgb}{0.000000,0.000000,0.000000}%
\pgfsetstrokecolor{currentstroke}%
\pgfsetdash{}{0pt}%
\pgfpathmoveto{\pgfqpoint{1.961458in}{1.634722in}}%
\pgfpathlineto{\pgfqpoint{1.961458in}{2.029861in}}%
\pgfpathlineto{\pgfqpoint{2.545139in}{2.029861in}}%
\pgfpathlineto{\pgfqpoint{2.545139in}{1.634722in}}%
\pgfpathlineto{\pgfqpoint{1.961458in}{1.634722in}}%
\pgfpathlineto{\pgfqpoint{1.961458in}{1.634722in}}%
\pgfpathclose%
\pgfusepath{fill}%
\end{pgfscope}%
\begin{pgfscope}%
\pgfpathrectangle{\pgfqpoint{0.794097in}{0.844444in}}{\pgfqpoint{2.918403in}{1.580556in}}%
\pgfusepath{clip}%
\pgfsetbuttcap%
\pgfsetroundjoin%
\definecolor{currentfill}{rgb}{0.890196,0.185621,0.152941}%
\pgfsetfillcolor{currentfill}%
\pgfsetlinewidth{0.000000pt}%
\definecolor{currentstroke}{rgb}{0.000000,0.000000,0.000000}%
\pgfsetstrokecolor{currentstroke}%
\pgfsetdash{}{0pt}%
\pgfpathmoveto{\pgfqpoint{2.545139in}{1.634722in}}%
\pgfpathlineto{\pgfqpoint{2.545139in}{2.029861in}}%
\pgfpathlineto{\pgfqpoint{3.128819in}{2.029861in}}%
\pgfpathlineto{\pgfqpoint{3.128819in}{1.634722in}}%
\pgfpathlineto{\pgfqpoint{2.545139in}{1.634722in}}%
\pgfpathlineto{\pgfqpoint{2.545139in}{1.634722in}}%
\pgfpathclose%
\pgfusepath{fill}%
\end{pgfscope}%
\begin{pgfscope}%
\pgfpathrectangle{\pgfqpoint{0.794097in}{0.844444in}}{\pgfqpoint{2.918403in}{1.580556in}}%
\pgfusepath{clip}%
\pgfsetbuttcap%
\pgfsetroundjoin%
\definecolor{currentfill}{rgb}{0.970288,0.360754,0.255133}%
\pgfsetfillcolor{currentfill}%
\pgfsetlinewidth{0.000000pt}%
\definecolor{currentstroke}{rgb}{0.000000,0.000000,0.000000}%
\pgfsetstrokecolor{currentstroke}%
\pgfsetdash{}{0pt}%
\pgfpathmoveto{\pgfqpoint{3.128819in}{1.634722in}}%
\pgfpathlineto{\pgfqpoint{3.128819in}{2.029861in}}%
\pgfpathlineto{\pgfqpoint{3.712500in}{2.029861in}}%
\pgfpathlineto{\pgfqpoint{3.712500in}{1.634722in}}%
\pgfpathlineto{\pgfqpoint{3.128819in}{1.634722in}}%
\pgfpathlineto{\pgfqpoint{3.128819in}{1.634722in}}%
\pgfpathclose%
\pgfusepath{fill}%
\end{pgfscope}%
\begin{pgfscope}%
\pgfpathrectangle{\pgfqpoint{0.794097in}{0.844444in}}{\pgfqpoint{2.918403in}{1.580556in}}%
\pgfusepath{clip}%
\pgfsetbuttcap%
\pgfsetroundjoin%
\definecolor{currentfill}{rgb}{0.647643,0.058962,0.082476}%
\pgfsetfillcolor{currentfill}%
\pgfsetlinewidth{0.000000pt}%
\definecolor{currentstroke}{rgb}{0.000000,0.000000,0.000000}%
\pgfsetstrokecolor{currentstroke}%
\pgfsetdash{}{0pt}%
\pgfpathmoveto{\pgfqpoint{0.794097in}{2.029861in}}%
\pgfpathlineto{\pgfqpoint{0.794097in}{2.425000in}}%
\pgfpathlineto{\pgfqpoint{1.377778in}{2.425000in}}%
\pgfpathlineto{\pgfqpoint{1.377778in}{2.029861in}}%
\pgfpathlineto{\pgfqpoint{0.794097in}{2.029861in}}%
\pgfpathlineto{\pgfqpoint{0.794097in}{2.029861in}}%
\pgfpathclose%
\pgfusepath{fill}%
\end{pgfscope}%
\begin{pgfscope}%
\pgfpathrectangle{\pgfqpoint{0.794097in}{0.844444in}}{\pgfqpoint{2.918403in}{1.580556in}}%
\pgfusepath{clip}%
\pgfsetbuttcap%
\pgfsetroundjoin%
\definecolor{currentfill}{rgb}{0.986098,0.487043,0.361553}%
\pgfsetfillcolor{currentfill}%
\pgfsetlinewidth{0.000000pt}%
\definecolor{currentstroke}{rgb}{0.000000,0.000000,0.000000}%
\pgfsetstrokecolor{currentstroke}%
\pgfsetdash{}{0pt}%
\pgfpathmoveto{\pgfqpoint{1.377778in}{2.029861in}}%
\pgfpathlineto{\pgfqpoint{1.377778in}{2.425000in}}%
\pgfpathlineto{\pgfqpoint{1.961458in}{2.425000in}}%
\pgfpathlineto{\pgfqpoint{1.961458in}{2.029861in}}%
\pgfpathlineto{\pgfqpoint{1.377778in}{2.029861in}}%
\pgfpathlineto{\pgfqpoint{1.377778in}{2.029861in}}%
\pgfpathclose%
\pgfusepath{fill}%
\end{pgfscope}%
\begin{pgfscope}%
\pgfpathrectangle{\pgfqpoint{0.794097in}{0.844444in}}{\pgfqpoint{2.918403in}{1.580556in}}%
\pgfusepath{clip}%
\pgfsetbuttcap%
\pgfsetroundjoin%
\definecolor{currentfill}{rgb}{0.988235,0.656409,0.543191}%
\pgfsetfillcolor{currentfill}%
\pgfsetlinewidth{0.000000pt}%
\definecolor{currentstroke}{rgb}{0.000000,0.000000,0.000000}%
\pgfsetstrokecolor{currentstroke}%
\pgfsetdash{}{0pt}%
\pgfpathmoveto{\pgfqpoint{1.961458in}{2.029861in}}%
\pgfpathlineto{\pgfqpoint{1.961458in}{2.425000in}}%
\pgfpathlineto{\pgfqpoint{2.545139in}{2.425000in}}%
\pgfpathlineto{\pgfqpoint{2.545139in}{2.029861in}}%
\pgfpathlineto{\pgfqpoint{1.961458in}{2.029861in}}%
\pgfpathlineto{\pgfqpoint{1.961458in}{2.029861in}}%
\pgfpathclose%
\pgfusepath{fill}%
\end{pgfscope}%
\begin{pgfscope}%
\pgfpathrectangle{\pgfqpoint{0.794097in}{0.844444in}}{\pgfqpoint{2.918403in}{1.580556in}}%
\pgfusepath{clip}%
\pgfsetbuttcap%
\pgfsetroundjoin%
\definecolor{currentfill}{rgb}{0.958478,0.314494,0.225606}%
\pgfsetfillcolor{currentfill}%
\pgfsetlinewidth{0.000000pt}%
\definecolor{currentstroke}{rgb}{0.000000,0.000000,0.000000}%
\pgfsetstrokecolor{currentstroke}%
\pgfsetdash{}{0pt}%
\pgfpathmoveto{\pgfqpoint{2.545139in}{2.029861in}}%
\pgfpathlineto{\pgfqpoint{2.545139in}{2.425000in}}%
\pgfpathlineto{\pgfqpoint{3.128819in}{2.425000in}}%
\pgfpathlineto{\pgfqpoint{3.128819in}{2.029861in}}%
\pgfpathlineto{\pgfqpoint{2.545139in}{2.029861in}}%
\pgfpathlineto{\pgfqpoint{2.545139in}{2.029861in}}%
\pgfpathclose%
\pgfusepath{fill}%
\end{pgfscope}%
\begin{pgfscope}%
\pgfpathrectangle{\pgfqpoint{0.794097in}{0.844444in}}{\pgfqpoint{2.918403in}{1.580556in}}%
\pgfusepath{clip}%
\pgfsetbuttcap%
\pgfsetroundjoin%
\definecolor{currentfill}{rgb}{0.990634,0.777716,0.690150}%
\pgfsetfillcolor{currentfill}%
\pgfsetlinewidth{0.000000pt}%
\definecolor{currentstroke}{rgb}{0.000000,0.000000,0.000000}%
\pgfsetstrokecolor{currentstroke}%
\pgfsetdash{}{0pt}%
\pgfpathmoveto{\pgfqpoint{3.128819in}{2.029861in}}%
\pgfpathlineto{\pgfqpoint{3.128819in}{2.425000in}}%
\pgfpathlineto{\pgfqpoint{3.712500in}{2.425000in}}%
\pgfpathlineto{\pgfqpoint{3.712500in}{2.029861in}}%
\pgfpathlineto{\pgfqpoint{3.128819in}{2.029861in}}%
\pgfpathlineto{\pgfqpoint{3.128819in}{2.029861in}}%
\pgfpathclose%
\pgfusepath{fill}%
\end{pgfscope}%
\begin{pgfscope}%
\pgfsetbuttcap%
\pgfsetroundjoin%
\definecolor{currentfill}{rgb}{0.000000,0.000000,0.000000}%
\pgfsetfillcolor{currentfill}%
\pgfsetlinewidth{0.803000pt}%
\definecolor{currentstroke}{rgb}{0.000000,0.000000,0.000000}%
\pgfsetstrokecolor{currentstroke}%
\pgfsetdash{}{0pt}%
\pgfsys@defobject{currentmarker}{\pgfqpoint{0.000000in}{-0.048611in}}{\pgfqpoint{0.000000in}{0.000000in}}{%
\pgfpathmoveto{\pgfqpoint{0.000000in}{0.000000in}}%
\pgfpathlineto{\pgfqpoint{0.000000in}{-0.048611in}}%
\pgfusepath{stroke,fill}%
}%
\begin{pgfscope}%
\pgfsys@transformshift{0.794097in}{0.844444in}%
\pgfsys@useobject{currentmarker}{}%
\end{pgfscope}%
\end{pgfscope}%
\begin{pgfscope}%
\definecolor{textcolor}{rgb}{0.000000,0.000000,0.000000}%
\pgfsetstrokecolor{textcolor}%
\pgfsetfillcolor{textcolor}%
\pgftext[x=0.794097in,y=0.747222in,,top]{\color{textcolor}\sffamily\fontsize{10.000000}{12.000000}\selectfont 0}%
\end{pgfscope}%
\begin{pgfscope}%
\pgfsetbuttcap%
\pgfsetroundjoin%
\definecolor{currentfill}{rgb}{0.000000,0.000000,0.000000}%
\pgfsetfillcolor{currentfill}%
\pgfsetlinewidth{0.803000pt}%
\definecolor{currentstroke}{rgb}{0.000000,0.000000,0.000000}%
\pgfsetstrokecolor{currentstroke}%
\pgfsetdash{}{0pt}%
\pgfsys@defobject{currentmarker}{\pgfqpoint{0.000000in}{-0.048611in}}{\pgfqpoint{0.000000in}{0.000000in}}{%
\pgfpathmoveto{\pgfqpoint{0.000000in}{0.000000in}}%
\pgfpathlineto{\pgfqpoint{0.000000in}{-0.048611in}}%
\pgfusepath{stroke,fill}%
}%
\begin{pgfscope}%
\pgfsys@transformshift{1.377778in}{0.844444in}%
\pgfsys@useobject{currentmarker}{}%
\end{pgfscope}%
\end{pgfscope}%
\begin{pgfscope}%
\definecolor{textcolor}{rgb}{0.000000,0.000000,0.000000}%
\pgfsetstrokecolor{textcolor}%
\pgfsetfillcolor{textcolor}%
\pgftext[x=1.377778in,y=0.747222in,,top]{\color{textcolor}\sffamily\fontsize{10.000000}{12.000000}\selectfont 1}%
\end{pgfscope}%
\begin{pgfscope}%
\pgfsetbuttcap%
\pgfsetroundjoin%
\definecolor{currentfill}{rgb}{0.000000,0.000000,0.000000}%
\pgfsetfillcolor{currentfill}%
\pgfsetlinewidth{0.803000pt}%
\definecolor{currentstroke}{rgb}{0.000000,0.000000,0.000000}%
\pgfsetstrokecolor{currentstroke}%
\pgfsetdash{}{0pt}%
\pgfsys@defobject{currentmarker}{\pgfqpoint{0.000000in}{-0.048611in}}{\pgfqpoint{0.000000in}{0.000000in}}{%
\pgfpathmoveto{\pgfqpoint{0.000000in}{0.000000in}}%
\pgfpathlineto{\pgfqpoint{0.000000in}{-0.048611in}}%
\pgfusepath{stroke,fill}%
}%
\begin{pgfscope}%
\pgfsys@transformshift{1.961458in}{0.844444in}%
\pgfsys@useobject{currentmarker}{}%
\end{pgfscope}%
\end{pgfscope}%
\begin{pgfscope}%
\definecolor{textcolor}{rgb}{0.000000,0.000000,0.000000}%
\pgfsetstrokecolor{textcolor}%
\pgfsetfillcolor{textcolor}%
\pgftext[x=1.961458in,y=0.747222in,,top]{\color{textcolor}\sffamily\fontsize{10.000000}{12.000000}\selectfont 2}%
\end{pgfscope}%
\begin{pgfscope}%
\pgfsetbuttcap%
\pgfsetroundjoin%
\definecolor{currentfill}{rgb}{0.000000,0.000000,0.000000}%
\pgfsetfillcolor{currentfill}%
\pgfsetlinewidth{0.803000pt}%
\definecolor{currentstroke}{rgb}{0.000000,0.000000,0.000000}%
\pgfsetstrokecolor{currentstroke}%
\pgfsetdash{}{0pt}%
\pgfsys@defobject{currentmarker}{\pgfqpoint{0.000000in}{-0.048611in}}{\pgfqpoint{0.000000in}{0.000000in}}{%
\pgfpathmoveto{\pgfqpoint{0.000000in}{0.000000in}}%
\pgfpathlineto{\pgfqpoint{0.000000in}{-0.048611in}}%
\pgfusepath{stroke,fill}%
}%
\begin{pgfscope}%
\pgfsys@transformshift{2.545139in}{0.844444in}%
\pgfsys@useobject{currentmarker}{}%
\end{pgfscope}%
\end{pgfscope}%
\begin{pgfscope}%
\definecolor{textcolor}{rgb}{0.000000,0.000000,0.000000}%
\pgfsetstrokecolor{textcolor}%
\pgfsetfillcolor{textcolor}%
\pgftext[x=2.545139in,y=0.747222in,,top]{\color{textcolor}\sffamily\fontsize{10.000000}{12.000000}\selectfont 3}%
\end{pgfscope}%
\begin{pgfscope}%
\pgfsetbuttcap%
\pgfsetroundjoin%
\definecolor{currentfill}{rgb}{0.000000,0.000000,0.000000}%
\pgfsetfillcolor{currentfill}%
\pgfsetlinewidth{0.803000pt}%
\definecolor{currentstroke}{rgb}{0.000000,0.000000,0.000000}%
\pgfsetstrokecolor{currentstroke}%
\pgfsetdash{}{0pt}%
\pgfsys@defobject{currentmarker}{\pgfqpoint{0.000000in}{-0.048611in}}{\pgfqpoint{0.000000in}{0.000000in}}{%
\pgfpathmoveto{\pgfqpoint{0.000000in}{0.000000in}}%
\pgfpathlineto{\pgfqpoint{0.000000in}{-0.048611in}}%
\pgfusepath{stroke,fill}%
}%
\begin{pgfscope}%
\pgfsys@transformshift{3.128819in}{0.844444in}%
\pgfsys@useobject{currentmarker}{}%
\end{pgfscope}%
\end{pgfscope}%
\begin{pgfscope}%
\definecolor{textcolor}{rgb}{0.000000,0.000000,0.000000}%
\pgfsetstrokecolor{textcolor}%
\pgfsetfillcolor{textcolor}%
\pgftext[x=3.128819in,y=0.747222in,,top]{\color{textcolor}\sffamily\fontsize{10.000000}{12.000000}\selectfont 4}%
\end{pgfscope}%
\begin{pgfscope}%
\definecolor{textcolor}{rgb}{0.000000,0.000000,0.000000}%
\pgfsetstrokecolor{textcolor}%
\pgfsetfillcolor{textcolor}%
\pgftext[x=2.253299in,y=0.557254in,,top]{\color{textcolor}\sffamily\fontsize{30.000000}{36.000000}\selectfont x axis}%
\end{pgfscope}%
\begin{pgfscope}%
\pgfsetbuttcap%
\pgfsetroundjoin%
\definecolor{currentfill}{rgb}{0.000000,0.000000,0.000000}%
\pgfsetfillcolor{currentfill}%
\pgfsetlinewidth{0.803000pt}%
\definecolor{currentstroke}{rgb}{0.000000,0.000000,0.000000}%
\pgfsetstrokecolor{currentstroke}%
\pgfsetdash{}{0pt}%
\pgfsys@defobject{currentmarker}{\pgfqpoint{-0.048611in}{0.000000in}}{\pgfqpoint{-0.000000in}{0.000000in}}{%
\pgfpathmoveto{\pgfqpoint{-0.000000in}{0.000000in}}%
\pgfpathlineto{\pgfqpoint{-0.048611in}{0.000000in}}%
\pgfusepath{stroke,fill}%
}%
\begin{pgfscope}%
\pgfsys@transformshift{0.794097in}{0.844444in}%
\pgfsys@useobject{currentmarker}{}%
\end{pgfscope}%
\end{pgfscope}%
\begin{pgfscope}%
\definecolor{textcolor}{rgb}{0.000000,0.000000,0.000000}%
\pgfsetstrokecolor{textcolor}%
\pgfsetfillcolor{textcolor}%
\pgftext[x=0.608510in, y=0.791683in, left, base]{\color{textcolor}\sffamily\fontsize{10.000000}{12.000000}\selectfont 0}%
\end{pgfscope}%
\begin{pgfscope}%
\pgfsetbuttcap%
\pgfsetroundjoin%
\definecolor{currentfill}{rgb}{0.000000,0.000000,0.000000}%
\pgfsetfillcolor{currentfill}%
\pgfsetlinewidth{0.803000pt}%
\definecolor{currentstroke}{rgb}{0.000000,0.000000,0.000000}%
\pgfsetstrokecolor{currentstroke}%
\pgfsetdash{}{0pt}%
\pgfsys@defobject{currentmarker}{\pgfqpoint{-0.048611in}{0.000000in}}{\pgfqpoint{-0.000000in}{0.000000in}}{%
\pgfpathmoveto{\pgfqpoint{-0.000000in}{0.000000in}}%
\pgfpathlineto{\pgfqpoint{-0.048611in}{0.000000in}}%
\pgfusepath{stroke,fill}%
}%
\begin{pgfscope}%
\pgfsys@transformshift{0.794097in}{1.239583in}%
\pgfsys@useobject{currentmarker}{}%
\end{pgfscope}%
\end{pgfscope}%
\begin{pgfscope}%
\definecolor{textcolor}{rgb}{0.000000,0.000000,0.000000}%
\pgfsetstrokecolor{textcolor}%
\pgfsetfillcolor{textcolor}%
\pgftext[x=0.608510in, y=1.186822in, left, base]{\color{textcolor}\sffamily\fontsize{10.000000}{12.000000}\selectfont 1}%
\end{pgfscope}%
\begin{pgfscope}%
\pgfsetbuttcap%
\pgfsetroundjoin%
\definecolor{currentfill}{rgb}{0.000000,0.000000,0.000000}%
\pgfsetfillcolor{currentfill}%
\pgfsetlinewidth{0.803000pt}%
\definecolor{currentstroke}{rgb}{0.000000,0.000000,0.000000}%
\pgfsetstrokecolor{currentstroke}%
\pgfsetdash{}{0pt}%
\pgfsys@defobject{currentmarker}{\pgfqpoint{-0.048611in}{0.000000in}}{\pgfqpoint{-0.000000in}{0.000000in}}{%
\pgfpathmoveto{\pgfqpoint{-0.000000in}{0.000000in}}%
\pgfpathlineto{\pgfqpoint{-0.048611in}{0.000000in}}%
\pgfusepath{stroke,fill}%
}%
\begin{pgfscope}%
\pgfsys@transformshift{0.794097in}{1.634722in}%
\pgfsys@useobject{currentmarker}{}%
\end{pgfscope}%
\end{pgfscope}%
\begin{pgfscope}%
\definecolor{textcolor}{rgb}{0.000000,0.000000,0.000000}%
\pgfsetstrokecolor{textcolor}%
\pgfsetfillcolor{textcolor}%
\pgftext[x=0.608510in, y=1.581961in, left, base]{\color{textcolor}\sffamily\fontsize{10.000000}{12.000000}\selectfont 2}%
\end{pgfscope}%
\begin{pgfscope}%
\pgfsetbuttcap%
\pgfsetroundjoin%
\definecolor{currentfill}{rgb}{0.000000,0.000000,0.000000}%
\pgfsetfillcolor{currentfill}%
\pgfsetlinewidth{0.803000pt}%
\definecolor{currentstroke}{rgb}{0.000000,0.000000,0.000000}%
\pgfsetstrokecolor{currentstroke}%
\pgfsetdash{}{0pt}%
\pgfsys@defobject{currentmarker}{\pgfqpoint{-0.048611in}{0.000000in}}{\pgfqpoint{-0.000000in}{0.000000in}}{%
\pgfpathmoveto{\pgfqpoint{-0.000000in}{0.000000in}}%
\pgfpathlineto{\pgfqpoint{-0.048611in}{0.000000in}}%
\pgfusepath{stroke,fill}%
}%
\begin{pgfscope}%
\pgfsys@transformshift{0.794097in}{2.029861in}%
\pgfsys@useobject{currentmarker}{}%
\end{pgfscope}%
\end{pgfscope}%
\begin{pgfscope}%
\definecolor{textcolor}{rgb}{0.000000,0.000000,0.000000}%
\pgfsetstrokecolor{textcolor}%
\pgfsetfillcolor{textcolor}%
\pgftext[x=0.608510in, y=1.977100in, left, base]{\color{textcolor}\sffamily\fontsize{10.000000}{12.000000}\selectfont 3}%
\end{pgfscope}%
\begin{pgfscope}%
\definecolor{textcolor}{rgb}{0.000000,0.000000,0.000000}%
\pgfsetstrokecolor{textcolor}%
\pgfsetfillcolor{textcolor}%
\pgftext[x=0.552954in,y=1.634722in,,bottom,rotate=90.000000]{\color{textcolor}\sffamily\fontsize{30.000000}{36.000000}\selectfont y axis}%
\end{pgfscope}%
\begin{pgfscope}%
\pgfsetrectcap%
\pgfsetmiterjoin%
\pgfsetlinewidth{0.803000pt}%
\definecolor{currentstroke}{rgb}{0.000000,0.000000,0.000000}%
\pgfsetstrokecolor{currentstroke}%
\pgfsetdash{}{0pt}%
\pgfpathmoveto{\pgfqpoint{0.794097in}{0.844444in}}%
\pgfpathlineto{\pgfqpoint{0.794097in}{2.425000in}}%
\pgfusepath{stroke}%
\end{pgfscope}%
\begin{pgfscope}%
\pgfsetrectcap%
\pgfsetmiterjoin%
\pgfsetlinewidth{0.803000pt}%
\definecolor{currentstroke}{rgb}{0.000000,0.000000,0.000000}%
\pgfsetstrokecolor{currentstroke}%
\pgfsetdash{}{0pt}%
\pgfpathmoveto{\pgfqpoint{3.712500in}{0.844444in}}%
\pgfpathlineto{\pgfqpoint{3.712500in}{2.425000in}}%
\pgfusepath{stroke}%
\end{pgfscope}%
\begin{pgfscope}%
\pgfsetrectcap%
\pgfsetmiterjoin%
\pgfsetlinewidth{0.803000pt}%
\definecolor{currentstroke}{rgb}{0.000000,0.000000,0.000000}%
\pgfsetstrokecolor{currentstroke}%
\pgfsetdash{}{0pt}%
\pgfpathmoveto{\pgfqpoint{0.794097in}{0.844444in}}%
\pgfpathlineto{\pgfqpoint{3.712500in}{0.844444in}}%
\pgfusepath{stroke}%
\end{pgfscope}%
\begin{pgfscope}%
\pgfsetrectcap%
\pgfsetmiterjoin%
\pgfsetlinewidth{0.803000pt}%
\definecolor{currentstroke}{rgb}{0.000000,0.000000,0.000000}%
\pgfsetstrokecolor{currentstroke}%
\pgfsetdash{}{0pt}%
\pgfpathmoveto{\pgfqpoint{0.794097in}{2.425000in}}%
\pgfpathlineto{\pgfqpoint{3.712500in}{2.425000in}}%
\pgfusepath{stroke}%
\end{pgfscope}%
\begin{pgfscope}%
\pgfsetbuttcap%
\pgfsetmiterjoin%
\definecolor{currentfill}{rgb}{1.000000,1.000000,1.000000}%
\pgfsetfillcolor{currentfill}%
\pgfsetlinewidth{0.000000pt}%
\definecolor{currentstroke}{rgb}{0.000000,0.000000,0.000000}%
\pgfsetstrokecolor{currentstroke}%
\pgfsetstrokeopacity{0.000000}%
\pgfsetdash{}{0pt}%
\pgfpathmoveto{\pgfqpoint{4.506597in}{0.844444in}}%
\pgfpathlineto{\pgfqpoint{7.425000in}{0.844444in}}%
\pgfpathlineto{\pgfqpoint{7.425000in}{2.425000in}}%
\pgfpathlineto{\pgfqpoint{4.506597in}{2.425000in}}%
\pgfpathlineto{\pgfqpoint{4.506597in}{0.844444in}}%
\pgfpathclose%
\pgfusepath{fill}%
\end{pgfscope}%
\begin{pgfscope}%
\pgfpathrectangle{\pgfqpoint{4.506597in}{0.844444in}}{\pgfqpoint{2.918403in}{1.580556in}}%
\pgfusepath{clip}%
\pgfsetbuttcap%
\pgfsetroundjoin%
\definecolor{currentfill}{rgb}{0.988235,0.661453,0.548973}%
\pgfsetfillcolor{currentfill}%
\pgfsetlinewidth{0.000000pt}%
\definecolor{currentstroke}{rgb}{0.000000,0.000000,0.000000}%
\pgfsetstrokecolor{currentstroke}%
\pgfsetdash{}{0pt}%
\pgfpathmoveto{\pgfqpoint{4.506597in}{0.844444in}}%
\pgfpathlineto{\pgfqpoint{4.506597in}{1.239583in}}%
\pgfpathlineto{\pgfqpoint{5.090278in}{1.239583in}}%
\pgfpathlineto{\pgfqpoint{5.090278in}{0.844444in}}%
\pgfpathlineto{\pgfqpoint{4.506597in}{0.844444in}}%
\pgfpathlineto{\pgfqpoint{4.506597in}{0.844444in}}%
\pgfpathclose%
\pgfusepath{fill}%
\end{pgfscope}%
\begin{pgfscope}%
\pgfpathrectangle{\pgfqpoint{4.506597in}{0.844444in}}{\pgfqpoint{2.918403in}{1.580556in}}%
\pgfusepath{clip}%
\pgfsetbuttcap%
\pgfsetroundjoin%
\definecolor{currentfill}{rgb}{0.988235,0.590834,0.468020}%
\pgfsetfillcolor{currentfill}%
\pgfsetlinewidth{0.000000pt}%
\definecolor{currentstroke}{rgb}{0.000000,0.000000,0.000000}%
\pgfsetstrokecolor{currentstroke}%
\pgfsetdash{}{0pt}%
\pgfpathmoveto{\pgfqpoint{5.090278in}{0.844444in}}%
\pgfpathlineto{\pgfqpoint{5.090278in}{1.239583in}}%
\pgfpathlineto{\pgfqpoint{5.673958in}{1.239583in}}%
\pgfpathlineto{\pgfqpoint{5.673958in}{0.844444in}}%
\pgfpathlineto{\pgfqpoint{5.090278in}{0.844444in}}%
\pgfpathlineto{\pgfqpoint{5.090278in}{0.844444in}}%
\pgfpathclose%
\pgfusepath{fill}%
\end{pgfscope}%
\begin{pgfscope}%
\pgfpathrectangle{\pgfqpoint{4.506597in}{0.844444in}}{\pgfqpoint{2.918403in}{1.580556in}}%
\pgfusepath{clip}%
\pgfsetbuttcap%
\pgfsetroundjoin%
\definecolor{currentfill}{rgb}{0.994079,0.841446,0.774548}%
\pgfsetfillcolor{currentfill}%
\pgfsetlinewidth{0.000000pt}%
\definecolor{currentstroke}{rgb}{0.000000,0.000000,0.000000}%
\pgfsetstrokecolor{currentstroke}%
\pgfsetdash{}{0pt}%
\pgfpathmoveto{\pgfqpoint{5.673958in}{0.844444in}}%
\pgfpathlineto{\pgfqpoint{5.673958in}{1.239583in}}%
\pgfpathlineto{\pgfqpoint{6.257639in}{1.239583in}}%
\pgfpathlineto{\pgfqpoint{6.257639in}{0.844444in}}%
\pgfpathlineto{\pgfqpoint{5.673958in}{0.844444in}}%
\pgfpathlineto{\pgfqpoint{5.673958in}{0.844444in}}%
\pgfpathclose%
\pgfusepath{fill}%
\end{pgfscope}%
\begin{pgfscope}%
\pgfpathrectangle{\pgfqpoint{4.506597in}{0.844444in}}{\pgfqpoint{2.918403in}{1.580556in}}%
\pgfusepath{clip}%
\pgfsetbuttcap%
\pgfsetroundjoin%
\definecolor{currentfill}{rgb}{1.000000,0.960784,0.941176}%
\pgfsetfillcolor{currentfill}%
\pgfsetlinewidth{0.000000pt}%
\definecolor{currentstroke}{rgb}{0.000000,0.000000,0.000000}%
\pgfsetstrokecolor{currentstroke}%
\pgfsetdash{}{0pt}%
\pgfpathmoveto{\pgfqpoint{6.257639in}{0.844444in}}%
\pgfpathlineto{\pgfqpoint{6.257639in}{1.239583in}}%
\pgfpathlineto{\pgfqpoint{6.841319in}{1.239583in}}%
\pgfpathlineto{\pgfqpoint{6.841319in}{0.844444in}}%
\pgfpathlineto{\pgfqpoint{6.257639in}{0.844444in}}%
\pgfpathlineto{\pgfqpoint{6.257639in}{0.844444in}}%
\pgfpathclose%
\pgfusepath{fill}%
\end{pgfscope}%
\begin{pgfscope}%
\pgfpathrectangle{\pgfqpoint{4.506597in}{0.844444in}}{\pgfqpoint{2.918403in}{1.580556in}}%
\pgfusepath{clip}%
\pgfsetbuttcap%
\pgfsetroundjoin%
\definecolor{currentfill}{rgb}{0.984744,0.432910,0.307420}%
\pgfsetfillcolor{currentfill}%
\pgfsetlinewidth{0.000000pt}%
\definecolor{currentstroke}{rgb}{0.000000,0.000000,0.000000}%
\pgfsetstrokecolor{currentstroke}%
\pgfsetdash{}{0pt}%
\pgfpathmoveto{\pgfqpoint{6.841319in}{0.844444in}}%
\pgfpathlineto{\pgfqpoint{6.841319in}{1.239583in}}%
\pgfpathlineto{\pgfqpoint{7.425000in}{1.239583in}}%
\pgfpathlineto{\pgfqpoint{7.425000in}{0.844444in}}%
\pgfpathlineto{\pgfqpoint{6.841319in}{0.844444in}}%
\pgfpathlineto{\pgfqpoint{6.841319in}{0.844444in}}%
\pgfpathclose%
\pgfusepath{fill}%
\end{pgfscope}%
\begin{pgfscope}%
\pgfpathrectangle{\pgfqpoint{4.506597in}{0.844444in}}{\pgfqpoint{2.918403in}{1.580556in}}%
\pgfusepath{clip}%
\pgfsetbuttcap%
\pgfsetroundjoin%
\definecolor{currentfill}{rgb}{0.988235,0.681630,0.572103}%
\pgfsetfillcolor{currentfill}%
\pgfsetlinewidth{0.000000pt}%
\definecolor{currentstroke}{rgb}{0.000000,0.000000,0.000000}%
\pgfsetstrokecolor{currentstroke}%
\pgfsetdash{}{0pt}%
\pgfpathmoveto{\pgfqpoint{4.506597in}{1.239583in}}%
\pgfpathlineto{\pgfqpoint{4.506597in}{1.634722in}}%
\pgfpathlineto{\pgfqpoint{5.090278in}{1.634722in}}%
\pgfpathlineto{\pgfqpoint{5.090278in}{1.239583in}}%
\pgfpathlineto{\pgfqpoint{4.506597in}{1.239583in}}%
\pgfpathlineto{\pgfqpoint{4.506597in}{1.239583in}}%
\pgfpathclose%
\pgfusepath{fill}%
\end{pgfscope}%
\begin{pgfscope}%
\pgfpathrectangle{\pgfqpoint{4.506597in}{0.844444in}}{\pgfqpoint{2.918403in}{1.580556in}}%
\pgfusepath{clip}%
\pgfsetbuttcap%
\pgfsetroundjoin%
\definecolor{currentfill}{rgb}{0.986098,0.487043,0.361553}%
\pgfsetfillcolor{currentfill}%
\pgfsetlinewidth{0.000000pt}%
\definecolor{currentstroke}{rgb}{0.000000,0.000000,0.000000}%
\pgfsetstrokecolor{currentstroke}%
\pgfsetdash{}{0pt}%
\pgfpathmoveto{\pgfqpoint{5.090278in}{1.239583in}}%
\pgfpathlineto{\pgfqpoint{5.090278in}{1.634722in}}%
\pgfpathlineto{\pgfqpoint{5.673958in}{1.634722in}}%
\pgfpathlineto{\pgfqpoint{5.673958in}{1.239583in}}%
\pgfpathlineto{\pgfqpoint{5.090278in}{1.239583in}}%
\pgfpathlineto{\pgfqpoint{5.090278in}{1.239583in}}%
\pgfpathclose%
\pgfusepath{fill}%
\end{pgfscope}%
\begin{pgfscope}%
\pgfpathrectangle{\pgfqpoint{4.506597in}{0.844444in}}{\pgfqpoint{2.918403in}{1.580556in}}%
\pgfusepath{clip}%
\pgfsetbuttcap%
\pgfsetroundjoin%
\definecolor{currentfill}{rgb}{0.986959,0.521492,0.396002}%
\pgfsetfillcolor{currentfill}%
\pgfsetlinewidth{0.000000pt}%
\definecolor{currentstroke}{rgb}{0.000000,0.000000,0.000000}%
\pgfsetstrokecolor{currentstroke}%
\pgfsetdash{}{0pt}%
\pgfpathmoveto{\pgfqpoint{5.673958in}{1.239583in}}%
\pgfpathlineto{\pgfqpoint{5.673958in}{1.634722in}}%
\pgfpathlineto{\pgfqpoint{6.257639in}{1.634722in}}%
\pgfpathlineto{\pgfqpoint{6.257639in}{1.239583in}}%
\pgfpathlineto{\pgfqpoint{5.673958in}{1.239583in}}%
\pgfpathlineto{\pgfqpoint{5.673958in}{1.239583in}}%
\pgfpathclose%
\pgfusepath{fill}%
\end{pgfscope}%
\begin{pgfscope}%
\pgfpathrectangle{\pgfqpoint{4.506597in}{0.844444in}}{\pgfqpoint{2.918403in}{1.580556in}}%
\pgfusepath{clip}%
\pgfsetbuttcap%
\pgfsetroundjoin%
\definecolor{currentfill}{rgb}{0.988066,0.565782,0.440292}%
\pgfsetfillcolor{currentfill}%
\pgfsetlinewidth{0.000000pt}%
\definecolor{currentstroke}{rgb}{0.000000,0.000000,0.000000}%
\pgfsetstrokecolor{currentstroke}%
\pgfsetdash{}{0pt}%
\pgfpathmoveto{\pgfqpoint{6.257639in}{1.239583in}}%
\pgfpathlineto{\pgfqpoint{6.257639in}{1.634722in}}%
\pgfpathlineto{\pgfqpoint{6.841319in}{1.634722in}}%
\pgfpathlineto{\pgfqpoint{6.841319in}{1.239583in}}%
\pgfpathlineto{\pgfqpoint{6.257639in}{1.239583in}}%
\pgfpathlineto{\pgfqpoint{6.257639in}{1.239583in}}%
\pgfpathclose%
\pgfusepath{fill}%
\end{pgfscope}%
\begin{pgfscope}%
\pgfpathrectangle{\pgfqpoint{4.506597in}{0.844444in}}{\pgfqpoint{2.918403in}{1.580556in}}%
\pgfusepath{clip}%
\pgfsetbuttcap%
\pgfsetroundjoin%
\definecolor{currentfill}{rgb}{0.988235,0.676586,0.566321}%
\pgfsetfillcolor{currentfill}%
\pgfsetlinewidth{0.000000pt}%
\definecolor{currentstroke}{rgb}{0.000000,0.000000,0.000000}%
\pgfsetstrokecolor{currentstroke}%
\pgfsetdash{}{0pt}%
\pgfpathmoveto{\pgfqpoint{6.841319in}{1.239583in}}%
\pgfpathlineto{\pgfqpoint{6.841319in}{1.634722in}}%
\pgfpathlineto{\pgfqpoint{7.425000in}{1.634722in}}%
\pgfpathlineto{\pgfqpoint{7.425000in}{1.239583in}}%
\pgfpathlineto{\pgfqpoint{6.841319in}{1.239583in}}%
\pgfpathlineto{\pgfqpoint{6.841319in}{1.239583in}}%
\pgfpathclose%
\pgfusepath{fill}%
\end{pgfscope}%
\begin{pgfscope}%
\pgfpathrectangle{\pgfqpoint{4.506597in}{0.844444in}}{\pgfqpoint{2.918403in}{1.580556in}}%
\pgfusepath{clip}%
\pgfsetbuttcap%
\pgfsetroundjoin%
\definecolor{currentfill}{rgb}{0.988235,0.580746,0.456455}%
\pgfsetfillcolor{currentfill}%
\pgfsetlinewidth{0.000000pt}%
\definecolor{currentstroke}{rgb}{0.000000,0.000000,0.000000}%
\pgfsetstrokecolor{currentstroke}%
\pgfsetdash{}{0pt}%
\pgfpathmoveto{\pgfqpoint{4.506597in}{1.634722in}}%
\pgfpathlineto{\pgfqpoint{4.506597in}{2.029861in}}%
\pgfpathlineto{\pgfqpoint{5.090278in}{2.029861in}}%
\pgfpathlineto{\pgfqpoint{5.090278in}{1.634722in}}%
\pgfpathlineto{\pgfqpoint{4.506597in}{1.634722in}}%
\pgfpathlineto{\pgfqpoint{4.506597in}{1.634722in}}%
\pgfpathclose%
\pgfusepath{fill}%
\end{pgfscope}%
\begin{pgfscope}%
\pgfpathrectangle{\pgfqpoint{4.506597in}{0.844444in}}{\pgfqpoint{2.918403in}{1.580556in}}%
\pgfusepath{clip}%
\pgfsetbuttcap%
\pgfsetroundjoin%
\definecolor{currentfill}{rgb}{0.984498,0.423068,0.297578}%
\pgfsetfillcolor{currentfill}%
\pgfsetlinewidth{0.000000pt}%
\definecolor{currentstroke}{rgb}{0.000000,0.000000,0.000000}%
\pgfsetstrokecolor{currentstroke}%
\pgfsetdash{}{0pt}%
\pgfpathmoveto{\pgfqpoint{5.090278in}{1.634722in}}%
\pgfpathlineto{\pgfqpoint{5.090278in}{2.029861in}}%
\pgfpathlineto{\pgfqpoint{5.673958in}{2.029861in}}%
\pgfpathlineto{\pgfqpoint{5.673958in}{1.634722in}}%
\pgfpathlineto{\pgfqpoint{5.090278in}{1.634722in}}%
\pgfpathlineto{\pgfqpoint{5.090278in}{1.634722in}}%
\pgfpathclose%
\pgfusepath{fill}%
\end{pgfscope}%
\begin{pgfscope}%
\pgfpathrectangle{\pgfqpoint{4.506597in}{0.844444in}}{\pgfqpoint{2.918403in}{1.580556in}}%
\pgfusepath{clip}%
\pgfsetbuttcap%
\pgfsetroundjoin%
\definecolor{currentfill}{rgb}{0.890196,0.185621,0.152941}%
\pgfsetfillcolor{currentfill}%
\pgfsetlinewidth{0.000000pt}%
\definecolor{currentstroke}{rgb}{0.000000,0.000000,0.000000}%
\pgfsetstrokecolor{currentstroke}%
\pgfsetdash{}{0pt}%
\pgfpathmoveto{\pgfqpoint{5.673958in}{1.634722in}}%
\pgfpathlineto{\pgfqpoint{5.673958in}{2.029861in}}%
\pgfpathlineto{\pgfqpoint{6.257639in}{2.029861in}}%
\pgfpathlineto{\pgfqpoint{6.257639in}{1.634722in}}%
\pgfpathlineto{\pgfqpoint{5.673958in}{1.634722in}}%
\pgfpathlineto{\pgfqpoint{5.673958in}{1.634722in}}%
\pgfpathclose%
\pgfusepath{fill}%
\end{pgfscope}%
\begin{pgfscope}%
\pgfpathrectangle{\pgfqpoint{4.506597in}{0.844444in}}{\pgfqpoint{2.918403in}{1.580556in}}%
\pgfusepath{clip}%
\pgfsetbuttcap%
\pgfsetroundjoin%
\definecolor{currentfill}{rgb}{0.970288,0.360754,0.255133}%
\pgfsetfillcolor{currentfill}%
\pgfsetlinewidth{0.000000pt}%
\definecolor{currentstroke}{rgb}{0.000000,0.000000,0.000000}%
\pgfsetstrokecolor{currentstroke}%
\pgfsetdash{}{0pt}%
\pgfpathmoveto{\pgfqpoint{6.257639in}{1.634722in}}%
\pgfpathlineto{\pgfqpoint{6.257639in}{2.029861in}}%
\pgfpathlineto{\pgfqpoint{6.841319in}{2.029861in}}%
\pgfpathlineto{\pgfqpoint{6.841319in}{1.634722in}}%
\pgfpathlineto{\pgfqpoint{6.257639in}{1.634722in}}%
\pgfpathlineto{\pgfqpoint{6.257639in}{1.634722in}}%
\pgfpathclose%
\pgfusepath{fill}%
\end{pgfscope}%
\begin{pgfscope}%
\pgfpathrectangle{\pgfqpoint{4.506597in}{0.844444in}}{\pgfqpoint{2.918403in}{1.580556in}}%
\pgfusepath{clip}%
\pgfsetbuttcap%
\pgfsetroundjoin%
\definecolor{currentfill}{rgb}{0.403922,0.000000,0.050980}%
\pgfsetfillcolor{currentfill}%
\pgfsetlinewidth{0.000000pt}%
\definecolor{currentstroke}{rgb}{0.000000,0.000000,0.000000}%
\pgfsetstrokecolor{currentstroke}%
\pgfsetdash{}{0pt}%
\pgfpathmoveto{\pgfqpoint{6.841319in}{1.634722in}}%
\pgfpathlineto{\pgfqpoint{6.841319in}{2.029861in}}%
\pgfpathlineto{\pgfqpoint{7.425000in}{2.029861in}}%
\pgfpathlineto{\pgfqpoint{7.425000in}{1.634722in}}%
\pgfpathlineto{\pgfqpoint{6.841319in}{1.634722in}}%
\pgfpathlineto{\pgfqpoint{6.841319in}{1.634722in}}%
\pgfpathclose%
\pgfusepath{fill}%
\end{pgfscope}%
\begin{pgfscope}%
\pgfpathrectangle{\pgfqpoint{4.506597in}{0.844444in}}{\pgfqpoint{2.918403in}{1.580556in}}%
\pgfusepath{clip}%
\pgfsetbuttcap%
\pgfsetroundjoin%
\definecolor{currentfill}{rgb}{0.986098,0.487043,0.361553}%
\pgfsetfillcolor{currentfill}%
\pgfsetlinewidth{0.000000pt}%
\definecolor{currentstroke}{rgb}{0.000000,0.000000,0.000000}%
\pgfsetstrokecolor{currentstroke}%
\pgfsetdash{}{0pt}%
\pgfpathmoveto{\pgfqpoint{4.506597in}{2.029861in}}%
\pgfpathlineto{\pgfqpoint{4.506597in}{2.425000in}}%
\pgfpathlineto{\pgfqpoint{5.090278in}{2.425000in}}%
\pgfpathlineto{\pgfqpoint{5.090278in}{2.029861in}}%
\pgfpathlineto{\pgfqpoint{4.506597in}{2.029861in}}%
\pgfpathlineto{\pgfqpoint{4.506597in}{2.029861in}}%
\pgfpathclose%
\pgfusepath{fill}%
\end{pgfscope}%
\begin{pgfscope}%
\pgfpathrectangle{\pgfqpoint{4.506597in}{0.844444in}}{\pgfqpoint{2.918403in}{1.580556in}}%
\pgfusepath{clip}%
\pgfsetbuttcap%
\pgfsetroundjoin%
\definecolor{currentfill}{rgb}{0.988235,0.656409,0.543191}%
\pgfsetfillcolor{currentfill}%
\pgfsetlinewidth{0.000000pt}%
\definecolor{currentstroke}{rgb}{0.000000,0.000000,0.000000}%
\pgfsetstrokecolor{currentstroke}%
\pgfsetdash{}{0pt}%
\pgfpathmoveto{\pgfqpoint{5.090278in}{2.029861in}}%
\pgfpathlineto{\pgfqpoint{5.090278in}{2.425000in}}%
\pgfpathlineto{\pgfqpoint{5.673958in}{2.425000in}}%
\pgfpathlineto{\pgfqpoint{5.673958in}{2.029861in}}%
\pgfpathlineto{\pgfqpoint{5.090278in}{2.029861in}}%
\pgfpathlineto{\pgfqpoint{5.090278in}{2.029861in}}%
\pgfpathclose%
\pgfusepath{fill}%
\end{pgfscope}%
\begin{pgfscope}%
\pgfpathrectangle{\pgfqpoint{4.506597in}{0.844444in}}{\pgfqpoint{2.918403in}{1.580556in}}%
\pgfusepath{clip}%
\pgfsetbuttcap%
\pgfsetroundjoin%
\definecolor{currentfill}{rgb}{0.958478,0.314494,0.225606}%
\pgfsetfillcolor{currentfill}%
\pgfsetlinewidth{0.000000pt}%
\definecolor{currentstroke}{rgb}{0.000000,0.000000,0.000000}%
\pgfsetstrokecolor{currentstroke}%
\pgfsetdash{}{0pt}%
\pgfpathmoveto{\pgfqpoint{5.673958in}{2.029861in}}%
\pgfpathlineto{\pgfqpoint{5.673958in}{2.425000in}}%
\pgfpathlineto{\pgfqpoint{6.257639in}{2.425000in}}%
\pgfpathlineto{\pgfqpoint{6.257639in}{2.029861in}}%
\pgfpathlineto{\pgfqpoint{5.673958in}{2.029861in}}%
\pgfpathlineto{\pgfqpoint{5.673958in}{2.029861in}}%
\pgfpathclose%
\pgfusepath{fill}%
\end{pgfscope}%
\begin{pgfscope}%
\pgfpathrectangle{\pgfqpoint{4.506597in}{0.844444in}}{\pgfqpoint{2.918403in}{1.580556in}}%
\pgfusepath{clip}%
\pgfsetbuttcap%
\pgfsetroundjoin%
\definecolor{currentfill}{rgb}{0.990634,0.777716,0.690150}%
\pgfsetfillcolor{currentfill}%
\pgfsetlinewidth{0.000000pt}%
\definecolor{currentstroke}{rgb}{0.000000,0.000000,0.000000}%
\pgfsetstrokecolor{currentstroke}%
\pgfsetdash{}{0pt}%
\pgfpathmoveto{\pgfqpoint{6.257639in}{2.029861in}}%
\pgfpathlineto{\pgfqpoint{6.257639in}{2.425000in}}%
\pgfpathlineto{\pgfqpoint{6.841319in}{2.425000in}}%
\pgfpathlineto{\pgfqpoint{6.841319in}{2.029861in}}%
\pgfpathlineto{\pgfqpoint{6.257639in}{2.029861in}}%
\pgfpathlineto{\pgfqpoint{6.257639in}{2.029861in}}%
\pgfpathclose%
\pgfusepath{fill}%
\end{pgfscope}%
\begin{pgfscope}%
\pgfpathrectangle{\pgfqpoint{4.506597in}{0.844444in}}{\pgfqpoint{2.918403in}{1.580556in}}%
\pgfusepath{clip}%
\pgfsetbuttcap%
\pgfsetroundjoin%
\definecolor{currentfill}{rgb}{0.647643,0.058962,0.082476}%
\pgfsetfillcolor{currentfill}%
\pgfsetlinewidth{0.000000pt}%
\definecolor{currentstroke}{rgb}{0.000000,0.000000,0.000000}%
\pgfsetstrokecolor{currentstroke}%
\pgfsetdash{}{0pt}%
\pgfpathmoveto{\pgfqpoint{6.841319in}{2.029861in}}%
\pgfpathlineto{\pgfqpoint{6.841319in}{2.425000in}}%
\pgfpathlineto{\pgfqpoint{7.425000in}{2.425000in}}%
\pgfpathlineto{\pgfqpoint{7.425000in}{2.029861in}}%
\pgfpathlineto{\pgfqpoint{6.841319in}{2.029861in}}%
\pgfpathlineto{\pgfqpoint{6.841319in}{2.029861in}}%
\pgfpathclose%
\pgfusepath{fill}%
\end{pgfscope}%
\begin{pgfscope}%
\pgfsetbuttcap%
\pgfsetroundjoin%
\definecolor{currentfill}{rgb}{0.000000,0.000000,0.000000}%
\pgfsetfillcolor{currentfill}%
\pgfsetlinewidth{0.803000pt}%
\definecolor{currentstroke}{rgb}{0.000000,0.000000,0.000000}%
\pgfsetstrokecolor{currentstroke}%
\pgfsetdash{}{0pt}%
\pgfsys@defobject{currentmarker}{\pgfqpoint{0.000000in}{-0.048611in}}{\pgfqpoint{0.000000in}{0.000000in}}{%
\pgfpathmoveto{\pgfqpoint{0.000000in}{0.000000in}}%
\pgfpathlineto{\pgfqpoint{0.000000in}{-0.048611in}}%
\pgfusepath{stroke,fill}%
}%
\begin{pgfscope}%
\pgfsys@transformshift{4.506597in}{0.844444in}%
\pgfsys@useobject{currentmarker}{}%
\end{pgfscope}%
\end{pgfscope}%
\begin{pgfscope}%
\definecolor{textcolor}{rgb}{0.000000,0.000000,0.000000}%
\pgfsetstrokecolor{textcolor}%
\pgfsetfillcolor{textcolor}%
\pgftext[x=4.506597in,y=0.747222in,,top]{\color{textcolor}\sffamily\fontsize{10.000000}{12.000000}\selectfont 0}%
\end{pgfscope}%
\begin{pgfscope}%
\pgfsetbuttcap%
\pgfsetroundjoin%
\definecolor{currentfill}{rgb}{0.000000,0.000000,0.000000}%
\pgfsetfillcolor{currentfill}%
\pgfsetlinewidth{0.803000pt}%
\definecolor{currentstroke}{rgb}{0.000000,0.000000,0.000000}%
\pgfsetstrokecolor{currentstroke}%
\pgfsetdash{}{0pt}%
\pgfsys@defobject{currentmarker}{\pgfqpoint{0.000000in}{-0.048611in}}{\pgfqpoint{0.000000in}{0.000000in}}{%
\pgfpathmoveto{\pgfqpoint{0.000000in}{0.000000in}}%
\pgfpathlineto{\pgfqpoint{0.000000in}{-0.048611in}}%
\pgfusepath{stroke,fill}%
}%
\begin{pgfscope}%
\pgfsys@transformshift{5.090278in}{0.844444in}%
\pgfsys@useobject{currentmarker}{}%
\end{pgfscope}%
\end{pgfscope}%
\begin{pgfscope}%
\definecolor{textcolor}{rgb}{0.000000,0.000000,0.000000}%
\pgfsetstrokecolor{textcolor}%
\pgfsetfillcolor{textcolor}%
\pgftext[x=5.090278in,y=0.747222in,,top]{\color{textcolor}\sffamily\fontsize{10.000000}{12.000000}\selectfont 1}%
\end{pgfscope}%
\begin{pgfscope}%
\pgfsetbuttcap%
\pgfsetroundjoin%
\definecolor{currentfill}{rgb}{0.000000,0.000000,0.000000}%
\pgfsetfillcolor{currentfill}%
\pgfsetlinewidth{0.803000pt}%
\definecolor{currentstroke}{rgb}{0.000000,0.000000,0.000000}%
\pgfsetstrokecolor{currentstroke}%
\pgfsetdash{}{0pt}%
\pgfsys@defobject{currentmarker}{\pgfqpoint{0.000000in}{-0.048611in}}{\pgfqpoint{0.000000in}{0.000000in}}{%
\pgfpathmoveto{\pgfqpoint{0.000000in}{0.000000in}}%
\pgfpathlineto{\pgfqpoint{0.000000in}{-0.048611in}}%
\pgfusepath{stroke,fill}%
}%
\begin{pgfscope}%
\pgfsys@transformshift{5.673958in}{0.844444in}%
\pgfsys@useobject{currentmarker}{}%
\end{pgfscope}%
\end{pgfscope}%
\begin{pgfscope}%
\definecolor{textcolor}{rgb}{0.000000,0.000000,0.000000}%
\pgfsetstrokecolor{textcolor}%
\pgfsetfillcolor{textcolor}%
\pgftext[x=5.673958in,y=0.747222in,,top]{\color{textcolor}\sffamily\fontsize{10.000000}{12.000000}\selectfont 2}%
\end{pgfscope}%
\begin{pgfscope}%
\pgfsetbuttcap%
\pgfsetroundjoin%
\definecolor{currentfill}{rgb}{0.000000,0.000000,0.000000}%
\pgfsetfillcolor{currentfill}%
\pgfsetlinewidth{0.803000pt}%
\definecolor{currentstroke}{rgb}{0.000000,0.000000,0.000000}%
\pgfsetstrokecolor{currentstroke}%
\pgfsetdash{}{0pt}%
\pgfsys@defobject{currentmarker}{\pgfqpoint{0.000000in}{-0.048611in}}{\pgfqpoint{0.000000in}{0.000000in}}{%
\pgfpathmoveto{\pgfqpoint{0.000000in}{0.000000in}}%
\pgfpathlineto{\pgfqpoint{0.000000in}{-0.048611in}}%
\pgfusepath{stroke,fill}%
}%
\begin{pgfscope}%
\pgfsys@transformshift{6.257639in}{0.844444in}%
\pgfsys@useobject{currentmarker}{}%
\end{pgfscope}%
\end{pgfscope}%
\begin{pgfscope}%
\definecolor{textcolor}{rgb}{0.000000,0.000000,0.000000}%
\pgfsetstrokecolor{textcolor}%
\pgfsetfillcolor{textcolor}%
\pgftext[x=6.257639in,y=0.747222in,,top]{\color{textcolor}\sffamily\fontsize{10.000000}{12.000000}\selectfont 3}%
\end{pgfscope}%
\begin{pgfscope}%
\pgfsetbuttcap%
\pgfsetroundjoin%
\definecolor{currentfill}{rgb}{0.000000,0.000000,0.000000}%
\pgfsetfillcolor{currentfill}%
\pgfsetlinewidth{0.803000pt}%
\definecolor{currentstroke}{rgb}{0.000000,0.000000,0.000000}%
\pgfsetstrokecolor{currentstroke}%
\pgfsetdash{}{0pt}%
\pgfsys@defobject{currentmarker}{\pgfqpoint{0.000000in}{-0.048611in}}{\pgfqpoint{0.000000in}{0.000000in}}{%
\pgfpathmoveto{\pgfqpoint{0.000000in}{0.000000in}}%
\pgfpathlineto{\pgfqpoint{0.000000in}{-0.048611in}}%
\pgfusepath{stroke,fill}%
}%
\begin{pgfscope}%
\pgfsys@transformshift{6.841319in}{0.844444in}%
\pgfsys@useobject{currentmarker}{}%
\end{pgfscope}%
\end{pgfscope}%
\begin{pgfscope}%
\definecolor{textcolor}{rgb}{0.000000,0.000000,0.000000}%
\pgfsetstrokecolor{textcolor}%
\pgfsetfillcolor{textcolor}%
\pgftext[x=6.841319in,y=0.747222in,,top]{\color{textcolor}\sffamily\fontsize{10.000000}{12.000000}\selectfont 4}%
\end{pgfscope}%
\begin{pgfscope}%
\definecolor{textcolor}{rgb}{0.000000,0.000000,0.000000}%
\pgfsetstrokecolor{textcolor}%
\pgfsetfillcolor{textcolor}%
\pgftext[x=5.965799in,y=0.557254in,,top]{\color{textcolor}\sffamily\fontsize{30.000000}{36.000000}\selectfont x axis}%
\end{pgfscope}%
\begin{pgfscope}%
\pgfsetbuttcap%
\pgfsetroundjoin%
\definecolor{currentfill}{rgb}{0.000000,0.000000,0.000000}%
\pgfsetfillcolor{currentfill}%
\pgfsetlinewidth{0.803000pt}%
\definecolor{currentstroke}{rgb}{0.000000,0.000000,0.000000}%
\pgfsetstrokecolor{currentstroke}%
\pgfsetdash{}{0pt}%
\pgfsys@defobject{currentmarker}{\pgfqpoint{-0.048611in}{0.000000in}}{\pgfqpoint{-0.000000in}{0.000000in}}{%
\pgfpathmoveto{\pgfqpoint{-0.000000in}{0.000000in}}%
\pgfpathlineto{\pgfqpoint{-0.048611in}{0.000000in}}%
\pgfusepath{stroke,fill}%
}%
\begin{pgfscope}%
\pgfsys@transformshift{4.506597in}{0.844444in}%
\pgfsys@useobject{currentmarker}{}%
\end{pgfscope}%
\end{pgfscope}%
\begin{pgfscope}%
\definecolor{textcolor}{rgb}{0.000000,0.000000,0.000000}%
\pgfsetstrokecolor{textcolor}%
\pgfsetfillcolor{textcolor}%
\pgftext[x=4.321010in, y=0.791683in, left, base]{\color{textcolor}\sffamily\fontsize{10.000000}{12.000000}\selectfont 0}%
\end{pgfscope}%
\begin{pgfscope}%
\pgfsetbuttcap%
\pgfsetroundjoin%
\definecolor{currentfill}{rgb}{0.000000,0.000000,0.000000}%
\pgfsetfillcolor{currentfill}%
\pgfsetlinewidth{0.803000pt}%
\definecolor{currentstroke}{rgb}{0.000000,0.000000,0.000000}%
\pgfsetstrokecolor{currentstroke}%
\pgfsetdash{}{0pt}%
\pgfsys@defobject{currentmarker}{\pgfqpoint{-0.048611in}{0.000000in}}{\pgfqpoint{-0.000000in}{0.000000in}}{%
\pgfpathmoveto{\pgfqpoint{-0.000000in}{0.000000in}}%
\pgfpathlineto{\pgfqpoint{-0.048611in}{0.000000in}}%
\pgfusepath{stroke,fill}%
}%
\begin{pgfscope}%
\pgfsys@transformshift{4.506597in}{1.239583in}%
\pgfsys@useobject{currentmarker}{}%
\end{pgfscope}%
\end{pgfscope}%
\begin{pgfscope}%
\definecolor{textcolor}{rgb}{0.000000,0.000000,0.000000}%
\pgfsetstrokecolor{textcolor}%
\pgfsetfillcolor{textcolor}%
\pgftext[x=4.321010in, y=1.186822in, left, base]{\color{textcolor}\sffamily\fontsize{10.000000}{12.000000}\selectfont 1}%
\end{pgfscope}%
\begin{pgfscope}%
\pgfsetbuttcap%
\pgfsetroundjoin%
\definecolor{currentfill}{rgb}{0.000000,0.000000,0.000000}%
\pgfsetfillcolor{currentfill}%
\pgfsetlinewidth{0.803000pt}%
\definecolor{currentstroke}{rgb}{0.000000,0.000000,0.000000}%
\pgfsetstrokecolor{currentstroke}%
\pgfsetdash{}{0pt}%
\pgfsys@defobject{currentmarker}{\pgfqpoint{-0.048611in}{0.000000in}}{\pgfqpoint{-0.000000in}{0.000000in}}{%
\pgfpathmoveto{\pgfqpoint{-0.000000in}{0.000000in}}%
\pgfpathlineto{\pgfqpoint{-0.048611in}{0.000000in}}%
\pgfusepath{stroke,fill}%
}%
\begin{pgfscope}%
\pgfsys@transformshift{4.506597in}{1.634722in}%
\pgfsys@useobject{currentmarker}{}%
\end{pgfscope}%
\end{pgfscope}%
\begin{pgfscope}%
\definecolor{textcolor}{rgb}{0.000000,0.000000,0.000000}%
\pgfsetstrokecolor{textcolor}%
\pgfsetfillcolor{textcolor}%
\pgftext[x=4.321010in, y=1.581961in, left, base]{\color{textcolor}\sffamily\fontsize{10.000000}{12.000000}\selectfont 2}%
\end{pgfscope}%
\begin{pgfscope}%
\pgfsetbuttcap%
\pgfsetroundjoin%
\definecolor{currentfill}{rgb}{0.000000,0.000000,0.000000}%
\pgfsetfillcolor{currentfill}%
\pgfsetlinewidth{0.803000pt}%
\definecolor{currentstroke}{rgb}{0.000000,0.000000,0.000000}%
\pgfsetstrokecolor{currentstroke}%
\pgfsetdash{}{0pt}%
\pgfsys@defobject{currentmarker}{\pgfqpoint{-0.048611in}{0.000000in}}{\pgfqpoint{-0.000000in}{0.000000in}}{%
\pgfpathmoveto{\pgfqpoint{-0.000000in}{0.000000in}}%
\pgfpathlineto{\pgfqpoint{-0.048611in}{0.000000in}}%
\pgfusepath{stroke,fill}%
}%
\begin{pgfscope}%
\pgfsys@transformshift{4.506597in}{2.029861in}%
\pgfsys@useobject{currentmarker}{}%
\end{pgfscope}%
\end{pgfscope}%
\begin{pgfscope}%
\definecolor{textcolor}{rgb}{0.000000,0.000000,0.000000}%
\pgfsetstrokecolor{textcolor}%
\pgfsetfillcolor{textcolor}%
\pgftext[x=4.321010in, y=1.977100in, left, base]{\color{textcolor}\sffamily\fontsize{10.000000}{12.000000}\selectfont 3}%
\end{pgfscope}%
\begin{pgfscope}%
\definecolor{textcolor}{rgb}{0.000000,0.000000,0.000000}%
\pgfsetstrokecolor{textcolor}%
\pgfsetfillcolor{textcolor}%
\pgftext[x=4.265454in,y=1.634722in,,bottom,rotate=90.000000]{\color{textcolor}\sffamily\fontsize{30.000000}{36.000000}\selectfont y axis}%
\end{pgfscope}%
\begin{pgfscope}%
\pgfsetrectcap%
\pgfsetmiterjoin%
\pgfsetlinewidth{0.803000pt}%
\definecolor{currentstroke}{rgb}{0.000000,0.000000,0.000000}%
\pgfsetstrokecolor{currentstroke}%
\pgfsetdash{}{0pt}%
\pgfpathmoveto{\pgfqpoint{4.506597in}{0.844444in}}%
\pgfpathlineto{\pgfqpoint{4.506597in}{2.425000in}}%
\pgfusepath{stroke}%
\end{pgfscope}%
\begin{pgfscope}%
\pgfsetrectcap%
\pgfsetmiterjoin%
\pgfsetlinewidth{0.803000pt}%
\definecolor{currentstroke}{rgb}{0.000000,0.000000,0.000000}%
\pgfsetstrokecolor{currentstroke}%
\pgfsetdash{}{0pt}%
\pgfpathmoveto{\pgfqpoint{7.425000in}{0.844444in}}%
\pgfpathlineto{\pgfqpoint{7.425000in}{2.425000in}}%
\pgfusepath{stroke}%
\end{pgfscope}%
\begin{pgfscope}%
\pgfsetrectcap%
\pgfsetmiterjoin%
\pgfsetlinewidth{0.803000pt}%
\definecolor{currentstroke}{rgb}{0.000000,0.000000,0.000000}%
\pgfsetstrokecolor{currentstroke}%
\pgfsetdash{}{0pt}%
\pgfpathmoveto{\pgfqpoint{4.506597in}{0.844444in}}%
\pgfpathlineto{\pgfqpoint{7.425000in}{0.844444in}}%
\pgfusepath{stroke}%
\end{pgfscope}%
\begin{pgfscope}%
\pgfsetrectcap%
\pgfsetmiterjoin%
\pgfsetlinewidth{0.803000pt}%
\definecolor{currentstroke}{rgb}{0.000000,0.000000,0.000000}%
\pgfsetstrokecolor{currentstroke}%
\pgfsetdash{}{0pt}%
\pgfpathmoveto{\pgfqpoint{4.506597in}{2.425000in}}%
\pgfpathlineto{\pgfqpoint{7.425000in}{2.425000in}}%
\pgfusepath{stroke}%
\end{pgfscope}%
\begin{pgfscope}%
\pgfsetbuttcap%
\pgfsetmiterjoin%
\definecolor{currentfill}{rgb}{1.000000,1.000000,1.000000}%
\pgfsetfillcolor{currentfill}%
\pgfsetlinewidth{0.000000pt}%
\definecolor{currentstroke}{rgb}{0.000000,0.000000,0.000000}%
\pgfsetstrokecolor{currentstroke}%
\pgfsetstrokeopacity{0.000000}%
\pgfsetdash{}{0pt}%
\pgfpathmoveto{\pgfqpoint{8.219097in}{0.844444in}}%
\pgfpathlineto{\pgfqpoint{11.137500in}{0.844444in}}%
\pgfpathlineto{\pgfqpoint{11.137500in}{2.425000in}}%
\pgfpathlineto{\pgfqpoint{8.219097in}{2.425000in}}%
\pgfpathlineto{\pgfqpoint{8.219097in}{0.844444in}}%
\pgfpathclose%
\pgfusepath{fill}%
\end{pgfscope}%
\begin{pgfscope}%
\pgfpathrectangle{\pgfqpoint{8.219097in}{0.844444in}}{\pgfqpoint{2.918403in}{1.580556in}}%
\pgfusepath{clip}%
\pgfsetbuttcap%
\pgfsetroundjoin%
\definecolor{currentfill}{rgb}{0.988235,0.590834,0.468020}%
\pgfsetfillcolor{currentfill}%
\pgfsetlinewidth{0.000000pt}%
\definecolor{currentstroke}{rgb}{0.000000,0.000000,0.000000}%
\pgfsetstrokecolor{currentstroke}%
\pgfsetdash{}{0pt}%
\pgfpathmoveto{\pgfqpoint{8.219097in}{0.844444in}}%
\pgfpathlineto{\pgfqpoint{8.219097in}{1.239583in}}%
\pgfpathlineto{\pgfqpoint{8.802778in}{1.239583in}}%
\pgfpathlineto{\pgfqpoint{8.802778in}{0.844444in}}%
\pgfpathlineto{\pgfqpoint{8.219097in}{0.844444in}}%
\pgfpathlineto{\pgfqpoint{8.219097in}{0.844444in}}%
\pgfpathclose%
\pgfusepath{fill}%
\end{pgfscope}%
\begin{pgfscope}%
\pgfpathrectangle{\pgfqpoint{8.219097in}{0.844444in}}{\pgfqpoint{2.918403in}{1.580556in}}%
\pgfusepath{clip}%
\pgfsetbuttcap%
\pgfsetroundjoin%
\definecolor{currentfill}{rgb}{0.994079,0.841446,0.774548}%
\pgfsetfillcolor{currentfill}%
\pgfsetlinewidth{0.000000pt}%
\definecolor{currentstroke}{rgb}{0.000000,0.000000,0.000000}%
\pgfsetstrokecolor{currentstroke}%
\pgfsetdash{}{0pt}%
\pgfpathmoveto{\pgfqpoint{8.802778in}{0.844444in}}%
\pgfpathlineto{\pgfqpoint{8.802778in}{1.239583in}}%
\pgfpathlineto{\pgfqpoint{9.386458in}{1.239583in}}%
\pgfpathlineto{\pgfqpoint{9.386458in}{0.844444in}}%
\pgfpathlineto{\pgfqpoint{8.802778in}{0.844444in}}%
\pgfpathlineto{\pgfqpoint{8.802778in}{0.844444in}}%
\pgfpathclose%
\pgfusepath{fill}%
\end{pgfscope}%
\begin{pgfscope}%
\pgfpathrectangle{\pgfqpoint{8.219097in}{0.844444in}}{\pgfqpoint{2.918403in}{1.580556in}}%
\pgfusepath{clip}%
\pgfsetbuttcap%
\pgfsetroundjoin%
\definecolor{currentfill}{rgb}{1.000000,0.960784,0.941176}%
\pgfsetfillcolor{currentfill}%
\pgfsetlinewidth{0.000000pt}%
\definecolor{currentstroke}{rgb}{0.000000,0.000000,0.000000}%
\pgfsetstrokecolor{currentstroke}%
\pgfsetdash{}{0pt}%
\pgfpathmoveto{\pgfqpoint{9.386458in}{0.844444in}}%
\pgfpathlineto{\pgfqpoint{9.386458in}{1.239583in}}%
\pgfpathlineto{\pgfqpoint{9.970139in}{1.239583in}}%
\pgfpathlineto{\pgfqpoint{9.970139in}{0.844444in}}%
\pgfpathlineto{\pgfqpoint{9.386458in}{0.844444in}}%
\pgfpathlineto{\pgfqpoint{9.386458in}{0.844444in}}%
\pgfpathclose%
\pgfusepath{fill}%
\end{pgfscope}%
\begin{pgfscope}%
\pgfpathrectangle{\pgfqpoint{8.219097in}{0.844444in}}{\pgfqpoint{2.918403in}{1.580556in}}%
\pgfusepath{clip}%
\pgfsetbuttcap%
\pgfsetroundjoin%
\definecolor{currentfill}{rgb}{0.984744,0.432910,0.307420}%
\pgfsetfillcolor{currentfill}%
\pgfsetlinewidth{0.000000pt}%
\definecolor{currentstroke}{rgb}{0.000000,0.000000,0.000000}%
\pgfsetstrokecolor{currentstroke}%
\pgfsetdash{}{0pt}%
\pgfpathmoveto{\pgfqpoint{9.970139in}{0.844444in}}%
\pgfpathlineto{\pgfqpoint{9.970139in}{1.239583in}}%
\pgfpathlineto{\pgfqpoint{10.553819in}{1.239583in}}%
\pgfpathlineto{\pgfqpoint{10.553819in}{0.844444in}}%
\pgfpathlineto{\pgfqpoint{9.970139in}{0.844444in}}%
\pgfpathlineto{\pgfqpoint{9.970139in}{0.844444in}}%
\pgfpathclose%
\pgfusepath{fill}%
\end{pgfscope}%
\begin{pgfscope}%
\pgfpathrectangle{\pgfqpoint{8.219097in}{0.844444in}}{\pgfqpoint{2.918403in}{1.580556in}}%
\pgfusepath{clip}%
\pgfsetbuttcap%
\pgfsetroundjoin%
\definecolor{currentfill}{rgb}{0.988235,0.661453,0.548973}%
\pgfsetfillcolor{currentfill}%
\pgfsetlinewidth{0.000000pt}%
\definecolor{currentstroke}{rgb}{0.000000,0.000000,0.000000}%
\pgfsetstrokecolor{currentstroke}%
\pgfsetdash{}{0pt}%
\pgfpathmoveto{\pgfqpoint{10.553819in}{0.844444in}}%
\pgfpathlineto{\pgfqpoint{10.553819in}{1.239583in}}%
\pgfpathlineto{\pgfqpoint{11.137500in}{1.239583in}}%
\pgfpathlineto{\pgfqpoint{11.137500in}{0.844444in}}%
\pgfpathlineto{\pgfqpoint{10.553819in}{0.844444in}}%
\pgfpathlineto{\pgfqpoint{10.553819in}{0.844444in}}%
\pgfpathclose%
\pgfusepath{fill}%
\end{pgfscope}%
\begin{pgfscope}%
\pgfpathrectangle{\pgfqpoint{8.219097in}{0.844444in}}{\pgfqpoint{2.918403in}{1.580556in}}%
\pgfusepath{clip}%
\pgfsetbuttcap%
\pgfsetroundjoin%
\definecolor{currentfill}{rgb}{0.986098,0.487043,0.361553}%
\pgfsetfillcolor{currentfill}%
\pgfsetlinewidth{0.000000pt}%
\definecolor{currentstroke}{rgb}{0.000000,0.000000,0.000000}%
\pgfsetstrokecolor{currentstroke}%
\pgfsetdash{}{0pt}%
\pgfpathmoveto{\pgfqpoint{8.219097in}{1.239583in}}%
\pgfpathlineto{\pgfqpoint{8.219097in}{1.634722in}}%
\pgfpathlineto{\pgfqpoint{8.802778in}{1.634722in}}%
\pgfpathlineto{\pgfqpoint{8.802778in}{1.239583in}}%
\pgfpathlineto{\pgfqpoint{8.219097in}{1.239583in}}%
\pgfpathlineto{\pgfqpoint{8.219097in}{1.239583in}}%
\pgfpathclose%
\pgfusepath{fill}%
\end{pgfscope}%
\begin{pgfscope}%
\pgfpathrectangle{\pgfqpoint{8.219097in}{0.844444in}}{\pgfqpoint{2.918403in}{1.580556in}}%
\pgfusepath{clip}%
\pgfsetbuttcap%
\pgfsetroundjoin%
\definecolor{currentfill}{rgb}{0.986959,0.521492,0.396002}%
\pgfsetfillcolor{currentfill}%
\pgfsetlinewidth{0.000000pt}%
\definecolor{currentstroke}{rgb}{0.000000,0.000000,0.000000}%
\pgfsetstrokecolor{currentstroke}%
\pgfsetdash{}{0pt}%
\pgfpathmoveto{\pgfqpoint{8.802778in}{1.239583in}}%
\pgfpathlineto{\pgfqpoint{8.802778in}{1.634722in}}%
\pgfpathlineto{\pgfqpoint{9.386458in}{1.634722in}}%
\pgfpathlineto{\pgfqpoint{9.386458in}{1.239583in}}%
\pgfpathlineto{\pgfqpoint{8.802778in}{1.239583in}}%
\pgfpathlineto{\pgfqpoint{8.802778in}{1.239583in}}%
\pgfpathclose%
\pgfusepath{fill}%
\end{pgfscope}%
\begin{pgfscope}%
\pgfpathrectangle{\pgfqpoint{8.219097in}{0.844444in}}{\pgfqpoint{2.918403in}{1.580556in}}%
\pgfusepath{clip}%
\pgfsetbuttcap%
\pgfsetroundjoin%
\definecolor{currentfill}{rgb}{0.988066,0.565782,0.440292}%
\pgfsetfillcolor{currentfill}%
\pgfsetlinewidth{0.000000pt}%
\definecolor{currentstroke}{rgb}{0.000000,0.000000,0.000000}%
\pgfsetstrokecolor{currentstroke}%
\pgfsetdash{}{0pt}%
\pgfpathmoveto{\pgfqpoint{9.386458in}{1.239583in}}%
\pgfpathlineto{\pgfqpoint{9.386458in}{1.634722in}}%
\pgfpathlineto{\pgfqpoint{9.970139in}{1.634722in}}%
\pgfpathlineto{\pgfqpoint{9.970139in}{1.239583in}}%
\pgfpathlineto{\pgfqpoint{9.386458in}{1.239583in}}%
\pgfpathlineto{\pgfqpoint{9.386458in}{1.239583in}}%
\pgfpathclose%
\pgfusepath{fill}%
\end{pgfscope}%
\begin{pgfscope}%
\pgfpathrectangle{\pgfqpoint{8.219097in}{0.844444in}}{\pgfqpoint{2.918403in}{1.580556in}}%
\pgfusepath{clip}%
\pgfsetbuttcap%
\pgfsetroundjoin%
\definecolor{currentfill}{rgb}{0.988235,0.676586,0.566321}%
\pgfsetfillcolor{currentfill}%
\pgfsetlinewidth{0.000000pt}%
\definecolor{currentstroke}{rgb}{0.000000,0.000000,0.000000}%
\pgfsetstrokecolor{currentstroke}%
\pgfsetdash{}{0pt}%
\pgfpathmoveto{\pgfqpoint{9.970139in}{1.239583in}}%
\pgfpathlineto{\pgfqpoint{9.970139in}{1.634722in}}%
\pgfpathlineto{\pgfqpoint{10.553819in}{1.634722in}}%
\pgfpathlineto{\pgfqpoint{10.553819in}{1.239583in}}%
\pgfpathlineto{\pgfqpoint{9.970139in}{1.239583in}}%
\pgfpathlineto{\pgfqpoint{9.970139in}{1.239583in}}%
\pgfpathclose%
\pgfusepath{fill}%
\end{pgfscope}%
\begin{pgfscope}%
\pgfpathrectangle{\pgfqpoint{8.219097in}{0.844444in}}{\pgfqpoint{2.918403in}{1.580556in}}%
\pgfusepath{clip}%
\pgfsetbuttcap%
\pgfsetroundjoin%
\definecolor{currentfill}{rgb}{0.988235,0.681630,0.572103}%
\pgfsetfillcolor{currentfill}%
\pgfsetlinewidth{0.000000pt}%
\definecolor{currentstroke}{rgb}{0.000000,0.000000,0.000000}%
\pgfsetstrokecolor{currentstroke}%
\pgfsetdash{}{0pt}%
\pgfpathmoveto{\pgfqpoint{10.553819in}{1.239583in}}%
\pgfpathlineto{\pgfqpoint{10.553819in}{1.634722in}}%
\pgfpathlineto{\pgfqpoint{11.137500in}{1.634722in}}%
\pgfpathlineto{\pgfqpoint{11.137500in}{1.239583in}}%
\pgfpathlineto{\pgfqpoint{10.553819in}{1.239583in}}%
\pgfpathlineto{\pgfqpoint{10.553819in}{1.239583in}}%
\pgfpathclose%
\pgfusepath{fill}%
\end{pgfscope}%
\begin{pgfscope}%
\pgfpathrectangle{\pgfqpoint{8.219097in}{0.844444in}}{\pgfqpoint{2.918403in}{1.580556in}}%
\pgfusepath{clip}%
\pgfsetbuttcap%
\pgfsetroundjoin%
\definecolor{currentfill}{rgb}{0.984498,0.423068,0.297578}%
\pgfsetfillcolor{currentfill}%
\pgfsetlinewidth{0.000000pt}%
\definecolor{currentstroke}{rgb}{0.000000,0.000000,0.000000}%
\pgfsetstrokecolor{currentstroke}%
\pgfsetdash{}{0pt}%
\pgfpathmoveto{\pgfqpoint{8.219097in}{1.634722in}}%
\pgfpathlineto{\pgfqpoint{8.219097in}{2.029861in}}%
\pgfpathlineto{\pgfqpoint{8.802778in}{2.029861in}}%
\pgfpathlineto{\pgfqpoint{8.802778in}{1.634722in}}%
\pgfpathlineto{\pgfqpoint{8.219097in}{1.634722in}}%
\pgfpathlineto{\pgfqpoint{8.219097in}{1.634722in}}%
\pgfpathclose%
\pgfusepath{fill}%
\end{pgfscope}%
\begin{pgfscope}%
\pgfpathrectangle{\pgfqpoint{8.219097in}{0.844444in}}{\pgfqpoint{2.918403in}{1.580556in}}%
\pgfusepath{clip}%
\pgfsetbuttcap%
\pgfsetroundjoin%
\definecolor{currentfill}{rgb}{0.890196,0.185621,0.152941}%
\pgfsetfillcolor{currentfill}%
\pgfsetlinewidth{0.000000pt}%
\definecolor{currentstroke}{rgb}{0.000000,0.000000,0.000000}%
\pgfsetstrokecolor{currentstroke}%
\pgfsetdash{}{0pt}%
\pgfpathmoveto{\pgfqpoint{8.802778in}{1.634722in}}%
\pgfpathlineto{\pgfqpoint{8.802778in}{2.029861in}}%
\pgfpathlineto{\pgfqpoint{9.386458in}{2.029861in}}%
\pgfpathlineto{\pgfqpoint{9.386458in}{1.634722in}}%
\pgfpathlineto{\pgfqpoint{8.802778in}{1.634722in}}%
\pgfpathlineto{\pgfqpoint{8.802778in}{1.634722in}}%
\pgfpathclose%
\pgfusepath{fill}%
\end{pgfscope}%
\begin{pgfscope}%
\pgfpathrectangle{\pgfqpoint{8.219097in}{0.844444in}}{\pgfqpoint{2.918403in}{1.580556in}}%
\pgfusepath{clip}%
\pgfsetbuttcap%
\pgfsetroundjoin%
\definecolor{currentfill}{rgb}{0.970288,0.360754,0.255133}%
\pgfsetfillcolor{currentfill}%
\pgfsetlinewidth{0.000000pt}%
\definecolor{currentstroke}{rgb}{0.000000,0.000000,0.000000}%
\pgfsetstrokecolor{currentstroke}%
\pgfsetdash{}{0pt}%
\pgfpathmoveto{\pgfqpoint{9.386458in}{1.634722in}}%
\pgfpathlineto{\pgfqpoint{9.386458in}{2.029861in}}%
\pgfpathlineto{\pgfqpoint{9.970139in}{2.029861in}}%
\pgfpathlineto{\pgfqpoint{9.970139in}{1.634722in}}%
\pgfpathlineto{\pgfqpoint{9.386458in}{1.634722in}}%
\pgfpathlineto{\pgfqpoint{9.386458in}{1.634722in}}%
\pgfpathclose%
\pgfusepath{fill}%
\end{pgfscope}%
\begin{pgfscope}%
\pgfpathrectangle{\pgfqpoint{8.219097in}{0.844444in}}{\pgfqpoint{2.918403in}{1.580556in}}%
\pgfusepath{clip}%
\pgfsetbuttcap%
\pgfsetroundjoin%
\definecolor{currentfill}{rgb}{0.403922,0.000000,0.050980}%
\pgfsetfillcolor{currentfill}%
\pgfsetlinewidth{0.000000pt}%
\definecolor{currentstroke}{rgb}{0.000000,0.000000,0.000000}%
\pgfsetstrokecolor{currentstroke}%
\pgfsetdash{}{0pt}%
\pgfpathmoveto{\pgfqpoint{9.970139in}{1.634722in}}%
\pgfpathlineto{\pgfqpoint{9.970139in}{2.029861in}}%
\pgfpathlineto{\pgfqpoint{10.553819in}{2.029861in}}%
\pgfpathlineto{\pgfqpoint{10.553819in}{1.634722in}}%
\pgfpathlineto{\pgfqpoint{9.970139in}{1.634722in}}%
\pgfpathlineto{\pgfqpoint{9.970139in}{1.634722in}}%
\pgfpathclose%
\pgfusepath{fill}%
\end{pgfscope}%
\begin{pgfscope}%
\pgfpathrectangle{\pgfqpoint{8.219097in}{0.844444in}}{\pgfqpoint{2.918403in}{1.580556in}}%
\pgfusepath{clip}%
\pgfsetbuttcap%
\pgfsetroundjoin%
\definecolor{currentfill}{rgb}{0.988235,0.580746,0.456455}%
\pgfsetfillcolor{currentfill}%
\pgfsetlinewidth{0.000000pt}%
\definecolor{currentstroke}{rgb}{0.000000,0.000000,0.000000}%
\pgfsetstrokecolor{currentstroke}%
\pgfsetdash{}{0pt}%
\pgfpathmoveto{\pgfqpoint{10.553819in}{1.634722in}}%
\pgfpathlineto{\pgfqpoint{10.553819in}{2.029861in}}%
\pgfpathlineto{\pgfqpoint{11.137500in}{2.029861in}}%
\pgfpathlineto{\pgfqpoint{11.137500in}{1.634722in}}%
\pgfpathlineto{\pgfqpoint{10.553819in}{1.634722in}}%
\pgfpathlineto{\pgfqpoint{10.553819in}{1.634722in}}%
\pgfpathclose%
\pgfusepath{fill}%
\end{pgfscope}%
\begin{pgfscope}%
\pgfpathrectangle{\pgfqpoint{8.219097in}{0.844444in}}{\pgfqpoint{2.918403in}{1.580556in}}%
\pgfusepath{clip}%
\pgfsetbuttcap%
\pgfsetroundjoin%
\definecolor{currentfill}{rgb}{0.988235,0.656409,0.543191}%
\pgfsetfillcolor{currentfill}%
\pgfsetlinewidth{0.000000pt}%
\definecolor{currentstroke}{rgb}{0.000000,0.000000,0.000000}%
\pgfsetstrokecolor{currentstroke}%
\pgfsetdash{}{0pt}%
\pgfpathmoveto{\pgfqpoint{8.219097in}{2.029861in}}%
\pgfpathlineto{\pgfqpoint{8.219097in}{2.425000in}}%
\pgfpathlineto{\pgfqpoint{8.802778in}{2.425000in}}%
\pgfpathlineto{\pgfqpoint{8.802778in}{2.029861in}}%
\pgfpathlineto{\pgfqpoint{8.219097in}{2.029861in}}%
\pgfpathlineto{\pgfqpoint{8.219097in}{2.029861in}}%
\pgfpathclose%
\pgfusepath{fill}%
\end{pgfscope}%
\begin{pgfscope}%
\pgfpathrectangle{\pgfqpoint{8.219097in}{0.844444in}}{\pgfqpoint{2.918403in}{1.580556in}}%
\pgfusepath{clip}%
\pgfsetbuttcap%
\pgfsetroundjoin%
\definecolor{currentfill}{rgb}{0.958478,0.314494,0.225606}%
\pgfsetfillcolor{currentfill}%
\pgfsetlinewidth{0.000000pt}%
\definecolor{currentstroke}{rgb}{0.000000,0.000000,0.000000}%
\pgfsetstrokecolor{currentstroke}%
\pgfsetdash{}{0pt}%
\pgfpathmoveto{\pgfqpoint{8.802778in}{2.029861in}}%
\pgfpathlineto{\pgfqpoint{8.802778in}{2.425000in}}%
\pgfpathlineto{\pgfqpoint{9.386458in}{2.425000in}}%
\pgfpathlineto{\pgfqpoint{9.386458in}{2.029861in}}%
\pgfpathlineto{\pgfqpoint{8.802778in}{2.029861in}}%
\pgfpathlineto{\pgfqpoint{8.802778in}{2.029861in}}%
\pgfpathclose%
\pgfusepath{fill}%
\end{pgfscope}%
\begin{pgfscope}%
\pgfpathrectangle{\pgfqpoint{8.219097in}{0.844444in}}{\pgfqpoint{2.918403in}{1.580556in}}%
\pgfusepath{clip}%
\pgfsetbuttcap%
\pgfsetroundjoin%
\definecolor{currentfill}{rgb}{0.990634,0.777716,0.690150}%
\pgfsetfillcolor{currentfill}%
\pgfsetlinewidth{0.000000pt}%
\definecolor{currentstroke}{rgb}{0.000000,0.000000,0.000000}%
\pgfsetstrokecolor{currentstroke}%
\pgfsetdash{}{0pt}%
\pgfpathmoveto{\pgfqpoint{9.386458in}{2.029861in}}%
\pgfpathlineto{\pgfqpoint{9.386458in}{2.425000in}}%
\pgfpathlineto{\pgfqpoint{9.970139in}{2.425000in}}%
\pgfpathlineto{\pgfqpoint{9.970139in}{2.029861in}}%
\pgfpathlineto{\pgfqpoint{9.386458in}{2.029861in}}%
\pgfpathlineto{\pgfqpoint{9.386458in}{2.029861in}}%
\pgfpathclose%
\pgfusepath{fill}%
\end{pgfscope}%
\begin{pgfscope}%
\pgfpathrectangle{\pgfqpoint{8.219097in}{0.844444in}}{\pgfqpoint{2.918403in}{1.580556in}}%
\pgfusepath{clip}%
\pgfsetbuttcap%
\pgfsetroundjoin%
\definecolor{currentfill}{rgb}{0.647643,0.058962,0.082476}%
\pgfsetfillcolor{currentfill}%
\pgfsetlinewidth{0.000000pt}%
\definecolor{currentstroke}{rgb}{0.000000,0.000000,0.000000}%
\pgfsetstrokecolor{currentstroke}%
\pgfsetdash{}{0pt}%
\pgfpathmoveto{\pgfqpoint{9.970139in}{2.029861in}}%
\pgfpathlineto{\pgfqpoint{9.970139in}{2.425000in}}%
\pgfpathlineto{\pgfqpoint{10.553819in}{2.425000in}}%
\pgfpathlineto{\pgfqpoint{10.553819in}{2.029861in}}%
\pgfpathlineto{\pgfqpoint{9.970139in}{2.029861in}}%
\pgfpathlineto{\pgfqpoint{9.970139in}{2.029861in}}%
\pgfpathclose%
\pgfusepath{fill}%
\end{pgfscope}%
\begin{pgfscope}%
\pgfpathrectangle{\pgfqpoint{8.219097in}{0.844444in}}{\pgfqpoint{2.918403in}{1.580556in}}%
\pgfusepath{clip}%
\pgfsetbuttcap%
\pgfsetroundjoin%
\definecolor{currentfill}{rgb}{0.986098,0.487043,0.361553}%
\pgfsetfillcolor{currentfill}%
\pgfsetlinewidth{0.000000pt}%
\definecolor{currentstroke}{rgb}{0.000000,0.000000,0.000000}%
\pgfsetstrokecolor{currentstroke}%
\pgfsetdash{}{0pt}%
\pgfpathmoveto{\pgfqpoint{10.553819in}{2.029861in}}%
\pgfpathlineto{\pgfqpoint{10.553819in}{2.425000in}}%
\pgfpathlineto{\pgfqpoint{11.137500in}{2.425000in}}%
\pgfpathlineto{\pgfqpoint{11.137500in}{2.029861in}}%
\pgfpathlineto{\pgfqpoint{10.553819in}{2.029861in}}%
\pgfpathlineto{\pgfqpoint{10.553819in}{2.029861in}}%
\pgfpathclose%
\pgfusepath{fill}%
\end{pgfscope}%
\begin{pgfscope}%
\pgfsetbuttcap%
\pgfsetroundjoin%
\definecolor{currentfill}{rgb}{0.000000,0.000000,0.000000}%
\pgfsetfillcolor{currentfill}%
\pgfsetlinewidth{0.803000pt}%
\definecolor{currentstroke}{rgb}{0.000000,0.000000,0.000000}%
\pgfsetstrokecolor{currentstroke}%
\pgfsetdash{}{0pt}%
\pgfsys@defobject{currentmarker}{\pgfqpoint{0.000000in}{-0.048611in}}{\pgfqpoint{0.000000in}{0.000000in}}{%
\pgfpathmoveto{\pgfqpoint{0.000000in}{0.000000in}}%
\pgfpathlineto{\pgfqpoint{0.000000in}{-0.048611in}}%
\pgfusepath{stroke,fill}%
}%
\begin{pgfscope}%
\pgfsys@transformshift{8.219097in}{0.844444in}%
\pgfsys@useobject{currentmarker}{}%
\end{pgfscope}%
\end{pgfscope}%
\begin{pgfscope}%
\definecolor{textcolor}{rgb}{0.000000,0.000000,0.000000}%
\pgfsetstrokecolor{textcolor}%
\pgfsetfillcolor{textcolor}%
\pgftext[x=8.219097in,y=0.747222in,,top]{\color{textcolor}\sffamily\fontsize{10.000000}{12.000000}\selectfont 0}%
\end{pgfscope}%
\begin{pgfscope}%
\pgfsetbuttcap%
\pgfsetroundjoin%
\definecolor{currentfill}{rgb}{0.000000,0.000000,0.000000}%
\pgfsetfillcolor{currentfill}%
\pgfsetlinewidth{0.803000pt}%
\definecolor{currentstroke}{rgb}{0.000000,0.000000,0.000000}%
\pgfsetstrokecolor{currentstroke}%
\pgfsetdash{}{0pt}%
\pgfsys@defobject{currentmarker}{\pgfqpoint{0.000000in}{-0.048611in}}{\pgfqpoint{0.000000in}{0.000000in}}{%
\pgfpathmoveto{\pgfqpoint{0.000000in}{0.000000in}}%
\pgfpathlineto{\pgfqpoint{0.000000in}{-0.048611in}}%
\pgfusepath{stroke,fill}%
}%
\begin{pgfscope}%
\pgfsys@transformshift{8.802778in}{0.844444in}%
\pgfsys@useobject{currentmarker}{}%
\end{pgfscope}%
\end{pgfscope}%
\begin{pgfscope}%
\definecolor{textcolor}{rgb}{0.000000,0.000000,0.000000}%
\pgfsetstrokecolor{textcolor}%
\pgfsetfillcolor{textcolor}%
\pgftext[x=8.802778in,y=0.747222in,,top]{\color{textcolor}\sffamily\fontsize{10.000000}{12.000000}\selectfont 1}%
\end{pgfscope}%
\begin{pgfscope}%
\pgfsetbuttcap%
\pgfsetroundjoin%
\definecolor{currentfill}{rgb}{0.000000,0.000000,0.000000}%
\pgfsetfillcolor{currentfill}%
\pgfsetlinewidth{0.803000pt}%
\definecolor{currentstroke}{rgb}{0.000000,0.000000,0.000000}%
\pgfsetstrokecolor{currentstroke}%
\pgfsetdash{}{0pt}%
\pgfsys@defobject{currentmarker}{\pgfqpoint{0.000000in}{-0.048611in}}{\pgfqpoint{0.000000in}{0.000000in}}{%
\pgfpathmoveto{\pgfqpoint{0.000000in}{0.000000in}}%
\pgfpathlineto{\pgfqpoint{0.000000in}{-0.048611in}}%
\pgfusepath{stroke,fill}%
}%
\begin{pgfscope}%
\pgfsys@transformshift{9.386458in}{0.844444in}%
\pgfsys@useobject{currentmarker}{}%
\end{pgfscope}%
\end{pgfscope}%
\begin{pgfscope}%
\definecolor{textcolor}{rgb}{0.000000,0.000000,0.000000}%
\pgfsetstrokecolor{textcolor}%
\pgfsetfillcolor{textcolor}%
\pgftext[x=9.386458in,y=0.747222in,,top]{\color{textcolor}\sffamily\fontsize{10.000000}{12.000000}\selectfont 2}%
\end{pgfscope}%
\begin{pgfscope}%
\pgfsetbuttcap%
\pgfsetroundjoin%
\definecolor{currentfill}{rgb}{0.000000,0.000000,0.000000}%
\pgfsetfillcolor{currentfill}%
\pgfsetlinewidth{0.803000pt}%
\definecolor{currentstroke}{rgb}{0.000000,0.000000,0.000000}%
\pgfsetstrokecolor{currentstroke}%
\pgfsetdash{}{0pt}%
\pgfsys@defobject{currentmarker}{\pgfqpoint{0.000000in}{-0.048611in}}{\pgfqpoint{0.000000in}{0.000000in}}{%
\pgfpathmoveto{\pgfqpoint{0.000000in}{0.000000in}}%
\pgfpathlineto{\pgfqpoint{0.000000in}{-0.048611in}}%
\pgfusepath{stroke,fill}%
}%
\begin{pgfscope}%
\pgfsys@transformshift{9.970139in}{0.844444in}%
\pgfsys@useobject{currentmarker}{}%
\end{pgfscope}%
\end{pgfscope}%
\begin{pgfscope}%
\definecolor{textcolor}{rgb}{0.000000,0.000000,0.000000}%
\pgfsetstrokecolor{textcolor}%
\pgfsetfillcolor{textcolor}%
\pgftext[x=9.970139in,y=0.747222in,,top]{\color{textcolor}\sffamily\fontsize{10.000000}{12.000000}\selectfont 3}%
\end{pgfscope}%
\begin{pgfscope}%
\pgfsetbuttcap%
\pgfsetroundjoin%
\definecolor{currentfill}{rgb}{0.000000,0.000000,0.000000}%
\pgfsetfillcolor{currentfill}%
\pgfsetlinewidth{0.803000pt}%
\definecolor{currentstroke}{rgb}{0.000000,0.000000,0.000000}%
\pgfsetstrokecolor{currentstroke}%
\pgfsetdash{}{0pt}%
\pgfsys@defobject{currentmarker}{\pgfqpoint{0.000000in}{-0.048611in}}{\pgfqpoint{0.000000in}{0.000000in}}{%
\pgfpathmoveto{\pgfqpoint{0.000000in}{0.000000in}}%
\pgfpathlineto{\pgfqpoint{0.000000in}{-0.048611in}}%
\pgfusepath{stroke,fill}%
}%
\begin{pgfscope}%
\pgfsys@transformshift{10.553819in}{0.844444in}%
\pgfsys@useobject{currentmarker}{}%
\end{pgfscope}%
\end{pgfscope}%
\begin{pgfscope}%
\definecolor{textcolor}{rgb}{0.000000,0.000000,0.000000}%
\pgfsetstrokecolor{textcolor}%
\pgfsetfillcolor{textcolor}%
\pgftext[x=10.553819in,y=0.747222in,,top]{\color{textcolor}\sffamily\fontsize{10.000000}{12.000000}\selectfont 4}%
\end{pgfscope}%
\begin{pgfscope}%
\definecolor{textcolor}{rgb}{0.000000,0.000000,0.000000}%
\pgfsetstrokecolor{textcolor}%
\pgfsetfillcolor{textcolor}%
\pgftext[x=9.678299in,y=0.557254in,,top]{\color{textcolor}\sffamily\fontsize{30.000000}{36.000000}\selectfont x axis}%
\end{pgfscope}%
\begin{pgfscope}%
\pgfsetbuttcap%
\pgfsetroundjoin%
\definecolor{currentfill}{rgb}{0.000000,0.000000,0.000000}%
\pgfsetfillcolor{currentfill}%
\pgfsetlinewidth{0.803000pt}%
\definecolor{currentstroke}{rgb}{0.000000,0.000000,0.000000}%
\pgfsetstrokecolor{currentstroke}%
\pgfsetdash{}{0pt}%
\pgfsys@defobject{currentmarker}{\pgfqpoint{-0.048611in}{0.000000in}}{\pgfqpoint{-0.000000in}{0.000000in}}{%
\pgfpathmoveto{\pgfqpoint{-0.000000in}{0.000000in}}%
\pgfpathlineto{\pgfqpoint{-0.048611in}{0.000000in}}%
\pgfusepath{stroke,fill}%
}%
\begin{pgfscope}%
\pgfsys@transformshift{8.219097in}{0.844444in}%
\pgfsys@useobject{currentmarker}{}%
\end{pgfscope}%
\end{pgfscope}%
\begin{pgfscope}%
\definecolor{textcolor}{rgb}{0.000000,0.000000,0.000000}%
\pgfsetstrokecolor{textcolor}%
\pgfsetfillcolor{textcolor}%
\pgftext[x=8.033510in, y=0.791683in, left, base]{\color{textcolor}\sffamily\fontsize{10.000000}{12.000000}\selectfont 0}%
\end{pgfscope}%
\begin{pgfscope}%
\pgfsetbuttcap%
\pgfsetroundjoin%
\definecolor{currentfill}{rgb}{0.000000,0.000000,0.000000}%
\pgfsetfillcolor{currentfill}%
\pgfsetlinewidth{0.803000pt}%
\definecolor{currentstroke}{rgb}{0.000000,0.000000,0.000000}%
\pgfsetstrokecolor{currentstroke}%
\pgfsetdash{}{0pt}%
\pgfsys@defobject{currentmarker}{\pgfqpoint{-0.048611in}{0.000000in}}{\pgfqpoint{-0.000000in}{0.000000in}}{%
\pgfpathmoveto{\pgfqpoint{-0.000000in}{0.000000in}}%
\pgfpathlineto{\pgfqpoint{-0.048611in}{0.000000in}}%
\pgfusepath{stroke,fill}%
}%
\begin{pgfscope}%
\pgfsys@transformshift{8.219097in}{1.239583in}%
\pgfsys@useobject{currentmarker}{}%
\end{pgfscope}%
\end{pgfscope}%
\begin{pgfscope}%
\definecolor{textcolor}{rgb}{0.000000,0.000000,0.000000}%
\pgfsetstrokecolor{textcolor}%
\pgfsetfillcolor{textcolor}%
\pgftext[x=8.033510in, y=1.186822in, left, base]{\color{textcolor}\sffamily\fontsize{10.000000}{12.000000}\selectfont 1}%
\end{pgfscope}%
\begin{pgfscope}%
\pgfsetbuttcap%
\pgfsetroundjoin%
\definecolor{currentfill}{rgb}{0.000000,0.000000,0.000000}%
\pgfsetfillcolor{currentfill}%
\pgfsetlinewidth{0.803000pt}%
\definecolor{currentstroke}{rgb}{0.000000,0.000000,0.000000}%
\pgfsetstrokecolor{currentstroke}%
\pgfsetdash{}{0pt}%
\pgfsys@defobject{currentmarker}{\pgfqpoint{-0.048611in}{0.000000in}}{\pgfqpoint{-0.000000in}{0.000000in}}{%
\pgfpathmoveto{\pgfqpoint{-0.000000in}{0.000000in}}%
\pgfpathlineto{\pgfqpoint{-0.048611in}{0.000000in}}%
\pgfusepath{stroke,fill}%
}%
\begin{pgfscope}%
\pgfsys@transformshift{8.219097in}{1.634722in}%
\pgfsys@useobject{currentmarker}{}%
\end{pgfscope}%
\end{pgfscope}%
\begin{pgfscope}%
\definecolor{textcolor}{rgb}{0.000000,0.000000,0.000000}%
\pgfsetstrokecolor{textcolor}%
\pgfsetfillcolor{textcolor}%
\pgftext[x=8.033510in, y=1.581961in, left, base]{\color{textcolor}\sffamily\fontsize{10.000000}{12.000000}\selectfont 2}%
\end{pgfscope}%
\begin{pgfscope}%
\pgfsetbuttcap%
\pgfsetroundjoin%
\definecolor{currentfill}{rgb}{0.000000,0.000000,0.000000}%
\pgfsetfillcolor{currentfill}%
\pgfsetlinewidth{0.803000pt}%
\definecolor{currentstroke}{rgb}{0.000000,0.000000,0.000000}%
\pgfsetstrokecolor{currentstroke}%
\pgfsetdash{}{0pt}%
\pgfsys@defobject{currentmarker}{\pgfqpoint{-0.048611in}{0.000000in}}{\pgfqpoint{-0.000000in}{0.000000in}}{%
\pgfpathmoveto{\pgfqpoint{-0.000000in}{0.000000in}}%
\pgfpathlineto{\pgfqpoint{-0.048611in}{0.000000in}}%
\pgfusepath{stroke,fill}%
}%
\begin{pgfscope}%
\pgfsys@transformshift{8.219097in}{2.029861in}%
\pgfsys@useobject{currentmarker}{}%
\end{pgfscope}%
\end{pgfscope}%
\begin{pgfscope}%
\definecolor{textcolor}{rgb}{0.000000,0.000000,0.000000}%
\pgfsetstrokecolor{textcolor}%
\pgfsetfillcolor{textcolor}%
\pgftext[x=8.033510in, y=1.977100in, left, base]{\color{textcolor}\sffamily\fontsize{10.000000}{12.000000}\selectfont 3}%
\end{pgfscope}%
\begin{pgfscope}%
\definecolor{textcolor}{rgb}{0.000000,0.000000,0.000000}%
\pgfsetstrokecolor{textcolor}%
\pgfsetfillcolor{textcolor}%
\pgftext[x=7.977954in,y=1.634722in,,bottom,rotate=90.000000]{\color{textcolor}\sffamily\fontsize{30.000000}{36.000000}\selectfont y axis}%
\end{pgfscope}%
\begin{pgfscope}%
\pgfsetrectcap%
\pgfsetmiterjoin%
\pgfsetlinewidth{0.803000pt}%
\definecolor{currentstroke}{rgb}{0.000000,0.000000,0.000000}%
\pgfsetstrokecolor{currentstroke}%
\pgfsetdash{}{0pt}%
\pgfpathmoveto{\pgfqpoint{8.219097in}{0.844444in}}%
\pgfpathlineto{\pgfqpoint{8.219097in}{2.425000in}}%
\pgfusepath{stroke}%
\end{pgfscope}%
\begin{pgfscope}%
\pgfsetrectcap%
\pgfsetmiterjoin%
\pgfsetlinewidth{0.803000pt}%
\definecolor{currentstroke}{rgb}{0.000000,0.000000,0.000000}%
\pgfsetstrokecolor{currentstroke}%
\pgfsetdash{}{0pt}%
\pgfpathmoveto{\pgfqpoint{11.137500in}{0.844444in}}%
\pgfpathlineto{\pgfqpoint{11.137500in}{2.425000in}}%
\pgfusepath{stroke}%
\end{pgfscope}%
\begin{pgfscope}%
\pgfsetrectcap%
\pgfsetmiterjoin%
\pgfsetlinewidth{0.803000pt}%
\definecolor{currentstroke}{rgb}{0.000000,0.000000,0.000000}%
\pgfsetstrokecolor{currentstroke}%
\pgfsetdash{}{0pt}%
\pgfpathmoveto{\pgfqpoint{8.219097in}{0.844444in}}%
\pgfpathlineto{\pgfqpoint{11.137500in}{0.844444in}}%
\pgfusepath{stroke}%
\end{pgfscope}%
\begin{pgfscope}%
\pgfsetrectcap%
\pgfsetmiterjoin%
\pgfsetlinewidth{0.803000pt}%
\definecolor{currentstroke}{rgb}{0.000000,0.000000,0.000000}%
\pgfsetstrokecolor{currentstroke}%
\pgfsetdash{}{0pt}%
\pgfpathmoveto{\pgfqpoint{8.219097in}{2.425000in}}%
\pgfpathlineto{\pgfqpoint{11.137500in}{2.425000in}}%
\pgfusepath{stroke}%
\end{pgfscope}%
\begin{pgfscope}%
\pgfsetbuttcap%
\pgfsetmiterjoin%
\definecolor{currentfill}{rgb}{1.000000,1.000000,1.000000}%
\pgfsetfillcolor{currentfill}%
\pgfsetlinewidth{0.000000pt}%
\definecolor{currentstroke}{rgb}{0.000000,0.000000,0.000000}%
\pgfsetstrokecolor{currentstroke}%
\pgfsetstrokeopacity{0.000000}%
\pgfsetdash{}{0pt}%
\pgfpathmoveto{\pgfqpoint{11.931597in}{0.844444in}}%
\pgfpathlineto{\pgfqpoint{14.850000in}{0.844444in}}%
\pgfpathlineto{\pgfqpoint{14.850000in}{2.425000in}}%
\pgfpathlineto{\pgfqpoint{11.931597in}{2.425000in}}%
\pgfpathlineto{\pgfqpoint{11.931597in}{0.844444in}}%
\pgfpathclose%
\pgfusepath{fill}%
\end{pgfscope}%
\begin{pgfscope}%
\pgfpathrectangle{\pgfqpoint{11.931597in}{0.844444in}}{\pgfqpoint{2.918403in}{1.580556in}}%
\pgfusepath{clip}%
\pgfsetbuttcap%
\pgfsetroundjoin%
\definecolor{currentfill}{rgb}{0.994079,0.841446,0.774548}%
\pgfsetfillcolor{currentfill}%
\pgfsetlinewidth{0.000000pt}%
\definecolor{currentstroke}{rgb}{0.000000,0.000000,0.000000}%
\pgfsetstrokecolor{currentstroke}%
\pgfsetdash{}{0pt}%
\pgfpathmoveto{\pgfqpoint{11.931597in}{0.844444in}}%
\pgfpathlineto{\pgfqpoint{11.931597in}{1.239583in}}%
\pgfpathlineto{\pgfqpoint{12.515278in}{1.239583in}}%
\pgfpathlineto{\pgfqpoint{12.515278in}{0.844444in}}%
\pgfpathlineto{\pgfqpoint{11.931597in}{0.844444in}}%
\pgfpathlineto{\pgfqpoint{11.931597in}{0.844444in}}%
\pgfpathclose%
\pgfusepath{fill}%
\end{pgfscope}%
\begin{pgfscope}%
\pgfpathrectangle{\pgfqpoint{11.931597in}{0.844444in}}{\pgfqpoint{2.918403in}{1.580556in}}%
\pgfusepath{clip}%
\pgfsetbuttcap%
\pgfsetroundjoin%
\definecolor{currentfill}{rgb}{1.000000,0.960784,0.941176}%
\pgfsetfillcolor{currentfill}%
\pgfsetlinewidth{0.000000pt}%
\definecolor{currentstroke}{rgb}{0.000000,0.000000,0.000000}%
\pgfsetstrokecolor{currentstroke}%
\pgfsetdash{}{0pt}%
\pgfpathmoveto{\pgfqpoint{12.515278in}{0.844444in}}%
\pgfpathlineto{\pgfqpoint{12.515278in}{1.239583in}}%
\pgfpathlineto{\pgfqpoint{13.098958in}{1.239583in}}%
\pgfpathlineto{\pgfqpoint{13.098958in}{0.844444in}}%
\pgfpathlineto{\pgfqpoint{12.515278in}{0.844444in}}%
\pgfpathlineto{\pgfqpoint{12.515278in}{0.844444in}}%
\pgfpathclose%
\pgfusepath{fill}%
\end{pgfscope}%
\begin{pgfscope}%
\pgfpathrectangle{\pgfqpoint{11.931597in}{0.844444in}}{\pgfqpoint{2.918403in}{1.580556in}}%
\pgfusepath{clip}%
\pgfsetbuttcap%
\pgfsetroundjoin%
\definecolor{currentfill}{rgb}{0.984744,0.432910,0.307420}%
\pgfsetfillcolor{currentfill}%
\pgfsetlinewidth{0.000000pt}%
\definecolor{currentstroke}{rgb}{0.000000,0.000000,0.000000}%
\pgfsetstrokecolor{currentstroke}%
\pgfsetdash{}{0pt}%
\pgfpathmoveto{\pgfqpoint{13.098958in}{0.844444in}}%
\pgfpathlineto{\pgfqpoint{13.098958in}{1.239583in}}%
\pgfpathlineto{\pgfqpoint{13.682639in}{1.239583in}}%
\pgfpathlineto{\pgfqpoint{13.682639in}{0.844444in}}%
\pgfpathlineto{\pgfqpoint{13.098958in}{0.844444in}}%
\pgfpathlineto{\pgfqpoint{13.098958in}{0.844444in}}%
\pgfpathclose%
\pgfusepath{fill}%
\end{pgfscope}%
\begin{pgfscope}%
\pgfpathrectangle{\pgfqpoint{11.931597in}{0.844444in}}{\pgfqpoint{2.918403in}{1.580556in}}%
\pgfusepath{clip}%
\pgfsetbuttcap%
\pgfsetroundjoin%
\definecolor{currentfill}{rgb}{0.988235,0.661453,0.548973}%
\pgfsetfillcolor{currentfill}%
\pgfsetlinewidth{0.000000pt}%
\definecolor{currentstroke}{rgb}{0.000000,0.000000,0.000000}%
\pgfsetstrokecolor{currentstroke}%
\pgfsetdash{}{0pt}%
\pgfpathmoveto{\pgfqpoint{13.682639in}{0.844444in}}%
\pgfpathlineto{\pgfqpoint{13.682639in}{1.239583in}}%
\pgfpathlineto{\pgfqpoint{14.266319in}{1.239583in}}%
\pgfpathlineto{\pgfqpoint{14.266319in}{0.844444in}}%
\pgfpathlineto{\pgfqpoint{13.682639in}{0.844444in}}%
\pgfpathlineto{\pgfqpoint{13.682639in}{0.844444in}}%
\pgfpathclose%
\pgfusepath{fill}%
\end{pgfscope}%
\begin{pgfscope}%
\pgfpathrectangle{\pgfqpoint{11.931597in}{0.844444in}}{\pgfqpoint{2.918403in}{1.580556in}}%
\pgfusepath{clip}%
\pgfsetbuttcap%
\pgfsetroundjoin%
\definecolor{currentfill}{rgb}{0.988235,0.590834,0.468020}%
\pgfsetfillcolor{currentfill}%
\pgfsetlinewidth{0.000000pt}%
\definecolor{currentstroke}{rgb}{0.000000,0.000000,0.000000}%
\pgfsetstrokecolor{currentstroke}%
\pgfsetdash{}{0pt}%
\pgfpathmoveto{\pgfqpoint{14.266319in}{0.844444in}}%
\pgfpathlineto{\pgfqpoint{14.266319in}{1.239583in}}%
\pgfpathlineto{\pgfqpoint{14.850000in}{1.239583in}}%
\pgfpathlineto{\pgfqpoint{14.850000in}{0.844444in}}%
\pgfpathlineto{\pgfqpoint{14.266319in}{0.844444in}}%
\pgfpathlineto{\pgfqpoint{14.266319in}{0.844444in}}%
\pgfpathclose%
\pgfusepath{fill}%
\end{pgfscope}%
\begin{pgfscope}%
\pgfpathrectangle{\pgfqpoint{11.931597in}{0.844444in}}{\pgfqpoint{2.918403in}{1.580556in}}%
\pgfusepath{clip}%
\pgfsetbuttcap%
\pgfsetroundjoin%
\definecolor{currentfill}{rgb}{0.986959,0.521492,0.396002}%
\pgfsetfillcolor{currentfill}%
\pgfsetlinewidth{0.000000pt}%
\definecolor{currentstroke}{rgb}{0.000000,0.000000,0.000000}%
\pgfsetstrokecolor{currentstroke}%
\pgfsetdash{}{0pt}%
\pgfpathmoveto{\pgfqpoint{11.931597in}{1.239583in}}%
\pgfpathlineto{\pgfqpoint{11.931597in}{1.634722in}}%
\pgfpathlineto{\pgfqpoint{12.515278in}{1.634722in}}%
\pgfpathlineto{\pgfqpoint{12.515278in}{1.239583in}}%
\pgfpathlineto{\pgfqpoint{11.931597in}{1.239583in}}%
\pgfpathlineto{\pgfqpoint{11.931597in}{1.239583in}}%
\pgfpathclose%
\pgfusepath{fill}%
\end{pgfscope}%
\begin{pgfscope}%
\pgfpathrectangle{\pgfqpoint{11.931597in}{0.844444in}}{\pgfqpoint{2.918403in}{1.580556in}}%
\pgfusepath{clip}%
\pgfsetbuttcap%
\pgfsetroundjoin%
\definecolor{currentfill}{rgb}{0.988066,0.565782,0.440292}%
\pgfsetfillcolor{currentfill}%
\pgfsetlinewidth{0.000000pt}%
\definecolor{currentstroke}{rgb}{0.000000,0.000000,0.000000}%
\pgfsetstrokecolor{currentstroke}%
\pgfsetdash{}{0pt}%
\pgfpathmoveto{\pgfqpoint{12.515278in}{1.239583in}}%
\pgfpathlineto{\pgfqpoint{12.515278in}{1.634722in}}%
\pgfpathlineto{\pgfqpoint{13.098958in}{1.634722in}}%
\pgfpathlineto{\pgfqpoint{13.098958in}{1.239583in}}%
\pgfpathlineto{\pgfqpoint{12.515278in}{1.239583in}}%
\pgfpathlineto{\pgfqpoint{12.515278in}{1.239583in}}%
\pgfpathclose%
\pgfusepath{fill}%
\end{pgfscope}%
\begin{pgfscope}%
\pgfpathrectangle{\pgfqpoint{11.931597in}{0.844444in}}{\pgfqpoint{2.918403in}{1.580556in}}%
\pgfusepath{clip}%
\pgfsetbuttcap%
\pgfsetroundjoin%
\definecolor{currentfill}{rgb}{0.988235,0.676586,0.566321}%
\pgfsetfillcolor{currentfill}%
\pgfsetlinewidth{0.000000pt}%
\definecolor{currentstroke}{rgb}{0.000000,0.000000,0.000000}%
\pgfsetstrokecolor{currentstroke}%
\pgfsetdash{}{0pt}%
\pgfpathmoveto{\pgfqpoint{13.098958in}{1.239583in}}%
\pgfpathlineto{\pgfqpoint{13.098958in}{1.634722in}}%
\pgfpathlineto{\pgfqpoint{13.682639in}{1.634722in}}%
\pgfpathlineto{\pgfqpoint{13.682639in}{1.239583in}}%
\pgfpathlineto{\pgfqpoint{13.098958in}{1.239583in}}%
\pgfpathlineto{\pgfqpoint{13.098958in}{1.239583in}}%
\pgfpathclose%
\pgfusepath{fill}%
\end{pgfscope}%
\begin{pgfscope}%
\pgfpathrectangle{\pgfqpoint{11.931597in}{0.844444in}}{\pgfqpoint{2.918403in}{1.580556in}}%
\pgfusepath{clip}%
\pgfsetbuttcap%
\pgfsetroundjoin%
\definecolor{currentfill}{rgb}{0.988235,0.681630,0.572103}%
\pgfsetfillcolor{currentfill}%
\pgfsetlinewidth{0.000000pt}%
\definecolor{currentstroke}{rgb}{0.000000,0.000000,0.000000}%
\pgfsetstrokecolor{currentstroke}%
\pgfsetdash{}{0pt}%
\pgfpathmoveto{\pgfqpoint{13.682639in}{1.239583in}}%
\pgfpathlineto{\pgfqpoint{13.682639in}{1.634722in}}%
\pgfpathlineto{\pgfqpoint{14.266319in}{1.634722in}}%
\pgfpathlineto{\pgfqpoint{14.266319in}{1.239583in}}%
\pgfpathlineto{\pgfqpoint{13.682639in}{1.239583in}}%
\pgfpathlineto{\pgfqpoint{13.682639in}{1.239583in}}%
\pgfpathclose%
\pgfusepath{fill}%
\end{pgfscope}%
\begin{pgfscope}%
\pgfpathrectangle{\pgfqpoint{11.931597in}{0.844444in}}{\pgfqpoint{2.918403in}{1.580556in}}%
\pgfusepath{clip}%
\pgfsetbuttcap%
\pgfsetroundjoin%
\definecolor{currentfill}{rgb}{0.986098,0.487043,0.361553}%
\pgfsetfillcolor{currentfill}%
\pgfsetlinewidth{0.000000pt}%
\definecolor{currentstroke}{rgb}{0.000000,0.000000,0.000000}%
\pgfsetstrokecolor{currentstroke}%
\pgfsetdash{}{0pt}%
\pgfpathmoveto{\pgfqpoint{14.266319in}{1.239583in}}%
\pgfpathlineto{\pgfqpoint{14.266319in}{1.634722in}}%
\pgfpathlineto{\pgfqpoint{14.850000in}{1.634722in}}%
\pgfpathlineto{\pgfqpoint{14.850000in}{1.239583in}}%
\pgfpathlineto{\pgfqpoint{14.266319in}{1.239583in}}%
\pgfpathlineto{\pgfqpoint{14.266319in}{1.239583in}}%
\pgfpathclose%
\pgfusepath{fill}%
\end{pgfscope}%
\begin{pgfscope}%
\pgfpathrectangle{\pgfqpoint{11.931597in}{0.844444in}}{\pgfqpoint{2.918403in}{1.580556in}}%
\pgfusepath{clip}%
\pgfsetbuttcap%
\pgfsetroundjoin%
\definecolor{currentfill}{rgb}{0.890196,0.185621,0.152941}%
\pgfsetfillcolor{currentfill}%
\pgfsetlinewidth{0.000000pt}%
\definecolor{currentstroke}{rgb}{0.000000,0.000000,0.000000}%
\pgfsetstrokecolor{currentstroke}%
\pgfsetdash{}{0pt}%
\pgfpathmoveto{\pgfqpoint{11.931597in}{1.634722in}}%
\pgfpathlineto{\pgfqpoint{11.931597in}{2.029861in}}%
\pgfpathlineto{\pgfqpoint{12.515278in}{2.029861in}}%
\pgfpathlineto{\pgfqpoint{12.515278in}{1.634722in}}%
\pgfpathlineto{\pgfqpoint{11.931597in}{1.634722in}}%
\pgfpathlineto{\pgfqpoint{11.931597in}{1.634722in}}%
\pgfpathclose%
\pgfusepath{fill}%
\end{pgfscope}%
\begin{pgfscope}%
\pgfpathrectangle{\pgfqpoint{11.931597in}{0.844444in}}{\pgfqpoint{2.918403in}{1.580556in}}%
\pgfusepath{clip}%
\pgfsetbuttcap%
\pgfsetroundjoin%
\definecolor{currentfill}{rgb}{0.970288,0.360754,0.255133}%
\pgfsetfillcolor{currentfill}%
\pgfsetlinewidth{0.000000pt}%
\definecolor{currentstroke}{rgb}{0.000000,0.000000,0.000000}%
\pgfsetstrokecolor{currentstroke}%
\pgfsetdash{}{0pt}%
\pgfpathmoveto{\pgfqpoint{12.515278in}{1.634722in}}%
\pgfpathlineto{\pgfqpoint{12.515278in}{2.029861in}}%
\pgfpathlineto{\pgfqpoint{13.098958in}{2.029861in}}%
\pgfpathlineto{\pgfqpoint{13.098958in}{1.634722in}}%
\pgfpathlineto{\pgfqpoint{12.515278in}{1.634722in}}%
\pgfpathlineto{\pgfqpoint{12.515278in}{1.634722in}}%
\pgfpathclose%
\pgfusepath{fill}%
\end{pgfscope}%
\begin{pgfscope}%
\pgfpathrectangle{\pgfqpoint{11.931597in}{0.844444in}}{\pgfqpoint{2.918403in}{1.580556in}}%
\pgfusepath{clip}%
\pgfsetbuttcap%
\pgfsetroundjoin%
\definecolor{currentfill}{rgb}{0.403922,0.000000,0.050980}%
\pgfsetfillcolor{currentfill}%
\pgfsetlinewidth{0.000000pt}%
\definecolor{currentstroke}{rgb}{0.000000,0.000000,0.000000}%
\pgfsetstrokecolor{currentstroke}%
\pgfsetdash{}{0pt}%
\pgfpathmoveto{\pgfqpoint{13.098958in}{1.634722in}}%
\pgfpathlineto{\pgfqpoint{13.098958in}{2.029861in}}%
\pgfpathlineto{\pgfqpoint{13.682639in}{2.029861in}}%
\pgfpathlineto{\pgfqpoint{13.682639in}{1.634722in}}%
\pgfpathlineto{\pgfqpoint{13.098958in}{1.634722in}}%
\pgfpathlineto{\pgfqpoint{13.098958in}{1.634722in}}%
\pgfpathclose%
\pgfusepath{fill}%
\end{pgfscope}%
\begin{pgfscope}%
\pgfpathrectangle{\pgfqpoint{11.931597in}{0.844444in}}{\pgfqpoint{2.918403in}{1.580556in}}%
\pgfusepath{clip}%
\pgfsetbuttcap%
\pgfsetroundjoin%
\definecolor{currentfill}{rgb}{0.988235,0.580746,0.456455}%
\pgfsetfillcolor{currentfill}%
\pgfsetlinewidth{0.000000pt}%
\definecolor{currentstroke}{rgb}{0.000000,0.000000,0.000000}%
\pgfsetstrokecolor{currentstroke}%
\pgfsetdash{}{0pt}%
\pgfpathmoveto{\pgfqpoint{13.682639in}{1.634722in}}%
\pgfpathlineto{\pgfqpoint{13.682639in}{2.029861in}}%
\pgfpathlineto{\pgfqpoint{14.266319in}{2.029861in}}%
\pgfpathlineto{\pgfqpoint{14.266319in}{1.634722in}}%
\pgfpathlineto{\pgfqpoint{13.682639in}{1.634722in}}%
\pgfpathlineto{\pgfqpoint{13.682639in}{1.634722in}}%
\pgfpathclose%
\pgfusepath{fill}%
\end{pgfscope}%
\begin{pgfscope}%
\pgfpathrectangle{\pgfqpoint{11.931597in}{0.844444in}}{\pgfqpoint{2.918403in}{1.580556in}}%
\pgfusepath{clip}%
\pgfsetbuttcap%
\pgfsetroundjoin%
\definecolor{currentfill}{rgb}{0.984498,0.423068,0.297578}%
\pgfsetfillcolor{currentfill}%
\pgfsetlinewidth{0.000000pt}%
\definecolor{currentstroke}{rgb}{0.000000,0.000000,0.000000}%
\pgfsetstrokecolor{currentstroke}%
\pgfsetdash{}{0pt}%
\pgfpathmoveto{\pgfqpoint{14.266319in}{1.634722in}}%
\pgfpathlineto{\pgfqpoint{14.266319in}{2.029861in}}%
\pgfpathlineto{\pgfqpoint{14.850000in}{2.029861in}}%
\pgfpathlineto{\pgfqpoint{14.850000in}{1.634722in}}%
\pgfpathlineto{\pgfqpoint{14.266319in}{1.634722in}}%
\pgfpathlineto{\pgfqpoint{14.266319in}{1.634722in}}%
\pgfpathclose%
\pgfusepath{fill}%
\end{pgfscope}%
\begin{pgfscope}%
\pgfpathrectangle{\pgfqpoint{11.931597in}{0.844444in}}{\pgfqpoint{2.918403in}{1.580556in}}%
\pgfusepath{clip}%
\pgfsetbuttcap%
\pgfsetroundjoin%
\definecolor{currentfill}{rgb}{0.958478,0.314494,0.225606}%
\pgfsetfillcolor{currentfill}%
\pgfsetlinewidth{0.000000pt}%
\definecolor{currentstroke}{rgb}{0.000000,0.000000,0.000000}%
\pgfsetstrokecolor{currentstroke}%
\pgfsetdash{}{0pt}%
\pgfpathmoveto{\pgfqpoint{11.931597in}{2.029861in}}%
\pgfpathlineto{\pgfqpoint{11.931597in}{2.425000in}}%
\pgfpathlineto{\pgfqpoint{12.515278in}{2.425000in}}%
\pgfpathlineto{\pgfqpoint{12.515278in}{2.029861in}}%
\pgfpathlineto{\pgfqpoint{11.931597in}{2.029861in}}%
\pgfpathlineto{\pgfqpoint{11.931597in}{2.029861in}}%
\pgfpathclose%
\pgfusepath{fill}%
\end{pgfscope}%
\begin{pgfscope}%
\pgfpathrectangle{\pgfqpoint{11.931597in}{0.844444in}}{\pgfqpoint{2.918403in}{1.580556in}}%
\pgfusepath{clip}%
\pgfsetbuttcap%
\pgfsetroundjoin%
\definecolor{currentfill}{rgb}{0.990634,0.777716,0.690150}%
\pgfsetfillcolor{currentfill}%
\pgfsetlinewidth{0.000000pt}%
\definecolor{currentstroke}{rgb}{0.000000,0.000000,0.000000}%
\pgfsetstrokecolor{currentstroke}%
\pgfsetdash{}{0pt}%
\pgfpathmoveto{\pgfqpoint{12.515278in}{2.029861in}}%
\pgfpathlineto{\pgfqpoint{12.515278in}{2.425000in}}%
\pgfpathlineto{\pgfqpoint{13.098958in}{2.425000in}}%
\pgfpathlineto{\pgfqpoint{13.098958in}{2.029861in}}%
\pgfpathlineto{\pgfqpoint{12.515278in}{2.029861in}}%
\pgfpathlineto{\pgfqpoint{12.515278in}{2.029861in}}%
\pgfpathclose%
\pgfusepath{fill}%
\end{pgfscope}%
\begin{pgfscope}%
\pgfpathrectangle{\pgfqpoint{11.931597in}{0.844444in}}{\pgfqpoint{2.918403in}{1.580556in}}%
\pgfusepath{clip}%
\pgfsetbuttcap%
\pgfsetroundjoin%
\definecolor{currentfill}{rgb}{0.647643,0.058962,0.082476}%
\pgfsetfillcolor{currentfill}%
\pgfsetlinewidth{0.000000pt}%
\definecolor{currentstroke}{rgb}{0.000000,0.000000,0.000000}%
\pgfsetstrokecolor{currentstroke}%
\pgfsetdash{}{0pt}%
\pgfpathmoveto{\pgfqpoint{13.098958in}{2.029861in}}%
\pgfpathlineto{\pgfqpoint{13.098958in}{2.425000in}}%
\pgfpathlineto{\pgfqpoint{13.682639in}{2.425000in}}%
\pgfpathlineto{\pgfqpoint{13.682639in}{2.029861in}}%
\pgfpathlineto{\pgfqpoint{13.098958in}{2.029861in}}%
\pgfpathlineto{\pgfqpoint{13.098958in}{2.029861in}}%
\pgfpathclose%
\pgfusepath{fill}%
\end{pgfscope}%
\begin{pgfscope}%
\pgfpathrectangle{\pgfqpoint{11.931597in}{0.844444in}}{\pgfqpoint{2.918403in}{1.580556in}}%
\pgfusepath{clip}%
\pgfsetbuttcap%
\pgfsetroundjoin%
\definecolor{currentfill}{rgb}{0.986098,0.487043,0.361553}%
\pgfsetfillcolor{currentfill}%
\pgfsetlinewidth{0.000000pt}%
\definecolor{currentstroke}{rgb}{0.000000,0.000000,0.000000}%
\pgfsetstrokecolor{currentstroke}%
\pgfsetdash{}{0pt}%
\pgfpathmoveto{\pgfqpoint{13.682639in}{2.029861in}}%
\pgfpathlineto{\pgfqpoint{13.682639in}{2.425000in}}%
\pgfpathlineto{\pgfqpoint{14.266319in}{2.425000in}}%
\pgfpathlineto{\pgfqpoint{14.266319in}{2.029861in}}%
\pgfpathlineto{\pgfqpoint{13.682639in}{2.029861in}}%
\pgfpathlineto{\pgfqpoint{13.682639in}{2.029861in}}%
\pgfpathclose%
\pgfusepath{fill}%
\end{pgfscope}%
\begin{pgfscope}%
\pgfpathrectangle{\pgfqpoint{11.931597in}{0.844444in}}{\pgfqpoint{2.918403in}{1.580556in}}%
\pgfusepath{clip}%
\pgfsetbuttcap%
\pgfsetroundjoin%
\definecolor{currentfill}{rgb}{0.988235,0.656409,0.543191}%
\pgfsetfillcolor{currentfill}%
\pgfsetlinewidth{0.000000pt}%
\definecolor{currentstroke}{rgb}{0.000000,0.000000,0.000000}%
\pgfsetstrokecolor{currentstroke}%
\pgfsetdash{}{0pt}%
\pgfpathmoveto{\pgfqpoint{14.266319in}{2.029861in}}%
\pgfpathlineto{\pgfqpoint{14.266319in}{2.425000in}}%
\pgfpathlineto{\pgfqpoint{14.850000in}{2.425000in}}%
\pgfpathlineto{\pgfqpoint{14.850000in}{2.029861in}}%
\pgfpathlineto{\pgfqpoint{14.266319in}{2.029861in}}%
\pgfpathlineto{\pgfqpoint{14.266319in}{2.029861in}}%
\pgfpathclose%
\pgfusepath{fill}%
\end{pgfscope}%
\begin{pgfscope}%
\pgfsetbuttcap%
\pgfsetroundjoin%
\definecolor{currentfill}{rgb}{0.000000,0.000000,0.000000}%
\pgfsetfillcolor{currentfill}%
\pgfsetlinewidth{0.803000pt}%
\definecolor{currentstroke}{rgb}{0.000000,0.000000,0.000000}%
\pgfsetstrokecolor{currentstroke}%
\pgfsetdash{}{0pt}%
\pgfsys@defobject{currentmarker}{\pgfqpoint{0.000000in}{-0.048611in}}{\pgfqpoint{0.000000in}{0.000000in}}{%
\pgfpathmoveto{\pgfqpoint{0.000000in}{0.000000in}}%
\pgfpathlineto{\pgfqpoint{0.000000in}{-0.048611in}}%
\pgfusepath{stroke,fill}%
}%
\begin{pgfscope}%
\pgfsys@transformshift{11.931597in}{0.844444in}%
\pgfsys@useobject{currentmarker}{}%
\end{pgfscope}%
\end{pgfscope}%
\begin{pgfscope}%
\definecolor{textcolor}{rgb}{0.000000,0.000000,0.000000}%
\pgfsetstrokecolor{textcolor}%
\pgfsetfillcolor{textcolor}%
\pgftext[x=11.931597in,y=0.747222in,,top]{\color{textcolor}\sffamily\fontsize{10.000000}{12.000000}\selectfont 0}%
\end{pgfscope}%
\begin{pgfscope}%
\pgfsetbuttcap%
\pgfsetroundjoin%
\definecolor{currentfill}{rgb}{0.000000,0.000000,0.000000}%
\pgfsetfillcolor{currentfill}%
\pgfsetlinewidth{0.803000pt}%
\definecolor{currentstroke}{rgb}{0.000000,0.000000,0.000000}%
\pgfsetstrokecolor{currentstroke}%
\pgfsetdash{}{0pt}%
\pgfsys@defobject{currentmarker}{\pgfqpoint{0.000000in}{-0.048611in}}{\pgfqpoint{0.000000in}{0.000000in}}{%
\pgfpathmoveto{\pgfqpoint{0.000000in}{0.000000in}}%
\pgfpathlineto{\pgfqpoint{0.000000in}{-0.048611in}}%
\pgfusepath{stroke,fill}%
}%
\begin{pgfscope}%
\pgfsys@transformshift{12.515278in}{0.844444in}%
\pgfsys@useobject{currentmarker}{}%
\end{pgfscope}%
\end{pgfscope}%
\begin{pgfscope}%
\definecolor{textcolor}{rgb}{0.000000,0.000000,0.000000}%
\pgfsetstrokecolor{textcolor}%
\pgfsetfillcolor{textcolor}%
\pgftext[x=12.515278in,y=0.747222in,,top]{\color{textcolor}\sffamily\fontsize{10.000000}{12.000000}\selectfont 1}%
\end{pgfscope}%
\begin{pgfscope}%
\pgfsetbuttcap%
\pgfsetroundjoin%
\definecolor{currentfill}{rgb}{0.000000,0.000000,0.000000}%
\pgfsetfillcolor{currentfill}%
\pgfsetlinewidth{0.803000pt}%
\definecolor{currentstroke}{rgb}{0.000000,0.000000,0.000000}%
\pgfsetstrokecolor{currentstroke}%
\pgfsetdash{}{0pt}%
\pgfsys@defobject{currentmarker}{\pgfqpoint{0.000000in}{-0.048611in}}{\pgfqpoint{0.000000in}{0.000000in}}{%
\pgfpathmoveto{\pgfqpoint{0.000000in}{0.000000in}}%
\pgfpathlineto{\pgfqpoint{0.000000in}{-0.048611in}}%
\pgfusepath{stroke,fill}%
}%
\begin{pgfscope}%
\pgfsys@transformshift{13.098958in}{0.844444in}%
\pgfsys@useobject{currentmarker}{}%
\end{pgfscope}%
\end{pgfscope}%
\begin{pgfscope}%
\definecolor{textcolor}{rgb}{0.000000,0.000000,0.000000}%
\pgfsetstrokecolor{textcolor}%
\pgfsetfillcolor{textcolor}%
\pgftext[x=13.098958in,y=0.747222in,,top]{\color{textcolor}\sffamily\fontsize{10.000000}{12.000000}\selectfont 2}%
\end{pgfscope}%
\begin{pgfscope}%
\pgfsetbuttcap%
\pgfsetroundjoin%
\definecolor{currentfill}{rgb}{0.000000,0.000000,0.000000}%
\pgfsetfillcolor{currentfill}%
\pgfsetlinewidth{0.803000pt}%
\definecolor{currentstroke}{rgb}{0.000000,0.000000,0.000000}%
\pgfsetstrokecolor{currentstroke}%
\pgfsetdash{}{0pt}%
\pgfsys@defobject{currentmarker}{\pgfqpoint{0.000000in}{-0.048611in}}{\pgfqpoint{0.000000in}{0.000000in}}{%
\pgfpathmoveto{\pgfqpoint{0.000000in}{0.000000in}}%
\pgfpathlineto{\pgfqpoint{0.000000in}{-0.048611in}}%
\pgfusepath{stroke,fill}%
}%
\begin{pgfscope}%
\pgfsys@transformshift{13.682639in}{0.844444in}%
\pgfsys@useobject{currentmarker}{}%
\end{pgfscope}%
\end{pgfscope}%
\begin{pgfscope}%
\definecolor{textcolor}{rgb}{0.000000,0.000000,0.000000}%
\pgfsetstrokecolor{textcolor}%
\pgfsetfillcolor{textcolor}%
\pgftext[x=13.682639in,y=0.747222in,,top]{\color{textcolor}\sffamily\fontsize{10.000000}{12.000000}\selectfont 3}%
\end{pgfscope}%
\begin{pgfscope}%
\pgfsetbuttcap%
\pgfsetroundjoin%
\definecolor{currentfill}{rgb}{0.000000,0.000000,0.000000}%
\pgfsetfillcolor{currentfill}%
\pgfsetlinewidth{0.803000pt}%
\definecolor{currentstroke}{rgb}{0.000000,0.000000,0.000000}%
\pgfsetstrokecolor{currentstroke}%
\pgfsetdash{}{0pt}%
\pgfsys@defobject{currentmarker}{\pgfqpoint{0.000000in}{-0.048611in}}{\pgfqpoint{0.000000in}{0.000000in}}{%
\pgfpathmoveto{\pgfqpoint{0.000000in}{0.000000in}}%
\pgfpathlineto{\pgfqpoint{0.000000in}{-0.048611in}}%
\pgfusepath{stroke,fill}%
}%
\begin{pgfscope}%
\pgfsys@transformshift{14.266319in}{0.844444in}%
\pgfsys@useobject{currentmarker}{}%
\end{pgfscope}%
\end{pgfscope}%
\begin{pgfscope}%
\definecolor{textcolor}{rgb}{0.000000,0.000000,0.000000}%
\pgfsetstrokecolor{textcolor}%
\pgfsetfillcolor{textcolor}%
\pgftext[x=14.266319in,y=0.747222in,,top]{\color{textcolor}\sffamily\fontsize{10.000000}{12.000000}\selectfont 4}%
\end{pgfscope}%
\begin{pgfscope}%
\definecolor{textcolor}{rgb}{0.000000,0.000000,0.000000}%
\pgfsetstrokecolor{textcolor}%
\pgfsetfillcolor{textcolor}%
\pgftext[x=13.390799in,y=0.557254in,,top]{\color{textcolor}\sffamily\fontsize{30.000000}{36.000000}\selectfont x axis}%
\end{pgfscope}%
\begin{pgfscope}%
\pgfsetbuttcap%
\pgfsetroundjoin%
\definecolor{currentfill}{rgb}{0.000000,0.000000,0.000000}%
\pgfsetfillcolor{currentfill}%
\pgfsetlinewidth{0.803000pt}%
\definecolor{currentstroke}{rgb}{0.000000,0.000000,0.000000}%
\pgfsetstrokecolor{currentstroke}%
\pgfsetdash{}{0pt}%
\pgfsys@defobject{currentmarker}{\pgfqpoint{-0.048611in}{0.000000in}}{\pgfqpoint{-0.000000in}{0.000000in}}{%
\pgfpathmoveto{\pgfqpoint{-0.000000in}{0.000000in}}%
\pgfpathlineto{\pgfqpoint{-0.048611in}{0.000000in}}%
\pgfusepath{stroke,fill}%
}%
\begin{pgfscope}%
\pgfsys@transformshift{11.931597in}{0.844444in}%
\pgfsys@useobject{currentmarker}{}%
\end{pgfscope}%
\end{pgfscope}%
\begin{pgfscope}%
\definecolor{textcolor}{rgb}{0.000000,0.000000,0.000000}%
\pgfsetstrokecolor{textcolor}%
\pgfsetfillcolor{textcolor}%
\pgftext[x=11.746010in, y=0.791683in, left, base]{\color{textcolor}\sffamily\fontsize{10.000000}{12.000000}\selectfont 0}%
\end{pgfscope}%
\begin{pgfscope}%
\pgfsetbuttcap%
\pgfsetroundjoin%
\definecolor{currentfill}{rgb}{0.000000,0.000000,0.000000}%
\pgfsetfillcolor{currentfill}%
\pgfsetlinewidth{0.803000pt}%
\definecolor{currentstroke}{rgb}{0.000000,0.000000,0.000000}%
\pgfsetstrokecolor{currentstroke}%
\pgfsetdash{}{0pt}%
\pgfsys@defobject{currentmarker}{\pgfqpoint{-0.048611in}{0.000000in}}{\pgfqpoint{-0.000000in}{0.000000in}}{%
\pgfpathmoveto{\pgfqpoint{-0.000000in}{0.000000in}}%
\pgfpathlineto{\pgfqpoint{-0.048611in}{0.000000in}}%
\pgfusepath{stroke,fill}%
}%
\begin{pgfscope}%
\pgfsys@transformshift{11.931597in}{1.239583in}%
\pgfsys@useobject{currentmarker}{}%
\end{pgfscope}%
\end{pgfscope}%
\begin{pgfscope}%
\definecolor{textcolor}{rgb}{0.000000,0.000000,0.000000}%
\pgfsetstrokecolor{textcolor}%
\pgfsetfillcolor{textcolor}%
\pgftext[x=11.746010in, y=1.186822in, left, base]{\color{textcolor}\sffamily\fontsize{10.000000}{12.000000}\selectfont 1}%
\end{pgfscope}%
\begin{pgfscope}%
\pgfsetbuttcap%
\pgfsetroundjoin%
\definecolor{currentfill}{rgb}{0.000000,0.000000,0.000000}%
\pgfsetfillcolor{currentfill}%
\pgfsetlinewidth{0.803000pt}%
\definecolor{currentstroke}{rgb}{0.000000,0.000000,0.000000}%
\pgfsetstrokecolor{currentstroke}%
\pgfsetdash{}{0pt}%
\pgfsys@defobject{currentmarker}{\pgfqpoint{-0.048611in}{0.000000in}}{\pgfqpoint{-0.000000in}{0.000000in}}{%
\pgfpathmoveto{\pgfqpoint{-0.000000in}{0.000000in}}%
\pgfpathlineto{\pgfqpoint{-0.048611in}{0.000000in}}%
\pgfusepath{stroke,fill}%
}%
\begin{pgfscope}%
\pgfsys@transformshift{11.931597in}{1.634722in}%
\pgfsys@useobject{currentmarker}{}%
\end{pgfscope}%
\end{pgfscope}%
\begin{pgfscope}%
\definecolor{textcolor}{rgb}{0.000000,0.000000,0.000000}%
\pgfsetstrokecolor{textcolor}%
\pgfsetfillcolor{textcolor}%
\pgftext[x=11.746010in, y=1.581961in, left, base]{\color{textcolor}\sffamily\fontsize{10.000000}{12.000000}\selectfont 2}%
\end{pgfscope}%
\begin{pgfscope}%
\pgfsetbuttcap%
\pgfsetroundjoin%
\definecolor{currentfill}{rgb}{0.000000,0.000000,0.000000}%
\pgfsetfillcolor{currentfill}%
\pgfsetlinewidth{0.803000pt}%
\definecolor{currentstroke}{rgb}{0.000000,0.000000,0.000000}%
\pgfsetstrokecolor{currentstroke}%
\pgfsetdash{}{0pt}%
\pgfsys@defobject{currentmarker}{\pgfqpoint{-0.048611in}{0.000000in}}{\pgfqpoint{-0.000000in}{0.000000in}}{%
\pgfpathmoveto{\pgfqpoint{-0.000000in}{0.000000in}}%
\pgfpathlineto{\pgfqpoint{-0.048611in}{0.000000in}}%
\pgfusepath{stroke,fill}%
}%
\begin{pgfscope}%
\pgfsys@transformshift{11.931597in}{2.029861in}%
\pgfsys@useobject{currentmarker}{}%
\end{pgfscope}%
\end{pgfscope}%
\begin{pgfscope}%
\definecolor{textcolor}{rgb}{0.000000,0.000000,0.000000}%
\pgfsetstrokecolor{textcolor}%
\pgfsetfillcolor{textcolor}%
\pgftext[x=11.746010in, y=1.977100in, left, base]{\color{textcolor}\sffamily\fontsize{10.000000}{12.000000}\selectfont 3}%
\end{pgfscope}%
\begin{pgfscope}%
\definecolor{textcolor}{rgb}{0.000000,0.000000,0.000000}%
\pgfsetstrokecolor{textcolor}%
\pgfsetfillcolor{textcolor}%
\pgftext[x=11.690454in,y=1.634722in,,bottom,rotate=90.000000]{\color{textcolor}\sffamily\fontsize{30.000000}{36.000000}\selectfont y axis}%
\end{pgfscope}%
\begin{pgfscope}%
\pgfsetrectcap%
\pgfsetmiterjoin%
\pgfsetlinewidth{0.803000pt}%
\definecolor{currentstroke}{rgb}{0.000000,0.000000,0.000000}%
\pgfsetstrokecolor{currentstroke}%
\pgfsetdash{}{0pt}%
\pgfpathmoveto{\pgfqpoint{11.931597in}{0.844444in}}%
\pgfpathlineto{\pgfqpoint{11.931597in}{2.425000in}}%
\pgfusepath{stroke}%
\end{pgfscope}%
\begin{pgfscope}%
\pgfsetrectcap%
\pgfsetmiterjoin%
\pgfsetlinewidth{0.803000pt}%
\definecolor{currentstroke}{rgb}{0.000000,0.000000,0.000000}%
\pgfsetstrokecolor{currentstroke}%
\pgfsetdash{}{0pt}%
\pgfpathmoveto{\pgfqpoint{14.850000in}{0.844444in}}%
\pgfpathlineto{\pgfqpoint{14.850000in}{2.425000in}}%
\pgfusepath{stroke}%
\end{pgfscope}%
\begin{pgfscope}%
\pgfsetrectcap%
\pgfsetmiterjoin%
\pgfsetlinewidth{0.803000pt}%
\definecolor{currentstroke}{rgb}{0.000000,0.000000,0.000000}%
\pgfsetstrokecolor{currentstroke}%
\pgfsetdash{}{0pt}%
\pgfpathmoveto{\pgfqpoint{11.931597in}{0.844444in}}%
\pgfpathlineto{\pgfqpoint{14.850000in}{0.844444in}}%
\pgfusepath{stroke}%
\end{pgfscope}%
\begin{pgfscope}%
\pgfsetrectcap%
\pgfsetmiterjoin%
\pgfsetlinewidth{0.803000pt}%
\definecolor{currentstroke}{rgb}{0.000000,0.000000,0.000000}%
\pgfsetstrokecolor{currentstroke}%
\pgfsetdash{}{0pt}%
\pgfpathmoveto{\pgfqpoint{11.931597in}{2.425000in}}%
\pgfpathlineto{\pgfqpoint{14.850000in}{2.425000in}}%
\pgfusepath{stroke}%
\end{pgfscope}%
\end{pgfpicture}%
\makeatother%
\endgroup%
}
\vspace*{-10mm}
\caption[Visualization of the streaming]{A visualization of the streaming operator for velocity 2 and 3. In the top row each element is shifted to the top. In the bottom  each element is shifted to the left.}
\label{fig:m1-shifting}
\end{figure}
The local densities and velocities are randomly initialized with an average of $\mu_{\rho}=\mu_{u}=0.5$ and a standard deviation of $std_{\rho}=std_{u}=0.01$ and used to calculate the equilibrium PDF.
For simplicity we only plot two velocity directions.
The top row in figure \ref{fig:m1-shifting} shows the streaming of each element to the top, i.e. using streaming direction $2$. In the bottom row each element is shifted to the left, which corresponds to the streaming in direction $3$.
This is the expected behaviour and concludes Milestone 1.

\section{Collision}
The collision term is the \textit{r.h.s} of the BTE \ref{eq:BTE} and represents the interaction between particles.
It is a complicated two particle scattering integral but can be approximated by a relaxation time approximation $\tau$ so that the PDF locally relaxes to an equilibrium distribution $f_{eq}(r,v,t)$.
The resulting discrete form of the BTE is shown in equation
\begin{equation}
  \label{eq:btw-discrete}
  \begin{aligned}
    f_{i}(r+ \nabla t c_{i},t+\nabla t)=f_{i}(r,t) + \omega \left( f_{i}^{eq}(r,t) - f_{i}(r,t) \right)
  \end{aligned}
\end{equation}
Because of the relaxation the PDF equilibrium is a local one and therefor depends on the local density $\rho$ and the local average velocity $u$.

The local density is just a summation over the velocities of the PDF, i.e. $\rho(r) = \sum_{i} f_{i}$.
An example implementation is shown in listing \ref{lst:rho}.
\begin{center}
\begin{lstlisting}[caption=Implementation of the local density,label=lst:rho, basicstyle=\small]
def local_density(f_cxy: np.array) -> np.array:
    r_xy = np.einsum("cxy -> xy", f_cxy)
    return r_xy
  \end{lstlisting}
\end{center}
The local average velocity is more complicated but effectively it is the summation over the velocity dimensions for each physical dimension divided by the local density.
It is calculated via $\textbf{u}(\textbf{r})=\frac{1}{\rho (\textbf{r})} \sum_{i} c_{i}f_{i}(\textbf{r})$ and respective code is shown in listing \ref{lst:vel}.
\begin{center}
  \begin{lstlisting}[caption=Implementation of the local average velocity.,label=lst:vel, basicstyle=\small]
def local_avg_velocity(f_cxy: np.array, r_xy: np.array):
    u_aij = np.einsum("ac, cxy->axy", C_CA.T, f_cxy) / r_xy
    return u_aij
  \end{lstlisting}
\end{center}
With this we can define the PDF equilibrium function as in equation \ref{eq:pdf-eq},
\begin{equation}
  \label{eq:pdf-eq}
  \begin{aligned}
    f_{i}^{eq} ( \rho , u ) = w_i \rho \left[ 1+3 c_i * u + \frac{9}{2}(c_i * u )^2 - \frac{3}{2} | u |^2 \right]
  \end{aligned}
\end{equation}
where $w_i$ for a D2Q9 lattice can be defined as follow: 
\begin{equation}
w_i = \left( \frac{4}{9}, \frac{1}{9}, \frac{1}{9}, \frac{1}{9}, \frac{1}{9}, \frac{1}{36}, \frac{1}{36}, \frac{1}{36}, \frac{1}{36} \right)
\end{equation}
It is important to note that the approximation in the collision term is not always accurate.
But under the assumption that all transport processes occur on a longer time scale it is appropriate to do.

\section{Boundaries}
The simplest boundaries are static walls. 
The bounce back from the walls are modeled to occur in the opposed direction. 
Listing \ref{lst:top} shows the code for top wall.
\begin{center}
  \begin{lstlisting}[caption=Bounce back at the top wall.,label=lst:top, basicstyle=\small]
def top_wall(f_cxy: np.array) -> np.array:
    f_cxy[4, :, -2] = f_cxy[2, :, -1]
    f_cxy[7, :, -2] = f_cxy[5, :, -1]
    f_cxy[8, :, -2] = f_cxy[6, :, -1]
    return f_cxy
  \end{lstlisting}
\end{center}
The wall reverses the effects of the streaming, just in the opposite direction.
For the bottom wall the velocity vectors would be reversed and instead of copying from the last row to the second last, the first row should be copied to the second column.
For the left and right wall, the left most (or right most) column should be copied to the second column with respect to the corresponding velocities.

Adding a velocity to the top wall leads to a bounce back with added or subtracted velocity as shown in listing \ref{lst:top-slide}. 
\begin{center}
  \begin{lstlisting}[caption=Bounce back at the top sliding wall.,label=lst:top-slide, basicstyle=\small]
def sliding_top_wall(f_cxy: np.array, velocity: float):
    r_top = ( f_cxy[[0, 1, 3], :, -2].sum(axis=0) + 
            2.0 * (f_cxy[2, :, -1] + f_cxy[5, :, -1] + f_cxy[6, :, -1]))
    f_cxy[4, :, -2] = f_cxy[2, :, -1]
    f_cxy[7, :, -2] = f_cxy[5, :, -1] - 6.0 * W_C[5] * r_top * velocity
    f_cxy[8, :, -2] = f_cxy[6, :, -1] + 6.0 * W_C[6] * r_top * velocity
    return f_cxy
  \end{lstlisting}
\end{center}
In the first line, the density at the sliding wall is calculated and later used in the update of the diagonal velocities. For other walls this could be implemented similarly, just with other velocities and row or columns depending on where the wall is located.

\section{Reynolds number}
The flow induced by a moving wall, in this report a moving lid, can be characterized by a certain Reynolds number (Re).
It expresses the ratio between the inertial terms and the viscous ones and can be calculated by equation \ref{eq:reynolds}.
\begin{equation}
    \nu = 1 / 3 * (1 / \omega - 1 / 2) \\
\end{equation}
\begin{equation}\label{eq:reynolds}
     Re = \frac{top-vel * L_x}{\nu}
\end{equation}


\section{Parallelization}
To run the code in an high performance computing environment, e.g. super computers, the code must be able to be used in parallel.
This means the different operations inside the LBM have to be able to be splitted up and calculated by different processes. 
To achieve this dry nodes are added around each splitted up section of the grid.
\begin{figure}[ht]
\centering
\includegraphics[width=0.5\columnwidth]{milestones/final/img/latttice-ghost-beschriftet.png}
\caption[Wet and dry nodes]{$3\times3$ grids where in blue we have the wet nodes and in gray are the dry nodes.}
\label{fig:para-ghost}
\end{figure}
Basically, two additional rows and columns are added to each sub grid. 
The nodes inside them are called dry nodes, because they are not part of the actual simulation but are solely used for computational purposes.


\chapter{Results}
In this chapter I will present the results obtained for the milestones. I chose to keep the order of the milestones and name them chapters accordingly.
In all experiments the lattice size includes dry nodes and the wet nodes.

\section{M3: Shear Wave Decay}
Shear wave decay describes the amplitude decay of shear waves, i.e. elastic waves travelling through the body of an object.
An interesting aspect is the effects of a certain $\omega$, and thereby the kinematic viscosity $\nu(\omega)$ of a fluid, on the shear wave decay.
The kinematic viscosity is especially relevant in fluid dynamics and is defined as the ratio of the dynamic viscosity $\mu$ over the density $\rho$ of the fluid as in equation \ref{eq:viscocity}.
\begin{equation}\label{eq:viscocity}.
\nu = \frac{\mu}{\rho}
\end{equation}
This milestone is separated in three different parts. 
The general motion is to set a sinusoidal density or velocity
perturbation and measure how fast it decays, i.e. the equilibrium state is reached.
For each part I use the following setup:
\begin{enumerate}
    \item $T = 3000$
    % \item $\mu_{\rho}=\mu_{u}=0.5$
    \item $\epsilon = 0.1$
    \item $L_{x}$, $L_{y}$ = 30, 30
\end{enumerate}
The code for this milestone can be found \href{https://github.com/jonas27/pylbm/tree/master/milestones/m3}{here}.

\paragraph{Density decay} 
In a closed fluid system with an unequal density, the density is expected to decay over time. In the LBM this decay is dependent on the collision frequency $\omega$ and the size of the system, where both parameters are expected to be inversely correlated to the time it takes to reach the equilibrium state.

To see if the simulation behaves according to the theoretical expectations, I simulate a sine wave in the densities over the length (x dimension) of the physical space.
Specifically, the initial density distribution is $\rho (\textbf{r},t_{0})=\rho_{0}+\epsilon \sin \left( \frac{2\pi x}{L_{x}} \right)$, where $L_{x}$ is the length of the domain in the x direction, and the initial local average velocity is set to zero, i.e. $\textbf{u}(\textbf{r},0)=0$.
I set $\rho_{0}=0.5$ and use three omega values for the simulation: $\omega=[0.5,1.0,1.7]$.
The density decay for these values is shown in \ref{fig:m3-1}.
\begin{figure}[ht]
\centering
\resizebox{\columnwidth}{!}{\large%% Creator: Matplotlib, PGF backend
%%
%% To include the figure in your LaTeX document, write
%%   \input{<filename>.pgf}
%%
%% Make sure the required packages are loaded in your preamble
%%   \usepackage{pgf}
%%
%% Also ensure that all the required font packages are loaded; for instance,
%% the lmodern package is sometimes necessary when using math font.
%%   \usepackage{lmodern}
%%
%% Figures using additional raster images can only be included by \input if
%% they are in the same directory as the main LaTeX file. For loading figures
%% from other directories you can use the `import` package
%%   \usepackage{import}
%%
%% and then include the figures with
%%   \import{<path to file>}{<filename>.pgf}
%%
%% Matplotlib used the following preamble
%%   \usepackage{fontspec}
%%   \setmainfont{DejaVuSerif.ttf}[Path=\detokenize{/home/joe/miniconda3/envs/high/lib/python3.9/site-packages/matplotlib/mpl-data/fonts/ttf/}]
%%   \setsansfont{DejaVuSans.ttf}[Path=\detokenize{/home/joe/miniconda3/envs/high/lib/python3.9/site-packages/matplotlib/mpl-data/fonts/ttf/}]
%%   \setmonofont{DejaVuSansMono.ttf}[Path=\detokenize{/home/joe/miniconda3/envs/high/lib/python3.9/site-packages/matplotlib/mpl-data/fonts/ttf/}]
%%
\begingroup%
\makeatletter%
\begin{pgfpicture}%
\pgfpathrectangle{\pgfpointorigin}{\pgfqpoint{15.000000in}{5.000000in}}%
\pgfusepath{use as bounding box, clip}%
\begin{pgfscope}%
\pgfsetbuttcap%
\pgfsetmiterjoin%
\pgfsetlinewidth{0.000000pt}%
\definecolor{currentstroke}{rgb}{1.000000,1.000000,1.000000}%
\pgfsetstrokecolor{currentstroke}%
\pgfsetstrokeopacity{0.000000}%
\pgfsetdash{}{0pt}%
\pgfpathmoveto{\pgfqpoint{0.000000in}{0.000000in}}%
\pgfpathlineto{\pgfqpoint{15.000000in}{0.000000in}}%
\pgfpathlineto{\pgfqpoint{15.000000in}{5.000000in}}%
\pgfpathlineto{\pgfqpoint{0.000000in}{5.000000in}}%
\pgfpathlineto{\pgfqpoint{0.000000in}{0.000000in}}%
\pgfpathclose%
\pgfusepath{}%
\end{pgfscope}%
\begin{pgfscope}%
\pgfsetbuttcap%
\pgfsetmiterjoin%
\definecolor{currentfill}{rgb}{1.000000,1.000000,1.000000}%
\pgfsetfillcolor{currentfill}%
\pgfsetlinewidth{0.000000pt}%
\definecolor{currentstroke}{rgb}{0.000000,0.000000,0.000000}%
\pgfsetstrokecolor{currentstroke}%
\pgfsetstrokeopacity{0.000000}%
\pgfsetdash{}{0pt}%
\pgfpathmoveto{\pgfqpoint{1.875000in}{0.625000in}}%
\pgfpathlineto{\pgfqpoint{13.500000in}{0.625000in}}%
\pgfpathlineto{\pgfqpoint{13.500000in}{4.400000in}}%
\pgfpathlineto{\pgfqpoint{1.875000in}{4.400000in}}%
\pgfpathlineto{\pgfqpoint{1.875000in}{0.625000in}}%
\pgfpathclose%
\pgfusepath{fill}%
\end{pgfscope}%
\begin{pgfscope}%
\pgfpathrectangle{\pgfqpoint{1.875000in}{0.625000in}}{\pgfqpoint{11.625000in}{3.775000in}}%
\pgfusepath{clip}%
\pgfsetrectcap%
\pgfsetroundjoin%
\pgfsetlinewidth{0.803000pt}%
\definecolor{currentstroke}{rgb}{0.690196,0.690196,0.690196}%
\pgfsetstrokecolor{currentstroke}%
\pgfsetdash{}{0pt}%
\pgfpathmoveto{\pgfqpoint{2.403409in}{0.625000in}}%
\pgfpathlineto{\pgfqpoint{2.403409in}{4.400000in}}%
\pgfusepath{stroke}%
\end{pgfscope}%
\begin{pgfscope}%
\pgfsetbuttcap%
\pgfsetroundjoin%
\definecolor{currentfill}{rgb}{0.000000,0.000000,0.000000}%
\pgfsetfillcolor{currentfill}%
\pgfsetlinewidth{0.803000pt}%
\definecolor{currentstroke}{rgb}{0.000000,0.000000,0.000000}%
\pgfsetstrokecolor{currentstroke}%
\pgfsetdash{}{0pt}%
\pgfsys@defobject{currentmarker}{\pgfqpoint{0.000000in}{-0.048611in}}{\pgfqpoint{0.000000in}{0.000000in}}{%
\pgfpathmoveto{\pgfqpoint{0.000000in}{0.000000in}}%
\pgfpathlineto{\pgfqpoint{0.000000in}{-0.048611in}}%
\pgfusepath{stroke,fill}%
}%
\begin{pgfscope}%
\pgfsys@transformshift{2.403409in}{0.625000in}%
\pgfsys@useobject{currentmarker}{}%
\end{pgfscope}%
\end{pgfscope}%
\begin{pgfscope}%
\definecolor{textcolor}{rgb}{0.000000,0.000000,0.000000}%
\pgfsetstrokecolor{textcolor}%
\pgfsetfillcolor{textcolor}%
\pgftext[x=2.403409in,y=0.527778in,,top]{\color{textcolor}\sffamily\fontsize{10.000000}{12.000000}\selectfont 0}%
\end{pgfscope}%
\begin{pgfscope}%
\pgfpathrectangle{\pgfqpoint{1.875000in}{0.625000in}}{\pgfqpoint{11.625000in}{3.775000in}}%
\pgfusepath{clip}%
\pgfsetrectcap%
\pgfsetroundjoin%
\pgfsetlinewidth{0.803000pt}%
\definecolor{currentstroke}{rgb}{0.690196,0.690196,0.690196}%
\pgfsetstrokecolor{currentstroke}%
\pgfsetdash{}{0pt}%
\pgfpathmoveto{\pgfqpoint{4.165360in}{0.625000in}}%
\pgfpathlineto{\pgfqpoint{4.165360in}{4.400000in}}%
\pgfusepath{stroke}%
\end{pgfscope}%
\begin{pgfscope}%
\pgfsetbuttcap%
\pgfsetroundjoin%
\definecolor{currentfill}{rgb}{0.000000,0.000000,0.000000}%
\pgfsetfillcolor{currentfill}%
\pgfsetlinewidth{0.803000pt}%
\definecolor{currentstroke}{rgb}{0.000000,0.000000,0.000000}%
\pgfsetstrokecolor{currentstroke}%
\pgfsetdash{}{0pt}%
\pgfsys@defobject{currentmarker}{\pgfqpoint{0.000000in}{-0.048611in}}{\pgfqpoint{0.000000in}{0.000000in}}{%
\pgfpathmoveto{\pgfqpoint{0.000000in}{0.000000in}}%
\pgfpathlineto{\pgfqpoint{0.000000in}{-0.048611in}}%
\pgfusepath{stroke,fill}%
}%
\begin{pgfscope}%
\pgfsys@transformshift{4.165360in}{0.625000in}%
\pgfsys@useobject{currentmarker}{}%
\end{pgfscope}%
\end{pgfscope}%
\begin{pgfscope}%
\definecolor{textcolor}{rgb}{0.000000,0.000000,0.000000}%
\pgfsetstrokecolor{textcolor}%
\pgfsetfillcolor{textcolor}%
\pgftext[x=4.165360in,y=0.527778in,,top]{\color{textcolor}\sffamily\fontsize{10.000000}{12.000000}\selectfont 500}%
\end{pgfscope}%
\begin{pgfscope}%
\pgfpathrectangle{\pgfqpoint{1.875000in}{0.625000in}}{\pgfqpoint{11.625000in}{3.775000in}}%
\pgfusepath{clip}%
\pgfsetrectcap%
\pgfsetroundjoin%
\pgfsetlinewidth{0.803000pt}%
\definecolor{currentstroke}{rgb}{0.690196,0.690196,0.690196}%
\pgfsetstrokecolor{currentstroke}%
\pgfsetdash{}{0pt}%
\pgfpathmoveto{\pgfqpoint{5.927311in}{0.625000in}}%
\pgfpathlineto{\pgfqpoint{5.927311in}{4.400000in}}%
\pgfusepath{stroke}%
\end{pgfscope}%
\begin{pgfscope}%
\pgfsetbuttcap%
\pgfsetroundjoin%
\definecolor{currentfill}{rgb}{0.000000,0.000000,0.000000}%
\pgfsetfillcolor{currentfill}%
\pgfsetlinewidth{0.803000pt}%
\definecolor{currentstroke}{rgb}{0.000000,0.000000,0.000000}%
\pgfsetstrokecolor{currentstroke}%
\pgfsetdash{}{0pt}%
\pgfsys@defobject{currentmarker}{\pgfqpoint{0.000000in}{-0.048611in}}{\pgfqpoint{0.000000in}{0.000000in}}{%
\pgfpathmoveto{\pgfqpoint{0.000000in}{0.000000in}}%
\pgfpathlineto{\pgfqpoint{0.000000in}{-0.048611in}}%
\pgfusepath{stroke,fill}%
}%
\begin{pgfscope}%
\pgfsys@transformshift{5.927311in}{0.625000in}%
\pgfsys@useobject{currentmarker}{}%
\end{pgfscope}%
\end{pgfscope}%
\begin{pgfscope}%
\definecolor{textcolor}{rgb}{0.000000,0.000000,0.000000}%
\pgfsetstrokecolor{textcolor}%
\pgfsetfillcolor{textcolor}%
\pgftext[x=5.927311in,y=0.527778in,,top]{\color{textcolor}\sffamily\fontsize{10.000000}{12.000000}\selectfont 1000}%
\end{pgfscope}%
\begin{pgfscope}%
\pgfpathrectangle{\pgfqpoint{1.875000in}{0.625000in}}{\pgfqpoint{11.625000in}{3.775000in}}%
\pgfusepath{clip}%
\pgfsetrectcap%
\pgfsetroundjoin%
\pgfsetlinewidth{0.803000pt}%
\definecolor{currentstroke}{rgb}{0.690196,0.690196,0.690196}%
\pgfsetstrokecolor{currentstroke}%
\pgfsetdash{}{0pt}%
\pgfpathmoveto{\pgfqpoint{7.689262in}{0.625000in}}%
\pgfpathlineto{\pgfqpoint{7.689262in}{4.400000in}}%
\pgfusepath{stroke}%
\end{pgfscope}%
\begin{pgfscope}%
\pgfsetbuttcap%
\pgfsetroundjoin%
\definecolor{currentfill}{rgb}{0.000000,0.000000,0.000000}%
\pgfsetfillcolor{currentfill}%
\pgfsetlinewidth{0.803000pt}%
\definecolor{currentstroke}{rgb}{0.000000,0.000000,0.000000}%
\pgfsetstrokecolor{currentstroke}%
\pgfsetdash{}{0pt}%
\pgfsys@defobject{currentmarker}{\pgfqpoint{0.000000in}{-0.048611in}}{\pgfqpoint{0.000000in}{0.000000in}}{%
\pgfpathmoveto{\pgfqpoint{0.000000in}{0.000000in}}%
\pgfpathlineto{\pgfqpoint{0.000000in}{-0.048611in}}%
\pgfusepath{stroke,fill}%
}%
\begin{pgfscope}%
\pgfsys@transformshift{7.689262in}{0.625000in}%
\pgfsys@useobject{currentmarker}{}%
\end{pgfscope}%
\end{pgfscope}%
\begin{pgfscope}%
\definecolor{textcolor}{rgb}{0.000000,0.000000,0.000000}%
\pgfsetstrokecolor{textcolor}%
\pgfsetfillcolor{textcolor}%
\pgftext[x=7.689262in,y=0.527778in,,top]{\color{textcolor}\sffamily\fontsize{10.000000}{12.000000}\selectfont 1500}%
\end{pgfscope}%
\begin{pgfscope}%
\pgfpathrectangle{\pgfqpoint{1.875000in}{0.625000in}}{\pgfqpoint{11.625000in}{3.775000in}}%
\pgfusepath{clip}%
\pgfsetrectcap%
\pgfsetroundjoin%
\pgfsetlinewidth{0.803000pt}%
\definecolor{currentstroke}{rgb}{0.690196,0.690196,0.690196}%
\pgfsetstrokecolor{currentstroke}%
\pgfsetdash{}{0pt}%
\pgfpathmoveto{\pgfqpoint{9.451213in}{0.625000in}}%
\pgfpathlineto{\pgfqpoint{9.451213in}{4.400000in}}%
\pgfusepath{stroke}%
\end{pgfscope}%
\begin{pgfscope}%
\pgfsetbuttcap%
\pgfsetroundjoin%
\definecolor{currentfill}{rgb}{0.000000,0.000000,0.000000}%
\pgfsetfillcolor{currentfill}%
\pgfsetlinewidth{0.803000pt}%
\definecolor{currentstroke}{rgb}{0.000000,0.000000,0.000000}%
\pgfsetstrokecolor{currentstroke}%
\pgfsetdash{}{0pt}%
\pgfsys@defobject{currentmarker}{\pgfqpoint{0.000000in}{-0.048611in}}{\pgfqpoint{0.000000in}{0.000000in}}{%
\pgfpathmoveto{\pgfqpoint{0.000000in}{0.000000in}}%
\pgfpathlineto{\pgfqpoint{0.000000in}{-0.048611in}}%
\pgfusepath{stroke,fill}%
}%
\begin{pgfscope}%
\pgfsys@transformshift{9.451213in}{0.625000in}%
\pgfsys@useobject{currentmarker}{}%
\end{pgfscope}%
\end{pgfscope}%
\begin{pgfscope}%
\definecolor{textcolor}{rgb}{0.000000,0.000000,0.000000}%
\pgfsetstrokecolor{textcolor}%
\pgfsetfillcolor{textcolor}%
\pgftext[x=9.451213in,y=0.527778in,,top]{\color{textcolor}\sffamily\fontsize{10.000000}{12.000000}\selectfont 2000}%
\end{pgfscope}%
\begin{pgfscope}%
\pgfpathrectangle{\pgfqpoint{1.875000in}{0.625000in}}{\pgfqpoint{11.625000in}{3.775000in}}%
\pgfusepath{clip}%
\pgfsetrectcap%
\pgfsetroundjoin%
\pgfsetlinewidth{0.803000pt}%
\definecolor{currentstroke}{rgb}{0.690196,0.690196,0.690196}%
\pgfsetstrokecolor{currentstroke}%
\pgfsetdash{}{0pt}%
\pgfpathmoveto{\pgfqpoint{11.213164in}{0.625000in}}%
\pgfpathlineto{\pgfqpoint{11.213164in}{4.400000in}}%
\pgfusepath{stroke}%
\end{pgfscope}%
\begin{pgfscope}%
\pgfsetbuttcap%
\pgfsetroundjoin%
\definecolor{currentfill}{rgb}{0.000000,0.000000,0.000000}%
\pgfsetfillcolor{currentfill}%
\pgfsetlinewidth{0.803000pt}%
\definecolor{currentstroke}{rgb}{0.000000,0.000000,0.000000}%
\pgfsetstrokecolor{currentstroke}%
\pgfsetdash{}{0pt}%
\pgfsys@defobject{currentmarker}{\pgfqpoint{0.000000in}{-0.048611in}}{\pgfqpoint{0.000000in}{0.000000in}}{%
\pgfpathmoveto{\pgfqpoint{0.000000in}{0.000000in}}%
\pgfpathlineto{\pgfqpoint{0.000000in}{-0.048611in}}%
\pgfusepath{stroke,fill}%
}%
\begin{pgfscope}%
\pgfsys@transformshift{11.213164in}{0.625000in}%
\pgfsys@useobject{currentmarker}{}%
\end{pgfscope}%
\end{pgfscope}%
\begin{pgfscope}%
\definecolor{textcolor}{rgb}{0.000000,0.000000,0.000000}%
\pgfsetstrokecolor{textcolor}%
\pgfsetfillcolor{textcolor}%
\pgftext[x=11.213164in,y=0.527778in,,top]{\color{textcolor}\sffamily\fontsize{10.000000}{12.000000}\selectfont 2500}%
\end{pgfscope}%
\begin{pgfscope}%
\pgfpathrectangle{\pgfqpoint{1.875000in}{0.625000in}}{\pgfqpoint{11.625000in}{3.775000in}}%
\pgfusepath{clip}%
\pgfsetrectcap%
\pgfsetroundjoin%
\pgfsetlinewidth{0.803000pt}%
\definecolor{currentstroke}{rgb}{0.690196,0.690196,0.690196}%
\pgfsetstrokecolor{currentstroke}%
\pgfsetdash{}{0pt}%
\pgfpathmoveto{\pgfqpoint{12.975115in}{0.625000in}}%
\pgfpathlineto{\pgfqpoint{12.975115in}{4.400000in}}%
\pgfusepath{stroke}%
\end{pgfscope}%
\begin{pgfscope}%
\pgfsetbuttcap%
\pgfsetroundjoin%
\definecolor{currentfill}{rgb}{0.000000,0.000000,0.000000}%
\pgfsetfillcolor{currentfill}%
\pgfsetlinewidth{0.803000pt}%
\definecolor{currentstroke}{rgb}{0.000000,0.000000,0.000000}%
\pgfsetstrokecolor{currentstroke}%
\pgfsetdash{}{0pt}%
\pgfsys@defobject{currentmarker}{\pgfqpoint{0.000000in}{-0.048611in}}{\pgfqpoint{0.000000in}{0.000000in}}{%
\pgfpathmoveto{\pgfqpoint{0.000000in}{0.000000in}}%
\pgfpathlineto{\pgfqpoint{0.000000in}{-0.048611in}}%
\pgfusepath{stroke,fill}%
}%
\begin{pgfscope}%
\pgfsys@transformshift{12.975115in}{0.625000in}%
\pgfsys@useobject{currentmarker}{}%
\end{pgfscope}%
\end{pgfscope}%
\begin{pgfscope}%
\definecolor{textcolor}{rgb}{0.000000,0.000000,0.000000}%
\pgfsetstrokecolor{textcolor}%
\pgfsetfillcolor{textcolor}%
\pgftext[x=12.975115in,y=0.527778in,,top]{\color{textcolor}\sffamily\fontsize{10.000000}{12.000000}\selectfont 3000}%
\end{pgfscope}%
\begin{pgfscope}%
\definecolor{textcolor}{rgb}{0.000000,0.000000,0.000000}%
\pgfsetstrokecolor{textcolor}%
\pgfsetfillcolor{textcolor}%
\pgftext[x=7.687500in,y=0.337809in,,top]{\color{textcolor}\sffamily\fontsize{10.000000}{12.000000}\selectfont time}%
\end{pgfscope}%
\begin{pgfscope}%
\pgfpathrectangle{\pgfqpoint{1.875000in}{0.625000in}}{\pgfqpoint{11.625000in}{3.775000in}}%
\pgfusepath{clip}%
\pgfsetrectcap%
\pgfsetroundjoin%
\pgfsetlinewidth{0.803000pt}%
\definecolor{currentstroke}{rgb}{0.690196,0.690196,0.690196}%
\pgfsetstrokecolor{currentstroke}%
\pgfsetdash{}{0pt}%
\pgfpathmoveto{\pgfqpoint{1.875000in}{0.714817in}}%
\pgfpathlineto{\pgfqpoint{13.500000in}{0.714817in}}%
\pgfusepath{stroke}%
\end{pgfscope}%
\begin{pgfscope}%
\pgfsetbuttcap%
\pgfsetroundjoin%
\definecolor{currentfill}{rgb}{0.000000,0.000000,0.000000}%
\pgfsetfillcolor{currentfill}%
\pgfsetlinewidth{0.803000pt}%
\definecolor{currentstroke}{rgb}{0.000000,0.000000,0.000000}%
\pgfsetstrokecolor{currentstroke}%
\pgfsetdash{}{0pt}%
\pgfsys@defobject{currentmarker}{\pgfqpoint{-0.048611in}{0.000000in}}{\pgfqpoint{-0.000000in}{0.000000in}}{%
\pgfpathmoveto{\pgfqpoint{-0.000000in}{0.000000in}}%
\pgfpathlineto{\pgfqpoint{-0.048611in}{0.000000in}}%
\pgfusepath{stroke,fill}%
}%
\begin{pgfscope}%
\pgfsys@transformshift{1.875000in}{0.714817in}%
\pgfsys@useobject{currentmarker}{}%
\end{pgfscope}%
\end{pgfscope}%
\begin{pgfscope}%
\definecolor{textcolor}{rgb}{0.000000,0.000000,0.000000}%
\pgfsetstrokecolor{textcolor}%
\pgfsetfillcolor{textcolor}%
\pgftext[x=1.380168in, y=0.662055in, left, base]{\color{textcolor}\sffamily\fontsize{10.000000}{12.000000}\selectfont 0.400}%
\end{pgfscope}%
\begin{pgfscope}%
\pgfpathrectangle{\pgfqpoint{1.875000in}{0.625000in}}{\pgfqpoint{11.625000in}{3.775000in}}%
\pgfusepath{clip}%
\pgfsetrectcap%
\pgfsetroundjoin%
\pgfsetlinewidth{0.803000pt}%
\definecolor{currentstroke}{rgb}{0.690196,0.690196,0.690196}%
\pgfsetstrokecolor{currentstroke}%
\pgfsetdash{}{0pt}%
\pgfpathmoveto{\pgfqpoint{1.875000in}{1.155222in}}%
\pgfpathlineto{\pgfqpoint{13.500000in}{1.155222in}}%
\pgfusepath{stroke}%
\end{pgfscope}%
\begin{pgfscope}%
\pgfsetbuttcap%
\pgfsetroundjoin%
\definecolor{currentfill}{rgb}{0.000000,0.000000,0.000000}%
\pgfsetfillcolor{currentfill}%
\pgfsetlinewidth{0.803000pt}%
\definecolor{currentstroke}{rgb}{0.000000,0.000000,0.000000}%
\pgfsetstrokecolor{currentstroke}%
\pgfsetdash{}{0pt}%
\pgfsys@defobject{currentmarker}{\pgfqpoint{-0.048611in}{0.000000in}}{\pgfqpoint{-0.000000in}{0.000000in}}{%
\pgfpathmoveto{\pgfqpoint{-0.000000in}{0.000000in}}%
\pgfpathlineto{\pgfqpoint{-0.048611in}{0.000000in}}%
\pgfusepath{stroke,fill}%
}%
\begin{pgfscope}%
\pgfsys@transformshift{1.875000in}{1.155222in}%
\pgfsys@useobject{currentmarker}{}%
\end{pgfscope}%
\end{pgfscope}%
\begin{pgfscope}%
\definecolor{textcolor}{rgb}{0.000000,0.000000,0.000000}%
\pgfsetstrokecolor{textcolor}%
\pgfsetfillcolor{textcolor}%
\pgftext[x=1.380168in, y=1.102461in, left, base]{\color{textcolor}\sffamily\fontsize{10.000000}{12.000000}\selectfont 0.425}%
\end{pgfscope}%
\begin{pgfscope}%
\pgfpathrectangle{\pgfqpoint{1.875000in}{0.625000in}}{\pgfqpoint{11.625000in}{3.775000in}}%
\pgfusepath{clip}%
\pgfsetrectcap%
\pgfsetroundjoin%
\pgfsetlinewidth{0.803000pt}%
\definecolor{currentstroke}{rgb}{0.690196,0.690196,0.690196}%
\pgfsetstrokecolor{currentstroke}%
\pgfsetdash{}{0pt}%
\pgfpathmoveto{\pgfqpoint{1.875000in}{1.595628in}}%
\pgfpathlineto{\pgfqpoint{13.500000in}{1.595628in}}%
\pgfusepath{stroke}%
\end{pgfscope}%
\begin{pgfscope}%
\pgfsetbuttcap%
\pgfsetroundjoin%
\definecolor{currentfill}{rgb}{0.000000,0.000000,0.000000}%
\pgfsetfillcolor{currentfill}%
\pgfsetlinewidth{0.803000pt}%
\definecolor{currentstroke}{rgb}{0.000000,0.000000,0.000000}%
\pgfsetstrokecolor{currentstroke}%
\pgfsetdash{}{0pt}%
\pgfsys@defobject{currentmarker}{\pgfqpoint{-0.048611in}{0.000000in}}{\pgfqpoint{-0.000000in}{0.000000in}}{%
\pgfpathmoveto{\pgfqpoint{-0.000000in}{0.000000in}}%
\pgfpathlineto{\pgfqpoint{-0.048611in}{0.000000in}}%
\pgfusepath{stroke,fill}%
}%
\begin{pgfscope}%
\pgfsys@transformshift{1.875000in}{1.595628in}%
\pgfsys@useobject{currentmarker}{}%
\end{pgfscope}%
\end{pgfscope}%
\begin{pgfscope}%
\definecolor{textcolor}{rgb}{0.000000,0.000000,0.000000}%
\pgfsetstrokecolor{textcolor}%
\pgfsetfillcolor{textcolor}%
\pgftext[x=1.380168in, y=1.542866in, left, base]{\color{textcolor}\sffamily\fontsize{10.000000}{12.000000}\selectfont 0.450}%
\end{pgfscope}%
\begin{pgfscope}%
\pgfpathrectangle{\pgfqpoint{1.875000in}{0.625000in}}{\pgfqpoint{11.625000in}{3.775000in}}%
\pgfusepath{clip}%
\pgfsetrectcap%
\pgfsetroundjoin%
\pgfsetlinewidth{0.803000pt}%
\definecolor{currentstroke}{rgb}{0.690196,0.690196,0.690196}%
\pgfsetstrokecolor{currentstroke}%
\pgfsetdash{}{0pt}%
\pgfpathmoveto{\pgfqpoint{1.875000in}{2.036033in}}%
\pgfpathlineto{\pgfqpoint{13.500000in}{2.036033in}}%
\pgfusepath{stroke}%
\end{pgfscope}%
\begin{pgfscope}%
\pgfsetbuttcap%
\pgfsetroundjoin%
\definecolor{currentfill}{rgb}{0.000000,0.000000,0.000000}%
\pgfsetfillcolor{currentfill}%
\pgfsetlinewidth{0.803000pt}%
\definecolor{currentstroke}{rgb}{0.000000,0.000000,0.000000}%
\pgfsetstrokecolor{currentstroke}%
\pgfsetdash{}{0pt}%
\pgfsys@defobject{currentmarker}{\pgfqpoint{-0.048611in}{0.000000in}}{\pgfqpoint{-0.000000in}{0.000000in}}{%
\pgfpathmoveto{\pgfqpoint{-0.000000in}{0.000000in}}%
\pgfpathlineto{\pgfqpoint{-0.048611in}{0.000000in}}%
\pgfusepath{stroke,fill}%
}%
\begin{pgfscope}%
\pgfsys@transformshift{1.875000in}{2.036033in}%
\pgfsys@useobject{currentmarker}{}%
\end{pgfscope}%
\end{pgfscope}%
\begin{pgfscope}%
\definecolor{textcolor}{rgb}{0.000000,0.000000,0.000000}%
\pgfsetstrokecolor{textcolor}%
\pgfsetfillcolor{textcolor}%
\pgftext[x=1.380168in, y=1.983271in, left, base]{\color{textcolor}\sffamily\fontsize{10.000000}{12.000000}\selectfont 0.475}%
\end{pgfscope}%
\begin{pgfscope}%
\pgfpathrectangle{\pgfqpoint{1.875000in}{0.625000in}}{\pgfqpoint{11.625000in}{3.775000in}}%
\pgfusepath{clip}%
\pgfsetrectcap%
\pgfsetroundjoin%
\pgfsetlinewidth{0.803000pt}%
\definecolor{currentstroke}{rgb}{0.690196,0.690196,0.690196}%
\pgfsetstrokecolor{currentstroke}%
\pgfsetdash{}{0pt}%
\pgfpathmoveto{\pgfqpoint{1.875000in}{2.476438in}}%
\pgfpathlineto{\pgfqpoint{13.500000in}{2.476438in}}%
\pgfusepath{stroke}%
\end{pgfscope}%
\begin{pgfscope}%
\pgfsetbuttcap%
\pgfsetroundjoin%
\definecolor{currentfill}{rgb}{0.000000,0.000000,0.000000}%
\pgfsetfillcolor{currentfill}%
\pgfsetlinewidth{0.803000pt}%
\definecolor{currentstroke}{rgb}{0.000000,0.000000,0.000000}%
\pgfsetstrokecolor{currentstroke}%
\pgfsetdash{}{0pt}%
\pgfsys@defobject{currentmarker}{\pgfqpoint{-0.048611in}{0.000000in}}{\pgfqpoint{-0.000000in}{0.000000in}}{%
\pgfpathmoveto{\pgfqpoint{-0.000000in}{0.000000in}}%
\pgfpathlineto{\pgfqpoint{-0.048611in}{0.000000in}}%
\pgfusepath{stroke,fill}%
}%
\begin{pgfscope}%
\pgfsys@transformshift{1.875000in}{2.476438in}%
\pgfsys@useobject{currentmarker}{}%
\end{pgfscope}%
\end{pgfscope}%
\begin{pgfscope}%
\definecolor{textcolor}{rgb}{0.000000,0.000000,0.000000}%
\pgfsetstrokecolor{textcolor}%
\pgfsetfillcolor{textcolor}%
\pgftext[x=1.380168in, y=2.423677in, left, base]{\color{textcolor}\sffamily\fontsize{10.000000}{12.000000}\selectfont 0.500}%
\end{pgfscope}%
\begin{pgfscope}%
\pgfpathrectangle{\pgfqpoint{1.875000in}{0.625000in}}{\pgfqpoint{11.625000in}{3.775000in}}%
\pgfusepath{clip}%
\pgfsetrectcap%
\pgfsetroundjoin%
\pgfsetlinewidth{0.803000pt}%
\definecolor{currentstroke}{rgb}{0.690196,0.690196,0.690196}%
\pgfsetstrokecolor{currentstroke}%
\pgfsetdash{}{0pt}%
\pgfpathmoveto{\pgfqpoint{1.875000in}{2.916843in}}%
\pgfpathlineto{\pgfqpoint{13.500000in}{2.916843in}}%
\pgfusepath{stroke}%
\end{pgfscope}%
\begin{pgfscope}%
\pgfsetbuttcap%
\pgfsetroundjoin%
\definecolor{currentfill}{rgb}{0.000000,0.000000,0.000000}%
\pgfsetfillcolor{currentfill}%
\pgfsetlinewidth{0.803000pt}%
\definecolor{currentstroke}{rgb}{0.000000,0.000000,0.000000}%
\pgfsetstrokecolor{currentstroke}%
\pgfsetdash{}{0pt}%
\pgfsys@defobject{currentmarker}{\pgfqpoint{-0.048611in}{0.000000in}}{\pgfqpoint{-0.000000in}{0.000000in}}{%
\pgfpathmoveto{\pgfqpoint{-0.000000in}{0.000000in}}%
\pgfpathlineto{\pgfqpoint{-0.048611in}{0.000000in}}%
\pgfusepath{stroke,fill}%
}%
\begin{pgfscope}%
\pgfsys@transformshift{1.875000in}{2.916843in}%
\pgfsys@useobject{currentmarker}{}%
\end{pgfscope}%
\end{pgfscope}%
\begin{pgfscope}%
\definecolor{textcolor}{rgb}{0.000000,0.000000,0.000000}%
\pgfsetstrokecolor{textcolor}%
\pgfsetfillcolor{textcolor}%
\pgftext[x=1.380168in, y=2.864082in, left, base]{\color{textcolor}\sffamily\fontsize{10.000000}{12.000000}\selectfont 0.525}%
\end{pgfscope}%
\begin{pgfscope}%
\pgfpathrectangle{\pgfqpoint{1.875000in}{0.625000in}}{\pgfqpoint{11.625000in}{3.775000in}}%
\pgfusepath{clip}%
\pgfsetrectcap%
\pgfsetroundjoin%
\pgfsetlinewidth{0.803000pt}%
\definecolor{currentstroke}{rgb}{0.690196,0.690196,0.690196}%
\pgfsetstrokecolor{currentstroke}%
\pgfsetdash{}{0pt}%
\pgfpathmoveto{\pgfqpoint{1.875000in}{3.357249in}}%
\pgfpathlineto{\pgfqpoint{13.500000in}{3.357249in}}%
\pgfusepath{stroke}%
\end{pgfscope}%
\begin{pgfscope}%
\pgfsetbuttcap%
\pgfsetroundjoin%
\definecolor{currentfill}{rgb}{0.000000,0.000000,0.000000}%
\pgfsetfillcolor{currentfill}%
\pgfsetlinewidth{0.803000pt}%
\definecolor{currentstroke}{rgb}{0.000000,0.000000,0.000000}%
\pgfsetstrokecolor{currentstroke}%
\pgfsetdash{}{0pt}%
\pgfsys@defobject{currentmarker}{\pgfqpoint{-0.048611in}{0.000000in}}{\pgfqpoint{-0.000000in}{0.000000in}}{%
\pgfpathmoveto{\pgfqpoint{-0.000000in}{0.000000in}}%
\pgfpathlineto{\pgfqpoint{-0.048611in}{0.000000in}}%
\pgfusepath{stroke,fill}%
}%
\begin{pgfscope}%
\pgfsys@transformshift{1.875000in}{3.357249in}%
\pgfsys@useobject{currentmarker}{}%
\end{pgfscope}%
\end{pgfscope}%
\begin{pgfscope}%
\definecolor{textcolor}{rgb}{0.000000,0.000000,0.000000}%
\pgfsetstrokecolor{textcolor}%
\pgfsetfillcolor{textcolor}%
\pgftext[x=1.380168in, y=3.304487in, left, base]{\color{textcolor}\sffamily\fontsize{10.000000}{12.000000}\selectfont 0.550}%
\end{pgfscope}%
\begin{pgfscope}%
\pgfpathrectangle{\pgfqpoint{1.875000in}{0.625000in}}{\pgfqpoint{11.625000in}{3.775000in}}%
\pgfusepath{clip}%
\pgfsetrectcap%
\pgfsetroundjoin%
\pgfsetlinewidth{0.803000pt}%
\definecolor{currentstroke}{rgb}{0.690196,0.690196,0.690196}%
\pgfsetstrokecolor{currentstroke}%
\pgfsetdash{}{0pt}%
\pgfpathmoveto{\pgfqpoint{1.875000in}{3.797654in}}%
\pgfpathlineto{\pgfqpoint{13.500000in}{3.797654in}}%
\pgfusepath{stroke}%
\end{pgfscope}%
\begin{pgfscope}%
\pgfsetbuttcap%
\pgfsetroundjoin%
\definecolor{currentfill}{rgb}{0.000000,0.000000,0.000000}%
\pgfsetfillcolor{currentfill}%
\pgfsetlinewidth{0.803000pt}%
\definecolor{currentstroke}{rgb}{0.000000,0.000000,0.000000}%
\pgfsetstrokecolor{currentstroke}%
\pgfsetdash{}{0pt}%
\pgfsys@defobject{currentmarker}{\pgfqpoint{-0.048611in}{0.000000in}}{\pgfqpoint{-0.000000in}{0.000000in}}{%
\pgfpathmoveto{\pgfqpoint{-0.000000in}{0.000000in}}%
\pgfpathlineto{\pgfqpoint{-0.048611in}{0.000000in}}%
\pgfusepath{stroke,fill}%
}%
\begin{pgfscope}%
\pgfsys@transformshift{1.875000in}{3.797654in}%
\pgfsys@useobject{currentmarker}{}%
\end{pgfscope}%
\end{pgfscope}%
\begin{pgfscope}%
\definecolor{textcolor}{rgb}{0.000000,0.000000,0.000000}%
\pgfsetstrokecolor{textcolor}%
\pgfsetfillcolor{textcolor}%
\pgftext[x=1.380168in, y=3.744893in, left, base]{\color{textcolor}\sffamily\fontsize{10.000000}{12.000000}\selectfont 0.575}%
\end{pgfscope}%
\begin{pgfscope}%
\pgfpathrectangle{\pgfqpoint{1.875000in}{0.625000in}}{\pgfqpoint{11.625000in}{3.775000in}}%
\pgfusepath{clip}%
\pgfsetrectcap%
\pgfsetroundjoin%
\pgfsetlinewidth{0.803000pt}%
\definecolor{currentstroke}{rgb}{0.690196,0.690196,0.690196}%
\pgfsetstrokecolor{currentstroke}%
\pgfsetdash{}{0pt}%
\pgfpathmoveto{\pgfqpoint{1.875000in}{4.238059in}}%
\pgfpathlineto{\pgfqpoint{13.500000in}{4.238059in}}%
\pgfusepath{stroke}%
\end{pgfscope}%
\begin{pgfscope}%
\pgfsetbuttcap%
\pgfsetroundjoin%
\definecolor{currentfill}{rgb}{0.000000,0.000000,0.000000}%
\pgfsetfillcolor{currentfill}%
\pgfsetlinewidth{0.803000pt}%
\definecolor{currentstroke}{rgb}{0.000000,0.000000,0.000000}%
\pgfsetstrokecolor{currentstroke}%
\pgfsetdash{}{0pt}%
\pgfsys@defobject{currentmarker}{\pgfqpoint{-0.048611in}{0.000000in}}{\pgfqpoint{-0.000000in}{0.000000in}}{%
\pgfpathmoveto{\pgfqpoint{-0.000000in}{0.000000in}}%
\pgfpathlineto{\pgfqpoint{-0.048611in}{0.000000in}}%
\pgfusepath{stroke,fill}%
}%
\begin{pgfscope}%
\pgfsys@transformshift{1.875000in}{4.238059in}%
\pgfsys@useobject{currentmarker}{}%
\end{pgfscope}%
\end{pgfscope}%
\begin{pgfscope}%
\definecolor{textcolor}{rgb}{0.000000,0.000000,0.000000}%
\pgfsetstrokecolor{textcolor}%
\pgfsetfillcolor{textcolor}%
\pgftext[x=1.380168in, y=4.185298in, left, base]{\color{textcolor}\sffamily\fontsize{10.000000}{12.000000}\selectfont 0.600}%
\end{pgfscope}%
\begin{pgfscope}%
\definecolor{textcolor}{rgb}{0.000000,0.000000,0.000000}%
\pgfsetstrokecolor{textcolor}%
\pgfsetfillcolor{textcolor}%
\pgftext[x=1.324612in,y=2.512500in,,bottom,rotate=90.000000]{\color{textcolor}\sffamily\fontsize{10.000000}{12.000000}\selectfont density}%
\end{pgfscope}%
\begin{pgfscope}%
\pgfpathrectangle{\pgfqpoint{1.875000in}{0.625000in}}{\pgfqpoint{11.625000in}{3.775000in}}%
\pgfusepath{clip}%
\pgfsetrectcap%
\pgfsetroundjoin%
\pgfsetlinewidth{1.505625pt}%
\definecolor{currentstroke}{rgb}{0.121569,0.466667,0.705882}%
\pgfsetstrokecolor{currentstroke}%
\pgfsetdash{}{0pt}%
\pgfpathmoveto{\pgfqpoint{2.403409in}{4.228409in}}%
\pgfpathlineto{\pgfqpoint{2.406933in}{4.215648in}}%
\pgfpathlineto{\pgfqpoint{2.410457in}{4.177858in}}%
\pgfpathlineto{\pgfqpoint{2.417505in}{4.033317in}}%
\pgfpathlineto{\pgfqpoint{2.424553in}{3.812573in}}%
\pgfpathlineto{\pgfqpoint{2.438648in}{3.234825in}}%
\pgfpathlineto{\pgfqpoint{2.463315in}{2.204303in}}%
\pgfpathlineto{\pgfqpoint{2.473887in}{1.871298in}}%
\pgfpathlineto{\pgfqpoint{2.484459in}{1.644024in}}%
\pgfpathlineto{\pgfqpoint{2.491507in}{1.561463in}}%
\pgfpathlineto{\pgfqpoint{2.495031in}{1.542614in}}%
\pgfpathlineto{\pgfqpoint{2.498554in}{1.539164in}}%
\pgfpathlineto{\pgfqpoint{2.502078in}{1.551178in}}%
\pgfpathlineto{\pgfqpoint{2.505602in}{1.578450in}}%
\pgfpathlineto{\pgfqpoint{2.512650in}{1.676319in}}%
\pgfpathlineto{\pgfqpoint{2.519698in}{1.824388in}}%
\pgfpathlineto{\pgfqpoint{2.533793in}{2.212856in}}%
\pgfpathlineto{\pgfqpoint{2.551413in}{2.691495in}}%
\pgfpathlineto{\pgfqpoint{2.561985in}{2.892483in}}%
\pgfpathlineto{\pgfqpoint{2.569032in}{2.977292in}}%
\pgfpathlineto{\pgfqpoint{2.576080in}{3.022425in}}%
\pgfpathlineto{\pgfqpoint{2.579604in}{3.030931in}}%
\pgfpathlineto{\pgfqpoint{2.583128in}{3.030751in}}%
\pgfpathlineto{\pgfqpoint{2.586652in}{3.022504in}}%
\pgfpathlineto{\pgfqpoint{2.593700in}{2.984566in}}%
\pgfpathlineto{\pgfqpoint{2.600748in}{2.922916in}}%
\pgfpathlineto{\pgfqpoint{2.611319in}{2.799023in}}%
\pgfpathlineto{\pgfqpoint{2.650082in}{2.300990in}}%
\pgfpathlineto{\pgfqpoint{2.660654in}{2.219550in}}%
\pgfpathlineto{\pgfqpoint{2.667702in}{2.188336in}}%
\pgfpathlineto{\pgfqpoint{2.671226in}{2.180089in}}%
\pgfpathlineto{\pgfqpoint{2.674750in}{2.176769in}}%
\pgfpathlineto{\pgfqpoint{2.678273in}{2.178292in}}%
\pgfpathlineto{\pgfqpoint{2.681797in}{2.184493in}}%
\pgfpathlineto{\pgfqpoint{2.688845in}{2.209887in}}%
\pgfpathlineto{\pgfqpoint{2.695893in}{2.250154in}}%
\pgfpathlineto{\pgfqpoint{2.706465in}{2.330110in}}%
\pgfpathlineto{\pgfqpoint{2.734656in}{2.555821in}}%
\pgfpathlineto{\pgfqpoint{2.745228in}{2.611395in}}%
\pgfpathlineto{\pgfqpoint{2.752275in}{2.634405in}}%
\pgfpathlineto{\pgfqpoint{2.759323in}{2.645847in}}%
\pgfpathlineto{\pgfqpoint{2.762847in}{2.647374in}}%
\pgfpathlineto{\pgfqpoint{2.766371in}{2.646258in}}%
\pgfpathlineto{\pgfqpoint{2.769895in}{2.642655in}}%
\pgfpathlineto{\pgfqpoint{2.776943in}{2.628741in}}%
\pgfpathlineto{\pgfqpoint{2.783990in}{2.607329in}}%
\pgfpathlineto{\pgfqpoint{2.798086in}{2.549782in}}%
\pgfpathlineto{\pgfqpoint{2.822753in}{2.442856in}}%
\pgfpathlineto{\pgfqpoint{2.833325in}{2.409304in}}%
\pgfpathlineto{\pgfqpoint{2.840373in}{2.393890in}}%
\pgfpathlineto{\pgfqpoint{2.847421in}{2.384648in}}%
\pgfpathlineto{\pgfqpoint{2.854469in}{2.381657in}}%
\pgfpathlineto{\pgfqpoint{2.861516in}{2.384618in}}%
\pgfpathlineto{\pgfqpoint{2.868564in}{2.392883in}}%
\pgfpathlineto{\pgfqpoint{2.879136in}{2.413132in}}%
\pgfpathlineto{\pgfqpoint{2.896755in}{2.457824in}}%
\pgfpathlineto{\pgfqpoint{2.914375in}{2.499796in}}%
\pgfpathlineto{\pgfqpoint{2.924947in}{2.517516in}}%
\pgfpathlineto{\pgfqpoint{2.931994in}{2.525104in}}%
\pgfpathlineto{\pgfqpoint{2.939042in}{2.529119in}}%
\pgfpathlineto{\pgfqpoint{2.946090in}{2.529652in}}%
\pgfpathlineto{\pgfqpoint{2.953138in}{2.526993in}}%
\pgfpathlineto{\pgfqpoint{2.963709in}{2.518029in}}%
\pgfpathlineto{\pgfqpoint{2.977805in}{2.499932in}}%
\pgfpathlineto{\pgfqpoint{3.005996in}{2.461969in}}%
\pgfpathlineto{\pgfqpoint{3.016568in}{2.452617in}}%
\pgfpathlineto{\pgfqpoint{3.027140in}{2.447532in}}%
\pgfpathlineto{\pgfqpoint{3.037711in}{2.446853in}}%
\pgfpathlineto{\pgfqpoint{3.048283in}{2.450125in}}%
\pgfpathlineto{\pgfqpoint{3.062379in}{2.459001in}}%
\pgfpathlineto{\pgfqpoint{3.101142in}{2.487502in}}%
\pgfpathlineto{\pgfqpoint{3.115237in}{2.492382in}}%
\pgfpathlineto{\pgfqpoint{3.125809in}{2.493143in}}%
\pgfpathlineto{\pgfqpoint{3.139905in}{2.490712in}}%
\pgfpathlineto{\pgfqpoint{3.157524in}{2.483945in}}%
\pgfpathlineto{\pgfqpoint{3.189239in}{2.470829in}}%
\pgfpathlineto{\pgfqpoint{3.206859in}{2.467369in}}%
\pgfpathlineto{\pgfqpoint{3.224478in}{2.467683in}}%
\pgfpathlineto{\pgfqpoint{3.245622in}{2.471755in}}%
\pgfpathlineto{\pgfqpoint{3.287908in}{2.480813in}}%
\pgfpathlineto{\pgfqpoint{3.309052in}{2.481608in}}%
\pgfpathlineto{\pgfqpoint{3.333719in}{2.479320in}}%
\pgfpathlineto{\pgfqpoint{3.383054in}{2.473721in}}%
\pgfpathlineto{\pgfqpoint{3.411245in}{2.473996in}}%
\pgfpathlineto{\pgfqpoint{3.499343in}{2.477861in}}%
\pgfpathlineto{\pgfqpoint{3.608584in}{2.476050in}}%
\pgfpathlineto{\pgfqpoint{3.693157in}{2.476730in}}%
\pgfpathlineto{\pgfqpoint{3.820018in}{2.476540in}}%
\pgfpathlineto{\pgfqpoint{4.119549in}{2.476410in}}%
\pgfpathlineto{\pgfqpoint{10.335712in}{2.476438in}}%
\pgfpathlineto{\pgfqpoint{12.971591in}{2.476438in}}%
\pgfpathlineto{\pgfqpoint{12.971591in}{2.476438in}}%
\pgfusepath{stroke}%
\end{pgfscope}%
\begin{pgfscope}%
\pgfpathrectangle{\pgfqpoint{1.875000in}{0.625000in}}{\pgfqpoint{11.625000in}{3.775000in}}%
\pgfusepath{clip}%
\pgfsetrectcap%
\pgfsetroundjoin%
\pgfsetlinewidth{1.505625pt}%
\definecolor{currentstroke}{rgb}{1.000000,0.498039,0.054902}%
\pgfsetstrokecolor{currentstroke}%
\pgfsetdash{}{0pt}%
\pgfpathmoveto{\pgfqpoint{2.403409in}{4.228409in}}%
\pgfpathlineto{\pgfqpoint{2.406933in}{4.215648in}}%
\pgfpathlineto{\pgfqpoint{2.410457in}{4.177795in}}%
\pgfpathlineto{\pgfqpoint{2.417505in}{4.031931in}}%
\pgfpathlineto{\pgfqpoint{2.424553in}{3.805741in}}%
\pgfpathlineto{\pgfqpoint{2.435124in}{3.360339in}}%
\pgfpathlineto{\pgfqpoint{2.470363in}{1.738712in}}%
\pgfpathlineto{\pgfqpoint{2.480935in}{1.385451in}}%
\pgfpathlineto{\pgfqpoint{2.487983in}{1.213051in}}%
\pgfpathlineto{\pgfqpoint{2.495031in}{1.102392in}}%
\pgfpathlineto{\pgfqpoint{2.498554in}{1.073843in}}%
\pgfpathlineto{\pgfqpoint{2.502078in}{1.065344in}}%
\pgfpathlineto{\pgfqpoint{2.505602in}{1.078634in}}%
\pgfpathlineto{\pgfqpoint{2.509126in}{1.115405in}}%
\pgfpathlineto{\pgfqpoint{2.516174in}{1.264849in}}%
\pgfpathlineto{\pgfqpoint{2.523222in}{1.518376in}}%
\pgfpathlineto{\pgfqpoint{2.533793in}{2.063094in}}%
\pgfpathlineto{\pgfqpoint{2.554937in}{3.227829in}}%
\pgfpathlineto{\pgfqpoint{2.561985in}{3.480818in}}%
\pgfpathlineto{\pgfqpoint{2.569032in}{3.635570in}}%
\pgfpathlineto{\pgfqpoint{2.572556in}{3.677607in}}%
\pgfpathlineto{\pgfqpoint{2.576080in}{3.698113in}}%
\pgfpathlineto{\pgfqpoint{2.579604in}{3.699014in}}%
\pgfpathlineto{\pgfqpoint{2.583128in}{3.682374in}}%
\pgfpathlineto{\pgfqpoint{2.590176in}{3.604743in}}%
\pgfpathlineto{\pgfqpoint{2.597224in}{3.480876in}}%
\pgfpathlineto{\pgfqpoint{2.607795in}{3.237289in}}%
\pgfpathlineto{\pgfqpoint{2.632463in}{2.560871in}}%
\pgfpathlineto{\pgfqpoint{2.653606in}{2.011353in}}%
\pgfpathlineto{\pgfqpoint{2.667702in}{1.723150in}}%
\pgfpathlineto{\pgfqpoint{2.674750in}{1.619670in}}%
\pgfpathlineto{\pgfqpoint{2.681797in}{1.555124in}}%
\pgfpathlineto{\pgfqpoint{2.685321in}{1.541079in}}%
\pgfpathlineto{\pgfqpoint{2.688845in}{1.541346in}}%
\pgfpathlineto{\pgfqpoint{2.692369in}{1.557546in}}%
\pgfpathlineto{\pgfqpoint{2.695893in}{1.591145in}}%
\pgfpathlineto{\pgfqpoint{2.702941in}{1.714390in}}%
\pgfpathlineto{\pgfqpoint{2.709989in}{1.911610in}}%
\pgfpathlineto{\pgfqpoint{2.724084in}{2.453698in}}%
\pgfpathlineto{\pgfqpoint{2.738180in}{2.966305in}}%
\pgfpathlineto{\pgfqpoint{2.745228in}{3.139987in}}%
\pgfpathlineto{\pgfqpoint{2.752275in}{3.245898in}}%
\pgfpathlineto{\pgfqpoint{2.755799in}{3.274596in}}%
\pgfpathlineto{\pgfqpoint{2.759323in}{3.288594in}}%
\pgfpathlineto{\pgfqpoint{2.762847in}{3.289266in}}%
\pgfpathlineto{\pgfqpoint{2.766371in}{3.278068in}}%
\pgfpathlineto{\pgfqpoint{2.773419in}{3.225805in}}%
\pgfpathlineto{\pgfqpoint{2.780467in}{3.142532in}}%
\pgfpathlineto{\pgfqpoint{2.791038in}{2.978813in}}%
\pgfpathlineto{\pgfqpoint{2.812182in}{2.588183in}}%
\pgfpathlineto{\pgfqpoint{2.836849in}{2.146493in}}%
\pgfpathlineto{\pgfqpoint{2.847421in}{1.995435in}}%
\pgfpathlineto{\pgfqpoint{2.857992in}{1.888977in}}%
\pgfpathlineto{\pgfqpoint{2.865040in}{1.853182in}}%
\pgfpathlineto{\pgfqpoint{2.868564in}{1.848487in}}%
\pgfpathlineto{\pgfqpoint{2.872088in}{1.853743in}}%
\pgfpathlineto{\pgfqpoint{2.875612in}{1.869680in}}%
\pgfpathlineto{\pgfqpoint{2.882660in}{1.935441in}}%
\pgfpathlineto{\pgfqpoint{2.889708in}{2.046063in}}%
\pgfpathlineto{\pgfqpoint{2.900279in}{2.279538in}}%
\pgfpathlineto{\pgfqpoint{2.921423in}{2.782140in}}%
\pgfpathlineto{\pgfqpoint{2.928470in}{2.898416in}}%
\pgfpathlineto{\pgfqpoint{2.935518in}{2.974650in}}%
\pgfpathlineto{\pgfqpoint{2.942566in}{3.011223in}}%
\pgfpathlineto{\pgfqpoint{2.946090in}{3.015839in}}%
\pgfpathlineto{\pgfqpoint{2.949614in}{3.012269in}}%
\pgfpathlineto{\pgfqpoint{2.953138in}{3.001292in}}%
\pgfpathlineto{\pgfqpoint{2.960186in}{2.960299in}}%
\pgfpathlineto{\pgfqpoint{2.970757in}{2.862498in}}%
\pgfpathlineto{\pgfqpoint{2.984853in}{2.691138in}}%
\pgfpathlineto{\pgfqpoint{3.020092in}{2.237019in}}%
\pgfpathlineto{\pgfqpoint{3.030664in}{2.135107in}}%
\pgfpathlineto{\pgfqpoint{3.037711in}{2.086420in}}%
\pgfpathlineto{\pgfqpoint{3.044759in}{2.057753in}}%
\pgfpathlineto{\pgfqpoint{3.048283in}{2.052148in}}%
\pgfpathlineto{\pgfqpoint{3.051807in}{2.052940in}}%
\pgfpathlineto{\pgfqpoint{3.055331in}{2.060467in}}%
\pgfpathlineto{\pgfqpoint{3.058855in}{2.074951in}}%
\pgfpathlineto{\pgfqpoint{3.065903in}{2.124909in}}%
\pgfpathlineto{\pgfqpoint{3.072950in}{2.201083in}}%
\pgfpathlineto{\pgfqpoint{3.083522in}{2.352265in}}%
\pgfpathlineto{\pgfqpoint{3.104666in}{2.671185in}}%
\pgfpathlineto{\pgfqpoint{3.115237in}{2.778470in}}%
\pgfpathlineto{\pgfqpoint{3.122285in}{2.819805in}}%
\pgfpathlineto{\pgfqpoint{3.125809in}{2.831280in}}%
\pgfpathlineto{\pgfqpoint{3.129333in}{2.836889in}}%
\pgfpathlineto{\pgfqpoint{3.132857in}{2.836947in}}%
\pgfpathlineto{\pgfqpoint{3.136381in}{2.831835in}}%
\pgfpathlineto{\pgfqpoint{3.143428in}{2.807811in}}%
\pgfpathlineto{\pgfqpoint{3.150476in}{2.768397in}}%
\pgfpathlineto{\pgfqpoint{3.161048in}{2.688178in}}%
\pgfpathlineto{\pgfqpoint{3.182191in}{2.490607in}}%
\pgfpathlineto{\pgfqpoint{3.199811in}{2.331152in}}%
\pgfpathlineto{\pgfqpoint{3.210383in}{2.255223in}}%
\pgfpathlineto{\pgfqpoint{3.217430in}{2.218134in}}%
\pgfpathlineto{\pgfqpoint{3.224478in}{2.194790in}}%
\pgfpathlineto{\pgfqpoint{3.228002in}{2.188977in}}%
\pgfpathlineto{\pgfqpoint{3.231526in}{2.187376in}}%
\pgfpathlineto{\pgfqpoint{3.235050in}{2.190156in}}%
\pgfpathlineto{\pgfqpoint{3.238574in}{2.197415in}}%
\pgfpathlineto{\pgfqpoint{3.245622in}{2.225341in}}%
\pgfpathlineto{\pgfqpoint{3.252669in}{2.270093in}}%
\pgfpathlineto{\pgfqpoint{3.263241in}{2.362219in}}%
\pgfpathlineto{\pgfqpoint{3.291432in}{2.631087in}}%
\pgfpathlineto{\pgfqpoint{3.298480in}{2.675721in}}%
\pgfpathlineto{\pgfqpoint{3.305528in}{2.705175in}}%
\pgfpathlineto{\pgfqpoint{3.312576in}{2.718995in}}%
\pgfpathlineto{\pgfqpoint{3.316100in}{2.720237in}}%
\pgfpathlineto{\pgfqpoint{3.319624in}{2.717921in}}%
\pgfpathlineto{\pgfqpoint{3.323147in}{2.712265in}}%
\pgfpathlineto{\pgfqpoint{3.330195in}{2.691950in}}%
\pgfpathlineto{\pgfqpoint{3.337243in}{2.661484in}}%
\pgfpathlineto{\pgfqpoint{3.347815in}{2.601824in}}%
\pgfpathlineto{\pgfqpoint{3.390102in}{2.335668in}}%
\pgfpathlineto{\pgfqpoint{3.400673in}{2.296231in}}%
\pgfpathlineto{\pgfqpoint{3.407721in}{2.282243in}}%
\pgfpathlineto{\pgfqpoint{3.411245in}{2.279411in}}%
\pgfpathlineto{\pgfqpoint{3.414769in}{2.279493in}}%
\pgfpathlineto{\pgfqpoint{3.418293in}{2.282540in}}%
\pgfpathlineto{\pgfqpoint{3.425341in}{2.297490in}}%
\pgfpathlineto{\pgfqpoint{3.432388in}{2.323628in}}%
\pgfpathlineto{\pgfqpoint{3.442960in}{2.380123in}}%
\pgfpathlineto{\pgfqpoint{3.478199in}{2.595678in}}%
\pgfpathlineto{\pgfqpoint{3.485247in}{2.621183in}}%
\pgfpathlineto{\pgfqpoint{3.492295in}{2.636648in}}%
\pgfpathlineto{\pgfqpoint{3.495819in}{2.640536in}}%
\pgfpathlineto{\pgfqpoint{3.499343in}{2.641908in}}%
\pgfpathlineto{\pgfqpoint{3.502866in}{2.640848in}}%
\pgfpathlineto{\pgfqpoint{3.506390in}{2.637470in}}%
\pgfpathlineto{\pgfqpoint{3.513438in}{2.624338in}}%
\pgfpathlineto{\pgfqpoint{3.520486in}{2.603856in}}%
\pgfpathlineto{\pgfqpoint{3.531058in}{2.562738in}}%
\pgfpathlineto{\pgfqpoint{3.573345in}{2.377389in}}%
\pgfpathlineto{\pgfqpoint{3.580392in}{2.358405in}}%
\pgfpathlineto{\pgfqpoint{3.587440in}{2.346212in}}%
\pgfpathlineto{\pgfqpoint{3.594488in}{2.341582in}}%
\pgfpathlineto{\pgfqpoint{3.598012in}{2.342243in}}%
\pgfpathlineto{\pgfqpoint{3.601536in}{2.344904in}}%
\pgfpathlineto{\pgfqpoint{3.608584in}{2.356086in}}%
\pgfpathlineto{\pgfqpoint{3.615631in}{2.374502in}}%
\pgfpathlineto{\pgfqpoint{3.626203in}{2.412986in}}%
\pgfpathlineto{\pgfqpoint{3.661442in}{2.556817in}}%
\pgfpathlineto{\pgfqpoint{3.668490in}{2.574214in}}%
\pgfpathlineto{\pgfqpoint{3.675538in}{2.585026in}}%
\pgfpathlineto{\pgfqpoint{3.682585in}{2.589019in}}%
\pgfpathlineto{\pgfqpoint{3.686109in}{2.588506in}}%
\pgfpathlineto{\pgfqpoint{3.693157in}{2.582768in}}%
\pgfpathlineto{\pgfqpoint{3.700205in}{2.571370in}}%
\pgfpathlineto{\pgfqpoint{3.710777in}{2.545759in}}%
\pgfpathlineto{\pgfqpoint{3.728396in}{2.489856in}}%
\pgfpathlineto{\pgfqpoint{3.749540in}{2.424006in}}%
\pgfpathlineto{\pgfqpoint{3.760111in}{2.400335in}}%
\pgfpathlineto{\pgfqpoint{3.767159in}{2.390036in}}%
\pgfpathlineto{\pgfqpoint{3.774207in}{2.384819in}}%
\pgfpathlineto{\pgfqpoint{3.781255in}{2.384987in}}%
\pgfpathlineto{\pgfqpoint{3.788303in}{2.390535in}}%
\pgfpathlineto{\pgfqpoint{3.795350in}{2.401122in}}%
\pgfpathlineto{\pgfqpoint{3.805922in}{2.424870in}}%
\pgfpathlineto{\pgfqpoint{3.823542in}{2.476045in}}%
\pgfpathlineto{\pgfqpoint{3.841161in}{2.523614in}}%
\pgfpathlineto{\pgfqpoint{3.851733in}{2.542842in}}%
\pgfpathlineto{\pgfqpoint{3.858781in}{2.550293in}}%
\pgfpathlineto{\pgfqpoint{3.865828in}{2.553155in}}%
\pgfpathlineto{\pgfqpoint{3.872876in}{2.551507in}}%
\pgfpathlineto{\pgfqpoint{3.879924in}{2.545679in}}%
\pgfpathlineto{\pgfqpoint{3.890496in}{2.530309in}}%
\pgfpathlineto{\pgfqpoint{3.904591in}{2.501316in}}%
\pgfpathlineto{\pgfqpoint{3.932783in}{2.439691in}}%
\pgfpathlineto{\pgfqpoint{3.943354in}{2.423690in}}%
\pgfpathlineto{\pgfqpoint{3.950402in}{2.416904in}}%
\pgfpathlineto{\pgfqpoint{3.957450in}{2.413658in}}%
\pgfpathlineto{\pgfqpoint{3.964498in}{2.414108in}}%
\pgfpathlineto{\pgfqpoint{3.971545in}{2.418196in}}%
\pgfpathlineto{\pgfqpoint{3.978593in}{2.425641in}}%
\pgfpathlineto{\pgfqpoint{3.989165in}{2.441967in}}%
\pgfpathlineto{\pgfqpoint{4.010308in}{2.483657in}}%
\pgfpathlineto{\pgfqpoint{4.024404in}{2.508671in}}%
\pgfpathlineto{\pgfqpoint{4.034976in}{2.521699in}}%
\pgfpathlineto{\pgfqpoint{4.042023in}{2.526786in}}%
\pgfpathlineto{\pgfqpoint{4.049071in}{2.528770in}}%
\pgfpathlineto{\pgfqpoint{4.056119in}{2.527677in}}%
\pgfpathlineto{\pgfqpoint{4.063167in}{2.523710in}}%
\pgfpathlineto{\pgfqpoint{4.073739in}{2.513162in}}%
\pgfpathlineto{\pgfqpoint{4.087834in}{2.493166in}}%
\pgfpathlineto{\pgfqpoint{4.116025in}{2.450799in}}%
\pgfpathlineto{\pgfqpoint{4.126597in}{2.439987in}}%
\pgfpathlineto{\pgfqpoint{4.137169in}{2.434164in}}%
\pgfpathlineto{\pgfqpoint{4.144217in}{2.433390in}}%
\pgfpathlineto{\pgfqpoint{4.151264in}{2.435131in}}%
\pgfpathlineto{\pgfqpoint{4.161836in}{2.442112in}}%
\pgfpathlineto{\pgfqpoint{4.172408in}{2.453335in}}%
\pgfpathlineto{\pgfqpoint{4.214695in}{2.504880in}}%
\pgfpathlineto{\pgfqpoint{4.225266in}{2.510813in}}%
\pgfpathlineto{\pgfqpoint{4.235838in}{2.512038in}}%
\pgfpathlineto{\pgfqpoint{4.246410in}{2.508680in}}%
\pgfpathlineto{\pgfqpoint{4.256981in}{2.501425in}}%
\pgfpathlineto{\pgfqpoint{4.271077in}{2.487655in}}%
\pgfpathlineto{\pgfqpoint{4.299268in}{2.458607in}}%
\pgfpathlineto{\pgfqpoint{4.309840in}{2.451302in}}%
\pgfpathlineto{\pgfqpoint{4.320412in}{2.447463in}}%
\pgfpathlineto{\pgfqpoint{4.330983in}{2.447475in}}%
\pgfpathlineto{\pgfqpoint{4.341555in}{2.451224in}}%
\pgfpathlineto{\pgfqpoint{4.355651in}{2.460927in}}%
\pgfpathlineto{\pgfqpoint{4.397938in}{2.495921in}}%
\pgfpathlineto{\pgfqpoint{4.408509in}{2.499931in}}%
\pgfpathlineto{\pgfqpoint{4.419081in}{2.500737in}}%
\pgfpathlineto{\pgfqpoint{4.429653in}{2.498412in}}%
\pgfpathlineto{\pgfqpoint{4.443748in}{2.491281in}}%
\pgfpathlineto{\pgfqpoint{4.468416in}{2.473371in}}%
\pgfpathlineto{\pgfqpoint{4.486035in}{2.462181in}}%
\pgfpathlineto{\pgfqpoint{4.500131in}{2.457149in}}%
\pgfpathlineto{\pgfqpoint{4.510702in}{2.456365in}}%
\pgfpathlineto{\pgfqpoint{4.521274in}{2.458191in}}%
\pgfpathlineto{\pgfqpoint{4.535370in}{2.464077in}}%
\pgfpathlineto{\pgfqpoint{4.563561in}{2.481251in}}%
\pgfpathlineto{\pgfqpoint{4.581180in}{2.489800in}}%
\pgfpathlineto{\pgfqpoint{4.595276in}{2.492924in}}%
\pgfpathlineto{\pgfqpoint{4.609372in}{2.492169in}}%
\pgfpathlineto{\pgfqpoint{4.623467in}{2.487961in}}%
\pgfpathlineto{\pgfqpoint{4.644611in}{2.477816in}}%
\pgfpathlineto{\pgfqpoint{4.669278in}{2.466589in}}%
\pgfpathlineto{\pgfqpoint{4.683374in}{2.463213in}}%
\pgfpathlineto{\pgfqpoint{4.697469in}{2.462977in}}%
\pgfpathlineto{\pgfqpoint{4.711565in}{2.465781in}}%
\pgfpathlineto{\pgfqpoint{4.732708in}{2.473824in}}%
\pgfpathlineto{\pgfqpoint{4.760899in}{2.484695in}}%
\pgfpathlineto{\pgfqpoint{4.778519in}{2.487707in}}%
\pgfpathlineto{\pgfqpoint{4.792615in}{2.487159in}}%
\pgfpathlineto{\pgfqpoint{4.810234in}{2.483243in}}%
\pgfpathlineto{\pgfqpoint{4.863093in}{2.467748in}}%
\pgfpathlineto{\pgfqpoint{4.880712in}{2.467262in}}%
\pgfpathlineto{\pgfqpoint{4.898332in}{2.469971in}}%
\pgfpathlineto{\pgfqpoint{4.961762in}{2.484143in}}%
\pgfpathlineto{\pgfqpoint{4.979381in}{2.483375in}}%
\pgfpathlineto{\pgfqpoint{5.004049in}{2.478666in}}%
\pgfpathlineto{\pgfqpoint{5.042812in}{2.470815in}}%
\pgfpathlineto{\pgfqpoint{5.063955in}{2.470182in}}%
\pgfpathlineto{\pgfqpoint{5.085098in}{2.472639in}}%
\pgfpathlineto{\pgfqpoint{5.145005in}{2.481706in}}%
\pgfpathlineto{\pgfqpoint{5.169672in}{2.480460in}}%
\pgfpathlineto{\pgfqpoint{5.247198in}{2.472174in}}%
\pgfpathlineto{\pgfqpoint{5.275389in}{2.474755in}}%
\pgfpathlineto{\pgfqpoint{5.324724in}{2.479961in}}%
\pgfpathlineto{\pgfqpoint{5.352915in}{2.479167in}}%
\pgfpathlineto{\pgfqpoint{5.433965in}{2.473639in}}%
\pgfpathlineto{\pgfqpoint{5.476252in}{2.476946in}}%
\pgfpathlineto{\pgfqpoint{5.515014in}{2.478916in}}%
\pgfpathlineto{\pgfqpoint{5.550253in}{2.477359in}}%
\pgfpathlineto{\pgfqpoint{5.606636in}{2.474391in}}%
\pgfpathlineto{\pgfqpoint{5.648923in}{2.476130in}}%
\pgfpathlineto{\pgfqpoint{5.701781in}{2.478114in}}%
\pgfpathlineto{\pgfqpoint{5.751116in}{2.476134in}}%
\pgfpathlineto{\pgfqpoint{5.800451in}{2.475141in}}%
\pgfpathlineto{\pgfqpoint{5.930835in}{2.476345in}}%
\pgfpathlineto{\pgfqpoint{5.990741in}{2.475665in}}%
\pgfpathlineto{\pgfqpoint{6.107030in}{2.476542in}}%
\pgfpathlineto{\pgfqpoint{6.184556in}{2.476079in}}%
\pgfpathlineto{\pgfqpoint{6.290273in}{2.476505in}}%
\pgfpathlineto{\pgfqpoint{6.385418in}{2.476434in}}%
\pgfpathlineto{\pgfqpoint{6.487611in}{2.476324in}}%
\pgfpathlineto{\pgfqpoint{6.603900in}{2.476678in}}%
\pgfpathlineto{\pgfqpoint{7.213535in}{2.476427in}}%
\pgfpathlineto{\pgfqpoint{7.854885in}{2.476442in}}%
\pgfpathlineto{\pgfqpoint{10.332188in}{2.476438in}}%
\pgfpathlineto{\pgfqpoint{12.971591in}{2.476438in}}%
\pgfpathlineto{\pgfqpoint{12.971591in}{2.476438in}}%
\pgfusepath{stroke}%
\end{pgfscope}%
\begin{pgfscope}%
\pgfpathrectangle{\pgfqpoint{1.875000in}{0.625000in}}{\pgfqpoint{11.625000in}{3.775000in}}%
\pgfusepath{clip}%
\pgfsetrectcap%
\pgfsetroundjoin%
\pgfsetlinewidth{1.505625pt}%
\definecolor{currentstroke}{rgb}{0.172549,0.627451,0.172549}%
\pgfsetstrokecolor{currentstroke}%
\pgfsetdash{}{0pt}%
\pgfpathmoveto{\pgfqpoint{2.403409in}{4.228409in}}%
\pgfpathlineto{\pgfqpoint{2.406933in}{4.215648in}}%
\pgfpathlineto{\pgfqpoint{2.410457in}{4.177707in}}%
\pgfpathlineto{\pgfqpoint{2.417505in}{4.030613in}}%
\pgfpathlineto{\pgfqpoint{2.424553in}{3.800647in}}%
\pgfpathlineto{\pgfqpoint{2.435124in}{3.342970in}}%
\pgfpathlineto{\pgfqpoint{2.477411in}{1.353601in}}%
\pgfpathlineto{\pgfqpoint{2.487983in}{1.027616in}}%
\pgfpathlineto{\pgfqpoint{2.495031in}{0.879877in}}%
\pgfpathlineto{\pgfqpoint{2.502078in}{0.802724in}}%
\pgfpathlineto{\pgfqpoint{2.505602in}{0.796591in}}%
\pgfpathlineto{\pgfqpoint{2.509126in}{0.816431in}}%
\pgfpathlineto{\pgfqpoint{2.512650in}{0.866337in}}%
\pgfpathlineto{\pgfqpoint{2.516174in}{0.950967in}}%
\pgfpathlineto{\pgfqpoint{2.523222in}{1.243147in}}%
\pgfpathlineto{\pgfqpoint{2.530270in}{1.715863in}}%
\pgfpathlineto{\pgfqpoint{2.554937in}{3.734588in}}%
\pgfpathlineto{\pgfqpoint{2.561985in}{4.007714in}}%
\pgfpathlineto{\pgfqpoint{2.565509in}{4.083162in}}%
\pgfpathlineto{\pgfqpoint{2.569032in}{4.124524in}}%
\pgfpathlineto{\pgfqpoint{2.572556in}{4.136611in}}%
\pgfpathlineto{\pgfqpoint{2.576080in}{4.123768in}}%
\pgfpathlineto{\pgfqpoint{2.579604in}{4.089851in}}%
\pgfpathlineto{\pgfqpoint{2.586652in}{3.971899in}}%
\pgfpathlineto{\pgfqpoint{2.597224in}{3.708161in}}%
\pgfpathlineto{\pgfqpoint{2.614843in}{3.153053in}}%
\pgfpathlineto{\pgfqpoint{2.653606in}{1.901483in}}%
\pgfpathlineto{\pgfqpoint{2.671226in}{1.428131in}}%
\pgfpathlineto{\pgfqpoint{2.685321in}{1.126581in}}%
\pgfpathlineto{\pgfqpoint{2.692369in}{1.016479in}}%
\pgfpathlineto{\pgfqpoint{2.699417in}{0.956051in}}%
\pgfpathlineto{\pgfqpoint{2.702941in}{0.958491in}}%
\pgfpathlineto{\pgfqpoint{2.706465in}{0.998076in}}%
\pgfpathlineto{\pgfqpoint{2.709989in}{1.094824in}}%
\pgfpathlineto{\pgfqpoint{2.713512in}{1.276174in}}%
\pgfpathlineto{\pgfqpoint{2.720560in}{1.989915in}}%
\pgfpathlineto{\pgfqpoint{2.734656in}{3.865475in}}%
\pgfpathlineto{\pgfqpoint{2.738180in}{4.042549in}}%
\pgfpathlineto{\pgfqpoint{2.741704in}{4.094391in}}%
\pgfpathlineto{\pgfqpoint{2.745228in}{4.076570in}}%
\pgfpathlineto{\pgfqpoint{2.755799in}{3.935858in}}%
\pgfpathlineto{\pgfqpoint{2.762847in}{3.813500in}}%
\pgfpathlineto{\pgfqpoint{2.776943in}{3.496595in}}%
\pgfpathlineto{\pgfqpoint{2.850945in}{1.733267in}}%
\pgfpathlineto{\pgfqpoint{2.868564in}{1.397715in}}%
\pgfpathlineto{\pgfqpoint{2.879136in}{1.232425in}}%
\pgfpathlineto{\pgfqpoint{2.886184in}{1.159886in}}%
\pgfpathlineto{\pgfqpoint{2.889708in}{1.151434in}}%
\pgfpathlineto{\pgfqpoint{2.893231in}{1.181034in}}%
\pgfpathlineto{\pgfqpoint{2.896755in}{1.276442in}}%
\pgfpathlineto{\pgfqpoint{2.900279in}{1.476663in}}%
\pgfpathlineto{\pgfqpoint{2.907327in}{2.308570in}}%
\pgfpathlineto{\pgfqpoint{2.917899in}{3.838183in}}%
\pgfpathlineto{\pgfqpoint{2.921423in}{4.031711in}}%
\pgfpathlineto{\pgfqpoint{2.924947in}{4.042856in}}%
\pgfpathlineto{\pgfqpoint{2.942566in}{3.620925in}}%
\pgfpathlineto{\pgfqpoint{2.953138in}{3.433330in}}%
\pgfpathlineto{\pgfqpoint{3.009520in}{2.309745in}}%
\pgfpathlineto{\pgfqpoint{3.037711in}{1.809208in}}%
\pgfpathlineto{\pgfqpoint{3.058855in}{1.482496in}}%
\pgfpathlineto{\pgfqpoint{3.065903in}{1.396440in}}%
\pgfpathlineto{\pgfqpoint{3.069427in}{1.366301in}}%
\pgfpathlineto{\pgfqpoint{3.072950in}{1.353669in}}%
\pgfpathlineto{\pgfqpoint{3.076474in}{1.372029in}}%
\pgfpathlineto{\pgfqpoint{3.079998in}{1.443549in}}%
\pgfpathlineto{\pgfqpoint{3.083522in}{1.599260in}}%
\pgfpathlineto{\pgfqpoint{3.087046in}{1.870465in}}%
\pgfpathlineto{\pgfqpoint{3.094094in}{2.750280in}}%
\pgfpathlineto{\pgfqpoint{3.101142in}{3.610609in}}%
\pgfpathlineto{\pgfqpoint{3.104666in}{3.817254in}}%
\pgfpathlineto{\pgfqpoint{3.108189in}{3.854704in}}%
\pgfpathlineto{\pgfqpoint{3.111713in}{3.780374in}}%
\pgfpathlineto{\pgfqpoint{3.122285in}{3.481013in}}%
\pgfpathlineto{\pgfqpoint{3.196287in}{2.277117in}}%
\pgfpathlineto{\pgfqpoint{3.228002in}{1.811968in}}%
\pgfpathlineto{\pgfqpoint{3.245622in}{1.590646in}}%
\pgfpathlineto{\pgfqpoint{3.252669in}{1.533795in}}%
\pgfpathlineto{\pgfqpoint{3.256193in}{1.526882in}}%
\pgfpathlineto{\pgfqpoint{3.259717in}{1.547950in}}%
\pgfpathlineto{\pgfqpoint{3.263241in}{1.614252in}}%
\pgfpathlineto{\pgfqpoint{3.266765in}{1.748159in}}%
\pgfpathlineto{\pgfqpoint{3.273813in}{2.287061in}}%
\pgfpathlineto{\pgfqpoint{3.284385in}{3.382440in}}%
\pgfpathlineto{\pgfqpoint{3.287908in}{3.580209in}}%
\pgfpathlineto{\pgfqpoint{3.291432in}{3.644552in}}%
\pgfpathlineto{\pgfqpoint{3.294956in}{3.608058in}}%
\pgfpathlineto{\pgfqpoint{3.309052in}{3.285482in}}%
\pgfpathlineto{\pgfqpoint{3.383054in}{2.254762in}}%
\pgfpathlineto{\pgfqpoint{3.414769in}{1.857516in}}%
\pgfpathlineto{\pgfqpoint{3.428865in}{1.709862in}}%
\pgfpathlineto{\pgfqpoint{3.435912in}{1.667507in}}%
\pgfpathlineto{\pgfqpoint{3.439436in}{1.667517in}}%
\pgfpathlineto{\pgfqpoint{3.442960in}{1.693144in}}%
\pgfpathlineto{\pgfqpoint{3.446484in}{1.757561in}}%
\pgfpathlineto{\pgfqpoint{3.450008in}{1.876382in}}%
\pgfpathlineto{\pgfqpoint{3.457056in}{2.318037in}}%
\pgfpathlineto{\pgfqpoint{3.467627in}{3.200187in}}%
\pgfpathlineto{\pgfqpoint{3.471151in}{3.379086in}}%
\pgfpathlineto{\pgfqpoint{3.474675in}{3.457634in}}%
\pgfpathlineto{\pgfqpoint{3.478199in}{3.452276in}}%
\pgfpathlineto{\pgfqpoint{3.485247in}{3.321573in}}%
\pgfpathlineto{\pgfqpoint{3.492295in}{3.188706in}}%
\pgfpathlineto{\pgfqpoint{3.502866in}{3.052131in}}%
\pgfpathlineto{\pgfqpoint{3.531058in}{2.701715in}}%
\pgfpathlineto{\pgfqpoint{3.569821in}{2.238644in}}%
\pgfpathlineto{\pgfqpoint{3.598012in}{1.929982in}}%
\pgfpathlineto{\pgfqpoint{3.612107in}{1.804956in}}%
\pgfpathlineto{\pgfqpoint{3.615631in}{1.785235in}}%
\pgfpathlineto{\pgfqpoint{3.619155in}{1.775646in}}%
\pgfpathlineto{\pgfqpoint{3.622679in}{1.781602in}}%
\pgfpathlineto{\pgfqpoint{3.626203in}{1.810791in}}%
\pgfpathlineto{\pgfqpoint{3.629727in}{1.873125in}}%
\pgfpathlineto{\pgfqpoint{3.633251in}{1.979343in}}%
\pgfpathlineto{\pgfqpoint{3.640299in}{2.346256in}}%
\pgfpathlineto{\pgfqpoint{3.654394in}{3.219153in}}%
\pgfpathlineto{\pgfqpoint{3.657918in}{3.302147in}}%
\pgfpathlineto{\pgfqpoint{3.661442in}{3.317549in}}%
\pgfpathlineto{\pgfqpoint{3.664966in}{3.286133in}}%
\pgfpathlineto{\pgfqpoint{3.682585in}{3.022983in}}%
\pgfpathlineto{\pgfqpoint{3.735444in}{2.445674in}}%
\pgfpathlineto{\pgfqpoint{3.774207in}{2.053106in}}%
\pgfpathlineto{\pgfqpoint{3.788303in}{1.929172in}}%
\pgfpathlineto{\pgfqpoint{3.795350in}{1.882812in}}%
\pgfpathlineto{\pgfqpoint{3.798874in}{1.868637in}}%
\pgfpathlineto{\pgfqpoint{3.802398in}{1.864487in}}%
\pgfpathlineto{\pgfqpoint{3.805922in}{1.875002in}}%
\pgfpathlineto{\pgfqpoint{3.809446in}{1.906346in}}%
\pgfpathlineto{\pgfqpoint{3.812970in}{1.965905in}}%
\pgfpathlineto{\pgfqpoint{3.820018in}{2.196176in}}%
\pgfpathlineto{\pgfqpoint{3.841161in}{3.175251in}}%
\pgfpathlineto{\pgfqpoint{3.844685in}{3.203143in}}%
\pgfpathlineto{\pgfqpoint{3.848209in}{3.189962in}}%
\pgfpathlineto{\pgfqpoint{3.855257in}{3.104327in}}%
\pgfpathlineto{\pgfqpoint{3.865828in}{2.968758in}}%
\pgfpathlineto{\pgfqpoint{3.886972in}{2.756661in}}%
\pgfpathlineto{\pgfqpoint{3.936306in}{2.281793in}}%
\pgfpathlineto{\pgfqpoint{3.964498in}{2.036211in}}%
\pgfpathlineto{\pgfqpoint{3.975069in}{1.964235in}}%
\pgfpathlineto{\pgfqpoint{3.982117in}{1.938332in}}%
\pgfpathlineto{\pgfqpoint{3.985641in}{1.938497in}}%
\pgfpathlineto{\pgfqpoint{3.989165in}{1.952283in}}%
\pgfpathlineto{\pgfqpoint{3.992689in}{1.984577in}}%
\pgfpathlineto{\pgfqpoint{3.996213in}{2.040839in}}%
\pgfpathlineto{\pgfqpoint{4.003261in}{2.242559in}}%
\pgfpathlineto{\pgfqpoint{4.024404in}{3.071940in}}%
\pgfpathlineto{\pgfqpoint{4.027928in}{3.106636in}}%
\pgfpathlineto{\pgfqpoint{4.031452in}{3.106201in}}%
\pgfpathlineto{\pgfqpoint{4.034976in}{3.082092in}}%
\pgfpathlineto{\pgfqpoint{4.070215in}{2.728402in}}%
\pgfpathlineto{\pgfqpoint{4.123073in}{2.268580in}}%
\pgfpathlineto{\pgfqpoint{4.147741in}{2.076760in}}%
\pgfpathlineto{\pgfqpoint{4.158312in}{2.015630in}}%
\pgfpathlineto{\pgfqpoint{4.161836in}{2.003150in}}%
\pgfpathlineto{\pgfqpoint{4.165360in}{1.997383in}}%
\pgfpathlineto{\pgfqpoint{4.168884in}{2.000875in}}%
\pgfpathlineto{\pgfqpoint{4.172408in}{2.016872in}}%
\pgfpathlineto{\pgfqpoint{4.175932in}{2.049210in}}%
\pgfpathlineto{\pgfqpoint{4.179456in}{2.101904in}}%
\pgfpathlineto{\pgfqpoint{4.186503in}{2.279407in}}%
\pgfpathlineto{\pgfqpoint{4.207647in}{2.987494in}}%
\pgfpathlineto{\pgfqpoint{4.211171in}{3.025305in}}%
\pgfpathlineto{\pgfqpoint{4.214695in}{3.033451in}}%
\pgfpathlineto{\pgfqpoint{4.218219in}{3.019792in}}%
\pgfpathlineto{\pgfqpoint{4.225266in}{2.959175in}}%
\pgfpathlineto{\pgfqpoint{4.242886in}{2.793045in}}%
\pgfpathlineto{\pgfqpoint{4.285173in}{2.449034in}}%
\pgfpathlineto{\pgfqpoint{4.320412in}{2.181288in}}%
\pgfpathlineto{\pgfqpoint{4.334507in}{2.091499in}}%
\pgfpathlineto{\pgfqpoint{4.341555in}{2.059912in}}%
\pgfpathlineto{\pgfqpoint{4.345079in}{2.050681in}}%
\pgfpathlineto{\pgfqpoint{4.348603in}{2.047967in}}%
\pgfpathlineto{\pgfqpoint{4.352127in}{2.053964in}}%
\pgfpathlineto{\pgfqpoint{4.355651in}{2.071340in}}%
\pgfpathlineto{\pgfqpoint{4.359175in}{2.103088in}}%
\pgfpathlineto{\pgfqpoint{4.366222in}{2.220650in}}%
\pgfpathlineto{\pgfqpoint{4.373270in}{2.414632in}}%
\pgfpathlineto{\pgfqpoint{4.387366in}{2.852043in}}%
\pgfpathlineto{\pgfqpoint{4.394414in}{2.956627in}}%
\pgfpathlineto{\pgfqpoint{4.397938in}{2.970331in}}%
\pgfpathlineto{\pgfqpoint{4.401461in}{2.964478in}}%
\pgfpathlineto{\pgfqpoint{4.408509in}{2.918572in}}%
\pgfpathlineto{\pgfqpoint{4.440224in}{2.659497in}}%
\pgfpathlineto{\pgfqpoint{4.489559in}{2.298773in}}%
\pgfpathlineto{\pgfqpoint{4.510702in}{2.161407in}}%
\pgfpathlineto{\pgfqpoint{4.521274in}{2.109751in}}%
\pgfpathlineto{\pgfqpoint{4.528322in}{2.091966in}}%
\pgfpathlineto{\pgfqpoint{4.531846in}{2.091685in}}%
\pgfpathlineto{\pgfqpoint{4.535370in}{2.099518in}}%
\pgfpathlineto{\pgfqpoint{4.538894in}{2.117640in}}%
\pgfpathlineto{\pgfqpoint{4.542418in}{2.148375in}}%
\pgfpathlineto{\pgfqpoint{4.549465in}{2.255417in}}%
\pgfpathlineto{\pgfqpoint{4.556513in}{2.424178in}}%
\pgfpathlineto{\pgfqpoint{4.570609in}{2.800331in}}%
\pgfpathlineto{\pgfqpoint{4.577657in}{2.898425in}}%
\pgfpathlineto{\pgfqpoint{4.581180in}{2.915549in}}%
\pgfpathlineto{\pgfqpoint{4.584704in}{2.915400in}}%
\pgfpathlineto{\pgfqpoint{4.588228in}{2.902629in}}%
\pgfpathlineto{\pgfqpoint{4.595276in}{2.856387in}}%
\pgfpathlineto{\pgfqpoint{4.626991in}{2.620558in}}%
\pgfpathlineto{\pgfqpoint{4.672802in}{2.311049in}}%
\pgfpathlineto{\pgfqpoint{4.693945in}{2.185949in}}%
\pgfpathlineto{\pgfqpoint{4.704517in}{2.141298in}}%
\pgfpathlineto{\pgfqpoint{4.708041in}{2.132408in}}%
\pgfpathlineto{\pgfqpoint{4.711565in}{2.128114in}}%
\pgfpathlineto{\pgfqpoint{4.715089in}{2.129745in}}%
\pgfpathlineto{\pgfqpoint{4.718613in}{2.138879in}}%
\pgfpathlineto{\pgfqpoint{4.722137in}{2.157277in}}%
\pgfpathlineto{\pgfqpoint{4.729184in}{2.228762in}}%
\pgfpathlineto{\pgfqpoint{4.736232in}{2.352624in}}%
\pgfpathlineto{\pgfqpoint{4.757376in}{2.812321in}}%
\pgfpathlineto{\pgfqpoint{4.760899in}{2.848886in}}%
\pgfpathlineto{\pgfqpoint{4.764423in}{2.867937in}}%
\pgfpathlineto{\pgfqpoint{4.767947in}{2.871869in}}%
\pgfpathlineto{\pgfqpoint{4.771471in}{2.864103in}}%
\pgfpathlineto{\pgfqpoint{4.778519in}{2.827254in}}%
\pgfpathlineto{\pgfqpoint{4.845473in}{2.385962in}}%
\pgfpathlineto{\pgfqpoint{4.870140in}{2.242486in}}%
\pgfpathlineto{\pgfqpoint{4.880712in}{2.192706in}}%
\pgfpathlineto{\pgfqpoint{4.887760in}{2.169319in}}%
\pgfpathlineto{\pgfqpoint{4.891284in}{2.162434in}}%
\pgfpathlineto{\pgfqpoint{4.894808in}{2.159972in}}%
\pgfpathlineto{\pgfqpoint{4.898332in}{2.163086in}}%
\pgfpathlineto{\pgfqpoint{4.901856in}{2.173096in}}%
\pgfpathlineto{\pgfqpoint{4.905379in}{2.191425in}}%
\pgfpathlineto{\pgfqpoint{4.912427in}{2.258226in}}%
\pgfpathlineto{\pgfqpoint{4.919475in}{2.368876in}}%
\pgfpathlineto{\pgfqpoint{4.940618in}{2.771838in}}%
\pgfpathlineto{\pgfqpoint{4.947666in}{2.826468in}}%
\pgfpathlineto{\pgfqpoint{4.951190in}{2.833248in}}%
\pgfpathlineto{\pgfqpoint{4.954714in}{2.829362in}}%
\pgfpathlineto{\pgfqpoint{4.961762in}{2.800528in}}%
\pgfpathlineto{\pgfqpoint{4.975857in}{2.712416in}}%
\pgfpathlineto{\pgfqpoint{5.007573in}{2.515523in}}%
\pgfpathlineto{\pgfqpoint{5.042812in}{2.312596in}}%
\pgfpathlineto{\pgfqpoint{5.060431in}{2.227480in}}%
\pgfpathlineto{\pgfqpoint{5.067479in}{2.202932in}}%
\pgfpathlineto{\pgfqpoint{5.074527in}{2.189176in}}%
\pgfpathlineto{\pgfqpoint{5.078051in}{2.188204in}}%
\pgfpathlineto{\pgfqpoint{5.081575in}{2.192445in}}%
\pgfpathlineto{\pgfqpoint{5.085098in}{2.203001in}}%
\pgfpathlineto{\pgfqpoint{5.088622in}{2.221011in}}%
\pgfpathlineto{\pgfqpoint{5.095670in}{2.283266in}}%
\pgfpathlineto{\pgfqpoint{5.102718in}{2.382431in}}%
\pgfpathlineto{\pgfqpoint{5.123861in}{2.737524in}}%
\pgfpathlineto{\pgfqpoint{5.130909in}{2.790254in}}%
\pgfpathlineto{\pgfqpoint{5.134433in}{2.798959in}}%
\pgfpathlineto{\pgfqpoint{5.137957in}{2.798037in}}%
\pgfpathlineto{\pgfqpoint{5.141481in}{2.789657in}}%
\pgfpathlineto{\pgfqpoint{5.148529in}{2.758678in}}%
\pgfpathlineto{\pgfqpoint{5.169672in}{2.634357in}}%
\pgfpathlineto{\pgfqpoint{5.208435in}{2.415067in}}%
\pgfpathlineto{\pgfqpoint{5.233102in}{2.289089in}}%
\pgfpathlineto{\pgfqpoint{5.243674in}{2.245270in}}%
\pgfpathlineto{\pgfqpoint{5.250722in}{2.223905in}}%
\pgfpathlineto{\pgfqpoint{5.257770in}{2.213109in}}%
\pgfpathlineto{\pgfqpoint{5.261294in}{2.213338in}}%
\pgfpathlineto{\pgfqpoint{5.264817in}{2.218418in}}%
\pgfpathlineto{\pgfqpoint{5.268341in}{2.229263in}}%
\pgfpathlineto{\pgfqpoint{5.275389in}{2.271753in}}%
\pgfpathlineto{\pgfqpoint{5.282437in}{2.345598in}}%
\pgfpathlineto{\pgfqpoint{5.293009in}{2.505419in}}%
\pgfpathlineto{\pgfqpoint{5.303580in}{2.667705in}}%
\pgfpathlineto{\pgfqpoint{5.310628in}{2.738656in}}%
\pgfpathlineto{\pgfqpoint{5.314152in}{2.758538in}}%
\pgfpathlineto{\pgfqpoint{5.317676in}{2.768479in}}%
\pgfpathlineto{\pgfqpoint{5.321200in}{2.769789in}}%
\pgfpathlineto{\pgfqpoint{5.324724in}{2.764146in}}%
\pgfpathlineto{\pgfqpoint{5.331772in}{2.738727in}}%
\pgfpathlineto{\pgfqpoint{5.345867in}{2.664476in}}%
\pgfpathlineto{\pgfqpoint{5.388154in}{2.436346in}}%
\pgfpathlineto{\pgfqpoint{5.412821in}{2.316339in}}%
\pgfpathlineto{\pgfqpoint{5.426917in}{2.261460in}}%
\pgfpathlineto{\pgfqpoint{5.433965in}{2.242901in}}%
\pgfpathlineto{\pgfqpoint{5.437489in}{2.237199in}}%
\pgfpathlineto{\pgfqpoint{5.441013in}{2.234615in}}%
\pgfpathlineto{\pgfqpoint{5.444536in}{2.235803in}}%
\pgfpathlineto{\pgfqpoint{5.448060in}{2.241488in}}%
\pgfpathlineto{\pgfqpoint{5.451584in}{2.252423in}}%
\pgfpathlineto{\pgfqpoint{5.458632in}{2.292786in}}%
\pgfpathlineto{\pgfqpoint{5.465680in}{2.360236in}}%
\pgfpathlineto{\pgfqpoint{5.476252in}{2.502394in}}%
\pgfpathlineto{\pgfqpoint{5.486823in}{2.646337in}}%
\pgfpathlineto{\pgfqpoint{5.493871in}{2.711364in}}%
\pgfpathlineto{\pgfqpoint{5.497395in}{2.730676in}}%
\pgfpathlineto{\pgfqpoint{5.500919in}{2.741346in}}%
\pgfpathlineto{\pgfqpoint{5.504443in}{2.744309in}}%
\pgfpathlineto{\pgfqpoint{5.507967in}{2.740857in}}%
\pgfpathlineto{\pgfqpoint{5.515014in}{2.720215in}}%
\pgfpathlineto{\pgfqpoint{5.525586in}{2.671524in}}%
\pgfpathlineto{\pgfqpoint{5.585493in}{2.371976in}}%
\pgfpathlineto{\pgfqpoint{5.603112in}{2.299017in}}%
\pgfpathlineto{\pgfqpoint{5.613684in}{2.267206in}}%
\pgfpathlineto{\pgfqpoint{5.620732in}{2.255604in}}%
\pgfpathlineto{\pgfqpoint{5.624255in}{2.254007in}}%
\pgfpathlineto{\pgfqpoint{5.627779in}{2.255953in}}%
\pgfpathlineto{\pgfqpoint{5.631303in}{2.262053in}}%
\pgfpathlineto{\pgfqpoint{5.634827in}{2.272928in}}%
\pgfpathlineto{\pgfqpoint{5.641875in}{2.311129in}}%
\pgfpathlineto{\pgfqpoint{5.648923in}{2.372807in}}%
\pgfpathlineto{\pgfqpoint{5.663018in}{2.545234in}}%
\pgfpathlineto{\pgfqpoint{5.673590in}{2.661356in}}%
\pgfpathlineto{\pgfqpoint{5.680638in}{2.706121in}}%
\pgfpathlineto{\pgfqpoint{5.684162in}{2.717148in}}%
\pgfpathlineto{\pgfqpoint{5.687686in}{2.721310in}}%
\pgfpathlineto{\pgfqpoint{5.691210in}{2.719595in}}%
\pgfpathlineto{\pgfqpoint{5.694733in}{2.713130in}}%
\pgfpathlineto{\pgfqpoint{5.701781in}{2.690253in}}%
\pgfpathlineto{\pgfqpoint{5.715877in}{2.626169in}}%
\pgfpathlineto{\pgfqpoint{5.761688in}{2.408494in}}%
\pgfpathlineto{\pgfqpoint{5.782831in}{2.322344in}}%
\pgfpathlineto{\pgfqpoint{5.793403in}{2.289847in}}%
\pgfpathlineto{\pgfqpoint{5.800451in}{2.275931in}}%
\pgfpathlineto{\pgfqpoint{5.803974in}{2.272329in}}%
\pgfpathlineto{\pgfqpoint{5.807498in}{2.271547in}}%
\pgfpathlineto{\pgfqpoint{5.811022in}{2.274080in}}%
\pgfpathlineto{\pgfqpoint{5.814546in}{2.280444in}}%
\pgfpathlineto{\pgfqpoint{5.821594in}{2.306628in}}%
\pgfpathlineto{\pgfqpoint{5.828642in}{2.352950in}}%
\pgfpathlineto{\pgfqpoint{5.839213in}{2.457160in}}%
\pgfpathlineto{\pgfqpoint{5.856833in}{2.642475in}}%
\pgfpathlineto{\pgfqpoint{5.863881in}{2.684413in}}%
\pgfpathlineto{\pgfqpoint{5.867405in}{2.695529in}}%
\pgfpathlineto{\pgfqpoint{5.870929in}{2.700534in}}%
\pgfpathlineto{\pgfqpoint{5.874452in}{2.700183in}}%
\pgfpathlineto{\pgfqpoint{5.877976in}{2.695380in}}%
\pgfpathlineto{\pgfqpoint{5.885024in}{2.676043in}}%
\pgfpathlineto{\pgfqpoint{5.895596in}{2.633593in}}%
\pgfpathlineto{\pgfqpoint{5.959026in}{2.356341in}}%
\pgfpathlineto{\pgfqpoint{5.973122in}{2.310918in}}%
\pgfpathlineto{\pgfqpoint{5.980170in}{2.295396in}}%
\pgfpathlineto{\pgfqpoint{5.987217in}{2.287567in}}%
\pgfpathlineto{\pgfqpoint{5.990741in}{2.287454in}}%
\pgfpathlineto{\pgfqpoint{5.994265in}{2.290434in}}%
\pgfpathlineto{\pgfqpoint{5.997789in}{2.296941in}}%
\pgfpathlineto{\pgfqpoint{6.004837in}{2.322112in}}%
\pgfpathlineto{\pgfqpoint{6.011885in}{2.365075in}}%
\pgfpathlineto{\pgfqpoint{6.022456in}{2.459356in}}%
\pgfpathlineto{\pgfqpoint{6.040076in}{2.625986in}}%
\pgfpathlineto{\pgfqpoint{6.047124in}{2.665162in}}%
\pgfpathlineto{\pgfqpoint{6.050648in}{2.676173in}}%
\pgfpathlineto{\pgfqpoint{6.054171in}{2.681747in}}%
\pgfpathlineto{\pgfqpoint{6.057695in}{2.682456in}}%
\pgfpathlineto{\pgfqpoint{6.061219in}{2.679020in}}%
\pgfpathlineto{\pgfqpoint{6.068267in}{2.662759in}}%
\pgfpathlineto{\pgfqpoint{6.078839in}{2.624553in}}%
\pgfpathlineto{\pgfqpoint{6.145793in}{2.351566in}}%
\pgfpathlineto{\pgfqpoint{6.156365in}{2.321564in}}%
\pgfpathlineto{\pgfqpoint{6.163412in}{2.307816in}}%
\pgfpathlineto{\pgfqpoint{6.170460in}{2.301483in}}%
\pgfpathlineto{\pgfqpoint{6.173984in}{2.301914in}}%
\pgfpathlineto{\pgfqpoint{6.177508in}{2.305224in}}%
\pgfpathlineto{\pgfqpoint{6.181032in}{2.311777in}}%
\pgfpathlineto{\pgfqpoint{6.188080in}{2.335873in}}%
\pgfpathlineto{\pgfqpoint{6.195128in}{2.375718in}}%
\pgfpathlineto{\pgfqpoint{6.205699in}{2.461230in}}%
\pgfpathlineto{\pgfqpoint{6.223319in}{2.611515in}}%
\pgfpathlineto{\pgfqpoint{6.230367in}{2.648037in}}%
\pgfpathlineto{\pgfqpoint{6.237414in}{2.664740in}}%
\pgfpathlineto{\pgfqpoint{6.240938in}{2.666260in}}%
\pgfpathlineto{\pgfqpoint{6.244462in}{2.663943in}}%
\pgfpathlineto{\pgfqpoint{6.251510in}{2.650340in}}%
\pgfpathlineto{\pgfqpoint{6.262082in}{2.616018in}}%
\pgfpathlineto{\pgfqpoint{6.286749in}{2.516370in}}%
\pgfpathlineto{\pgfqpoint{6.314940in}{2.406018in}}%
\pgfpathlineto{\pgfqpoint{6.332560in}{2.348801in}}%
\pgfpathlineto{\pgfqpoint{6.343131in}{2.324604in}}%
\pgfpathlineto{\pgfqpoint{6.350179in}{2.315717in}}%
\pgfpathlineto{\pgfqpoint{6.353703in}{2.314221in}}%
\pgfpathlineto{\pgfqpoint{6.357227in}{2.315088in}}%
\pgfpathlineto{\pgfqpoint{6.360751in}{2.318632in}}%
\pgfpathlineto{\pgfqpoint{6.367799in}{2.334926in}}%
\pgfpathlineto{\pgfqpoint{6.374847in}{2.364885in}}%
\pgfpathlineto{\pgfqpoint{6.381894in}{2.408516in}}%
\pgfpathlineto{\pgfqpoint{6.413609in}{2.632760in}}%
\pgfpathlineto{\pgfqpoint{6.420657in}{2.649324in}}%
\pgfpathlineto{\pgfqpoint{6.424181in}{2.651455in}}%
\pgfpathlineto{\pgfqpoint{6.427705in}{2.650048in}}%
\pgfpathlineto{\pgfqpoint{6.431229in}{2.645628in}}%
\pgfpathlineto{\pgfqpoint{6.438277in}{2.629863in}}%
\pgfpathlineto{\pgfqpoint{6.448848in}{2.595586in}}%
\pgfpathlineto{\pgfqpoint{6.512279in}{2.365848in}}%
\pgfpathlineto{\pgfqpoint{6.522850in}{2.340648in}}%
\pgfpathlineto{\pgfqpoint{6.529898in}{2.329864in}}%
\pgfpathlineto{\pgfqpoint{6.536946in}{2.325901in}}%
\pgfpathlineto{\pgfqpoint{6.540470in}{2.327114in}}%
\pgfpathlineto{\pgfqpoint{6.543994in}{2.330813in}}%
\pgfpathlineto{\pgfqpoint{6.551042in}{2.346632in}}%
\pgfpathlineto{\pgfqpoint{6.558089in}{2.374726in}}%
\pgfpathlineto{\pgfqpoint{6.568661in}{2.438671in}}%
\pgfpathlineto{\pgfqpoint{6.589805in}{2.587468in}}%
\pgfpathlineto{\pgfqpoint{6.596852in}{2.619091in}}%
\pgfpathlineto{\pgfqpoint{6.603900in}{2.635333in}}%
\pgfpathlineto{\pgfqpoint{6.607424in}{2.637915in}}%
\pgfpathlineto{\pgfqpoint{6.610948in}{2.637241in}}%
\pgfpathlineto{\pgfqpoint{6.614472in}{2.633746in}}%
\pgfpathlineto{\pgfqpoint{6.621520in}{2.620094in}}%
\pgfpathlineto{\pgfqpoint{6.632091in}{2.588939in}}%
\pgfpathlineto{\pgfqpoint{6.656759in}{2.499823in}}%
\pgfpathlineto{\pgfqpoint{6.681426in}{2.413618in}}%
\pgfpathlineto{\pgfqpoint{6.695522in}{2.372349in}}%
\pgfpathlineto{\pgfqpoint{6.706093in}{2.349223in}}%
\pgfpathlineto{\pgfqpoint{6.713141in}{2.339671in}}%
\pgfpathlineto{\pgfqpoint{6.720189in}{2.336631in}}%
\pgfpathlineto{\pgfqpoint{6.723713in}{2.338114in}}%
\pgfpathlineto{\pgfqpoint{6.727237in}{2.341902in}}%
\pgfpathlineto{\pgfqpoint{6.734285in}{2.357183in}}%
\pgfpathlineto{\pgfqpoint{6.741332in}{2.383507in}}%
\pgfpathlineto{\pgfqpoint{6.751904in}{2.442189in}}%
\pgfpathlineto{\pgfqpoint{6.773047in}{2.577435in}}%
\pgfpathlineto{\pgfqpoint{6.780095in}{2.606830in}}%
\pgfpathlineto{\pgfqpoint{6.787143in}{2.622617in}}%
\pgfpathlineto{\pgfqpoint{6.790667in}{2.625521in}}%
\pgfpathlineto{\pgfqpoint{6.794191in}{2.625436in}}%
\pgfpathlineto{\pgfqpoint{6.797715in}{2.622717in}}%
\pgfpathlineto{\pgfqpoint{6.804763in}{2.610918in}}%
\pgfpathlineto{\pgfqpoint{6.815334in}{2.582625in}}%
\pgfpathlineto{\pgfqpoint{6.836478in}{2.511190in}}%
\pgfpathlineto{\pgfqpoint{6.864669in}{2.417147in}}%
\pgfpathlineto{\pgfqpoint{6.878765in}{2.378472in}}%
\pgfpathlineto{\pgfqpoint{6.889336in}{2.357229in}}%
\pgfpathlineto{\pgfqpoint{6.896384in}{2.348765in}}%
\pgfpathlineto{\pgfqpoint{6.903432in}{2.346504in}}%
\pgfpathlineto{\pgfqpoint{6.906956in}{2.348192in}}%
\pgfpathlineto{\pgfqpoint{6.910480in}{2.352017in}}%
\pgfpathlineto{\pgfqpoint{6.917527in}{2.366716in}}%
\pgfpathlineto{\pgfqpoint{6.924575in}{2.391365in}}%
\pgfpathlineto{\pgfqpoint{6.935147in}{2.445288in}}%
\pgfpathlineto{\pgfqpoint{6.956290in}{2.568487in}}%
\pgfpathlineto{\pgfqpoint{6.963338in}{2.595802in}}%
\pgfpathlineto{\pgfqpoint{6.970386in}{2.611046in}}%
\pgfpathlineto{\pgfqpoint{6.973910in}{2.614168in}}%
\pgfpathlineto{\pgfqpoint{6.977434in}{2.614550in}}%
\pgfpathlineto{\pgfqpoint{6.980958in}{2.612481in}}%
\pgfpathlineto{\pgfqpoint{6.988005in}{2.602302in}}%
\pgfpathlineto{\pgfqpoint{6.998577in}{2.576625in}}%
\pgfpathlineto{\pgfqpoint{7.016197in}{2.521321in}}%
\pgfpathlineto{\pgfqpoint{7.047912in}{2.420510in}}%
\pgfpathlineto{\pgfqpoint{7.062007in}{2.384245in}}%
\pgfpathlineto{\pgfqpoint{7.072579in}{2.364713in}}%
\pgfpathlineto{\pgfqpoint{7.079627in}{2.357209in}}%
\pgfpathlineto{\pgfqpoint{7.086675in}{2.355602in}}%
\pgfpathlineto{\pgfqpoint{7.090199in}{2.357441in}}%
\pgfpathlineto{\pgfqpoint{7.097246in}{2.367195in}}%
\pgfpathlineto{\pgfqpoint{7.104294in}{2.385767in}}%
\pgfpathlineto{\pgfqpoint{7.111342in}{2.413179in}}%
\pgfpathlineto{\pgfqpoint{7.125438in}{2.487227in}}%
\pgfpathlineto{\pgfqpoint{7.139533in}{2.560481in}}%
\pgfpathlineto{\pgfqpoint{7.146581in}{2.585860in}}%
\pgfpathlineto{\pgfqpoint{7.153629in}{2.600501in}}%
\pgfpathlineto{\pgfqpoint{7.157153in}{2.603760in}}%
\pgfpathlineto{\pgfqpoint{7.160677in}{2.604509in}}%
\pgfpathlineto{\pgfqpoint{7.164201in}{2.602982in}}%
\pgfpathlineto{\pgfqpoint{7.171248in}{2.594216in}}%
\pgfpathlineto{\pgfqpoint{7.181820in}{2.570924in}}%
\pgfpathlineto{\pgfqpoint{7.199440in}{2.519219in}}%
\pgfpathlineto{\pgfqpoint{7.231155in}{2.423714in}}%
\pgfpathlineto{\pgfqpoint{7.245250in}{2.389693in}}%
\pgfpathlineto{\pgfqpoint{7.255822in}{2.371716in}}%
\pgfpathlineto{\pgfqpoint{7.262870in}{2.365058in}}%
\pgfpathlineto{\pgfqpoint{7.269918in}{2.363998in}}%
\pgfpathlineto{\pgfqpoint{7.273442in}{2.365941in}}%
\pgfpathlineto{\pgfqpoint{7.280489in}{2.375442in}}%
\pgfpathlineto{\pgfqpoint{7.287537in}{2.392974in}}%
\pgfpathlineto{\pgfqpoint{7.294585in}{2.418433in}}%
\pgfpathlineto{\pgfqpoint{7.308681in}{2.486278in}}%
\pgfpathlineto{\pgfqpoint{7.322776in}{2.553295in}}%
\pgfpathlineto{\pgfqpoint{7.329824in}{2.576875in}}%
\pgfpathlineto{\pgfqpoint{7.336872in}{2.590879in}}%
\pgfpathlineto{\pgfqpoint{7.340396in}{2.594211in}}%
\pgfpathlineto{\pgfqpoint{7.343920in}{2.595243in}}%
\pgfpathlineto{\pgfqpoint{7.347443in}{2.594167in}}%
\pgfpathlineto{\pgfqpoint{7.354491in}{2.586631in}}%
\pgfpathlineto{\pgfqpoint{7.361539in}{2.573552in}}%
\pgfpathlineto{\pgfqpoint{7.375635in}{2.537589in}}%
\pgfpathlineto{\pgfqpoint{7.421445in}{2.409698in}}%
\pgfpathlineto{\pgfqpoint{7.432017in}{2.388448in}}%
\pgfpathlineto{\pgfqpoint{7.439065in}{2.378274in}}%
\pgfpathlineto{\pgfqpoint{7.446113in}{2.372363in}}%
\pgfpathlineto{\pgfqpoint{7.453161in}{2.371754in}}%
\pgfpathlineto{\pgfqpoint{7.460208in}{2.377475in}}%
\pgfpathlineto{\pgfqpoint{7.467256in}{2.390313in}}%
\pgfpathlineto{\pgfqpoint{7.474304in}{2.410491in}}%
\pgfpathlineto{\pgfqpoint{7.484876in}{2.452631in}}%
\pgfpathlineto{\pgfqpoint{7.509543in}{2.558784in}}%
\pgfpathlineto{\pgfqpoint{7.516591in}{2.576528in}}%
\pgfpathlineto{\pgfqpoint{7.523639in}{2.585442in}}%
\pgfpathlineto{\pgfqpoint{7.527162in}{2.586690in}}%
\pgfpathlineto{\pgfqpoint{7.530686in}{2.585987in}}%
\pgfpathlineto{\pgfqpoint{7.537734in}{2.579520in}}%
\pgfpathlineto{\pgfqpoint{7.544782in}{2.567736in}}%
\pgfpathlineto{\pgfqpoint{7.555354in}{2.543562in}}%
\pgfpathlineto{\pgfqpoint{7.580021in}{2.475855in}}%
\pgfpathlineto{\pgfqpoint{7.601164in}{2.421399in}}%
\pgfpathlineto{\pgfqpoint{7.615260in}{2.393762in}}%
\pgfpathlineto{\pgfqpoint{7.622308in}{2.384420in}}%
\pgfpathlineto{\pgfqpoint{7.629356in}{2.379166in}}%
\pgfpathlineto{\pgfqpoint{7.636403in}{2.378929in}}%
\pgfpathlineto{\pgfqpoint{7.643451in}{2.384597in}}%
\pgfpathlineto{\pgfqpoint{7.650499in}{2.396808in}}%
\pgfpathlineto{\pgfqpoint{7.657547in}{2.415671in}}%
\pgfpathlineto{\pgfqpoint{7.668119in}{2.454567in}}%
\pgfpathlineto{\pgfqpoint{7.692786in}{2.552064in}}%
\pgfpathlineto{\pgfqpoint{7.699834in}{2.568705in}}%
\pgfpathlineto{\pgfqpoint{7.706881in}{2.577382in}}%
\pgfpathlineto{\pgfqpoint{7.710405in}{2.578790in}}%
\pgfpathlineto{\pgfqpoint{7.713929in}{2.578393in}}%
\pgfpathlineto{\pgfqpoint{7.720977in}{2.572855in}}%
\pgfpathlineto{\pgfqpoint{7.728025in}{2.562238in}}%
\pgfpathlineto{\pgfqpoint{7.738597in}{2.539925in}}%
\pgfpathlineto{\pgfqpoint{7.759740in}{2.485447in}}%
\pgfpathlineto{\pgfqpoint{7.784407in}{2.424632in}}%
\pgfpathlineto{\pgfqpoint{7.798503in}{2.398770in}}%
\pgfpathlineto{\pgfqpoint{7.805551in}{2.390182in}}%
\pgfpathlineto{\pgfqpoint{7.812599in}{2.385506in}}%
\pgfpathlineto{\pgfqpoint{7.819646in}{2.385573in}}%
\pgfpathlineto{\pgfqpoint{7.826694in}{2.391146in}}%
\pgfpathlineto{\pgfqpoint{7.833742in}{2.402737in}}%
\pgfpathlineto{\pgfqpoint{7.840790in}{2.420368in}}%
\pgfpathlineto{\pgfqpoint{7.851361in}{2.456303in}}%
\pgfpathlineto{\pgfqpoint{7.876029in}{2.545973in}}%
\pgfpathlineto{\pgfqpoint{7.883077in}{2.561567in}}%
\pgfpathlineto{\pgfqpoint{7.890124in}{2.569968in}}%
\pgfpathlineto{\pgfqpoint{7.897172in}{2.571345in}}%
\pgfpathlineto{\pgfqpoint{7.904220in}{2.566612in}}%
\pgfpathlineto{\pgfqpoint{7.911268in}{2.557044in}}%
\pgfpathlineto{\pgfqpoint{7.921839in}{2.536454in}}%
\pgfpathlineto{\pgfqpoint{7.942983in}{2.485221in}}%
\pgfpathlineto{\pgfqpoint{7.967650in}{2.427700in}}%
\pgfpathlineto{\pgfqpoint{7.978222in}{2.408615in}}%
\pgfpathlineto{\pgfqpoint{7.988794in}{2.395585in}}%
\pgfpathlineto{\pgfqpoint{7.995841in}{2.391420in}}%
\pgfpathlineto{\pgfqpoint{8.002889in}{2.391731in}}%
\pgfpathlineto{\pgfqpoint{8.009937in}{2.397175in}}%
\pgfpathlineto{\pgfqpoint{8.016985in}{2.408161in}}%
\pgfpathlineto{\pgfqpoint{8.024033in}{2.424636in}}%
\pgfpathlineto{\pgfqpoint{8.034604in}{2.457863in}}%
\pgfpathlineto{\pgfqpoint{8.059272in}{2.540438in}}%
\pgfpathlineto{\pgfqpoint{8.066319in}{2.555045in}}%
\pgfpathlineto{\pgfqpoint{8.073367in}{2.563142in}}%
\pgfpathlineto{\pgfqpoint{8.080415in}{2.564800in}}%
\pgfpathlineto{\pgfqpoint{8.087463in}{2.560766in}}%
\pgfpathlineto{\pgfqpoint{8.094511in}{2.552142in}}%
\pgfpathlineto{\pgfqpoint{8.105082in}{2.533143in}}%
\pgfpathlineto{\pgfqpoint{8.122702in}{2.493348in}}%
\pgfpathlineto{\pgfqpoint{8.147369in}{2.437540in}}%
\pgfpathlineto{\pgfqpoint{8.161465in}{2.412707in}}%
\pgfpathlineto{\pgfqpoint{8.172037in}{2.400655in}}%
\pgfpathlineto{\pgfqpoint{8.179084in}{2.396938in}}%
\pgfpathlineto{\pgfqpoint{8.186132in}{2.397444in}}%
\pgfpathlineto{\pgfqpoint{8.193180in}{2.402735in}}%
\pgfpathlineto{\pgfqpoint{8.200228in}{2.413129in}}%
\pgfpathlineto{\pgfqpoint{8.207276in}{2.428523in}}%
\pgfpathlineto{\pgfqpoint{8.217847in}{2.459270in}}%
\pgfpathlineto{\pgfqpoint{8.242515in}{2.535400in}}%
\pgfpathlineto{\pgfqpoint{8.249562in}{2.549076in}}%
\pgfpathlineto{\pgfqpoint{8.256610in}{2.556853in}}%
\pgfpathlineto{\pgfqpoint{8.263658in}{2.558722in}}%
\pgfpathlineto{\pgfqpoint{8.270706in}{2.555293in}}%
\pgfpathlineto{\pgfqpoint{8.277754in}{2.547517in}}%
\pgfpathlineto{\pgfqpoint{8.288325in}{2.529988in}}%
\pgfpathlineto{\pgfqpoint{8.305945in}{2.492650in}}%
\pgfpathlineto{\pgfqpoint{8.330612in}{2.439900in}}%
\pgfpathlineto{\pgfqpoint{8.344708in}{2.416570in}}%
\pgfpathlineto{\pgfqpoint{8.355279in}{2.405411in}}%
\pgfpathlineto{\pgfqpoint{8.362327in}{2.402090in}}%
\pgfpathlineto{\pgfqpoint{8.369375in}{2.402748in}}%
\pgfpathlineto{\pgfqpoint{8.376423in}{2.407866in}}%
\pgfpathlineto{\pgfqpoint{8.383471in}{2.417688in}}%
\pgfpathlineto{\pgfqpoint{8.394042in}{2.440780in}}%
\pgfpathlineto{\pgfqpoint{8.411662in}{2.492913in}}%
\pgfpathlineto{\pgfqpoint{8.425757in}{2.530807in}}%
\pgfpathlineto{\pgfqpoint{8.432805in}{2.543608in}}%
\pgfpathlineto{\pgfqpoint{8.439853in}{2.551054in}}%
\pgfpathlineto{\pgfqpoint{8.446901in}{2.553077in}}%
\pgfpathlineto{\pgfqpoint{8.453949in}{2.550170in}}%
\pgfpathlineto{\pgfqpoint{8.460996in}{2.543156in}}%
\pgfpathlineto{\pgfqpoint{8.471568in}{2.526985in}}%
\pgfpathlineto{\pgfqpoint{8.489188in}{2.491969in}}%
\pgfpathlineto{\pgfqpoint{8.517379in}{2.435977in}}%
\pgfpathlineto{\pgfqpoint{8.527951in}{2.420215in}}%
\pgfpathlineto{\pgfqpoint{8.538522in}{2.409875in}}%
\pgfpathlineto{\pgfqpoint{8.545570in}{2.406901in}}%
\pgfpathlineto{\pgfqpoint{8.552618in}{2.407675in}}%
\pgfpathlineto{\pgfqpoint{8.559666in}{2.412607in}}%
\pgfpathlineto{\pgfqpoint{8.566714in}{2.421878in}}%
\pgfpathlineto{\pgfqpoint{8.577285in}{2.443405in}}%
\pgfpathlineto{\pgfqpoint{8.594905in}{2.491559in}}%
\pgfpathlineto{\pgfqpoint{8.609000in}{2.526611in}}%
\pgfpathlineto{\pgfqpoint{8.616048in}{2.538592in}}%
\pgfpathlineto{\pgfqpoint{8.623096in}{2.545702in}}%
\pgfpathlineto{\pgfqpoint{8.630144in}{2.547832in}}%
\pgfpathlineto{\pgfqpoint{8.637192in}{2.545377in}}%
\pgfpathlineto{\pgfqpoint{8.644239in}{2.539048in}}%
\pgfpathlineto{\pgfqpoint{8.654811in}{2.524127in}}%
\pgfpathlineto{\pgfqpoint{8.672431in}{2.491306in}}%
\pgfpathlineto{\pgfqpoint{8.700622in}{2.438447in}}%
\pgfpathlineto{\pgfqpoint{8.711194in}{2.423654in}}%
\pgfpathlineto{\pgfqpoint{8.721765in}{2.414063in}}%
\pgfpathlineto{\pgfqpoint{8.728813in}{2.411396in}}%
\pgfpathlineto{\pgfqpoint{8.735861in}{2.412257in}}%
\pgfpathlineto{\pgfqpoint{8.742909in}{2.416993in}}%
\pgfpathlineto{\pgfqpoint{8.749956in}{2.425734in}}%
\pgfpathlineto{\pgfqpoint{8.760528in}{2.445804in}}%
\pgfpathlineto{\pgfqpoint{8.778148in}{2.490323in}}%
\pgfpathlineto{\pgfqpoint{8.792243in}{2.522774in}}%
\pgfpathlineto{\pgfqpoint{8.799291in}{2.533984in}}%
\pgfpathlineto{\pgfqpoint{8.806339in}{2.540759in}}%
\pgfpathlineto{\pgfqpoint{8.813387in}{2.542957in}}%
\pgfpathlineto{\pgfqpoint{8.820434in}{2.540893in}}%
\pgfpathlineto{\pgfqpoint{8.827482in}{2.535179in}}%
\pgfpathlineto{\pgfqpoint{8.838054in}{2.521411in}}%
\pgfpathlineto{\pgfqpoint{8.855673in}{2.490659in}}%
\pgfpathlineto{\pgfqpoint{8.883865in}{2.440782in}}%
\pgfpathlineto{\pgfqpoint{8.894436in}{2.426897in}}%
\pgfpathlineto{\pgfqpoint{8.905008in}{2.417994in}}%
\pgfpathlineto{\pgfqpoint{8.912056in}{2.415597in}}%
\pgfpathlineto{\pgfqpoint{8.919104in}{2.416519in}}%
\pgfpathlineto{\pgfqpoint{8.926152in}{2.421053in}}%
\pgfpathlineto{\pgfqpoint{8.933199in}{2.429286in}}%
\pgfpathlineto{\pgfqpoint{8.943771in}{2.447999in}}%
\pgfpathlineto{\pgfqpoint{8.961391in}{2.489192in}}%
\pgfpathlineto{\pgfqpoint{8.975486in}{2.519258in}}%
\pgfpathlineto{\pgfqpoint{8.986058in}{2.533496in}}%
\pgfpathlineto{\pgfqpoint{8.993106in}{2.537828in}}%
\pgfpathlineto{\pgfqpoint{9.000153in}{2.538030in}}%
\pgfpathlineto{\pgfqpoint{9.007201in}{2.534508in}}%
\pgfpathlineto{\pgfqpoint{9.017773in}{2.523612in}}%
\pgfpathlineto{\pgfqpoint{9.031869in}{2.502229in}}%
\pgfpathlineto{\pgfqpoint{9.070631in}{2.438204in}}%
\pgfpathlineto{\pgfqpoint{9.081203in}{2.426595in}}%
\pgfpathlineto{\pgfqpoint{9.088251in}{2.421682in}}%
\pgfpathlineto{\pgfqpoint{9.095299in}{2.419524in}}%
\pgfpathlineto{\pgfqpoint{9.102347in}{2.420486in}}%
\pgfpathlineto{\pgfqpoint{9.109394in}{2.424815in}}%
\pgfpathlineto{\pgfqpoint{9.116442in}{2.432564in}}%
\pgfpathlineto{\pgfqpoint{9.127014in}{2.450013in}}%
\pgfpathlineto{\pgfqpoint{9.148157in}{2.495833in}}%
\pgfpathlineto{\pgfqpoint{9.158729in}{2.516034in}}%
\pgfpathlineto{\pgfqpoint{9.169301in}{2.529390in}}%
\pgfpathlineto{\pgfqpoint{9.176349in}{2.533570in}}%
\pgfpathlineto{\pgfqpoint{9.183396in}{2.533932in}}%
\pgfpathlineto{\pgfqpoint{9.190444in}{2.530813in}}%
\pgfpathlineto{\pgfqpoint{9.201016in}{2.520818in}}%
\pgfpathlineto{\pgfqpoint{9.215111in}{2.500879in}}%
\pgfpathlineto{\pgfqpoint{9.253874in}{2.440571in}}%
\pgfpathlineto{\pgfqpoint{9.264446in}{2.429697in}}%
\pgfpathlineto{\pgfqpoint{9.275018in}{2.423822in}}%
\pgfpathlineto{\pgfqpoint{9.282066in}{2.423305in}}%
\pgfpathlineto{\pgfqpoint{9.289113in}{2.425843in}}%
\pgfpathlineto{\pgfqpoint{9.296161in}{2.431560in}}%
\pgfpathlineto{\pgfqpoint{9.306733in}{2.445810in}}%
\pgfpathlineto{\pgfqpoint{9.320829in}{2.472587in}}%
\pgfpathlineto{\pgfqpoint{9.341972in}{2.513074in}}%
\pgfpathlineto{\pgfqpoint{9.352544in}{2.525599in}}%
\pgfpathlineto{\pgfqpoint{9.359591in}{2.529620in}}%
\pgfpathlineto{\pgfqpoint{9.366639in}{2.530110in}}%
\pgfpathlineto{\pgfqpoint{9.373687in}{2.527348in}}%
\pgfpathlineto{\pgfqpoint{9.384259in}{2.518177in}}%
\pgfpathlineto{\pgfqpoint{9.398354in}{2.499586in}}%
\pgfpathlineto{\pgfqpoint{9.437117in}{2.442802in}}%
\pgfpathlineto{\pgfqpoint{9.447689in}{2.432617in}}%
\pgfpathlineto{\pgfqpoint{9.458261in}{2.427182in}}%
\pgfpathlineto{\pgfqpoint{9.465309in}{2.426767in}}%
\pgfpathlineto{\pgfqpoint{9.472356in}{2.429211in}}%
\pgfpathlineto{\pgfqpoint{9.479404in}{2.434609in}}%
\pgfpathlineto{\pgfqpoint{9.489976in}{2.447932in}}%
\pgfpathlineto{\pgfqpoint{9.504071in}{2.472795in}}%
\pgfpathlineto{\pgfqpoint{9.525215in}{2.510352in}}%
\pgfpathlineto{\pgfqpoint{9.535787in}{2.522097in}}%
\pgfpathlineto{\pgfqpoint{9.542834in}{2.525955in}}%
\pgfpathlineto{\pgfqpoint{9.549882in}{2.526545in}}%
\pgfpathlineto{\pgfqpoint{9.556930in}{2.524100in}}%
\pgfpathlineto{\pgfqpoint{9.567502in}{2.515681in}}%
\pgfpathlineto{\pgfqpoint{9.581597in}{2.498350in}}%
\pgfpathlineto{\pgfqpoint{9.620360in}{2.444905in}}%
\pgfpathlineto{\pgfqpoint{9.630932in}{2.435362in}}%
\pgfpathlineto{\pgfqpoint{9.641504in}{2.430329in}}%
\pgfpathlineto{\pgfqpoint{9.648551in}{2.430000in}}%
\pgfpathlineto{\pgfqpoint{9.655599in}{2.432345in}}%
\pgfpathlineto{\pgfqpoint{9.662647in}{2.437435in}}%
\pgfpathlineto{\pgfqpoint{9.673219in}{2.449890in}}%
\pgfpathlineto{\pgfqpoint{9.687314in}{2.472986in}}%
\pgfpathlineto{\pgfqpoint{9.708458in}{2.507848in}}%
\pgfpathlineto{\pgfqpoint{9.719029in}{2.518859in}}%
\pgfpathlineto{\pgfqpoint{9.726077in}{2.522552in}}%
\pgfpathlineto{\pgfqpoint{9.733125in}{2.523220in}}%
\pgfpathlineto{\pgfqpoint{9.740173in}{2.521055in}}%
\pgfpathlineto{\pgfqpoint{9.750745in}{2.513324in}}%
\pgfpathlineto{\pgfqpoint{9.764840in}{2.497170in}}%
\pgfpathlineto{\pgfqpoint{9.807127in}{2.443497in}}%
\pgfpathlineto{\pgfqpoint{9.817699in}{2.435861in}}%
\pgfpathlineto{\pgfqpoint{9.828270in}{2.432844in}}%
\pgfpathlineto{\pgfqpoint{9.835318in}{2.433821in}}%
\pgfpathlineto{\pgfqpoint{9.842366in}{2.437345in}}%
\pgfpathlineto{\pgfqpoint{9.852938in}{2.447278in}}%
\pgfpathlineto{\pgfqpoint{9.867033in}{2.467409in}}%
\pgfpathlineto{\pgfqpoint{9.891701in}{2.505541in}}%
\pgfpathlineto{\pgfqpoint{9.902272in}{2.515863in}}%
\pgfpathlineto{\pgfqpoint{9.909320in}{2.519392in}}%
\pgfpathlineto{\pgfqpoint{9.916368in}{2.520119in}}%
\pgfpathlineto{\pgfqpoint{9.923416in}{2.518201in}}%
\pgfpathlineto{\pgfqpoint{9.933987in}{2.511100in}}%
\pgfpathlineto{\pgfqpoint{9.948083in}{2.496044in}}%
\pgfpathlineto{\pgfqpoint{9.990370in}{2.445570in}}%
\pgfpathlineto{\pgfqpoint{10.000942in}{2.438427in}}%
\pgfpathlineto{\pgfqpoint{10.011513in}{2.435654in}}%
\pgfpathlineto{\pgfqpoint{10.018561in}{2.436611in}}%
\pgfpathlineto{\pgfqpoint{10.025609in}{2.439943in}}%
\pgfpathlineto{\pgfqpoint{10.036181in}{2.449248in}}%
\pgfpathlineto{\pgfqpoint{10.050276in}{2.467984in}}%
\pgfpathlineto{\pgfqpoint{10.074944in}{2.503414in}}%
\pgfpathlineto{\pgfqpoint{10.085515in}{2.513090in}}%
\pgfpathlineto{\pgfqpoint{10.092563in}{2.516456in}}%
\pgfpathlineto{\pgfqpoint{10.099611in}{2.517226in}}%
\pgfpathlineto{\pgfqpoint{10.106659in}{2.515528in}}%
\pgfpathlineto{\pgfqpoint{10.117230in}{2.509002in}}%
\pgfpathlineto{\pgfqpoint{10.131326in}{2.494971in}}%
\pgfpathlineto{\pgfqpoint{10.173613in}{2.447519in}}%
\pgfpathlineto{\pgfqpoint{10.184185in}{2.440837in}}%
\pgfpathlineto{\pgfqpoint{10.194756in}{2.438282in}}%
\pgfpathlineto{\pgfqpoint{10.201804in}{2.439214in}}%
\pgfpathlineto{\pgfqpoint{10.208852in}{2.442361in}}%
\pgfpathlineto{\pgfqpoint{10.219424in}{2.451074in}}%
\pgfpathlineto{\pgfqpoint{10.233519in}{2.468517in}}%
\pgfpathlineto{\pgfqpoint{10.258186in}{2.501452in}}%
\pgfpathlineto{\pgfqpoint{10.268758in}{2.510521in}}%
\pgfpathlineto{\pgfqpoint{10.275806in}{2.513727in}}%
\pgfpathlineto{\pgfqpoint{10.282854in}{2.514526in}}%
\pgfpathlineto{\pgfqpoint{10.289902in}{2.513024in}}%
\pgfpathlineto{\pgfqpoint{10.300473in}{2.507024in}}%
\pgfpathlineto{\pgfqpoint{10.314569in}{2.493949in}}%
\pgfpathlineto{\pgfqpoint{10.356856in}{2.449351in}}%
\pgfpathlineto{\pgfqpoint{10.367427in}{2.443098in}}%
\pgfpathlineto{\pgfqpoint{10.377999in}{2.440741in}}%
\pgfpathlineto{\pgfqpoint{10.385047in}{2.441643in}}%
\pgfpathlineto{\pgfqpoint{10.395619in}{2.446859in}}%
\pgfpathlineto{\pgfqpoint{10.406190in}{2.456358in}}%
\pgfpathlineto{\pgfqpoint{10.423810in}{2.478298in}}%
\pgfpathlineto{\pgfqpoint{10.441429in}{2.499641in}}%
\pgfpathlineto{\pgfqpoint{10.452001in}{2.508141in}}%
\pgfpathlineto{\pgfqpoint{10.462573in}{2.511876in}}%
\pgfpathlineto{\pgfqpoint{10.469621in}{2.511600in}}%
\pgfpathlineto{\pgfqpoint{10.480192in}{2.507419in}}%
\pgfpathlineto{\pgfqpoint{10.490764in}{2.499604in}}%
\pgfpathlineto{\pgfqpoint{10.508383in}{2.481854in}}%
\pgfpathlineto{\pgfqpoint{10.533051in}{2.456625in}}%
\pgfpathlineto{\pgfqpoint{10.547146in}{2.446799in}}%
\pgfpathlineto{\pgfqpoint{10.557718in}{2.443319in}}%
\pgfpathlineto{\pgfqpoint{10.568290in}{2.443910in}}%
\pgfpathlineto{\pgfqpoint{10.578862in}{2.448815in}}%
\pgfpathlineto{\pgfqpoint{10.589433in}{2.457689in}}%
\pgfpathlineto{\pgfqpoint{10.607053in}{2.478102in}}%
\pgfpathlineto{\pgfqpoint{10.624672in}{2.497968in}}%
\pgfpathlineto{\pgfqpoint{10.635244in}{2.505934in}}%
\pgfpathlineto{\pgfqpoint{10.645816in}{2.509502in}}%
\pgfpathlineto{\pgfqpoint{10.656387in}{2.508481in}}%
\pgfpathlineto{\pgfqpoint{10.666959in}{2.503405in}}%
\pgfpathlineto{\pgfqpoint{10.681055in}{2.492052in}}%
\pgfpathlineto{\pgfqpoint{10.723341in}{2.452692in}}%
\pgfpathlineto{\pgfqpoint{10.733913in}{2.447211in}}%
\pgfpathlineto{\pgfqpoint{10.744485in}{2.445194in}}%
\pgfpathlineto{\pgfqpoint{10.755057in}{2.447120in}}%
\pgfpathlineto{\pgfqpoint{10.765628in}{2.453026in}}%
\pgfpathlineto{\pgfqpoint{10.779724in}{2.466029in}}%
\pgfpathlineto{\pgfqpoint{10.811439in}{2.499316in}}%
\pgfpathlineto{\pgfqpoint{10.822011in}{2.505498in}}%
\pgfpathlineto{\pgfqpoint{10.832582in}{2.507464in}}%
\pgfpathlineto{\pgfqpoint{10.843154in}{2.505252in}}%
\pgfpathlineto{\pgfqpoint{10.853726in}{2.499495in}}%
\pgfpathlineto{\pgfqpoint{10.867821in}{2.488023in}}%
\pgfpathlineto{\pgfqpoint{10.903060in}{2.456532in}}%
\pgfpathlineto{\pgfqpoint{10.917156in}{2.449078in}}%
\pgfpathlineto{\pgfqpoint{10.927728in}{2.447208in}}%
\pgfpathlineto{\pgfqpoint{10.938300in}{2.449034in}}%
\pgfpathlineto{\pgfqpoint{10.948871in}{2.454568in}}%
\pgfpathlineto{\pgfqpoint{10.962967in}{2.466693in}}%
\pgfpathlineto{\pgfqpoint{10.994682in}{2.497696in}}%
\pgfpathlineto{\pgfqpoint{11.005254in}{2.503509in}}%
\pgfpathlineto{\pgfqpoint{11.015825in}{2.505417in}}%
\pgfpathlineto{\pgfqpoint{11.026397in}{2.503427in}}%
\pgfpathlineto{\pgfqpoint{11.036969in}{2.498102in}}%
\pgfpathlineto{\pgfqpoint{11.051064in}{2.487391in}}%
\pgfpathlineto{\pgfqpoint{11.089827in}{2.455638in}}%
\pgfpathlineto{\pgfqpoint{11.100399in}{2.450828in}}%
\pgfpathlineto{\pgfqpoint{11.110971in}{2.449093in}}%
\pgfpathlineto{\pgfqpoint{11.121542in}{2.450820in}}%
\pgfpathlineto{\pgfqpoint{11.132114in}{2.456003in}}%
\pgfpathlineto{\pgfqpoint{11.146210in}{2.467310in}}%
\pgfpathlineto{\pgfqpoint{11.177925in}{2.496196in}}%
\pgfpathlineto{\pgfqpoint{11.188497in}{2.501660in}}%
\pgfpathlineto{\pgfqpoint{11.199068in}{2.503506in}}%
\pgfpathlineto{\pgfqpoint{11.209640in}{2.501715in}}%
\pgfpathlineto{\pgfqpoint{11.220212in}{2.496788in}}%
\pgfpathlineto{\pgfqpoint{11.234307in}{2.486791in}}%
\pgfpathlineto{\pgfqpoint{11.273070in}{2.456976in}}%
\pgfpathlineto{\pgfqpoint{11.283642in}{2.452470in}}%
\pgfpathlineto{\pgfqpoint{11.294214in}{2.450856in}}%
\pgfpathlineto{\pgfqpoint{11.304785in}{2.452487in}}%
\pgfpathlineto{\pgfqpoint{11.315357in}{2.457339in}}%
\pgfpathlineto{\pgfqpoint{11.329453in}{2.467885in}}%
\pgfpathlineto{\pgfqpoint{11.361168in}{2.494806in}}%
\pgfpathlineto{\pgfqpoint{11.371739in}{2.499941in}}%
\pgfpathlineto{\pgfqpoint{11.382311in}{2.501722in}}%
\pgfpathlineto{\pgfqpoint{11.392883in}{2.500111in}}%
\pgfpathlineto{\pgfqpoint{11.403455in}{2.495551in}}%
\pgfpathlineto{\pgfqpoint{11.421074in}{2.483530in}}%
\pgfpathlineto{\pgfqpoint{11.452789in}{2.460151in}}%
\pgfpathlineto{\pgfqpoint{11.466885in}{2.454008in}}%
\pgfpathlineto{\pgfqpoint{11.477457in}{2.452506in}}%
\pgfpathlineto{\pgfqpoint{11.488028in}{2.454044in}}%
\pgfpathlineto{\pgfqpoint{11.498600in}{2.458584in}}%
\pgfpathlineto{\pgfqpoint{11.512696in}{2.468421in}}%
\pgfpathlineto{\pgfqpoint{11.544411in}{2.493517in}}%
\pgfpathlineto{\pgfqpoint{11.554982in}{2.498342in}}%
\pgfpathlineto{\pgfqpoint{11.565554in}{2.500057in}}%
\pgfpathlineto{\pgfqpoint{11.576126in}{2.498607in}}%
\pgfpathlineto{\pgfqpoint{11.590221in}{2.492478in}}%
\pgfpathlineto{\pgfqpoint{11.607841in}{2.480571in}}%
\pgfpathlineto{\pgfqpoint{11.636032in}{2.461211in}}%
\pgfpathlineto{\pgfqpoint{11.650128in}{2.455450in}}%
\pgfpathlineto{\pgfqpoint{11.660699in}{2.454050in}}%
\pgfpathlineto{\pgfqpoint{11.671271in}{2.455497in}}%
\pgfpathlineto{\pgfqpoint{11.681843in}{2.459745in}}%
\pgfpathlineto{\pgfqpoint{11.695938in}{2.468921in}}%
\pgfpathlineto{\pgfqpoint{11.731177in}{2.494129in}}%
\pgfpathlineto{\pgfqpoint{11.741749in}{2.497738in}}%
\pgfpathlineto{\pgfqpoint{11.752321in}{2.498386in}}%
\pgfpathlineto{\pgfqpoint{11.762893in}{2.496156in}}%
\pgfpathlineto{\pgfqpoint{11.776988in}{2.489547in}}%
\pgfpathlineto{\pgfqpoint{11.801655in}{2.473009in}}%
\pgfpathlineto{\pgfqpoint{11.822799in}{2.460513in}}%
\pgfpathlineto{\pgfqpoint{11.836894in}{2.456080in}}%
\pgfpathlineto{\pgfqpoint{11.847466in}{2.455647in}}%
\pgfpathlineto{\pgfqpoint{11.858038in}{2.457902in}}%
\pgfpathlineto{\pgfqpoint{11.872134in}{2.464730in}}%
\pgfpathlineto{\pgfqpoint{11.893277in}{2.479763in}}%
\pgfpathlineto{\pgfqpoint{11.910896in}{2.491212in}}%
\pgfpathlineto{\pgfqpoint{11.924992in}{2.496308in}}%
\pgfpathlineto{\pgfqpoint{11.935564in}{2.496957in}}%
\pgfpathlineto{\pgfqpoint{11.946135in}{2.494916in}}%
\pgfpathlineto{\pgfqpoint{11.960231in}{2.488770in}}%
\pgfpathlineto{\pgfqpoint{11.984898in}{2.473280in}}%
\pgfpathlineto{\pgfqpoint{12.006042in}{2.461547in}}%
\pgfpathlineto{\pgfqpoint{12.020137in}{2.457392in}}%
\pgfpathlineto{\pgfqpoint{12.030709in}{2.456990in}}%
\pgfpathlineto{\pgfqpoint{12.041281in}{2.459103in}}%
\pgfpathlineto{\pgfqpoint{12.055376in}{2.465480in}}%
\pgfpathlineto{\pgfqpoint{12.080044in}{2.481886in}}%
\pgfpathlineto{\pgfqpoint{12.097663in}{2.491769in}}%
\pgfpathlineto{\pgfqpoint{12.111759in}{2.495481in}}%
\pgfpathlineto{\pgfqpoint{12.122331in}{2.495265in}}%
\pgfpathlineto{\pgfqpoint{12.136426in}{2.491286in}}%
\pgfpathlineto{\pgfqpoint{12.154046in}{2.482142in}}%
\pgfpathlineto{\pgfqpoint{12.189285in}{2.462516in}}%
\pgfpathlineto{\pgfqpoint{12.203380in}{2.458620in}}%
\pgfpathlineto{\pgfqpoint{12.213952in}{2.458247in}}%
\pgfpathlineto{\pgfqpoint{12.228048in}{2.461387in}}%
\pgfpathlineto{\pgfqpoint{12.242143in}{2.468138in}}%
\pgfpathlineto{\pgfqpoint{12.284430in}{2.491972in}}%
\pgfpathlineto{\pgfqpoint{12.298526in}{2.494430in}}%
\pgfpathlineto{\pgfqpoint{12.312621in}{2.492661in}}%
\pgfpathlineto{\pgfqpoint{12.326717in}{2.487346in}}%
\pgfpathlineto{\pgfqpoint{12.354908in}{2.471789in}}%
\pgfpathlineto{\pgfqpoint{12.372528in}{2.463424in}}%
\pgfpathlineto{\pgfqpoint{12.386623in}{2.459771in}}%
\pgfpathlineto{\pgfqpoint{12.400719in}{2.459796in}}%
\pgfpathlineto{\pgfqpoint{12.414814in}{2.463660in}}%
\pgfpathlineto{\pgfqpoint{12.432434in}{2.472655in}}%
\pgfpathlineto{\pgfqpoint{12.460625in}{2.488338in}}%
\pgfpathlineto{\pgfqpoint{12.474721in}{2.492576in}}%
\pgfpathlineto{\pgfqpoint{12.488816in}{2.492917in}}%
\pgfpathlineto{\pgfqpoint{12.502912in}{2.489514in}}%
\pgfpathlineto{\pgfqpoint{12.520531in}{2.481542in}}%
\pgfpathlineto{\pgfqpoint{12.555770in}{2.464274in}}%
\pgfpathlineto{\pgfqpoint{12.569866in}{2.460848in}}%
\pgfpathlineto{\pgfqpoint{12.583962in}{2.460871in}}%
\pgfpathlineto{\pgfqpoint{12.598057in}{2.464481in}}%
\pgfpathlineto{\pgfqpoint{12.615677in}{2.472873in}}%
\pgfpathlineto{\pgfqpoint{12.647392in}{2.488819in}}%
\pgfpathlineto{\pgfqpoint{12.661488in}{2.491938in}}%
\pgfpathlineto{\pgfqpoint{12.675583in}{2.491372in}}%
\pgfpathlineto{\pgfqpoint{12.689679in}{2.487465in}}%
\pgfpathlineto{\pgfqpoint{12.710822in}{2.477737in}}%
\pgfpathlineto{\pgfqpoint{12.739013in}{2.465070in}}%
\pgfpathlineto{\pgfqpoint{12.753109in}{2.461856in}}%
\pgfpathlineto{\pgfqpoint{12.767205in}{2.461876in}}%
\pgfpathlineto{\pgfqpoint{12.781300in}{2.465249in}}%
\pgfpathlineto{\pgfqpoint{12.798920in}{2.473078in}}%
\pgfpathlineto{\pgfqpoint{12.830635in}{2.487970in}}%
\pgfpathlineto{\pgfqpoint{12.844730in}{2.490912in}}%
\pgfpathlineto{\pgfqpoint{12.858826in}{2.490418in}}%
\pgfpathlineto{\pgfqpoint{12.872922in}{2.486792in}}%
\pgfpathlineto{\pgfqpoint{12.894065in}{2.477697in}}%
\pgfpathlineto{\pgfqpoint{12.922256in}{2.465815in}}%
\pgfpathlineto{\pgfqpoint{12.936352in}{2.462800in}}%
\pgfpathlineto{\pgfqpoint{12.950447in}{2.462816in}}%
\pgfpathlineto{\pgfqpoint{12.964543in}{2.465966in}}%
\pgfpathlineto{\pgfqpoint{12.971591in}{2.468546in}}%
\pgfpathlineto{\pgfqpoint{12.971591in}{2.468546in}}%
\pgfusepath{stroke}%
\end{pgfscope}%
\begin{pgfscope}%
\pgfsetrectcap%
\pgfsetmiterjoin%
\pgfsetlinewidth{0.803000pt}%
\definecolor{currentstroke}{rgb}{0.000000,0.000000,0.000000}%
\pgfsetstrokecolor{currentstroke}%
\pgfsetdash{}{0pt}%
\pgfpathmoveto{\pgfqpoint{1.875000in}{0.625000in}}%
\pgfpathlineto{\pgfqpoint{1.875000in}{4.400000in}}%
\pgfusepath{stroke}%
\end{pgfscope}%
\begin{pgfscope}%
\pgfsetrectcap%
\pgfsetmiterjoin%
\pgfsetlinewidth{0.803000pt}%
\definecolor{currentstroke}{rgb}{0.000000,0.000000,0.000000}%
\pgfsetstrokecolor{currentstroke}%
\pgfsetdash{}{0pt}%
\pgfpathmoveto{\pgfqpoint{13.500000in}{0.625000in}}%
\pgfpathlineto{\pgfqpoint{13.500000in}{4.400000in}}%
\pgfusepath{stroke}%
\end{pgfscope}%
\begin{pgfscope}%
\pgfsetrectcap%
\pgfsetmiterjoin%
\pgfsetlinewidth{0.803000pt}%
\definecolor{currentstroke}{rgb}{0.000000,0.000000,0.000000}%
\pgfsetstrokecolor{currentstroke}%
\pgfsetdash{}{0pt}%
\pgfpathmoveto{\pgfqpoint{1.875000in}{0.625000in}}%
\pgfpathlineto{\pgfqpoint{13.500000in}{0.625000in}}%
\pgfusepath{stroke}%
\end{pgfscope}%
\begin{pgfscope}%
\pgfsetrectcap%
\pgfsetmiterjoin%
\pgfsetlinewidth{0.803000pt}%
\definecolor{currentstroke}{rgb}{0.000000,0.000000,0.000000}%
\pgfsetstrokecolor{currentstroke}%
\pgfsetdash{}{0pt}%
\pgfpathmoveto{\pgfqpoint{1.875000in}{4.400000in}}%
\pgfpathlineto{\pgfqpoint{13.500000in}{4.400000in}}%
\pgfusepath{stroke}%
\end{pgfscope}%
\begin{pgfscope}%
\pgfsetbuttcap%
\pgfsetmiterjoin%
\definecolor{currentfill}{rgb}{1.000000,1.000000,1.000000}%
\pgfsetfillcolor{currentfill}%
\pgfsetfillopacity{0.800000}%
\pgfsetlinewidth{1.003750pt}%
\definecolor{currentstroke}{rgb}{0.800000,0.800000,0.800000}%
\pgfsetstrokecolor{currentstroke}%
\pgfsetstrokeopacity{0.800000}%
\pgfsetdash{}{0pt}%
\pgfpathmoveto{\pgfqpoint{12.142090in}{3.677317in}}%
\pgfpathlineto{\pgfqpoint{13.402778in}{3.677317in}}%
\pgfpathquadraticcurveto{\pgfqpoint{13.430556in}{3.677317in}}{\pgfqpoint{13.430556in}{3.705095in}}%
\pgfpathlineto{\pgfqpoint{13.430556in}{4.302778in}}%
\pgfpathquadraticcurveto{\pgfqpoint{13.430556in}{4.330556in}}{\pgfqpoint{13.402778in}{4.330556in}}%
\pgfpathlineto{\pgfqpoint{12.142090in}{4.330556in}}%
\pgfpathquadraticcurveto{\pgfqpoint{12.114312in}{4.330556in}}{\pgfqpoint{12.114312in}{4.302778in}}%
\pgfpathlineto{\pgfqpoint{12.114312in}{3.705095in}}%
\pgfpathquadraticcurveto{\pgfqpoint{12.114312in}{3.677317in}}{\pgfqpoint{12.142090in}{3.677317in}}%
\pgfpathlineto{\pgfqpoint{12.142090in}{3.677317in}}%
\pgfpathclose%
\pgfusepath{stroke,fill}%
\end{pgfscope}%
\begin{pgfscope}%
\pgfsetrectcap%
\pgfsetroundjoin%
\pgfsetlinewidth{1.505625pt}%
\definecolor{currentstroke}{rgb}{0.121569,0.466667,0.705882}%
\pgfsetstrokecolor{currentstroke}%
\pgfsetdash{}{0pt}%
\pgfpathmoveto{\pgfqpoint{12.169868in}{4.218088in}}%
\pgfpathlineto{\pgfqpoint{12.308757in}{4.218088in}}%
\pgfpathlineto{\pgfqpoint{12.447645in}{4.218088in}}%
\pgfusepath{stroke}%
\end{pgfscope}%
\begin{pgfscope}%
\definecolor{textcolor}{rgb}{0.000000,0.000000,0.000000}%
\pgfsetstrokecolor{textcolor}%
\pgfsetfillcolor{textcolor}%
\pgftext[x=12.558757in,y=4.169477in,left,base]{\color{textcolor}\sffamily\fontsize{10.000000}{12.000000}\selectfont omega=0.5}%
\end{pgfscope}%
\begin{pgfscope}%
\pgfsetrectcap%
\pgfsetroundjoin%
\pgfsetlinewidth{1.505625pt}%
\definecolor{currentstroke}{rgb}{1.000000,0.498039,0.054902}%
\pgfsetstrokecolor{currentstroke}%
\pgfsetdash{}{0pt}%
\pgfpathmoveto{\pgfqpoint{12.169868in}{4.014231in}}%
\pgfpathlineto{\pgfqpoint{12.308757in}{4.014231in}}%
\pgfpathlineto{\pgfqpoint{12.447645in}{4.014231in}}%
\pgfusepath{stroke}%
\end{pgfscope}%
\begin{pgfscope}%
\definecolor{textcolor}{rgb}{0.000000,0.000000,0.000000}%
\pgfsetstrokecolor{textcolor}%
\pgfsetfillcolor{textcolor}%
\pgftext[x=12.558757in,y=3.965620in,left,base]{\color{textcolor}\sffamily\fontsize{10.000000}{12.000000}\selectfont omega=1}%
\end{pgfscope}%
\begin{pgfscope}%
\pgfsetrectcap%
\pgfsetroundjoin%
\pgfsetlinewidth{1.505625pt}%
\definecolor{currentstroke}{rgb}{0.172549,0.627451,0.172549}%
\pgfsetstrokecolor{currentstroke}%
\pgfsetdash{}{0pt}%
\pgfpathmoveto{\pgfqpoint{12.169868in}{3.810374in}}%
\pgfpathlineto{\pgfqpoint{12.308757in}{3.810374in}}%
\pgfpathlineto{\pgfqpoint{12.447645in}{3.810374in}}%
\pgfusepath{stroke}%
\end{pgfscope}%
\begin{pgfscope}%
\definecolor{textcolor}{rgb}{0.000000,0.000000,0.000000}%
\pgfsetstrokecolor{textcolor}%
\pgfsetfillcolor{textcolor}%
\pgftext[x=12.558757in,y=3.761763in,left,base]{\color{textcolor}\sffamily\fontsize{10.000000}{12.000000}\selectfont omega=1.7}%
\end{pgfscope}%
\end{pgfpicture}%
\makeatother%
\endgroup%
}
\vspace*{-10mm}
\caption[Density decay]{The density decay for $\omega=[0.5,1.0,1.7]$. The less viscose the fluid the faster the amplitude decays.}
\label{fig:m3-1}
\end{figure}
It shows that all curves start at the same amplitude at $t=0$ but that the smaller $\omega$ the faster the density decays which is inline with expectations. 
For $t\rightarrow \infty$ the density amplitude of all three curves converges towards $0$ and stabilizes around $\rho_{0}$.
The higher $\omega$, i.e. the collision frequency, the more the initial sine wave is able to propagate through the system. Conversely, lowering $\omega$ increases the kinematic viscosity of the simulated fluid and decreasing the time it takes for the system to get into the equilibrium state.

% Because $\omega = \frac{1}{\tau}$

\paragraph{Velocity decay}
Similarly, the velocity in an imbalanced state is expected to converge towards the equilibrium.
To test this, an initial distribution of $\rho_{0}=1$ $u_{y}(r,0)=0$ and $u_{x}(r,t_{0})=\rho_{0}+\epsilon \sin \left( \frac{2\pi x}{L_{y}} \right)$ was chosen, where $L_{y}$ is the length of the domain in the y direction, i.e. the height.
\begin{figure}[ht]
\centering
\resizebox{\columnwidth}{!}{\large%% Creator: Matplotlib, PGF backend
%%
%% To include the figure in your LaTeX document, write
%%   \input{<filename>.pgf}
%%
%% Make sure the required packages are loaded in your preamble
%%   \usepackage{pgf}
%%
%% Also ensure that all the required font packages are loaded; for instance,
%% the lmodern package is sometimes necessary when using math font.
%%   \usepackage{lmodern}
%%
%% Figures using additional raster images can only be included by \input if
%% they are in the same directory as the main LaTeX file. For loading figures
%% from other directories you can use the `import` package
%%   \usepackage{import}
%%
%% and then include the figures with
%%   \import{<path to file>}{<filename>.pgf}
%%
%% Matplotlib used the following preamble
%%   \usepackage{fontspec}
%%   \setmainfont{DejaVuSerif.ttf}[Path=\detokenize{/home/joe/miniconda3/envs/high/lib/python3.9/site-packages/matplotlib/mpl-data/fonts/ttf/}]
%%   \setsansfont{DejaVuSans.ttf}[Path=\detokenize{/home/joe/miniconda3/envs/high/lib/python3.9/site-packages/matplotlib/mpl-data/fonts/ttf/}]
%%   \setmonofont{DejaVuSansMono.ttf}[Path=\detokenize{/home/joe/miniconda3/envs/high/lib/python3.9/site-packages/matplotlib/mpl-data/fonts/ttf/}]
%%
\begingroup%
\makeatletter%
\begin{pgfpicture}%
\pgfpathrectangle{\pgfpointorigin}{\pgfqpoint{20.000000in}{6.000000in}}%
\pgfusepath{use as bounding box, clip}%
\begin{pgfscope}%
\pgfsetbuttcap%
\pgfsetmiterjoin%
\pgfsetlinewidth{0.000000pt}%
\definecolor{currentstroke}{rgb}{1.000000,1.000000,1.000000}%
\pgfsetstrokecolor{currentstroke}%
\pgfsetstrokeopacity{0.000000}%
\pgfsetdash{}{0pt}%
\pgfpathmoveto{\pgfqpoint{0.000000in}{0.000000in}}%
\pgfpathlineto{\pgfqpoint{20.000000in}{0.000000in}}%
\pgfpathlineto{\pgfqpoint{20.000000in}{6.000000in}}%
\pgfpathlineto{\pgfqpoint{0.000000in}{6.000000in}}%
\pgfpathlineto{\pgfqpoint{0.000000in}{0.000000in}}%
\pgfpathclose%
\pgfusepath{}%
\end{pgfscope}%
\begin{pgfscope}%
\pgfsetbuttcap%
\pgfsetmiterjoin%
\definecolor{currentfill}{rgb}{1.000000,1.000000,1.000000}%
\pgfsetfillcolor{currentfill}%
\pgfsetlinewidth{0.000000pt}%
\definecolor{currentstroke}{rgb}{0.000000,0.000000,0.000000}%
\pgfsetstrokecolor{currentstroke}%
\pgfsetstrokeopacity{0.000000}%
\pgfsetdash{}{0pt}%
\pgfpathmoveto{\pgfqpoint{0.939653in}{0.865000in}}%
\pgfpathlineto{\pgfqpoint{6.535076in}{0.865000in}}%
\pgfpathlineto{\pgfqpoint{6.535076in}{5.440556in}}%
\pgfpathlineto{\pgfqpoint{0.939653in}{5.440556in}}%
\pgfpathlineto{\pgfqpoint{0.939653in}{0.865000in}}%
\pgfpathclose%
\pgfusepath{fill}%
\end{pgfscope}%
\begin{pgfscope}%
\pgfpathrectangle{\pgfqpoint{0.939653in}{0.865000in}}{\pgfqpoint{5.595423in}{4.575556in}}%
\pgfusepath{clip}%
\pgfsetrectcap%
\pgfsetroundjoin%
\pgfsetlinewidth{0.803000pt}%
\definecolor{currentstroke}{rgb}{0.690196,0.690196,0.690196}%
\pgfsetstrokecolor{currentstroke}%
\pgfsetdash{}{0pt}%
\pgfpathmoveto{\pgfqpoint{1.179981in}{0.865000in}}%
\pgfpathlineto{\pgfqpoint{1.179981in}{5.440556in}}%
\pgfusepath{stroke}%
\end{pgfscope}%
\begin{pgfscope}%
\pgfsetbuttcap%
\pgfsetroundjoin%
\definecolor{currentfill}{rgb}{0.000000,0.000000,0.000000}%
\pgfsetfillcolor{currentfill}%
\pgfsetlinewidth{0.803000pt}%
\definecolor{currentstroke}{rgb}{0.000000,0.000000,0.000000}%
\pgfsetstrokecolor{currentstroke}%
\pgfsetdash{}{0pt}%
\pgfsys@defobject{currentmarker}{\pgfqpoint{0.000000in}{-0.048611in}}{\pgfqpoint{0.000000in}{0.000000in}}{%
\pgfpathmoveto{\pgfqpoint{0.000000in}{0.000000in}}%
\pgfpathlineto{\pgfqpoint{0.000000in}{-0.048611in}}%
\pgfusepath{stroke,fill}%
}%
\begin{pgfscope}%
\pgfsys@transformshift{1.179981in}{0.865000in}%
\pgfsys@useobject{currentmarker}{}%
\end{pgfscope}%
\end{pgfscope}%
\begin{pgfscope}%
\definecolor{textcolor}{rgb}{0.000000,0.000000,0.000000}%
\pgfsetstrokecolor{textcolor}%
\pgfsetfillcolor{textcolor}%
\pgftext[x=1.179981in,y=0.767778in,,top]{\color{textcolor}\sffamily\fontsize{16.000000}{19.200000}\selectfont \ensuremath{-}0.10}%
\end{pgfscope}%
\begin{pgfscope}%
\pgfpathrectangle{\pgfqpoint{0.939653in}{0.865000in}}{\pgfqpoint{5.595423in}{4.575556in}}%
\pgfusepath{clip}%
\pgfsetrectcap%
\pgfsetroundjoin%
\pgfsetlinewidth{0.803000pt}%
\definecolor{currentstroke}{rgb}{0.690196,0.690196,0.690196}%
\pgfsetstrokecolor{currentstroke}%
\pgfsetdash{}{0pt}%
\pgfpathmoveto{\pgfqpoint{2.458672in}{0.865000in}}%
\pgfpathlineto{\pgfqpoint{2.458672in}{5.440556in}}%
\pgfusepath{stroke}%
\end{pgfscope}%
\begin{pgfscope}%
\pgfsetbuttcap%
\pgfsetroundjoin%
\definecolor{currentfill}{rgb}{0.000000,0.000000,0.000000}%
\pgfsetfillcolor{currentfill}%
\pgfsetlinewidth{0.803000pt}%
\definecolor{currentstroke}{rgb}{0.000000,0.000000,0.000000}%
\pgfsetstrokecolor{currentstroke}%
\pgfsetdash{}{0pt}%
\pgfsys@defobject{currentmarker}{\pgfqpoint{0.000000in}{-0.048611in}}{\pgfqpoint{0.000000in}{0.000000in}}{%
\pgfpathmoveto{\pgfqpoint{0.000000in}{0.000000in}}%
\pgfpathlineto{\pgfqpoint{0.000000in}{-0.048611in}}%
\pgfusepath{stroke,fill}%
}%
\begin{pgfscope}%
\pgfsys@transformshift{2.458672in}{0.865000in}%
\pgfsys@useobject{currentmarker}{}%
\end{pgfscope}%
\end{pgfscope}%
\begin{pgfscope}%
\definecolor{textcolor}{rgb}{0.000000,0.000000,0.000000}%
\pgfsetstrokecolor{textcolor}%
\pgfsetfillcolor{textcolor}%
\pgftext[x=2.458672in,y=0.767778in,,top]{\color{textcolor}\sffamily\fontsize{16.000000}{19.200000}\selectfont \ensuremath{-}0.05}%
\end{pgfscope}%
\begin{pgfscope}%
\pgfpathrectangle{\pgfqpoint{0.939653in}{0.865000in}}{\pgfqpoint{5.595423in}{4.575556in}}%
\pgfusepath{clip}%
\pgfsetrectcap%
\pgfsetroundjoin%
\pgfsetlinewidth{0.803000pt}%
\definecolor{currentstroke}{rgb}{0.690196,0.690196,0.690196}%
\pgfsetstrokecolor{currentstroke}%
\pgfsetdash{}{0pt}%
\pgfpathmoveto{\pgfqpoint{3.737364in}{0.865000in}}%
\pgfpathlineto{\pgfqpoint{3.737364in}{5.440556in}}%
\pgfusepath{stroke}%
\end{pgfscope}%
\begin{pgfscope}%
\pgfsetbuttcap%
\pgfsetroundjoin%
\definecolor{currentfill}{rgb}{0.000000,0.000000,0.000000}%
\pgfsetfillcolor{currentfill}%
\pgfsetlinewidth{0.803000pt}%
\definecolor{currentstroke}{rgb}{0.000000,0.000000,0.000000}%
\pgfsetstrokecolor{currentstroke}%
\pgfsetdash{}{0pt}%
\pgfsys@defobject{currentmarker}{\pgfqpoint{0.000000in}{-0.048611in}}{\pgfqpoint{0.000000in}{0.000000in}}{%
\pgfpathmoveto{\pgfqpoint{0.000000in}{0.000000in}}%
\pgfpathlineto{\pgfqpoint{0.000000in}{-0.048611in}}%
\pgfusepath{stroke,fill}%
}%
\begin{pgfscope}%
\pgfsys@transformshift{3.737364in}{0.865000in}%
\pgfsys@useobject{currentmarker}{}%
\end{pgfscope}%
\end{pgfscope}%
\begin{pgfscope}%
\definecolor{textcolor}{rgb}{0.000000,0.000000,0.000000}%
\pgfsetstrokecolor{textcolor}%
\pgfsetfillcolor{textcolor}%
\pgftext[x=3.737364in,y=0.767778in,,top]{\color{textcolor}\sffamily\fontsize{16.000000}{19.200000}\selectfont 0.00}%
\end{pgfscope}%
\begin{pgfscope}%
\pgfpathrectangle{\pgfqpoint{0.939653in}{0.865000in}}{\pgfqpoint{5.595423in}{4.575556in}}%
\pgfusepath{clip}%
\pgfsetrectcap%
\pgfsetroundjoin%
\pgfsetlinewidth{0.803000pt}%
\definecolor{currentstroke}{rgb}{0.690196,0.690196,0.690196}%
\pgfsetstrokecolor{currentstroke}%
\pgfsetdash{}{0pt}%
\pgfpathmoveto{\pgfqpoint{5.016056in}{0.865000in}}%
\pgfpathlineto{\pgfqpoint{5.016056in}{5.440556in}}%
\pgfusepath{stroke}%
\end{pgfscope}%
\begin{pgfscope}%
\pgfsetbuttcap%
\pgfsetroundjoin%
\definecolor{currentfill}{rgb}{0.000000,0.000000,0.000000}%
\pgfsetfillcolor{currentfill}%
\pgfsetlinewidth{0.803000pt}%
\definecolor{currentstroke}{rgb}{0.000000,0.000000,0.000000}%
\pgfsetstrokecolor{currentstroke}%
\pgfsetdash{}{0pt}%
\pgfsys@defobject{currentmarker}{\pgfqpoint{0.000000in}{-0.048611in}}{\pgfqpoint{0.000000in}{0.000000in}}{%
\pgfpathmoveto{\pgfqpoint{0.000000in}{0.000000in}}%
\pgfpathlineto{\pgfqpoint{0.000000in}{-0.048611in}}%
\pgfusepath{stroke,fill}%
}%
\begin{pgfscope}%
\pgfsys@transformshift{5.016056in}{0.865000in}%
\pgfsys@useobject{currentmarker}{}%
\end{pgfscope}%
\end{pgfscope}%
\begin{pgfscope}%
\definecolor{textcolor}{rgb}{0.000000,0.000000,0.000000}%
\pgfsetstrokecolor{textcolor}%
\pgfsetfillcolor{textcolor}%
\pgftext[x=5.016056in,y=0.767778in,,top]{\color{textcolor}\sffamily\fontsize{16.000000}{19.200000}\selectfont 0.05}%
\end{pgfscope}%
\begin{pgfscope}%
\pgfpathrectangle{\pgfqpoint{0.939653in}{0.865000in}}{\pgfqpoint{5.595423in}{4.575556in}}%
\pgfusepath{clip}%
\pgfsetrectcap%
\pgfsetroundjoin%
\pgfsetlinewidth{0.803000pt}%
\definecolor{currentstroke}{rgb}{0.690196,0.690196,0.690196}%
\pgfsetstrokecolor{currentstroke}%
\pgfsetdash{}{0pt}%
\pgfpathmoveto{\pgfqpoint{6.294748in}{0.865000in}}%
\pgfpathlineto{\pgfqpoint{6.294748in}{5.440556in}}%
\pgfusepath{stroke}%
\end{pgfscope}%
\begin{pgfscope}%
\pgfsetbuttcap%
\pgfsetroundjoin%
\definecolor{currentfill}{rgb}{0.000000,0.000000,0.000000}%
\pgfsetfillcolor{currentfill}%
\pgfsetlinewidth{0.803000pt}%
\definecolor{currentstroke}{rgb}{0.000000,0.000000,0.000000}%
\pgfsetstrokecolor{currentstroke}%
\pgfsetdash{}{0pt}%
\pgfsys@defobject{currentmarker}{\pgfqpoint{0.000000in}{-0.048611in}}{\pgfqpoint{0.000000in}{0.000000in}}{%
\pgfpathmoveto{\pgfqpoint{0.000000in}{0.000000in}}%
\pgfpathlineto{\pgfqpoint{0.000000in}{-0.048611in}}%
\pgfusepath{stroke,fill}%
}%
\begin{pgfscope}%
\pgfsys@transformshift{6.294748in}{0.865000in}%
\pgfsys@useobject{currentmarker}{}%
\end{pgfscope}%
\end{pgfscope}%
\begin{pgfscope}%
\definecolor{textcolor}{rgb}{0.000000,0.000000,0.000000}%
\pgfsetstrokecolor{textcolor}%
\pgfsetfillcolor{textcolor}%
\pgftext[x=6.294748in,y=0.767778in,,top]{\color{textcolor}\sffamily\fontsize{16.000000}{19.200000}\selectfont 0.10}%
\end{pgfscope}%
\begin{pgfscope}%
\definecolor{textcolor}{rgb}{0.000000,0.000000,0.000000}%
\pgfsetstrokecolor{textcolor}%
\pgfsetfillcolor{textcolor}%
\pgftext[x=3.737364in,y=0.497162in,,top]{\color{textcolor}\sffamily\fontsize{20.000000}{24.000000}\selectfont velocity}%
\end{pgfscope}%
\begin{pgfscope}%
\pgfpathrectangle{\pgfqpoint{0.939653in}{0.865000in}}{\pgfqpoint{5.595423in}{4.575556in}}%
\pgfusepath{clip}%
\pgfsetrectcap%
\pgfsetroundjoin%
\pgfsetlinewidth{0.803000pt}%
\definecolor{currentstroke}{rgb}{0.690196,0.690196,0.690196}%
\pgfsetstrokecolor{currentstroke}%
\pgfsetdash{}{0pt}%
\pgfpathmoveto{\pgfqpoint{0.939653in}{1.072980in}}%
\pgfpathlineto{\pgfqpoint{6.535076in}{1.072980in}}%
\pgfusepath{stroke}%
\end{pgfscope}%
\begin{pgfscope}%
\pgfsetbuttcap%
\pgfsetroundjoin%
\definecolor{currentfill}{rgb}{0.000000,0.000000,0.000000}%
\pgfsetfillcolor{currentfill}%
\pgfsetlinewidth{0.803000pt}%
\definecolor{currentstroke}{rgb}{0.000000,0.000000,0.000000}%
\pgfsetstrokecolor{currentstroke}%
\pgfsetdash{}{0pt}%
\pgfsys@defobject{currentmarker}{\pgfqpoint{-0.048611in}{0.000000in}}{\pgfqpoint{-0.000000in}{0.000000in}}{%
\pgfpathmoveto{\pgfqpoint{-0.000000in}{0.000000in}}%
\pgfpathlineto{\pgfqpoint{-0.048611in}{0.000000in}}%
\pgfusepath{stroke,fill}%
}%
\begin{pgfscope}%
\pgfsys@transformshift{0.939653in}{1.072980in}%
\pgfsys@useobject{currentmarker}{}%
\end{pgfscope}%
\end{pgfscope}%
\begin{pgfscope}%
\definecolor{textcolor}{rgb}{0.000000,0.000000,0.000000}%
\pgfsetstrokecolor{textcolor}%
\pgfsetfillcolor{textcolor}%
\pgftext[x=0.701046in, y=0.988561in, left, base]{\color{textcolor}\sffamily\fontsize{16.000000}{19.200000}\selectfont 0}%
\end{pgfscope}%
\begin{pgfscope}%
\pgfpathrectangle{\pgfqpoint{0.939653in}{0.865000in}}{\pgfqpoint{5.595423in}{4.575556in}}%
\pgfusepath{clip}%
\pgfsetrectcap%
\pgfsetroundjoin%
\pgfsetlinewidth{0.803000pt}%
\definecolor{currentstroke}{rgb}{0.690196,0.690196,0.690196}%
\pgfsetstrokecolor{currentstroke}%
\pgfsetdash{}{0pt}%
\pgfpathmoveto{\pgfqpoint{0.939653in}{1.790152in}}%
\pgfpathlineto{\pgfqpoint{6.535076in}{1.790152in}}%
\pgfusepath{stroke}%
\end{pgfscope}%
\begin{pgfscope}%
\pgfsetbuttcap%
\pgfsetroundjoin%
\definecolor{currentfill}{rgb}{0.000000,0.000000,0.000000}%
\pgfsetfillcolor{currentfill}%
\pgfsetlinewidth{0.803000pt}%
\definecolor{currentstroke}{rgb}{0.000000,0.000000,0.000000}%
\pgfsetstrokecolor{currentstroke}%
\pgfsetdash{}{0pt}%
\pgfsys@defobject{currentmarker}{\pgfqpoint{-0.048611in}{0.000000in}}{\pgfqpoint{-0.000000in}{0.000000in}}{%
\pgfpathmoveto{\pgfqpoint{-0.000000in}{0.000000in}}%
\pgfpathlineto{\pgfqpoint{-0.048611in}{0.000000in}}%
\pgfusepath{stroke,fill}%
}%
\begin{pgfscope}%
\pgfsys@transformshift{0.939653in}{1.790152in}%
\pgfsys@useobject{currentmarker}{}%
\end{pgfscope}%
\end{pgfscope}%
\begin{pgfscope}%
\definecolor{textcolor}{rgb}{0.000000,0.000000,0.000000}%
\pgfsetstrokecolor{textcolor}%
\pgfsetfillcolor{textcolor}%
\pgftext[x=0.701046in, y=1.705733in, left, base]{\color{textcolor}\sffamily\fontsize{16.000000}{19.200000}\selectfont 5}%
\end{pgfscope}%
\begin{pgfscope}%
\pgfpathrectangle{\pgfqpoint{0.939653in}{0.865000in}}{\pgfqpoint{5.595423in}{4.575556in}}%
\pgfusepath{clip}%
\pgfsetrectcap%
\pgfsetroundjoin%
\pgfsetlinewidth{0.803000pt}%
\definecolor{currentstroke}{rgb}{0.690196,0.690196,0.690196}%
\pgfsetstrokecolor{currentstroke}%
\pgfsetdash{}{0pt}%
\pgfpathmoveto{\pgfqpoint{0.939653in}{2.507323in}}%
\pgfpathlineto{\pgfqpoint{6.535076in}{2.507323in}}%
\pgfusepath{stroke}%
\end{pgfscope}%
\begin{pgfscope}%
\pgfsetbuttcap%
\pgfsetroundjoin%
\definecolor{currentfill}{rgb}{0.000000,0.000000,0.000000}%
\pgfsetfillcolor{currentfill}%
\pgfsetlinewidth{0.803000pt}%
\definecolor{currentstroke}{rgb}{0.000000,0.000000,0.000000}%
\pgfsetstrokecolor{currentstroke}%
\pgfsetdash{}{0pt}%
\pgfsys@defobject{currentmarker}{\pgfqpoint{-0.048611in}{0.000000in}}{\pgfqpoint{-0.000000in}{0.000000in}}{%
\pgfpathmoveto{\pgfqpoint{-0.000000in}{0.000000in}}%
\pgfpathlineto{\pgfqpoint{-0.048611in}{0.000000in}}%
\pgfusepath{stroke,fill}%
}%
\begin{pgfscope}%
\pgfsys@transformshift{0.939653in}{2.507323in}%
\pgfsys@useobject{currentmarker}{}%
\end{pgfscope}%
\end{pgfscope}%
\begin{pgfscope}%
\definecolor{textcolor}{rgb}{0.000000,0.000000,0.000000}%
\pgfsetstrokecolor{textcolor}%
\pgfsetfillcolor{textcolor}%
\pgftext[x=0.559661in, y=2.422905in, left, base]{\color{textcolor}\sffamily\fontsize{16.000000}{19.200000}\selectfont 10}%
\end{pgfscope}%
\begin{pgfscope}%
\pgfpathrectangle{\pgfqpoint{0.939653in}{0.865000in}}{\pgfqpoint{5.595423in}{4.575556in}}%
\pgfusepath{clip}%
\pgfsetrectcap%
\pgfsetroundjoin%
\pgfsetlinewidth{0.803000pt}%
\definecolor{currentstroke}{rgb}{0.690196,0.690196,0.690196}%
\pgfsetstrokecolor{currentstroke}%
\pgfsetdash{}{0pt}%
\pgfpathmoveto{\pgfqpoint{0.939653in}{3.224495in}}%
\pgfpathlineto{\pgfqpoint{6.535076in}{3.224495in}}%
\pgfusepath{stroke}%
\end{pgfscope}%
\begin{pgfscope}%
\pgfsetbuttcap%
\pgfsetroundjoin%
\definecolor{currentfill}{rgb}{0.000000,0.000000,0.000000}%
\pgfsetfillcolor{currentfill}%
\pgfsetlinewidth{0.803000pt}%
\definecolor{currentstroke}{rgb}{0.000000,0.000000,0.000000}%
\pgfsetstrokecolor{currentstroke}%
\pgfsetdash{}{0pt}%
\pgfsys@defobject{currentmarker}{\pgfqpoint{-0.048611in}{0.000000in}}{\pgfqpoint{-0.000000in}{0.000000in}}{%
\pgfpathmoveto{\pgfqpoint{-0.000000in}{0.000000in}}%
\pgfpathlineto{\pgfqpoint{-0.048611in}{0.000000in}}%
\pgfusepath{stroke,fill}%
}%
\begin{pgfscope}%
\pgfsys@transformshift{0.939653in}{3.224495in}%
\pgfsys@useobject{currentmarker}{}%
\end{pgfscope}%
\end{pgfscope}%
\begin{pgfscope}%
\definecolor{textcolor}{rgb}{0.000000,0.000000,0.000000}%
\pgfsetstrokecolor{textcolor}%
\pgfsetfillcolor{textcolor}%
\pgftext[x=0.559661in, y=3.140077in, left, base]{\color{textcolor}\sffamily\fontsize{16.000000}{19.200000}\selectfont 15}%
\end{pgfscope}%
\begin{pgfscope}%
\pgfpathrectangle{\pgfqpoint{0.939653in}{0.865000in}}{\pgfqpoint{5.595423in}{4.575556in}}%
\pgfusepath{clip}%
\pgfsetrectcap%
\pgfsetroundjoin%
\pgfsetlinewidth{0.803000pt}%
\definecolor{currentstroke}{rgb}{0.690196,0.690196,0.690196}%
\pgfsetstrokecolor{currentstroke}%
\pgfsetdash{}{0pt}%
\pgfpathmoveto{\pgfqpoint{0.939653in}{3.941667in}}%
\pgfpathlineto{\pgfqpoint{6.535076in}{3.941667in}}%
\pgfusepath{stroke}%
\end{pgfscope}%
\begin{pgfscope}%
\pgfsetbuttcap%
\pgfsetroundjoin%
\definecolor{currentfill}{rgb}{0.000000,0.000000,0.000000}%
\pgfsetfillcolor{currentfill}%
\pgfsetlinewidth{0.803000pt}%
\definecolor{currentstroke}{rgb}{0.000000,0.000000,0.000000}%
\pgfsetstrokecolor{currentstroke}%
\pgfsetdash{}{0pt}%
\pgfsys@defobject{currentmarker}{\pgfqpoint{-0.048611in}{0.000000in}}{\pgfqpoint{-0.000000in}{0.000000in}}{%
\pgfpathmoveto{\pgfqpoint{-0.000000in}{0.000000in}}%
\pgfpathlineto{\pgfqpoint{-0.048611in}{0.000000in}}%
\pgfusepath{stroke,fill}%
}%
\begin{pgfscope}%
\pgfsys@transformshift{0.939653in}{3.941667in}%
\pgfsys@useobject{currentmarker}{}%
\end{pgfscope}%
\end{pgfscope}%
\begin{pgfscope}%
\definecolor{textcolor}{rgb}{0.000000,0.000000,0.000000}%
\pgfsetstrokecolor{textcolor}%
\pgfsetfillcolor{textcolor}%
\pgftext[x=0.559661in, y=3.857248in, left, base]{\color{textcolor}\sffamily\fontsize{16.000000}{19.200000}\selectfont 20}%
\end{pgfscope}%
\begin{pgfscope}%
\pgfpathrectangle{\pgfqpoint{0.939653in}{0.865000in}}{\pgfqpoint{5.595423in}{4.575556in}}%
\pgfusepath{clip}%
\pgfsetrectcap%
\pgfsetroundjoin%
\pgfsetlinewidth{0.803000pt}%
\definecolor{currentstroke}{rgb}{0.690196,0.690196,0.690196}%
\pgfsetstrokecolor{currentstroke}%
\pgfsetdash{}{0pt}%
\pgfpathmoveto{\pgfqpoint{0.939653in}{4.658838in}}%
\pgfpathlineto{\pgfqpoint{6.535076in}{4.658838in}}%
\pgfusepath{stroke}%
\end{pgfscope}%
\begin{pgfscope}%
\pgfsetbuttcap%
\pgfsetroundjoin%
\definecolor{currentfill}{rgb}{0.000000,0.000000,0.000000}%
\pgfsetfillcolor{currentfill}%
\pgfsetlinewidth{0.803000pt}%
\definecolor{currentstroke}{rgb}{0.000000,0.000000,0.000000}%
\pgfsetstrokecolor{currentstroke}%
\pgfsetdash{}{0pt}%
\pgfsys@defobject{currentmarker}{\pgfqpoint{-0.048611in}{0.000000in}}{\pgfqpoint{-0.000000in}{0.000000in}}{%
\pgfpathmoveto{\pgfqpoint{-0.000000in}{0.000000in}}%
\pgfpathlineto{\pgfqpoint{-0.048611in}{0.000000in}}%
\pgfusepath{stroke,fill}%
}%
\begin{pgfscope}%
\pgfsys@transformshift{0.939653in}{4.658838in}%
\pgfsys@useobject{currentmarker}{}%
\end{pgfscope}%
\end{pgfscope}%
\begin{pgfscope}%
\definecolor{textcolor}{rgb}{0.000000,0.000000,0.000000}%
\pgfsetstrokecolor{textcolor}%
\pgfsetfillcolor{textcolor}%
\pgftext[x=0.559661in, y=4.574420in, left, base]{\color{textcolor}\sffamily\fontsize{16.000000}{19.200000}\selectfont 25}%
\end{pgfscope}%
\begin{pgfscope}%
\pgfpathrectangle{\pgfqpoint{0.939653in}{0.865000in}}{\pgfqpoint{5.595423in}{4.575556in}}%
\pgfusepath{clip}%
\pgfsetrectcap%
\pgfsetroundjoin%
\pgfsetlinewidth{0.803000pt}%
\definecolor{currentstroke}{rgb}{0.690196,0.690196,0.690196}%
\pgfsetstrokecolor{currentstroke}%
\pgfsetdash{}{0pt}%
\pgfpathmoveto{\pgfqpoint{0.939653in}{5.376010in}}%
\pgfpathlineto{\pgfqpoint{6.535076in}{5.376010in}}%
\pgfusepath{stroke}%
\end{pgfscope}%
\begin{pgfscope}%
\pgfsetbuttcap%
\pgfsetroundjoin%
\definecolor{currentfill}{rgb}{0.000000,0.000000,0.000000}%
\pgfsetfillcolor{currentfill}%
\pgfsetlinewidth{0.803000pt}%
\definecolor{currentstroke}{rgb}{0.000000,0.000000,0.000000}%
\pgfsetstrokecolor{currentstroke}%
\pgfsetdash{}{0pt}%
\pgfsys@defobject{currentmarker}{\pgfqpoint{-0.048611in}{0.000000in}}{\pgfqpoint{-0.000000in}{0.000000in}}{%
\pgfpathmoveto{\pgfqpoint{-0.000000in}{0.000000in}}%
\pgfpathlineto{\pgfqpoint{-0.048611in}{0.000000in}}%
\pgfusepath{stroke,fill}%
}%
\begin{pgfscope}%
\pgfsys@transformshift{0.939653in}{5.376010in}%
\pgfsys@useobject{currentmarker}{}%
\end{pgfscope}%
\end{pgfscope}%
\begin{pgfscope}%
\definecolor{textcolor}{rgb}{0.000000,0.000000,0.000000}%
\pgfsetstrokecolor{textcolor}%
\pgfsetfillcolor{textcolor}%
\pgftext[x=0.559661in, y=5.291592in, left, base]{\color{textcolor}\sffamily\fontsize{16.000000}{19.200000}\selectfont 30}%
\end{pgfscope}%
\begin{pgfscope}%
\definecolor{textcolor}{rgb}{0.000000,0.000000,0.000000}%
\pgfsetstrokecolor{textcolor}%
\pgfsetfillcolor{textcolor}%
\pgftext[x=0.504106in,y=3.152778in,,bottom,rotate=90.000000]{\color{textcolor}\sffamily\fontsize{20.000000}{24.000000}\selectfont y}%
\end{pgfscope}%
\begin{pgfscope}%
\pgfpathrectangle{\pgfqpoint{0.939653in}{0.865000in}}{\pgfqpoint{5.595423in}{4.575556in}}%
\pgfusepath{clip}%
\pgfsetrectcap%
\pgfsetroundjoin%
\pgfsetlinewidth{1.505625pt}%
\definecolor{currentstroke}{rgb}{0.121569,0.466667,0.705882}%
\pgfsetstrokecolor{currentstroke}%
\pgfsetdash{}{0pt}%
\pgfpathmoveto{\pgfqpoint{3.737364in}{1.072980in}}%
\pgfpathlineto{\pgfqpoint{4.269074in}{1.216414in}}%
\pgfpathlineto{\pgfqpoint{4.777546in}{1.359848in}}%
\pgfpathlineto{\pgfqpoint{5.240557in}{1.503283in}}%
\pgfpathlineto{\pgfqpoint{5.637871in}{1.646717in}}%
\pgfpathlineto{\pgfqpoint{5.952124in}{1.790152in}}%
\pgfpathlineto{\pgfqpoint{6.169581in}{1.933586in}}%
\pgfpathlineto{\pgfqpoint{6.280738in}{2.077020in}}%
\pgfpathlineto{\pgfqpoint{6.280738in}{2.220455in}}%
\pgfpathlineto{\pgfqpoint{6.169581in}{2.363889in}}%
\pgfpathlineto{\pgfqpoint{5.952124in}{2.507323in}}%
\pgfpathlineto{\pgfqpoint{5.637871in}{2.650758in}}%
\pgfpathlineto{\pgfqpoint{5.240557in}{2.794192in}}%
\pgfpathlineto{\pgfqpoint{4.777546in}{2.937626in}}%
\pgfpathlineto{\pgfqpoint{4.269074in}{3.081061in}}%
\pgfpathlineto{\pgfqpoint{3.737364in}{3.224495in}}%
\pgfpathlineto{\pgfqpoint{3.205654in}{3.367929in}}%
\pgfpathlineto{\pgfqpoint{2.697183in}{3.511364in}}%
\pgfpathlineto{\pgfqpoint{2.234172in}{3.654798in}}%
\pgfpathlineto{\pgfqpoint{1.836858in}{3.798232in}}%
\pgfpathlineto{\pgfqpoint{1.522605in}{3.941667in}}%
\pgfpathlineto{\pgfqpoint{1.305148in}{4.085101in}}%
\pgfpathlineto{\pgfqpoint{1.193990in}{4.228535in}}%
\pgfpathlineto{\pgfqpoint{1.193990in}{4.371970in}}%
\pgfpathlineto{\pgfqpoint{1.305148in}{4.515404in}}%
\pgfpathlineto{\pgfqpoint{1.522605in}{4.658838in}}%
\pgfpathlineto{\pgfqpoint{1.836858in}{4.802273in}}%
\pgfpathlineto{\pgfqpoint{2.234172in}{4.945707in}}%
\pgfpathlineto{\pgfqpoint{2.697183in}{5.089141in}}%
\pgfpathlineto{\pgfqpoint{3.205654in}{5.232576in}}%
\pgfusepath{stroke}%
\end{pgfscope}%
\begin{pgfscope}%
\pgfpathrectangle{\pgfqpoint{0.939653in}{0.865000in}}{\pgfqpoint{5.595423in}{4.575556in}}%
\pgfusepath{clip}%
\pgfsetrectcap%
\pgfsetroundjoin%
\pgfsetlinewidth{1.505625pt}%
\definecolor{currentstroke}{rgb}{1.000000,0.498039,0.054902}%
\pgfsetstrokecolor{currentstroke}%
\pgfsetdash{}{0pt}%
\pgfpathmoveto{\pgfqpoint{3.737364in}{1.072980in}}%
\pgfpathlineto{\pgfqpoint{3.737376in}{1.216414in}}%
\pgfpathlineto{\pgfqpoint{3.737388in}{1.359848in}}%
\pgfpathlineto{\pgfqpoint{3.737398in}{1.503283in}}%
\pgfpathlineto{\pgfqpoint{3.737407in}{1.646717in}}%
\pgfpathlineto{\pgfqpoint{3.737414in}{1.790152in}}%
\pgfpathlineto{\pgfqpoint{3.737419in}{1.933586in}}%
\pgfpathlineto{\pgfqpoint{3.737422in}{2.077020in}}%
\pgfpathlineto{\pgfqpoint{3.737422in}{2.220455in}}%
\pgfpathlineto{\pgfqpoint{3.737419in}{2.363889in}}%
\pgfpathlineto{\pgfqpoint{3.737414in}{2.507323in}}%
\pgfpathlineto{\pgfqpoint{3.737407in}{2.650758in}}%
\pgfpathlineto{\pgfqpoint{3.737398in}{2.794192in}}%
\pgfpathlineto{\pgfqpoint{3.737388in}{2.937626in}}%
\pgfpathlineto{\pgfqpoint{3.737376in}{3.081061in}}%
\pgfpathlineto{\pgfqpoint{3.737364in}{3.224495in}}%
\pgfpathlineto{\pgfqpoint{3.737352in}{3.367929in}}%
\pgfpathlineto{\pgfqpoint{3.737341in}{3.511364in}}%
\pgfpathlineto{\pgfqpoint{3.737330in}{3.654798in}}%
\pgfpathlineto{\pgfqpoint{3.737321in}{3.798232in}}%
\pgfpathlineto{\pgfqpoint{3.737314in}{3.941667in}}%
\pgfpathlineto{\pgfqpoint{3.737309in}{4.085101in}}%
\pgfpathlineto{\pgfqpoint{3.737307in}{4.228535in}}%
\pgfpathlineto{\pgfqpoint{3.737307in}{4.371970in}}%
\pgfpathlineto{\pgfqpoint{3.737309in}{4.515404in}}%
\pgfpathlineto{\pgfqpoint{3.737314in}{4.658838in}}%
\pgfpathlineto{\pgfqpoint{3.737321in}{4.802273in}}%
\pgfpathlineto{\pgfqpoint{3.737330in}{4.945707in}}%
\pgfpathlineto{\pgfqpoint{3.737341in}{5.089141in}}%
\pgfpathlineto{\pgfqpoint{3.737352in}{5.232576in}}%
\pgfusepath{stroke}%
\end{pgfscope}%
\begin{pgfscope}%
\pgfpathrectangle{\pgfqpoint{0.939653in}{0.865000in}}{\pgfqpoint{5.595423in}{4.575556in}}%
\pgfusepath{clip}%
\pgfsetrectcap%
\pgfsetroundjoin%
\pgfsetlinewidth{1.505625pt}%
\definecolor{currentstroke}{rgb}{0.172549,0.627451,0.172549}%
\pgfsetstrokecolor{currentstroke}%
\pgfsetdash{}{0pt}%
\pgfpathmoveto{\pgfqpoint{3.737364in}{1.072980in}}%
\pgfpathlineto{\pgfqpoint{3.737364in}{1.216414in}}%
\pgfpathlineto{\pgfqpoint{3.737364in}{1.359848in}}%
\pgfpathlineto{\pgfqpoint{3.737364in}{1.503283in}}%
\pgfpathlineto{\pgfqpoint{3.737364in}{1.646717in}}%
\pgfpathlineto{\pgfqpoint{3.737364in}{1.790152in}}%
\pgfpathlineto{\pgfqpoint{3.737364in}{1.933586in}}%
\pgfpathlineto{\pgfqpoint{3.737364in}{2.077020in}}%
\pgfpathlineto{\pgfqpoint{3.737364in}{2.220455in}}%
\pgfpathlineto{\pgfqpoint{3.737364in}{2.363889in}}%
\pgfpathlineto{\pgfqpoint{3.737364in}{2.507323in}}%
\pgfpathlineto{\pgfqpoint{3.737364in}{2.650758in}}%
\pgfpathlineto{\pgfqpoint{3.737364in}{2.794192in}}%
\pgfpathlineto{\pgfqpoint{3.737364in}{2.937626in}}%
\pgfpathlineto{\pgfqpoint{3.737364in}{3.081061in}}%
\pgfpathlineto{\pgfqpoint{3.737364in}{3.224495in}}%
\pgfpathlineto{\pgfqpoint{3.737364in}{3.367929in}}%
\pgfpathlineto{\pgfqpoint{3.737364in}{3.511364in}}%
\pgfpathlineto{\pgfqpoint{3.737364in}{3.654798in}}%
\pgfpathlineto{\pgfqpoint{3.737364in}{3.798232in}}%
\pgfpathlineto{\pgfqpoint{3.737364in}{3.941667in}}%
\pgfpathlineto{\pgfqpoint{3.737364in}{4.085101in}}%
\pgfpathlineto{\pgfqpoint{3.737364in}{4.228535in}}%
\pgfpathlineto{\pgfqpoint{3.737364in}{4.371970in}}%
\pgfpathlineto{\pgfqpoint{3.737364in}{4.515404in}}%
\pgfpathlineto{\pgfqpoint{3.737364in}{4.658838in}}%
\pgfpathlineto{\pgfqpoint{3.737364in}{4.802273in}}%
\pgfpathlineto{\pgfqpoint{3.737364in}{4.945707in}}%
\pgfpathlineto{\pgfqpoint{3.737364in}{5.089141in}}%
\pgfpathlineto{\pgfqpoint{3.737364in}{5.232576in}}%
\pgfusepath{stroke}%
\end{pgfscope}%
\begin{pgfscope}%
\pgfpathrectangle{\pgfqpoint{0.939653in}{0.865000in}}{\pgfqpoint{5.595423in}{4.575556in}}%
\pgfusepath{clip}%
\pgfsetrectcap%
\pgfsetroundjoin%
\pgfsetlinewidth{1.505625pt}%
\definecolor{currentstroke}{rgb}{0.839216,0.152941,0.156863}%
\pgfsetstrokecolor{currentstroke}%
\pgfsetdash{}{0pt}%
\pgfpathmoveto{\pgfqpoint{3.737364in}{1.072980in}}%
\pgfpathlineto{\pgfqpoint{3.737364in}{1.216414in}}%
\pgfpathlineto{\pgfqpoint{3.737364in}{1.359848in}}%
\pgfpathlineto{\pgfqpoint{3.737364in}{1.503283in}}%
\pgfpathlineto{\pgfqpoint{3.737364in}{1.646717in}}%
\pgfpathlineto{\pgfqpoint{3.737364in}{1.790152in}}%
\pgfpathlineto{\pgfqpoint{3.737364in}{1.933586in}}%
\pgfpathlineto{\pgfqpoint{3.737364in}{2.077020in}}%
\pgfpathlineto{\pgfqpoint{3.737364in}{2.220455in}}%
\pgfpathlineto{\pgfqpoint{3.737364in}{2.363889in}}%
\pgfpathlineto{\pgfqpoint{3.737364in}{2.507323in}}%
\pgfpathlineto{\pgfqpoint{3.737364in}{2.650758in}}%
\pgfpathlineto{\pgfqpoint{3.737364in}{2.794192in}}%
\pgfpathlineto{\pgfqpoint{3.737364in}{2.937626in}}%
\pgfpathlineto{\pgfqpoint{3.737364in}{3.081061in}}%
\pgfpathlineto{\pgfqpoint{3.737364in}{3.224495in}}%
\pgfpathlineto{\pgfqpoint{3.737364in}{3.367929in}}%
\pgfpathlineto{\pgfqpoint{3.737364in}{3.511364in}}%
\pgfpathlineto{\pgfqpoint{3.737364in}{3.654798in}}%
\pgfpathlineto{\pgfqpoint{3.737364in}{3.798232in}}%
\pgfpathlineto{\pgfqpoint{3.737364in}{3.941667in}}%
\pgfpathlineto{\pgfqpoint{3.737364in}{4.085101in}}%
\pgfpathlineto{\pgfqpoint{3.737364in}{4.228535in}}%
\pgfpathlineto{\pgfqpoint{3.737364in}{4.371970in}}%
\pgfpathlineto{\pgfqpoint{3.737364in}{4.515404in}}%
\pgfpathlineto{\pgfqpoint{3.737364in}{4.658838in}}%
\pgfpathlineto{\pgfqpoint{3.737364in}{4.802273in}}%
\pgfpathlineto{\pgfqpoint{3.737364in}{4.945707in}}%
\pgfpathlineto{\pgfqpoint{3.737364in}{5.089141in}}%
\pgfpathlineto{\pgfqpoint{3.737364in}{5.232576in}}%
\pgfusepath{stroke}%
\end{pgfscope}%
\begin{pgfscope}%
\pgfpathrectangle{\pgfqpoint{0.939653in}{0.865000in}}{\pgfqpoint{5.595423in}{4.575556in}}%
\pgfusepath{clip}%
\pgfsetrectcap%
\pgfsetroundjoin%
\pgfsetlinewidth{1.505625pt}%
\definecolor{currentstroke}{rgb}{0.580392,0.403922,0.741176}%
\pgfsetstrokecolor{currentstroke}%
\pgfsetdash{}{0pt}%
\pgfpathmoveto{\pgfqpoint{3.737364in}{1.072980in}}%
\pgfpathlineto{\pgfqpoint{3.737364in}{1.216414in}}%
\pgfpathlineto{\pgfqpoint{3.737364in}{1.359848in}}%
\pgfpathlineto{\pgfqpoint{3.737364in}{1.503283in}}%
\pgfpathlineto{\pgfqpoint{3.737364in}{1.646717in}}%
\pgfpathlineto{\pgfqpoint{3.737364in}{1.790152in}}%
\pgfpathlineto{\pgfqpoint{3.737364in}{1.933586in}}%
\pgfpathlineto{\pgfqpoint{3.737364in}{2.077020in}}%
\pgfpathlineto{\pgfqpoint{3.737364in}{2.220455in}}%
\pgfpathlineto{\pgfqpoint{3.737364in}{2.363889in}}%
\pgfpathlineto{\pgfqpoint{3.737364in}{2.507323in}}%
\pgfpathlineto{\pgfqpoint{3.737364in}{2.650758in}}%
\pgfpathlineto{\pgfqpoint{3.737364in}{2.794192in}}%
\pgfpathlineto{\pgfqpoint{3.737364in}{2.937626in}}%
\pgfpathlineto{\pgfqpoint{3.737364in}{3.081061in}}%
\pgfpathlineto{\pgfqpoint{3.737364in}{3.224495in}}%
\pgfpathlineto{\pgfqpoint{3.737364in}{3.367929in}}%
\pgfpathlineto{\pgfqpoint{3.737364in}{3.511364in}}%
\pgfpathlineto{\pgfqpoint{3.737364in}{3.654798in}}%
\pgfpathlineto{\pgfqpoint{3.737364in}{3.798232in}}%
\pgfpathlineto{\pgfqpoint{3.737364in}{3.941667in}}%
\pgfpathlineto{\pgfqpoint{3.737364in}{4.085101in}}%
\pgfpathlineto{\pgfqpoint{3.737364in}{4.228535in}}%
\pgfpathlineto{\pgfqpoint{3.737364in}{4.371970in}}%
\pgfpathlineto{\pgfqpoint{3.737364in}{4.515404in}}%
\pgfpathlineto{\pgfqpoint{3.737364in}{4.658838in}}%
\pgfpathlineto{\pgfqpoint{3.737364in}{4.802273in}}%
\pgfpathlineto{\pgfqpoint{3.737364in}{4.945707in}}%
\pgfpathlineto{\pgfqpoint{3.737364in}{5.089141in}}%
\pgfpathlineto{\pgfqpoint{3.737364in}{5.232576in}}%
\pgfusepath{stroke}%
\end{pgfscope}%
\begin{pgfscope}%
\pgfpathrectangle{\pgfqpoint{0.939653in}{0.865000in}}{\pgfqpoint{5.595423in}{4.575556in}}%
\pgfusepath{clip}%
\pgfsetrectcap%
\pgfsetroundjoin%
\pgfsetlinewidth{1.505625pt}%
\definecolor{currentstroke}{rgb}{0.549020,0.337255,0.294118}%
\pgfsetstrokecolor{currentstroke}%
\pgfsetdash{}{0pt}%
\pgfpathmoveto{\pgfqpoint{3.737364in}{1.072980in}}%
\pgfpathlineto{\pgfqpoint{3.737364in}{1.216414in}}%
\pgfpathlineto{\pgfqpoint{3.737364in}{1.359848in}}%
\pgfpathlineto{\pgfqpoint{3.737364in}{1.503283in}}%
\pgfpathlineto{\pgfqpoint{3.737364in}{1.646717in}}%
\pgfpathlineto{\pgfqpoint{3.737364in}{1.790152in}}%
\pgfpathlineto{\pgfqpoint{3.737364in}{1.933586in}}%
\pgfpathlineto{\pgfqpoint{3.737364in}{2.077020in}}%
\pgfpathlineto{\pgfqpoint{3.737364in}{2.220455in}}%
\pgfpathlineto{\pgfqpoint{3.737364in}{2.363889in}}%
\pgfpathlineto{\pgfqpoint{3.737364in}{2.507323in}}%
\pgfpathlineto{\pgfqpoint{3.737364in}{2.650758in}}%
\pgfpathlineto{\pgfqpoint{3.737364in}{2.794192in}}%
\pgfpathlineto{\pgfqpoint{3.737364in}{2.937626in}}%
\pgfpathlineto{\pgfqpoint{3.737364in}{3.081061in}}%
\pgfpathlineto{\pgfqpoint{3.737364in}{3.224495in}}%
\pgfpathlineto{\pgfqpoint{3.737364in}{3.367929in}}%
\pgfpathlineto{\pgfqpoint{3.737364in}{3.511364in}}%
\pgfpathlineto{\pgfqpoint{3.737364in}{3.654798in}}%
\pgfpathlineto{\pgfqpoint{3.737364in}{3.798232in}}%
\pgfpathlineto{\pgfqpoint{3.737364in}{3.941667in}}%
\pgfpathlineto{\pgfqpoint{3.737364in}{4.085101in}}%
\pgfpathlineto{\pgfqpoint{3.737364in}{4.228535in}}%
\pgfpathlineto{\pgfqpoint{3.737364in}{4.371970in}}%
\pgfpathlineto{\pgfqpoint{3.737364in}{4.515404in}}%
\pgfpathlineto{\pgfqpoint{3.737364in}{4.658838in}}%
\pgfpathlineto{\pgfqpoint{3.737364in}{4.802273in}}%
\pgfpathlineto{\pgfqpoint{3.737364in}{4.945707in}}%
\pgfpathlineto{\pgfqpoint{3.737364in}{5.089141in}}%
\pgfpathlineto{\pgfqpoint{3.737364in}{5.232576in}}%
\pgfusepath{stroke}%
\end{pgfscope}%
\begin{pgfscope}%
\pgfsetrectcap%
\pgfsetmiterjoin%
\pgfsetlinewidth{0.803000pt}%
\definecolor{currentstroke}{rgb}{0.000000,0.000000,0.000000}%
\pgfsetstrokecolor{currentstroke}%
\pgfsetdash{}{0pt}%
\pgfpathmoveto{\pgfqpoint{0.939653in}{0.865000in}}%
\pgfpathlineto{\pgfqpoint{0.939653in}{5.440556in}}%
\pgfusepath{stroke}%
\end{pgfscope}%
\begin{pgfscope}%
\pgfsetrectcap%
\pgfsetmiterjoin%
\pgfsetlinewidth{0.803000pt}%
\definecolor{currentstroke}{rgb}{0.000000,0.000000,0.000000}%
\pgfsetstrokecolor{currentstroke}%
\pgfsetdash{}{0pt}%
\pgfpathmoveto{\pgfqpoint{6.535076in}{0.865000in}}%
\pgfpathlineto{\pgfqpoint{6.535076in}{5.440556in}}%
\pgfusepath{stroke}%
\end{pgfscope}%
\begin{pgfscope}%
\pgfsetrectcap%
\pgfsetmiterjoin%
\pgfsetlinewidth{0.803000pt}%
\definecolor{currentstroke}{rgb}{0.000000,0.000000,0.000000}%
\pgfsetstrokecolor{currentstroke}%
\pgfsetdash{}{0pt}%
\pgfpathmoveto{\pgfqpoint{0.939653in}{0.865000in}}%
\pgfpathlineto{\pgfqpoint{6.535076in}{0.865000in}}%
\pgfusepath{stroke}%
\end{pgfscope}%
\begin{pgfscope}%
\pgfsetrectcap%
\pgfsetmiterjoin%
\pgfsetlinewidth{0.803000pt}%
\definecolor{currentstroke}{rgb}{0.000000,0.000000,0.000000}%
\pgfsetstrokecolor{currentstroke}%
\pgfsetdash{}{0pt}%
\pgfpathmoveto{\pgfqpoint{0.939653in}{5.440556in}}%
\pgfpathlineto{\pgfqpoint{6.535076in}{5.440556in}}%
\pgfusepath{stroke}%
\end{pgfscope}%
\begin{pgfscope}%
\definecolor{textcolor}{rgb}{0.000000,0.000000,0.000000}%
\pgfsetstrokecolor{textcolor}%
\pgfsetfillcolor{textcolor}%
\pgftext[x=3.737364in,y=5.523889in,,base]{\color{textcolor}\sffamily\fontsize{20.000000}{24.000000}\selectfont Velocity decay (\(\displaystyle \omega\)=0.3)}%
\end{pgfscope}%
\begin{pgfscope}%
\pgfsetbuttcap%
\pgfsetmiterjoin%
\definecolor{currentfill}{rgb}{1.000000,1.000000,1.000000}%
\pgfsetfillcolor{currentfill}%
\pgfsetfillopacity{0.800000}%
\pgfsetlinewidth{1.003750pt}%
\definecolor{currentstroke}{rgb}{0.800000,0.800000,0.800000}%
\pgfsetstrokecolor{currentstroke}%
\pgfsetstrokeopacity{0.800000}%
\pgfsetdash{}{0pt}%
\pgfpathmoveto{\pgfqpoint{1.134097in}{1.003889in}}%
\pgfpathlineto{\pgfqpoint{3.071570in}{1.003889in}}%
\pgfpathquadraticcurveto{\pgfqpoint{3.127126in}{1.003889in}}{\pgfqpoint{3.127126in}{1.059444in}}%
\pgfpathlineto{\pgfqpoint{3.127126in}{3.477954in}}%
\pgfpathquadraticcurveto{\pgfqpoint{3.127126in}{3.533510in}}{\pgfqpoint{3.071570in}{3.533510in}}%
\pgfpathlineto{\pgfqpoint{1.134097in}{3.533510in}}%
\pgfpathquadraticcurveto{\pgfqpoint{1.078542in}{3.533510in}}{\pgfqpoint{1.078542in}{3.477954in}}%
\pgfpathlineto{\pgfqpoint{1.078542in}{1.059444in}}%
\pgfpathquadraticcurveto{\pgfqpoint{1.078542in}{1.003889in}}{\pgfqpoint{1.134097in}{1.003889in}}%
\pgfpathlineto{\pgfqpoint{1.134097in}{1.003889in}}%
\pgfpathclose%
\pgfusepath{stroke,fill}%
\end{pgfscope}%
\begin{pgfscope}%
\pgfsetrectcap%
\pgfsetroundjoin%
\pgfsetlinewidth{1.505625pt}%
\definecolor{currentstroke}{rgb}{0.121569,0.466667,0.705882}%
\pgfsetstrokecolor{currentstroke}%
\pgfsetdash{}{0pt}%
\pgfpathmoveto{\pgfqpoint{1.189653in}{3.308575in}}%
\pgfpathlineto{\pgfqpoint{1.467431in}{3.308575in}}%
\pgfpathlineto{\pgfqpoint{1.745208in}{3.308575in}}%
\pgfusepath{stroke}%
\end{pgfscope}%
\begin{pgfscope}%
\definecolor{textcolor}{rgb}{0.000000,0.000000,0.000000}%
\pgfsetstrokecolor{textcolor}%
\pgfsetfillcolor{textcolor}%
\pgftext[x=1.967431in,y=3.211353in,left,base]{\color{textcolor}\sffamily\fontsize{20.000000}{24.000000}\selectfont t=0}%
\end{pgfscope}%
\begin{pgfscope}%
\pgfsetrectcap%
\pgfsetroundjoin%
\pgfsetlinewidth{1.505625pt}%
\definecolor{currentstroke}{rgb}{1.000000,0.498039,0.054902}%
\pgfsetstrokecolor{currentstroke}%
\pgfsetdash{}{0pt}%
\pgfpathmoveto{\pgfqpoint{1.189653in}{2.900860in}}%
\pgfpathlineto{\pgfqpoint{1.467431in}{2.900860in}}%
\pgfpathlineto{\pgfqpoint{1.745208in}{2.900860in}}%
\pgfusepath{stroke}%
\end{pgfscope}%
\begin{pgfscope}%
\definecolor{textcolor}{rgb}{0.000000,0.000000,0.000000}%
\pgfsetstrokecolor{textcolor}%
\pgfsetfillcolor{textcolor}%
\pgftext[x=1.967431in,y=2.803638in,left,base]{\color{textcolor}\sffamily\fontsize{20.000000}{24.000000}\selectfont t=300}%
\end{pgfscope}%
\begin{pgfscope}%
\pgfsetrectcap%
\pgfsetroundjoin%
\pgfsetlinewidth{1.505625pt}%
\definecolor{currentstroke}{rgb}{0.172549,0.627451,0.172549}%
\pgfsetstrokecolor{currentstroke}%
\pgfsetdash{}{0pt}%
\pgfpathmoveto{\pgfqpoint{1.189653in}{2.493146in}}%
\pgfpathlineto{\pgfqpoint{1.467431in}{2.493146in}}%
\pgfpathlineto{\pgfqpoint{1.745208in}{2.493146in}}%
\pgfusepath{stroke}%
\end{pgfscope}%
\begin{pgfscope}%
\definecolor{textcolor}{rgb}{0.000000,0.000000,0.000000}%
\pgfsetstrokecolor{textcolor}%
\pgfsetfillcolor{textcolor}%
\pgftext[x=1.967431in,y=2.395923in,left,base]{\color{textcolor}\sffamily\fontsize{20.000000}{24.000000}\selectfont t=600}%
\end{pgfscope}%
\begin{pgfscope}%
\pgfsetrectcap%
\pgfsetroundjoin%
\pgfsetlinewidth{1.505625pt}%
\definecolor{currentstroke}{rgb}{0.839216,0.152941,0.156863}%
\pgfsetstrokecolor{currentstroke}%
\pgfsetdash{}{0pt}%
\pgfpathmoveto{\pgfqpoint{1.189653in}{2.085431in}}%
\pgfpathlineto{\pgfqpoint{1.467431in}{2.085431in}}%
\pgfpathlineto{\pgfqpoint{1.745208in}{2.085431in}}%
\pgfusepath{stroke}%
\end{pgfscope}%
\begin{pgfscope}%
\definecolor{textcolor}{rgb}{0.000000,0.000000,0.000000}%
\pgfsetstrokecolor{textcolor}%
\pgfsetfillcolor{textcolor}%
\pgftext[x=1.967431in,y=1.988209in,left,base]{\color{textcolor}\sffamily\fontsize{20.000000}{24.000000}\selectfont t=900}%
\end{pgfscope}%
\begin{pgfscope}%
\pgfsetrectcap%
\pgfsetroundjoin%
\pgfsetlinewidth{1.505625pt}%
\definecolor{currentstroke}{rgb}{0.580392,0.403922,0.741176}%
\pgfsetstrokecolor{currentstroke}%
\pgfsetdash{}{0pt}%
\pgfpathmoveto{\pgfqpoint{1.189653in}{1.677717in}}%
\pgfpathlineto{\pgfqpoint{1.467431in}{1.677717in}}%
\pgfpathlineto{\pgfqpoint{1.745208in}{1.677717in}}%
\pgfusepath{stroke}%
\end{pgfscope}%
\begin{pgfscope}%
\definecolor{textcolor}{rgb}{0.000000,0.000000,0.000000}%
\pgfsetstrokecolor{textcolor}%
\pgfsetfillcolor{textcolor}%
\pgftext[x=1.967431in,y=1.580494in,left,base]{\color{textcolor}\sffamily\fontsize{20.000000}{24.000000}\selectfont t=1200}%
\end{pgfscope}%
\begin{pgfscope}%
\pgfsetrectcap%
\pgfsetroundjoin%
\pgfsetlinewidth{1.505625pt}%
\definecolor{currentstroke}{rgb}{0.549020,0.337255,0.294118}%
\pgfsetstrokecolor{currentstroke}%
\pgfsetdash{}{0pt}%
\pgfpathmoveto{\pgfqpoint{1.189653in}{1.270002in}}%
\pgfpathlineto{\pgfqpoint{1.467431in}{1.270002in}}%
\pgfpathlineto{\pgfqpoint{1.745208in}{1.270002in}}%
\pgfusepath{stroke}%
\end{pgfscope}%
\begin{pgfscope}%
\definecolor{textcolor}{rgb}{0.000000,0.000000,0.000000}%
\pgfsetstrokecolor{textcolor}%
\pgfsetfillcolor{textcolor}%
\pgftext[x=1.967431in,y=1.172780in,left,base]{\color{textcolor}\sffamily\fontsize{20.000000}{24.000000}\selectfont t=1500}%
\end{pgfscope}%
\begin{pgfscope}%
\pgfsetbuttcap%
\pgfsetmiterjoin%
\definecolor{currentfill}{rgb}{1.000000,1.000000,1.000000}%
\pgfsetfillcolor{currentfill}%
\pgfsetlinewidth{0.000000pt}%
\definecolor{currentstroke}{rgb}{0.000000,0.000000,0.000000}%
\pgfsetstrokecolor{currentstroke}%
\pgfsetstrokeopacity{0.000000}%
\pgfsetdash{}{0pt}%
\pgfpathmoveto{\pgfqpoint{7.526319in}{0.865000in}}%
\pgfpathlineto{\pgfqpoint{13.121743in}{0.865000in}}%
\pgfpathlineto{\pgfqpoint{13.121743in}{5.440556in}}%
\pgfpathlineto{\pgfqpoint{7.526319in}{5.440556in}}%
\pgfpathlineto{\pgfqpoint{7.526319in}{0.865000in}}%
\pgfpathclose%
\pgfusepath{fill}%
\end{pgfscope}%
\begin{pgfscope}%
\pgfpathrectangle{\pgfqpoint{7.526319in}{0.865000in}}{\pgfqpoint{5.595423in}{4.575556in}}%
\pgfusepath{clip}%
\pgfsetrectcap%
\pgfsetroundjoin%
\pgfsetlinewidth{0.803000pt}%
\definecolor{currentstroke}{rgb}{0.690196,0.690196,0.690196}%
\pgfsetstrokecolor{currentstroke}%
\pgfsetdash{}{0pt}%
\pgfpathmoveto{\pgfqpoint{7.766647in}{0.865000in}}%
\pgfpathlineto{\pgfqpoint{7.766647in}{5.440556in}}%
\pgfusepath{stroke}%
\end{pgfscope}%
\begin{pgfscope}%
\pgfsetbuttcap%
\pgfsetroundjoin%
\definecolor{currentfill}{rgb}{0.000000,0.000000,0.000000}%
\pgfsetfillcolor{currentfill}%
\pgfsetlinewidth{0.803000pt}%
\definecolor{currentstroke}{rgb}{0.000000,0.000000,0.000000}%
\pgfsetstrokecolor{currentstroke}%
\pgfsetdash{}{0pt}%
\pgfsys@defobject{currentmarker}{\pgfqpoint{0.000000in}{-0.048611in}}{\pgfqpoint{0.000000in}{0.000000in}}{%
\pgfpathmoveto{\pgfqpoint{0.000000in}{0.000000in}}%
\pgfpathlineto{\pgfqpoint{0.000000in}{-0.048611in}}%
\pgfusepath{stroke,fill}%
}%
\begin{pgfscope}%
\pgfsys@transformshift{7.766647in}{0.865000in}%
\pgfsys@useobject{currentmarker}{}%
\end{pgfscope}%
\end{pgfscope}%
\begin{pgfscope}%
\definecolor{textcolor}{rgb}{0.000000,0.000000,0.000000}%
\pgfsetstrokecolor{textcolor}%
\pgfsetfillcolor{textcolor}%
\pgftext[x=7.766647in,y=0.767778in,,top]{\color{textcolor}\sffamily\fontsize{16.000000}{19.200000}\selectfont \ensuremath{-}0.10}%
\end{pgfscope}%
\begin{pgfscope}%
\pgfpathrectangle{\pgfqpoint{7.526319in}{0.865000in}}{\pgfqpoint{5.595423in}{4.575556in}}%
\pgfusepath{clip}%
\pgfsetrectcap%
\pgfsetroundjoin%
\pgfsetlinewidth{0.803000pt}%
\definecolor{currentstroke}{rgb}{0.690196,0.690196,0.690196}%
\pgfsetstrokecolor{currentstroke}%
\pgfsetdash{}{0pt}%
\pgfpathmoveto{\pgfqpoint{9.045339in}{0.865000in}}%
\pgfpathlineto{\pgfqpoint{9.045339in}{5.440556in}}%
\pgfusepath{stroke}%
\end{pgfscope}%
\begin{pgfscope}%
\pgfsetbuttcap%
\pgfsetroundjoin%
\definecolor{currentfill}{rgb}{0.000000,0.000000,0.000000}%
\pgfsetfillcolor{currentfill}%
\pgfsetlinewidth{0.803000pt}%
\definecolor{currentstroke}{rgb}{0.000000,0.000000,0.000000}%
\pgfsetstrokecolor{currentstroke}%
\pgfsetdash{}{0pt}%
\pgfsys@defobject{currentmarker}{\pgfqpoint{0.000000in}{-0.048611in}}{\pgfqpoint{0.000000in}{0.000000in}}{%
\pgfpathmoveto{\pgfqpoint{0.000000in}{0.000000in}}%
\pgfpathlineto{\pgfqpoint{0.000000in}{-0.048611in}}%
\pgfusepath{stroke,fill}%
}%
\begin{pgfscope}%
\pgfsys@transformshift{9.045339in}{0.865000in}%
\pgfsys@useobject{currentmarker}{}%
\end{pgfscope}%
\end{pgfscope}%
\begin{pgfscope}%
\definecolor{textcolor}{rgb}{0.000000,0.000000,0.000000}%
\pgfsetstrokecolor{textcolor}%
\pgfsetfillcolor{textcolor}%
\pgftext[x=9.045339in,y=0.767778in,,top]{\color{textcolor}\sffamily\fontsize{16.000000}{19.200000}\selectfont \ensuremath{-}0.05}%
\end{pgfscope}%
\begin{pgfscope}%
\pgfpathrectangle{\pgfqpoint{7.526319in}{0.865000in}}{\pgfqpoint{5.595423in}{4.575556in}}%
\pgfusepath{clip}%
\pgfsetrectcap%
\pgfsetroundjoin%
\pgfsetlinewidth{0.803000pt}%
\definecolor{currentstroke}{rgb}{0.690196,0.690196,0.690196}%
\pgfsetstrokecolor{currentstroke}%
\pgfsetdash{}{0pt}%
\pgfpathmoveto{\pgfqpoint{10.324031in}{0.865000in}}%
\pgfpathlineto{\pgfqpoint{10.324031in}{5.440556in}}%
\pgfusepath{stroke}%
\end{pgfscope}%
\begin{pgfscope}%
\pgfsetbuttcap%
\pgfsetroundjoin%
\definecolor{currentfill}{rgb}{0.000000,0.000000,0.000000}%
\pgfsetfillcolor{currentfill}%
\pgfsetlinewidth{0.803000pt}%
\definecolor{currentstroke}{rgb}{0.000000,0.000000,0.000000}%
\pgfsetstrokecolor{currentstroke}%
\pgfsetdash{}{0pt}%
\pgfsys@defobject{currentmarker}{\pgfqpoint{0.000000in}{-0.048611in}}{\pgfqpoint{0.000000in}{0.000000in}}{%
\pgfpathmoveto{\pgfqpoint{0.000000in}{0.000000in}}%
\pgfpathlineto{\pgfqpoint{0.000000in}{-0.048611in}}%
\pgfusepath{stroke,fill}%
}%
\begin{pgfscope}%
\pgfsys@transformshift{10.324031in}{0.865000in}%
\pgfsys@useobject{currentmarker}{}%
\end{pgfscope}%
\end{pgfscope}%
\begin{pgfscope}%
\definecolor{textcolor}{rgb}{0.000000,0.000000,0.000000}%
\pgfsetstrokecolor{textcolor}%
\pgfsetfillcolor{textcolor}%
\pgftext[x=10.324031in,y=0.767778in,,top]{\color{textcolor}\sffamily\fontsize{16.000000}{19.200000}\selectfont 0.00}%
\end{pgfscope}%
\begin{pgfscope}%
\pgfpathrectangle{\pgfqpoint{7.526319in}{0.865000in}}{\pgfqpoint{5.595423in}{4.575556in}}%
\pgfusepath{clip}%
\pgfsetrectcap%
\pgfsetroundjoin%
\pgfsetlinewidth{0.803000pt}%
\definecolor{currentstroke}{rgb}{0.690196,0.690196,0.690196}%
\pgfsetstrokecolor{currentstroke}%
\pgfsetdash{}{0pt}%
\pgfpathmoveto{\pgfqpoint{11.602723in}{0.865000in}}%
\pgfpathlineto{\pgfqpoint{11.602723in}{5.440556in}}%
\pgfusepath{stroke}%
\end{pgfscope}%
\begin{pgfscope}%
\pgfsetbuttcap%
\pgfsetroundjoin%
\definecolor{currentfill}{rgb}{0.000000,0.000000,0.000000}%
\pgfsetfillcolor{currentfill}%
\pgfsetlinewidth{0.803000pt}%
\definecolor{currentstroke}{rgb}{0.000000,0.000000,0.000000}%
\pgfsetstrokecolor{currentstroke}%
\pgfsetdash{}{0pt}%
\pgfsys@defobject{currentmarker}{\pgfqpoint{0.000000in}{-0.048611in}}{\pgfqpoint{0.000000in}{0.000000in}}{%
\pgfpathmoveto{\pgfqpoint{0.000000in}{0.000000in}}%
\pgfpathlineto{\pgfqpoint{0.000000in}{-0.048611in}}%
\pgfusepath{stroke,fill}%
}%
\begin{pgfscope}%
\pgfsys@transformshift{11.602723in}{0.865000in}%
\pgfsys@useobject{currentmarker}{}%
\end{pgfscope}%
\end{pgfscope}%
\begin{pgfscope}%
\definecolor{textcolor}{rgb}{0.000000,0.000000,0.000000}%
\pgfsetstrokecolor{textcolor}%
\pgfsetfillcolor{textcolor}%
\pgftext[x=11.602723in,y=0.767778in,,top]{\color{textcolor}\sffamily\fontsize{16.000000}{19.200000}\selectfont 0.05}%
\end{pgfscope}%
\begin{pgfscope}%
\pgfpathrectangle{\pgfqpoint{7.526319in}{0.865000in}}{\pgfqpoint{5.595423in}{4.575556in}}%
\pgfusepath{clip}%
\pgfsetrectcap%
\pgfsetroundjoin%
\pgfsetlinewidth{0.803000pt}%
\definecolor{currentstroke}{rgb}{0.690196,0.690196,0.690196}%
\pgfsetstrokecolor{currentstroke}%
\pgfsetdash{}{0pt}%
\pgfpathmoveto{\pgfqpoint{12.881415in}{0.865000in}}%
\pgfpathlineto{\pgfqpoint{12.881415in}{5.440556in}}%
\pgfusepath{stroke}%
\end{pgfscope}%
\begin{pgfscope}%
\pgfsetbuttcap%
\pgfsetroundjoin%
\definecolor{currentfill}{rgb}{0.000000,0.000000,0.000000}%
\pgfsetfillcolor{currentfill}%
\pgfsetlinewidth{0.803000pt}%
\definecolor{currentstroke}{rgb}{0.000000,0.000000,0.000000}%
\pgfsetstrokecolor{currentstroke}%
\pgfsetdash{}{0pt}%
\pgfsys@defobject{currentmarker}{\pgfqpoint{0.000000in}{-0.048611in}}{\pgfqpoint{0.000000in}{0.000000in}}{%
\pgfpathmoveto{\pgfqpoint{0.000000in}{0.000000in}}%
\pgfpathlineto{\pgfqpoint{0.000000in}{-0.048611in}}%
\pgfusepath{stroke,fill}%
}%
\begin{pgfscope}%
\pgfsys@transformshift{12.881415in}{0.865000in}%
\pgfsys@useobject{currentmarker}{}%
\end{pgfscope}%
\end{pgfscope}%
\begin{pgfscope}%
\definecolor{textcolor}{rgb}{0.000000,0.000000,0.000000}%
\pgfsetstrokecolor{textcolor}%
\pgfsetfillcolor{textcolor}%
\pgftext[x=12.881415in,y=0.767778in,,top]{\color{textcolor}\sffamily\fontsize{16.000000}{19.200000}\selectfont 0.10}%
\end{pgfscope}%
\begin{pgfscope}%
\definecolor{textcolor}{rgb}{0.000000,0.000000,0.000000}%
\pgfsetstrokecolor{textcolor}%
\pgfsetfillcolor{textcolor}%
\pgftext[x=10.324031in,y=0.497162in,,top]{\color{textcolor}\sffamily\fontsize{20.000000}{24.000000}\selectfont velocity}%
\end{pgfscope}%
\begin{pgfscope}%
\pgfpathrectangle{\pgfqpoint{7.526319in}{0.865000in}}{\pgfqpoint{5.595423in}{4.575556in}}%
\pgfusepath{clip}%
\pgfsetrectcap%
\pgfsetroundjoin%
\pgfsetlinewidth{0.803000pt}%
\definecolor{currentstroke}{rgb}{0.690196,0.690196,0.690196}%
\pgfsetstrokecolor{currentstroke}%
\pgfsetdash{}{0pt}%
\pgfpathmoveto{\pgfqpoint{7.526319in}{1.072980in}}%
\pgfpathlineto{\pgfqpoint{13.121743in}{1.072980in}}%
\pgfusepath{stroke}%
\end{pgfscope}%
\begin{pgfscope}%
\pgfsetbuttcap%
\pgfsetroundjoin%
\definecolor{currentfill}{rgb}{0.000000,0.000000,0.000000}%
\pgfsetfillcolor{currentfill}%
\pgfsetlinewidth{0.803000pt}%
\definecolor{currentstroke}{rgb}{0.000000,0.000000,0.000000}%
\pgfsetstrokecolor{currentstroke}%
\pgfsetdash{}{0pt}%
\pgfsys@defobject{currentmarker}{\pgfqpoint{-0.048611in}{0.000000in}}{\pgfqpoint{-0.000000in}{0.000000in}}{%
\pgfpathmoveto{\pgfqpoint{-0.000000in}{0.000000in}}%
\pgfpathlineto{\pgfqpoint{-0.048611in}{0.000000in}}%
\pgfusepath{stroke,fill}%
}%
\begin{pgfscope}%
\pgfsys@transformshift{7.526319in}{1.072980in}%
\pgfsys@useobject{currentmarker}{}%
\end{pgfscope}%
\end{pgfscope}%
\begin{pgfscope}%
\definecolor{textcolor}{rgb}{0.000000,0.000000,0.000000}%
\pgfsetstrokecolor{textcolor}%
\pgfsetfillcolor{textcolor}%
\pgftext[x=7.287713in, y=0.988561in, left, base]{\color{textcolor}\sffamily\fontsize{16.000000}{19.200000}\selectfont 0}%
\end{pgfscope}%
\begin{pgfscope}%
\pgfpathrectangle{\pgfqpoint{7.526319in}{0.865000in}}{\pgfqpoint{5.595423in}{4.575556in}}%
\pgfusepath{clip}%
\pgfsetrectcap%
\pgfsetroundjoin%
\pgfsetlinewidth{0.803000pt}%
\definecolor{currentstroke}{rgb}{0.690196,0.690196,0.690196}%
\pgfsetstrokecolor{currentstroke}%
\pgfsetdash{}{0pt}%
\pgfpathmoveto{\pgfqpoint{7.526319in}{1.790152in}}%
\pgfpathlineto{\pgfqpoint{13.121743in}{1.790152in}}%
\pgfusepath{stroke}%
\end{pgfscope}%
\begin{pgfscope}%
\pgfsetbuttcap%
\pgfsetroundjoin%
\definecolor{currentfill}{rgb}{0.000000,0.000000,0.000000}%
\pgfsetfillcolor{currentfill}%
\pgfsetlinewidth{0.803000pt}%
\definecolor{currentstroke}{rgb}{0.000000,0.000000,0.000000}%
\pgfsetstrokecolor{currentstroke}%
\pgfsetdash{}{0pt}%
\pgfsys@defobject{currentmarker}{\pgfqpoint{-0.048611in}{0.000000in}}{\pgfqpoint{-0.000000in}{0.000000in}}{%
\pgfpathmoveto{\pgfqpoint{-0.000000in}{0.000000in}}%
\pgfpathlineto{\pgfqpoint{-0.048611in}{0.000000in}}%
\pgfusepath{stroke,fill}%
}%
\begin{pgfscope}%
\pgfsys@transformshift{7.526319in}{1.790152in}%
\pgfsys@useobject{currentmarker}{}%
\end{pgfscope}%
\end{pgfscope}%
\begin{pgfscope}%
\definecolor{textcolor}{rgb}{0.000000,0.000000,0.000000}%
\pgfsetstrokecolor{textcolor}%
\pgfsetfillcolor{textcolor}%
\pgftext[x=7.287713in, y=1.705733in, left, base]{\color{textcolor}\sffamily\fontsize{16.000000}{19.200000}\selectfont 5}%
\end{pgfscope}%
\begin{pgfscope}%
\pgfpathrectangle{\pgfqpoint{7.526319in}{0.865000in}}{\pgfqpoint{5.595423in}{4.575556in}}%
\pgfusepath{clip}%
\pgfsetrectcap%
\pgfsetroundjoin%
\pgfsetlinewidth{0.803000pt}%
\definecolor{currentstroke}{rgb}{0.690196,0.690196,0.690196}%
\pgfsetstrokecolor{currentstroke}%
\pgfsetdash{}{0pt}%
\pgfpathmoveto{\pgfqpoint{7.526319in}{2.507323in}}%
\pgfpathlineto{\pgfqpoint{13.121743in}{2.507323in}}%
\pgfusepath{stroke}%
\end{pgfscope}%
\begin{pgfscope}%
\pgfsetbuttcap%
\pgfsetroundjoin%
\definecolor{currentfill}{rgb}{0.000000,0.000000,0.000000}%
\pgfsetfillcolor{currentfill}%
\pgfsetlinewidth{0.803000pt}%
\definecolor{currentstroke}{rgb}{0.000000,0.000000,0.000000}%
\pgfsetstrokecolor{currentstroke}%
\pgfsetdash{}{0pt}%
\pgfsys@defobject{currentmarker}{\pgfqpoint{-0.048611in}{0.000000in}}{\pgfqpoint{-0.000000in}{0.000000in}}{%
\pgfpathmoveto{\pgfqpoint{-0.000000in}{0.000000in}}%
\pgfpathlineto{\pgfqpoint{-0.048611in}{0.000000in}}%
\pgfusepath{stroke,fill}%
}%
\begin{pgfscope}%
\pgfsys@transformshift{7.526319in}{2.507323in}%
\pgfsys@useobject{currentmarker}{}%
\end{pgfscope}%
\end{pgfscope}%
\begin{pgfscope}%
\definecolor{textcolor}{rgb}{0.000000,0.000000,0.000000}%
\pgfsetstrokecolor{textcolor}%
\pgfsetfillcolor{textcolor}%
\pgftext[x=7.146328in, y=2.422905in, left, base]{\color{textcolor}\sffamily\fontsize{16.000000}{19.200000}\selectfont 10}%
\end{pgfscope}%
\begin{pgfscope}%
\pgfpathrectangle{\pgfqpoint{7.526319in}{0.865000in}}{\pgfqpoint{5.595423in}{4.575556in}}%
\pgfusepath{clip}%
\pgfsetrectcap%
\pgfsetroundjoin%
\pgfsetlinewidth{0.803000pt}%
\definecolor{currentstroke}{rgb}{0.690196,0.690196,0.690196}%
\pgfsetstrokecolor{currentstroke}%
\pgfsetdash{}{0pt}%
\pgfpathmoveto{\pgfqpoint{7.526319in}{3.224495in}}%
\pgfpathlineto{\pgfqpoint{13.121743in}{3.224495in}}%
\pgfusepath{stroke}%
\end{pgfscope}%
\begin{pgfscope}%
\pgfsetbuttcap%
\pgfsetroundjoin%
\definecolor{currentfill}{rgb}{0.000000,0.000000,0.000000}%
\pgfsetfillcolor{currentfill}%
\pgfsetlinewidth{0.803000pt}%
\definecolor{currentstroke}{rgb}{0.000000,0.000000,0.000000}%
\pgfsetstrokecolor{currentstroke}%
\pgfsetdash{}{0pt}%
\pgfsys@defobject{currentmarker}{\pgfqpoint{-0.048611in}{0.000000in}}{\pgfqpoint{-0.000000in}{0.000000in}}{%
\pgfpathmoveto{\pgfqpoint{-0.000000in}{0.000000in}}%
\pgfpathlineto{\pgfqpoint{-0.048611in}{0.000000in}}%
\pgfusepath{stroke,fill}%
}%
\begin{pgfscope}%
\pgfsys@transformshift{7.526319in}{3.224495in}%
\pgfsys@useobject{currentmarker}{}%
\end{pgfscope}%
\end{pgfscope}%
\begin{pgfscope}%
\definecolor{textcolor}{rgb}{0.000000,0.000000,0.000000}%
\pgfsetstrokecolor{textcolor}%
\pgfsetfillcolor{textcolor}%
\pgftext[x=7.146328in, y=3.140077in, left, base]{\color{textcolor}\sffamily\fontsize{16.000000}{19.200000}\selectfont 15}%
\end{pgfscope}%
\begin{pgfscope}%
\pgfpathrectangle{\pgfqpoint{7.526319in}{0.865000in}}{\pgfqpoint{5.595423in}{4.575556in}}%
\pgfusepath{clip}%
\pgfsetrectcap%
\pgfsetroundjoin%
\pgfsetlinewidth{0.803000pt}%
\definecolor{currentstroke}{rgb}{0.690196,0.690196,0.690196}%
\pgfsetstrokecolor{currentstroke}%
\pgfsetdash{}{0pt}%
\pgfpathmoveto{\pgfqpoint{7.526319in}{3.941667in}}%
\pgfpathlineto{\pgfqpoint{13.121743in}{3.941667in}}%
\pgfusepath{stroke}%
\end{pgfscope}%
\begin{pgfscope}%
\pgfsetbuttcap%
\pgfsetroundjoin%
\definecolor{currentfill}{rgb}{0.000000,0.000000,0.000000}%
\pgfsetfillcolor{currentfill}%
\pgfsetlinewidth{0.803000pt}%
\definecolor{currentstroke}{rgb}{0.000000,0.000000,0.000000}%
\pgfsetstrokecolor{currentstroke}%
\pgfsetdash{}{0pt}%
\pgfsys@defobject{currentmarker}{\pgfqpoint{-0.048611in}{0.000000in}}{\pgfqpoint{-0.000000in}{0.000000in}}{%
\pgfpathmoveto{\pgfqpoint{-0.000000in}{0.000000in}}%
\pgfpathlineto{\pgfqpoint{-0.048611in}{0.000000in}}%
\pgfusepath{stroke,fill}%
}%
\begin{pgfscope}%
\pgfsys@transformshift{7.526319in}{3.941667in}%
\pgfsys@useobject{currentmarker}{}%
\end{pgfscope}%
\end{pgfscope}%
\begin{pgfscope}%
\definecolor{textcolor}{rgb}{0.000000,0.000000,0.000000}%
\pgfsetstrokecolor{textcolor}%
\pgfsetfillcolor{textcolor}%
\pgftext[x=7.146328in, y=3.857248in, left, base]{\color{textcolor}\sffamily\fontsize{16.000000}{19.200000}\selectfont 20}%
\end{pgfscope}%
\begin{pgfscope}%
\pgfpathrectangle{\pgfqpoint{7.526319in}{0.865000in}}{\pgfqpoint{5.595423in}{4.575556in}}%
\pgfusepath{clip}%
\pgfsetrectcap%
\pgfsetroundjoin%
\pgfsetlinewidth{0.803000pt}%
\definecolor{currentstroke}{rgb}{0.690196,0.690196,0.690196}%
\pgfsetstrokecolor{currentstroke}%
\pgfsetdash{}{0pt}%
\pgfpathmoveto{\pgfqpoint{7.526319in}{4.658838in}}%
\pgfpathlineto{\pgfqpoint{13.121743in}{4.658838in}}%
\pgfusepath{stroke}%
\end{pgfscope}%
\begin{pgfscope}%
\pgfsetbuttcap%
\pgfsetroundjoin%
\definecolor{currentfill}{rgb}{0.000000,0.000000,0.000000}%
\pgfsetfillcolor{currentfill}%
\pgfsetlinewidth{0.803000pt}%
\definecolor{currentstroke}{rgb}{0.000000,0.000000,0.000000}%
\pgfsetstrokecolor{currentstroke}%
\pgfsetdash{}{0pt}%
\pgfsys@defobject{currentmarker}{\pgfqpoint{-0.048611in}{0.000000in}}{\pgfqpoint{-0.000000in}{0.000000in}}{%
\pgfpathmoveto{\pgfqpoint{-0.000000in}{0.000000in}}%
\pgfpathlineto{\pgfqpoint{-0.048611in}{0.000000in}}%
\pgfusepath{stroke,fill}%
}%
\begin{pgfscope}%
\pgfsys@transformshift{7.526319in}{4.658838in}%
\pgfsys@useobject{currentmarker}{}%
\end{pgfscope}%
\end{pgfscope}%
\begin{pgfscope}%
\definecolor{textcolor}{rgb}{0.000000,0.000000,0.000000}%
\pgfsetstrokecolor{textcolor}%
\pgfsetfillcolor{textcolor}%
\pgftext[x=7.146328in, y=4.574420in, left, base]{\color{textcolor}\sffamily\fontsize{16.000000}{19.200000}\selectfont 25}%
\end{pgfscope}%
\begin{pgfscope}%
\pgfpathrectangle{\pgfqpoint{7.526319in}{0.865000in}}{\pgfqpoint{5.595423in}{4.575556in}}%
\pgfusepath{clip}%
\pgfsetrectcap%
\pgfsetroundjoin%
\pgfsetlinewidth{0.803000pt}%
\definecolor{currentstroke}{rgb}{0.690196,0.690196,0.690196}%
\pgfsetstrokecolor{currentstroke}%
\pgfsetdash{}{0pt}%
\pgfpathmoveto{\pgfqpoint{7.526319in}{5.376010in}}%
\pgfpathlineto{\pgfqpoint{13.121743in}{5.376010in}}%
\pgfusepath{stroke}%
\end{pgfscope}%
\begin{pgfscope}%
\pgfsetbuttcap%
\pgfsetroundjoin%
\definecolor{currentfill}{rgb}{0.000000,0.000000,0.000000}%
\pgfsetfillcolor{currentfill}%
\pgfsetlinewidth{0.803000pt}%
\definecolor{currentstroke}{rgb}{0.000000,0.000000,0.000000}%
\pgfsetstrokecolor{currentstroke}%
\pgfsetdash{}{0pt}%
\pgfsys@defobject{currentmarker}{\pgfqpoint{-0.048611in}{0.000000in}}{\pgfqpoint{-0.000000in}{0.000000in}}{%
\pgfpathmoveto{\pgfqpoint{-0.000000in}{0.000000in}}%
\pgfpathlineto{\pgfqpoint{-0.048611in}{0.000000in}}%
\pgfusepath{stroke,fill}%
}%
\begin{pgfscope}%
\pgfsys@transformshift{7.526319in}{5.376010in}%
\pgfsys@useobject{currentmarker}{}%
\end{pgfscope}%
\end{pgfscope}%
\begin{pgfscope}%
\definecolor{textcolor}{rgb}{0.000000,0.000000,0.000000}%
\pgfsetstrokecolor{textcolor}%
\pgfsetfillcolor{textcolor}%
\pgftext[x=7.146328in, y=5.291592in, left, base]{\color{textcolor}\sffamily\fontsize{16.000000}{19.200000}\selectfont 30}%
\end{pgfscope}%
\begin{pgfscope}%
\definecolor{textcolor}{rgb}{0.000000,0.000000,0.000000}%
\pgfsetstrokecolor{textcolor}%
\pgfsetfillcolor{textcolor}%
\pgftext[x=7.090772in,y=3.152778in,,bottom,rotate=90.000000]{\color{textcolor}\sffamily\fontsize{20.000000}{24.000000}\selectfont y}%
\end{pgfscope}%
\begin{pgfscope}%
\pgfpathrectangle{\pgfqpoint{7.526319in}{0.865000in}}{\pgfqpoint{5.595423in}{4.575556in}}%
\pgfusepath{clip}%
\pgfsetrectcap%
\pgfsetroundjoin%
\pgfsetlinewidth{1.505625pt}%
\definecolor{currentstroke}{rgb}{0.121569,0.466667,0.705882}%
\pgfsetstrokecolor{currentstroke}%
\pgfsetdash{}{0pt}%
\pgfpathmoveto{\pgfqpoint{10.324031in}{1.072980in}}%
\pgfpathlineto{\pgfqpoint{10.855741in}{1.216414in}}%
\pgfpathlineto{\pgfqpoint{11.364213in}{1.359848in}}%
\pgfpathlineto{\pgfqpoint{11.827223in}{1.503283in}}%
\pgfpathlineto{\pgfqpoint{12.224537in}{1.646717in}}%
\pgfpathlineto{\pgfqpoint{12.538790in}{1.790152in}}%
\pgfpathlineto{\pgfqpoint{12.756247in}{1.933586in}}%
\pgfpathlineto{\pgfqpoint{12.867405in}{2.077020in}}%
\pgfpathlineto{\pgfqpoint{12.867405in}{2.220455in}}%
\pgfpathlineto{\pgfqpoint{12.756247in}{2.363889in}}%
\pgfpathlineto{\pgfqpoint{12.538790in}{2.507323in}}%
\pgfpathlineto{\pgfqpoint{12.224537in}{2.650758in}}%
\pgfpathlineto{\pgfqpoint{11.827223in}{2.794192in}}%
\pgfpathlineto{\pgfqpoint{11.364213in}{2.937626in}}%
\pgfpathlineto{\pgfqpoint{10.855741in}{3.081061in}}%
\pgfpathlineto{\pgfqpoint{10.324031in}{3.224495in}}%
\pgfpathlineto{\pgfqpoint{9.792321in}{3.367929in}}%
\pgfpathlineto{\pgfqpoint{9.283849in}{3.511364in}}%
\pgfpathlineto{\pgfqpoint{8.820839in}{3.654798in}}%
\pgfpathlineto{\pgfqpoint{8.423524in}{3.798232in}}%
\pgfpathlineto{\pgfqpoint{8.109272in}{3.941667in}}%
\pgfpathlineto{\pgfqpoint{7.891815in}{4.085101in}}%
\pgfpathlineto{\pgfqpoint{7.780657in}{4.228535in}}%
\pgfpathlineto{\pgfqpoint{7.780657in}{4.371970in}}%
\pgfpathlineto{\pgfqpoint{7.891815in}{4.515404in}}%
\pgfpathlineto{\pgfqpoint{8.109272in}{4.658838in}}%
\pgfpathlineto{\pgfqpoint{8.423524in}{4.802273in}}%
\pgfpathlineto{\pgfqpoint{8.820839in}{4.945707in}}%
\pgfpathlineto{\pgfqpoint{9.283849in}{5.089141in}}%
\pgfpathlineto{\pgfqpoint{9.792321in}{5.232576in}}%
\pgfusepath{stroke}%
\end{pgfscope}%
\begin{pgfscope}%
\pgfpathrectangle{\pgfqpoint{7.526319in}{0.865000in}}{\pgfqpoint{5.595423in}{4.575556in}}%
\pgfusepath{clip}%
\pgfsetrectcap%
\pgfsetroundjoin%
\pgfsetlinewidth{1.505625pt}%
\definecolor{currentstroke}{rgb}{1.000000,0.498039,0.054902}%
\pgfsetstrokecolor{currentstroke}%
\pgfsetdash{}{0pt}%
\pgfpathmoveto{\pgfqpoint{10.324031in}{1.072980in}}%
\pgfpathlineto{\pgfqpoint{10.412111in}{1.216414in}}%
\pgfpathlineto{\pgfqpoint{10.496341in}{1.359848in}}%
\pgfpathlineto{\pgfqpoint{10.573040in}{1.503283in}}%
\pgfpathlineto{\pgfqpoint{10.638857in}{1.646717in}}%
\pgfpathlineto{\pgfqpoint{10.690914in}{1.790152in}}%
\pgfpathlineto{\pgfqpoint{10.726937in}{1.933586in}}%
\pgfpathlineto{\pgfqpoint{10.745350in}{2.077020in}}%
\pgfpathlineto{\pgfqpoint{10.745350in}{2.220455in}}%
\pgfpathlineto{\pgfqpoint{10.726937in}{2.363889in}}%
\pgfpathlineto{\pgfqpoint{10.690914in}{2.507323in}}%
\pgfpathlineto{\pgfqpoint{10.638857in}{2.650758in}}%
\pgfpathlineto{\pgfqpoint{10.573040in}{2.794192in}}%
\pgfpathlineto{\pgfqpoint{10.496341in}{2.937626in}}%
\pgfpathlineto{\pgfqpoint{10.412111in}{3.081061in}}%
\pgfpathlineto{\pgfqpoint{10.324031in}{3.224495in}}%
\pgfpathlineto{\pgfqpoint{10.235951in}{3.367929in}}%
\pgfpathlineto{\pgfqpoint{10.151721in}{3.511364in}}%
\pgfpathlineto{\pgfqpoint{10.075022in}{3.654798in}}%
\pgfpathlineto{\pgfqpoint{10.009205in}{3.798232in}}%
\pgfpathlineto{\pgfqpoint{9.957148in}{3.941667in}}%
\pgfpathlineto{\pgfqpoint{9.921125in}{4.085101in}}%
\pgfpathlineto{\pgfqpoint{9.902712in}{4.228535in}}%
\pgfpathlineto{\pgfqpoint{9.902712in}{4.371970in}}%
\pgfpathlineto{\pgfqpoint{9.921125in}{4.515404in}}%
\pgfpathlineto{\pgfqpoint{9.957148in}{4.658838in}}%
\pgfpathlineto{\pgfqpoint{10.009205in}{4.802273in}}%
\pgfpathlineto{\pgfqpoint{10.075022in}{4.945707in}}%
\pgfpathlineto{\pgfqpoint{10.151721in}{5.089141in}}%
\pgfpathlineto{\pgfqpoint{10.235951in}{5.232576in}}%
\pgfusepath{stroke}%
\end{pgfscope}%
\begin{pgfscope}%
\pgfpathrectangle{\pgfqpoint{7.526319in}{0.865000in}}{\pgfqpoint{5.595423in}{4.575556in}}%
\pgfusepath{clip}%
\pgfsetrectcap%
\pgfsetroundjoin%
\pgfsetlinewidth{1.505625pt}%
\definecolor{currentstroke}{rgb}{0.172549,0.627451,0.172549}%
\pgfsetstrokecolor{currentstroke}%
\pgfsetdash{}{0pt}%
\pgfpathmoveto{\pgfqpoint{10.324031in}{1.072980in}}%
\pgfpathlineto{\pgfqpoint{10.338639in}{1.216414in}}%
\pgfpathlineto{\pgfqpoint{10.352609in}{1.359848in}}%
\pgfpathlineto{\pgfqpoint{10.365330in}{1.503283in}}%
\pgfpathlineto{\pgfqpoint{10.376246in}{1.646717in}}%
\pgfpathlineto{\pgfqpoint{10.384880in}{1.790152in}}%
\pgfpathlineto{\pgfqpoint{10.390854in}{1.933586in}}%
\pgfpathlineto{\pgfqpoint{10.393908in}{2.077020in}}%
\pgfpathlineto{\pgfqpoint{10.393908in}{2.220455in}}%
\pgfpathlineto{\pgfqpoint{10.390854in}{2.363889in}}%
\pgfpathlineto{\pgfqpoint{10.384880in}{2.507323in}}%
\pgfpathlineto{\pgfqpoint{10.376246in}{2.650758in}}%
\pgfpathlineto{\pgfqpoint{10.365330in}{2.794192in}}%
\pgfpathlineto{\pgfqpoint{10.352609in}{2.937626in}}%
\pgfpathlineto{\pgfqpoint{10.338639in}{3.081061in}}%
\pgfpathlineto{\pgfqpoint{10.324031in}{3.224495in}}%
\pgfpathlineto{\pgfqpoint{10.309423in}{3.367929in}}%
\pgfpathlineto{\pgfqpoint{10.295453in}{3.511364in}}%
\pgfpathlineto{\pgfqpoint{10.282732in}{3.654798in}}%
\pgfpathlineto{\pgfqpoint{10.271816in}{3.798232in}}%
\pgfpathlineto{\pgfqpoint{10.263182in}{3.941667in}}%
\pgfpathlineto{\pgfqpoint{10.257208in}{4.085101in}}%
\pgfpathlineto{\pgfqpoint{10.254154in}{4.228535in}}%
\pgfpathlineto{\pgfqpoint{10.254154in}{4.371970in}}%
\pgfpathlineto{\pgfqpoint{10.257208in}{4.515404in}}%
\pgfpathlineto{\pgfqpoint{10.263182in}{4.658838in}}%
\pgfpathlineto{\pgfqpoint{10.271816in}{4.802273in}}%
\pgfpathlineto{\pgfqpoint{10.282732in}{4.945707in}}%
\pgfpathlineto{\pgfqpoint{10.295453in}{5.089141in}}%
\pgfpathlineto{\pgfqpoint{10.309423in}{5.232576in}}%
\pgfusepath{stroke}%
\end{pgfscope}%
\begin{pgfscope}%
\pgfpathrectangle{\pgfqpoint{7.526319in}{0.865000in}}{\pgfqpoint{5.595423in}{4.575556in}}%
\pgfusepath{clip}%
\pgfsetrectcap%
\pgfsetroundjoin%
\pgfsetlinewidth{1.505625pt}%
\definecolor{currentstroke}{rgb}{0.839216,0.152941,0.156863}%
\pgfsetstrokecolor{currentstroke}%
\pgfsetdash{}{0pt}%
\pgfpathmoveto{\pgfqpoint{10.324031in}{1.072980in}}%
\pgfpathlineto{\pgfqpoint{10.326454in}{1.216414in}}%
\pgfpathlineto{\pgfqpoint{10.328771in}{1.359848in}}%
\pgfpathlineto{\pgfqpoint{10.330881in}{1.503283in}}%
\pgfpathlineto{\pgfqpoint{10.332691in}{1.646717in}}%
\pgfpathlineto{\pgfqpoint{10.334123in}{1.790152in}}%
\pgfpathlineto{\pgfqpoint{10.335114in}{1.933586in}}%
\pgfpathlineto{\pgfqpoint{10.335620in}{2.077020in}}%
\pgfpathlineto{\pgfqpoint{10.335620in}{2.220455in}}%
\pgfpathlineto{\pgfqpoint{10.335114in}{2.363889in}}%
\pgfpathlineto{\pgfqpoint{10.334123in}{2.507323in}}%
\pgfpathlineto{\pgfqpoint{10.332691in}{2.650758in}}%
\pgfpathlineto{\pgfqpoint{10.330881in}{2.794192in}}%
\pgfpathlineto{\pgfqpoint{10.328771in}{2.937626in}}%
\pgfpathlineto{\pgfqpoint{10.326454in}{3.081061in}}%
\pgfpathlineto{\pgfqpoint{10.324031in}{3.224495in}}%
\pgfpathlineto{\pgfqpoint{10.321608in}{3.367929in}}%
\pgfpathlineto{\pgfqpoint{10.319291in}{3.511364in}}%
\pgfpathlineto{\pgfqpoint{10.317181in}{3.654798in}}%
\pgfpathlineto{\pgfqpoint{10.315371in}{3.798232in}}%
\pgfpathlineto{\pgfqpoint{10.313939in}{3.941667in}}%
\pgfpathlineto{\pgfqpoint{10.312948in}{4.085101in}}%
\pgfpathlineto{\pgfqpoint{10.312442in}{4.228535in}}%
\pgfpathlineto{\pgfqpoint{10.312442in}{4.371970in}}%
\pgfpathlineto{\pgfqpoint{10.312948in}{4.515404in}}%
\pgfpathlineto{\pgfqpoint{10.313939in}{4.658838in}}%
\pgfpathlineto{\pgfqpoint{10.315371in}{4.802273in}}%
\pgfpathlineto{\pgfqpoint{10.317181in}{4.945707in}}%
\pgfpathlineto{\pgfqpoint{10.319291in}{5.089141in}}%
\pgfpathlineto{\pgfqpoint{10.321608in}{5.232576in}}%
\pgfusepath{stroke}%
\end{pgfscope}%
\begin{pgfscope}%
\pgfpathrectangle{\pgfqpoint{7.526319in}{0.865000in}}{\pgfqpoint{5.595423in}{4.575556in}}%
\pgfusepath{clip}%
\pgfsetrectcap%
\pgfsetroundjoin%
\pgfsetlinewidth{1.505625pt}%
\definecolor{currentstroke}{rgb}{0.580392,0.403922,0.741176}%
\pgfsetstrokecolor{currentstroke}%
\pgfsetdash{}{0pt}%
\pgfpathmoveto{\pgfqpoint{10.324031in}{1.072980in}}%
\pgfpathlineto{\pgfqpoint{10.324433in}{1.216414in}}%
\pgfpathlineto{\pgfqpoint{10.324817in}{1.359848in}}%
\pgfpathlineto{\pgfqpoint{10.325167in}{1.503283in}}%
\pgfpathlineto{\pgfqpoint{10.325467in}{1.646717in}}%
\pgfpathlineto{\pgfqpoint{10.325705in}{1.790152in}}%
\pgfpathlineto{\pgfqpoint{10.325869in}{1.933586in}}%
\pgfpathlineto{\pgfqpoint{10.325953in}{2.077020in}}%
\pgfpathlineto{\pgfqpoint{10.325953in}{2.220455in}}%
\pgfpathlineto{\pgfqpoint{10.325869in}{2.363889in}}%
\pgfpathlineto{\pgfqpoint{10.325705in}{2.507323in}}%
\pgfpathlineto{\pgfqpoint{10.325467in}{2.650758in}}%
\pgfpathlineto{\pgfqpoint{10.325167in}{2.794192in}}%
\pgfpathlineto{\pgfqpoint{10.324817in}{2.937626in}}%
\pgfpathlineto{\pgfqpoint{10.324433in}{3.081061in}}%
\pgfpathlineto{\pgfqpoint{10.324031in}{3.224495in}}%
\pgfpathlineto{\pgfqpoint{10.323629in}{3.367929in}}%
\pgfpathlineto{\pgfqpoint{10.323245in}{3.511364in}}%
\pgfpathlineto{\pgfqpoint{10.322895in}{3.654798in}}%
\pgfpathlineto{\pgfqpoint{10.322595in}{3.798232in}}%
\pgfpathlineto{\pgfqpoint{10.322357in}{3.941667in}}%
\pgfpathlineto{\pgfqpoint{10.322193in}{4.085101in}}%
\pgfpathlineto{\pgfqpoint{10.322109in}{4.228535in}}%
\pgfpathlineto{\pgfqpoint{10.322109in}{4.371970in}}%
\pgfpathlineto{\pgfqpoint{10.322193in}{4.515404in}}%
\pgfpathlineto{\pgfqpoint{10.322357in}{4.658838in}}%
\pgfpathlineto{\pgfqpoint{10.322595in}{4.802273in}}%
\pgfpathlineto{\pgfqpoint{10.322895in}{4.945707in}}%
\pgfpathlineto{\pgfqpoint{10.323245in}{5.089141in}}%
\pgfpathlineto{\pgfqpoint{10.323629in}{5.232576in}}%
\pgfusepath{stroke}%
\end{pgfscope}%
\begin{pgfscope}%
\pgfpathrectangle{\pgfqpoint{7.526319in}{0.865000in}}{\pgfqpoint{5.595423in}{4.575556in}}%
\pgfusepath{clip}%
\pgfsetrectcap%
\pgfsetroundjoin%
\pgfsetlinewidth{1.505625pt}%
\definecolor{currentstroke}{rgb}{0.549020,0.337255,0.294118}%
\pgfsetstrokecolor{currentstroke}%
\pgfsetdash{}{0pt}%
\pgfpathmoveto{\pgfqpoint{10.324031in}{1.072980in}}%
\pgfpathlineto{\pgfqpoint{10.324098in}{1.216414in}}%
\pgfpathlineto{\pgfqpoint{10.324161in}{1.359848in}}%
\pgfpathlineto{\pgfqpoint{10.324219in}{1.503283in}}%
\pgfpathlineto{\pgfqpoint{10.324269in}{1.646717in}}%
\pgfpathlineto{\pgfqpoint{10.324309in}{1.790152in}}%
\pgfpathlineto{\pgfqpoint{10.324336in}{1.933586in}}%
\pgfpathlineto{\pgfqpoint{10.324350in}{2.077020in}}%
\pgfpathlineto{\pgfqpoint{10.324350in}{2.220455in}}%
\pgfpathlineto{\pgfqpoint{10.324336in}{2.363889in}}%
\pgfpathlineto{\pgfqpoint{10.324309in}{2.507323in}}%
\pgfpathlineto{\pgfqpoint{10.324269in}{2.650758in}}%
\pgfpathlineto{\pgfqpoint{10.324219in}{2.794192in}}%
\pgfpathlineto{\pgfqpoint{10.324161in}{2.937626in}}%
\pgfpathlineto{\pgfqpoint{10.324098in}{3.081061in}}%
\pgfpathlineto{\pgfqpoint{10.324031in}{3.224495in}}%
\pgfpathlineto{\pgfqpoint{10.323964in}{3.367929in}}%
\pgfpathlineto{\pgfqpoint{10.323901in}{3.511364in}}%
\pgfpathlineto{\pgfqpoint{10.323843in}{3.654798in}}%
\pgfpathlineto{\pgfqpoint{10.323793in}{3.798232in}}%
\pgfpathlineto{\pgfqpoint{10.323753in}{3.941667in}}%
\pgfpathlineto{\pgfqpoint{10.323726in}{4.085101in}}%
\pgfpathlineto{\pgfqpoint{10.323712in}{4.228535in}}%
\pgfpathlineto{\pgfqpoint{10.323712in}{4.371970in}}%
\pgfpathlineto{\pgfqpoint{10.323726in}{4.515404in}}%
\pgfpathlineto{\pgfqpoint{10.323753in}{4.658838in}}%
\pgfpathlineto{\pgfqpoint{10.323793in}{4.802273in}}%
\pgfpathlineto{\pgfqpoint{10.323843in}{4.945707in}}%
\pgfpathlineto{\pgfqpoint{10.323901in}{5.089141in}}%
\pgfpathlineto{\pgfqpoint{10.323964in}{5.232576in}}%
\pgfusepath{stroke}%
\end{pgfscope}%
\begin{pgfscope}%
\pgfsetrectcap%
\pgfsetmiterjoin%
\pgfsetlinewidth{0.803000pt}%
\definecolor{currentstroke}{rgb}{0.000000,0.000000,0.000000}%
\pgfsetstrokecolor{currentstroke}%
\pgfsetdash{}{0pt}%
\pgfpathmoveto{\pgfqpoint{7.526319in}{0.865000in}}%
\pgfpathlineto{\pgfqpoint{7.526319in}{5.440556in}}%
\pgfusepath{stroke}%
\end{pgfscope}%
\begin{pgfscope}%
\pgfsetrectcap%
\pgfsetmiterjoin%
\pgfsetlinewidth{0.803000pt}%
\definecolor{currentstroke}{rgb}{0.000000,0.000000,0.000000}%
\pgfsetstrokecolor{currentstroke}%
\pgfsetdash{}{0pt}%
\pgfpathmoveto{\pgfqpoint{13.121743in}{0.865000in}}%
\pgfpathlineto{\pgfqpoint{13.121743in}{5.440556in}}%
\pgfusepath{stroke}%
\end{pgfscope}%
\begin{pgfscope}%
\pgfsetrectcap%
\pgfsetmiterjoin%
\pgfsetlinewidth{0.803000pt}%
\definecolor{currentstroke}{rgb}{0.000000,0.000000,0.000000}%
\pgfsetstrokecolor{currentstroke}%
\pgfsetdash{}{0pt}%
\pgfpathmoveto{\pgfqpoint{7.526319in}{0.865000in}}%
\pgfpathlineto{\pgfqpoint{13.121743in}{0.865000in}}%
\pgfusepath{stroke}%
\end{pgfscope}%
\begin{pgfscope}%
\pgfsetrectcap%
\pgfsetmiterjoin%
\pgfsetlinewidth{0.803000pt}%
\definecolor{currentstroke}{rgb}{0.000000,0.000000,0.000000}%
\pgfsetstrokecolor{currentstroke}%
\pgfsetdash{}{0pt}%
\pgfpathmoveto{\pgfqpoint{7.526319in}{5.440556in}}%
\pgfpathlineto{\pgfqpoint{13.121743in}{5.440556in}}%
\pgfusepath{stroke}%
\end{pgfscope}%
\begin{pgfscope}%
\definecolor{textcolor}{rgb}{0.000000,0.000000,0.000000}%
\pgfsetstrokecolor{textcolor}%
\pgfsetfillcolor{textcolor}%
\pgftext[x=10.324031in,y=5.523889in,,base]{\color{textcolor}\sffamily\fontsize{20.000000}{24.000000}\selectfont Velocity decay (\(\displaystyle \omega\)=1.1)}%
\end{pgfscope}%
\begin{pgfscope}%
\pgfsetbuttcap%
\pgfsetmiterjoin%
\definecolor{currentfill}{rgb}{1.000000,1.000000,1.000000}%
\pgfsetfillcolor{currentfill}%
\pgfsetfillopacity{0.800000}%
\pgfsetlinewidth{1.003750pt}%
\definecolor{currentstroke}{rgb}{0.800000,0.800000,0.800000}%
\pgfsetstrokecolor{currentstroke}%
\pgfsetstrokeopacity{0.800000}%
\pgfsetdash{}{0pt}%
\pgfpathmoveto{\pgfqpoint{7.720764in}{1.003889in}}%
\pgfpathlineto{\pgfqpoint{9.658237in}{1.003889in}}%
\pgfpathquadraticcurveto{\pgfqpoint{9.713792in}{1.003889in}}{\pgfqpoint{9.713792in}{1.059444in}}%
\pgfpathlineto{\pgfqpoint{9.713792in}{3.477954in}}%
\pgfpathquadraticcurveto{\pgfqpoint{9.713792in}{3.533510in}}{\pgfqpoint{9.658237in}{3.533510in}}%
\pgfpathlineto{\pgfqpoint{7.720764in}{3.533510in}}%
\pgfpathquadraticcurveto{\pgfqpoint{7.665208in}{3.533510in}}{\pgfqpoint{7.665208in}{3.477954in}}%
\pgfpathlineto{\pgfqpoint{7.665208in}{1.059444in}}%
\pgfpathquadraticcurveto{\pgfqpoint{7.665208in}{1.003889in}}{\pgfqpoint{7.720764in}{1.003889in}}%
\pgfpathlineto{\pgfqpoint{7.720764in}{1.003889in}}%
\pgfpathclose%
\pgfusepath{stroke,fill}%
\end{pgfscope}%
\begin{pgfscope}%
\pgfsetrectcap%
\pgfsetroundjoin%
\pgfsetlinewidth{1.505625pt}%
\definecolor{currentstroke}{rgb}{0.121569,0.466667,0.705882}%
\pgfsetstrokecolor{currentstroke}%
\pgfsetdash{}{0pt}%
\pgfpathmoveto{\pgfqpoint{7.776319in}{3.308575in}}%
\pgfpathlineto{\pgfqpoint{8.054097in}{3.308575in}}%
\pgfpathlineto{\pgfqpoint{8.331875in}{3.308575in}}%
\pgfusepath{stroke}%
\end{pgfscope}%
\begin{pgfscope}%
\definecolor{textcolor}{rgb}{0.000000,0.000000,0.000000}%
\pgfsetstrokecolor{textcolor}%
\pgfsetfillcolor{textcolor}%
\pgftext[x=8.554097in,y=3.211353in,left,base]{\color{textcolor}\sffamily\fontsize{20.000000}{24.000000}\selectfont t=0}%
\end{pgfscope}%
\begin{pgfscope}%
\pgfsetrectcap%
\pgfsetroundjoin%
\pgfsetlinewidth{1.505625pt}%
\definecolor{currentstroke}{rgb}{1.000000,0.498039,0.054902}%
\pgfsetstrokecolor{currentstroke}%
\pgfsetdash{}{0pt}%
\pgfpathmoveto{\pgfqpoint{7.776319in}{2.900860in}}%
\pgfpathlineto{\pgfqpoint{8.054097in}{2.900860in}}%
\pgfpathlineto{\pgfqpoint{8.331875in}{2.900860in}}%
\pgfusepath{stroke}%
\end{pgfscope}%
\begin{pgfscope}%
\definecolor{textcolor}{rgb}{0.000000,0.000000,0.000000}%
\pgfsetstrokecolor{textcolor}%
\pgfsetfillcolor{textcolor}%
\pgftext[x=8.554097in,y=2.803638in,left,base]{\color{textcolor}\sffamily\fontsize{20.000000}{24.000000}\selectfont t=300}%
\end{pgfscope}%
\begin{pgfscope}%
\pgfsetrectcap%
\pgfsetroundjoin%
\pgfsetlinewidth{1.505625pt}%
\definecolor{currentstroke}{rgb}{0.172549,0.627451,0.172549}%
\pgfsetstrokecolor{currentstroke}%
\pgfsetdash{}{0pt}%
\pgfpathmoveto{\pgfqpoint{7.776319in}{2.493146in}}%
\pgfpathlineto{\pgfqpoint{8.054097in}{2.493146in}}%
\pgfpathlineto{\pgfqpoint{8.331875in}{2.493146in}}%
\pgfusepath{stroke}%
\end{pgfscope}%
\begin{pgfscope}%
\definecolor{textcolor}{rgb}{0.000000,0.000000,0.000000}%
\pgfsetstrokecolor{textcolor}%
\pgfsetfillcolor{textcolor}%
\pgftext[x=8.554097in,y=2.395923in,left,base]{\color{textcolor}\sffamily\fontsize{20.000000}{24.000000}\selectfont t=600}%
\end{pgfscope}%
\begin{pgfscope}%
\pgfsetrectcap%
\pgfsetroundjoin%
\pgfsetlinewidth{1.505625pt}%
\definecolor{currentstroke}{rgb}{0.839216,0.152941,0.156863}%
\pgfsetstrokecolor{currentstroke}%
\pgfsetdash{}{0pt}%
\pgfpathmoveto{\pgfqpoint{7.776319in}{2.085431in}}%
\pgfpathlineto{\pgfqpoint{8.054097in}{2.085431in}}%
\pgfpathlineto{\pgfqpoint{8.331875in}{2.085431in}}%
\pgfusepath{stroke}%
\end{pgfscope}%
\begin{pgfscope}%
\definecolor{textcolor}{rgb}{0.000000,0.000000,0.000000}%
\pgfsetstrokecolor{textcolor}%
\pgfsetfillcolor{textcolor}%
\pgftext[x=8.554097in,y=1.988209in,left,base]{\color{textcolor}\sffamily\fontsize{20.000000}{24.000000}\selectfont t=900}%
\end{pgfscope}%
\begin{pgfscope}%
\pgfsetrectcap%
\pgfsetroundjoin%
\pgfsetlinewidth{1.505625pt}%
\definecolor{currentstroke}{rgb}{0.580392,0.403922,0.741176}%
\pgfsetstrokecolor{currentstroke}%
\pgfsetdash{}{0pt}%
\pgfpathmoveto{\pgfqpoint{7.776319in}{1.677717in}}%
\pgfpathlineto{\pgfqpoint{8.054097in}{1.677717in}}%
\pgfpathlineto{\pgfqpoint{8.331875in}{1.677717in}}%
\pgfusepath{stroke}%
\end{pgfscope}%
\begin{pgfscope}%
\definecolor{textcolor}{rgb}{0.000000,0.000000,0.000000}%
\pgfsetstrokecolor{textcolor}%
\pgfsetfillcolor{textcolor}%
\pgftext[x=8.554097in,y=1.580494in,left,base]{\color{textcolor}\sffamily\fontsize{20.000000}{24.000000}\selectfont t=1200}%
\end{pgfscope}%
\begin{pgfscope}%
\pgfsetrectcap%
\pgfsetroundjoin%
\pgfsetlinewidth{1.505625pt}%
\definecolor{currentstroke}{rgb}{0.549020,0.337255,0.294118}%
\pgfsetstrokecolor{currentstroke}%
\pgfsetdash{}{0pt}%
\pgfpathmoveto{\pgfqpoint{7.776319in}{1.270002in}}%
\pgfpathlineto{\pgfqpoint{8.054097in}{1.270002in}}%
\pgfpathlineto{\pgfqpoint{8.331875in}{1.270002in}}%
\pgfusepath{stroke}%
\end{pgfscope}%
\begin{pgfscope}%
\definecolor{textcolor}{rgb}{0.000000,0.000000,0.000000}%
\pgfsetstrokecolor{textcolor}%
\pgfsetfillcolor{textcolor}%
\pgftext[x=8.554097in,y=1.172780in,left,base]{\color{textcolor}\sffamily\fontsize{20.000000}{24.000000}\selectfont t=1500}%
\end{pgfscope}%
\begin{pgfscope}%
\pgfsetbuttcap%
\pgfsetmiterjoin%
\definecolor{currentfill}{rgb}{1.000000,1.000000,1.000000}%
\pgfsetfillcolor{currentfill}%
\pgfsetlinewidth{0.000000pt}%
\definecolor{currentstroke}{rgb}{0.000000,0.000000,0.000000}%
\pgfsetstrokecolor{currentstroke}%
\pgfsetstrokeopacity{0.000000}%
\pgfsetdash{}{0pt}%
\pgfpathmoveto{\pgfqpoint{14.112986in}{0.865000in}}%
\pgfpathlineto{\pgfqpoint{19.708409in}{0.865000in}}%
\pgfpathlineto{\pgfqpoint{19.708409in}{5.440556in}}%
\pgfpathlineto{\pgfqpoint{14.112986in}{5.440556in}}%
\pgfpathlineto{\pgfqpoint{14.112986in}{0.865000in}}%
\pgfpathclose%
\pgfusepath{fill}%
\end{pgfscope}%
\begin{pgfscope}%
\pgfpathrectangle{\pgfqpoint{14.112986in}{0.865000in}}{\pgfqpoint{5.595423in}{4.575556in}}%
\pgfusepath{clip}%
\pgfsetrectcap%
\pgfsetroundjoin%
\pgfsetlinewidth{0.803000pt}%
\definecolor{currentstroke}{rgb}{0.690196,0.690196,0.690196}%
\pgfsetstrokecolor{currentstroke}%
\pgfsetdash{}{0pt}%
\pgfpathmoveto{\pgfqpoint{14.353314in}{0.865000in}}%
\pgfpathlineto{\pgfqpoint{14.353314in}{5.440556in}}%
\pgfusepath{stroke}%
\end{pgfscope}%
\begin{pgfscope}%
\pgfsetbuttcap%
\pgfsetroundjoin%
\definecolor{currentfill}{rgb}{0.000000,0.000000,0.000000}%
\pgfsetfillcolor{currentfill}%
\pgfsetlinewidth{0.803000pt}%
\definecolor{currentstroke}{rgb}{0.000000,0.000000,0.000000}%
\pgfsetstrokecolor{currentstroke}%
\pgfsetdash{}{0pt}%
\pgfsys@defobject{currentmarker}{\pgfqpoint{0.000000in}{-0.048611in}}{\pgfqpoint{0.000000in}{0.000000in}}{%
\pgfpathmoveto{\pgfqpoint{0.000000in}{0.000000in}}%
\pgfpathlineto{\pgfqpoint{0.000000in}{-0.048611in}}%
\pgfusepath{stroke,fill}%
}%
\begin{pgfscope}%
\pgfsys@transformshift{14.353314in}{0.865000in}%
\pgfsys@useobject{currentmarker}{}%
\end{pgfscope}%
\end{pgfscope}%
\begin{pgfscope}%
\definecolor{textcolor}{rgb}{0.000000,0.000000,0.000000}%
\pgfsetstrokecolor{textcolor}%
\pgfsetfillcolor{textcolor}%
\pgftext[x=14.353314in,y=0.767778in,,top]{\color{textcolor}\sffamily\fontsize{16.000000}{19.200000}\selectfont \ensuremath{-}0.10}%
\end{pgfscope}%
\begin{pgfscope}%
\pgfpathrectangle{\pgfqpoint{14.112986in}{0.865000in}}{\pgfqpoint{5.595423in}{4.575556in}}%
\pgfusepath{clip}%
\pgfsetrectcap%
\pgfsetroundjoin%
\pgfsetlinewidth{0.803000pt}%
\definecolor{currentstroke}{rgb}{0.690196,0.690196,0.690196}%
\pgfsetstrokecolor{currentstroke}%
\pgfsetdash{}{0pt}%
\pgfpathmoveto{\pgfqpoint{15.632006in}{0.865000in}}%
\pgfpathlineto{\pgfqpoint{15.632006in}{5.440556in}}%
\pgfusepath{stroke}%
\end{pgfscope}%
\begin{pgfscope}%
\pgfsetbuttcap%
\pgfsetroundjoin%
\definecolor{currentfill}{rgb}{0.000000,0.000000,0.000000}%
\pgfsetfillcolor{currentfill}%
\pgfsetlinewidth{0.803000pt}%
\definecolor{currentstroke}{rgb}{0.000000,0.000000,0.000000}%
\pgfsetstrokecolor{currentstroke}%
\pgfsetdash{}{0pt}%
\pgfsys@defobject{currentmarker}{\pgfqpoint{0.000000in}{-0.048611in}}{\pgfqpoint{0.000000in}{0.000000in}}{%
\pgfpathmoveto{\pgfqpoint{0.000000in}{0.000000in}}%
\pgfpathlineto{\pgfqpoint{0.000000in}{-0.048611in}}%
\pgfusepath{stroke,fill}%
}%
\begin{pgfscope}%
\pgfsys@transformshift{15.632006in}{0.865000in}%
\pgfsys@useobject{currentmarker}{}%
\end{pgfscope}%
\end{pgfscope}%
\begin{pgfscope}%
\definecolor{textcolor}{rgb}{0.000000,0.000000,0.000000}%
\pgfsetstrokecolor{textcolor}%
\pgfsetfillcolor{textcolor}%
\pgftext[x=15.632006in,y=0.767778in,,top]{\color{textcolor}\sffamily\fontsize{16.000000}{19.200000}\selectfont \ensuremath{-}0.05}%
\end{pgfscope}%
\begin{pgfscope}%
\pgfpathrectangle{\pgfqpoint{14.112986in}{0.865000in}}{\pgfqpoint{5.595423in}{4.575556in}}%
\pgfusepath{clip}%
\pgfsetrectcap%
\pgfsetroundjoin%
\pgfsetlinewidth{0.803000pt}%
\definecolor{currentstroke}{rgb}{0.690196,0.690196,0.690196}%
\pgfsetstrokecolor{currentstroke}%
\pgfsetdash{}{0pt}%
\pgfpathmoveto{\pgfqpoint{16.910698in}{0.865000in}}%
\pgfpathlineto{\pgfqpoint{16.910698in}{5.440556in}}%
\pgfusepath{stroke}%
\end{pgfscope}%
\begin{pgfscope}%
\pgfsetbuttcap%
\pgfsetroundjoin%
\definecolor{currentfill}{rgb}{0.000000,0.000000,0.000000}%
\pgfsetfillcolor{currentfill}%
\pgfsetlinewidth{0.803000pt}%
\definecolor{currentstroke}{rgb}{0.000000,0.000000,0.000000}%
\pgfsetstrokecolor{currentstroke}%
\pgfsetdash{}{0pt}%
\pgfsys@defobject{currentmarker}{\pgfqpoint{0.000000in}{-0.048611in}}{\pgfqpoint{0.000000in}{0.000000in}}{%
\pgfpathmoveto{\pgfqpoint{0.000000in}{0.000000in}}%
\pgfpathlineto{\pgfqpoint{0.000000in}{-0.048611in}}%
\pgfusepath{stroke,fill}%
}%
\begin{pgfscope}%
\pgfsys@transformshift{16.910698in}{0.865000in}%
\pgfsys@useobject{currentmarker}{}%
\end{pgfscope}%
\end{pgfscope}%
\begin{pgfscope}%
\definecolor{textcolor}{rgb}{0.000000,0.000000,0.000000}%
\pgfsetstrokecolor{textcolor}%
\pgfsetfillcolor{textcolor}%
\pgftext[x=16.910698in,y=0.767778in,,top]{\color{textcolor}\sffamily\fontsize{16.000000}{19.200000}\selectfont 0.00}%
\end{pgfscope}%
\begin{pgfscope}%
\pgfpathrectangle{\pgfqpoint{14.112986in}{0.865000in}}{\pgfqpoint{5.595423in}{4.575556in}}%
\pgfusepath{clip}%
\pgfsetrectcap%
\pgfsetroundjoin%
\pgfsetlinewidth{0.803000pt}%
\definecolor{currentstroke}{rgb}{0.690196,0.690196,0.690196}%
\pgfsetstrokecolor{currentstroke}%
\pgfsetdash{}{0pt}%
\pgfpathmoveto{\pgfqpoint{18.189390in}{0.865000in}}%
\pgfpathlineto{\pgfqpoint{18.189390in}{5.440556in}}%
\pgfusepath{stroke}%
\end{pgfscope}%
\begin{pgfscope}%
\pgfsetbuttcap%
\pgfsetroundjoin%
\definecolor{currentfill}{rgb}{0.000000,0.000000,0.000000}%
\pgfsetfillcolor{currentfill}%
\pgfsetlinewidth{0.803000pt}%
\definecolor{currentstroke}{rgb}{0.000000,0.000000,0.000000}%
\pgfsetstrokecolor{currentstroke}%
\pgfsetdash{}{0pt}%
\pgfsys@defobject{currentmarker}{\pgfqpoint{0.000000in}{-0.048611in}}{\pgfqpoint{0.000000in}{0.000000in}}{%
\pgfpathmoveto{\pgfqpoint{0.000000in}{0.000000in}}%
\pgfpathlineto{\pgfqpoint{0.000000in}{-0.048611in}}%
\pgfusepath{stroke,fill}%
}%
\begin{pgfscope}%
\pgfsys@transformshift{18.189390in}{0.865000in}%
\pgfsys@useobject{currentmarker}{}%
\end{pgfscope}%
\end{pgfscope}%
\begin{pgfscope}%
\definecolor{textcolor}{rgb}{0.000000,0.000000,0.000000}%
\pgfsetstrokecolor{textcolor}%
\pgfsetfillcolor{textcolor}%
\pgftext[x=18.189390in,y=0.767778in,,top]{\color{textcolor}\sffamily\fontsize{16.000000}{19.200000}\selectfont 0.05}%
\end{pgfscope}%
\begin{pgfscope}%
\pgfpathrectangle{\pgfqpoint{14.112986in}{0.865000in}}{\pgfqpoint{5.595423in}{4.575556in}}%
\pgfusepath{clip}%
\pgfsetrectcap%
\pgfsetroundjoin%
\pgfsetlinewidth{0.803000pt}%
\definecolor{currentstroke}{rgb}{0.690196,0.690196,0.690196}%
\pgfsetstrokecolor{currentstroke}%
\pgfsetdash{}{0pt}%
\pgfpathmoveto{\pgfqpoint{19.468081in}{0.865000in}}%
\pgfpathlineto{\pgfqpoint{19.468081in}{5.440556in}}%
\pgfusepath{stroke}%
\end{pgfscope}%
\begin{pgfscope}%
\pgfsetbuttcap%
\pgfsetroundjoin%
\definecolor{currentfill}{rgb}{0.000000,0.000000,0.000000}%
\pgfsetfillcolor{currentfill}%
\pgfsetlinewidth{0.803000pt}%
\definecolor{currentstroke}{rgb}{0.000000,0.000000,0.000000}%
\pgfsetstrokecolor{currentstroke}%
\pgfsetdash{}{0pt}%
\pgfsys@defobject{currentmarker}{\pgfqpoint{0.000000in}{-0.048611in}}{\pgfqpoint{0.000000in}{0.000000in}}{%
\pgfpathmoveto{\pgfqpoint{0.000000in}{0.000000in}}%
\pgfpathlineto{\pgfqpoint{0.000000in}{-0.048611in}}%
\pgfusepath{stroke,fill}%
}%
\begin{pgfscope}%
\pgfsys@transformshift{19.468081in}{0.865000in}%
\pgfsys@useobject{currentmarker}{}%
\end{pgfscope}%
\end{pgfscope}%
\begin{pgfscope}%
\definecolor{textcolor}{rgb}{0.000000,0.000000,0.000000}%
\pgfsetstrokecolor{textcolor}%
\pgfsetfillcolor{textcolor}%
\pgftext[x=19.468081in,y=0.767778in,,top]{\color{textcolor}\sffamily\fontsize{16.000000}{19.200000}\selectfont 0.10}%
\end{pgfscope}%
\begin{pgfscope}%
\definecolor{textcolor}{rgb}{0.000000,0.000000,0.000000}%
\pgfsetstrokecolor{textcolor}%
\pgfsetfillcolor{textcolor}%
\pgftext[x=16.910698in,y=0.497162in,,top]{\color{textcolor}\sffamily\fontsize{20.000000}{24.000000}\selectfont velocity}%
\end{pgfscope}%
\begin{pgfscope}%
\pgfpathrectangle{\pgfqpoint{14.112986in}{0.865000in}}{\pgfqpoint{5.595423in}{4.575556in}}%
\pgfusepath{clip}%
\pgfsetrectcap%
\pgfsetroundjoin%
\pgfsetlinewidth{0.803000pt}%
\definecolor{currentstroke}{rgb}{0.690196,0.690196,0.690196}%
\pgfsetstrokecolor{currentstroke}%
\pgfsetdash{}{0pt}%
\pgfpathmoveto{\pgfqpoint{14.112986in}{1.072980in}}%
\pgfpathlineto{\pgfqpoint{19.708409in}{1.072980in}}%
\pgfusepath{stroke}%
\end{pgfscope}%
\begin{pgfscope}%
\pgfsetbuttcap%
\pgfsetroundjoin%
\definecolor{currentfill}{rgb}{0.000000,0.000000,0.000000}%
\pgfsetfillcolor{currentfill}%
\pgfsetlinewidth{0.803000pt}%
\definecolor{currentstroke}{rgb}{0.000000,0.000000,0.000000}%
\pgfsetstrokecolor{currentstroke}%
\pgfsetdash{}{0pt}%
\pgfsys@defobject{currentmarker}{\pgfqpoint{-0.048611in}{0.000000in}}{\pgfqpoint{-0.000000in}{0.000000in}}{%
\pgfpathmoveto{\pgfqpoint{-0.000000in}{0.000000in}}%
\pgfpathlineto{\pgfqpoint{-0.048611in}{0.000000in}}%
\pgfusepath{stroke,fill}%
}%
\begin{pgfscope}%
\pgfsys@transformshift{14.112986in}{1.072980in}%
\pgfsys@useobject{currentmarker}{}%
\end{pgfscope}%
\end{pgfscope}%
\begin{pgfscope}%
\definecolor{textcolor}{rgb}{0.000000,0.000000,0.000000}%
\pgfsetstrokecolor{textcolor}%
\pgfsetfillcolor{textcolor}%
\pgftext[x=13.874379in, y=0.988561in, left, base]{\color{textcolor}\sffamily\fontsize{16.000000}{19.200000}\selectfont 0}%
\end{pgfscope}%
\begin{pgfscope}%
\pgfpathrectangle{\pgfqpoint{14.112986in}{0.865000in}}{\pgfqpoint{5.595423in}{4.575556in}}%
\pgfusepath{clip}%
\pgfsetrectcap%
\pgfsetroundjoin%
\pgfsetlinewidth{0.803000pt}%
\definecolor{currentstroke}{rgb}{0.690196,0.690196,0.690196}%
\pgfsetstrokecolor{currentstroke}%
\pgfsetdash{}{0pt}%
\pgfpathmoveto{\pgfqpoint{14.112986in}{1.790152in}}%
\pgfpathlineto{\pgfqpoint{19.708409in}{1.790152in}}%
\pgfusepath{stroke}%
\end{pgfscope}%
\begin{pgfscope}%
\pgfsetbuttcap%
\pgfsetroundjoin%
\definecolor{currentfill}{rgb}{0.000000,0.000000,0.000000}%
\pgfsetfillcolor{currentfill}%
\pgfsetlinewidth{0.803000pt}%
\definecolor{currentstroke}{rgb}{0.000000,0.000000,0.000000}%
\pgfsetstrokecolor{currentstroke}%
\pgfsetdash{}{0pt}%
\pgfsys@defobject{currentmarker}{\pgfqpoint{-0.048611in}{0.000000in}}{\pgfqpoint{-0.000000in}{0.000000in}}{%
\pgfpathmoveto{\pgfqpoint{-0.000000in}{0.000000in}}%
\pgfpathlineto{\pgfqpoint{-0.048611in}{0.000000in}}%
\pgfusepath{stroke,fill}%
}%
\begin{pgfscope}%
\pgfsys@transformshift{14.112986in}{1.790152in}%
\pgfsys@useobject{currentmarker}{}%
\end{pgfscope}%
\end{pgfscope}%
\begin{pgfscope}%
\definecolor{textcolor}{rgb}{0.000000,0.000000,0.000000}%
\pgfsetstrokecolor{textcolor}%
\pgfsetfillcolor{textcolor}%
\pgftext[x=13.874379in, y=1.705733in, left, base]{\color{textcolor}\sffamily\fontsize{16.000000}{19.200000}\selectfont 5}%
\end{pgfscope}%
\begin{pgfscope}%
\pgfpathrectangle{\pgfqpoint{14.112986in}{0.865000in}}{\pgfqpoint{5.595423in}{4.575556in}}%
\pgfusepath{clip}%
\pgfsetrectcap%
\pgfsetroundjoin%
\pgfsetlinewidth{0.803000pt}%
\definecolor{currentstroke}{rgb}{0.690196,0.690196,0.690196}%
\pgfsetstrokecolor{currentstroke}%
\pgfsetdash{}{0pt}%
\pgfpathmoveto{\pgfqpoint{14.112986in}{2.507323in}}%
\pgfpathlineto{\pgfqpoint{19.708409in}{2.507323in}}%
\pgfusepath{stroke}%
\end{pgfscope}%
\begin{pgfscope}%
\pgfsetbuttcap%
\pgfsetroundjoin%
\definecolor{currentfill}{rgb}{0.000000,0.000000,0.000000}%
\pgfsetfillcolor{currentfill}%
\pgfsetlinewidth{0.803000pt}%
\definecolor{currentstroke}{rgb}{0.000000,0.000000,0.000000}%
\pgfsetstrokecolor{currentstroke}%
\pgfsetdash{}{0pt}%
\pgfsys@defobject{currentmarker}{\pgfqpoint{-0.048611in}{0.000000in}}{\pgfqpoint{-0.000000in}{0.000000in}}{%
\pgfpathmoveto{\pgfqpoint{-0.000000in}{0.000000in}}%
\pgfpathlineto{\pgfqpoint{-0.048611in}{0.000000in}}%
\pgfusepath{stroke,fill}%
}%
\begin{pgfscope}%
\pgfsys@transformshift{14.112986in}{2.507323in}%
\pgfsys@useobject{currentmarker}{}%
\end{pgfscope}%
\end{pgfscope}%
\begin{pgfscope}%
\definecolor{textcolor}{rgb}{0.000000,0.000000,0.000000}%
\pgfsetstrokecolor{textcolor}%
\pgfsetfillcolor{textcolor}%
\pgftext[x=13.732995in, y=2.422905in, left, base]{\color{textcolor}\sffamily\fontsize{16.000000}{19.200000}\selectfont 10}%
\end{pgfscope}%
\begin{pgfscope}%
\pgfpathrectangle{\pgfqpoint{14.112986in}{0.865000in}}{\pgfqpoint{5.595423in}{4.575556in}}%
\pgfusepath{clip}%
\pgfsetrectcap%
\pgfsetroundjoin%
\pgfsetlinewidth{0.803000pt}%
\definecolor{currentstroke}{rgb}{0.690196,0.690196,0.690196}%
\pgfsetstrokecolor{currentstroke}%
\pgfsetdash{}{0pt}%
\pgfpathmoveto{\pgfqpoint{14.112986in}{3.224495in}}%
\pgfpathlineto{\pgfqpoint{19.708409in}{3.224495in}}%
\pgfusepath{stroke}%
\end{pgfscope}%
\begin{pgfscope}%
\pgfsetbuttcap%
\pgfsetroundjoin%
\definecolor{currentfill}{rgb}{0.000000,0.000000,0.000000}%
\pgfsetfillcolor{currentfill}%
\pgfsetlinewidth{0.803000pt}%
\definecolor{currentstroke}{rgb}{0.000000,0.000000,0.000000}%
\pgfsetstrokecolor{currentstroke}%
\pgfsetdash{}{0pt}%
\pgfsys@defobject{currentmarker}{\pgfqpoint{-0.048611in}{0.000000in}}{\pgfqpoint{-0.000000in}{0.000000in}}{%
\pgfpathmoveto{\pgfqpoint{-0.000000in}{0.000000in}}%
\pgfpathlineto{\pgfqpoint{-0.048611in}{0.000000in}}%
\pgfusepath{stroke,fill}%
}%
\begin{pgfscope}%
\pgfsys@transformshift{14.112986in}{3.224495in}%
\pgfsys@useobject{currentmarker}{}%
\end{pgfscope}%
\end{pgfscope}%
\begin{pgfscope}%
\definecolor{textcolor}{rgb}{0.000000,0.000000,0.000000}%
\pgfsetstrokecolor{textcolor}%
\pgfsetfillcolor{textcolor}%
\pgftext[x=13.732995in, y=3.140077in, left, base]{\color{textcolor}\sffamily\fontsize{16.000000}{19.200000}\selectfont 15}%
\end{pgfscope}%
\begin{pgfscope}%
\pgfpathrectangle{\pgfqpoint{14.112986in}{0.865000in}}{\pgfqpoint{5.595423in}{4.575556in}}%
\pgfusepath{clip}%
\pgfsetrectcap%
\pgfsetroundjoin%
\pgfsetlinewidth{0.803000pt}%
\definecolor{currentstroke}{rgb}{0.690196,0.690196,0.690196}%
\pgfsetstrokecolor{currentstroke}%
\pgfsetdash{}{0pt}%
\pgfpathmoveto{\pgfqpoint{14.112986in}{3.941667in}}%
\pgfpathlineto{\pgfqpoint{19.708409in}{3.941667in}}%
\pgfusepath{stroke}%
\end{pgfscope}%
\begin{pgfscope}%
\pgfsetbuttcap%
\pgfsetroundjoin%
\definecolor{currentfill}{rgb}{0.000000,0.000000,0.000000}%
\pgfsetfillcolor{currentfill}%
\pgfsetlinewidth{0.803000pt}%
\definecolor{currentstroke}{rgb}{0.000000,0.000000,0.000000}%
\pgfsetstrokecolor{currentstroke}%
\pgfsetdash{}{0pt}%
\pgfsys@defobject{currentmarker}{\pgfqpoint{-0.048611in}{0.000000in}}{\pgfqpoint{-0.000000in}{0.000000in}}{%
\pgfpathmoveto{\pgfqpoint{-0.000000in}{0.000000in}}%
\pgfpathlineto{\pgfqpoint{-0.048611in}{0.000000in}}%
\pgfusepath{stroke,fill}%
}%
\begin{pgfscope}%
\pgfsys@transformshift{14.112986in}{3.941667in}%
\pgfsys@useobject{currentmarker}{}%
\end{pgfscope}%
\end{pgfscope}%
\begin{pgfscope}%
\definecolor{textcolor}{rgb}{0.000000,0.000000,0.000000}%
\pgfsetstrokecolor{textcolor}%
\pgfsetfillcolor{textcolor}%
\pgftext[x=13.732995in, y=3.857248in, left, base]{\color{textcolor}\sffamily\fontsize{16.000000}{19.200000}\selectfont 20}%
\end{pgfscope}%
\begin{pgfscope}%
\pgfpathrectangle{\pgfqpoint{14.112986in}{0.865000in}}{\pgfqpoint{5.595423in}{4.575556in}}%
\pgfusepath{clip}%
\pgfsetrectcap%
\pgfsetroundjoin%
\pgfsetlinewidth{0.803000pt}%
\definecolor{currentstroke}{rgb}{0.690196,0.690196,0.690196}%
\pgfsetstrokecolor{currentstroke}%
\pgfsetdash{}{0pt}%
\pgfpathmoveto{\pgfqpoint{14.112986in}{4.658838in}}%
\pgfpathlineto{\pgfqpoint{19.708409in}{4.658838in}}%
\pgfusepath{stroke}%
\end{pgfscope}%
\begin{pgfscope}%
\pgfsetbuttcap%
\pgfsetroundjoin%
\definecolor{currentfill}{rgb}{0.000000,0.000000,0.000000}%
\pgfsetfillcolor{currentfill}%
\pgfsetlinewidth{0.803000pt}%
\definecolor{currentstroke}{rgb}{0.000000,0.000000,0.000000}%
\pgfsetstrokecolor{currentstroke}%
\pgfsetdash{}{0pt}%
\pgfsys@defobject{currentmarker}{\pgfqpoint{-0.048611in}{0.000000in}}{\pgfqpoint{-0.000000in}{0.000000in}}{%
\pgfpathmoveto{\pgfqpoint{-0.000000in}{0.000000in}}%
\pgfpathlineto{\pgfqpoint{-0.048611in}{0.000000in}}%
\pgfusepath{stroke,fill}%
}%
\begin{pgfscope}%
\pgfsys@transformshift{14.112986in}{4.658838in}%
\pgfsys@useobject{currentmarker}{}%
\end{pgfscope}%
\end{pgfscope}%
\begin{pgfscope}%
\definecolor{textcolor}{rgb}{0.000000,0.000000,0.000000}%
\pgfsetstrokecolor{textcolor}%
\pgfsetfillcolor{textcolor}%
\pgftext[x=13.732995in, y=4.574420in, left, base]{\color{textcolor}\sffamily\fontsize{16.000000}{19.200000}\selectfont 25}%
\end{pgfscope}%
\begin{pgfscope}%
\pgfpathrectangle{\pgfqpoint{14.112986in}{0.865000in}}{\pgfqpoint{5.595423in}{4.575556in}}%
\pgfusepath{clip}%
\pgfsetrectcap%
\pgfsetroundjoin%
\pgfsetlinewidth{0.803000pt}%
\definecolor{currentstroke}{rgb}{0.690196,0.690196,0.690196}%
\pgfsetstrokecolor{currentstroke}%
\pgfsetdash{}{0pt}%
\pgfpathmoveto{\pgfqpoint{14.112986in}{5.376010in}}%
\pgfpathlineto{\pgfqpoint{19.708409in}{5.376010in}}%
\pgfusepath{stroke}%
\end{pgfscope}%
\begin{pgfscope}%
\pgfsetbuttcap%
\pgfsetroundjoin%
\definecolor{currentfill}{rgb}{0.000000,0.000000,0.000000}%
\pgfsetfillcolor{currentfill}%
\pgfsetlinewidth{0.803000pt}%
\definecolor{currentstroke}{rgb}{0.000000,0.000000,0.000000}%
\pgfsetstrokecolor{currentstroke}%
\pgfsetdash{}{0pt}%
\pgfsys@defobject{currentmarker}{\pgfqpoint{-0.048611in}{0.000000in}}{\pgfqpoint{-0.000000in}{0.000000in}}{%
\pgfpathmoveto{\pgfqpoint{-0.000000in}{0.000000in}}%
\pgfpathlineto{\pgfqpoint{-0.048611in}{0.000000in}}%
\pgfusepath{stroke,fill}%
}%
\begin{pgfscope}%
\pgfsys@transformshift{14.112986in}{5.376010in}%
\pgfsys@useobject{currentmarker}{}%
\end{pgfscope}%
\end{pgfscope}%
\begin{pgfscope}%
\definecolor{textcolor}{rgb}{0.000000,0.000000,0.000000}%
\pgfsetstrokecolor{textcolor}%
\pgfsetfillcolor{textcolor}%
\pgftext[x=13.732995in, y=5.291592in, left, base]{\color{textcolor}\sffamily\fontsize{16.000000}{19.200000}\selectfont 30}%
\end{pgfscope}%
\begin{pgfscope}%
\definecolor{textcolor}{rgb}{0.000000,0.000000,0.000000}%
\pgfsetstrokecolor{textcolor}%
\pgfsetfillcolor{textcolor}%
\pgftext[x=13.677439in,y=3.152778in,,bottom,rotate=90.000000]{\color{textcolor}\sffamily\fontsize{20.000000}{24.000000}\selectfont y}%
\end{pgfscope}%
\begin{pgfscope}%
\pgfpathrectangle{\pgfqpoint{14.112986in}{0.865000in}}{\pgfqpoint{5.595423in}{4.575556in}}%
\pgfusepath{clip}%
\pgfsetrectcap%
\pgfsetroundjoin%
\pgfsetlinewidth{1.505625pt}%
\definecolor{currentstroke}{rgb}{0.121569,0.466667,0.705882}%
\pgfsetstrokecolor{currentstroke}%
\pgfsetdash{}{0pt}%
\pgfpathmoveto{\pgfqpoint{16.910698in}{1.072980in}}%
\pgfpathlineto{\pgfqpoint{17.442408in}{1.216414in}}%
\pgfpathlineto{\pgfqpoint{17.950879in}{1.359848in}}%
\pgfpathlineto{\pgfqpoint{18.413890in}{1.503283in}}%
\pgfpathlineto{\pgfqpoint{18.811204in}{1.646717in}}%
\pgfpathlineto{\pgfqpoint{19.125457in}{1.790152in}}%
\pgfpathlineto{\pgfqpoint{19.342914in}{1.933586in}}%
\pgfpathlineto{\pgfqpoint{19.454072in}{2.077020in}}%
\pgfpathlineto{\pgfqpoint{19.454072in}{2.220455in}}%
\pgfpathlineto{\pgfqpoint{19.342914in}{2.363889in}}%
\pgfpathlineto{\pgfqpoint{19.125457in}{2.507323in}}%
\pgfpathlineto{\pgfqpoint{18.811204in}{2.650758in}}%
\pgfpathlineto{\pgfqpoint{18.413890in}{2.794192in}}%
\pgfpathlineto{\pgfqpoint{17.950879in}{2.937626in}}%
\pgfpathlineto{\pgfqpoint{17.442408in}{3.081061in}}%
\pgfpathlineto{\pgfqpoint{16.910698in}{3.224495in}}%
\pgfpathlineto{\pgfqpoint{16.378988in}{3.367929in}}%
\pgfpathlineto{\pgfqpoint{15.870516in}{3.511364in}}%
\pgfpathlineto{\pgfqpoint{15.407505in}{3.654798in}}%
\pgfpathlineto{\pgfqpoint{15.010191in}{3.798232in}}%
\pgfpathlineto{\pgfqpoint{14.695938in}{3.941667in}}%
\pgfpathlineto{\pgfqpoint{14.478481in}{4.085101in}}%
\pgfpathlineto{\pgfqpoint{14.367324in}{4.228535in}}%
\pgfpathlineto{\pgfqpoint{14.367324in}{4.371970in}}%
\pgfpathlineto{\pgfqpoint{14.478481in}{4.515404in}}%
\pgfpathlineto{\pgfqpoint{14.695938in}{4.658838in}}%
\pgfpathlineto{\pgfqpoint{15.010191in}{4.802273in}}%
\pgfpathlineto{\pgfqpoint{15.407505in}{4.945707in}}%
\pgfpathlineto{\pgfqpoint{15.870516in}{5.089141in}}%
\pgfpathlineto{\pgfqpoint{16.378988in}{5.232576in}}%
\pgfusepath{stroke}%
\end{pgfscope}%
\begin{pgfscope}%
\pgfpathrectangle{\pgfqpoint{14.112986in}{0.865000in}}{\pgfqpoint{5.595423in}{4.575556in}}%
\pgfusepath{clip}%
\pgfsetrectcap%
\pgfsetroundjoin%
\pgfsetlinewidth{1.505625pt}%
\definecolor{currentstroke}{rgb}{1.000000,0.498039,0.054902}%
\pgfsetstrokecolor{currentstroke}%
\pgfsetdash{}{0pt}%
\pgfpathmoveto{\pgfqpoint{16.910698in}{1.072980in}}%
\pgfpathlineto{\pgfqpoint{17.269985in}{1.216414in}}%
\pgfpathlineto{\pgfqpoint{17.613570in}{1.359848in}}%
\pgfpathlineto{\pgfqpoint{17.926435in}{1.503283in}}%
\pgfpathlineto{\pgfqpoint{18.194909in}{1.646717in}}%
\pgfpathlineto{\pgfqpoint{18.407256in}{1.790152in}}%
\pgfpathlineto{\pgfqpoint{18.554196in}{1.933586in}}%
\pgfpathlineto{\pgfqpoint{18.629307in}{2.077020in}}%
\pgfpathlineto{\pgfqpoint{18.629307in}{2.220455in}}%
\pgfpathlineto{\pgfqpoint{18.554196in}{2.363889in}}%
\pgfpathlineto{\pgfqpoint{18.407256in}{2.507323in}}%
\pgfpathlineto{\pgfqpoint{18.194909in}{2.650758in}}%
\pgfpathlineto{\pgfqpoint{17.926435in}{2.794192in}}%
\pgfpathlineto{\pgfqpoint{17.613570in}{2.937626in}}%
\pgfpathlineto{\pgfqpoint{17.269985in}{3.081061in}}%
\pgfpathlineto{\pgfqpoint{16.910698in}{3.224495in}}%
\pgfpathlineto{\pgfqpoint{16.551410in}{3.367929in}}%
\pgfpathlineto{\pgfqpoint{16.207826in}{3.511364in}}%
\pgfpathlineto{\pgfqpoint{15.894960in}{3.654798in}}%
\pgfpathlineto{\pgfqpoint{15.626487in}{3.798232in}}%
\pgfpathlineto{\pgfqpoint{15.414140in}{3.941667in}}%
\pgfpathlineto{\pgfqpoint{15.267200in}{4.085101in}}%
\pgfpathlineto{\pgfqpoint{15.192088in}{4.228535in}}%
\pgfpathlineto{\pgfqpoint{15.192088in}{4.371970in}}%
\pgfpathlineto{\pgfqpoint{15.267200in}{4.515404in}}%
\pgfpathlineto{\pgfqpoint{15.414140in}{4.658838in}}%
\pgfpathlineto{\pgfqpoint{15.626487in}{4.802273in}}%
\pgfpathlineto{\pgfqpoint{15.894960in}{4.945707in}}%
\pgfpathlineto{\pgfqpoint{16.207826in}{5.089141in}}%
\pgfpathlineto{\pgfqpoint{16.551410in}{5.232576in}}%
\pgfusepath{stroke}%
\end{pgfscope}%
\begin{pgfscope}%
\pgfpathrectangle{\pgfqpoint{14.112986in}{0.865000in}}{\pgfqpoint{5.595423in}{4.575556in}}%
\pgfusepath{clip}%
\pgfsetrectcap%
\pgfsetroundjoin%
\pgfsetlinewidth{1.505625pt}%
\definecolor{currentstroke}{rgb}{0.172549,0.627451,0.172549}%
\pgfsetstrokecolor{currentstroke}%
\pgfsetdash{}{0pt}%
\pgfpathmoveto{\pgfqpoint{16.910698in}{1.072980in}}%
\pgfpathlineto{\pgfqpoint{17.154341in}{1.216414in}}%
\pgfpathlineto{\pgfqpoint{17.387336in}{1.359848in}}%
\pgfpathlineto{\pgfqpoint{17.599500in}{1.503283in}}%
\pgfpathlineto{\pgfqpoint{17.781560in}{1.646717in}}%
\pgfpathlineto{\pgfqpoint{17.925558in}{1.790152in}}%
\pgfpathlineto{\pgfqpoint{18.025203in}{1.933586in}}%
\pgfpathlineto{\pgfqpoint{18.076138in}{2.077020in}}%
\pgfpathlineto{\pgfqpoint{18.076138in}{2.220455in}}%
\pgfpathlineto{\pgfqpoint{18.025203in}{2.363889in}}%
\pgfpathlineto{\pgfqpoint{17.925558in}{2.507323in}}%
\pgfpathlineto{\pgfqpoint{17.781560in}{2.650758in}}%
\pgfpathlineto{\pgfqpoint{17.599500in}{2.794192in}}%
\pgfpathlineto{\pgfqpoint{17.387336in}{2.937626in}}%
\pgfpathlineto{\pgfqpoint{17.154341in}{3.081061in}}%
\pgfpathlineto{\pgfqpoint{16.910698in}{3.224495in}}%
\pgfpathlineto{\pgfqpoint{16.667054in}{3.367929in}}%
\pgfpathlineto{\pgfqpoint{16.434059in}{3.511364in}}%
\pgfpathlineto{\pgfqpoint{16.221895in}{3.654798in}}%
\pgfpathlineto{\pgfqpoint{16.039836in}{3.798232in}}%
\pgfpathlineto{\pgfqpoint{15.895837in}{3.941667in}}%
\pgfpathlineto{\pgfqpoint{15.796192in}{4.085101in}}%
\pgfpathlineto{\pgfqpoint{15.745257in}{4.228535in}}%
\pgfpathlineto{\pgfqpoint{15.745257in}{4.371970in}}%
\pgfpathlineto{\pgfqpoint{15.796192in}{4.515404in}}%
\pgfpathlineto{\pgfqpoint{15.895837in}{4.658838in}}%
\pgfpathlineto{\pgfqpoint{16.039836in}{4.802273in}}%
\pgfpathlineto{\pgfqpoint{16.221895in}{4.945707in}}%
\pgfpathlineto{\pgfqpoint{16.434059in}{5.089141in}}%
\pgfpathlineto{\pgfqpoint{16.667054in}{5.232576in}}%
\pgfusepath{stroke}%
\end{pgfscope}%
\begin{pgfscope}%
\pgfpathrectangle{\pgfqpoint{14.112986in}{0.865000in}}{\pgfqpoint{5.595423in}{4.575556in}}%
\pgfusepath{clip}%
\pgfsetrectcap%
\pgfsetroundjoin%
\pgfsetlinewidth{1.505625pt}%
\definecolor{currentstroke}{rgb}{0.839216,0.152941,0.156863}%
\pgfsetstrokecolor{currentstroke}%
\pgfsetdash{}{0pt}%
\pgfpathmoveto{\pgfqpoint{16.910698in}{1.072980in}}%
\pgfpathlineto{\pgfqpoint{17.075920in}{1.216414in}}%
\pgfpathlineto{\pgfqpoint{17.233921in}{1.359848in}}%
\pgfpathlineto{\pgfqpoint{17.377795in}{1.503283in}}%
\pgfpathlineto{\pgfqpoint{17.501255in}{1.646717in}}%
\pgfpathlineto{\pgfqpoint{17.598905in}{1.790152in}}%
\pgfpathlineto{\pgfqpoint{17.666477in}{1.933586in}}%
\pgfpathlineto{\pgfqpoint{17.701018in}{2.077020in}}%
\pgfpathlineto{\pgfqpoint{17.701018in}{2.220455in}}%
\pgfpathlineto{\pgfqpoint{17.666477in}{2.363889in}}%
\pgfpathlineto{\pgfqpoint{17.598905in}{2.507323in}}%
\pgfpathlineto{\pgfqpoint{17.501255in}{2.650758in}}%
\pgfpathlineto{\pgfqpoint{17.377795in}{2.794192in}}%
\pgfpathlineto{\pgfqpoint{17.233921in}{2.937626in}}%
\pgfpathlineto{\pgfqpoint{17.075920in}{3.081061in}}%
\pgfpathlineto{\pgfqpoint{16.910698in}{3.224495in}}%
\pgfpathlineto{\pgfqpoint{16.745476in}{3.367929in}}%
\pgfpathlineto{\pgfqpoint{16.587475in}{3.511364in}}%
\pgfpathlineto{\pgfqpoint{16.443600in}{3.654798in}}%
\pgfpathlineto{\pgfqpoint{16.320140in}{3.798232in}}%
\pgfpathlineto{\pgfqpoint{16.222490in}{3.941667in}}%
\pgfpathlineto{\pgfqpoint{16.154918in}{4.085101in}}%
\pgfpathlineto{\pgfqpoint{16.120377in}{4.228535in}}%
\pgfpathlineto{\pgfqpoint{16.120377in}{4.371970in}}%
\pgfpathlineto{\pgfqpoint{16.154918in}{4.515404in}}%
\pgfpathlineto{\pgfqpoint{16.222490in}{4.658838in}}%
\pgfpathlineto{\pgfqpoint{16.320140in}{4.802273in}}%
\pgfpathlineto{\pgfqpoint{16.443600in}{4.945707in}}%
\pgfpathlineto{\pgfqpoint{16.587475in}{5.089141in}}%
\pgfpathlineto{\pgfqpoint{16.745476in}{5.232576in}}%
\pgfusepath{stroke}%
\end{pgfscope}%
\begin{pgfscope}%
\pgfpathrectangle{\pgfqpoint{14.112986in}{0.865000in}}{\pgfqpoint{5.595423in}{4.575556in}}%
\pgfusepath{clip}%
\pgfsetrectcap%
\pgfsetroundjoin%
\pgfsetlinewidth{1.505625pt}%
\definecolor{currentstroke}{rgb}{0.580392,0.403922,0.741176}%
\pgfsetstrokecolor{currentstroke}%
\pgfsetdash{}{0pt}%
\pgfpathmoveto{\pgfqpoint{16.910698in}{1.072980in}}%
\pgfpathlineto{\pgfqpoint{17.022740in}{1.216414in}}%
\pgfpathlineto{\pgfqpoint{17.129885in}{1.359848in}}%
\pgfpathlineto{\pgfqpoint{17.227450in}{1.503283in}}%
\pgfpathlineto{\pgfqpoint{17.311172in}{1.646717in}}%
\pgfpathlineto{\pgfqpoint{17.377392in}{1.790152in}}%
\pgfpathlineto{\pgfqpoint{17.423214in}{1.933586in}}%
\pgfpathlineto{\pgfqpoint{17.446638in}{2.077020in}}%
\pgfpathlineto{\pgfqpoint{17.446638in}{2.220455in}}%
\pgfpathlineto{\pgfqpoint{17.423214in}{2.363889in}}%
\pgfpathlineto{\pgfqpoint{17.377392in}{2.507323in}}%
\pgfpathlineto{\pgfqpoint{17.311172in}{2.650758in}}%
\pgfpathlineto{\pgfqpoint{17.227450in}{2.794192in}}%
\pgfpathlineto{\pgfqpoint{17.129885in}{2.937626in}}%
\pgfpathlineto{\pgfqpoint{17.022740in}{3.081061in}}%
\pgfpathlineto{\pgfqpoint{16.910698in}{3.224495in}}%
\pgfpathlineto{\pgfqpoint{16.798656in}{3.367929in}}%
\pgfpathlineto{\pgfqpoint{16.691511in}{3.511364in}}%
\pgfpathlineto{\pgfqpoint{16.593945in}{3.654798in}}%
\pgfpathlineto{\pgfqpoint{16.510223in}{3.798232in}}%
\pgfpathlineto{\pgfqpoint{16.444004in}{3.941667in}}%
\pgfpathlineto{\pgfqpoint{16.398181in}{4.085101in}}%
\pgfpathlineto{\pgfqpoint{16.374758in}{4.228535in}}%
\pgfpathlineto{\pgfqpoint{16.374758in}{4.371970in}}%
\pgfpathlineto{\pgfqpoint{16.398181in}{4.515404in}}%
\pgfpathlineto{\pgfqpoint{16.444004in}{4.658838in}}%
\pgfpathlineto{\pgfqpoint{16.510223in}{4.802273in}}%
\pgfpathlineto{\pgfqpoint{16.593945in}{4.945707in}}%
\pgfpathlineto{\pgfqpoint{16.691511in}{5.089141in}}%
\pgfpathlineto{\pgfqpoint{16.798656in}{5.232576in}}%
\pgfusepath{stroke}%
\end{pgfscope}%
\begin{pgfscope}%
\pgfpathrectangle{\pgfqpoint{14.112986in}{0.865000in}}{\pgfqpoint{5.595423in}{4.575556in}}%
\pgfusepath{clip}%
\pgfsetrectcap%
\pgfsetroundjoin%
\pgfsetlinewidth{1.505625pt}%
\definecolor{currentstroke}{rgb}{0.549020,0.337255,0.294118}%
\pgfsetstrokecolor{currentstroke}%
\pgfsetdash{}{0pt}%
\pgfpathmoveto{\pgfqpoint{16.910698in}{1.072980in}}%
\pgfpathlineto{\pgfqpoint{16.986677in}{1.216414in}}%
\pgfpathlineto{\pgfqpoint{17.059335in}{1.359848in}}%
\pgfpathlineto{\pgfqpoint{17.125497in}{1.503283in}}%
\pgfpathlineto{\pgfqpoint{17.182272in}{1.646717in}}%
\pgfpathlineto{\pgfqpoint{17.227177in}{1.790152in}}%
\pgfpathlineto{\pgfqpoint{17.258251in}{1.933586in}}%
\pgfpathlineto{\pgfqpoint{17.274135in}{2.077020in}}%
\pgfpathlineto{\pgfqpoint{17.274135in}{2.220455in}}%
\pgfpathlineto{\pgfqpoint{17.258251in}{2.363889in}}%
\pgfpathlineto{\pgfqpoint{17.227177in}{2.507323in}}%
\pgfpathlineto{\pgfqpoint{17.182272in}{2.650758in}}%
\pgfpathlineto{\pgfqpoint{17.125497in}{2.794192in}}%
\pgfpathlineto{\pgfqpoint{17.059335in}{2.937626in}}%
\pgfpathlineto{\pgfqpoint{16.986677in}{3.081061in}}%
\pgfpathlineto{\pgfqpoint{16.910698in}{3.224495in}}%
\pgfpathlineto{\pgfqpoint{16.834719in}{3.367929in}}%
\pgfpathlineto{\pgfqpoint{16.762060in}{3.511364in}}%
\pgfpathlineto{\pgfqpoint{16.695898in}{3.654798in}}%
\pgfpathlineto{\pgfqpoint{16.639124in}{3.798232in}}%
\pgfpathlineto{\pgfqpoint{16.594218in}{3.941667in}}%
\pgfpathlineto{\pgfqpoint{16.563145in}{4.085101in}}%
\pgfpathlineto{\pgfqpoint{16.547261in}{4.228535in}}%
\pgfpathlineto{\pgfqpoint{16.547261in}{4.371970in}}%
\pgfpathlineto{\pgfqpoint{16.563145in}{4.515404in}}%
\pgfpathlineto{\pgfqpoint{16.594218in}{4.658838in}}%
\pgfpathlineto{\pgfqpoint{16.639124in}{4.802273in}}%
\pgfpathlineto{\pgfqpoint{16.695898in}{4.945707in}}%
\pgfpathlineto{\pgfqpoint{16.762060in}{5.089141in}}%
\pgfpathlineto{\pgfqpoint{16.834719in}{5.232576in}}%
\pgfusepath{stroke}%
\end{pgfscope}%
\begin{pgfscope}%
\pgfsetrectcap%
\pgfsetmiterjoin%
\pgfsetlinewidth{0.803000pt}%
\definecolor{currentstroke}{rgb}{0.000000,0.000000,0.000000}%
\pgfsetstrokecolor{currentstroke}%
\pgfsetdash{}{0pt}%
\pgfpathmoveto{\pgfqpoint{14.112986in}{0.865000in}}%
\pgfpathlineto{\pgfqpoint{14.112986in}{5.440556in}}%
\pgfusepath{stroke}%
\end{pgfscope}%
\begin{pgfscope}%
\pgfsetrectcap%
\pgfsetmiterjoin%
\pgfsetlinewidth{0.803000pt}%
\definecolor{currentstroke}{rgb}{0.000000,0.000000,0.000000}%
\pgfsetstrokecolor{currentstroke}%
\pgfsetdash{}{0pt}%
\pgfpathmoveto{\pgfqpoint{19.708409in}{0.865000in}}%
\pgfpathlineto{\pgfqpoint{19.708409in}{5.440556in}}%
\pgfusepath{stroke}%
\end{pgfscope}%
\begin{pgfscope}%
\pgfsetrectcap%
\pgfsetmiterjoin%
\pgfsetlinewidth{0.803000pt}%
\definecolor{currentstroke}{rgb}{0.000000,0.000000,0.000000}%
\pgfsetstrokecolor{currentstroke}%
\pgfsetdash{}{0pt}%
\pgfpathmoveto{\pgfqpoint{14.112986in}{0.865000in}}%
\pgfpathlineto{\pgfqpoint{19.708409in}{0.865000in}}%
\pgfusepath{stroke}%
\end{pgfscope}%
\begin{pgfscope}%
\pgfsetrectcap%
\pgfsetmiterjoin%
\pgfsetlinewidth{0.803000pt}%
\definecolor{currentstroke}{rgb}{0.000000,0.000000,0.000000}%
\pgfsetstrokecolor{currentstroke}%
\pgfsetdash{}{0pt}%
\pgfpathmoveto{\pgfqpoint{14.112986in}{5.440556in}}%
\pgfpathlineto{\pgfqpoint{19.708409in}{5.440556in}}%
\pgfusepath{stroke}%
\end{pgfscope}%
\begin{pgfscope}%
\definecolor{textcolor}{rgb}{0.000000,0.000000,0.000000}%
\pgfsetstrokecolor{textcolor}%
\pgfsetfillcolor{textcolor}%
\pgftext[x=16.910698in,y=5.523889in,,base]{\color{textcolor}\sffamily\fontsize{20.000000}{24.000000}\selectfont Velocity decay (\(\displaystyle \omega\)=1.7)}%
\end{pgfscope}%
\begin{pgfscope}%
\pgfsetbuttcap%
\pgfsetmiterjoin%
\definecolor{currentfill}{rgb}{1.000000,1.000000,1.000000}%
\pgfsetfillcolor{currentfill}%
\pgfsetfillopacity{0.800000}%
\pgfsetlinewidth{1.003750pt}%
\definecolor{currentstroke}{rgb}{0.800000,0.800000,0.800000}%
\pgfsetstrokecolor{currentstroke}%
\pgfsetstrokeopacity{0.800000}%
\pgfsetdash{}{0pt}%
\pgfpathmoveto{\pgfqpoint{14.307431in}{1.003889in}}%
\pgfpathlineto{\pgfqpoint{16.244903in}{1.003889in}}%
\pgfpathquadraticcurveto{\pgfqpoint{16.300459in}{1.003889in}}{\pgfqpoint{16.300459in}{1.059444in}}%
\pgfpathlineto{\pgfqpoint{16.300459in}{3.477954in}}%
\pgfpathquadraticcurveto{\pgfqpoint{16.300459in}{3.533510in}}{\pgfqpoint{16.244903in}{3.533510in}}%
\pgfpathlineto{\pgfqpoint{14.307431in}{3.533510in}}%
\pgfpathquadraticcurveto{\pgfqpoint{14.251875in}{3.533510in}}{\pgfqpoint{14.251875in}{3.477954in}}%
\pgfpathlineto{\pgfqpoint{14.251875in}{1.059444in}}%
\pgfpathquadraticcurveto{\pgfqpoint{14.251875in}{1.003889in}}{\pgfqpoint{14.307431in}{1.003889in}}%
\pgfpathlineto{\pgfqpoint{14.307431in}{1.003889in}}%
\pgfpathclose%
\pgfusepath{stroke,fill}%
\end{pgfscope}%
\begin{pgfscope}%
\pgfsetrectcap%
\pgfsetroundjoin%
\pgfsetlinewidth{1.505625pt}%
\definecolor{currentstroke}{rgb}{0.121569,0.466667,0.705882}%
\pgfsetstrokecolor{currentstroke}%
\pgfsetdash{}{0pt}%
\pgfpathmoveto{\pgfqpoint{14.362986in}{3.308575in}}%
\pgfpathlineto{\pgfqpoint{14.640764in}{3.308575in}}%
\pgfpathlineto{\pgfqpoint{14.918542in}{3.308575in}}%
\pgfusepath{stroke}%
\end{pgfscope}%
\begin{pgfscope}%
\definecolor{textcolor}{rgb}{0.000000,0.000000,0.000000}%
\pgfsetstrokecolor{textcolor}%
\pgfsetfillcolor{textcolor}%
\pgftext[x=15.140764in,y=3.211353in,left,base]{\color{textcolor}\sffamily\fontsize{20.000000}{24.000000}\selectfont t=0}%
\end{pgfscope}%
\begin{pgfscope}%
\pgfsetrectcap%
\pgfsetroundjoin%
\pgfsetlinewidth{1.505625pt}%
\definecolor{currentstroke}{rgb}{1.000000,0.498039,0.054902}%
\pgfsetstrokecolor{currentstroke}%
\pgfsetdash{}{0pt}%
\pgfpathmoveto{\pgfqpoint{14.362986in}{2.900860in}}%
\pgfpathlineto{\pgfqpoint{14.640764in}{2.900860in}}%
\pgfpathlineto{\pgfqpoint{14.918542in}{2.900860in}}%
\pgfusepath{stroke}%
\end{pgfscope}%
\begin{pgfscope}%
\definecolor{textcolor}{rgb}{0.000000,0.000000,0.000000}%
\pgfsetstrokecolor{textcolor}%
\pgfsetfillcolor{textcolor}%
\pgftext[x=15.140764in,y=2.803638in,left,base]{\color{textcolor}\sffamily\fontsize{20.000000}{24.000000}\selectfont t=300}%
\end{pgfscope}%
\begin{pgfscope}%
\pgfsetrectcap%
\pgfsetroundjoin%
\pgfsetlinewidth{1.505625pt}%
\definecolor{currentstroke}{rgb}{0.172549,0.627451,0.172549}%
\pgfsetstrokecolor{currentstroke}%
\pgfsetdash{}{0pt}%
\pgfpathmoveto{\pgfqpoint{14.362986in}{2.493146in}}%
\pgfpathlineto{\pgfqpoint{14.640764in}{2.493146in}}%
\pgfpathlineto{\pgfqpoint{14.918542in}{2.493146in}}%
\pgfusepath{stroke}%
\end{pgfscope}%
\begin{pgfscope}%
\definecolor{textcolor}{rgb}{0.000000,0.000000,0.000000}%
\pgfsetstrokecolor{textcolor}%
\pgfsetfillcolor{textcolor}%
\pgftext[x=15.140764in,y=2.395923in,left,base]{\color{textcolor}\sffamily\fontsize{20.000000}{24.000000}\selectfont t=600}%
\end{pgfscope}%
\begin{pgfscope}%
\pgfsetrectcap%
\pgfsetroundjoin%
\pgfsetlinewidth{1.505625pt}%
\definecolor{currentstroke}{rgb}{0.839216,0.152941,0.156863}%
\pgfsetstrokecolor{currentstroke}%
\pgfsetdash{}{0pt}%
\pgfpathmoveto{\pgfqpoint{14.362986in}{2.085431in}}%
\pgfpathlineto{\pgfqpoint{14.640764in}{2.085431in}}%
\pgfpathlineto{\pgfqpoint{14.918542in}{2.085431in}}%
\pgfusepath{stroke}%
\end{pgfscope}%
\begin{pgfscope}%
\definecolor{textcolor}{rgb}{0.000000,0.000000,0.000000}%
\pgfsetstrokecolor{textcolor}%
\pgfsetfillcolor{textcolor}%
\pgftext[x=15.140764in,y=1.988209in,left,base]{\color{textcolor}\sffamily\fontsize{20.000000}{24.000000}\selectfont t=900}%
\end{pgfscope}%
\begin{pgfscope}%
\pgfsetrectcap%
\pgfsetroundjoin%
\pgfsetlinewidth{1.505625pt}%
\definecolor{currentstroke}{rgb}{0.580392,0.403922,0.741176}%
\pgfsetstrokecolor{currentstroke}%
\pgfsetdash{}{0pt}%
\pgfpathmoveto{\pgfqpoint{14.362986in}{1.677717in}}%
\pgfpathlineto{\pgfqpoint{14.640764in}{1.677717in}}%
\pgfpathlineto{\pgfqpoint{14.918542in}{1.677717in}}%
\pgfusepath{stroke}%
\end{pgfscope}%
\begin{pgfscope}%
\definecolor{textcolor}{rgb}{0.000000,0.000000,0.000000}%
\pgfsetstrokecolor{textcolor}%
\pgfsetfillcolor{textcolor}%
\pgftext[x=15.140764in,y=1.580494in,left,base]{\color{textcolor}\sffamily\fontsize{20.000000}{24.000000}\selectfont t=1200}%
\end{pgfscope}%
\begin{pgfscope}%
\pgfsetrectcap%
\pgfsetroundjoin%
\pgfsetlinewidth{1.505625pt}%
\definecolor{currentstroke}{rgb}{0.549020,0.337255,0.294118}%
\pgfsetstrokecolor{currentstroke}%
\pgfsetdash{}{0pt}%
\pgfpathmoveto{\pgfqpoint{14.362986in}{1.270002in}}%
\pgfpathlineto{\pgfqpoint{14.640764in}{1.270002in}}%
\pgfpathlineto{\pgfqpoint{14.918542in}{1.270002in}}%
\pgfusepath{stroke}%
\end{pgfscope}%
\begin{pgfscope}%
\definecolor{textcolor}{rgb}{0.000000,0.000000,0.000000}%
\pgfsetstrokecolor{textcolor}%
\pgfsetfillcolor{textcolor}%
\pgftext[x=15.140764in,y=1.172780in,left,base]{\color{textcolor}\sffamily\fontsize{20.000000}{24.000000}\selectfont t=1500}%
\end{pgfscope}%
\end{pgfpicture}%
\makeatother%
\endgroup%
}
\vspace*{-10mm}
\caption[Velocity decay]{The velocity decay for $\omega=[0.3,1.1,1.7]$. The more viscose the fluid the faster the amplitude decays.}
\label{fig:m3-2-vel}
\end{figure}
The velocity decays much slower with higher $\omega$ compared to smaller values, i.e. a positive correlation between collision frequency and the time to reach the equilibrium.
Similarly to the density decay, this is in line with expectations and shows that the simulation is working.

The comparison between the numerical decay of the initial perturbation to the theoretical decay is shown in \ref{fig:m3-2-norm-vel} for $0.1 \leq \omega \leq 1.9 $.
\begin{figure}[ht]
\centering
\resizebox{0.9\columnwidth}{!}{\large%% Creator: Matplotlib, PGF backend
%%
%% To include the figure in your LaTeX document, write
%%   \input{<filename>.pgf}
%%
%% Make sure the required packages are loaded in your preamble
%%   \usepackage{pgf}
%%
%% Also ensure that all the required font packages are loaded; for instance,
%% the lmodern package is sometimes necessary when using math font.
%%   \usepackage{lmodern}
%%
%% Figures using additional raster images can only be included by \input if
%% they are in the same directory as the main LaTeX file. For loading figures
%% from other directories you can use the `import` package
%%   \usepackage{import}
%%
%% and then include the figures with
%%   \import{<path to file>}{<filename>.pgf}
%%
%% Matplotlib used the following preamble
%%   \usepackage{fontspec}
%%   \setmainfont{DejaVuSerif.ttf}[Path=\detokenize{/home/joe/miniconda3/envs/high/lib/python3.9/site-packages/matplotlib/mpl-data/fonts/ttf/}]
%%   \setsansfont{DejaVuSans.ttf}[Path=\detokenize{/home/joe/miniconda3/envs/high/lib/python3.9/site-packages/matplotlib/mpl-data/fonts/ttf/}]
%%   \setmonofont{DejaVuSansMono.ttf}[Path=\detokenize{/home/joe/miniconda3/envs/high/lib/python3.9/site-packages/matplotlib/mpl-data/fonts/ttf/}]
%%
\begingroup%
\makeatletter%
\begin{pgfpicture}%
\pgfpathrectangle{\pgfpointorigin}{\pgfqpoint{15.000000in}{8.000000in}}%
\pgfusepath{use as bounding box, clip}%
\begin{pgfscope}%
\pgfsetbuttcap%
\pgfsetmiterjoin%
\pgfsetlinewidth{0.000000pt}%
\definecolor{currentstroke}{rgb}{1.000000,1.000000,1.000000}%
\pgfsetstrokecolor{currentstroke}%
\pgfsetstrokeopacity{0.000000}%
\pgfsetdash{}{0pt}%
\pgfpathmoveto{\pgfqpoint{0.000000in}{0.000000in}}%
\pgfpathlineto{\pgfqpoint{15.000000in}{0.000000in}}%
\pgfpathlineto{\pgfqpoint{15.000000in}{8.000000in}}%
\pgfpathlineto{\pgfqpoint{0.000000in}{8.000000in}}%
\pgfpathlineto{\pgfqpoint{0.000000in}{0.000000in}}%
\pgfpathclose%
\pgfusepath{}%
\end{pgfscope}%
\begin{pgfscope}%
\pgfsetbuttcap%
\pgfsetmiterjoin%
\definecolor{currentfill}{rgb}{1.000000,1.000000,1.000000}%
\pgfsetfillcolor{currentfill}%
\pgfsetlinewidth{0.000000pt}%
\definecolor{currentstroke}{rgb}{0.000000,0.000000,0.000000}%
\pgfsetstrokecolor{currentstroke}%
\pgfsetstrokeopacity{0.000000}%
\pgfsetdash{}{0pt}%
\pgfpathmoveto{\pgfqpoint{1.875000in}{1.000000in}}%
\pgfpathlineto{\pgfqpoint{7.159091in}{1.000000in}}%
\pgfpathlineto{\pgfqpoint{7.159091in}{7.040000in}}%
\pgfpathlineto{\pgfqpoint{1.875000in}{7.040000in}}%
\pgfpathlineto{\pgfqpoint{1.875000in}{1.000000in}}%
\pgfpathclose%
\pgfusepath{fill}%
\end{pgfscope}%
\begin{pgfscope}%
\pgfpathrectangle{\pgfqpoint{1.875000in}{1.000000in}}{\pgfqpoint{5.284091in}{6.040000in}}%
\pgfusepath{clip}%
\pgfsetrectcap%
\pgfsetroundjoin%
\pgfsetlinewidth{0.803000pt}%
\definecolor{currentstroke}{rgb}{0.690196,0.690196,0.690196}%
\pgfsetstrokecolor{currentstroke}%
\pgfsetdash{}{0pt}%
\pgfpathmoveto{\pgfqpoint{2.115186in}{1.000000in}}%
\pgfpathlineto{\pgfqpoint{2.115186in}{7.040000in}}%
\pgfusepath{stroke}%
\end{pgfscope}%
\begin{pgfscope}%
\pgfsetbuttcap%
\pgfsetroundjoin%
\definecolor{currentfill}{rgb}{0.000000,0.000000,0.000000}%
\pgfsetfillcolor{currentfill}%
\pgfsetlinewidth{0.803000pt}%
\definecolor{currentstroke}{rgb}{0.000000,0.000000,0.000000}%
\pgfsetstrokecolor{currentstroke}%
\pgfsetdash{}{0pt}%
\pgfsys@defobject{currentmarker}{\pgfqpoint{0.000000in}{-0.048611in}}{\pgfqpoint{0.000000in}{0.000000in}}{%
\pgfpathmoveto{\pgfqpoint{0.000000in}{0.000000in}}%
\pgfpathlineto{\pgfqpoint{0.000000in}{-0.048611in}}%
\pgfusepath{stroke,fill}%
}%
\begin{pgfscope}%
\pgfsys@transformshift{2.115186in}{1.000000in}%
\pgfsys@useobject{currentmarker}{}%
\end{pgfscope}%
\end{pgfscope}%
\begin{pgfscope}%
\definecolor{textcolor}{rgb}{0.000000,0.000000,0.000000}%
\pgfsetstrokecolor{textcolor}%
\pgfsetfillcolor{textcolor}%
\pgftext[x=2.115186in,y=0.902778in,,top]{\color{textcolor}\sffamily\fontsize{16.000000}{19.200000}\selectfont 0}%
\end{pgfscope}%
\begin{pgfscope}%
\pgfpathrectangle{\pgfqpoint{1.875000in}{1.000000in}}{\pgfqpoint{5.284091in}{6.040000in}}%
\pgfusepath{clip}%
\pgfsetrectcap%
\pgfsetroundjoin%
\pgfsetlinewidth{0.803000pt}%
\definecolor{currentstroke}{rgb}{0.690196,0.690196,0.690196}%
\pgfsetstrokecolor{currentstroke}%
\pgfsetdash{}{0pt}%
\pgfpathmoveto{\pgfqpoint{2.916073in}{1.000000in}}%
\pgfpathlineto{\pgfqpoint{2.916073in}{7.040000in}}%
\pgfusepath{stroke}%
\end{pgfscope}%
\begin{pgfscope}%
\pgfsetbuttcap%
\pgfsetroundjoin%
\definecolor{currentfill}{rgb}{0.000000,0.000000,0.000000}%
\pgfsetfillcolor{currentfill}%
\pgfsetlinewidth{0.803000pt}%
\definecolor{currentstroke}{rgb}{0.000000,0.000000,0.000000}%
\pgfsetstrokecolor{currentstroke}%
\pgfsetdash{}{0pt}%
\pgfsys@defobject{currentmarker}{\pgfqpoint{0.000000in}{-0.048611in}}{\pgfqpoint{0.000000in}{0.000000in}}{%
\pgfpathmoveto{\pgfqpoint{0.000000in}{0.000000in}}%
\pgfpathlineto{\pgfqpoint{0.000000in}{-0.048611in}}%
\pgfusepath{stroke,fill}%
}%
\begin{pgfscope}%
\pgfsys@transformshift{2.916073in}{1.000000in}%
\pgfsys@useobject{currentmarker}{}%
\end{pgfscope}%
\end{pgfscope}%
\begin{pgfscope}%
\definecolor{textcolor}{rgb}{0.000000,0.000000,0.000000}%
\pgfsetstrokecolor{textcolor}%
\pgfsetfillcolor{textcolor}%
\pgftext[x=2.916073in,y=0.902778in,,top]{\color{textcolor}\sffamily\fontsize{16.000000}{19.200000}\selectfont 500}%
\end{pgfscope}%
\begin{pgfscope}%
\pgfpathrectangle{\pgfqpoint{1.875000in}{1.000000in}}{\pgfqpoint{5.284091in}{6.040000in}}%
\pgfusepath{clip}%
\pgfsetrectcap%
\pgfsetroundjoin%
\pgfsetlinewidth{0.803000pt}%
\definecolor{currentstroke}{rgb}{0.690196,0.690196,0.690196}%
\pgfsetstrokecolor{currentstroke}%
\pgfsetdash{}{0pt}%
\pgfpathmoveto{\pgfqpoint{3.716960in}{1.000000in}}%
\pgfpathlineto{\pgfqpoint{3.716960in}{7.040000in}}%
\pgfusepath{stroke}%
\end{pgfscope}%
\begin{pgfscope}%
\pgfsetbuttcap%
\pgfsetroundjoin%
\definecolor{currentfill}{rgb}{0.000000,0.000000,0.000000}%
\pgfsetfillcolor{currentfill}%
\pgfsetlinewidth{0.803000pt}%
\definecolor{currentstroke}{rgb}{0.000000,0.000000,0.000000}%
\pgfsetstrokecolor{currentstroke}%
\pgfsetdash{}{0pt}%
\pgfsys@defobject{currentmarker}{\pgfqpoint{0.000000in}{-0.048611in}}{\pgfqpoint{0.000000in}{0.000000in}}{%
\pgfpathmoveto{\pgfqpoint{0.000000in}{0.000000in}}%
\pgfpathlineto{\pgfqpoint{0.000000in}{-0.048611in}}%
\pgfusepath{stroke,fill}%
}%
\begin{pgfscope}%
\pgfsys@transformshift{3.716960in}{1.000000in}%
\pgfsys@useobject{currentmarker}{}%
\end{pgfscope}%
\end{pgfscope}%
\begin{pgfscope}%
\definecolor{textcolor}{rgb}{0.000000,0.000000,0.000000}%
\pgfsetstrokecolor{textcolor}%
\pgfsetfillcolor{textcolor}%
\pgftext[x=3.716960in,y=0.902778in,,top]{\color{textcolor}\sffamily\fontsize{16.000000}{19.200000}\selectfont 1000}%
\end{pgfscope}%
\begin{pgfscope}%
\pgfpathrectangle{\pgfqpoint{1.875000in}{1.000000in}}{\pgfqpoint{5.284091in}{6.040000in}}%
\pgfusepath{clip}%
\pgfsetrectcap%
\pgfsetroundjoin%
\pgfsetlinewidth{0.803000pt}%
\definecolor{currentstroke}{rgb}{0.690196,0.690196,0.690196}%
\pgfsetstrokecolor{currentstroke}%
\pgfsetdash{}{0pt}%
\pgfpathmoveto{\pgfqpoint{4.517846in}{1.000000in}}%
\pgfpathlineto{\pgfqpoint{4.517846in}{7.040000in}}%
\pgfusepath{stroke}%
\end{pgfscope}%
\begin{pgfscope}%
\pgfsetbuttcap%
\pgfsetroundjoin%
\definecolor{currentfill}{rgb}{0.000000,0.000000,0.000000}%
\pgfsetfillcolor{currentfill}%
\pgfsetlinewidth{0.803000pt}%
\definecolor{currentstroke}{rgb}{0.000000,0.000000,0.000000}%
\pgfsetstrokecolor{currentstroke}%
\pgfsetdash{}{0pt}%
\pgfsys@defobject{currentmarker}{\pgfqpoint{0.000000in}{-0.048611in}}{\pgfqpoint{0.000000in}{0.000000in}}{%
\pgfpathmoveto{\pgfqpoint{0.000000in}{0.000000in}}%
\pgfpathlineto{\pgfqpoint{0.000000in}{-0.048611in}}%
\pgfusepath{stroke,fill}%
}%
\begin{pgfscope}%
\pgfsys@transformshift{4.517846in}{1.000000in}%
\pgfsys@useobject{currentmarker}{}%
\end{pgfscope}%
\end{pgfscope}%
\begin{pgfscope}%
\definecolor{textcolor}{rgb}{0.000000,0.000000,0.000000}%
\pgfsetstrokecolor{textcolor}%
\pgfsetfillcolor{textcolor}%
\pgftext[x=4.517846in,y=0.902778in,,top]{\color{textcolor}\sffamily\fontsize{16.000000}{19.200000}\selectfont 1500}%
\end{pgfscope}%
\begin{pgfscope}%
\pgfpathrectangle{\pgfqpoint{1.875000in}{1.000000in}}{\pgfqpoint{5.284091in}{6.040000in}}%
\pgfusepath{clip}%
\pgfsetrectcap%
\pgfsetroundjoin%
\pgfsetlinewidth{0.803000pt}%
\definecolor{currentstroke}{rgb}{0.690196,0.690196,0.690196}%
\pgfsetstrokecolor{currentstroke}%
\pgfsetdash{}{0pt}%
\pgfpathmoveto{\pgfqpoint{5.318733in}{1.000000in}}%
\pgfpathlineto{\pgfqpoint{5.318733in}{7.040000in}}%
\pgfusepath{stroke}%
\end{pgfscope}%
\begin{pgfscope}%
\pgfsetbuttcap%
\pgfsetroundjoin%
\definecolor{currentfill}{rgb}{0.000000,0.000000,0.000000}%
\pgfsetfillcolor{currentfill}%
\pgfsetlinewidth{0.803000pt}%
\definecolor{currentstroke}{rgb}{0.000000,0.000000,0.000000}%
\pgfsetstrokecolor{currentstroke}%
\pgfsetdash{}{0pt}%
\pgfsys@defobject{currentmarker}{\pgfqpoint{0.000000in}{-0.048611in}}{\pgfqpoint{0.000000in}{0.000000in}}{%
\pgfpathmoveto{\pgfqpoint{0.000000in}{0.000000in}}%
\pgfpathlineto{\pgfqpoint{0.000000in}{-0.048611in}}%
\pgfusepath{stroke,fill}%
}%
\begin{pgfscope}%
\pgfsys@transformshift{5.318733in}{1.000000in}%
\pgfsys@useobject{currentmarker}{}%
\end{pgfscope}%
\end{pgfscope}%
\begin{pgfscope}%
\definecolor{textcolor}{rgb}{0.000000,0.000000,0.000000}%
\pgfsetstrokecolor{textcolor}%
\pgfsetfillcolor{textcolor}%
\pgftext[x=5.318733in,y=0.902778in,,top]{\color{textcolor}\sffamily\fontsize{16.000000}{19.200000}\selectfont 2000}%
\end{pgfscope}%
\begin{pgfscope}%
\pgfpathrectangle{\pgfqpoint{1.875000in}{1.000000in}}{\pgfqpoint{5.284091in}{6.040000in}}%
\pgfusepath{clip}%
\pgfsetrectcap%
\pgfsetroundjoin%
\pgfsetlinewidth{0.803000pt}%
\definecolor{currentstroke}{rgb}{0.690196,0.690196,0.690196}%
\pgfsetstrokecolor{currentstroke}%
\pgfsetdash{}{0pt}%
\pgfpathmoveto{\pgfqpoint{6.119620in}{1.000000in}}%
\pgfpathlineto{\pgfqpoint{6.119620in}{7.040000in}}%
\pgfusepath{stroke}%
\end{pgfscope}%
\begin{pgfscope}%
\pgfsetbuttcap%
\pgfsetroundjoin%
\definecolor{currentfill}{rgb}{0.000000,0.000000,0.000000}%
\pgfsetfillcolor{currentfill}%
\pgfsetlinewidth{0.803000pt}%
\definecolor{currentstroke}{rgb}{0.000000,0.000000,0.000000}%
\pgfsetstrokecolor{currentstroke}%
\pgfsetdash{}{0pt}%
\pgfsys@defobject{currentmarker}{\pgfqpoint{0.000000in}{-0.048611in}}{\pgfqpoint{0.000000in}{0.000000in}}{%
\pgfpathmoveto{\pgfqpoint{0.000000in}{0.000000in}}%
\pgfpathlineto{\pgfqpoint{0.000000in}{-0.048611in}}%
\pgfusepath{stroke,fill}%
}%
\begin{pgfscope}%
\pgfsys@transformshift{6.119620in}{1.000000in}%
\pgfsys@useobject{currentmarker}{}%
\end{pgfscope}%
\end{pgfscope}%
\begin{pgfscope}%
\definecolor{textcolor}{rgb}{0.000000,0.000000,0.000000}%
\pgfsetstrokecolor{textcolor}%
\pgfsetfillcolor{textcolor}%
\pgftext[x=6.119620in,y=0.902778in,,top]{\color{textcolor}\sffamily\fontsize{16.000000}{19.200000}\selectfont 2500}%
\end{pgfscope}%
\begin{pgfscope}%
\pgfpathrectangle{\pgfqpoint{1.875000in}{1.000000in}}{\pgfqpoint{5.284091in}{6.040000in}}%
\pgfusepath{clip}%
\pgfsetrectcap%
\pgfsetroundjoin%
\pgfsetlinewidth{0.803000pt}%
\definecolor{currentstroke}{rgb}{0.690196,0.690196,0.690196}%
\pgfsetstrokecolor{currentstroke}%
\pgfsetdash{}{0pt}%
\pgfpathmoveto{\pgfqpoint{6.920507in}{1.000000in}}%
\pgfpathlineto{\pgfqpoint{6.920507in}{7.040000in}}%
\pgfusepath{stroke}%
\end{pgfscope}%
\begin{pgfscope}%
\pgfsetbuttcap%
\pgfsetroundjoin%
\definecolor{currentfill}{rgb}{0.000000,0.000000,0.000000}%
\pgfsetfillcolor{currentfill}%
\pgfsetlinewidth{0.803000pt}%
\definecolor{currentstroke}{rgb}{0.000000,0.000000,0.000000}%
\pgfsetstrokecolor{currentstroke}%
\pgfsetdash{}{0pt}%
\pgfsys@defobject{currentmarker}{\pgfqpoint{0.000000in}{-0.048611in}}{\pgfqpoint{0.000000in}{0.000000in}}{%
\pgfpathmoveto{\pgfqpoint{0.000000in}{0.000000in}}%
\pgfpathlineto{\pgfqpoint{0.000000in}{-0.048611in}}%
\pgfusepath{stroke,fill}%
}%
\begin{pgfscope}%
\pgfsys@transformshift{6.920507in}{1.000000in}%
\pgfsys@useobject{currentmarker}{}%
\end{pgfscope}%
\end{pgfscope}%
\begin{pgfscope}%
\definecolor{textcolor}{rgb}{0.000000,0.000000,0.000000}%
\pgfsetstrokecolor{textcolor}%
\pgfsetfillcolor{textcolor}%
\pgftext[x=6.920507in,y=0.902778in,,top]{\color{textcolor}\sffamily\fontsize{16.000000}{19.200000}\selectfont 3000}%
\end{pgfscope}%
\begin{pgfscope}%
\definecolor{textcolor}{rgb}{0.000000,0.000000,0.000000}%
\pgfsetstrokecolor{textcolor}%
\pgfsetfillcolor{textcolor}%
\pgftext[x=4.517045in,y=0.632162in,,top]{\color{textcolor}\sffamily\fontsize{20.000000}{24.000000}\selectfont time}%
\end{pgfscope}%
\begin{pgfscope}%
\pgfpathrectangle{\pgfqpoint{1.875000in}{1.000000in}}{\pgfqpoint{5.284091in}{6.040000in}}%
\pgfusepath{clip}%
\pgfsetrectcap%
\pgfsetroundjoin%
\pgfsetlinewidth{0.803000pt}%
\definecolor{currentstroke}{rgb}{0.690196,0.690196,0.690196}%
\pgfsetstrokecolor{currentstroke}%
\pgfsetdash{}{0pt}%
\pgfpathmoveto{\pgfqpoint{1.875000in}{1.259454in}}%
\pgfpathlineto{\pgfqpoint{7.159091in}{1.259454in}}%
\pgfusepath{stroke}%
\end{pgfscope}%
\begin{pgfscope}%
\pgfsetbuttcap%
\pgfsetroundjoin%
\definecolor{currentfill}{rgb}{0.000000,0.000000,0.000000}%
\pgfsetfillcolor{currentfill}%
\pgfsetlinewidth{0.803000pt}%
\definecolor{currentstroke}{rgb}{0.000000,0.000000,0.000000}%
\pgfsetstrokecolor{currentstroke}%
\pgfsetdash{}{0pt}%
\pgfsys@defobject{currentmarker}{\pgfqpoint{-0.048611in}{0.000000in}}{\pgfqpoint{-0.000000in}{0.000000in}}{%
\pgfpathmoveto{\pgfqpoint{-0.000000in}{0.000000in}}%
\pgfpathlineto{\pgfqpoint{-0.048611in}{0.000000in}}%
\pgfusepath{stroke,fill}%
}%
\begin{pgfscope}%
\pgfsys@transformshift{1.875000in}{1.259454in}%
\pgfsys@useobject{currentmarker}{}%
\end{pgfscope}%
\end{pgfscope}%
\begin{pgfscope}%
\definecolor{textcolor}{rgb}{0.000000,0.000000,0.000000}%
\pgfsetstrokecolor{textcolor}%
\pgfsetfillcolor{textcolor}%
\pgftext[x=1.424371in, y=1.175035in, left, base]{\color{textcolor}\sffamily\fontsize{16.000000}{19.200000}\selectfont 0.5}%
\end{pgfscope}%
\begin{pgfscope}%
\pgfpathrectangle{\pgfqpoint{1.875000in}{1.000000in}}{\pgfqpoint{5.284091in}{6.040000in}}%
\pgfusepath{clip}%
\pgfsetrectcap%
\pgfsetroundjoin%
\pgfsetlinewidth{0.803000pt}%
\definecolor{currentstroke}{rgb}{0.690196,0.690196,0.690196}%
\pgfsetstrokecolor{currentstroke}%
\pgfsetdash{}{0pt}%
\pgfpathmoveto{\pgfqpoint{1.875000in}{2.358412in}}%
\pgfpathlineto{\pgfqpoint{7.159091in}{2.358412in}}%
\pgfusepath{stroke}%
\end{pgfscope}%
\begin{pgfscope}%
\pgfsetbuttcap%
\pgfsetroundjoin%
\definecolor{currentfill}{rgb}{0.000000,0.000000,0.000000}%
\pgfsetfillcolor{currentfill}%
\pgfsetlinewidth{0.803000pt}%
\definecolor{currentstroke}{rgb}{0.000000,0.000000,0.000000}%
\pgfsetstrokecolor{currentstroke}%
\pgfsetdash{}{0pt}%
\pgfsys@defobject{currentmarker}{\pgfqpoint{-0.048611in}{0.000000in}}{\pgfqpoint{-0.000000in}{0.000000in}}{%
\pgfpathmoveto{\pgfqpoint{-0.000000in}{0.000000in}}%
\pgfpathlineto{\pgfqpoint{-0.048611in}{0.000000in}}%
\pgfusepath{stroke,fill}%
}%
\begin{pgfscope}%
\pgfsys@transformshift{1.875000in}{2.358412in}%
\pgfsys@useobject{currentmarker}{}%
\end{pgfscope}%
\end{pgfscope}%
\begin{pgfscope}%
\definecolor{textcolor}{rgb}{0.000000,0.000000,0.000000}%
\pgfsetstrokecolor{textcolor}%
\pgfsetfillcolor{textcolor}%
\pgftext[x=1.424371in, y=2.273993in, left, base]{\color{textcolor}\sffamily\fontsize{16.000000}{19.200000}\selectfont 0.6}%
\end{pgfscope}%
\begin{pgfscope}%
\pgfpathrectangle{\pgfqpoint{1.875000in}{1.000000in}}{\pgfqpoint{5.284091in}{6.040000in}}%
\pgfusepath{clip}%
\pgfsetrectcap%
\pgfsetroundjoin%
\pgfsetlinewidth{0.803000pt}%
\definecolor{currentstroke}{rgb}{0.690196,0.690196,0.690196}%
\pgfsetstrokecolor{currentstroke}%
\pgfsetdash{}{0pt}%
\pgfpathmoveto{\pgfqpoint{1.875000in}{3.457370in}}%
\pgfpathlineto{\pgfqpoint{7.159091in}{3.457370in}}%
\pgfusepath{stroke}%
\end{pgfscope}%
\begin{pgfscope}%
\pgfsetbuttcap%
\pgfsetroundjoin%
\definecolor{currentfill}{rgb}{0.000000,0.000000,0.000000}%
\pgfsetfillcolor{currentfill}%
\pgfsetlinewidth{0.803000pt}%
\definecolor{currentstroke}{rgb}{0.000000,0.000000,0.000000}%
\pgfsetstrokecolor{currentstroke}%
\pgfsetdash{}{0pt}%
\pgfsys@defobject{currentmarker}{\pgfqpoint{-0.048611in}{0.000000in}}{\pgfqpoint{-0.000000in}{0.000000in}}{%
\pgfpathmoveto{\pgfqpoint{-0.000000in}{0.000000in}}%
\pgfpathlineto{\pgfqpoint{-0.048611in}{0.000000in}}%
\pgfusepath{stroke,fill}%
}%
\begin{pgfscope}%
\pgfsys@transformshift{1.875000in}{3.457370in}%
\pgfsys@useobject{currentmarker}{}%
\end{pgfscope}%
\end{pgfscope}%
\begin{pgfscope}%
\definecolor{textcolor}{rgb}{0.000000,0.000000,0.000000}%
\pgfsetstrokecolor{textcolor}%
\pgfsetfillcolor{textcolor}%
\pgftext[x=1.424371in, y=3.372951in, left, base]{\color{textcolor}\sffamily\fontsize{16.000000}{19.200000}\selectfont 0.7}%
\end{pgfscope}%
\begin{pgfscope}%
\pgfpathrectangle{\pgfqpoint{1.875000in}{1.000000in}}{\pgfqpoint{5.284091in}{6.040000in}}%
\pgfusepath{clip}%
\pgfsetrectcap%
\pgfsetroundjoin%
\pgfsetlinewidth{0.803000pt}%
\definecolor{currentstroke}{rgb}{0.690196,0.690196,0.690196}%
\pgfsetstrokecolor{currentstroke}%
\pgfsetdash{}{0pt}%
\pgfpathmoveto{\pgfqpoint{1.875000in}{4.556328in}}%
\pgfpathlineto{\pgfqpoint{7.159091in}{4.556328in}}%
\pgfusepath{stroke}%
\end{pgfscope}%
\begin{pgfscope}%
\pgfsetbuttcap%
\pgfsetroundjoin%
\definecolor{currentfill}{rgb}{0.000000,0.000000,0.000000}%
\pgfsetfillcolor{currentfill}%
\pgfsetlinewidth{0.803000pt}%
\definecolor{currentstroke}{rgb}{0.000000,0.000000,0.000000}%
\pgfsetstrokecolor{currentstroke}%
\pgfsetdash{}{0pt}%
\pgfsys@defobject{currentmarker}{\pgfqpoint{-0.048611in}{0.000000in}}{\pgfqpoint{-0.000000in}{0.000000in}}{%
\pgfpathmoveto{\pgfqpoint{-0.000000in}{0.000000in}}%
\pgfpathlineto{\pgfqpoint{-0.048611in}{0.000000in}}%
\pgfusepath{stroke,fill}%
}%
\begin{pgfscope}%
\pgfsys@transformshift{1.875000in}{4.556328in}%
\pgfsys@useobject{currentmarker}{}%
\end{pgfscope}%
\end{pgfscope}%
\begin{pgfscope}%
\definecolor{textcolor}{rgb}{0.000000,0.000000,0.000000}%
\pgfsetstrokecolor{textcolor}%
\pgfsetfillcolor{textcolor}%
\pgftext[x=1.424371in, y=4.471910in, left, base]{\color{textcolor}\sffamily\fontsize{16.000000}{19.200000}\selectfont 0.8}%
\end{pgfscope}%
\begin{pgfscope}%
\pgfpathrectangle{\pgfqpoint{1.875000in}{1.000000in}}{\pgfqpoint{5.284091in}{6.040000in}}%
\pgfusepath{clip}%
\pgfsetrectcap%
\pgfsetroundjoin%
\pgfsetlinewidth{0.803000pt}%
\definecolor{currentstroke}{rgb}{0.690196,0.690196,0.690196}%
\pgfsetstrokecolor{currentstroke}%
\pgfsetdash{}{0pt}%
\pgfpathmoveto{\pgfqpoint{1.875000in}{5.655286in}}%
\pgfpathlineto{\pgfqpoint{7.159091in}{5.655286in}}%
\pgfusepath{stroke}%
\end{pgfscope}%
\begin{pgfscope}%
\pgfsetbuttcap%
\pgfsetroundjoin%
\definecolor{currentfill}{rgb}{0.000000,0.000000,0.000000}%
\pgfsetfillcolor{currentfill}%
\pgfsetlinewidth{0.803000pt}%
\definecolor{currentstroke}{rgb}{0.000000,0.000000,0.000000}%
\pgfsetstrokecolor{currentstroke}%
\pgfsetdash{}{0pt}%
\pgfsys@defobject{currentmarker}{\pgfqpoint{-0.048611in}{0.000000in}}{\pgfqpoint{-0.000000in}{0.000000in}}{%
\pgfpathmoveto{\pgfqpoint{-0.000000in}{0.000000in}}%
\pgfpathlineto{\pgfqpoint{-0.048611in}{0.000000in}}%
\pgfusepath{stroke,fill}%
}%
\begin{pgfscope}%
\pgfsys@transformshift{1.875000in}{5.655286in}%
\pgfsys@useobject{currentmarker}{}%
\end{pgfscope}%
\end{pgfscope}%
\begin{pgfscope}%
\definecolor{textcolor}{rgb}{0.000000,0.000000,0.000000}%
\pgfsetstrokecolor{textcolor}%
\pgfsetfillcolor{textcolor}%
\pgftext[x=1.424371in, y=5.570868in, left, base]{\color{textcolor}\sffamily\fontsize{16.000000}{19.200000}\selectfont 0.9}%
\end{pgfscope}%
\begin{pgfscope}%
\pgfpathrectangle{\pgfqpoint{1.875000in}{1.000000in}}{\pgfqpoint{5.284091in}{6.040000in}}%
\pgfusepath{clip}%
\pgfsetrectcap%
\pgfsetroundjoin%
\pgfsetlinewidth{0.803000pt}%
\definecolor{currentstroke}{rgb}{0.690196,0.690196,0.690196}%
\pgfsetstrokecolor{currentstroke}%
\pgfsetdash{}{0pt}%
\pgfpathmoveto{\pgfqpoint{1.875000in}{6.754244in}}%
\pgfpathlineto{\pgfqpoint{7.159091in}{6.754244in}}%
\pgfusepath{stroke}%
\end{pgfscope}%
\begin{pgfscope}%
\pgfsetbuttcap%
\pgfsetroundjoin%
\definecolor{currentfill}{rgb}{0.000000,0.000000,0.000000}%
\pgfsetfillcolor{currentfill}%
\pgfsetlinewidth{0.803000pt}%
\definecolor{currentstroke}{rgb}{0.000000,0.000000,0.000000}%
\pgfsetstrokecolor{currentstroke}%
\pgfsetdash{}{0pt}%
\pgfsys@defobject{currentmarker}{\pgfqpoint{-0.048611in}{0.000000in}}{\pgfqpoint{-0.000000in}{0.000000in}}{%
\pgfpathmoveto{\pgfqpoint{-0.000000in}{0.000000in}}%
\pgfpathlineto{\pgfqpoint{-0.048611in}{0.000000in}}%
\pgfusepath{stroke,fill}%
}%
\begin{pgfscope}%
\pgfsys@transformshift{1.875000in}{6.754244in}%
\pgfsys@useobject{currentmarker}{}%
\end{pgfscope}%
\end{pgfscope}%
\begin{pgfscope}%
\definecolor{textcolor}{rgb}{0.000000,0.000000,0.000000}%
\pgfsetstrokecolor{textcolor}%
\pgfsetfillcolor{textcolor}%
\pgftext[x=1.424371in, y=6.669826in, left, base]{\color{textcolor}\sffamily\fontsize{16.000000}{19.200000}\selectfont 1.0}%
\end{pgfscope}%
\begin{pgfscope}%
\definecolor{textcolor}{rgb}{0.000000,0.000000,0.000000}%
\pgfsetstrokecolor{textcolor}%
\pgfsetfillcolor{textcolor}%
\pgftext[x=1.368815in,y=4.020000in,,bottom,rotate=90.000000]{\color{textcolor}\sffamily\fontsize{20.000000}{24.000000}\selectfont \(\displaystyle a(t)/a(0)\)}%
\end{pgfscope}%
\begin{pgfscope}%
\pgfpathrectangle{\pgfqpoint{1.875000in}{1.000000in}}{\pgfqpoint{5.284091in}{6.040000in}}%
\pgfusepath{clip}%
\pgfsetrectcap%
\pgfsetroundjoin%
\pgfsetlinewidth{1.505625pt}%
\definecolor{currentstroke}{rgb}{0.121569,0.466667,0.705882}%
\pgfsetstrokecolor{currentstroke}%
\pgfsetdash{}{0pt}%
\pgfpathmoveto{\pgfqpoint{2.115186in}{6.754244in}}%
\pgfpathlineto{\pgfqpoint{2.116788in}{6.726304in}}%
\pgfpathlineto{\pgfqpoint{2.119991in}{6.468240in}}%
\pgfpathlineto{\pgfqpoint{2.126398in}{5.468051in}}%
\pgfpathlineto{\pgfqpoint{2.136009in}{3.982118in}}%
\pgfpathlineto{\pgfqpoint{2.142416in}{3.441178in}}%
\pgfpathlineto{\pgfqpoint{2.147221in}{3.256996in}}%
\pgfpathlineto{\pgfqpoint{2.152027in}{3.173041in}}%
\pgfpathlineto{\pgfqpoint{2.160036in}{3.064890in}}%
\pgfpathlineto{\pgfqpoint{2.166443in}{2.917358in}}%
\pgfpathlineto{\pgfqpoint{2.193673in}{2.172147in}}%
\pgfpathlineto{\pgfqpoint{2.203283in}{2.025823in}}%
\pgfpathlineto{\pgfqpoint{2.219301in}{1.835865in}}%
\pgfpathlineto{\pgfqpoint{2.233717in}{1.693824in}}%
\pgfpathlineto{\pgfqpoint{2.246531in}{1.598639in}}%
\pgfpathlineto{\pgfqpoint{2.259346in}{1.526979in}}%
\pgfpathlineto{\pgfqpoint{2.272160in}{1.471757in}}%
\pgfpathlineto{\pgfqpoint{2.286576in}{1.423310in}}%
\pgfpathlineto{\pgfqpoint{2.299390in}{1.390319in}}%
\pgfpathlineto{\pgfqpoint{2.313806in}{1.361895in}}%
\pgfpathlineto{\pgfqpoint{2.328222in}{1.340338in}}%
\pgfpathlineto{\pgfqpoint{2.342638in}{1.324029in}}%
\pgfpathlineto{\pgfqpoint{2.358656in}{1.310686in}}%
\pgfpathlineto{\pgfqpoint{2.376275in}{1.300212in}}%
\pgfpathlineto{\pgfqpoint{2.397098in}{1.291635in}}%
\pgfpathlineto{\pgfqpoint{2.421125in}{1.285205in}}%
\pgfpathlineto{\pgfqpoint{2.451558in}{1.280404in}}%
\pgfpathlineto{\pgfqpoint{2.494806in}{1.277066in}}%
\pgfpathlineto{\pgfqpoint{2.565284in}{1.275179in}}%
\pgfpathlineto{\pgfqpoint{2.741479in}{1.274566in}}%
\pgfpathlineto{\pgfqpoint{6.806781in}{1.274545in}}%
\pgfpathlineto{\pgfqpoint{6.918905in}{1.274545in}}%
\pgfpathlineto{\pgfqpoint{6.918905in}{1.274545in}}%
\pgfusepath{stroke}%
\end{pgfscope}%
\begin{pgfscope}%
\pgfpathrectangle{\pgfqpoint{1.875000in}{1.000000in}}{\pgfqpoint{5.284091in}{6.040000in}}%
\pgfusepath{clip}%
\pgfsetrectcap%
\pgfsetroundjoin%
\pgfsetlinewidth{1.505625pt}%
\definecolor{currentstroke}{rgb}{1.000000,0.498039,0.054902}%
\pgfsetstrokecolor{currentstroke}%
\pgfsetdash{}{0pt}%
\pgfpathmoveto{\pgfqpoint{2.115186in}{6.754244in}}%
\pgfpathlineto{\pgfqpoint{2.116788in}{6.726304in}}%
\pgfpathlineto{\pgfqpoint{2.119991in}{6.521851in}}%
\pgfpathlineto{\pgfqpoint{2.147221in}{4.170120in}}%
\pgfpathlineto{\pgfqpoint{2.156832in}{3.627384in}}%
\pgfpathlineto{\pgfqpoint{2.169646in}{3.046530in}}%
\pgfpathlineto{\pgfqpoint{2.184062in}{2.561636in}}%
\pgfpathlineto{\pgfqpoint{2.198478in}{2.214119in}}%
\pgfpathlineto{\pgfqpoint{2.208089in}{2.037220in}}%
\pgfpathlineto{\pgfqpoint{2.220903in}{1.848546in}}%
\pgfpathlineto{\pgfqpoint{2.235319in}{1.691816in}}%
\pgfpathlineto{\pgfqpoint{2.248133in}{1.590651in}}%
\pgfpathlineto{\pgfqpoint{2.259346in}{1.521728in}}%
\pgfpathlineto{\pgfqpoint{2.272160in}{1.460455in}}%
\pgfpathlineto{\pgfqpoint{2.286576in}{1.409975in}}%
\pgfpathlineto{\pgfqpoint{2.296186in}{1.384602in}}%
\pgfpathlineto{\pgfqpoint{2.309001in}{1.357431in}}%
\pgfpathlineto{\pgfqpoint{2.323417in}{1.334750in}}%
\pgfpathlineto{\pgfqpoint{2.339434in}{1.316905in}}%
\pgfpathlineto{\pgfqpoint{2.353850in}{1.305488in}}%
\pgfpathlineto{\pgfqpoint{2.368266in}{1.297119in}}%
\pgfpathlineto{\pgfqpoint{2.385886in}{1.289871in}}%
\pgfpathlineto{\pgfqpoint{2.408311in}{1.283878in}}%
\pgfpathlineto{\pgfqpoint{2.433939in}{1.279876in}}%
\pgfpathlineto{\pgfqpoint{2.475585in}{1.276678in}}%
\pgfpathlineto{\pgfqpoint{2.541258in}{1.275050in}}%
\pgfpathlineto{\pgfqpoint{2.725462in}{1.274554in}}%
\pgfpathlineto{\pgfqpoint{6.918905in}{1.274545in}}%
\pgfpathlineto{\pgfqpoint{6.918905in}{1.274545in}}%
\pgfusepath{stroke}%
\end{pgfscope}%
\begin{pgfscope}%
\pgfpathrectangle{\pgfqpoint{1.875000in}{1.000000in}}{\pgfqpoint{5.284091in}{6.040000in}}%
\pgfusepath{clip}%
\pgfsetrectcap%
\pgfsetroundjoin%
\pgfsetlinewidth{1.505625pt}%
\definecolor{currentstroke}{rgb}{0.172549,0.627451,0.172549}%
\pgfsetstrokecolor{currentstroke}%
\pgfsetdash{}{0pt}%
\pgfpathmoveto{\pgfqpoint{2.115186in}{6.754244in}}%
\pgfpathlineto{\pgfqpoint{2.116788in}{6.726304in}}%
\pgfpathlineto{\pgfqpoint{2.121593in}{6.471848in}}%
\pgfpathlineto{\pgfqpoint{2.140814in}{5.334920in}}%
\pgfpathlineto{\pgfqpoint{2.155230in}{4.646986in}}%
\pgfpathlineto{\pgfqpoint{2.171248in}{4.017860in}}%
\pgfpathlineto{\pgfqpoint{2.187266in}{3.506092in}}%
\pgfpathlineto{\pgfqpoint{2.203283in}{3.089791in}}%
\pgfpathlineto{\pgfqpoint{2.219301in}{2.751148in}}%
\pgfpathlineto{\pgfqpoint{2.235319in}{2.475677in}}%
\pgfpathlineto{\pgfqpoint{2.251337in}{2.251595in}}%
\pgfpathlineto{\pgfqpoint{2.267354in}{2.069316in}}%
\pgfpathlineto{\pgfqpoint{2.283372in}{1.921041in}}%
\pgfpathlineto{\pgfqpoint{2.299390in}{1.800428in}}%
\pgfpathlineto{\pgfqpoint{2.315408in}{1.702315in}}%
\pgfpathlineto{\pgfqpoint{2.331425in}{1.622507in}}%
\pgfpathlineto{\pgfqpoint{2.349045in}{1.551870in}}%
\pgfpathlineto{\pgfqpoint{2.365063in}{1.500136in}}%
\pgfpathlineto{\pgfqpoint{2.381080in}{1.458053in}}%
\pgfpathlineto{\pgfqpoint{2.397098in}{1.423819in}}%
\pgfpathlineto{\pgfqpoint{2.413116in}{1.395972in}}%
\pgfpathlineto{\pgfqpoint{2.430735in}{1.371298in}}%
\pgfpathlineto{\pgfqpoint{2.446753in}{1.353251in}}%
\pgfpathlineto{\pgfqpoint{2.464373in}{1.337237in}}%
\pgfpathlineto{\pgfqpoint{2.481992in}{1.324462in}}%
\pgfpathlineto{\pgfqpoint{2.499612in}{1.314276in}}%
\pgfpathlineto{\pgfqpoint{2.518833in}{1.305469in}}%
\pgfpathlineto{\pgfqpoint{2.555674in}{1.293861in}}%
\pgfpathlineto{\pgfqpoint{2.592514in}{1.286568in}}%
\pgfpathlineto{\pgfqpoint{2.634161in}{1.281557in}}%
\pgfpathlineto{\pgfqpoint{2.699833in}{1.277556in}}%
\pgfpathlineto{\pgfqpoint{2.800745in}{1.275367in}}%
\pgfpathlineto{\pgfqpoint{3.020188in}{1.274594in}}%
\pgfpathlineto{\pgfqpoint{5.816885in}{1.274545in}}%
\pgfpathlineto{\pgfqpoint{6.918905in}{1.274545in}}%
\pgfpathlineto{\pgfqpoint{6.918905in}{1.274545in}}%
\pgfusepath{stroke}%
\end{pgfscope}%
\begin{pgfscope}%
\pgfpathrectangle{\pgfqpoint{1.875000in}{1.000000in}}{\pgfqpoint{5.284091in}{6.040000in}}%
\pgfusepath{clip}%
\pgfsetrectcap%
\pgfsetroundjoin%
\pgfsetlinewidth{1.505625pt}%
\definecolor{currentstroke}{rgb}{0.839216,0.152941,0.156863}%
\pgfsetstrokecolor{currentstroke}%
\pgfsetdash{}{0pt}%
\pgfpathmoveto{\pgfqpoint{2.115186in}{6.754244in}}%
\pgfpathlineto{\pgfqpoint{2.118389in}{6.673204in}}%
\pgfpathlineto{\pgfqpoint{2.161637in}{5.066506in}}%
\pgfpathlineto{\pgfqpoint{2.169646in}{4.838338in}}%
\pgfpathlineto{\pgfqpoint{2.195275in}{4.151834in}}%
\pgfpathlineto{\pgfqpoint{2.238523in}{3.306410in}}%
\pgfpathlineto{\pgfqpoint{2.249735in}{3.129940in}}%
\pgfpathlineto{\pgfqpoint{2.275363in}{2.772795in}}%
\pgfpathlineto{\pgfqpoint{2.312204in}{2.388859in}}%
\pgfpathlineto{\pgfqpoint{2.341036in}{2.153032in}}%
\pgfpathlineto{\pgfqpoint{2.382682in}{1.902177in}}%
\pgfpathlineto{\pgfqpoint{2.395496in}{1.840325in}}%
\pgfpathlineto{\pgfqpoint{2.416319in}{1.751230in}}%
\pgfpathlineto{\pgfqpoint{2.425930in}{1.715898in}}%
\pgfpathlineto{\pgfqpoint{2.451558in}{1.631081in}}%
\pgfpathlineto{\pgfqpoint{2.481992in}{1.553293in}}%
\pgfpathlineto{\pgfqpoint{2.490001in}{1.536365in}}%
\pgfpathlineto{\pgfqpoint{2.517231in}{1.483474in}}%
\pgfpathlineto{\pgfqpoint{2.563683in}{1.417547in}}%
\pgfpathlineto{\pgfqpoint{2.589311in}{1.391129in}}%
\pgfpathlineto{\pgfqpoint{2.611736in}{1.371217in}}%
\pgfpathlineto{\pgfqpoint{2.638966in}{1.352294in}}%
\pgfpathlineto{\pgfqpoint{2.661391in}{1.339061in}}%
\pgfpathlineto{\pgfqpoint{2.714249in}{1.316643in}}%
\pgfpathlineto{\pgfqpoint{2.755895in}{1.304411in}}%
\pgfpathlineto{\pgfqpoint{2.795940in}{1.296188in}}%
\pgfpathlineto{\pgfqpoint{2.847196in}{1.288793in}}%
\pgfpathlineto{\pgfqpoint{2.901657in}{1.283647in}}%
\pgfpathlineto{\pgfqpoint{3.008976in}{1.278355in}}%
\pgfpathlineto{\pgfqpoint{3.122702in}{1.276052in}}%
\pgfpathlineto{\pgfqpoint{3.322923in}{1.274840in}}%
\pgfpathlineto{\pgfqpoint{4.131819in}{1.274546in}}%
\pgfpathlineto{\pgfqpoint{6.918905in}{1.274545in}}%
\pgfpathlineto{\pgfqpoint{6.918905in}{1.274545in}}%
\pgfusepath{stroke}%
\end{pgfscope}%
\begin{pgfscope}%
\pgfpathrectangle{\pgfqpoint{1.875000in}{1.000000in}}{\pgfqpoint{5.284091in}{6.040000in}}%
\pgfusepath{clip}%
\pgfsetrectcap%
\pgfsetroundjoin%
\pgfsetlinewidth{1.505625pt}%
\definecolor{currentstroke}{rgb}{0.580392,0.403922,0.741176}%
\pgfsetstrokecolor{currentstroke}%
\pgfsetdash{}{0pt}%
\pgfpathmoveto{\pgfqpoint{2.115186in}{6.754244in}}%
\pgfpathlineto{\pgfqpoint{2.119991in}{6.647806in}}%
\pgfpathlineto{\pgfqpoint{2.164841in}{5.480593in}}%
\pgfpathlineto{\pgfqpoint{2.174452in}{5.278465in}}%
\pgfpathlineto{\pgfqpoint{2.184062in}{5.075828in}}%
\pgfpathlineto{\pgfqpoint{2.228912in}{4.250052in}}%
\pgfpathlineto{\pgfqpoint{2.238523in}{4.107155in}}%
\pgfpathlineto{\pgfqpoint{2.248133in}{3.963737in}}%
\pgfpathlineto{\pgfqpoint{2.292983in}{3.379523in}}%
\pgfpathlineto{\pgfqpoint{2.302593in}{3.278501in}}%
\pgfpathlineto{\pgfqpoint{2.312204in}{3.176996in}}%
\pgfpathlineto{\pgfqpoint{2.358656in}{2.753155in}}%
\pgfpathlineto{\pgfqpoint{2.368266in}{2.678538in}}%
\pgfpathlineto{\pgfqpoint{2.376275in}{2.620421in}}%
\pgfpathlineto{\pgfqpoint{2.422726in}{2.320572in}}%
\pgfpathlineto{\pgfqpoint{2.432337in}{2.267819in}}%
\pgfpathlineto{\pgfqpoint{2.440346in}{2.226676in}}%
\pgfpathlineto{\pgfqpoint{2.486797in}{2.014545in}}%
\pgfpathlineto{\pgfqpoint{2.496408in}{1.977250in}}%
\pgfpathlineto{\pgfqpoint{2.504417in}{1.948123in}}%
\pgfpathlineto{\pgfqpoint{2.550868in}{1.798050in}}%
\pgfpathlineto{\pgfqpoint{2.562081in}{1.766662in}}%
\pgfpathlineto{\pgfqpoint{2.568488in}{1.751063in}}%
\pgfpathlineto{\pgfqpoint{2.614939in}{1.644893in}}%
\pgfpathlineto{\pgfqpoint{2.626152in}{1.622702in}}%
\pgfpathlineto{\pgfqpoint{2.632559in}{1.611655in}}%
\pgfpathlineto{\pgfqpoint{2.680612in}{1.534585in}}%
\pgfpathlineto{\pgfqpoint{2.728665in}{1.475135in}}%
\pgfpathlineto{\pgfqpoint{2.778320in}{1.428053in}}%
\pgfpathlineto{\pgfqpoint{2.819966in}{1.396529in}}%
\pgfpathlineto{\pgfqpoint{2.850400in}{1.378237in}}%
\pgfpathlineto{\pgfqpoint{2.861612in}{1.372447in}}%
\pgfpathlineto{\pgfqpoint{2.887241in}{1.359618in}}%
\pgfpathlineto{\pgfqpoint{2.965728in}{1.330130in}}%
\pgfpathlineto{\pgfqpoint{3.069843in}{1.306337in}}%
\pgfpathlineto{\pgfqpoint{3.119498in}{1.298844in}}%
\pgfpathlineto{\pgfqpoint{3.170755in}{1.292922in}}%
\pgfpathlineto{\pgfqpoint{3.217206in}{1.288865in}}%
\pgfpathlineto{\pgfqpoint{3.279675in}{1.284774in}}%
\pgfpathlineto{\pgfqpoint{3.367773in}{1.280891in}}%
\pgfpathlineto{\pgfqpoint{3.558384in}{1.276808in}}%
\pgfpathlineto{\pgfqpoint{3.910774in}{1.274883in}}%
\pgfpathlineto{\pgfqpoint{5.198600in}{1.274546in}}%
\pgfpathlineto{\pgfqpoint{6.918905in}{1.274545in}}%
\pgfpathlineto{\pgfqpoint{6.918905in}{1.274545in}}%
\pgfusepath{stroke}%
\end{pgfscope}%
\begin{pgfscope}%
\pgfpathrectangle{\pgfqpoint{1.875000in}{1.000000in}}{\pgfqpoint{5.284091in}{6.040000in}}%
\pgfusepath{clip}%
\pgfsetrectcap%
\pgfsetroundjoin%
\pgfsetlinewidth{1.505625pt}%
\definecolor{currentstroke}{rgb}{0.549020,0.337255,0.294118}%
\pgfsetstrokecolor{currentstroke}%
\pgfsetdash{}{0pt}%
\pgfpathmoveto{\pgfqpoint{2.115186in}{6.754244in}}%
\pgfpathlineto{\pgfqpoint{2.129602in}{6.479349in}}%
\pgfpathlineto{\pgfqpoint{2.132805in}{6.419216in}}%
\pgfpathlineto{\pgfqpoint{2.140814in}{6.288437in}}%
\pgfpathlineto{\pgfqpoint{2.153629in}{6.060759in}}%
\pgfpathlineto{\pgfqpoint{2.161637in}{5.908474in}}%
\pgfpathlineto{\pgfqpoint{2.164841in}{5.854916in}}%
\pgfpathlineto{\pgfqpoint{2.172850in}{5.738495in}}%
\pgfpathlineto{\pgfqpoint{2.185664in}{5.535784in}}%
\pgfpathlineto{\pgfqpoint{2.193673in}{5.400215in}}%
\pgfpathlineto{\pgfqpoint{2.196876in}{5.352511in}}%
\pgfpathlineto{\pgfqpoint{2.204885in}{5.248872in}}%
\pgfpathlineto{\pgfqpoint{2.217699in}{5.068390in}}%
\pgfpathlineto{\pgfqpoint{2.225708in}{4.947702in}}%
\pgfpathlineto{\pgfqpoint{2.228912in}{4.905214in}}%
\pgfpathlineto{\pgfqpoint{2.236921in}{4.812953in}}%
\pgfpathlineto{\pgfqpoint{2.249735in}{4.652263in}}%
\pgfpathlineto{\pgfqpoint{2.257744in}{4.544822in}}%
\pgfpathlineto{\pgfqpoint{2.262549in}{4.493211in}}%
\pgfpathlineto{\pgfqpoint{2.270558in}{4.404548in}}%
\pgfpathlineto{\pgfqpoint{2.278567in}{4.311565in}}%
\pgfpathlineto{\pgfqpoint{2.286576in}{4.227909in}}%
\pgfpathlineto{\pgfqpoint{2.294585in}{4.140169in}}%
\pgfpathlineto{\pgfqpoint{2.302593in}{4.061239in}}%
\pgfpathlineto{\pgfqpoint{2.310602in}{3.978446in}}%
\pgfpathlineto{\pgfqpoint{2.318611in}{3.903975in}}%
\pgfpathlineto{\pgfqpoint{2.326620in}{3.825851in}}%
\pgfpathlineto{\pgfqpoint{2.334629in}{3.755586in}}%
\pgfpathlineto{\pgfqpoint{2.342638in}{3.681867in}}%
\pgfpathlineto{\pgfqpoint{2.350647in}{3.615570in}}%
\pgfpathlineto{\pgfqpoint{2.358656in}{3.546008in}}%
\pgfpathlineto{\pgfqpoint{2.366664in}{3.483457in}}%
\pgfpathlineto{\pgfqpoint{2.374673in}{3.417817in}}%
\pgfpathlineto{\pgfqpoint{2.382682in}{3.358799in}}%
\pgfpathlineto{\pgfqpoint{2.390691in}{3.296861in}}%
\pgfpathlineto{\pgfqpoint{2.398700in}{3.241176in}}%
\pgfpathlineto{\pgfqpoint{2.406709in}{3.182731in}}%
\pgfpathlineto{\pgfqpoint{2.414718in}{3.130191in}}%
\pgfpathlineto{\pgfqpoint{2.422726in}{3.075042in}}%
\pgfpathlineto{\pgfqpoint{2.430735in}{3.025470in}}%
\pgfpathlineto{\pgfqpoint{2.438744in}{2.973430in}}%
\pgfpathlineto{\pgfqpoint{2.446753in}{2.926658in}}%
\pgfpathlineto{\pgfqpoint{2.454762in}{2.877552in}}%
\pgfpathlineto{\pgfqpoint{2.462771in}{2.833423in}}%
\pgfpathlineto{\pgfqpoint{2.470780in}{2.787086in}}%
\pgfpathlineto{\pgfqpoint{2.478789in}{2.745449in}}%
\pgfpathlineto{\pgfqpoint{2.486797in}{2.701725in}}%
\pgfpathlineto{\pgfqpoint{2.494806in}{2.662440in}}%
\pgfpathlineto{\pgfqpoint{2.502815in}{2.621182in}}%
\pgfpathlineto{\pgfqpoint{2.510824in}{2.584115in}}%
\pgfpathlineto{\pgfqpoint{2.518833in}{2.545184in}}%
\pgfpathlineto{\pgfqpoint{2.526842in}{2.510211in}}%
\pgfpathlineto{\pgfqpoint{2.534851in}{2.473475in}}%
\pgfpathlineto{\pgfqpoint{2.542859in}{2.440478in}}%
\pgfpathlineto{\pgfqpoint{2.550868in}{2.405813in}}%
\pgfpathlineto{\pgfqpoint{2.558877in}{2.374679in}}%
\pgfpathlineto{\pgfqpoint{2.566886in}{2.341969in}}%
\pgfpathlineto{\pgfqpoint{2.574895in}{2.312594in}}%
\pgfpathlineto{\pgfqpoint{2.582904in}{2.281729in}}%
\pgfpathlineto{\pgfqpoint{2.590913in}{2.254013in}}%
\pgfpathlineto{\pgfqpoint{2.598922in}{2.224888in}}%
\pgfpathlineto{\pgfqpoint{2.606930in}{2.198738in}}%
\pgfpathlineto{\pgfqpoint{2.614939in}{2.171255in}}%
\pgfpathlineto{\pgfqpoint{2.622948in}{2.146582in}}%
\pgfpathlineto{\pgfqpoint{2.630957in}{2.120649in}}%
\pgfpathlineto{\pgfqpoint{2.638966in}{2.097369in}}%
\pgfpathlineto{\pgfqpoint{2.646975in}{2.072898in}}%
\pgfpathlineto{\pgfqpoint{2.654984in}{2.050934in}}%
\pgfpathlineto{\pgfqpoint{2.662993in}{2.027843in}}%
\pgfpathlineto{\pgfqpoint{2.671001in}{2.007119in}}%
\pgfpathlineto{\pgfqpoint{2.680612in}{1.981997in}}%
\pgfpathlineto{\pgfqpoint{2.688621in}{1.961027in}}%
\pgfpathlineto{\pgfqpoint{2.695028in}{1.945217in}}%
\pgfpathlineto{\pgfqpoint{2.703037in}{1.926768in}}%
\pgfpathlineto{\pgfqpoint{2.712648in}{1.904400in}}%
\pgfpathlineto{\pgfqpoint{2.720656in}{1.885732in}}%
\pgfpathlineto{\pgfqpoint{2.727063in}{1.871654in}}%
\pgfpathlineto{\pgfqpoint{2.735072in}{1.855230in}}%
\pgfpathlineto{\pgfqpoint{2.744683in}{1.835314in}}%
\pgfpathlineto{\pgfqpoint{2.752692in}{1.818695in}}%
\pgfpathlineto{\pgfqpoint{2.759099in}{1.806159in}}%
\pgfpathlineto{\pgfqpoint{2.767108in}{1.791539in}}%
\pgfpathlineto{\pgfqpoint{2.776718in}{1.773806in}}%
\pgfpathlineto{\pgfqpoint{2.784727in}{1.759011in}}%
\pgfpathlineto{\pgfqpoint{2.791134in}{1.747849in}}%
\pgfpathlineto{\pgfqpoint{2.799143in}{1.734833in}}%
\pgfpathlineto{\pgfqpoint{2.808754in}{1.719044in}}%
\pgfpathlineto{\pgfqpoint{2.818365in}{1.702722in}}%
\pgfpathlineto{\pgfqpoint{2.823170in}{1.695934in}}%
\pgfpathlineto{\pgfqpoint{2.831179in}{1.684347in}}%
\pgfpathlineto{\pgfqpoint{2.840789in}{1.670289in}}%
\pgfpathlineto{\pgfqpoint{2.850400in}{1.655759in}}%
\pgfpathlineto{\pgfqpoint{2.855205in}{1.649714in}}%
\pgfpathlineto{\pgfqpoint{2.864816in}{1.636893in}}%
\pgfpathlineto{\pgfqpoint{2.871223in}{1.628541in}}%
\pgfpathlineto{\pgfqpoint{2.880834in}{1.616444in}}%
\pgfpathlineto{\pgfqpoint{2.887241in}{1.608563in}}%
\pgfpathlineto{\pgfqpoint{2.896851in}{1.597150in}}%
\pgfpathlineto{\pgfqpoint{2.903259in}{1.589713in}}%
\pgfpathlineto{\pgfqpoint{2.912869in}{1.578944in}}%
\pgfpathlineto{\pgfqpoint{2.919276in}{1.571926in}}%
\pgfpathlineto{\pgfqpoint{2.928887in}{1.561765in}}%
\pgfpathlineto{\pgfqpoint{2.935294in}{1.555143in}}%
\pgfpathlineto{\pgfqpoint{2.944905in}{1.545557in}}%
\pgfpathlineto{\pgfqpoint{2.952914in}{1.538067in}}%
\pgfpathlineto{\pgfqpoint{2.962524in}{1.528395in}}%
\pgfpathlineto{\pgfqpoint{2.967330in}{1.524366in}}%
\pgfpathlineto{\pgfqpoint{2.976940in}{1.515831in}}%
\pgfpathlineto{\pgfqpoint{2.984949in}{1.509163in}}%
\pgfpathlineto{\pgfqpoint{2.994560in}{1.500552in}}%
\pgfpathlineto{\pgfqpoint{2.999365in}{1.496964in}}%
\pgfpathlineto{\pgfqpoint{3.008976in}{1.489366in}}%
\pgfpathlineto{\pgfqpoint{3.016984in}{1.483429in}}%
\pgfpathlineto{\pgfqpoint{3.041011in}{1.465804in}}%
\pgfpathlineto{\pgfqpoint{3.049020in}{1.460517in}}%
\pgfpathlineto{\pgfqpoint{3.074648in}{1.443583in}}%
\pgfpathlineto{\pgfqpoint{3.081055in}{1.440119in}}%
\pgfpathlineto{\pgfqpoint{3.106684in}{1.425043in}}%
\pgfpathlineto{\pgfqpoint{3.113091in}{1.421958in}}%
\pgfpathlineto{\pgfqpoint{3.207596in}{1.379066in}}%
\pgfpathlineto{\pgfqpoint{3.233224in}{1.369801in}}%
\pgfpathlineto{\pgfqpoint{3.252445in}{1.363079in}}%
\pgfpathlineto{\pgfqpoint{3.274870in}{1.356585in}}%
\pgfpathlineto{\pgfqpoint{3.350153in}{1.336774in}}%
\pgfpathlineto{\pgfqpoint{3.378985in}{1.330606in}}%
\pgfpathlineto{\pgfqpoint{3.398207in}{1.326822in}}%
\pgfpathlineto{\pgfqpoint{3.441454in}{1.319310in}}%
\pgfpathlineto{\pgfqpoint{3.471888in}{1.314675in}}%
\pgfpathlineto{\pgfqpoint{3.513534in}{1.309046in}}%
\pgfpathlineto{\pgfqpoint{3.651287in}{1.295429in}}%
\pgfpathlineto{\pgfqpoint{3.700942in}{1.291953in}}%
\pgfpathlineto{\pgfqpoint{3.989261in}{1.280664in}}%
\pgfpathlineto{\pgfqpoint{4.380094in}{1.276036in}}%
\pgfpathlineto{\pgfqpoint{4.929502in}{1.274749in}}%
\pgfpathlineto{\pgfqpoint{6.918905in}{1.274546in}}%
\pgfpathlineto{\pgfqpoint{6.918905in}{1.274546in}}%
\pgfusepath{stroke}%
\end{pgfscope}%
\begin{pgfscope}%
\pgfpathrectangle{\pgfqpoint{1.875000in}{1.000000in}}{\pgfqpoint{5.284091in}{6.040000in}}%
\pgfusepath{clip}%
\pgfsetrectcap%
\pgfsetroundjoin%
\pgfsetlinewidth{1.505625pt}%
\definecolor{currentstroke}{rgb}{0.890196,0.466667,0.760784}%
\pgfsetstrokecolor{currentstroke}%
\pgfsetdash{}{0pt}%
\pgfpathmoveto{\pgfqpoint{2.115186in}{6.754244in}}%
\pgfpathlineto{\pgfqpoint{2.118389in}{6.719018in}}%
\pgfpathlineto{\pgfqpoint{2.124797in}{6.644579in}}%
\pgfpathlineto{\pgfqpoint{2.136009in}{6.499146in}}%
\pgfpathlineto{\pgfqpoint{2.142416in}{6.419752in}}%
\pgfpathlineto{\pgfqpoint{2.150425in}{6.316672in}}%
\pgfpathlineto{\pgfqpoint{2.156832in}{6.248777in}}%
\pgfpathlineto{\pgfqpoint{2.168044in}{6.114069in}}%
\pgfpathlineto{\pgfqpoint{2.174452in}{6.040525in}}%
\pgfpathlineto{\pgfqpoint{2.182460in}{5.945045in}}%
\pgfpathlineto{\pgfqpoint{2.188868in}{5.882152in}}%
\pgfpathlineto{\pgfqpoint{2.200080in}{5.757375in}}%
\pgfpathlineto{\pgfqpoint{2.206487in}{5.689248in}}%
\pgfpathlineto{\pgfqpoint{2.214496in}{5.600809in}}%
\pgfpathlineto{\pgfqpoint{2.220903in}{5.542548in}}%
\pgfpathlineto{\pgfqpoint{2.232115in}{5.426970in}}%
\pgfpathlineto{\pgfqpoint{2.238523in}{5.363863in}}%
\pgfpathlineto{\pgfqpoint{2.246531in}{5.281944in}}%
\pgfpathlineto{\pgfqpoint{2.252938in}{5.227976in}}%
\pgfpathlineto{\pgfqpoint{2.264151in}{5.120917in}}%
\pgfpathlineto{\pgfqpoint{2.270558in}{5.062460in}}%
\pgfpathlineto{\pgfqpoint{2.278567in}{4.986581in}}%
\pgfpathlineto{\pgfqpoint{2.284974in}{4.936589in}}%
\pgfpathlineto{\pgfqpoint{2.296186in}{4.837422in}}%
\pgfpathlineto{\pgfqpoint{2.302593in}{4.783271in}}%
\pgfpathlineto{\pgfqpoint{2.310602in}{4.712988in}}%
\pgfpathlineto{\pgfqpoint{2.317009in}{4.666678in}}%
\pgfpathlineto{\pgfqpoint{2.328222in}{4.574822in}}%
\pgfpathlineto{\pgfqpoint{2.334629in}{4.524661in}}%
\pgfpathlineto{\pgfqpoint{2.342638in}{4.459560in}}%
\pgfpathlineto{\pgfqpoint{2.349045in}{4.416662in}}%
\pgfpathlineto{\pgfqpoint{2.360257in}{4.331577in}}%
\pgfpathlineto{\pgfqpoint{2.366664in}{4.285111in}}%
\pgfpathlineto{\pgfqpoint{2.374673in}{4.224810in}}%
\pgfpathlineto{\pgfqpoint{2.381080in}{4.185072in}}%
\pgfpathlineto{\pgfqpoint{2.392293in}{4.106260in}}%
\pgfpathlineto{\pgfqpoint{2.398700in}{4.063217in}}%
\pgfpathlineto{\pgfqpoint{2.406709in}{4.007363in}}%
\pgfpathlineto{\pgfqpoint{2.413116in}{3.970552in}}%
\pgfpathlineto{\pgfqpoint{2.424328in}{3.897550in}}%
\pgfpathlineto{\pgfqpoint{2.430735in}{3.857678in}}%
\pgfpathlineto{\pgfqpoint{2.438744in}{3.805942in}}%
\pgfpathlineto{\pgfqpoint{2.445151in}{3.771843in}}%
\pgfpathlineto{\pgfqpoint{2.456364in}{3.704223in}}%
\pgfpathlineto{\pgfqpoint{2.462771in}{3.667288in}}%
\pgfpathlineto{\pgfqpoint{2.470780in}{3.619367in}}%
\pgfpathlineto{\pgfqpoint{2.477187in}{3.587780in}}%
\pgfpathlineto{\pgfqpoint{2.488399in}{3.525145in}}%
\pgfpathlineto{\pgfqpoint{2.494806in}{3.490931in}}%
\pgfpathlineto{\pgfqpoint{2.502815in}{3.446543in}}%
\pgfpathlineto{\pgfqpoint{2.509222in}{3.417284in}}%
\pgfpathlineto{\pgfqpoint{2.520435in}{3.359265in}}%
\pgfpathlineto{\pgfqpoint{2.526842in}{3.327573in}}%
\pgfpathlineto{\pgfqpoint{2.534851in}{3.286458in}}%
\pgfpathlineto{\pgfqpoint{2.541258in}{3.259353in}}%
\pgfpathlineto{\pgfqpoint{2.552470in}{3.205612in}}%
\pgfpathlineto{\pgfqpoint{2.558877in}{3.176254in}}%
\pgfpathlineto{\pgfqpoint{2.566886in}{3.138171in}}%
\pgfpathlineto{\pgfqpoint{2.573293in}{3.113063in}}%
\pgfpathlineto{\pgfqpoint{2.584506in}{3.063284in}}%
\pgfpathlineto{\pgfqpoint{2.590913in}{3.036089in}}%
\pgfpathlineto{\pgfqpoint{2.598922in}{3.000813in}}%
\pgfpathlineto{\pgfqpoint{2.605329in}{2.977556in}}%
\pgfpathlineto{\pgfqpoint{2.616541in}{2.931446in}}%
\pgfpathlineto{\pgfqpoint{2.622948in}{2.906254in}}%
\pgfpathlineto{\pgfqpoint{2.630957in}{2.873580in}}%
\pgfpathlineto{\pgfqpoint{2.637364in}{2.852036in}}%
\pgfpathlineto{\pgfqpoint{2.648577in}{2.809325in}}%
\pgfpathlineto{\pgfqpoint{2.654984in}{2.785989in}}%
\pgfpathlineto{\pgfqpoint{2.662993in}{2.755724in}}%
\pgfpathlineto{\pgfqpoint{2.669400in}{2.735767in}}%
\pgfpathlineto{\pgfqpoint{2.680612in}{2.696204in}}%
\pgfpathlineto{\pgfqpoint{2.687019in}{2.674588in}}%
\pgfpathlineto{\pgfqpoint{2.695028in}{2.646555in}}%
\pgfpathlineto{\pgfqpoint{2.701435in}{2.628068in}}%
\pgfpathlineto{\pgfqpoint{2.712648in}{2.591422in}}%
\pgfpathlineto{\pgfqpoint{2.719055in}{2.571398in}}%
\pgfpathlineto{\pgfqpoint{2.727063in}{2.545432in}}%
\pgfpathlineto{\pgfqpoint{2.733471in}{2.528306in}}%
\pgfpathlineto{\pgfqpoint{2.744683in}{2.494362in}}%
\pgfpathlineto{\pgfqpoint{2.751090in}{2.475814in}}%
\pgfpathlineto{\pgfqpoint{2.759099in}{2.451762in}}%
\pgfpathlineto{\pgfqpoint{2.765506in}{2.435898in}}%
\pgfpathlineto{\pgfqpoint{2.776718in}{2.404456in}}%
\pgfpathlineto{\pgfqpoint{2.783126in}{2.387274in}}%
\pgfpathlineto{\pgfqpoint{2.791134in}{2.364996in}}%
\pgfpathlineto{\pgfqpoint{2.797542in}{2.350301in}}%
\pgfpathlineto{\pgfqpoint{2.808754in}{2.321177in}}%
\pgfpathlineto{\pgfqpoint{2.815161in}{2.305260in}}%
\pgfpathlineto{\pgfqpoint{2.823170in}{2.284625in}}%
\pgfpathlineto{\pgfqpoint{2.829577in}{2.271012in}}%
\pgfpathlineto{\pgfqpoint{2.840789in}{2.244035in}}%
\pgfpathlineto{\pgfqpoint{2.847196in}{2.229292in}}%
\pgfpathlineto{\pgfqpoint{2.855205in}{2.210178in}}%
\pgfpathlineto{\pgfqpoint{2.861612in}{2.197568in}}%
\pgfpathlineto{\pgfqpoint{2.872825in}{2.172579in}}%
\pgfpathlineto{\pgfqpoint{2.879232in}{2.158922in}}%
\pgfpathlineto{\pgfqpoint{2.887241in}{2.141217in}}%
\pgfpathlineto{\pgfqpoint{2.893648in}{2.129537in}}%
\pgfpathlineto{\pgfqpoint{2.904860in}{2.106390in}}%
\pgfpathlineto{\pgfqpoint{2.911267in}{2.093739in}}%
\pgfpathlineto{\pgfqpoint{2.919276in}{2.077340in}}%
\pgfpathlineto{\pgfqpoint{2.925683in}{2.066520in}}%
\pgfpathlineto{\pgfqpoint{2.936896in}{2.045080in}}%
\pgfpathlineto{\pgfqpoint{2.943303in}{2.033360in}}%
\pgfpathlineto{\pgfqpoint{2.951312in}{2.018170in}}%
\pgfpathlineto{\pgfqpoint{2.957719in}{2.008147in}}%
\pgfpathlineto{\pgfqpoint{2.968931in}{1.988288in}}%
\pgfpathlineto{\pgfqpoint{2.975338in}{1.977432in}}%
\pgfpathlineto{\pgfqpoint{2.983347in}{1.963362in}}%
\pgfpathlineto{\pgfqpoint{2.989754in}{1.954077in}}%
\pgfpathlineto{\pgfqpoint{3.000967in}{1.935682in}}%
\pgfpathlineto{\pgfqpoint{3.007374in}{1.925626in}}%
\pgfpathlineto{\pgfqpoint{3.015383in}{1.912593in}}%
\pgfpathlineto{\pgfqpoint{3.021790in}{1.903992in}}%
\pgfpathlineto{\pgfqpoint{3.033002in}{1.886953in}}%
\pgfpathlineto{\pgfqpoint{3.039409in}{1.877638in}}%
\pgfpathlineto{\pgfqpoint{3.047418in}{1.865566in}}%
\pgfpathlineto{\pgfqpoint{3.053825in}{1.857599in}}%
\pgfpathlineto{\pgfqpoint{3.065038in}{1.841816in}}%
\pgfpathlineto{\pgfqpoint{3.071445in}{1.833187in}}%
\pgfpathlineto{\pgfqpoint{3.079454in}{1.822005in}}%
\pgfpathlineto{\pgfqpoint{3.085861in}{1.814625in}}%
\pgfpathlineto{\pgfqpoint{3.097073in}{1.800005in}}%
\pgfpathlineto{\pgfqpoint{3.105082in}{1.789467in}}%
\pgfpathlineto{\pgfqpoint{3.109887in}{1.782820in}}%
\pgfpathlineto{\pgfqpoint{3.116294in}{1.776850in}}%
\pgfpathlineto{\pgfqpoint{3.143525in}{1.744279in}}%
\pgfpathlineto{\pgfqpoint{3.149932in}{1.737946in}}%
\pgfpathlineto{\pgfqpoint{3.161144in}{1.725402in}}%
\pgfpathlineto{\pgfqpoint{3.169153in}{1.716360in}}%
\pgfpathlineto{\pgfqpoint{3.173958in}{1.710657in}}%
\pgfpathlineto{\pgfqpoint{3.180365in}{1.705534in}}%
\pgfpathlineto{\pgfqpoint{3.207596in}{1.677588in}}%
\pgfpathlineto{\pgfqpoint{3.215604in}{1.670372in}}%
\pgfpathlineto{\pgfqpoint{3.223613in}{1.662451in}}%
\pgfpathlineto{\pgfqpoint{3.231622in}{1.655506in}}%
\pgfpathlineto{\pgfqpoint{3.239631in}{1.647882in}}%
\pgfpathlineto{\pgfqpoint{3.247640in}{1.641198in}}%
\pgfpathlineto{\pgfqpoint{3.255649in}{1.633860in}}%
\pgfpathlineto{\pgfqpoint{3.263658in}{1.627427in}}%
\pgfpathlineto{\pgfqpoint{3.271666in}{1.620365in}}%
\pgfpathlineto{\pgfqpoint{3.279675in}{1.614174in}}%
\pgfpathlineto{\pgfqpoint{3.287684in}{1.607377in}}%
\pgfpathlineto{\pgfqpoint{3.295693in}{1.601418in}}%
\pgfpathlineto{\pgfqpoint{3.303702in}{1.594877in}}%
\pgfpathlineto{\pgfqpoint{3.311711in}{1.589141in}}%
\pgfpathlineto{\pgfqpoint{3.319720in}{1.582846in}}%
\pgfpathlineto{\pgfqpoint{3.327729in}{1.577326in}}%
\pgfpathlineto{\pgfqpoint{3.337339in}{1.570457in}}%
\pgfpathlineto{\pgfqpoint{3.345348in}{1.564521in}}%
\pgfpathlineto{\pgfqpoint{3.351755in}{1.560123in}}%
\pgfpathlineto{\pgfqpoint{3.359764in}{1.555009in}}%
\pgfpathlineto{\pgfqpoint{3.369375in}{1.548647in}}%
\pgfpathlineto{\pgfqpoint{3.378985in}{1.541709in}}%
\pgfpathlineto{\pgfqpoint{3.383791in}{1.539075in}}%
\pgfpathlineto{\pgfqpoint{3.393401in}{1.533060in}}%
\pgfpathlineto{\pgfqpoint{3.399808in}{1.529140in}}%
\pgfpathlineto{\pgfqpoint{3.409419in}{1.523351in}}%
\pgfpathlineto{\pgfqpoint{3.415826in}{1.519578in}}%
\pgfpathlineto{\pgfqpoint{3.425437in}{1.514006in}}%
\pgfpathlineto{\pgfqpoint{3.431844in}{1.510375in}}%
\pgfpathlineto{\pgfqpoint{3.441454in}{1.505012in}}%
\pgfpathlineto{\pgfqpoint{3.447862in}{1.501518in}}%
\pgfpathlineto{\pgfqpoint{3.457472in}{1.496357in}}%
\pgfpathlineto{\pgfqpoint{3.463879in}{1.492993in}}%
\pgfpathlineto{\pgfqpoint{3.473490in}{1.488026in}}%
\pgfpathlineto{\pgfqpoint{3.479897in}{1.484789in}}%
\pgfpathlineto{\pgfqpoint{3.489508in}{1.480008in}}%
\pgfpathlineto{\pgfqpoint{3.497517in}{1.476340in}}%
\pgfpathlineto{\pgfqpoint{3.507127in}{1.471232in}}%
\pgfpathlineto{\pgfqpoint{3.511933in}{1.469293in}}%
\pgfpathlineto{\pgfqpoint{3.521543in}{1.464864in}}%
\pgfpathlineto{\pgfqpoint{3.529552in}{1.461467in}}%
\pgfpathlineto{\pgfqpoint{3.551977in}{1.451708in}}%
\pgfpathlineto{\pgfqpoint{3.563189in}{1.447141in}}%
\pgfpathlineto{\pgfqpoint{3.656092in}{1.412535in}}%
\pgfpathlineto{\pgfqpoint{3.667305in}{1.408674in}}%
\pgfpathlineto{\pgfqpoint{3.673712in}{1.406989in}}%
\pgfpathlineto{\pgfqpoint{3.699340in}{1.398788in}}%
\pgfpathlineto{\pgfqpoint{3.707349in}{1.396838in}}%
\pgfpathlineto{\pgfqpoint{3.814668in}{1.368863in}}%
\pgfpathlineto{\pgfqpoint{3.835491in}{1.364578in}}%
\pgfpathlineto{\pgfqpoint{3.946013in}{1.343633in}}%
\pgfpathlineto{\pgfqpoint{4.155846in}{1.316415in}}%
\pgfpathlineto{\pgfqpoint{4.309616in}{1.303387in}}%
\pgfpathlineto{\pgfqpoint{4.362474in}{1.300081in}}%
\pgfpathlineto{\pgfqpoint{4.633174in}{1.287902in}}%
\pgfpathlineto{\pgfqpoint{5.088078in}{1.279057in}}%
\pgfpathlineto{\pgfqpoint{5.395618in}{1.276705in}}%
\pgfpathlineto{\pgfqpoint{5.791256in}{1.275386in}}%
\pgfpathlineto{\pgfqpoint{6.882064in}{1.274607in}}%
\pgfpathlineto{\pgfqpoint{6.918905in}{1.274602in}}%
\pgfpathlineto{\pgfqpoint{6.918905in}{1.274602in}}%
\pgfusepath{stroke}%
\end{pgfscope}%
\begin{pgfscope}%
\pgfpathrectangle{\pgfqpoint{1.875000in}{1.000000in}}{\pgfqpoint{5.284091in}{6.040000in}}%
\pgfusepath{clip}%
\pgfsetrectcap%
\pgfsetroundjoin%
\pgfsetlinewidth{1.505625pt}%
\definecolor{currentstroke}{rgb}{0.498039,0.498039,0.498039}%
\pgfsetstrokecolor{currentstroke}%
\pgfsetdash{}{0pt}%
\pgfpathmoveto{\pgfqpoint{2.115186in}{6.754244in}}%
\pgfpathlineto{\pgfqpoint{2.116788in}{6.726304in}}%
\pgfpathlineto{\pgfqpoint{2.118389in}{6.734289in}}%
\pgfpathlineto{\pgfqpoint{2.129602in}{6.632185in}}%
\pgfpathlineto{\pgfqpoint{2.132805in}{6.606847in}}%
\pgfpathlineto{\pgfqpoint{2.136009in}{6.595622in}}%
\pgfpathlineto{\pgfqpoint{2.140814in}{6.561376in}}%
\pgfpathlineto{\pgfqpoint{2.150425in}{6.477554in}}%
\pgfpathlineto{\pgfqpoint{2.153629in}{6.462034in}}%
\pgfpathlineto{\pgfqpoint{2.158434in}{6.421865in}}%
\pgfpathlineto{\pgfqpoint{2.164841in}{6.359869in}}%
\pgfpathlineto{\pgfqpoint{2.168044in}{6.349161in}}%
\pgfpathlineto{\pgfqpoint{2.172850in}{6.316497in}}%
\pgfpathlineto{\pgfqpoint{2.182460in}{6.236565in}}%
\pgfpathlineto{\pgfqpoint{2.185664in}{6.221759in}}%
\pgfpathlineto{\pgfqpoint{2.190469in}{6.183445in}}%
\pgfpathlineto{\pgfqpoint{2.196876in}{6.124333in}}%
\pgfpathlineto{\pgfqpoint{2.200080in}{6.114116in}}%
\pgfpathlineto{\pgfqpoint{2.204885in}{6.082959in}}%
\pgfpathlineto{\pgfqpoint{2.214496in}{6.006738in}}%
\pgfpathlineto{\pgfqpoint{2.217699in}{5.992614in}}%
\pgfpathlineto{\pgfqpoint{2.222505in}{5.956068in}}%
\pgfpathlineto{\pgfqpoint{2.228912in}{5.899705in}}%
\pgfpathlineto{\pgfqpoint{2.232115in}{5.889959in}}%
\pgfpathlineto{\pgfqpoint{2.236921in}{5.860239in}}%
\pgfpathlineto{\pgfqpoint{2.246531in}{5.787555in}}%
\pgfpathlineto{\pgfqpoint{2.249735in}{5.774082in}}%
\pgfpathlineto{\pgfqpoint{2.254540in}{5.739224in}}%
\pgfpathlineto{\pgfqpoint{2.260947in}{5.685482in}}%
\pgfpathlineto{\pgfqpoint{2.264151in}{5.676183in}}%
\pgfpathlineto{\pgfqpoint{2.268956in}{5.647835in}}%
\pgfpathlineto{\pgfqpoint{2.278567in}{5.578525in}}%
\pgfpathlineto{\pgfqpoint{2.281770in}{5.565672in}}%
\pgfpathlineto{\pgfqpoint{2.286576in}{5.532423in}}%
\pgfpathlineto{\pgfqpoint{2.292983in}{5.481181in}}%
\pgfpathlineto{\pgfqpoint{2.296186in}{5.472310in}}%
\pgfpathlineto{\pgfqpoint{2.300992in}{5.445269in}}%
\pgfpathlineto{\pgfqpoint{2.310602in}{5.379176in}}%
\pgfpathlineto{\pgfqpoint{2.313806in}{5.366915in}}%
\pgfpathlineto{\pgfqpoint{2.318611in}{5.335202in}}%
\pgfpathlineto{\pgfqpoint{2.325018in}{5.286343in}}%
\pgfpathlineto{\pgfqpoint{2.328222in}{5.277879in}}%
\pgfpathlineto{\pgfqpoint{2.333027in}{5.252086in}}%
\pgfpathlineto{\pgfqpoint{2.342638in}{5.189061in}}%
\pgfpathlineto{\pgfqpoint{2.345841in}{5.177364in}}%
\pgfpathlineto{\pgfqpoint{2.350647in}{5.147115in}}%
\pgfpathlineto{\pgfqpoint{2.357054in}{5.100529in}}%
\pgfpathlineto{\pgfqpoint{2.360257in}{5.092454in}}%
\pgfpathlineto{\pgfqpoint{2.365063in}{5.067851in}}%
\pgfpathlineto{\pgfqpoint{2.374673in}{5.007751in}}%
\pgfpathlineto{\pgfqpoint{2.377877in}{4.996593in}}%
\pgfpathlineto{\pgfqpoint{2.382682in}{4.967741in}}%
\pgfpathlineto{\pgfqpoint{2.389089in}{4.923321in}}%
\pgfpathlineto{\pgfqpoint{2.392293in}{4.915617in}}%
\pgfpathlineto{\pgfqpoint{2.397098in}{4.892149in}}%
\pgfpathlineto{\pgfqpoint{2.406709in}{4.834839in}}%
\pgfpathlineto{\pgfqpoint{2.409912in}{4.824195in}}%
\pgfpathlineto{\pgfqpoint{2.414718in}{4.796675in}}%
\pgfpathlineto{\pgfqpoint{2.421125in}{4.754321in}}%
\pgfpathlineto{\pgfqpoint{2.424328in}{4.746971in}}%
\pgfpathlineto{\pgfqpoint{2.429134in}{4.724586in}}%
\pgfpathlineto{\pgfqpoint{2.438744in}{4.669936in}}%
\pgfpathlineto{\pgfqpoint{2.443550in}{4.652491in}}%
\pgfpathlineto{\pgfqpoint{2.448355in}{4.621886in}}%
\pgfpathlineto{\pgfqpoint{2.453160in}{4.593149in}}%
\pgfpathlineto{\pgfqpoint{2.456364in}{4.586136in}}%
\pgfpathlineto{\pgfqpoint{2.461169in}{4.564784in}}%
\pgfpathlineto{\pgfqpoint{2.470780in}{4.512670in}}%
\pgfpathlineto{\pgfqpoint{2.475585in}{4.496030in}}%
\pgfpathlineto{\pgfqpoint{2.480390in}{4.466838in}}%
\pgfpathlineto{\pgfqpoint{2.485196in}{4.439441in}}%
\pgfpathlineto{\pgfqpoint{2.488399in}{4.432751in}}%
\pgfpathlineto{\pgfqpoint{2.493205in}{4.412384in}}%
\pgfpathlineto{\pgfqpoint{2.502815in}{4.362689in}}%
\pgfpathlineto{\pgfqpoint{2.507620in}{4.346816in}}%
\pgfpathlineto{\pgfqpoint{2.512426in}{4.318973in}}%
\pgfpathlineto{\pgfqpoint{2.517231in}{4.292853in}}%
\pgfpathlineto{\pgfqpoint{2.520435in}{4.286470in}}%
\pgfpathlineto{\pgfqpoint{2.525240in}{4.267043in}}%
\pgfpathlineto{\pgfqpoint{2.534851in}{4.219655in}}%
\pgfpathlineto{\pgfqpoint{2.539656in}{4.204513in}}%
\pgfpathlineto{\pgfqpoint{2.544461in}{4.177956in}}%
\pgfpathlineto{\pgfqpoint{2.549267in}{4.153054in}}%
\pgfpathlineto{\pgfqpoint{2.552470in}{4.146965in}}%
\pgfpathlineto{\pgfqpoint{2.557275in}{4.128434in}}%
\pgfpathlineto{\pgfqpoint{2.566886in}{4.083245in}}%
\pgfpathlineto{\pgfqpoint{2.571691in}{4.068801in}}%
\pgfpathlineto{\pgfqpoint{2.576497in}{4.043471in}}%
\pgfpathlineto{\pgfqpoint{2.581302in}{4.019731in}}%
\pgfpathlineto{\pgfqpoint{2.584506in}{4.013921in}}%
\pgfpathlineto{\pgfqpoint{2.589311in}{3.996245in}}%
\pgfpathlineto{\pgfqpoint{2.598922in}{3.953154in}}%
\pgfpathlineto{\pgfqpoint{2.603727in}{3.939376in}}%
\pgfpathlineto{\pgfqpoint{2.608532in}{3.915215in}}%
\pgfpathlineto{\pgfqpoint{2.613338in}{3.892582in}}%
\pgfpathlineto{\pgfqpoint{2.616541in}{3.887039in}}%
\pgfpathlineto{\pgfqpoint{2.621346in}{3.870179in}}%
\pgfpathlineto{\pgfqpoint{2.630957in}{3.829088in}}%
\pgfpathlineto{\pgfqpoint{2.635762in}{3.815945in}}%
\pgfpathlineto{\pgfqpoint{2.642169in}{3.783058in}}%
\pgfpathlineto{\pgfqpoint{2.645373in}{3.771323in}}%
\pgfpathlineto{\pgfqpoint{2.648577in}{3.766035in}}%
\pgfpathlineto{\pgfqpoint{2.653382in}{3.749952in}}%
\pgfpathlineto{\pgfqpoint{2.662993in}{3.710768in}}%
\pgfpathlineto{\pgfqpoint{2.667798in}{3.698231in}}%
\pgfpathlineto{\pgfqpoint{2.674205in}{3.666863in}}%
\pgfpathlineto{\pgfqpoint{2.677408in}{3.655680in}}%
\pgfpathlineto{\pgfqpoint{2.680612in}{3.650635in}}%
\pgfpathlineto{\pgfqpoint{2.685417in}{3.635294in}}%
\pgfpathlineto{\pgfqpoint{2.695028in}{3.597929in}}%
\pgfpathlineto{\pgfqpoint{2.699833in}{3.585970in}}%
\pgfpathlineto{\pgfqpoint{2.706240in}{3.556051in}}%
\pgfpathlineto{\pgfqpoint{2.709444in}{3.545393in}}%
\pgfpathlineto{\pgfqpoint{2.712648in}{3.540580in}}%
\pgfpathlineto{\pgfqpoint{2.717453in}{3.525947in}}%
\pgfpathlineto{\pgfqpoint{2.727063in}{3.490316in}}%
\pgfpathlineto{\pgfqpoint{2.731869in}{3.478908in}}%
\pgfpathlineto{\pgfqpoint{2.738276in}{3.450372in}}%
\pgfpathlineto{\pgfqpoint{2.741479in}{3.440214in}}%
\pgfpathlineto{\pgfqpoint{2.744683in}{3.435622in}}%
\pgfpathlineto{\pgfqpoint{2.749488in}{3.421665in}}%
\pgfpathlineto{\pgfqpoint{2.759099in}{3.387688in}}%
\pgfpathlineto{\pgfqpoint{2.763904in}{3.376805in}}%
\pgfpathlineto{\pgfqpoint{2.770311in}{3.349587in}}%
\pgfpathlineto{\pgfqpoint{2.773515in}{3.339907in}}%
\pgfpathlineto{\pgfqpoint{2.776718in}{3.335526in}}%
\pgfpathlineto{\pgfqpoint{2.781524in}{3.322213in}}%
\pgfpathlineto{\pgfqpoint{2.791134in}{3.289812in}}%
\pgfpathlineto{\pgfqpoint{2.795940in}{3.279432in}}%
\pgfpathlineto{\pgfqpoint{2.802347in}{3.253471in}}%
\pgfpathlineto{\pgfqpoint{2.805550in}{3.244246in}}%
\pgfpathlineto{\pgfqpoint{2.808754in}{3.240066in}}%
\pgfpathlineto{\pgfqpoint{2.813559in}{3.227367in}}%
\pgfpathlineto{\pgfqpoint{2.823170in}{3.196471in}}%
\pgfpathlineto{\pgfqpoint{2.827975in}{3.186569in}}%
\pgfpathlineto{\pgfqpoint{2.834382in}{3.161807in}}%
\pgfpathlineto{\pgfqpoint{2.837586in}{3.153015in}}%
\pgfpathlineto{\pgfqpoint{2.840789in}{3.149028in}}%
\pgfpathlineto{\pgfqpoint{2.845595in}{3.136914in}}%
\pgfpathlineto{\pgfqpoint{2.855205in}{3.107452in}}%
\pgfpathlineto{\pgfqpoint{2.860011in}{3.098007in}}%
\pgfpathlineto{\pgfqpoint{2.866418in}{3.074389in}}%
\pgfpathlineto{\pgfqpoint{2.869621in}{3.066010in}}%
\pgfpathlineto{\pgfqpoint{2.872825in}{3.062206in}}%
\pgfpathlineto{\pgfqpoint{2.877630in}{3.050652in}}%
\pgfpathlineto{\pgfqpoint{2.887241in}{3.022557in}}%
\pgfpathlineto{\pgfqpoint{2.892046in}{3.013547in}}%
\pgfpathlineto{\pgfqpoint{2.898453in}{2.991021in}}%
\pgfpathlineto{\pgfqpoint{2.901657in}{2.983035in}}%
\pgfpathlineto{\pgfqpoint{2.904860in}{2.979406in}}%
\pgfpathlineto{\pgfqpoint{2.909666in}{2.968384in}}%
\pgfpathlineto{\pgfqpoint{2.919276in}{2.941594in}}%
\pgfpathlineto{\pgfqpoint{2.924082in}{2.932999in}}%
\pgfpathlineto{\pgfqpoint{2.930489in}{2.911513in}}%
\pgfpathlineto{\pgfqpoint{2.933692in}{2.903903in}}%
\pgfpathlineto{\pgfqpoint{2.936896in}{2.900441in}}%
\pgfpathlineto{\pgfqpoint{2.941701in}{2.889928in}}%
\pgfpathlineto{\pgfqpoint{2.951312in}{2.864381in}}%
\pgfpathlineto{\pgfqpoint{2.956117in}{2.856182in}}%
\pgfpathlineto{\pgfqpoint{2.962524in}{2.835689in}}%
\pgfpathlineto{\pgfqpoint{2.965728in}{2.828437in}}%
\pgfpathlineto{\pgfqpoint{2.968931in}{2.825133in}}%
\pgfpathlineto{\pgfqpoint{2.973737in}{2.815105in}}%
\pgfpathlineto{\pgfqpoint{2.983347in}{2.790744in}}%
\pgfpathlineto{\pgfqpoint{2.988153in}{2.782923in}}%
\pgfpathlineto{\pgfqpoint{2.994560in}{2.763377in}}%
\pgfpathlineto{\pgfqpoint{2.997763in}{2.756465in}}%
\pgfpathlineto{\pgfqpoint{3.002569in}{2.750766in}}%
\pgfpathlineto{\pgfqpoint{3.007374in}{2.739287in}}%
\pgfpathlineto{\pgfqpoint{3.013781in}{2.721739in}}%
\pgfpathlineto{\pgfqpoint{3.018586in}{2.716172in}}%
\pgfpathlineto{\pgfqpoint{3.023392in}{2.704962in}}%
\pgfpathlineto{\pgfqpoint{3.029799in}{2.687827in}}%
\pgfpathlineto{\pgfqpoint{3.034604in}{2.682390in}}%
\pgfpathlineto{\pgfqpoint{3.039409in}{2.671442in}}%
\pgfpathlineto{\pgfqpoint{3.045816in}{2.654710in}}%
\pgfpathlineto{\pgfqpoint{3.050622in}{2.649399in}}%
\pgfpathlineto{\pgfqpoint{3.055427in}{2.638706in}}%
\pgfpathlineto{\pgfqpoint{3.061834in}{2.622368in}}%
\pgfpathlineto{\pgfqpoint{3.066639in}{2.617181in}}%
\pgfpathlineto{\pgfqpoint{3.071445in}{2.606738in}}%
\pgfpathlineto{\pgfqpoint{3.077852in}{2.590785in}}%
\pgfpathlineto{\pgfqpoint{3.082657in}{2.585718in}}%
\pgfpathlineto{\pgfqpoint{3.087463in}{2.575520in}}%
\pgfpathlineto{\pgfqpoint{3.093870in}{2.559941in}}%
\pgfpathlineto{\pgfqpoint{3.098675in}{2.554993in}}%
\pgfpathlineto{\pgfqpoint{3.103480in}{2.545032in}}%
\pgfpathlineto{\pgfqpoint{3.109887in}{2.529821in}}%
\pgfpathlineto{\pgfqpoint{3.114693in}{2.524987in}}%
\pgfpathlineto{\pgfqpoint{3.119498in}{2.515259in}}%
\pgfpathlineto{\pgfqpoint{3.125905in}{2.500406in}}%
\pgfpathlineto{\pgfqpoint{3.130710in}{2.495685in}}%
\pgfpathlineto{\pgfqpoint{3.135516in}{2.486184in}}%
\pgfpathlineto{\pgfqpoint{3.141923in}{2.471680in}}%
\pgfpathlineto{\pgfqpoint{3.146728in}{2.467069in}}%
\pgfpathlineto{\pgfqpoint{3.151533in}{2.457790in}}%
\pgfpathlineto{\pgfqpoint{3.157941in}{2.443628in}}%
\pgfpathlineto{\pgfqpoint{3.162746in}{2.439124in}}%
\pgfpathlineto{\pgfqpoint{3.167551in}{2.430062in}}%
\pgfpathlineto{\pgfqpoint{3.173958in}{2.416232in}}%
\pgfpathlineto{\pgfqpoint{3.178764in}{2.411834in}}%
\pgfpathlineto{\pgfqpoint{3.183569in}{2.402983in}}%
\pgfpathlineto{\pgfqpoint{3.189976in}{2.389479in}}%
\pgfpathlineto{\pgfqpoint{3.194781in}{2.385183in}}%
\pgfpathlineto{\pgfqpoint{3.199587in}{2.376539in}}%
\pgfpathlineto{\pgfqpoint{3.205994in}{2.363353in}}%
\pgfpathlineto{\pgfqpoint{3.210799in}{2.359157in}}%
\pgfpathlineto{\pgfqpoint{3.215604in}{2.350715in}}%
\pgfpathlineto{\pgfqpoint{3.222012in}{2.337839in}}%
\pgfpathlineto{\pgfqpoint{3.226817in}{2.333741in}}%
\pgfpathlineto{\pgfqpoint{3.231622in}{2.325496in}}%
\pgfpathlineto{\pgfqpoint{3.238029in}{2.312923in}}%
\pgfpathlineto{\pgfqpoint{3.242835in}{2.308920in}}%
\pgfpathlineto{\pgfqpoint{3.247640in}{2.300868in}}%
\pgfpathlineto{\pgfqpoint{3.254047in}{2.288591in}}%
\pgfpathlineto{\pgfqpoint{3.258852in}{2.284681in}}%
\pgfpathlineto{\pgfqpoint{3.265259in}{2.273331in}}%
\pgfpathlineto{\pgfqpoint{3.270065in}{2.264828in}}%
\pgfpathlineto{\pgfqpoint{3.274870in}{2.261010in}}%
\pgfpathlineto{\pgfqpoint{3.281277in}{2.249925in}}%
\pgfpathlineto{\pgfqpoint{3.286082in}{2.241623in}}%
\pgfpathlineto{\pgfqpoint{3.290888in}{2.237893in}}%
\pgfpathlineto{\pgfqpoint{3.297295in}{2.227068in}}%
\pgfpathlineto{\pgfqpoint{3.302100in}{2.218962in}}%
\pgfpathlineto{\pgfqpoint{3.306906in}{2.215318in}}%
\pgfpathlineto{\pgfqpoint{3.313313in}{2.204746in}}%
\pgfpathlineto{\pgfqpoint{3.318118in}{2.196831in}}%
\pgfpathlineto{\pgfqpoint{3.322923in}{2.193273in}}%
\pgfpathlineto{\pgfqpoint{3.329330in}{2.182947in}}%
\pgfpathlineto{\pgfqpoint{3.334136in}{2.175219in}}%
\pgfpathlineto{\pgfqpoint{3.338941in}{2.171744in}}%
\pgfpathlineto{\pgfqpoint{3.345348in}{2.161659in}}%
\pgfpathlineto{\pgfqpoint{3.350153in}{2.154114in}}%
\pgfpathlineto{\pgfqpoint{3.354959in}{2.150719in}}%
\pgfpathlineto{\pgfqpoint{3.361366in}{2.140870in}}%
\pgfpathlineto{\pgfqpoint{3.366171in}{2.133503in}}%
\pgfpathlineto{\pgfqpoint{3.370976in}{2.130187in}}%
\pgfpathlineto{\pgfqpoint{3.377384in}{2.120569in}}%
\pgfpathlineto{\pgfqpoint{3.382189in}{2.113375in}}%
\pgfpathlineto{\pgfqpoint{3.386994in}{2.110136in}}%
\pgfpathlineto{\pgfqpoint{3.393401in}{2.100743in}}%
\pgfpathlineto{\pgfqpoint{3.398207in}{2.093718in}}%
\pgfpathlineto{\pgfqpoint{3.403012in}{2.090555in}}%
\pgfpathlineto{\pgfqpoint{3.409419in}{2.081381in}}%
\pgfpathlineto{\pgfqpoint{3.414224in}{2.074523in}}%
\pgfpathlineto{\pgfqpoint{3.419030in}{2.071433in}}%
\pgfpathlineto{\pgfqpoint{3.425437in}{2.062473in}}%
\pgfpathlineto{\pgfqpoint{3.430242in}{2.055777in}}%
\pgfpathlineto{\pgfqpoint{3.435047in}{2.052759in}}%
\pgfpathlineto{\pgfqpoint{3.441454in}{2.044009in}}%
\pgfpathlineto{\pgfqpoint{3.446260in}{2.037470in}}%
\pgfpathlineto{\pgfqpoint{3.451065in}{2.034523in}}%
\pgfpathlineto{\pgfqpoint{3.457472in}{2.025977in}}%
\pgfpathlineto{\pgfqpoint{3.462278in}{2.019593in}}%
\pgfpathlineto{\pgfqpoint{3.467083in}{2.016714in}}%
\pgfpathlineto{\pgfqpoint{3.473490in}{2.008368in}}%
\pgfpathlineto{\pgfqpoint{3.478295in}{2.002134in}}%
\pgfpathlineto{\pgfqpoint{3.483101in}{1.999322in}}%
\pgfpathlineto{\pgfqpoint{3.489508in}{1.991171in}}%
\pgfpathlineto{\pgfqpoint{3.494313in}{1.985084in}}%
\pgfpathlineto{\pgfqpoint{3.499118in}{1.982338in}}%
\pgfpathlineto{\pgfqpoint{3.505525in}{1.974377in}}%
\pgfpathlineto{\pgfqpoint{3.510331in}{1.968434in}}%
\pgfpathlineto{\pgfqpoint{3.515136in}{1.965752in}}%
\pgfpathlineto{\pgfqpoint{3.521543in}{1.957977in}}%
\pgfpathlineto{\pgfqpoint{3.526348in}{1.952174in}}%
\pgfpathlineto{\pgfqpoint{3.531154in}{1.949555in}}%
\pgfpathlineto{\pgfqpoint{3.537561in}{1.941961in}}%
\pgfpathlineto{\pgfqpoint{3.542366in}{1.936295in}}%
\pgfpathlineto{\pgfqpoint{3.547172in}{1.933737in}}%
\pgfpathlineto{\pgfqpoint{3.553579in}{1.926321in}}%
\pgfpathlineto{\pgfqpoint{3.558384in}{1.920789in}}%
\pgfpathlineto{\pgfqpoint{3.563189in}{1.918289in}}%
\pgfpathlineto{\pgfqpoint{3.569596in}{1.911047in}}%
\pgfpathlineto{\pgfqpoint{3.574402in}{1.905645in}}%
\pgfpathlineto{\pgfqpoint{3.579207in}{1.903204in}}%
\pgfpathlineto{\pgfqpoint{3.585614in}{1.896131in}}%
\pgfpathlineto{\pgfqpoint{3.590419in}{1.890857in}}%
\pgfpathlineto{\pgfqpoint{3.596827in}{1.887141in}}%
\pgfpathlineto{\pgfqpoint{3.603234in}{1.879184in}}%
\pgfpathlineto{\pgfqpoint{3.606437in}{1.876415in}}%
\pgfpathlineto{\pgfqpoint{3.612844in}{1.872786in}}%
\pgfpathlineto{\pgfqpoint{3.619251in}{1.865015in}}%
\pgfpathlineto{\pgfqpoint{3.622455in}{1.862311in}}%
\pgfpathlineto{\pgfqpoint{3.628862in}{1.858767in}}%
\pgfpathlineto{\pgfqpoint{3.635269in}{1.851177in}}%
\pgfpathlineto{\pgfqpoint{3.638473in}{1.848538in}}%
\pgfpathlineto{\pgfqpoint{3.644880in}{1.845076in}}%
\pgfpathlineto{\pgfqpoint{3.651287in}{1.837664in}}%
\pgfpathlineto{\pgfqpoint{3.654490in}{1.835087in}}%
\pgfpathlineto{\pgfqpoint{3.660897in}{1.831706in}}%
\pgfpathlineto{\pgfqpoint{3.667305in}{1.824468in}}%
\pgfpathlineto{\pgfqpoint{3.670508in}{1.821952in}}%
\pgfpathlineto{\pgfqpoint{3.676915in}{1.818650in}}%
\pgfpathlineto{\pgfqpoint{3.683322in}{1.811581in}}%
\pgfpathlineto{\pgfqpoint{3.686526in}{1.809125in}}%
\pgfpathlineto{\pgfqpoint{3.692933in}{1.805900in}}%
\pgfpathlineto{\pgfqpoint{3.699340in}{1.798995in}}%
\pgfpathlineto{\pgfqpoint{3.702544in}{1.796598in}}%
\pgfpathlineto{\pgfqpoint{3.708951in}{1.793448in}}%
\pgfpathlineto{\pgfqpoint{3.715358in}{1.786705in}}%
\pgfpathlineto{\pgfqpoint{3.718561in}{1.784365in}}%
\pgfpathlineto{\pgfqpoint{3.724968in}{1.781288in}}%
\pgfpathlineto{\pgfqpoint{3.731376in}{1.774703in}}%
\pgfpathlineto{\pgfqpoint{3.734579in}{1.772418in}}%
\pgfpathlineto{\pgfqpoint{3.740986in}{1.769413in}}%
\pgfpathlineto{\pgfqpoint{3.747393in}{1.762982in}}%
\pgfpathlineto{\pgfqpoint{3.750597in}{1.760752in}}%
\pgfpathlineto{\pgfqpoint{3.757004in}{1.757817in}}%
\pgfpathlineto{\pgfqpoint{3.784234in}{1.737836in}}%
\pgfpathlineto{\pgfqpoint{3.790641in}{1.734230in}}%
\pgfpathlineto{\pgfqpoint{3.801854in}{1.726394in}}%
\pgfpathlineto{\pgfqpoint{3.808261in}{1.722090in}}%
\pgfpathlineto{\pgfqpoint{3.816270in}{1.716378in}}%
\pgfpathlineto{\pgfqpoint{3.822677in}{1.712938in}}%
\pgfpathlineto{\pgfqpoint{3.833889in}{1.705466in}}%
\pgfpathlineto{\pgfqpoint{3.840296in}{1.701360in}}%
\pgfpathlineto{\pgfqpoint{3.848305in}{1.695913in}}%
\pgfpathlineto{\pgfqpoint{3.854712in}{1.692632in}}%
\pgfpathlineto{\pgfqpoint{3.865924in}{1.685506in}}%
\pgfpathlineto{\pgfqpoint{3.872332in}{1.681591in}}%
\pgfpathlineto{\pgfqpoint{3.880340in}{1.676397in}}%
\pgfpathlineto{\pgfqpoint{3.886748in}{1.673267in}}%
\pgfpathlineto{\pgfqpoint{3.897960in}{1.666472in}}%
\pgfpathlineto{\pgfqpoint{3.904367in}{1.662737in}}%
\pgfpathlineto{\pgfqpoint{3.912376in}{1.657784in}}%
\pgfpathlineto{\pgfqpoint{3.918783in}{1.654798in}}%
\pgfpathlineto{\pgfqpoint{3.929995in}{1.648318in}}%
\pgfpathlineto{\pgfqpoint{3.938004in}{1.643465in}}%
\pgfpathlineto{\pgfqpoint{3.942810in}{1.640346in}}%
\pgfpathlineto{\pgfqpoint{3.949217in}{1.638135in}}%
\pgfpathlineto{\pgfqpoint{3.978049in}{1.622653in}}%
\pgfpathlineto{\pgfqpoint{3.986058in}{1.618132in}}%
\pgfpathlineto{\pgfqpoint{3.990863in}{1.615229in}}%
\pgfpathlineto{\pgfqpoint{3.998872in}{1.612284in}}%
\pgfpathlineto{\pgfqpoint{4.010084in}{1.606529in}}%
\pgfpathlineto{\pgfqpoint{4.018093in}{1.602217in}}%
\pgfpathlineto{\pgfqpoint{4.022898in}{1.599449in}}%
\pgfpathlineto{\pgfqpoint{4.030907in}{1.596640in}}%
\pgfpathlineto{\pgfqpoint{4.042120in}{1.591153in}}%
\pgfpathlineto{\pgfqpoint{4.050128in}{1.587040in}}%
\pgfpathlineto{\pgfqpoint{4.054934in}{1.584401in}}%
\pgfpathlineto{\pgfqpoint{4.062943in}{1.581721in}}%
\pgfpathlineto{\pgfqpoint{4.075757in}{1.575964in}}%
\pgfpathlineto{\pgfqpoint{4.083766in}{1.571395in}}%
\pgfpathlineto{\pgfqpoint{4.086969in}{1.570049in}}%
\pgfpathlineto{\pgfqpoint{4.094978in}{1.567493in}}%
\pgfpathlineto{\pgfqpoint{4.107792in}{1.562003in}}%
\pgfpathlineto{\pgfqpoint{4.115801in}{1.557645in}}%
\pgfpathlineto{\pgfqpoint{4.119005in}{1.556362in}}%
\pgfpathlineto{\pgfqpoint{4.127014in}{1.553924in}}%
\pgfpathlineto{\pgfqpoint{4.139828in}{1.548688in}}%
\pgfpathlineto{\pgfqpoint{4.147837in}{1.544531in}}%
\pgfpathlineto{\pgfqpoint{4.152642in}{1.543079in}}%
\pgfpathlineto{\pgfqpoint{4.160651in}{1.540171in}}%
\pgfpathlineto{\pgfqpoint{4.168660in}{1.536786in}}%
\pgfpathlineto{\pgfqpoint{4.176669in}{1.533946in}}%
\pgfpathlineto{\pgfqpoint{4.184677in}{1.530641in}}%
\pgfpathlineto{\pgfqpoint{4.192686in}{1.527867in}}%
\pgfpathlineto{\pgfqpoint{4.202297in}{1.524315in}}%
\pgfpathlineto{\pgfqpoint{4.210306in}{1.521068in}}%
\pgfpathlineto{\pgfqpoint{4.216713in}{1.518779in}}%
\pgfpathlineto{\pgfqpoint{4.224722in}{1.516134in}}%
\pgfpathlineto{\pgfqpoint{4.234332in}{1.512746in}}%
\pgfpathlineto{\pgfqpoint{4.243943in}{1.508725in}}%
\pgfpathlineto{\pgfqpoint{4.248748in}{1.507467in}}%
\pgfpathlineto{\pgfqpoint{4.258359in}{1.504139in}}%
\pgfpathlineto{\pgfqpoint{4.264766in}{1.502009in}}%
\pgfpathlineto{\pgfqpoint{4.274377in}{1.498759in}}%
\pgfpathlineto{\pgfqpoint{4.280784in}{1.496679in}}%
\pgfpathlineto{\pgfqpoint{4.290394in}{1.493505in}}%
\pgfpathlineto{\pgfqpoint{4.296802in}{1.491474in}}%
\pgfpathlineto{\pgfqpoint{4.306412in}{1.488373in}}%
\pgfpathlineto{\pgfqpoint{4.312819in}{1.486390in}}%
\pgfpathlineto{\pgfqpoint{4.322430in}{1.483362in}}%
\pgfpathlineto{\pgfqpoint{4.328837in}{1.481426in}}%
\pgfpathlineto{\pgfqpoint{4.338448in}{1.478469in}}%
\pgfpathlineto{\pgfqpoint{4.344855in}{1.476578in}}%
\pgfpathlineto{\pgfqpoint{4.354465in}{1.473690in}}%
\pgfpathlineto{\pgfqpoint{4.360873in}{1.471844in}}%
\pgfpathlineto{\pgfqpoint{4.370483in}{1.469023in}}%
\pgfpathlineto{\pgfqpoint{4.376890in}{1.467220in}}%
\pgfpathlineto{\pgfqpoint{4.386501in}{1.464466in}}%
\pgfpathlineto{\pgfqpoint{4.392908in}{1.462705in}}%
\pgfpathlineto{\pgfqpoint{4.402519in}{1.460015in}}%
\pgfpathlineto{\pgfqpoint{4.410528in}{1.458057in}}%
\pgfpathlineto{\pgfqpoint{4.420138in}{1.454957in}}%
\pgfpathlineto{\pgfqpoint{4.424943in}{1.453990in}}%
\pgfpathlineto{\pgfqpoint{4.434554in}{1.451424in}}%
\pgfpathlineto{\pgfqpoint{4.442563in}{1.449557in}}%
\pgfpathlineto{\pgfqpoint{4.464988in}{1.443823in}}%
\pgfpathlineto{\pgfqpoint{4.476200in}{1.441160in}}%
\pgfpathlineto{\pgfqpoint{4.650794in}{1.403131in}}%
\pgfpathlineto{\pgfqpoint{4.676422in}{1.397994in}}%
\pgfpathlineto{\pgfqpoint{4.684431in}{1.396961in}}%
\pgfpathlineto{\pgfqpoint{4.873440in}{1.366927in}}%
\pgfpathlineto{\pgfqpoint{4.899068in}{1.363469in}}%
\pgfpathlineto{\pgfqpoint{4.918290in}{1.360695in}}%
\pgfpathlineto{\pgfqpoint{4.932706in}{1.359016in}}%
\pgfpathlineto{\pgfqpoint{4.950325in}{1.356705in}}%
\pgfpathlineto{\pgfqpoint{4.964741in}{1.355104in}}%
\pgfpathlineto{\pgfqpoint{4.982361in}{1.352900in}}%
\pgfpathlineto{\pgfqpoint{5.001582in}{1.350964in}}%
\pgfpathlineto{\pgfqpoint{5.028812in}{1.347813in}}%
\pgfpathlineto{\pgfqpoint{5.048033in}{1.345779in}}%
\pgfpathlineto{\pgfqpoint{5.073662in}{1.343293in}}%
\pgfpathlineto{\pgfqpoint{5.102494in}{1.340483in}}%
\pgfpathlineto{\pgfqpoint{5.195397in}{1.331964in}}%
\pgfpathlineto{\pgfqpoint{5.269078in}{1.325883in}}%
\pgfpathlineto{\pgfqpoint{5.294707in}{1.324154in}}%
\pgfpathlineto{\pgfqpoint{5.406831in}{1.316566in}}%
\pgfpathlineto{\pgfqpoint{5.533371in}{1.309382in}}%
\pgfpathlineto{\pgfqpoint{5.816885in}{1.297369in}}%
\pgfpathlineto{\pgfqpoint{5.893770in}{1.294906in}}%
\pgfpathlineto{\pgfqpoint{5.964248in}{1.292942in}}%
\pgfpathlineto{\pgfqpoint{6.588940in}{1.281842in}}%
\pgfpathlineto{\pgfqpoint{6.918905in}{1.279009in}}%
\pgfpathlineto{\pgfqpoint{6.918905in}{1.279009in}}%
\pgfusepath{stroke}%
\end{pgfscope}%
\begin{pgfscope}%
\pgfpathrectangle{\pgfqpoint{1.875000in}{1.000000in}}{\pgfqpoint{5.284091in}{6.040000in}}%
\pgfusepath{clip}%
\pgfsetrectcap%
\pgfsetroundjoin%
\pgfsetlinewidth{1.505625pt}%
\definecolor{currentstroke}{rgb}{0.737255,0.741176,0.133333}%
\pgfsetstrokecolor{currentstroke}%
\pgfsetdash{}{0pt}%
\pgfpathmoveto{\pgfqpoint{2.115186in}{6.754244in}}%
\pgfpathlineto{\pgfqpoint{2.116788in}{6.726304in}}%
\pgfpathlineto{\pgfqpoint{2.118389in}{6.749560in}}%
\pgfpathlineto{\pgfqpoint{2.119991in}{6.734363in}}%
\pgfpathlineto{\pgfqpoint{2.121593in}{6.740239in}}%
\pgfpathlineto{\pgfqpoint{2.131204in}{6.667181in}}%
\pgfpathlineto{\pgfqpoint{2.132805in}{6.671213in}}%
\pgfpathlineto{\pgfqpoint{2.136009in}{6.671888in}}%
\pgfpathlineto{\pgfqpoint{2.139213in}{6.663045in}}%
\pgfpathlineto{\pgfqpoint{2.144018in}{6.632092in}}%
\pgfpathlineto{\pgfqpoint{2.147221in}{6.599923in}}%
\pgfpathlineto{\pgfqpoint{2.150425in}{6.605287in}}%
\pgfpathlineto{\pgfqpoint{2.152027in}{6.604450in}}%
\pgfpathlineto{\pgfqpoint{2.155230in}{6.595740in}}%
\pgfpathlineto{\pgfqpoint{2.160036in}{6.565259in}}%
\pgfpathlineto{\pgfqpoint{2.163239in}{6.533486in}}%
\pgfpathlineto{\pgfqpoint{2.166443in}{6.538775in}}%
\pgfpathlineto{\pgfqpoint{2.168044in}{6.537939in}}%
\pgfpathlineto{\pgfqpoint{2.171248in}{6.529337in}}%
\pgfpathlineto{\pgfqpoint{2.176053in}{6.499233in}}%
\pgfpathlineto{\pgfqpoint{2.179257in}{6.467868in}}%
\pgfpathlineto{\pgfqpoint{2.182460in}{6.473087in}}%
\pgfpathlineto{\pgfqpoint{2.184062in}{6.472260in}}%
\pgfpathlineto{\pgfqpoint{2.187266in}{6.463763in}}%
\pgfpathlineto{\pgfqpoint{2.192071in}{6.434031in}}%
\pgfpathlineto{\pgfqpoint{2.195275in}{6.403068in}}%
\pgfpathlineto{\pgfqpoint{2.198478in}{6.408220in}}%
\pgfpathlineto{\pgfqpoint{2.200080in}{6.407402in}}%
\pgfpathlineto{\pgfqpoint{2.203283in}{6.399008in}}%
\pgfpathlineto{\pgfqpoint{2.208089in}{6.369643in}}%
\pgfpathlineto{\pgfqpoint{2.211292in}{6.339076in}}%
\pgfpathlineto{\pgfqpoint{2.214496in}{6.344161in}}%
\pgfpathlineto{\pgfqpoint{2.216098in}{6.343352in}}%
\pgfpathlineto{\pgfqpoint{2.219301in}{6.335061in}}%
\pgfpathlineto{\pgfqpoint{2.224107in}{6.306058in}}%
\pgfpathlineto{\pgfqpoint{2.227310in}{6.275884in}}%
\pgfpathlineto{\pgfqpoint{2.230514in}{6.280903in}}%
\pgfpathlineto{\pgfqpoint{2.232115in}{6.280102in}}%
\pgfpathlineto{\pgfqpoint{2.235319in}{6.271912in}}%
\pgfpathlineto{\pgfqpoint{2.240124in}{6.243267in}}%
\pgfpathlineto{\pgfqpoint{2.243328in}{6.213479in}}%
\pgfpathlineto{\pgfqpoint{2.246531in}{6.218433in}}%
\pgfpathlineto{\pgfqpoint{2.248133in}{6.217641in}}%
\pgfpathlineto{\pgfqpoint{2.251337in}{6.209550in}}%
\pgfpathlineto{\pgfqpoint{2.256142in}{6.181260in}}%
\pgfpathlineto{\pgfqpoint{2.259346in}{6.151853in}}%
\pgfpathlineto{\pgfqpoint{2.262549in}{6.156743in}}%
\pgfpathlineto{\pgfqpoint{2.264151in}{6.155960in}}%
\pgfpathlineto{\pgfqpoint{2.267354in}{6.147967in}}%
\pgfpathlineto{\pgfqpoint{2.272160in}{6.120026in}}%
\pgfpathlineto{\pgfqpoint{2.275363in}{6.090997in}}%
\pgfpathlineto{\pgfqpoint{2.278567in}{6.095823in}}%
\pgfpathlineto{\pgfqpoint{2.280169in}{6.095048in}}%
\pgfpathlineto{\pgfqpoint{2.283372in}{6.087153in}}%
\pgfpathlineto{\pgfqpoint{2.288177in}{6.059556in}}%
\pgfpathlineto{\pgfqpoint{2.291381in}{6.030899in}}%
\pgfpathlineto{\pgfqpoint{2.294585in}{6.035662in}}%
\pgfpathlineto{\pgfqpoint{2.296186in}{6.034896in}}%
\pgfpathlineto{\pgfqpoint{2.299390in}{6.027097in}}%
\pgfpathlineto{\pgfqpoint{2.304195in}{5.999841in}}%
\pgfpathlineto{\pgfqpoint{2.307399in}{5.971552in}}%
\pgfpathlineto{\pgfqpoint{2.310602in}{5.976253in}}%
\pgfpathlineto{\pgfqpoint{2.312204in}{5.975495in}}%
\pgfpathlineto{\pgfqpoint{2.315408in}{5.967791in}}%
\pgfpathlineto{\pgfqpoint{2.320213in}{5.940872in}}%
\pgfpathlineto{\pgfqpoint{2.323417in}{5.912944in}}%
\pgfpathlineto{\pgfqpoint{2.326620in}{5.917585in}}%
\pgfpathlineto{\pgfqpoint{2.328222in}{5.916835in}}%
\pgfpathlineto{\pgfqpoint{2.331425in}{5.909225in}}%
\pgfpathlineto{\pgfqpoint{2.336231in}{5.882638in}}%
\pgfpathlineto{\pgfqpoint{2.339434in}{5.855069in}}%
\pgfpathlineto{\pgfqpoint{2.342638in}{5.859649in}}%
\pgfpathlineto{\pgfqpoint{2.344240in}{5.858908in}}%
\pgfpathlineto{\pgfqpoint{2.347443in}{5.851390in}}%
\pgfpathlineto{\pgfqpoint{2.352248in}{5.825131in}}%
\pgfpathlineto{\pgfqpoint{2.355452in}{5.797915in}}%
\pgfpathlineto{\pgfqpoint{2.358656in}{5.802436in}}%
\pgfpathlineto{\pgfqpoint{2.360257in}{5.801702in}}%
\pgfpathlineto{\pgfqpoint{2.363461in}{5.794276in}}%
\pgfpathlineto{\pgfqpoint{2.368266in}{5.768342in}}%
\pgfpathlineto{\pgfqpoint{2.371470in}{5.741474in}}%
\pgfpathlineto{\pgfqpoint{2.374673in}{5.745936in}}%
\pgfpathlineto{\pgfqpoint{2.376275in}{5.745211in}}%
\pgfpathlineto{\pgfqpoint{2.379479in}{5.737875in}}%
\pgfpathlineto{\pgfqpoint{2.384284in}{5.712261in}}%
\pgfpathlineto{\pgfqpoint{2.387487in}{5.685738in}}%
\pgfpathlineto{\pgfqpoint{2.390691in}{5.690142in}}%
\pgfpathlineto{\pgfqpoint{2.392293in}{5.689425in}}%
\pgfpathlineto{\pgfqpoint{2.395496in}{5.682178in}}%
\pgfpathlineto{\pgfqpoint{2.400302in}{5.656880in}}%
\pgfpathlineto{\pgfqpoint{2.403505in}{5.630697in}}%
\pgfpathlineto{\pgfqpoint{2.406709in}{5.635044in}}%
\pgfpathlineto{\pgfqpoint{2.408311in}{5.634335in}}%
\pgfpathlineto{\pgfqpoint{2.411514in}{5.627176in}}%
\pgfpathlineto{\pgfqpoint{2.416319in}{5.602190in}}%
\pgfpathlineto{\pgfqpoint{2.419523in}{5.576343in}}%
\pgfpathlineto{\pgfqpoint{2.422726in}{5.580633in}}%
\pgfpathlineto{\pgfqpoint{2.424328in}{5.579932in}}%
\pgfpathlineto{\pgfqpoint{2.427532in}{5.572860in}}%
\pgfpathlineto{\pgfqpoint{2.432337in}{5.548183in}}%
\pgfpathlineto{\pgfqpoint{2.435541in}{5.522667in}}%
\pgfpathlineto{\pgfqpoint{2.438744in}{5.526902in}}%
\pgfpathlineto{\pgfqpoint{2.440346in}{5.526208in}}%
\pgfpathlineto{\pgfqpoint{2.443550in}{5.519222in}}%
\pgfpathlineto{\pgfqpoint{2.448355in}{5.494850in}}%
\pgfpathlineto{\pgfqpoint{2.451558in}{5.469661in}}%
\pgfpathlineto{\pgfqpoint{2.454762in}{5.473841in}}%
\pgfpathlineto{\pgfqpoint{2.456364in}{5.473154in}}%
\pgfpathlineto{\pgfqpoint{2.459567in}{5.466254in}}%
\pgfpathlineto{\pgfqpoint{2.464373in}{5.442182in}}%
\pgfpathlineto{\pgfqpoint{2.467576in}{5.417317in}}%
\pgfpathlineto{\pgfqpoint{2.470780in}{5.421442in}}%
\pgfpathlineto{\pgfqpoint{2.472381in}{5.420763in}}%
\pgfpathlineto{\pgfqpoint{2.475585in}{5.413946in}}%
\pgfpathlineto{\pgfqpoint{2.480390in}{5.390172in}}%
\pgfpathlineto{\pgfqpoint{2.483594in}{5.365625in}}%
\pgfpathlineto{\pgfqpoint{2.486797in}{5.369697in}}%
\pgfpathlineto{\pgfqpoint{2.488399in}{5.369025in}}%
\pgfpathlineto{\pgfqpoint{2.491603in}{5.362292in}}%
\pgfpathlineto{\pgfqpoint{2.496408in}{5.338811in}}%
\pgfpathlineto{\pgfqpoint{2.499612in}{5.314578in}}%
\pgfpathlineto{\pgfqpoint{2.502815in}{5.318597in}}%
\pgfpathlineto{\pgfqpoint{2.504417in}{5.317933in}}%
\pgfpathlineto{\pgfqpoint{2.507620in}{5.311281in}}%
\pgfpathlineto{\pgfqpoint{2.512426in}{5.288090in}}%
\pgfpathlineto{\pgfqpoint{2.515629in}{5.264169in}}%
\pgfpathlineto{\pgfqpoint{2.518833in}{5.268135in}}%
\pgfpathlineto{\pgfqpoint{2.520435in}{5.267478in}}%
\pgfpathlineto{\pgfqpoint{2.523638in}{5.260908in}}%
\pgfpathlineto{\pgfqpoint{2.528444in}{5.238003in}}%
\pgfpathlineto{\pgfqpoint{2.531647in}{5.214388in}}%
\pgfpathlineto{\pgfqpoint{2.534851in}{5.218303in}}%
\pgfpathlineto{\pgfqpoint{2.536452in}{5.217653in}}%
\pgfpathlineto{\pgfqpoint{2.539656in}{5.211163in}}%
\pgfpathlineto{\pgfqpoint{2.544461in}{5.188541in}}%
\pgfpathlineto{\pgfqpoint{2.547665in}{5.165228in}}%
\pgfpathlineto{\pgfqpoint{2.550868in}{5.169092in}}%
\pgfpathlineto{\pgfqpoint{2.552470in}{5.168450in}}%
\pgfpathlineto{\pgfqpoint{2.555674in}{5.162038in}}%
\pgfpathlineto{\pgfqpoint{2.560479in}{5.139696in}}%
\pgfpathlineto{\pgfqpoint{2.563683in}{5.116682in}}%
\pgfpathlineto{\pgfqpoint{2.566886in}{5.120496in}}%
\pgfpathlineto{\pgfqpoint{2.568488in}{5.119860in}}%
\pgfpathlineto{\pgfqpoint{2.571691in}{5.113527in}}%
\pgfpathlineto{\pgfqpoint{2.576497in}{5.091460in}}%
\pgfpathlineto{\pgfqpoint{2.579700in}{5.068742in}}%
\pgfpathlineto{\pgfqpoint{2.582904in}{5.072506in}}%
\pgfpathlineto{\pgfqpoint{2.584506in}{5.071877in}}%
\pgfpathlineto{\pgfqpoint{2.587709in}{5.065621in}}%
\pgfpathlineto{\pgfqpoint{2.592514in}{5.043827in}}%
\pgfpathlineto{\pgfqpoint{2.595718in}{5.021400in}}%
\pgfpathlineto{\pgfqpoint{2.598922in}{5.025115in}}%
\pgfpathlineto{\pgfqpoint{2.602125in}{5.022224in}}%
\pgfpathlineto{\pgfqpoint{2.605329in}{5.012765in}}%
\pgfpathlineto{\pgfqpoint{2.610134in}{4.986373in}}%
\pgfpathlineto{\pgfqpoint{2.611736in}{4.974648in}}%
\pgfpathlineto{\pgfqpoint{2.614939in}{4.978315in}}%
\pgfpathlineto{\pgfqpoint{2.618143in}{4.975458in}}%
\pgfpathlineto{\pgfqpoint{2.621346in}{4.966116in}}%
\pgfpathlineto{\pgfqpoint{2.626152in}{4.940050in}}%
\pgfpathlineto{\pgfqpoint{2.627754in}{4.928480in}}%
\pgfpathlineto{\pgfqpoint{2.630957in}{4.932099in}}%
\pgfpathlineto{\pgfqpoint{2.634161in}{4.929276in}}%
\pgfpathlineto{\pgfqpoint{2.637364in}{4.920048in}}%
\pgfpathlineto{\pgfqpoint{2.642169in}{4.894305in}}%
\pgfpathlineto{\pgfqpoint{2.643771in}{4.882888in}}%
\pgfpathlineto{\pgfqpoint{2.646975in}{4.886460in}}%
\pgfpathlineto{\pgfqpoint{2.650178in}{4.883671in}}%
\pgfpathlineto{\pgfqpoint{2.653382in}{4.874556in}}%
\pgfpathlineto{\pgfqpoint{2.658187in}{4.849131in}}%
\pgfpathlineto{\pgfqpoint{2.659789in}{4.837864in}}%
\pgfpathlineto{\pgfqpoint{2.662993in}{4.841390in}}%
\pgfpathlineto{\pgfqpoint{2.666196in}{4.838634in}}%
\pgfpathlineto{\pgfqpoint{2.669400in}{4.829631in}}%
\pgfpathlineto{\pgfqpoint{2.674205in}{4.804521in}}%
\pgfpathlineto{\pgfqpoint{2.675807in}{4.793403in}}%
\pgfpathlineto{\pgfqpoint{2.679010in}{4.796883in}}%
\pgfpathlineto{\pgfqpoint{2.682214in}{4.794159in}}%
\pgfpathlineto{\pgfqpoint{2.685417in}{4.785267in}}%
\pgfpathlineto{\pgfqpoint{2.690223in}{4.760467in}}%
\pgfpathlineto{\pgfqpoint{2.691824in}{4.749496in}}%
\pgfpathlineto{\pgfqpoint{2.695028in}{4.752931in}}%
\pgfpathlineto{\pgfqpoint{2.698232in}{4.750239in}}%
\pgfpathlineto{\pgfqpoint{2.701435in}{4.741456in}}%
\pgfpathlineto{\pgfqpoint{2.706240in}{4.716964in}}%
\pgfpathlineto{\pgfqpoint{2.707842in}{4.706137in}}%
\pgfpathlineto{\pgfqpoint{2.711046in}{4.709528in}}%
\pgfpathlineto{\pgfqpoint{2.714249in}{4.706868in}}%
\pgfpathlineto{\pgfqpoint{2.717453in}{4.698192in}}%
\pgfpathlineto{\pgfqpoint{2.722258in}{4.674003in}}%
\pgfpathlineto{\pgfqpoint{2.723860in}{4.663320in}}%
\pgfpathlineto{\pgfqpoint{2.727063in}{4.666666in}}%
\pgfpathlineto{\pgfqpoint{2.730267in}{4.664037in}}%
\pgfpathlineto{\pgfqpoint{2.733471in}{4.655469in}}%
\pgfpathlineto{\pgfqpoint{2.738276in}{4.631578in}}%
\pgfpathlineto{\pgfqpoint{2.739878in}{4.621036in}}%
\pgfpathlineto{\pgfqpoint{2.743081in}{4.624339in}}%
\pgfpathlineto{\pgfqpoint{2.746285in}{4.621741in}}%
\pgfpathlineto{\pgfqpoint{2.749488in}{4.613278in}}%
\pgfpathlineto{\pgfqpoint{2.754294in}{4.589683in}}%
\pgfpathlineto{\pgfqpoint{2.755895in}{4.579280in}}%
\pgfpathlineto{\pgfqpoint{2.759099in}{4.582540in}}%
\pgfpathlineto{\pgfqpoint{2.762302in}{4.579973in}}%
\pgfpathlineto{\pgfqpoint{2.765506in}{4.571613in}}%
\pgfpathlineto{\pgfqpoint{2.770311in}{4.548311in}}%
\pgfpathlineto{\pgfqpoint{2.771913in}{4.538045in}}%
\pgfpathlineto{\pgfqpoint{2.775117in}{4.541263in}}%
\pgfpathlineto{\pgfqpoint{2.778320in}{4.538726in}}%
\pgfpathlineto{\pgfqpoint{2.781524in}{4.530469in}}%
\pgfpathlineto{\pgfqpoint{2.786329in}{4.507454in}}%
\pgfpathlineto{\pgfqpoint{2.787931in}{4.497325in}}%
\pgfpathlineto{\pgfqpoint{2.791134in}{4.500500in}}%
\pgfpathlineto{\pgfqpoint{2.794338in}{4.497994in}}%
\pgfpathlineto{\pgfqpoint{2.797542in}{4.489838in}}%
\pgfpathlineto{\pgfqpoint{2.802347in}{4.467108in}}%
\pgfpathlineto{\pgfqpoint{2.803949in}{4.457112in}}%
\pgfpathlineto{\pgfqpoint{2.807152in}{4.460247in}}%
\pgfpathlineto{\pgfqpoint{2.810356in}{4.457770in}}%
\pgfpathlineto{\pgfqpoint{2.813559in}{4.449714in}}%
\pgfpathlineto{\pgfqpoint{2.818365in}{4.427266in}}%
\pgfpathlineto{\pgfqpoint{2.819966in}{4.417402in}}%
\pgfpathlineto{\pgfqpoint{2.823170in}{4.420495in}}%
\pgfpathlineto{\pgfqpoint{2.826373in}{4.418048in}}%
\pgfpathlineto{\pgfqpoint{2.829577in}{4.410091in}}%
\pgfpathlineto{\pgfqpoint{2.834382in}{4.387920in}}%
\pgfpathlineto{\pgfqpoint{2.835984in}{4.378186in}}%
\pgfpathlineto{\pgfqpoint{2.839188in}{4.381240in}}%
\pgfpathlineto{\pgfqpoint{2.842391in}{4.378821in}}%
\pgfpathlineto{\pgfqpoint{2.845595in}{4.370962in}}%
\pgfpathlineto{\pgfqpoint{2.850400in}{4.349066in}}%
\pgfpathlineto{\pgfqpoint{2.852002in}{4.339461in}}%
\pgfpathlineto{\pgfqpoint{2.855205in}{4.342474in}}%
\pgfpathlineto{\pgfqpoint{2.858409in}{4.340084in}}%
\pgfpathlineto{\pgfqpoint{2.861612in}{4.332322in}}%
\pgfpathlineto{\pgfqpoint{2.866418in}{4.310696in}}%
\pgfpathlineto{\pgfqpoint{2.868020in}{4.301218in}}%
\pgfpathlineto{\pgfqpoint{2.871223in}{4.304193in}}%
\pgfpathlineto{\pgfqpoint{2.874427in}{4.301831in}}%
\pgfpathlineto{\pgfqpoint{2.877630in}{4.294164in}}%
\pgfpathlineto{\pgfqpoint{2.882436in}{4.272806in}}%
\pgfpathlineto{\pgfqpoint{2.884037in}{4.263453in}}%
\pgfpathlineto{\pgfqpoint{2.887241in}{4.266388in}}%
\pgfpathlineto{\pgfqpoint{2.890444in}{4.264055in}}%
\pgfpathlineto{\pgfqpoint{2.893648in}{4.256482in}}%
\pgfpathlineto{\pgfqpoint{2.898453in}{4.235388in}}%
\pgfpathlineto{\pgfqpoint{2.900055in}{4.226158in}}%
\pgfpathlineto{\pgfqpoint{2.903259in}{4.229056in}}%
\pgfpathlineto{\pgfqpoint{2.906462in}{4.226750in}}%
\pgfpathlineto{\pgfqpoint{2.909666in}{4.219270in}}%
\pgfpathlineto{\pgfqpoint{2.914471in}{4.198437in}}%
\pgfpathlineto{\pgfqpoint{2.916073in}{4.189329in}}%
\pgfpathlineto{\pgfqpoint{2.919276in}{4.192190in}}%
\pgfpathlineto{\pgfqpoint{2.922480in}{4.189911in}}%
\pgfpathlineto{\pgfqpoint{2.925683in}{4.182523in}}%
\pgfpathlineto{\pgfqpoint{2.930489in}{4.161947in}}%
\pgfpathlineto{\pgfqpoint{2.932090in}{4.152960in}}%
\pgfpathlineto{\pgfqpoint{2.935294in}{4.155783in}}%
\pgfpathlineto{\pgfqpoint{2.938498in}{4.153531in}}%
\pgfpathlineto{\pgfqpoint{2.941701in}{4.146234in}}%
\pgfpathlineto{\pgfqpoint{2.946506in}{4.125913in}}%
\pgfpathlineto{\pgfqpoint{2.948108in}{4.117045in}}%
\pgfpathlineto{\pgfqpoint{2.951312in}{4.119831in}}%
\pgfpathlineto{\pgfqpoint{2.954515in}{4.117605in}}%
\pgfpathlineto{\pgfqpoint{2.957719in}{4.110398in}}%
\pgfpathlineto{\pgfqpoint{2.962524in}{4.090328in}}%
\pgfpathlineto{\pgfqpoint{2.964126in}{4.081577in}}%
\pgfpathlineto{\pgfqpoint{2.967330in}{4.084327in}}%
\pgfpathlineto{\pgfqpoint{2.970533in}{4.082128in}}%
\pgfpathlineto{\pgfqpoint{2.973737in}{4.075009in}}%
\pgfpathlineto{\pgfqpoint{2.978542in}{4.055188in}}%
\pgfpathlineto{\pgfqpoint{2.980144in}{4.046552in}}%
\pgfpathlineto{\pgfqpoint{2.983347in}{4.049267in}}%
\pgfpathlineto{\pgfqpoint{2.986551in}{4.047094in}}%
\pgfpathlineto{\pgfqpoint{2.989754in}{4.040062in}}%
\pgfpathlineto{\pgfqpoint{2.994560in}{4.020486in}}%
\pgfpathlineto{\pgfqpoint{2.996161in}{4.011964in}}%
\pgfpathlineto{\pgfqpoint{2.999365in}{4.014643in}}%
\pgfpathlineto{\pgfqpoint{3.002569in}{4.012496in}}%
\pgfpathlineto{\pgfqpoint{3.005772in}{4.005551in}}%
\pgfpathlineto{\pgfqpoint{3.010577in}{3.986217in}}%
\pgfpathlineto{\pgfqpoint{3.012179in}{3.977808in}}%
\pgfpathlineto{\pgfqpoint{3.015383in}{3.980452in}}%
\pgfpathlineto{\pgfqpoint{3.018586in}{3.978330in}}%
\pgfpathlineto{\pgfqpoint{3.021790in}{3.971470in}}%
\pgfpathlineto{\pgfqpoint{3.026595in}{3.952376in}}%
\pgfpathlineto{\pgfqpoint{3.028197in}{3.944078in}}%
\pgfpathlineto{\pgfqpoint{3.031400in}{3.946688in}}%
\pgfpathlineto{\pgfqpoint{3.034604in}{3.944591in}}%
\pgfpathlineto{\pgfqpoint{3.037808in}{3.937815in}}%
\pgfpathlineto{\pgfqpoint{3.042613in}{3.918957in}}%
\pgfpathlineto{\pgfqpoint{3.044215in}{3.910769in}}%
\pgfpathlineto{\pgfqpoint{3.047418in}{3.913345in}}%
\pgfpathlineto{\pgfqpoint{3.050622in}{3.911273in}}%
\pgfpathlineto{\pgfqpoint{3.053825in}{3.904580in}}%
\pgfpathlineto{\pgfqpoint{3.058631in}{3.885955in}}%
\pgfpathlineto{\pgfqpoint{3.060232in}{3.877875in}}%
\pgfpathlineto{\pgfqpoint{3.063436in}{3.880418in}}%
\pgfpathlineto{\pgfqpoint{3.066639in}{3.878370in}}%
\pgfpathlineto{\pgfqpoint{3.069843in}{3.871759in}}%
\pgfpathlineto{\pgfqpoint{3.074648in}{3.853365in}}%
\pgfpathlineto{\pgfqpoint{3.076250in}{3.845392in}}%
\pgfpathlineto{\pgfqpoint{3.079454in}{3.847901in}}%
\pgfpathlineto{\pgfqpoint{3.082657in}{3.845878in}}%
\pgfpathlineto{\pgfqpoint{3.085861in}{3.839349in}}%
\pgfpathlineto{\pgfqpoint{3.090666in}{3.821182in}}%
\pgfpathlineto{\pgfqpoint{3.092268in}{3.813314in}}%
\pgfpathlineto{\pgfqpoint{3.095471in}{3.815791in}}%
\pgfpathlineto{\pgfqpoint{3.098675in}{3.813792in}}%
\pgfpathlineto{\pgfqpoint{3.101878in}{3.807342in}}%
\pgfpathlineto{\pgfqpoint{3.106684in}{3.789400in}}%
\pgfpathlineto{\pgfqpoint{3.108286in}{3.781637in}}%
\pgfpathlineto{\pgfqpoint{3.111489in}{3.784081in}}%
\pgfpathlineto{\pgfqpoint{3.114693in}{3.782105in}}%
\pgfpathlineto{\pgfqpoint{3.117896in}{3.775735in}}%
\pgfpathlineto{\pgfqpoint{3.122702in}{3.758015in}}%
\pgfpathlineto{\pgfqpoint{3.124303in}{3.750354in}}%
\pgfpathlineto{\pgfqpoint{3.127507in}{3.752767in}}%
\pgfpathlineto{\pgfqpoint{3.130710in}{3.750815in}}%
\pgfpathlineto{\pgfqpoint{3.133914in}{3.744523in}}%
\pgfpathlineto{\pgfqpoint{3.138719in}{3.727022in}}%
\pgfpathlineto{\pgfqpoint{3.140321in}{3.719462in}}%
\pgfpathlineto{\pgfqpoint{3.143525in}{3.721844in}}%
\pgfpathlineto{\pgfqpoint{3.146728in}{3.719914in}}%
\pgfpathlineto{\pgfqpoint{3.149932in}{3.713700in}}%
\pgfpathlineto{\pgfqpoint{3.154737in}{3.696415in}}%
\pgfpathlineto{\pgfqpoint{3.156339in}{3.688956in}}%
\pgfpathlineto{\pgfqpoint{3.159542in}{3.691306in}}%
\pgfpathlineto{\pgfqpoint{3.162746in}{3.689400in}}%
\pgfpathlineto{\pgfqpoint{3.165949in}{3.683261in}}%
\pgfpathlineto{\pgfqpoint{3.170755in}{3.666191in}}%
\pgfpathlineto{\pgfqpoint{3.172357in}{3.658830in}}%
\pgfpathlineto{\pgfqpoint{3.175560in}{3.661150in}}%
\pgfpathlineto{\pgfqpoint{3.178764in}{3.659266in}}%
\pgfpathlineto{\pgfqpoint{3.181967in}{3.653203in}}%
\pgfpathlineto{\pgfqpoint{3.186772in}{3.636343in}}%
\pgfpathlineto{\pgfqpoint{3.188374in}{3.629080in}}%
\pgfpathlineto{\pgfqpoint{3.191578in}{3.631369in}}%
\pgfpathlineto{\pgfqpoint{3.194781in}{3.629508in}}%
\pgfpathlineto{\pgfqpoint{3.197985in}{3.623519in}}%
\pgfpathlineto{\pgfqpoint{3.202790in}{3.606869in}}%
\pgfpathlineto{\pgfqpoint{3.204392in}{3.599701in}}%
\pgfpathlineto{\pgfqpoint{3.207596in}{3.601961in}}%
\pgfpathlineto{\pgfqpoint{3.210799in}{3.600121in}}%
\pgfpathlineto{\pgfqpoint{3.214003in}{3.594206in}}%
\pgfpathlineto{\pgfqpoint{3.218808in}{3.577762in}}%
\pgfpathlineto{\pgfqpoint{3.220410in}{3.570689in}}%
\pgfpathlineto{\pgfqpoint{3.223613in}{3.572919in}}%
\pgfpathlineto{\pgfqpoint{3.226817in}{3.571101in}}%
\pgfpathlineto{\pgfqpoint{3.230020in}{3.565259in}}%
\pgfpathlineto{\pgfqpoint{3.234826in}{3.549018in}}%
\pgfpathlineto{\pgfqpoint{3.236427in}{3.542038in}}%
\pgfpathlineto{\pgfqpoint{3.239631in}{3.544240in}}%
\pgfpathlineto{\pgfqpoint{3.242835in}{3.542444in}}%
\pgfpathlineto{\pgfqpoint{3.246038in}{3.536673in}}%
\pgfpathlineto{\pgfqpoint{3.250843in}{3.520633in}}%
\pgfpathlineto{\pgfqpoint{3.252445in}{3.513746in}}%
\pgfpathlineto{\pgfqpoint{3.255649in}{3.515918in}}%
\pgfpathlineto{\pgfqpoint{3.258852in}{3.514144in}}%
\pgfpathlineto{\pgfqpoint{3.262056in}{3.508444in}}%
\pgfpathlineto{\pgfqpoint{3.266861in}{3.492602in}}%
\pgfpathlineto{\pgfqpoint{3.268463in}{3.485806in}}%
\pgfpathlineto{\pgfqpoint{3.271666in}{3.487950in}}%
\pgfpathlineto{\pgfqpoint{3.274870in}{3.486197in}}%
\pgfpathlineto{\pgfqpoint{3.278074in}{3.480567in}}%
\pgfpathlineto{\pgfqpoint{3.282879in}{3.464921in}}%
\pgfpathlineto{\pgfqpoint{3.284481in}{3.458215in}}%
\pgfpathlineto{\pgfqpoint{3.287684in}{3.460331in}}%
\pgfpathlineto{\pgfqpoint{3.290888in}{3.458598in}}%
\pgfpathlineto{\pgfqpoint{3.294091in}{3.453038in}}%
\pgfpathlineto{\pgfqpoint{3.298897in}{3.437585in}}%
\pgfpathlineto{\pgfqpoint{3.300498in}{3.430968in}}%
\pgfpathlineto{\pgfqpoint{3.303702in}{3.433057in}}%
\pgfpathlineto{\pgfqpoint{3.306906in}{3.431344in}}%
\pgfpathlineto{\pgfqpoint{3.310109in}{3.425852in}}%
\pgfpathlineto{\pgfqpoint{3.314914in}{3.410591in}}%
\pgfpathlineto{\pgfqpoint{3.316516in}{3.404061in}}%
\pgfpathlineto{\pgfqpoint{3.319720in}{3.406123in}}%
\pgfpathlineto{\pgfqpoint{3.322923in}{3.404431in}}%
\pgfpathlineto{\pgfqpoint{3.326127in}{3.399006in}}%
\pgfpathlineto{\pgfqpoint{3.330932in}{3.383933in}}%
\pgfpathlineto{\pgfqpoint{3.332534in}{3.377490in}}%
\pgfpathlineto{\pgfqpoint{3.335737in}{3.379525in}}%
\pgfpathlineto{\pgfqpoint{3.338941in}{3.377853in}}%
\pgfpathlineto{\pgfqpoint{3.342145in}{3.372494in}}%
\pgfpathlineto{\pgfqpoint{3.346950in}{3.357609in}}%
\pgfpathlineto{\pgfqpoint{3.348552in}{3.351250in}}%
\pgfpathlineto{\pgfqpoint{3.351755in}{3.353259in}}%
\pgfpathlineto{\pgfqpoint{3.354959in}{3.351607in}}%
\pgfpathlineto{\pgfqpoint{3.358162in}{3.346314in}}%
\pgfpathlineto{\pgfqpoint{3.362968in}{3.331612in}}%
\pgfpathlineto{\pgfqpoint{3.364569in}{3.325338in}}%
\pgfpathlineto{\pgfqpoint{3.367773in}{3.327321in}}%
\pgfpathlineto{\pgfqpoint{3.370976in}{3.325688in}}%
\pgfpathlineto{\pgfqpoint{3.374180in}{3.320460in}}%
\pgfpathlineto{\pgfqpoint{3.378985in}{3.305940in}}%
\pgfpathlineto{\pgfqpoint{3.380587in}{3.299749in}}%
\pgfpathlineto{\pgfqpoint{3.383791in}{3.301706in}}%
\pgfpathlineto{\pgfqpoint{3.386994in}{3.300092in}}%
\pgfpathlineto{\pgfqpoint{3.390198in}{3.294929in}}%
\pgfpathlineto{\pgfqpoint{3.395003in}{3.280589in}}%
\pgfpathlineto{\pgfqpoint{3.396605in}{3.274480in}}%
\pgfpathlineto{\pgfqpoint{3.399808in}{3.276411in}}%
\pgfpathlineto{\pgfqpoint{3.403012in}{3.274816in}}%
\pgfpathlineto{\pgfqpoint{3.406215in}{3.269717in}}%
\pgfpathlineto{\pgfqpoint{3.411021in}{3.255554in}}%
\pgfpathlineto{\pgfqpoint{3.412623in}{3.249525in}}%
\pgfpathlineto{\pgfqpoint{3.415826in}{3.251432in}}%
\pgfpathlineto{\pgfqpoint{3.419030in}{3.249856in}}%
\pgfpathlineto{\pgfqpoint{3.422233in}{3.244819in}}%
\pgfpathlineto{\pgfqpoint{3.427039in}{3.230831in}}%
\pgfpathlineto{\pgfqpoint{3.428640in}{3.224883in}}%
\pgfpathlineto{\pgfqpoint{3.431844in}{3.226764in}}%
\pgfpathlineto{\pgfqpoint{3.435047in}{3.225207in}}%
\pgfpathlineto{\pgfqpoint{3.438251in}{3.220232in}}%
\pgfpathlineto{\pgfqpoint{3.443056in}{3.206417in}}%
\pgfpathlineto{\pgfqpoint{3.444658in}{3.200547in}}%
\pgfpathlineto{\pgfqpoint{3.447862in}{3.202404in}}%
\pgfpathlineto{\pgfqpoint{3.451065in}{3.200865in}}%
\pgfpathlineto{\pgfqpoint{3.454269in}{3.195951in}}%
\pgfpathlineto{\pgfqpoint{3.459074in}{3.182308in}}%
\pgfpathlineto{\pgfqpoint{3.460676in}{3.176515in}}%
\pgfpathlineto{\pgfqpoint{3.463879in}{3.178348in}}%
\pgfpathlineto{\pgfqpoint{3.467083in}{3.176828in}}%
\pgfpathlineto{\pgfqpoint{3.470286in}{3.171974in}}%
\pgfpathlineto{\pgfqpoint{3.475092in}{3.158499in}}%
\pgfpathlineto{\pgfqpoint{3.476694in}{3.152783in}}%
\pgfpathlineto{\pgfqpoint{3.479897in}{3.154592in}}%
\pgfpathlineto{\pgfqpoint{3.483101in}{3.153090in}}%
\pgfpathlineto{\pgfqpoint{3.487906in}{3.144671in}}%
\pgfpathlineto{\pgfqpoint{3.492711in}{3.129348in}}%
\pgfpathlineto{\pgfqpoint{3.494313in}{3.130650in}}%
\pgfpathlineto{\pgfqpoint{3.497517in}{3.130798in}}%
\pgfpathlineto{\pgfqpoint{3.500720in}{3.127686in}}%
\pgfpathlineto{\pgfqpoint{3.505525in}{3.116951in}}%
\pgfpathlineto{\pgfqpoint{3.508729in}{3.106204in}}%
\pgfpathlineto{\pgfqpoint{3.511933in}{3.107966in}}%
\pgfpathlineto{\pgfqpoint{3.515136in}{3.106499in}}%
\pgfpathlineto{\pgfqpoint{3.519941in}{3.098287in}}%
\pgfpathlineto{\pgfqpoint{3.524747in}{3.083350in}}%
\pgfpathlineto{\pgfqpoint{3.526348in}{3.084619in}}%
\pgfpathlineto{\pgfqpoint{3.529552in}{3.084762in}}%
\pgfpathlineto{\pgfqpoint{3.532756in}{3.081724in}}%
\pgfpathlineto{\pgfqpoint{3.537561in}{3.071253in}}%
\pgfpathlineto{\pgfqpoint{3.540764in}{3.060780in}}%
\pgfpathlineto{\pgfqpoint{3.543968in}{3.062497in}}%
\pgfpathlineto{\pgfqpoint{3.547172in}{3.061064in}}%
\pgfpathlineto{\pgfqpoint{3.551977in}{3.053053in}}%
\pgfpathlineto{\pgfqpoint{3.556782in}{3.038492in}}%
\pgfpathlineto{\pgfqpoint{3.558384in}{3.039729in}}%
\pgfpathlineto{\pgfqpoint{3.561588in}{3.039867in}}%
\pgfpathlineto{\pgfqpoint{3.564791in}{3.036903in}}%
\pgfpathlineto{\pgfqpoint{3.569596in}{3.026689in}}%
\pgfpathlineto{\pgfqpoint{3.572800in}{3.016483in}}%
\pgfpathlineto{\pgfqpoint{3.576003in}{3.018155in}}%
\pgfpathlineto{\pgfqpoint{3.579207in}{3.016756in}}%
\pgfpathlineto{\pgfqpoint{3.584012in}{3.008941in}}%
\pgfpathlineto{\pgfqpoint{3.588818in}{2.994747in}}%
\pgfpathlineto{\pgfqpoint{3.590419in}{2.995953in}}%
\pgfpathlineto{\pgfqpoint{3.593623in}{2.996085in}}%
\pgfpathlineto{\pgfqpoint{3.596827in}{2.993193in}}%
\pgfpathlineto{\pgfqpoint{3.601632in}{2.983229in}}%
\pgfpathlineto{\pgfqpoint{3.604835in}{2.973284in}}%
\pgfpathlineto{\pgfqpoint{3.608039in}{2.974913in}}%
\pgfpathlineto{\pgfqpoint{3.611242in}{2.973547in}}%
\pgfpathlineto{\pgfqpoint{3.616048in}{2.965922in}}%
\pgfpathlineto{\pgfqpoint{3.620853in}{2.952087in}}%
\pgfpathlineto{\pgfqpoint{3.622455in}{2.953262in}}%
\pgfpathlineto{\pgfqpoint{3.625658in}{2.953389in}}%
\pgfpathlineto{\pgfqpoint{3.628862in}{2.950567in}}%
\pgfpathlineto{\pgfqpoint{3.633667in}{2.940848in}}%
\pgfpathlineto{\pgfqpoint{3.636871in}{2.931156in}}%
\pgfpathlineto{\pgfqpoint{3.640074in}{2.932743in}}%
\pgfpathlineto{\pgfqpoint{3.643278in}{2.931409in}}%
\pgfpathlineto{\pgfqpoint{3.648083in}{2.923971in}}%
\pgfpathlineto{\pgfqpoint{3.652889in}{2.910485in}}%
\pgfpathlineto{\pgfqpoint{3.654490in}{2.911630in}}%
\pgfpathlineto{\pgfqpoint{3.657694in}{2.911752in}}%
\pgfpathlineto{\pgfqpoint{3.660897in}{2.908998in}}%
\pgfpathlineto{\pgfqpoint{3.665703in}{2.899518in}}%
\pgfpathlineto{\pgfqpoint{3.668906in}{2.890073in}}%
\pgfpathlineto{\pgfqpoint{3.672110in}{2.891619in}}%
\pgfpathlineto{\pgfqpoint{3.675313in}{2.890316in}}%
\pgfpathlineto{\pgfqpoint{3.680119in}{2.883061in}}%
\pgfpathlineto{\pgfqpoint{3.684924in}{2.869915in}}%
\pgfpathlineto{\pgfqpoint{3.686526in}{2.871030in}}%
\pgfpathlineto{\pgfqpoint{3.689729in}{2.871148in}}%
\pgfpathlineto{\pgfqpoint{3.692933in}{2.868461in}}%
\pgfpathlineto{\pgfqpoint{3.697738in}{2.859213in}}%
\pgfpathlineto{\pgfqpoint{3.700942in}{2.850009in}}%
\pgfpathlineto{\pgfqpoint{3.704145in}{2.851515in}}%
\pgfpathlineto{\pgfqpoint{3.707349in}{2.850243in}}%
\pgfpathlineto{\pgfqpoint{3.712154in}{2.843165in}}%
\pgfpathlineto{\pgfqpoint{3.716960in}{2.830351in}}%
\pgfpathlineto{\pgfqpoint{3.718561in}{2.831438in}}%
\pgfpathlineto{\pgfqpoint{3.721765in}{2.831551in}}%
\pgfpathlineto{\pgfqpoint{3.724968in}{2.828929in}}%
\pgfpathlineto{\pgfqpoint{3.729774in}{2.819908in}}%
\pgfpathlineto{\pgfqpoint{3.732977in}{2.810938in}}%
\pgfpathlineto{\pgfqpoint{3.736181in}{2.812405in}}%
\pgfpathlineto{\pgfqpoint{3.739384in}{2.811164in}}%
\pgfpathlineto{\pgfqpoint{3.744190in}{2.804258in}}%
\pgfpathlineto{\pgfqpoint{3.748995in}{2.791768in}}%
\pgfpathlineto{\pgfqpoint{3.750597in}{2.792827in}}%
\pgfpathlineto{\pgfqpoint{3.753800in}{2.792936in}}%
\pgfpathlineto{\pgfqpoint{3.757004in}{2.790377in}}%
\pgfpathlineto{\pgfqpoint{3.761809in}{2.781578in}}%
\pgfpathlineto{\pgfqpoint{3.765013in}{2.772836in}}%
\pgfpathlineto{\pgfqpoint{3.768216in}{2.774265in}}%
\pgfpathlineto{\pgfqpoint{3.771420in}{2.773053in}}%
\pgfpathlineto{\pgfqpoint{3.776225in}{2.766317in}}%
\pgfpathlineto{\pgfqpoint{3.781030in}{2.754141in}}%
\pgfpathlineto{\pgfqpoint{3.782632in}{2.755174in}}%
\pgfpathlineto{\pgfqpoint{3.785836in}{2.755278in}}%
\pgfpathlineto{\pgfqpoint{3.789039in}{2.752781in}}%
\pgfpathlineto{\pgfqpoint{3.793845in}{2.744198in}}%
\pgfpathlineto{\pgfqpoint{3.797048in}{2.735680in}}%
\pgfpathlineto{\pgfqpoint{3.800252in}{2.737072in}}%
\pgfpathlineto{\pgfqpoint{3.803455in}{2.735888in}}%
\pgfpathlineto{\pgfqpoint{3.808261in}{2.729317in}}%
\pgfpathlineto{\pgfqpoint{3.813066in}{2.717448in}}%
\pgfpathlineto{\pgfqpoint{3.814668in}{2.718454in}}%
\pgfpathlineto{\pgfqpoint{3.817871in}{2.718555in}}%
\pgfpathlineto{\pgfqpoint{3.821075in}{2.716118in}}%
\pgfpathlineto{\pgfqpoint{3.825880in}{2.707746in}}%
\pgfpathlineto{\pgfqpoint{3.829084in}{2.699444in}}%
\pgfpathlineto{\pgfqpoint{3.832287in}{2.700801in}}%
\pgfpathlineto{\pgfqpoint{3.835491in}{2.699645in}}%
\pgfpathlineto{\pgfqpoint{3.840296in}{2.693234in}}%
\pgfpathlineto{\pgfqpoint{3.845101in}{2.681665in}}%
\pgfpathlineto{\pgfqpoint{3.846703in}{2.682645in}}%
\pgfpathlineto{\pgfqpoint{3.849907in}{2.682742in}}%
\pgfpathlineto{\pgfqpoint{3.853110in}{2.680364in}}%
\pgfpathlineto{\pgfqpoint{3.857916in}{2.672197in}}%
\pgfpathlineto{\pgfqpoint{3.861119in}{2.664108in}}%
\pgfpathlineto{\pgfqpoint{3.864323in}{2.665429in}}%
\pgfpathlineto{\pgfqpoint{3.867526in}{2.664300in}}%
\pgfpathlineto{\pgfqpoint{3.872332in}{2.658046in}}%
\pgfpathlineto{\pgfqpoint{3.877137in}{2.646769in}}%
\pgfpathlineto{\pgfqpoint{3.878739in}{2.647724in}}%
\pgfpathlineto{\pgfqpoint{3.881942in}{2.647817in}}%
\pgfpathlineto{\pgfqpoint{3.885146in}{2.645497in}}%
\pgfpathlineto{\pgfqpoint{3.889951in}{2.637531in}}%
\pgfpathlineto{\pgfqpoint{3.893155in}{2.629647in}}%
\pgfpathlineto{\pgfqpoint{3.896358in}{2.630934in}}%
\pgfpathlineto{\pgfqpoint{3.899562in}{2.629832in}}%
\pgfpathlineto{\pgfqpoint{3.904367in}{2.623732in}}%
\pgfpathlineto{\pgfqpoint{3.909172in}{2.612739in}}%
\pgfpathlineto{\pgfqpoint{3.910774in}{2.613670in}}%
\pgfpathlineto{\pgfqpoint{3.913978in}{2.613759in}}%
\pgfpathlineto{\pgfqpoint{3.917181in}{2.611495in}}%
\pgfpathlineto{\pgfqpoint{3.921987in}{2.603724in}}%
\pgfpathlineto{\pgfqpoint{3.925190in}{2.596042in}}%
\pgfpathlineto{\pgfqpoint{3.929995in}{2.597048in}}%
\pgfpathlineto{\pgfqpoint{3.934801in}{2.592827in}}%
\pgfpathlineto{\pgfqpoint{3.939606in}{2.583436in}}%
\pgfpathlineto{\pgfqpoint{3.941208in}{2.579553in}}%
\pgfpathlineto{\pgfqpoint{3.946013in}{2.580545in}}%
\pgfpathlineto{\pgfqpoint{3.950818in}{2.576376in}}%
\pgfpathlineto{\pgfqpoint{3.955624in}{2.567101in}}%
\pgfpathlineto{\pgfqpoint{3.957226in}{2.563269in}}%
\pgfpathlineto{\pgfqpoint{3.962031in}{2.564248in}}%
\pgfpathlineto{\pgfqpoint{3.966836in}{2.560130in}}%
\pgfpathlineto{\pgfqpoint{3.971642in}{2.550970in}}%
\pgfpathlineto{\pgfqpoint{3.973243in}{2.547189in}}%
\pgfpathlineto{\pgfqpoint{3.978049in}{2.548155in}}%
\pgfpathlineto{\pgfqpoint{3.982854in}{2.544088in}}%
\pgfpathlineto{\pgfqpoint{3.987659in}{2.535041in}}%
\pgfpathlineto{\pgfqpoint{3.989261in}{2.531310in}}%
\pgfpathlineto{\pgfqpoint{3.994066in}{2.532263in}}%
\pgfpathlineto{\pgfqpoint{3.998872in}{2.528245in}}%
\pgfpathlineto{\pgfqpoint{4.003677in}{2.519310in}}%
\pgfpathlineto{\pgfqpoint{4.005279in}{2.515628in}}%
\pgfpathlineto{\pgfqpoint{4.010084in}{2.516568in}}%
\pgfpathlineto{\pgfqpoint{4.014889in}{2.512600in}}%
\pgfpathlineto{\pgfqpoint{4.019695in}{2.503775in}}%
\pgfpathlineto{\pgfqpoint{4.021297in}{2.500143in}}%
\pgfpathlineto{\pgfqpoint{4.026102in}{2.501070in}}%
\pgfpathlineto{\pgfqpoint{4.030907in}{2.497150in}}%
\pgfpathlineto{\pgfqpoint{4.035712in}{2.488435in}}%
\pgfpathlineto{\pgfqpoint{4.037314in}{2.484850in}}%
\pgfpathlineto{\pgfqpoint{4.042120in}{2.485765in}}%
\pgfpathlineto{\pgfqpoint{4.046925in}{2.481893in}}%
\pgfpathlineto{\pgfqpoint{4.051730in}{2.473286in}}%
\pgfpathlineto{\pgfqpoint{4.053332in}{2.469749in}}%
\pgfpathlineto{\pgfqpoint{4.058137in}{2.470651in}}%
\pgfpathlineto{\pgfqpoint{4.062943in}{2.466826in}}%
\pgfpathlineto{\pgfqpoint{4.067748in}{2.458325in}}%
\pgfpathlineto{\pgfqpoint{4.069350in}{2.454835in}}%
\pgfpathlineto{\pgfqpoint{4.074155in}{2.455726in}}%
\pgfpathlineto{\pgfqpoint{4.078960in}{2.451948in}}%
\pgfpathlineto{\pgfqpoint{4.086969in}{2.440915in}}%
\pgfpathlineto{\pgfqpoint{4.091775in}{2.440253in}}%
\pgfpathlineto{\pgfqpoint{4.096580in}{2.434995in}}%
\pgfpathlineto{\pgfqpoint{4.102987in}{2.426362in}}%
\pgfpathlineto{\pgfqpoint{4.107792in}{2.425706in}}%
\pgfpathlineto{\pgfqpoint{4.112598in}{2.420513in}}%
\pgfpathlineto{\pgfqpoint{4.119005in}{2.411990in}}%
\pgfpathlineto{\pgfqpoint{4.123810in}{2.411342in}}%
\pgfpathlineto{\pgfqpoint{4.128615in}{2.406212in}}%
\pgfpathlineto{\pgfqpoint{4.135022in}{2.397797in}}%
\pgfpathlineto{\pgfqpoint{4.139828in}{2.397156in}}%
\pgfpathlineto{\pgfqpoint{4.144633in}{2.392090in}}%
\pgfpathlineto{\pgfqpoint{4.151040in}{2.383781in}}%
\pgfpathlineto{\pgfqpoint{4.155846in}{2.383147in}}%
\pgfpathlineto{\pgfqpoint{4.160651in}{2.378143in}}%
\pgfpathlineto{\pgfqpoint{4.167058in}{2.369940in}}%
\pgfpathlineto{\pgfqpoint{4.171863in}{2.369314in}}%
\pgfpathlineto{\pgfqpoint{4.176669in}{2.364371in}}%
\pgfpathlineto{\pgfqpoint{4.183076in}{2.356272in}}%
\pgfpathlineto{\pgfqpoint{4.187881in}{2.355652in}}%
\pgfpathlineto{\pgfqpoint{4.192686in}{2.350771in}}%
\pgfpathlineto{\pgfqpoint{4.199093in}{2.342775in}}%
\pgfpathlineto{\pgfqpoint{4.203899in}{2.342162in}}%
\pgfpathlineto{\pgfqpoint{4.208704in}{2.337340in}}%
\pgfpathlineto{\pgfqpoint{4.215111in}{2.329445in}}%
\pgfpathlineto{\pgfqpoint{4.219916in}{2.328839in}}%
\pgfpathlineto{\pgfqpoint{4.224722in}{2.324077in}}%
\pgfpathlineto{\pgfqpoint{4.231129in}{2.316283in}}%
\pgfpathlineto{\pgfqpoint{4.235934in}{2.315683in}}%
\pgfpathlineto{\pgfqpoint{4.240740in}{2.310980in}}%
\pgfpathlineto{\pgfqpoint{4.247147in}{2.303284in}}%
\pgfpathlineto{\pgfqpoint{4.251952in}{2.302691in}}%
\pgfpathlineto{\pgfqpoint{4.256757in}{2.298046in}}%
\pgfpathlineto{\pgfqpoint{4.263164in}{2.290447in}}%
\pgfpathlineto{\pgfqpoint{4.267970in}{2.289862in}}%
\pgfpathlineto{\pgfqpoint{4.272775in}{2.285273in}}%
\pgfpathlineto{\pgfqpoint{4.279182in}{2.277771in}}%
\pgfpathlineto{\pgfqpoint{4.283987in}{2.277192in}}%
\pgfpathlineto{\pgfqpoint{4.288793in}{2.272660in}}%
\pgfpathlineto{\pgfqpoint{4.295200in}{2.265253in}}%
\pgfpathlineto{\pgfqpoint{4.300005in}{2.264680in}}%
\pgfpathlineto{\pgfqpoint{4.304810in}{2.260204in}}%
\pgfpathlineto{\pgfqpoint{4.311218in}{2.252891in}}%
\pgfpathlineto{\pgfqpoint{4.316023in}{2.252325in}}%
\pgfpathlineto{\pgfqpoint{4.320828in}{2.247904in}}%
\pgfpathlineto{\pgfqpoint{4.327235in}{2.240684in}}%
\pgfpathlineto{\pgfqpoint{4.332041in}{2.240124in}}%
\pgfpathlineto{\pgfqpoint{4.336846in}{2.235757in}}%
\pgfpathlineto{\pgfqpoint{4.343253in}{2.228628in}}%
\pgfpathlineto{\pgfqpoint{4.348058in}{2.228074in}}%
\pgfpathlineto{\pgfqpoint{4.352864in}{2.223762in}}%
\pgfpathlineto{\pgfqpoint{4.359271in}{2.216724in}}%
\pgfpathlineto{\pgfqpoint{4.364076in}{2.216176in}}%
\pgfpathlineto{\pgfqpoint{4.368881in}{2.211916in}}%
\pgfpathlineto{\pgfqpoint{4.375288in}{2.204967in}}%
\pgfpathlineto{\pgfqpoint{4.380094in}{2.204426in}}%
\pgfpathlineto{\pgfqpoint{4.384899in}{2.200218in}}%
\pgfpathlineto{\pgfqpoint{4.391306in}{2.193358in}}%
\pgfpathlineto{\pgfqpoint{4.396112in}{2.192822in}}%
\pgfpathlineto{\pgfqpoint{4.400917in}{2.188667in}}%
\pgfpathlineto{\pgfqpoint{4.407324in}{2.181893in}}%
\pgfpathlineto{\pgfqpoint{4.412129in}{2.181363in}}%
\pgfpathlineto{\pgfqpoint{4.416935in}{2.177259in}}%
\pgfpathlineto{\pgfqpoint{4.423342in}{2.170571in}}%
\pgfpathlineto{\pgfqpoint{4.428147in}{2.170047in}}%
\pgfpathlineto{\pgfqpoint{4.432952in}{2.165994in}}%
\pgfpathlineto{\pgfqpoint{4.439359in}{2.159391in}}%
\pgfpathlineto{\pgfqpoint{4.444165in}{2.158873in}}%
\pgfpathlineto{\pgfqpoint{4.448970in}{2.154869in}}%
\pgfpathlineto{\pgfqpoint{4.455377in}{2.148350in}}%
\pgfpathlineto{\pgfqpoint{4.460182in}{2.147838in}}%
\pgfpathlineto{\pgfqpoint{4.464988in}{2.143883in}}%
\pgfpathlineto{\pgfqpoint{4.471395in}{2.137447in}}%
\pgfpathlineto{\pgfqpoint{4.476200in}{2.136940in}}%
\pgfpathlineto{\pgfqpoint{4.481006in}{2.133034in}}%
\pgfpathlineto{\pgfqpoint{4.487413in}{2.126679in}}%
\pgfpathlineto{\pgfqpoint{4.492218in}{2.126179in}}%
\pgfpathlineto{\pgfqpoint{4.497023in}{2.122321in}}%
\pgfpathlineto{\pgfqpoint{4.503430in}{2.116047in}}%
\pgfpathlineto{\pgfqpoint{4.508236in}{2.115552in}}%
\pgfpathlineto{\pgfqpoint{4.513041in}{2.111741in}}%
\pgfpathlineto{\pgfqpoint{4.519448in}{2.105547in}}%
\pgfpathlineto{\pgfqpoint{4.524253in}{2.105057in}}%
\pgfpathlineto{\pgfqpoint{4.529059in}{2.101294in}}%
\pgfpathlineto{\pgfqpoint{4.535466in}{2.095177in}}%
\pgfpathlineto{\pgfqpoint{4.540271in}{2.094693in}}%
\pgfpathlineto{\pgfqpoint{4.545076in}{2.090976in}}%
\pgfpathlineto{\pgfqpoint{4.551484in}{2.084938in}}%
\pgfpathlineto{\pgfqpoint{4.556289in}{2.084459in}}%
\pgfpathlineto{\pgfqpoint{4.561094in}{2.080788in}}%
\pgfpathlineto{\pgfqpoint{4.567501in}{2.074826in}}%
\pgfpathlineto{\pgfqpoint{4.572307in}{2.074353in}}%
\pgfpathlineto{\pgfqpoint{4.577112in}{2.070727in}}%
\pgfpathlineto{\pgfqpoint{4.583519in}{2.064840in}}%
\pgfpathlineto{\pgfqpoint{4.588324in}{2.064372in}}%
\pgfpathlineto{\pgfqpoint{4.593130in}{2.060791in}}%
\pgfpathlineto{\pgfqpoint{4.599537in}{2.054979in}}%
\pgfpathlineto{\pgfqpoint{4.604342in}{2.054516in}}%
\pgfpathlineto{\pgfqpoint{4.609147in}{2.050979in}}%
\pgfpathlineto{\pgfqpoint{4.615555in}{2.045241in}}%
\pgfpathlineto{\pgfqpoint{4.620360in}{2.044783in}}%
\pgfpathlineto{\pgfqpoint{4.625165in}{2.041289in}}%
\pgfpathlineto{\pgfqpoint{4.631572in}{2.035624in}}%
\pgfpathlineto{\pgfqpoint{4.636378in}{2.035172in}}%
\pgfpathlineto{\pgfqpoint{4.641183in}{2.031721in}}%
\pgfpathlineto{\pgfqpoint{4.647590in}{2.026128in}}%
\pgfpathlineto{\pgfqpoint{4.652395in}{2.025680in}}%
\pgfpathlineto{\pgfqpoint{4.657201in}{2.022272in}}%
\pgfpathlineto{\pgfqpoint{4.663608in}{2.016750in}}%
\pgfpathlineto{\pgfqpoint{4.668413in}{2.016307in}}%
\pgfpathlineto{\pgfqpoint{4.673218in}{2.012941in}}%
\pgfpathlineto{\pgfqpoint{4.679625in}{2.007488in}}%
\pgfpathlineto{\pgfqpoint{4.684431in}{2.007051in}}%
\pgfpathlineto{\pgfqpoint{4.689236in}{2.003726in}}%
\pgfpathlineto{\pgfqpoint{4.695643in}{1.998343in}}%
\pgfpathlineto{\pgfqpoint{4.700449in}{1.997910in}}%
\pgfpathlineto{\pgfqpoint{4.705254in}{1.994626in}}%
\pgfpathlineto{\pgfqpoint{4.711661in}{1.989311in}}%
\pgfpathlineto{\pgfqpoint{4.716466in}{1.988884in}}%
\pgfpathlineto{\pgfqpoint{4.721272in}{1.985640in}}%
\pgfpathlineto{\pgfqpoint{4.727679in}{1.980393in}}%
\pgfpathlineto{\pgfqpoint{4.732484in}{1.979970in}}%
\pgfpathlineto{\pgfqpoint{4.737289in}{1.976766in}}%
\pgfpathlineto{\pgfqpoint{4.743696in}{1.971585in}}%
\pgfpathlineto{\pgfqpoint{4.748502in}{1.971167in}}%
\pgfpathlineto{\pgfqpoint{4.753307in}{1.968003in}}%
\pgfpathlineto{\pgfqpoint{4.759714in}{1.962888in}}%
\pgfpathlineto{\pgfqpoint{4.764519in}{1.962474in}}%
\pgfpathlineto{\pgfqpoint{4.769325in}{1.959349in}}%
\pgfpathlineto{\pgfqpoint{4.775732in}{1.954299in}}%
\pgfpathlineto{\pgfqpoint{4.780537in}{1.953890in}}%
\pgfpathlineto{\pgfqpoint{4.785343in}{1.950803in}}%
\pgfpathlineto{\pgfqpoint{4.791750in}{1.945817in}}%
\pgfpathlineto{\pgfqpoint{4.796555in}{1.945413in}}%
\pgfpathlineto{\pgfqpoint{4.801360in}{1.942364in}}%
\pgfpathlineto{\pgfqpoint{4.807767in}{1.937441in}}%
\pgfpathlineto{\pgfqpoint{4.812573in}{1.937041in}}%
\pgfpathlineto{\pgfqpoint{4.817378in}{1.934030in}}%
\pgfpathlineto{\pgfqpoint{4.823785in}{1.929169in}}%
\pgfpathlineto{\pgfqpoint{4.828590in}{1.928774in}}%
\pgfpathlineto{\pgfqpoint{4.833396in}{1.925800in}}%
\pgfpathlineto{\pgfqpoint{4.839803in}{1.921001in}}%
\pgfpathlineto{\pgfqpoint{4.844608in}{1.920610in}}%
\pgfpathlineto{\pgfqpoint{4.849413in}{1.917673in}}%
\pgfpathlineto{\pgfqpoint{4.855821in}{1.912935in}}%
\pgfpathlineto{\pgfqpoint{4.860626in}{1.912548in}}%
\pgfpathlineto{\pgfqpoint{4.867033in}{1.908126in}}%
\pgfpathlineto{\pgfqpoint{4.871838in}{1.904969in}}%
\pgfpathlineto{\pgfqpoint{4.876644in}{1.904587in}}%
\pgfpathlineto{\pgfqpoint{4.883051in}{1.900219in}}%
\pgfpathlineto{\pgfqpoint{4.887856in}{1.897103in}}%
\pgfpathlineto{\pgfqpoint{4.892661in}{1.896725in}}%
\pgfpathlineto{\pgfqpoint{4.899068in}{1.892411in}}%
\pgfpathlineto{\pgfqpoint{4.903874in}{1.889335in}}%
\pgfpathlineto{\pgfqpoint{4.908679in}{1.888961in}}%
\pgfpathlineto{\pgfqpoint{4.915086in}{1.884700in}}%
\pgfpathlineto{\pgfqpoint{4.919892in}{1.881663in}}%
\pgfpathlineto{\pgfqpoint{4.924697in}{1.881294in}}%
\pgfpathlineto{\pgfqpoint{4.931104in}{1.877086in}}%
\pgfpathlineto{\pgfqpoint{4.935909in}{1.874088in}}%
\pgfpathlineto{\pgfqpoint{4.940715in}{1.873723in}}%
\pgfpathlineto{\pgfqpoint{4.947122in}{1.869566in}}%
\pgfpathlineto{\pgfqpoint{4.951927in}{1.866607in}}%
\pgfpathlineto{\pgfqpoint{4.956732in}{1.866246in}}%
\pgfpathlineto{\pgfqpoint{4.963139in}{1.862141in}}%
\pgfpathlineto{\pgfqpoint{4.967945in}{1.859219in}}%
\pgfpathlineto{\pgfqpoint{4.974352in}{1.858230in}}%
\pgfpathlineto{\pgfqpoint{4.980759in}{1.853162in}}%
\pgfpathlineto{\pgfqpoint{4.983962in}{1.851924in}}%
\pgfpathlineto{\pgfqpoint{4.990370in}{1.850946in}}%
\pgfpathlineto{\pgfqpoint{4.996777in}{1.845941in}}%
\pgfpathlineto{\pgfqpoint{4.999980in}{1.844719in}}%
\pgfpathlineto{\pgfqpoint{5.006387in}{1.843754in}}%
\pgfpathlineto{\pgfqpoint{5.012794in}{1.838810in}}%
\pgfpathlineto{\pgfqpoint{5.015998in}{1.837605in}}%
\pgfpathlineto{\pgfqpoint{5.022405in}{1.836651in}}%
\pgfpathlineto{\pgfqpoint{5.028812in}{1.831768in}}%
\pgfpathlineto{\pgfqpoint{5.032016in}{1.830579in}}%
\pgfpathlineto{\pgfqpoint{5.038423in}{1.829636in}}%
\pgfpathlineto{\pgfqpoint{5.044830in}{1.824814in}}%
\pgfpathlineto{\pgfqpoint{5.048033in}{1.823641in}}%
\pgfpathlineto{\pgfqpoint{5.054440in}{1.822709in}}%
\pgfpathlineto{\pgfqpoint{5.060848in}{1.817947in}}%
\pgfpathlineto{\pgfqpoint{5.064051in}{1.816790in}}%
\pgfpathlineto{\pgfqpoint{5.070458in}{1.815869in}}%
\pgfpathlineto{\pgfqpoint{5.076865in}{1.811165in}}%
\pgfpathlineto{\pgfqpoint{5.080069in}{1.810023in}}%
\pgfpathlineto{\pgfqpoint{5.086476in}{1.809114in}}%
\pgfpathlineto{\pgfqpoint{5.092883in}{1.804468in}}%
\pgfpathlineto{\pgfqpoint{5.096087in}{1.803342in}}%
\pgfpathlineto{\pgfqpoint{5.102494in}{1.802443in}}%
\pgfpathlineto{\pgfqpoint{5.108901in}{1.797855in}}%
\pgfpathlineto{\pgfqpoint{5.112104in}{1.796744in}}%
\pgfpathlineto{\pgfqpoint{5.118511in}{1.795855in}}%
\pgfpathlineto{\pgfqpoint{5.124919in}{1.791324in}}%
\pgfpathlineto{\pgfqpoint{5.128122in}{1.790228in}}%
\pgfpathlineto{\pgfqpoint{5.134529in}{1.789350in}}%
\pgfpathlineto{\pgfqpoint{5.140936in}{1.784875in}}%
\pgfpathlineto{\pgfqpoint{5.144140in}{1.783793in}}%
\pgfpathlineto{\pgfqpoint{5.150547in}{1.782926in}}%
\pgfpathlineto{\pgfqpoint{5.156954in}{1.778506in}}%
\pgfpathlineto{\pgfqpoint{5.160158in}{1.777439in}}%
\pgfpathlineto{\pgfqpoint{5.166565in}{1.776582in}}%
\pgfpathlineto{\pgfqpoint{5.172972in}{1.772217in}}%
\pgfpathlineto{\pgfqpoint{5.176175in}{1.771164in}}%
\pgfpathlineto{\pgfqpoint{5.182582in}{1.770317in}}%
\pgfpathlineto{\pgfqpoint{5.188989in}{1.766006in}}%
\pgfpathlineto{\pgfqpoint{5.192193in}{1.764967in}}%
\pgfpathlineto{\pgfqpoint{5.198600in}{1.764130in}}%
\pgfpathlineto{\pgfqpoint{5.205007in}{1.759872in}}%
\pgfpathlineto{\pgfqpoint{5.208211in}{1.758848in}}%
\pgfpathlineto{\pgfqpoint{5.214618in}{1.758021in}}%
\pgfpathlineto{\pgfqpoint{5.221025in}{1.753816in}}%
\pgfpathlineto{\pgfqpoint{5.224228in}{1.752805in}}%
\pgfpathlineto{\pgfqpoint{5.230636in}{1.751988in}}%
\pgfpathlineto{\pgfqpoint{5.237043in}{1.747834in}}%
\pgfpathlineto{\pgfqpoint{5.240246in}{1.746837in}}%
\pgfpathlineto{\pgfqpoint{5.246653in}{1.746030in}}%
\pgfpathlineto{\pgfqpoint{5.253060in}{1.741928in}}%
\pgfpathlineto{\pgfqpoint{5.256264in}{1.740944in}}%
\pgfpathlineto{\pgfqpoint{5.262671in}{1.740146in}}%
\pgfpathlineto{\pgfqpoint{5.269078in}{1.736095in}}%
\pgfpathlineto{\pgfqpoint{5.272282in}{1.735124in}}%
\pgfpathlineto{\pgfqpoint{5.278689in}{1.734336in}}%
\pgfpathlineto{\pgfqpoint{5.285096in}{1.730335in}}%
\pgfpathlineto{\pgfqpoint{5.288299in}{1.729377in}}%
\pgfpathlineto{\pgfqpoint{5.294707in}{1.728598in}}%
\pgfpathlineto{\pgfqpoint{5.301114in}{1.724647in}}%
\pgfpathlineto{\pgfqpoint{5.304317in}{1.723702in}}%
\pgfpathlineto{\pgfqpoint{5.310724in}{1.722932in}}%
\pgfpathlineto{\pgfqpoint{5.317131in}{1.719029in}}%
\pgfpathlineto{\pgfqpoint{5.320335in}{1.718097in}}%
\pgfpathlineto{\pgfqpoint{5.326742in}{1.717337in}}%
\pgfpathlineto{\pgfqpoint{5.333149in}{1.713482in}}%
\pgfpathlineto{\pgfqpoint{5.336353in}{1.712563in}}%
\pgfpathlineto{\pgfqpoint{5.342760in}{1.711811in}}%
\pgfpathlineto{\pgfqpoint{5.369990in}{1.701798in}}%
\pgfpathlineto{\pgfqpoint{5.376397in}{1.700316in}}%
\pgfpathlineto{\pgfqpoint{5.386008in}{1.696467in}}%
\pgfpathlineto{\pgfqpoint{5.392415in}{1.695002in}}%
\pgfpathlineto{\pgfqpoint{5.402025in}{1.691202in}}%
\pgfpathlineto{\pgfqpoint{5.408432in}{1.689755in}}%
\pgfpathlineto{\pgfqpoint{5.418043in}{1.686003in}}%
\pgfpathlineto{\pgfqpoint{5.424450in}{1.684574in}}%
\pgfpathlineto{\pgfqpoint{5.434061in}{1.680869in}}%
\pgfpathlineto{\pgfqpoint{5.440468in}{1.679457in}}%
\pgfpathlineto{\pgfqpoint{5.450079in}{1.675799in}}%
\pgfpathlineto{\pgfqpoint{5.456486in}{1.674404in}}%
\pgfpathlineto{\pgfqpoint{5.466096in}{1.670792in}}%
\pgfpathlineto{\pgfqpoint{5.472503in}{1.669414in}}%
\pgfpathlineto{\pgfqpoint{5.482114in}{1.665848in}}%
\pgfpathlineto{\pgfqpoint{5.488521in}{1.664487in}}%
\pgfpathlineto{\pgfqpoint{5.498132in}{1.660965in}}%
\pgfpathlineto{\pgfqpoint{5.504539in}{1.659621in}}%
\pgfpathlineto{\pgfqpoint{5.514150in}{1.656143in}}%
\pgfpathlineto{\pgfqpoint{5.520557in}{1.654815in}}%
\pgfpathlineto{\pgfqpoint{5.530167in}{1.651381in}}%
\pgfpathlineto{\pgfqpoint{5.536574in}{1.650070in}}%
\pgfpathlineto{\pgfqpoint{5.546185in}{1.646679in}}%
\pgfpathlineto{\pgfqpoint{5.552592in}{1.645384in}}%
\pgfpathlineto{\pgfqpoint{5.563804in}{1.641957in}}%
\pgfpathlineto{\pgfqpoint{5.570212in}{1.640035in}}%
\pgfpathlineto{\pgfqpoint{5.576619in}{1.637368in}}%
\pgfpathlineto{\pgfqpoint{5.583026in}{1.636740in}}%
\pgfpathlineto{\pgfqpoint{5.611858in}{1.628374in}}%
\pgfpathlineto{\pgfqpoint{5.619867in}{1.625675in}}%
\pgfpathlineto{\pgfqpoint{5.624672in}{1.623955in}}%
\pgfpathlineto{\pgfqpoint{5.632681in}{1.622815in}}%
\pgfpathlineto{\pgfqpoint{5.643893in}{1.619599in}}%
\pgfpathlineto{\pgfqpoint{5.651902in}{1.616966in}}%
\pgfpathlineto{\pgfqpoint{5.656707in}{1.615290in}}%
\pgfpathlineto{\pgfqpoint{5.664716in}{1.614177in}}%
\pgfpathlineto{\pgfqpoint{5.675929in}{1.611041in}}%
\pgfpathlineto{\pgfqpoint{5.683938in}{1.608473in}}%
\pgfpathlineto{\pgfqpoint{5.688743in}{1.606839in}}%
\pgfpathlineto{\pgfqpoint{5.696752in}{1.605754in}}%
\pgfpathlineto{\pgfqpoint{5.707964in}{1.602696in}}%
\pgfpathlineto{\pgfqpoint{5.715973in}{1.600190in}}%
\pgfpathlineto{\pgfqpoint{5.720778in}{1.598598in}}%
\pgfpathlineto{\pgfqpoint{5.728787in}{1.597539in}}%
\pgfpathlineto{\pgfqpoint{5.740000in}{1.594557in}}%
\pgfpathlineto{\pgfqpoint{5.748008in}{1.592113in}}%
\pgfpathlineto{\pgfqpoint{5.752814in}{1.590562in}}%
\pgfpathlineto{\pgfqpoint{5.760823in}{1.589528in}}%
\pgfpathlineto{\pgfqpoint{5.772035in}{1.586621in}}%
\pgfpathlineto{\pgfqpoint{5.780044in}{1.584236in}}%
\pgfpathlineto{\pgfqpoint{5.784849in}{1.582725in}}%
\pgfpathlineto{\pgfqpoint{5.792858in}{1.581716in}}%
\pgfpathlineto{\pgfqpoint{5.804071in}{1.578881in}}%
\pgfpathlineto{\pgfqpoint{5.812079in}{1.576555in}}%
\pgfpathlineto{\pgfqpoint{5.816885in}{1.575082in}}%
\pgfpathlineto{\pgfqpoint{5.824894in}{1.574097in}}%
\pgfpathlineto{\pgfqpoint{5.836106in}{1.571333in}}%
\pgfpathlineto{\pgfqpoint{5.844115in}{1.569064in}}%
\pgfpathlineto{\pgfqpoint{5.848920in}{1.567629in}}%
\pgfpathlineto{\pgfqpoint{5.856929in}{1.566667in}}%
\pgfpathlineto{\pgfqpoint{5.868141in}{1.563973in}}%
\pgfpathlineto{\pgfqpoint{5.876150in}{1.561759in}}%
\pgfpathlineto{\pgfqpoint{5.880956in}{1.560360in}}%
\pgfpathlineto{\pgfqpoint{5.888965in}{1.559422in}}%
\pgfpathlineto{\pgfqpoint{5.900177in}{1.556794in}}%
\pgfpathlineto{\pgfqpoint{5.908186in}{1.554635in}}%
\pgfpathlineto{\pgfqpoint{5.912991in}{1.553272in}}%
\pgfpathlineto{\pgfqpoint{5.921000in}{1.552356in}}%
\pgfpathlineto{\pgfqpoint{5.932212in}{1.549794in}}%
\pgfpathlineto{\pgfqpoint{5.940221in}{1.547688in}}%
\pgfpathlineto{\pgfqpoint{5.945027in}{1.546359in}}%
\pgfpathlineto{\pgfqpoint{5.953035in}{1.545466in}}%
\pgfpathlineto{\pgfqpoint{5.965850in}{1.542791in}}%
\pgfpathlineto{\pgfqpoint{5.973859in}{1.540153in}}%
\pgfpathlineto{\pgfqpoint{5.978664in}{1.539677in}}%
\pgfpathlineto{\pgfqpoint{5.986673in}{1.538225in}}%
\pgfpathlineto{\pgfqpoint{5.994682in}{1.536369in}}%
\pgfpathlineto{\pgfqpoint{6.002690in}{1.534934in}}%
\pgfpathlineto{\pgfqpoint{6.010699in}{1.533101in}}%
\pgfpathlineto{\pgfqpoint{6.018708in}{1.531685in}}%
\pgfpathlineto{\pgfqpoint{6.026717in}{1.529875in}}%
\pgfpathlineto{\pgfqpoint{6.034726in}{1.528476in}}%
\pgfpathlineto{\pgfqpoint{6.042735in}{1.526689in}}%
\pgfpathlineto{\pgfqpoint{6.050744in}{1.525307in}}%
\pgfpathlineto{\pgfqpoint{6.058753in}{1.523543in}}%
\pgfpathlineto{\pgfqpoint{6.066761in}{1.522178in}}%
\pgfpathlineto{\pgfqpoint{6.074770in}{1.520436in}}%
\pgfpathlineto{\pgfqpoint{6.082779in}{1.519087in}}%
\pgfpathlineto{\pgfqpoint{6.090788in}{1.517368in}}%
\pgfpathlineto{\pgfqpoint{6.098797in}{1.516036in}}%
\pgfpathlineto{\pgfqpoint{6.106806in}{1.514338in}}%
\pgfpathlineto{\pgfqpoint{6.114815in}{1.513022in}}%
\pgfpathlineto{\pgfqpoint{6.122823in}{1.511346in}}%
\pgfpathlineto{\pgfqpoint{6.130832in}{1.510046in}}%
\pgfpathlineto{\pgfqpoint{6.138841in}{1.508391in}}%
\pgfpathlineto{\pgfqpoint{6.148452in}{1.506545in}}%
\pgfpathlineto{\pgfqpoint{6.154859in}{1.505473in}}%
\pgfpathlineto{\pgfqpoint{6.164470in}{1.503650in}}%
\pgfpathlineto{\pgfqpoint{6.170877in}{1.502591in}}%
\pgfpathlineto{\pgfqpoint{6.180487in}{1.500791in}}%
\pgfpathlineto{\pgfqpoint{6.186894in}{1.499746in}}%
\pgfpathlineto{\pgfqpoint{6.196505in}{1.497967in}}%
\pgfpathlineto{\pgfqpoint{6.202912in}{1.496936in}}%
\pgfpathlineto{\pgfqpoint{6.212523in}{1.495179in}}%
\pgfpathlineto{\pgfqpoint{6.218930in}{1.494161in}}%
\pgfpathlineto{\pgfqpoint{6.228541in}{1.492426in}}%
\pgfpathlineto{\pgfqpoint{6.234948in}{1.491420in}}%
\pgfpathlineto{\pgfqpoint{6.244558in}{1.489707in}}%
\pgfpathlineto{\pgfqpoint{6.250965in}{1.488714in}}%
\pgfpathlineto{\pgfqpoint{6.260576in}{1.487022in}}%
\pgfpathlineto{\pgfqpoint{6.266983in}{1.486042in}}%
\pgfpathlineto{\pgfqpoint{6.276594in}{1.484370in}}%
\pgfpathlineto{\pgfqpoint{6.283001in}{1.483403in}}%
\pgfpathlineto{\pgfqpoint{6.292611in}{1.481751in}}%
\pgfpathlineto{\pgfqpoint{6.299019in}{1.480796in}}%
\pgfpathlineto{\pgfqpoint{6.308629in}{1.479166in}}%
\pgfpathlineto{\pgfqpoint{6.315036in}{1.478223in}}%
\pgfpathlineto{\pgfqpoint{6.324647in}{1.476612in}}%
\pgfpathlineto{\pgfqpoint{6.331054in}{1.475681in}}%
\pgfpathlineto{\pgfqpoint{6.340665in}{1.474090in}}%
\pgfpathlineto{\pgfqpoint{6.347072in}{1.473172in}}%
\pgfpathlineto{\pgfqpoint{6.356682in}{1.471600in}}%
\pgfpathlineto{\pgfqpoint{6.363090in}{1.470693in}}%
\pgfpathlineto{\pgfqpoint{6.372700in}{1.469141in}}%
\pgfpathlineto{\pgfqpoint{6.379107in}{1.468245in}}%
\pgfpathlineto{\pgfqpoint{6.388718in}{1.466712in}}%
\pgfpathlineto{\pgfqpoint{6.395125in}{1.465828in}}%
\pgfpathlineto{\pgfqpoint{6.404736in}{1.464314in}}%
\pgfpathlineto{\pgfqpoint{6.411143in}{1.463442in}}%
\pgfpathlineto{\pgfqpoint{6.420753in}{1.461946in}}%
\pgfpathlineto{\pgfqpoint{6.427160in}{1.461085in}}%
\pgfpathlineto{\pgfqpoint{6.436771in}{1.459607in}}%
\pgfpathlineto{\pgfqpoint{6.443178in}{1.458757in}}%
\pgfpathlineto{\pgfqpoint{6.452789in}{1.457298in}}%
\pgfpathlineto{\pgfqpoint{6.459196in}{1.456458in}}%
\pgfpathlineto{\pgfqpoint{6.468807in}{1.455017in}}%
\pgfpathlineto{\pgfqpoint{6.475214in}{1.454188in}}%
\pgfpathlineto{\pgfqpoint{6.484824in}{1.452765in}}%
\pgfpathlineto{\pgfqpoint{6.491231in}{1.451947in}}%
\pgfpathlineto{\pgfqpoint{6.500842in}{1.450541in}}%
\pgfpathlineto{\pgfqpoint{6.507249in}{1.449733in}}%
\pgfpathlineto{\pgfqpoint{6.516860in}{1.448344in}}%
\pgfpathlineto{\pgfqpoint{6.523267in}{1.447547in}}%
\pgfpathlineto{\pgfqpoint{6.532878in}{1.446176in}}%
\pgfpathlineto{\pgfqpoint{6.539285in}{1.445388in}}%
\pgfpathlineto{\pgfqpoint{6.548895in}{1.444034in}}%
\pgfpathlineto{\pgfqpoint{6.556904in}{1.443218in}}%
\pgfpathlineto{\pgfqpoint{6.579329in}{1.440231in}}%
\pgfpathlineto{\pgfqpoint{6.590541in}{1.438925in}}%
\pgfpathlineto{\pgfqpoint{6.662621in}{1.429326in}}%
\pgfpathlineto{\pgfqpoint{6.670630in}{1.428921in}}%
\pgfpathlineto{\pgfqpoint{6.758728in}{1.418092in}}%
\pgfpathlineto{\pgfqpoint{6.766736in}{1.417718in}}%
\pgfpathlineto{\pgfqpoint{6.886869in}{1.404371in}}%
\pgfpathlineto{\pgfqpoint{6.896480in}{1.403890in}}%
\pgfpathlineto{\pgfqpoint{6.918905in}{1.401151in}}%
\pgfpathlineto{\pgfqpoint{6.918905in}{1.401151in}}%
\pgfusepath{stroke}%
\end{pgfscope}%
\begin{pgfscope}%
\pgfpathrectangle{\pgfqpoint{1.875000in}{1.000000in}}{\pgfqpoint{5.284091in}{6.040000in}}%
\pgfusepath{clip}%
\pgfsetrectcap%
\pgfsetroundjoin%
\pgfsetlinewidth{1.505625pt}%
\definecolor{currentstroke}{rgb}{0.090196,0.745098,0.811765}%
\pgfsetstrokecolor{currentstroke}%
\pgfsetdash{}{0pt}%
\pgfpathmoveto{\pgfqpoint{2.115186in}{6.754244in}}%
\pgfpathlineto{\pgfqpoint{2.116788in}{6.726304in}}%
\pgfpathlineto{\pgfqpoint{2.118389in}{6.764831in}}%
\pgfpathlineto{\pgfqpoint{2.119991in}{6.741469in}}%
\pgfpathlineto{\pgfqpoint{2.121593in}{6.765455in}}%
\pgfpathlineto{\pgfqpoint{2.123195in}{6.746932in}}%
\pgfpathlineto{\pgfqpoint{2.124797in}{6.756868in}}%
\pgfpathlineto{\pgfqpoint{2.129602in}{6.726756in}}%
\pgfpathlineto{\pgfqpoint{2.131204in}{6.713989in}}%
\pgfpathlineto{\pgfqpoint{2.136009in}{6.735854in}}%
\pgfpathlineto{\pgfqpoint{2.137611in}{6.732248in}}%
\pgfpathlineto{\pgfqpoint{2.139213in}{6.736966in}}%
\pgfpathlineto{\pgfqpoint{2.140814in}{6.728025in}}%
\pgfpathlineto{\pgfqpoint{2.142416in}{6.727983in}}%
\pgfpathlineto{\pgfqpoint{2.147221in}{6.692281in}}%
\pgfpathlineto{\pgfqpoint{2.150425in}{6.707803in}}%
\pgfpathlineto{\pgfqpoint{2.152027in}{6.714050in}}%
\pgfpathlineto{\pgfqpoint{2.155230in}{6.714289in}}%
\pgfpathlineto{\pgfqpoint{2.158434in}{6.705078in}}%
\pgfpathlineto{\pgfqpoint{2.163239in}{6.673893in}}%
\pgfpathlineto{\pgfqpoint{2.168044in}{6.692513in}}%
\pgfpathlineto{\pgfqpoint{2.169646in}{6.694405in}}%
\pgfpathlineto{\pgfqpoint{2.172850in}{6.690336in}}%
\pgfpathlineto{\pgfqpoint{2.176053in}{6.676751in}}%
\pgfpathlineto{\pgfqpoint{2.179257in}{6.653713in}}%
\pgfpathlineto{\pgfqpoint{2.184062in}{6.672380in}}%
\pgfpathlineto{\pgfqpoint{2.185664in}{6.674002in}}%
\pgfpathlineto{\pgfqpoint{2.187266in}{6.673077in}}%
\pgfpathlineto{\pgfqpoint{2.190469in}{6.664303in}}%
\pgfpathlineto{\pgfqpoint{2.195275in}{6.633552in}}%
\pgfpathlineto{\pgfqpoint{2.200080in}{6.652241in}}%
\pgfpathlineto{\pgfqpoint{2.201682in}{6.653785in}}%
\pgfpathlineto{\pgfqpoint{2.203283in}{6.652921in}}%
\pgfpathlineto{\pgfqpoint{2.206487in}{6.644167in}}%
\pgfpathlineto{\pgfqpoint{2.211292in}{6.613508in}}%
\pgfpathlineto{\pgfqpoint{2.216098in}{6.632156in}}%
\pgfpathlineto{\pgfqpoint{2.217699in}{6.633661in}}%
\pgfpathlineto{\pgfqpoint{2.219301in}{6.632828in}}%
\pgfpathlineto{\pgfqpoint{2.222505in}{6.624101in}}%
\pgfpathlineto{\pgfqpoint{2.227310in}{6.593543in}}%
\pgfpathlineto{\pgfqpoint{2.232115in}{6.612133in}}%
\pgfpathlineto{\pgfqpoint{2.233717in}{6.613625in}}%
\pgfpathlineto{\pgfqpoint{2.235319in}{6.612796in}}%
\pgfpathlineto{\pgfqpoint{2.238523in}{6.604095in}}%
\pgfpathlineto{\pgfqpoint{2.243328in}{6.573672in}}%
\pgfpathlineto{\pgfqpoint{2.248133in}{6.592175in}}%
\pgfpathlineto{\pgfqpoint{2.249735in}{6.593673in}}%
\pgfpathlineto{\pgfqpoint{2.251337in}{6.592833in}}%
\pgfpathlineto{\pgfqpoint{2.254540in}{6.584163in}}%
\pgfpathlineto{\pgfqpoint{2.259346in}{6.553871in}}%
\pgfpathlineto{\pgfqpoint{2.264151in}{6.572297in}}%
\pgfpathlineto{\pgfqpoint{2.265753in}{6.573788in}}%
\pgfpathlineto{\pgfqpoint{2.267354in}{6.572951in}}%
\pgfpathlineto{\pgfqpoint{2.270558in}{6.564311in}}%
\pgfpathlineto{\pgfqpoint{2.275363in}{6.534140in}}%
\pgfpathlineto{\pgfqpoint{2.280169in}{6.552495in}}%
\pgfpathlineto{\pgfqpoint{2.281770in}{6.553977in}}%
\pgfpathlineto{\pgfqpoint{2.283372in}{6.553143in}}%
\pgfpathlineto{\pgfqpoint{2.286576in}{6.544533in}}%
\pgfpathlineto{\pgfqpoint{2.291381in}{6.514483in}}%
\pgfpathlineto{\pgfqpoint{2.296186in}{6.532766in}}%
\pgfpathlineto{\pgfqpoint{2.297788in}{6.534242in}}%
\pgfpathlineto{\pgfqpoint{2.299390in}{6.533409in}}%
\pgfpathlineto{\pgfqpoint{2.302593in}{6.524828in}}%
\pgfpathlineto{\pgfqpoint{2.307399in}{6.494901in}}%
\pgfpathlineto{\pgfqpoint{2.312204in}{6.513111in}}%
\pgfpathlineto{\pgfqpoint{2.313806in}{6.514580in}}%
\pgfpathlineto{\pgfqpoint{2.315408in}{6.513748in}}%
\pgfpathlineto{\pgfqpoint{2.318611in}{6.505196in}}%
\pgfpathlineto{\pgfqpoint{2.323417in}{6.475392in}}%
\pgfpathlineto{\pgfqpoint{2.328222in}{6.493529in}}%
\pgfpathlineto{\pgfqpoint{2.329824in}{6.494991in}}%
\pgfpathlineto{\pgfqpoint{2.331425in}{6.494161in}}%
\pgfpathlineto{\pgfqpoint{2.334629in}{6.485638in}}%
\pgfpathlineto{\pgfqpoint{2.339434in}{6.455955in}}%
\pgfpathlineto{\pgfqpoint{2.344240in}{6.474021in}}%
\pgfpathlineto{\pgfqpoint{2.345841in}{6.475475in}}%
\pgfpathlineto{\pgfqpoint{2.347443in}{6.474648in}}%
\pgfpathlineto{\pgfqpoint{2.350647in}{6.466154in}}%
\pgfpathlineto{\pgfqpoint{2.355452in}{6.436592in}}%
\pgfpathlineto{\pgfqpoint{2.360257in}{6.454585in}}%
\pgfpathlineto{\pgfqpoint{2.361859in}{6.456033in}}%
\pgfpathlineto{\pgfqpoint{2.363461in}{6.455207in}}%
\pgfpathlineto{\pgfqpoint{2.366664in}{6.446742in}}%
\pgfpathlineto{\pgfqpoint{2.371470in}{6.417300in}}%
\pgfpathlineto{\pgfqpoint{2.376275in}{6.435222in}}%
\pgfpathlineto{\pgfqpoint{2.377877in}{6.436663in}}%
\pgfpathlineto{\pgfqpoint{2.379479in}{6.435839in}}%
\pgfpathlineto{\pgfqpoint{2.382682in}{6.427403in}}%
\pgfpathlineto{\pgfqpoint{2.387487in}{6.398081in}}%
\pgfpathlineto{\pgfqpoint{2.392293in}{6.415932in}}%
\pgfpathlineto{\pgfqpoint{2.393895in}{6.417366in}}%
\pgfpathlineto{\pgfqpoint{2.395496in}{6.416543in}}%
\pgfpathlineto{\pgfqpoint{2.398700in}{6.408136in}}%
\pgfpathlineto{\pgfqpoint{2.403505in}{6.378934in}}%
\pgfpathlineto{\pgfqpoint{2.408311in}{6.396713in}}%
\pgfpathlineto{\pgfqpoint{2.409912in}{6.398141in}}%
\pgfpathlineto{\pgfqpoint{2.411514in}{6.397320in}}%
\pgfpathlineto{\pgfqpoint{2.414718in}{6.388941in}}%
\pgfpathlineto{\pgfqpoint{2.419523in}{6.359858in}}%
\pgfpathlineto{\pgfqpoint{2.424328in}{6.377567in}}%
\pgfpathlineto{\pgfqpoint{2.425930in}{6.378987in}}%
\pgfpathlineto{\pgfqpoint{2.427532in}{6.378168in}}%
\pgfpathlineto{\pgfqpoint{2.430735in}{6.369817in}}%
\pgfpathlineto{\pgfqpoint{2.435541in}{6.340853in}}%
\pgfpathlineto{\pgfqpoint{2.440346in}{6.358492in}}%
\pgfpathlineto{\pgfqpoint{2.441948in}{6.359906in}}%
\pgfpathlineto{\pgfqpoint{2.443550in}{6.359088in}}%
\pgfpathlineto{\pgfqpoint{2.446753in}{6.350766in}}%
\pgfpathlineto{\pgfqpoint{2.451558in}{6.321919in}}%
\pgfpathlineto{\pgfqpoint{2.456364in}{6.339488in}}%
\pgfpathlineto{\pgfqpoint{2.457965in}{6.340895in}}%
\pgfpathlineto{\pgfqpoint{2.459567in}{6.340079in}}%
\pgfpathlineto{\pgfqpoint{2.462771in}{6.331785in}}%
\pgfpathlineto{\pgfqpoint{2.467576in}{6.303057in}}%
\pgfpathlineto{\pgfqpoint{2.472381in}{6.320555in}}%
\pgfpathlineto{\pgfqpoint{2.473983in}{6.321956in}}%
\pgfpathlineto{\pgfqpoint{2.475585in}{6.321141in}}%
\pgfpathlineto{\pgfqpoint{2.478789in}{6.312876in}}%
\pgfpathlineto{\pgfqpoint{2.483594in}{6.284264in}}%
\pgfpathlineto{\pgfqpoint{2.488399in}{6.301693in}}%
\pgfpathlineto{\pgfqpoint{2.490001in}{6.303087in}}%
\pgfpathlineto{\pgfqpoint{2.491603in}{6.302274in}}%
\pgfpathlineto{\pgfqpoint{2.494806in}{6.294037in}}%
\pgfpathlineto{\pgfqpoint{2.499612in}{6.265542in}}%
\pgfpathlineto{\pgfqpoint{2.504417in}{6.282902in}}%
\pgfpathlineto{\pgfqpoint{2.506019in}{6.284289in}}%
\pgfpathlineto{\pgfqpoint{2.507620in}{6.283478in}}%
\pgfpathlineto{\pgfqpoint{2.510824in}{6.275269in}}%
\pgfpathlineto{\pgfqpoint{2.515629in}{6.246890in}}%
\pgfpathlineto{\pgfqpoint{2.520435in}{6.264181in}}%
\pgfpathlineto{\pgfqpoint{2.522036in}{6.265562in}}%
\pgfpathlineto{\pgfqpoint{2.523638in}{6.264752in}}%
\pgfpathlineto{\pgfqpoint{2.526842in}{6.256571in}}%
\pgfpathlineto{\pgfqpoint{2.531647in}{6.228307in}}%
\pgfpathlineto{\pgfqpoint{2.536452in}{6.245529in}}%
\pgfpathlineto{\pgfqpoint{2.538054in}{6.246904in}}%
\pgfpathlineto{\pgfqpoint{2.539656in}{6.246096in}}%
\pgfpathlineto{\pgfqpoint{2.542859in}{6.237942in}}%
\pgfpathlineto{\pgfqpoint{2.547665in}{6.209794in}}%
\pgfpathlineto{\pgfqpoint{2.552470in}{6.226948in}}%
\pgfpathlineto{\pgfqpoint{2.554072in}{6.228316in}}%
\pgfpathlineto{\pgfqpoint{2.555674in}{6.227510in}}%
\pgfpathlineto{\pgfqpoint{2.558877in}{6.219384in}}%
\pgfpathlineto{\pgfqpoint{2.563683in}{6.191351in}}%
\pgfpathlineto{\pgfqpoint{2.568488in}{6.208436in}}%
\pgfpathlineto{\pgfqpoint{2.570090in}{6.209797in}}%
\pgfpathlineto{\pgfqpoint{2.571691in}{6.208993in}}%
\pgfpathlineto{\pgfqpoint{2.574895in}{6.200895in}}%
\pgfpathlineto{\pgfqpoint{2.579700in}{6.172976in}}%
\pgfpathlineto{\pgfqpoint{2.584506in}{6.189993in}}%
\pgfpathlineto{\pgfqpoint{2.586107in}{6.191348in}}%
\pgfpathlineto{\pgfqpoint{2.587709in}{6.190545in}}%
\pgfpathlineto{\pgfqpoint{2.590913in}{6.182475in}}%
\pgfpathlineto{\pgfqpoint{2.595718in}{6.154669in}}%
\pgfpathlineto{\pgfqpoint{2.600523in}{6.171619in}}%
\pgfpathlineto{\pgfqpoint{2.602125in}{6.172968in}}%
\pgfpathlineto{\pgfqpoint{2.603727in}{6.172166in}}%
\pgfpathlineto{\pgfqpoint{2.606930in}{6.164123in}}%
\pgfpathlineto{\pgfqpoint{2.611736in}{6.136432in}}%
\pgfpathlineto{\pgfqpoint{2.616541in}{6.153314in}}%
\pgfpathlineto{\pgfqpoint{2.618143in}{6.154656in}}%
\pgfpathlineto{\pgfqpoint{2.619745in}{6.153856in}}%
\pgfpathlineto{\pgfqpoint{2.622948in}{6.145841in}}%
\pgfpathlineto{\pgfqpoint{2.627754in}{6.118262in}}%
\pgfpathlineto{\pgfqpoint{2.632559in}{6.135077in}}%
\pgfpathlineto{\pgfqpoint{2.634161in}{6.136413in}}%
\pgfpathlineto{\pgfqpoint{2.635762in}{6.135615in}}%
\pgfpathlineto{\pgfqpoint{2.638966in}{6.127627in}}%
\pgfpathlineto{\pgfqpoint{2.643771in}{6.100160in}}%
\pgfpathlineto{\pgfqpoint{2.648577in}{6.116909in}}%
\pgfpathlineto{\pgfqpoint{2.650178in}{6.118238in}}%
\pgfpathlineto{\pgfqpoint{2.651780in}{6.117442in}}%
\pgfpathlineto{\pgfqpoint{2.654984in}{6.109481in}}%
\pgfpathlineto{\pgfqpoint{2.659789in}{6.082126in}}%
\pgfpathlineto{\pgfqpoint{2.664594in}{6.098808in}}%
\pgfpathlineto{\pgfqpoint{2.666196in}{6.100131in}}%
\pgfpathlineto{\pgfqpoint{2.667798in}{6.099336in}}%
\pgfpathlineto{\pgfqpoint{2.671001in}{6.091402in}}%
\pgfpathlineto{\pgfqpoint{2.675807in}{6.064159in}}%
\pgfpathlineto{\pgfqpoint{2.680612in}{6.080775in}}%
\pgfpathlineto{\pgfqpoint{2.682214in}{6.082092in}}%
\pgfpathlineto{\pgfqpoint{2.683816in}{6.081299in}}%
\pgfpathlineto{\pgfqpoint{2.687019in}{6.073392in}}%
\pgfpathlineto{\pgfqpoint{2.691824in}{6.046260in}}%
\pgfpathlineto{\pgfqpoint{2.696630in}{6.062809in}}%
\pgfpathlineto{\pgfqpoint{2.698232in}{6.064120in}}%
\pgfpathlineto{\pgfqpoint{2.699833in}{6.063328in}}%
\pgfpathlineto{\pgfqpoint{2.703037in}{6.055448in}}%
\pgfpathlineto{\pgfqpoint{2.707842in}{6.028427in}}%
\pgfpathlineto{\pgfqpoint{2.712648in}{6.044911in}}%
\pgfpathlineto{\pgfqpoint{2.714249in}{6.046215in}}%
\pgfpathlineto{\pgfqpoint{2.715851in}{6.045425in}}%
\pgfpathlineto{\pgfqpoint{2.719055in}{6.037572in}}%
\pgfpathlineto{\pgfqpoint{2.723860in}{6.010661in}}%
\pgfpathlineto{\pgfqpoint{2.728665in}{6.027079in}}%
\pgfpathlineto{\pgfqpoint{2.730267in}{6.028377in}}%
\pgfpathlineto{\pgfqpoint{2.731869in}{6.027589in}}%
\pgfpathlineto{\pgfqpoint{2.735072in}{6.019763in}}%
\pgfpathlineto{\pgfqpoint{2.739878in}{5.992961in}}%
\pgfpathlineto{\pgfqpoint{2.744683in}{6.009314in}}%
\pgfpathlineto{\pgfqpoint{2.746285in}{6.010606in}}%
\pgfpathlineto{\pgfqpoint{2.747887in}{6.009820in}}%
\pgfpathlineto{\pgfqpoint{2.751090in}{6.002020in}}%
\pgfpathlineto{\pgfqpoint{2.755895in}{5.975328in}}%
\pgfpathlineto{\pgfqpoint{2.760701in}{5.991616in}}%
\pgfpathlineto{\pgfqpoint{2.762302in}{5.992902in}}%
\pgfpathlineto{\pgfqpoint{2.763904in}{5.992117in}}%
\pgfpathlineto{\pgfqpoint{2.767108in}{5.984343in}}%
\pgfpathlineto{\pgfqpoint{2.771913in}{5.957760in}}%
\pgfpathlineto{\pgfqpoint{2.776718in}{5.973983in}}%
\pgfpathlineto{\pgfqpoint{2.778320in}{5.975263in}}%
\pgfpathlineto{\pgfqpoint{2.779922in}{5.974480in}}%
\pgfpathlineto{\pgfqpoint{2.783126in}{5.966733in}}%
\pgfpathlineto{\pgfqpoint{2.787931in}{5.940258in}}%
\pgfpathlineto{\pgfqpoint{2.792736in}{5.956417in}}%
\pgfpathlineto{\pgfqpoint{2.794338in}{5.957690in}}%
\pgfpathlineto{\pgfqpoint{2.795940in}{5.956909in}}%
\pgfpathlineto{\pgfqpoint{2.799143in}{5.949189in}}%
\pgfpathlineto{\pgfqpoint{2.803949in}{5.922822in}}%
\pgfpathlineto{\pgfqpoint{2.808754in}{5.938916in}}%
\pgfpathlineto{\pgfqpoint{2.810356in}{5.940184in}}%
\pgfpathlineto{\pgfqpoint{2.811957in}{5.939404in}}%
\pgfpathlineto{\pgfqpoint{2.815161in}{5.931710in}}%
\pgfpathlineto{\pgfqpoint{2.819966in}{5.905450in}}%
\pgfpathlineto{\pgfqpoint{2.824772in}{5.921481in}}%
\pgfpathlineto{\pgfqpoint{2.826373in}{5.922742in}}%
\pgfpathlineto{\pgfqpoint{2.827975in}{5.921964in}}%
\pgfpathlineto{\pgfqpoint{2.831179in}{5.914296in}}%
\pgfpathlineto{\pgfqpoint{2.835984in}{5.888144in}}%
\pgfpathlineto{\pgfqpoint{2.840789in}{5.904110in}}%
\pgfpathlineto{\pgfqpoint{2.842391in}{5.905366in}}%
\pgfpathlineto{\pgfqpoint{2.843993in}{5.904589in}}%
\pgfpathlineto{\pgfqpoint{2.847196in}{5.896948in}}%
\pgfpathlineto{\pgfqpoint{2.852002in}{5.870902in}}%
\pgfpathlineto{\pgfqpoint{2.856807in}{5.886805in}}%
\pgfpathlineto{\pgfqpoint{2.858409in}{5.888055in}}%
\pgfpathlineto{\pgfqpoint{2.860011in}{5.887280in}}%
\pgfpathlineto{\pgfqpoint{2.863214in}{5.879664in}}%
\pgfpathlineto{\pgfqpoint{2.868020in}{5.853725in}}%
\pgfpathlineto{\pgfqpoint{2.872825in}{5.869564in}}%
\pgfpathlineto{\pgfqpoint{2.874427in}{5.870808in}}%
\pgfpathlineto{\pgfqpoint{2.876028in}{5.870035in}}%
\pgfpathlineto{\pgfqpoint{2.879232in}{5.862445in}}%
\pgfpathlineto{\pgfqpoint{2.884037in}{5.836612in}}%
\pgfpathlineto{\pgfqpoint{2.888843in}{5.852388in}}%
\pgfpathlineto{\pgfqpoint{2.890444in}{5.853626in}}%
\pgfpathlineto{\pgfqpoint{2.892046in}{5.852854in}}%
\pgfpathlineto{\pgfqpoint{2.895250in}{5.845291in}}%
\pgfpathlineto{\pgfqpoint{2.900055in}{5.819563in}}%
\pgfpathlineto{\pgfqpoint{2.904860in}{5.835276in}}%
\pgfpathlineto{\pgfqpoint{2.906462in}{5.836508in}}%
\pgfpathlineto{\pgfqpoint{2.908064in}{5.835738in}}%
\pgfpathlineto{\pgfqpoint{2.911267in}{5.828200in}}%
\pgfpathlineto{\pgfqpoint{2.916073in}{5.802577in}}%
\pgfpathlineto{\pgfqpoint{2.920878in}{5.818228in}}%
\pgfpathlineto{\pgfqpoint{2.922480in}{5.819455in}}%
\pgfpathlineto{\pgfqpoint{2.924082in}{5.818686in}}%
\pgfpathlineto{\pgfqpoint{2.927285in}{5.811174in}}%
\pgfpathlineto{\pgfqpoint{2.932090in}{5.785655in}}%
\pgfpathlineto{\pgfqpoint{2.936896in}{5.801244in}}%
\pgfpathlineto{\pgfqpoint{2.938498in}{5.802464in}}%
\pgfpathlineto{\pgfqpoint{2.940099in}{5.801698in}}%
\pgfpathlineto{\pgfqpoint{2.943303in}{5.794211in}}%
\pgfpathlineto{\pgfqpoint{2.948108in}{5.768796in}}%
\pgfpathlineto{\pgfqpoint{2.952914in}{5.784323in}}%
\pgfpathlineto{\pgfqpoint{2.954515in}{5.785538in}}%
\pgfpathlineto{\pgfqpoint{2.956117in}{5.784773in}}%
\pgfpathlineto{\pgfqpoint{2.959321in}{5.777311in}}%
\pgfpathlineto{\pgfqpoint{2.964126in}{5.752000in}}%
\pgfpathlineto{\pgfqpoint{2.968931in}{5.767466in}}%
\pgfpathlineto{\pgfqpoint{2.970533in}{5.768675in}}%
\pgfpathlineto{\pgfqpoint{2.972135in}{5.767911in}}%
\pgfpathlineto{\pgfqpoint{2.975338in}{5.760475in}}%
\pgfpathlineto{\pgfqpoint{2.980144in}{5.735267in}}%
\pgfpathlineto{\pgfqpoint{2.984949in}{5.750671in}}%
\pgfpathlineto{\pgfqpoint{2.986551in}{5.751874in}}%
\pgfpathlineto{\pgfqpoint{2.988153in}{5.751112in}}%
\pgfpathlineto{\pgfqpoint{2.991356in}{5.743702in}}%
\pgfpathlineto{\pgfqpoint{2.996161in}{5.718597in}}%
\pgfpathlineto{\pgfqpoint{3.000967in}{5.733940in}}%
\pgfpathlineto{\pgfqpoint{3.002569in}{5.735137in}}%
\pgfpathlineto{\pgfqpoint{3.004170in}{5.734376in}}%
\pgfpathlineto{\pgfqpoint{3.007374in}{5.726991in}}%
\pgfpathlineto{\pgfqpoint{3.012179in}{5.701989in}}%
\pgfpathlineto{\pgfqpoint{3.016984in}{5.717270in}}%
\pgfpathlineto{\pgfqpoint{3.018586in}{5.718462in}}%
\pgfpathlineto{\pgfqpoint{3.020188in}{5.717703in}}%
\pgfpathlineto{\pgfqpoint{3.023392in}{5.710343in}}%
\pgfpathlineto{\pgfqpoint{3.028197in}{5.685443in}}%
\pgfpathlineto{\pgfqpoint{3.033002in}{5.700663in}}%
\pgfpathlineto{\pgfqpoint{3.034604in}{5.701849in}}%
\pgfpathlineto{\pgfqpoint{3.036206in}{5.701092in}}%
\pgfpathlineto{\pgfqpoint{3.039409in}{5.693757in}}%
\pgfpathlineto{\pgfqpoint{3.044215in}{5.668959in}}%
\pgfpathlineto{\pgfqpoint{3.049020in}{5.684119in}}%
\pgfpathlineto{\pgfqpoint{3.050622in}{5.685299in}}%
\pgfpathlineto{\pgfqpoint{3.052224in}{5.684543in}}%
\pgfpathlineto{\pgfqpoint{3.055427in}{5.677233in}}%
\pgfpathlineto{\pgfqpoint{3.060232in}{5.652536in}}%
\pgfpathlineto{\pgfqpoint{3.065038in}{5.667636in}}%
\pgfpathlineto{\pgfqpoint{3.066639in}{5.668810in}}%
\pgfpathlineto{\pgfqpoint{3.068241in}{5.668056in}}%
\pgfpathlineto{\pgfqpoint{3.071445in}{5.660771in}}%
\pgfpathlineto{\pgfqpoint{3.076250in}{5.636175in}}%
\pgfpathlineto{\pgfqpoint{3.081055in}{5.651214in}}%
\pgfpathlineto{\pgfqpoint{3.082657in}{5.652383in}}%
\pgfpathlineto{\pgfqpoint{3.084259in}{5.651631in}}%
\pgfpathlineto{\pgfqpoint{3.087463in}{5.644371in}}%
\pgfpathlineto{\pgfqpoint{3.092268in}{5.619874in}}%
\pgfpathlineto{\pgfqpoint{3.097073in}{5.634854in}}%
\pgfpathlineto{\pgfqpoint{3.098675in}{5.636018in}}%
\pgfpathlineto{\pgfqpoint{3.100277in}{5.635267in}}%
\pgfpathlineto{\pgfqpoint{3.103480in}{5.628031in}}%
\pgfpathlineto{\pgfqpoint{3.108286in}{5.603635in}}%
\pgfpathlineto{\pgfqpoint{3.113091in}{5.618556in}}%
\pgfpathlineto{\pgfqpoint{3.114693in}{5.619713in}}%
\pgfpathlineto{\pgfqpoint{3.116294in}{5.618964in}}%
\pgfpathlineto{\pgfqpoint{3.119498in}{5.611753in}}%
\pgfpathlineto{\pgfqpoint{3.124303in}{5.587457in}}%
\pgfpathlineto{\pgfqpoint{3.129109in}{5.602318in}}%
\pgfpathlineto{\pgfqpoint{3.132312in}{5.602723in}}%
\pgfpathlineto{\pgfqpoint{3.135516in}{5.595536in}}%
\pgfpathlineto{\pgfqpoint{3.140321in}{5.571339in}}%
\pgfpathlineto{\pgfqpoint{3.145126in}{5.586140in}}%
\pgfpathlineto{\pgfqpoint{3.148330in}{5.586542in}}%
\pgfpathlineto{\pgfqpoint{3.151533in}{5.579380in}}%
\pgfpathlineto{\pgfqpoint{3.156339in}{5.555281in}}%
\pgfpathlineto{\pgfqpoint{3.161144in}{5.570024in}}%
\pgfpathlineto{\pgfqpoint{3.164348in}{5.570421in}}%
\pgfpathlineto{\pgfqpoint{3.167551in}{5.563283in}}%
\pgfpathlineto{\pgfqpoint{3.172357in}{5.539283in}}%
\pgfpathlineto{\pgfqpoint{3.177162in}{5.553967in}}%
\pgfpathlineto{\pgfqpoint{3.180365in}{5.554361in}}%
\pgfpathlineto{\pgfqpoint{3.183569in}{5.547247in}}%
\pgfpathlineto{\pgfqpoint{3.188374in}{5.523345in}}%
\pgfpathlineto{\pgfqpoint{3.193180in}{5.537971in}}%
\pgfpathlineto{\pgfqpoint{3.196383in}{5.538360in}}%
\pgfpathlineto{\pgfqpoint{3.199587in}{5.531271in}}%
\pgfpathlineto{\pgfqpoint{3.204392in}{5.507466in}}%
\pgfpathlineto{\pgfqpoint{3.209197in}{5.522034in}}%
\pgfpathlineto{\pgfqpoint{3.212401in}{5.522420in}}%
\pgfpathlineto{\pgfqpoint{3.215604in}{5.515355in}}%
\pgfpathlineto{\pgfqpoint{3.220410in}{5.491647in}}%
\pgfpathlineto{\pgfqpoint{3.225215in}{5.506157in}}%
\pgfpathlineto{\pgfqpoint{3.228419in}{5.506539in}}%
\pgfpathlineto{\pgfqpoint{3.231622in}{5.499498in}}%
\pgfpathlineto{\pgfqpoint{3.236427in}{5.475887in}}%
\pgfpathlineto{\pgfqpoint{3.241233in}{5.490339in}}%
\pgfpathlineto{\pgfqpoint{3.244436in}{5.490717in}}%
\pgfpathlineto{\pgfqpoint{3.247640in}{5.483701in}}%
\pgfpathlineto{\pgfqpoint{3.252445in}{5.460186in}}%
\pgfpathlineto{\pgfqpoint{3.257251in}{5.474581in}}%
\pgfpathlineto{\pgfqpoint{3.260454in}{5.474955in}}%
\pgfpathlineto{\pgfqpoint{3.263658in}{5.467962in}}%
\pgfpathlineto{\pgfqpoint{3.268463in}{5.444544in}}%
\pgfpathlineto{\pgfqpoint{3.273268in}{5.458881in}}%
\pgfpathlineto{\pgfqpoint{3.276472in}{5.459252in}}%
\pgfpathlineto{\pgfqpoint{3.279675in}{5.452283in}}%
\pgfpathlineto{\pgfqpoint{3.284481in}{5.428960in}}%
\pgfpathlineto{\pgfqpoint{3.289286in}{5.443240in}}%
\pgfpathlineto{\pgfqpoint{3.292490in}{5.443607in}}%
\pgfpathlineto{\pgfqpoint{3.295693in}{5.436662in}}%
\pgfpathlineto{\pgfqpoint{3.300498in}{5.413434in}}%
\pgfpathlineto{\pgfqpoint{3.305304in}{5.427657in}}%
\pgfpathlineto{\pgfqpoint{3.308507in}{5.428021in}}%
\pgfpathlineto{\pgfqpoint{3.311711in}{5.421099in}}%
\pgfpathlineto{\pgfqpoint{3.316516in}{5.397966in}}%
\pgfpathlineto{\pgfqpoint{3.321321in}{5.412133in}}%
\pgfpathlineto{\pgfqpoint{3.324525in}{5.412493in}}%
\pgfpathlineto{\pgfqpoint{3.327729in}{5.405595in}}%
\pgfpathlineto{\pgfqpoint{3.332534in}{5.382556in}}%
\pgfpathlineto{\pgfqpoint{3.337339in}{5.396667in}}%
\pgfpathlineto{\pgfqpoint{3.340543in}{5.397023in}}%
\pgfpathlineto{\pgfqpoint{3.343746in}{5.390148in}}%
\pgfpathlineto{\pgfqpoint{3.348552in}{5.367204in}}%
\pgfpathlineto{\pgfqpoint{3.353357in}{5.381258in}}%
\pgfpathlineto{\pgfqpoint{3.356560in}{5.381611in}}%
\pgfpathlineto{\pgfqpoint{3.359764in}{5.374760in}}%
\pgfpathlineto{\pgfqpoint{3.364569in}{5.351909in}}%
\pgfpathlineto{\pgfqpoint{3.369375in}{5.365907in}}%
\pgfpathlineto{\pgfqpoint{3.372578in}{5.366256in}}%
\pgfpathlineto{\pgfqpoint{3.375782in}{5.359429in}}%
\pgfpathlineto{\pgfqpoint{3.380587in}{5.336671in}}%
\pgfpathlineto{\pgfqpoint{3.385392in}{5.350614in}}%
\pgfpathlineto{\pgfqpoint{3.388596in}{5.350959in}}%
\pgfpathlineto{\pgfqpoint{3.391800in}{5.344155in}}%
\pgfpathlineto{\pgfqpoint{3.396605in}{5.321490in}}%
\pgfpathlineto{\pgfqpoint{3.401410in}{5.335377in}}%
\pgfpathlineto{\pgfqpoint{3.404614in}{5.335719in}}%
\pgfpathlineto{\pgfqpoint{3.407817in}{5.328938in}}%
\pgfpathlineto{\pgfqpoint{3.412623in}{5.306366in}}%
\pgfpathlineto{\pgfqpoint{3.417428in}{5.320198in}}%
\pgfpathlineto{\pgfqpoint{3.420631in}{5.320536in}}%
\pgfpathlineto{\pgfqpoint{3.423835in}{5.313779in}}%
\pgfpathlineto{\pgfqpoint{3.428640in}{5.291299in}}%
\pgfpathlineto{\pgfqpoint{3.433446in}{5.305075in}}%
\pgfpathlineto{\pgfqpoint{3.436649in}{5.305410in}}%
\pgfpathlineto{\pgfqpoint{3.439853in}{5.298675in}}%
\pgfpathlineto{\pgfqpoint{3.444658in}{5.276287in}}%
\pgfpathlineto{\pgfqpoint{3.449463in}{5.290009in}}%
\pgfpathlineto{\pgfqpoint{3.452667in}{5.290341in}}%
\pgfpathlineto{\pgfqpoint{3.455870in}{5.283629in}}%
\pgfpathlineto{\pgfqpoint{3.460676in}{5.261332in}}%
\pgfpathlineto{\pgfqpoint{3.465481in}{5.274999in}}%
\pgfpathlineto{\pgfqpoint{3.468685in}{5.275328in}}%
\pgfpathlineto{\pgfqpoint{3.471888in}{5.268638in}}%
\pgfpathlineto{\pgfqpoint{3.476694in}{5.246433in}}%
\pgfpathlineto{\pgfqpoint{3.481499in}{5.260046in}}%
\pgfpathlineto{\pgfqpoint{3.484702in}{5.260370in}}%
\pgfpathlineto{\pgfqpoint{3.487906in}{5.253704in}}%
\pgfpathlineto{\pgfqpoint{3.492711in}{5.231589in}}%
\pgfpathlineto{\pgfqpoint{3.497517in}{5.245148in}}%
\pgfpathlineto{\pgfqpoint{3.500720in}{5.245469in}}%
\pgfpathlineto{\pgfqpoint{3.503924in}{5.238826in}}%
\pgfpathlineto{\pgfqpoint{3.508729in}{5.216801in}}%
\pgfpathlineto{\pgfqpoint{3.513534in}{5.230306in}}%
\pgfpathlineto{\pgfqpoint{3.516738in}{5.230624in}}%
\pgfpathlineto{\pgfqpoint{3.519941in}{5.224003in}}%
\pgfpathlineto{\pgfqpoint{3.524747in}{5.202068in}}%
\pgfpathlineto{\pgfqpoint{3.529552in}{5.215519in}}%
\pgfpathlineto{\pgfqpoint{3.532756in}{5.215834in}}%
\pgfpathlineto{\pgfqpoint{3.535959in}{5.209235in}}%
\pgfpathlineto{\pgfqpoint{3.540764in}{5.187390in}}%
\pgfpathlineto{\pgfqpoint{3.545570in}{5.200788in}}%
\pgfpathlineto{\pgfqpoint{3.548773in}{5.201099in}}%
\pgfpathlineto{\pgfqpoint{3.551977in}{5.194523in}}%
\pgfpathlineto{\pgfqpoint{3.556782in}{5.172767in}}%
\pgfpathlineto{\pgfqpoint{3.561588in}{5.186112in}}%
\pgfpathlineto{\pgfqpoint{3.564791in}{5.186419in}}%
\pgfpathlineto{\pgfqpoint{3.567995in}{5.179866in}}%
\pgfpathlineto{\pgfqpoint{3.572800in}{5.158199in}}%
\pgfpathlineto{\pgfqpoint{3.577605in}{5.171490in}}%
\pgfpathlineto{\pgfqpoint{3.580809in}{5.171795in}}%
\pgfpathlineto{\pgfqpoint{3.584012in}{5.165264in}}%
\pgfpathlineto{\pgfqpoint{3.588818in}{5.143685in}}%
\pgfpathlineto{\pgfqpoint{3.593623in}{5.156923in}}%
\pgfpathlineto{\pgfqpoint{3.596827in}{5.157225in}}%
\pgfpathlineto{\pgfqpoint{3.600030in}{5.150716in}}%
\pgfpathlineto{\pgfqpoint{3.604835in}{5.129225in}}%
\pgfpathlineto{\pgfqpoint{3.609641in}{5.142411in}}%
\pgfpathlineto{\pgfqpoint{3.612844in}{5.142709in}}%
\pgfpathlineto{\pgfqpoint{3.616048in}{5.136222in}}%
\pgfpathlineto{\pgfqpoint{3.620853in}{5.114820in}}%
\pgfpathlineto{\pgfqpoint{3.625658in}{5.127953in}}%
\pgfpathlineto{\pgfqpoint{3.628862in}{5.128248in}}%
\pgfpathlineto{\pgfqpoint{3.632066in}{5.121783in}}%
\pgfpathlineto{\pgfqpoint{3.636871in}{5.100468in}}%
\pgfpathlineto{\pgfqpoint{3.641676in}{5.113549in}}%
\pgfpathlineto{\pgfqpoint{3.644880in}{5.113840in}}%
\pgfpathlineto{\pgfqpoint{3.648083in}{5.107398in}}%
\pgfpathlineto{\pgfqpoint{3.652889in}{5.086170in}}%
\pgfpathlineto{\pgfqpoint{3.657694in}{5.099199in}}%
\pgfpathlineto{\pgfqpoint{3.660897in}{5.099487in}}%
\pgfpathlineto{\pgfqpoint{3.664101in}{5.093067in}}%
\pgfpathlineto{\pgfqpoint{3.668906in}{5.071925in}}%
\pgfpathlineto{\pgfqpoint{3.673712in}{5.084902in}}%
\pgfpathlineto{\pgfqpoint{3.676915in}{5.085187in}}%
\pgfpathlineto{\pgfqpoint{3.680119in}{5.078789in}}%
\pgfpathlineto{\pgfqpoint{3.684924in}{5.057734in}}%
\pgfpathlineto{\pgfqpoint{3.689729in}{5.070659in}}%
\pgfpathlineto{\pgfqpoint{3.692933in}{5.070941in}}%
\pgfpathlineto{\pgfqpoint{3.696136in}{5.064564in}}%
\pgfpathlineto{\pgfqpoint{3.700942in}{5.043595in}}%
\pgfpathlineto{\pgfqpoint{3.705747in}{5.056469in}}%
\pgfpathlineto{\pgfqpoint{3.708951in}{5.056748in}}%
\pgfpathlineto{\pgfqpoint{3.712154in}{5.050393in}}%
\pgfpathlineto{\pgfqpoint{3.716960in}{5.029510in}}%
\pgfpathlineto{\pgfqpoint{3.721765in}{5.042332in}}%
\pgfpathlineto{\pgfqpoint{3.724968in}{5.042608in}}%
\pgfpathlineto{\pgfqpoint{3.728172in}{5.036275in}}%
\pgfpathlineto{\pgfqpoint{3.732977in}{5.015477in}}%
\pgfpathlineto{\pgfqpoint{3.737783in}{5.028248in}}%
\pgfpathlineto{\pgfqpoint{3.740986in}{5.028521in}}%
\pgfpathlineto{\pgfqpoint{3.744190in}{5.022209in}}%
\pgfpathlineto{\pgfqpoint{3.748995in}{5.001496in}}%
\pgfpathlineto{\pgfqpoint{3.753800in}{5.014217in}}%
\pgfpathlineto{\pgfqpoint{3.757004in}{5.014486in}}%
\pgfpathlineto{\pgfqpoint{3.760207in}{5.008196in}}%
\pgfpathlineto{\pgfqpoint{3.765013in}{4.987568in}}%
\pgfpathlineto{\pgfqpoint{3.769818in}{5.000238in}}%
\pgfpathlineto{\pgfqpoint{3.773022in}{5.000504in}}%
\pgfpathlineto{\pgfqpoint{3.776225in}{4.994236in}}%
\pgfpathlineto{\pgfqpoint{3.781030in}{4.973692in}}%
\pgfpathlineto{\pgfqpoint{3.785836in}{4.986311in}}%
\pgfpathlineto{\pgfqpoint{3.789039in}{4.986575in}}%
\pgfpathlineto{\pgfqpoint{3.792243in}{4.980328in}}%
\pgfpathlineto{\pgfqpoint{3.797048in}{4.959867in}}%
\pgfpathlineto{\pgfqpoint{3.801854in}{4.972437in}}%
\pgfpathlineto{\pgfqpoint{3.805057in}{4.972697in}}%
\pgfpathlineto{\pgfqpoint{3.808261in}{4.966471in}}%
\pgfpathlineto{\pgfqpoint{3.813066in}{4.946095in}}%
\pgfpathlineto{\pgfqpoint{3.817871in}{4.958614in}}%
\pgfpathlineto{\pgfqpoint{3.821075in}{4.958871in}}%
\pgfpathlineto{\pgfqpoint{3.824278in}{4.952667in}}%
\pgfpathlineto{\pgfqpoint{3.829084in}{4.932373in}}%
\pgfpathlineto{\pgfqpoint{3.833889in}{4.944843in}}%
\pgfpathlineto{\pgfqpoint{3.837093in}{4.945097in}}%
\pgfpathlineto{\pgfqpoint{3.840296in}{4.938914in}}%
\pgfpathlineto{\pgfqpoint{3.845101in}{4.918703in}}%
\pgfpathlineto{\pgfqpoint{3.849907in}{4.931123in}}%
\pgfpathlineto{\pgfqpoint{3.853110in}{4.931375in}}%
\pgfpathlineto{\pgfqpoint{3.856314in}{4.925213in}}%
\pgfpathlineto{\pgfqpoint{3.861119in}{4.905085in}}%
\pgfpathlineto{\pgfqpoint{3.865924in}{4.917455in}}%
\pgfpathlineto{\pgfqpoint{3.869128in}{4.917704in}}%
\pgfpathlineto{\pgfqpoint{3.872332in}{4.911562in}}%
\pgfpathlineto{\pgfqpoint{3.877137in}{4.891517in}}%
\pgfpathlineto{\pgfqpoint{3.881942in}{4.903838in}}%
\pgfpathlineto{\pgfqpoint{3.885146in}{4.904083in}}%
\pgfpathlineto{\pgfqpoint{3.888349in}{4.897963in}}%
\pgfpathlineto{\pgfqpoint{3.893155in}{4.877999in}}%
\pgfpathlineto{\pgfqpoint{3.897960in}{4.890272in}}%
\pgfpathlineto{\pgfqpoint{3.901164in}{4.890514in}}%
\pgfpathlineto{\pgfqpoint{3.904367in}{4.884415in}}%
\pgfpathlineto{\pgfqpoint{3.909172in}{4.864533in}}%
\pgfpathlineto{\pgfqpoint{3.913978in}{4.876756in}}%
\pgfpathlineto{\pgfqpoint{3.917181in}{4.876996in}}%
\pgfpathlineto{\pgfqpoint{3.920385in}{4.870917in}}%
\pgfpathlineto{\pgfqpoint{3.925190in}{4.851116in}}%
\pgfpathlineto{\pgfqpoint{3.929995in}{4.863291in}}%
\pgfpathlineto{\pgfqpoint{3.933199in}{4.863528in}}%
\pgfpathlineto{\pgfqpoint{3.936403in}{4.857470in}}%
\pgfpathlineto{\pgfqpoint{3.941208in}{4.837750in}}%
\pgfpathlineto{\pgfqpoint{3.946013in}{4.849876in}}%
\pgfpathlineto{\pgfqpoint{3.949217in}{4.850110in}}%
\pgfpathlineto{\pgfqpoint{3.952420in}{4.844073in}}%
\pgfpathlineto{\pgfqpoint{3.957226in}{4.824434in}}%
\pgfpathlineto{\pgfqpoint{3.962031in}{4.836512in}}%
\pgfpathlineto{\pgfqpoint{3.965234in}{4.836743in}}%
\pgfpathlineto{\pgfqpoint{3.968438in}{4.830726in}}%
\pgfpathlineto{\pgfqpoint{3.973243in}{4.811167in}}%
\pgfpathlineto{\pgfqpoint{3.978049in}{4.823197in}}%
\pgfpathlineto{\pgfqpoint{3.981252in}{4.823425in}}%
\pgfpathlineto{\pgfqpoint{3.984456in}{4.817429in}}%
\pgfpathlineto{\pgfqpoint{3.989261in}{4.797950in}}%
\pgfpathlineto{\pgfqpoint{3.994066in}{4.809932in}}%
\pgfpathlineto{\pgfqpoint{3.997270in}{4.810158in}}%
\pgfpathlineto{\pgfqpoint{4.000473in}{4.804182in}}%
\pgfpathlineto{\pgfqpoint{4.005279in}{4.784782in}}%
\pgfpathlineto{\pgfqpoint{4.010084in}{4.796717in}}%
\pgfpathlineto{\pgfqpoint{4.013288in}{4.796940in}}%
\pgfpathlineto{\pgfqpoint{4.016491in}{4.790984in}}%
\pgfpathlineto{\pgfqpoint{4.021297in}{4.771664in}}%
\pgfpathlineto{\pgfqpoint{4.026102in}{4.783551in}}%
\pgfpathlineto{\pgfqpoint{4.029305in}{4.783771in}}%
\pgfpathlineto{\pgfqpoint{4.032509in}{4.777836in}}%
\pgfpathlineto{\pgfqpoint{4.037314in}{4.758595in}}%
\pgfpathlineto{\pgfqpoint{4.042120in}{4.770434in}}%
\pgfpathlineto{\pgfqpoint{4.045323in}{4.770651in}}%
\pgfpathlineto{\pgfqpoint{4.048527in}{4.764737in}}%
\pgfpathlineto{\pgfqpoint{4.053332in}{4.745574in}}%
\pgfpathlineto{\pgfqpoint{4.058137in}{4.757366in}}%
\pgfpathlineto{\pgfqpoint{4.061341in}{4.757581in}}%
\pgfpathlineto{\pgfqpoint{4.064544in}{4.751687in}}%
\pgfpathlineto{\pgfqpoint{4.069350in}{4.732602in}}%
\pgfpathlineto{\pgfqpoint{4.074155in}{4.744348in}}%
\pgfpathlineto{\pgfqpoint{4.077359in}{4.744559in}}%
\pgfpathlineto{\pgfqpoint{4.080562in}{4.738685in}}%
\pgfpathlineto{\pgfqpoint{4.085367in}{4.719679in}}%
\pgfpathlineto{\pgfqpoint{4.090173in}{4.731378in}}%
\pgfpathlineto{\pgfqpoint{4.093376in}{4.731587in}}%
\pgfpathlineto{\pgfqpoint{4.096580in}{4.725733in}}%
\pgfpathlineto{\pgfqpoint{4.101385in}{4.706804in}}%
\pgfpathlineto{\pgfqpoint{4.106191in}{4.718456in}}%
\pgfpathlineto{\pgfqpoint{4.109394in}{4.718662in}}%
\pgfpathlineto{\pgfqpoint{4.112598in}{4.712828in}}%
\pgfpathlineto{\pgfqpoint{4.117403in}{4.693977in}}%
\pgfpathlineto{\pgfqpoint{4.122208in}{4.705582in}}%
\pgfpathlineto{\pgfqpoint{4.125412in}{4.705786in}}%
\pgfpathlineto{\pgfqpoint{4.128615in}{4.699972in}}%
\pgfpathlineto{\pgfqpoint{4.133421in}{4.681198in}}%
\pgfpathlineto{\pgfqpoint{4.138226in}{4.692757in}}%
\pgfpathlineto{\pgfqpoint{4.141430in}{4.692958in}}%
\pgfpathlineto{\pgfqpoint{4.144633in}{4.687164in}}%
\pgfpathlineto{\pgfqpoint{4.149438in}{4.668467in}}%
\pgfpathlineto{\pgfqpoint{4.154244in}{4.679980in}}%
\pgfpathlineto{\pgfqpoint{4.157447in}{4.680178in}}%
\pgfpathlineto{\pgfqpoint{4.160651in}{4.674404in}}%
\pgfpathlineto{\pgfqpoint{4.165456in}{4.655783in}}%
\pgfpathlineto{\pgfqpoint{4.170261in}{4.667250in}}%
\pgfpathlineto{\pgfqpoint{4.173465in}{4.667446in}}%
\pgfpathlineto{\pgfqpoint{4.176669in}{4.661691in}}%
\pgfpathlineto{\pgfqpoint{4.181474in}{4.643147in}}%
\pgfpathlineto{\pgfqpoint{4.186279in}{4.654568in}}%
\pgfpathlineto{\pgfqpoint{4.189483in}{4.654762in}}%
\pgfpathlineto{\pgfqpoint{4.192686in}{4.649027in}}%
\pgfpathlineto{\pgfqpoint{4.197492in}{4.630558in}}%
\pgfpathlineto{\pgfqpoint{4.202297in}{4.641934in}}%
\pgfpathlineto{\pgfqpoint{4.205500in}{4.642125in}}%
\pgfpathlineto{\pgfqpoint{4.208704in}{4.636409in}}%
\pgfpathlineto{\pgfqpoint{4.213509in}{4.618016in}}%
\pgfpathlineto{\pgfqpoint{4.218315in}{4.629347in}}%
\pgfpathlineto{\pgfqpoint{4.221518in}{4.629535in}}%
\pgfpathlineto{\pgfqpoint{4.224722in}{4.623839in}}%
\pgfpathlineto{\pgfqpoint{4.229527in}{4.605520in}}%
\pgfpathlineto{\pgfqpoint{4.234332in}{4.616806in}}%
\pgfpathlineto{\pgfqpoint{4.237536in}{4.616992in}}%
\pgfpathlineto{\pgfqpoint{4.240740in}{4.611315in}}%
\pgfpathlineto{\pgfqpoint{4.245545in}{4.593072in}}%
\pgfpathlineto{\pgfqpoint{4.250350in}{4.604313in}}%
\pgfpathlineto{\pgfqpoint{4.253554in}{4.604496in}}%
\pgfpathlineto{\pgfqpoint{4.256757in}{4.598839in}}%
\pgfpathlineto{\pgfqpoint{4.261563in}{4.580670in}}%
\pgfpathlineto{\pgfqpoint{4.266368in}{4.591866in}}%
\pgfpathlineto{\pgfqpoint{4.269571in}{4.592047in}}%
\pgfpathlineto{\pgfqpoint{4.272775in}{4.586409in}}%
\pgfpathlineto{\pgfqpoint{4.277580in}{4.568314in}}%
\pgfpathlineto{\pgfqpoint{4.282386in}{4.579466in}}%
\pgfpathlineto{\pgfqpoint{4.285589in}{4.579644in}}%
\pgfpathlineto{\pgfqpoint{4.288793in}{4.574025in}}%
\pgfpathlineto{\pgfqpoint{4.293598in}{4.556005in}}%
\pgfpathlineto{\pgfqpoint{4.298403in}{4.567112in}}%
\pgfpathlineto{\pgfqpoint{4.301607in}{4.567288in}}%
\pgfpathlineto{\pgfqpoint{4.304810in}{4.561688in}}%
\pgfpathlineto{\pgfqpoint{4.309616in}{4.543742in}}%
\pgfpathlineto{\pgfqpoint{4.314421in}{4.554805in}}%
\pgfpathlineto{\pgfqpoint{4.317625in}{4.554978in}}%
\pgfpathlineto{\pgfqpoint{4.320828in}{4.549397in}}%
\pgfpathlineto{\pgfqpoint{4.325634in}{4.531524in}}%
\pgfpathlineto{\pgfqpoint{4.330439in}{4.542543in}}%
\pgfpathlineto{\pgfqpoint{4.333642in}{4.542714in}}%
\pgfpathlineto{\pgfqpoint{4.336846in}{4.537152in}}%
\pgfpathlineto{\pgfqpoint{4.341651in}{4.519352in}}%
\pgfpathlineto{\pgfqpoint{4.346457in}{4.530327in}}%
\pgfpathlineto{\pgfqpoint{4.349660in}{4.530495in}}%
\pgfpathlineto{\pgfqpoint{4.352864in}{4.524953in}}%
\pgfpathlineto{\pgfqpoint{4.357669in}{4.507226in}}%
\pgfpathlineto{\pgfqpoint{4.362474in}{4.518157in}}%
\pgfpathlineto{\pgfqpoint{4.365678in}{4.518323in}}%
\pgfpathlineto{\pgfqpoint{4.368881in}{4.512799in}}%
\pgfpathlineto{\pgfqpoint{4.373687in}{4.495145in}}%
\pgfpathlineto{\pgfqpoint{4.378492in}{4.506032in}}%
\pgfpathlineto{\pgfqpoint{4.381696in}{4.506196in}}%
\pgfpathlineto{\pgfqpoint{4.384899in}{4.500691in}}%
\pgfpathlineto{\pgfqpoint{4.389704in}{4.483109in}}%
\pgfpathlineto{\pgfqpoint{4.394510in}{4.493953in}}%
\pgfpathlineto{\pgfqpoint{4.397713in}{4.494114in}}%
\pgfpathlineto{\pgfqpoint{4.400917in}{4.488628in}}%
\pgfpathlineto{\pgfqpoint{4.405722in}{4.471118in}}%
\pgfpathlineto{\pgfqpoint{4.410528in}{4.481919in}}%
\pgfpathlineto{\pgfqpoint{4.413731in}{4.482078in}}%
\pgfpathlineto{\pgfqpoint{4.416935in}{4.476611in}}%
\pgfpathlineto{\pgfqpoint{4.421740in}{4.459171in}}%
\pgfpathlineto{\pgfqpoint{4.426545in}{4.469930in}}%
\pgfpathlineto{\pgfqpoint{4.429749in}{4.470086in}}%
\pgfpathlineto{\pgfqpoint{4.432952in}{4.464638in}}%
\pgfpathlineto{\pgfqpoint{4.437758in}{4.447270in}}%
\pgfpathlineto{\pgfqpoint{4.442563in}{4.457985in}}%
\pgfpathlineto{\pgfqpoint{4.445767in}{4.458139in}}%
\pgfpathlineto{\pgfqpoint{4.448970in}{4.452710in}}%
\pgfpathlineto{\pgfqpoint{4.453775in}{4.435413in}}%
\pgfpathlineto{\pgfqpoint{4.458581in}{4.446085in}}%
\pgfpathlineto{\pgfqpoint{4.461784in}{4.446237in}}%
\pgfpathlineto{\pgfqpoint{4.464988in}{4.440826in}}%
\pgfpathlineto{\pgfqpoint{4.469793in}{4.423600in}}%
\pgfpathlineto{\pgfqpoint{4.474598in}{4.434230in}}%
\pgfpathlineto{\pgfqpoint{4.477802in}{4.434380in}}%
\pgfpathlineto{\pgfqpoint{4.481006in}{4.428987in}}%
\pgfpathlineto{\pgfqpoint{4.485811in}{4.411831in}}%
\pgfpathlineto{\pgfqpoint{4.490616in}{4.422419in}}%
\pgfpathlineto{\pgfqpoint{4.493820in}{4.422566in}}%
\pgfpathlineto{\pgfqpoint{4.497023in}{4.417192in}}%
\pgfpathlineto{\pgfqpoint{4.501829in}{4.400107in}}%
\pgfpathlineto{\pgfqpoint{4.506634in}{4.410653in}}%
\pgfpathlineto{\pgfqpoint{4.509837in}{4.410797in}}%
\pgfpathlineto{\pgfqpoint{4.513041in}{4.405442in}}%
\pgfpathlineto{\pgfqpoint{4.517846in}{4.388426in}}%
\pgfpathlineto{\pgfqpoint{4.522652in}{4.398930in}}%
\pgfpathlineto{\pgfqpoint{4.525855in}{4.399072in}}%
\pgfpathlineto{\pgfqpoint{4.529059in}{4.393735in}}%
\pgfpathlineto{\pgfqpoint{4.533864in}{4.376789in}}%
\pgfpathlineto{\pgfqpoint{4.538669in}{4.387251in}}%
\pgfpathlineto{\pgfqpoint{4.541873in}{4.387391in}}%
\pgfpathlineto{\pgfqpoint{4.545076in}{4.382072in}}%
\pgfpathlineto{\pgfqpoint{4.549882in}{4.365195in}}%
\pgfpathlineto{\pgfqpoint{4.554687in}{4.375615in}}%
\pgfpathlineto{\pgfqpoint{4.557891in}{4.375753in}}%
\pgfpathlineto{\pgfqpoint{4.561094in}{4.370452in}}%
\pgfpathlineto{\pgfqpoint{4.565900in}{4.353645in}}%
\pgfpathlineto{\pgfqpoint{4.570705in}{4.364024in}}%
\pgfpathlineto{\pgfqpoint{4.573908in}{4.364159in}}%
\pgfpathlineto{\pgfqpoint{4.577112in}{4.358876in}}%
\pgfpathlineto{\pgfqpoint{4.581917in}{4.342138in}}%
\pgfpathlineto{\pgfqpoint{4.586723in}{4.352475in}}%
\pgfpathlineto{\pgfqpoint{4.589926in}{4.352609in}}%
\pgfpathlineto{\pgfqpoint{4.593130in}{4.347344in}}%
\pgfpathlineto{\pgfqpoint{4.597935in}{4.330674in}}%
\pgfpathlineto{\pgfqpoint{4.602740in}{4.340970in}}%
\pgfpathlineto{\pgfqpoint{4.605944in}{4.341101in}}%
\pgfpathlineto{\pgfqpoint{4.609147in}{4.335854in}}%
\pgfpathlineto{\pgfqpoint{4.613953in}{4.319252in}}%
\pgfpathlineto{\pgfqpoint{4.618758in}{4.329507in}}%
\pgfpathlineto{\pgfqpoint{4.621962in}{4.329637in}}%
\pgfpathlineto{\pgfqpoint{4.625165in}{4.324408in}}%
\pgfpathlineto{\pgfqpoint{4.629970in}{4.307874in}}%
\pgfpathlineto{\pgfqpoint{4.634776in}{4.318088in}}%
\pgfpathlineto{\pgfqpoint{4.637979in}{4.318215in}}%
\pgfpathlineto{\pgfqpoint{4.641183in}{4.313004in}}%
\pgfpathlineto{\pgfqpoint{4.645988in}{4.296538in}}%
\pgfpathlineto{\pgfqpoint{4.650794in}{4.306711in}}%
\pgfpathlineto{\pgfqpoint{4.653997in}{4.306836in}}%
\pgfpathlineto{\pgfqpoint{4.657201in}{4.301643in}}%
\pgfpathlineto{\pgfqpoint{4.662006in}{4.285244in}}%
\pgfpathlineto{\pgfqpoint{4.666811in}{4.295377in}}%
\pgfpathlineto{\pgfqpoint{4.670015in}{4.295500in}}%
\pgfpathlineto{\pgfqpoint{4.673218in}{4.290324in}}%
\pgfpathlineto{\pgfqpoint{4.678024in}{4.273992in}}%
\pgfpathlineto{\pgfqpoint{4.682829in}{4.284085in}}%
\pgfpathlineto{\pgfqpoint{4.686033in}{4.284206in}}%
\pgfpathlineto{\pgfqpoint{4.689236in}{4.279048in}}%
\pgfpathlineto{\pgfqpoint{4.694041in}{4.262783in}}%
\pgfpathlineto{\pgfqpoint{4.698847in}{4.272835in}}%
\pgfpathlineto{\pgfqpoint{4.702050in}{4.272954in}}%
\pgfpathlineto{\pgfqpoint{4.705254in}{4.267814in}}%
\pgfpathlineto{\pgfqpoint{4.710059in}{4.251615in}}%
\pgfpathlineto{\pgfqpoint{4.714864in}{4.261628in}}%
\pgfpathlineto{\pgfqpoint{4.718068in}{4.261744in}}%
\pgfpathlineto{\pgfqpoint{4.721272in}{4.256621in}}%
\pgfpathlineto{\pgfqpoint{4.726077in}{4.240489in}}%
\pgfpathlineto{\pgfqpoint{4.730882in}{4.250462in}}%
\pgfpathlineto{\pgfqpoint{4.734086in}{4.250576in}}%
\pgfpathlineto{\pgfqpoint{4.737289in}{4.245471in}}%
\pgfpathlineto{\pgfqpoint{4.742095in}{4.229405in}}%
\pgfpathlineto{\pgfqpoint{4.746900in}{4.239338in}}%
\pgfpathlineto{\pgfqpoint{4.750104in}{4.239450in}}%
\pgfpathlineto{\pgfqpoint{4.753307in}{4.234363in}}%
\pgfpathlineto{\pgfqpoint{4.758112in}{4.218362in}}%
\pgfpathlineto{\pgfqpoint{4.762918in}{4.228255in}}%
\pgfpathlineto{\pgfqpoint{4.766121in}{4.228366in}}%
\pgfpathlineto{\pgfqpoint{4.769325in}{4.223295in}}%
\pgfpathlineto{\pgfqpoint{4.774130in}{4.207361in}}%
\pgfpathlineto{\pgfqpoint{4.778935in}{4.217214in}}%
\pgfpathlineto{\pgfqpoint{4.782139in}{4.217323in}}%
\pgfpathlineto{\pgfqpoint{4.785343in}{4.212270in}}%
\pgfpathlineto{\pgfqpoint{4.790148in}{4.196400in}}%
\pgfpathlineto{\pgfqpoint{4.794953in}{4.206215in}}%
\pgfpathlineto{\pgfqpoint{4.798157in}{4.206321in}}%
\pgfpathlineto{\pgfqpoint{4.801360in}{4.201285in}}%
\pgfpathlineto{\pgfqpoint{4.806166in}{4.185481in}}%
\pgfpathlineto{\pgfqpoint{4.810971in}{4.195256in}}%
\pgfpathlineto{\pgfqpoint{4.814174in}{4.195360in}}%
\pgfpathlineto{\pgfqpoint{4.817378in}{4.190342in}}%
\pgfpathlineto{\pgfqpoint{4.822183in}{4.174602in}}%
\pgfpathlineto{\pgfqpoint{4.826989in}{4.184338in}}%
\pgfpathlineto{\pgfqpoint{4.830192in}{4.184441in}}%
\pgfpathlineto{\pgfqpoint{4.833396in}{4.179439in}}%
\pgfpathlineto{\pgfqpoint{4.838201in}{4.163764in}}%
\pgfpathlineto{\pgfqpoint{4.843006in}{4.173461in}}%
\pgfpathlineto{\pgfqpoint{4.846210in}{4.173562in}}%
\pgfpathlineto{\pgfqpoint{4.849413in}{4.168578in}}%
\pgfpathlineto{\pgfqpoint{4.854219in}{4.152966in}}%
\pgfpathlineto{\pgfqpoint{4.859024in}{4.162625in}}%
\pgfpathlineto{\pgfqpoint{4.862228in}{4.162724in}}%
\pgfpathlineto{\pgfqpoint{4.865431in}{4.157757in}}%
\pgfpathlineto{\pgfqpoint{4.870237in}{4.142209in}}%
\pgfpathlineto{\pgfqpoint{4.875042in}{4.151830in}}%
\pgfpathlineto{\pgfqpoint{4.878245in}{4.151926in}}%
\pgfpathlineto{\pgfqpoint{4.881449in}{4.146976in}}%
\pgfpathlineto{\pgfqpoint{4.886254in}{4.131492in}}%
\pgfpathlineto{\pgfqpoint{4.891060in}{4.141074in}}%
\pgfpathlineto{\pgfqpoint{4.894263in}{4.141169in}}%
\pgfpathlineto{\pgfqpoint{4.897467in}{4.136236in}}%
\pgfpathlineto{\pgfqpoint{4.902272in}{4.120815in}}%
\pgfpathlineto{\pgfqpoint{4.907077in}{4.130359in}}%
\pgfpathlineto{\pgfqpoint{4.910281in}{4.130452in}}%
\pgfpathlineto{\pgfqpoint{4.913484in}{4.125535in}}%
\pgfpathlineto{\pgfqpoint{4.918290in}{4.110178in}}%
\pgfpathlineto{\pgfqpoint{4.923095in}{4.119684in}}%
\pgfpathlineto{\pgfqpoint{4.926299in}{4.119775in}}%
\pgfpathlineto{\pgfqpoint{4.929502in}{4.114875in}}%
\pgfpathlineto{\pgfqpoint{4.934307in}{4.099581in}}%
\pgfpathlineto{\pgfqpoint{4.939113in}{4.109049in}}%
\pgfpathlineto{\pgfqpoint{4.942316in}{4.109138in}}%
\pgfpathlineto{\pgfqpoint{4.945520in}{4.104255in}}%
\pgfpathlineto{\pgfqpoint{4.950325in}{4.089023in}}%
\pgfpathlineto{\pgfqpoint{4.955131in}{4.098453in}}%
\pgfpathlineto{\pgfqpoint{4.958334in}{4.098540in}}%
\pgfpathlineto{\pgfqpoint{4.961538in}{4.093674in}}%
\pgfpathlineto{\pgfqpoint{4.966343in}{4.078505in}}%
\pgfpathlineto{\pgfqpoint{4.971148in}{4.087898in}}%
\pgfpathlineto{\pgfqpoint{4.974352in}{4.087983in}}%
\pgfpathlineto{\pgfqpoint{4.977555in}{4.083133in}}%
\pgfpathlineto{\pgfqpoint{4.982361in}{4.068026in}}%
\pgfpathlineto{\pgfqpoint{4.987166in}{4.077381in}}%
\pgfpathlineto{\pgfqpoint{4.990370in}{4.077464in}}%
\pgfpathlineto{\pgfqpoint{4.993573in}{4.072632in}}%
\pgfpathlineto{\pgfqpoint{4.998378in}{4.057586in}}%
\pgfpathlineto{\pgfqpoint{5.003184in}{4.066904in}}%
\pgfpathlineto{\pgfqpoint{5.006387in}{4.066985in}}%
\pgfpathlineto{\pgfqpoint{5.009591in}{4.062169in}}%
\pgfpathlineto{\pgfqpoint{5.014396in}{4.047185in}}%
\pgfpathlineto{\pgfqpoint{5.019201in}{4.056466in}}%
\pgfpathlineto{\pgfqpoint{5.022405in}{4.056546in}}%
\pgfpathlineto{\pgfqpoint{5.025609in}{4.051746in}}%
\pgfpathlineto{\pgfqpoint{5.030414in}{4.036824in}}%
\pgfpathlineto{\pgfqpoint{5.035219in}{4.046068in}}%
\pgfpathlineto{\pgfqpoint{5.038423in}{4.046145in}}%
\pgfpathlineto{\pgfqpoint{5.041626in}{4.041362in}}%
\pgfpathlineto{\pgfqpoint{5.046432in}{4.026500in}}%
\pgfpathlineto{\pgfqpoint{5.051237in}{4.035708in}}%
\pgfpathlineto{\pgfqpoint{5.054440in}{4.035783in}}%
\pgfpathlineto{\pgfqpoint{5.057644in}{4.031016in}}%
\pgfpathlineto{\pgfqpoint{5.062449in}{4.016216in}}%
\pgfpathlineto{\pgfqpoint{5.067255in}{4.025386in}}%
\pgfpathlineto{\pgfqpoint{5.070458in}{4.025460in}}%
\pgfpathlineto{\pgfqpoint{5.073662in}{4.020709in}}%
\pgfpathlineto{\pgfqpoint{5.078467in}{4.005970in}}%
\pgfpathlineto{\pgfqpoint{5.083272in}{4.015104in}}%
\pgfpathlineto{\pgfqpoint{5.086476in}{4.015176in}}%
\pgfpathlineto{\pgfqpoint{5.089680in}{4.010441in}}%
\pgfpathlineto{\pgfqpoint{5.094485in}{3.995762in}}%
\pgfpathlineto{\pgfqpoint{5.099290in}{4.004859in}}%
\pgfpathlineto{\pgfqpoint{5.102494in}{4.004930in}}%
\pgfpathlineto{\pgfqpoint{5.105697in}{4.000211in}}%
\pgfpathlineto{\pgfqpoint{5.110503in}{3.985592in}}%
\pgfpathlineto{\pgfqpoint{5.115308in}{3.994653in}}%
\pgfpathlineto{\pgfqpoint{5.118511in}{3.994722in}}%
\pgfpathlineto{\pgfqpoint{5.121715in}{3.990020in}}%
\pgfpathlineto{\pgfqpoint{5.126520in}{3.975460in}}%
\pgfpathlineto{\pgfqpoint{5.131326in}{3.984486in}}%
\pgfpathlineto{\pgfqpoint{5.134529in}{3.984552in}}%
\pgfpathlineto{\pgfqpoint{5.137733in}{3.979866in}}%
\pgfpathlineto{\pgfqpoint{5.142538in}{3.965367in}}%
\pgfpathlineto{\pgfqpoint{5.147343in}{3.974356in}}%
\pgfpathlineto{\pgfqpoint{5.150547in}{3.974421in}}%
\pgfpathlineto{\pgfqpoint{5.153750in}{3.969751in}}%
\pgfpathlineto{\pgfqpoint{5.158556in}{3.955310in}}%
\pgfpathlineto{\pgfqpoint{5.163361in}{3.964264in}}%
\pgfpathlineto{\pgfqpoint{5.166565in}{3.964327in}}%
\pgfpathlineto{\pgfqpoint{5.169768in}{3.959673in}}%
\pgfpathlineto{\pgfqpoint{5.174574in}{3.945292in}}%
\pgfpathlineto{\pgfqpoint{5.179379in}{3.954210in}}%
\pgfpathlineto{\pgfqpoint{5.182582in}{3.954271in}}%
\pgfpathlineto{\pgfqpoint{5.185786in}{3.949633in}}%
\pgfpathlineto{\pgfqpoint{5.190591in}{3.935311in}}%
\pgfpathlineto{\pgfqpoint{5.195397in}{3.944193in}}%
\pgfpathlineto{\pgfqpoint{5.198600in}{3.944253in}}%
\pgfpathlineto{\pgfqpoint{5.201804in}{3.939631in}}%
\pgfpathlineto{\pgfqpoint{5.206609in}{3.925367in}}%
\pgfpathlineto{\pgfqpoint{5.211414in}{3.934214in}}%
\pgfpathlineto{\pgfqpoint{5.214618in}{3.934272in}}%
\pgfpathlineto{\pgfqpoint{5.217821in}{3.929666in}}%
\pgfpathlineto{\pgfqpoint{5.222627in}{3.915460in}}%
\pgfpathlineto{\pgfqpoint{5.227432in}{3.924272in}}%
\pgfpathlineto{\pgfqpoint{5.230636in}{3.924328in}}%
\pgfpathlineto{\pgfqpoint{5.233839in}{3.919738in}}%
\pgfpathlineto{\pgfqpoint{5.238644in}{3.905591in}}%
\pgfpathlineto{\pgfqpoint{5.243450in}{3.914367in}}%
\pgfpathlineto{\pgfqpoint{5.246653in}{3.914422in}}%
\pgfpathlineto{\pgfqpoint{5.249857in}{3.909847in}}%
\pgfpathlineto{\pgfqpoint{5.254662in}{3.895758in}}%
\pgfpathlineto{\pgfqpoint{5.259468in}{3.904499in}}%
\pgfpathlineto{\pgfqpoint{5.262671in}{3.904553in}}%
\pgfpathlineto{\pgfqpoint{5.265875in}{3.899993in}}%
\pgfpathlineto{\pgfqpoint{5.270680in}{3.885962in}}%
\pgfpathlineto{\pgfqpoint{5.275485in}{3.894669in}}%
\pgfpathlineto{\pgfqpoint{5.278689in}{3.894720in}}%
\pgfpathlineto{\pgfqpoint{5.281892in}{3.890177in}}%
\pgfpathlineto{\pgfqpoint{5.286698in}{3.876203in}}%
\pgfpathlineto{\pgfqpoint{5.291503in}{3.884875in}}%
\pgfpathlineto{\pgfqpoint{5.294707in}{3.884925in}}%
\pgfpathlineto{\pgfqpoint{5.297910in}{3.880396in}}%
\pgfpathlineto{\pgfqpoint{5.302715in}{3.866480in}}%
\pgfpathlineto{\pgfqpoint{5.307521in}{3.875117in}}%
\pgfpathlineto{\pgfqpoint{5.310724in}{3.875165in}}%
\pgfpathlineto{\pgfqpoint{5.313928in}{3.870653in}}%
\pgfpathlineto{\pgfqpoint{5.318733in}{3.856793in}}%
\pgfpathlineto{\pgfqpoint{5.323538in}{3.865396in}}%
\pgfpathlineto{\pgfqpoint{5.326742in}{3.865443in}}%
\pgfpathlineto{\pgfqpoint{5.329946in}{3.860946in}}%
\pgfpathlineto{\pgfqpoint{5.334751in}{3.847143in}}%
\pgfpathlineto{\pgfqpoint{5.339556in}{3.855711in}}%
\pgfpathlineto{\pgfqpoint{5.342760in}{3.855757in}}%
\pgfpathlineto{\pgfqpoint{5.345963in}{3.851275in}}%
\pgfpathlineto{\pgfqpoint{5.350769in}{3.837529in}}%
\pgfpathlineto{\pgfqpoint{5.355574in}{3.846063in}}%
\pgfpathlineto{\pgfqpoint{5.358777in}{3.846107in}}%
\pgfpathlineto{\pgfqpoint{5.361981in}{3.841640in}}%
\pgfpathlineto{\pgfqpoint{5.366786in}{3.827950in}}%
\pgfpathlineto{\pgfqpoint{5.371592in}{3.836451in}}%
\pgfpathlineto{\pgfqpoint{5.374795in}{3.836493in}}%
\pgfpathlineto{\pgfqpoint{5.377999in}{3.832041in}}%
\pgfpathlineto{\pgfqpoint{5.382804in}{3.818408in}}%
\pgfpathlineto{\pgfqpoint{5.387609in}{3.826874in}}%
\pgfpathlineto{\pgfqpoint{5.390813in}{3.826915in}}%
\pgfpathlineto{\pgfqpoint{5.394016in}{3.822479in}}%
\pgfpathlineto{\pgfqpoint{5.398822in}{3.808901in}}%
\pgfpathlineto{\pgfqpoint{5.403627in}{3.817333in}}%
\pgfpathlineto{\pgfqpoint{5.406831in}{3.817372in}}%
\pgfpathlineto{\pgfqpoint{5.410034in}{3.812952in}}%
\pgfpathlineto{\pgfqpoint{5.414840in}{3.799429in}}%
\pgfpathlineto{\pgfqpoint{5.419645in}{3.807828in}}%
\pgfpathlineto{\pgfqpoint{5.422848in}{3.807866in}}%
\pgfpathlineto{\pgfqpoint{5.426052in}{3.803460in}}%
\pgfpathlineto{\pgfqpoint{5.430857in}{3.789993in}}%
\pgfpathlineto{\pgfqpoint{5.435663in}{3.798359in}}%
\pgfpathlineto{\pgfqpoint{5.438866in}{3.798395in}}%
\pgfpathlineto{\pgfqpoint{5.442070in}{3.794004in}}%
\pgfpathlineto{\pgfqpoint{5.446875in}{3.780593in}}%
\pgfpathlineto{\pgfqpoint{5.451680in}{3.788925in}}%
\pgfpathlineto{\pgfqpoint{5.454884in}{3.788959in}}%
\pgfpathlineto{\pgfqpoint{5.458087in}{3.784584in}}%
\pgfpathlineto{\pgfqpoint{5.462893in}{3.771227in}}%
\pgfpathlineto{\pgfqpoint{5.467698in}{3.779526in}}%
\pgfpathlineto{\pgfqpoint{5.470902in}{3.779559in}}%
\pgfpathlineto{\pgfqpoint{5.474105in}{3.775198in}}%
\pgfpathlineto{\pgfqpoint{5.478910in}{3.761897in}}%
\pgfpathlineto{\pgfqpoint{5.483716in}{3.770162in}}%
\pgfpathlineto{\pgfqpoint{5.486919in}{3.770194in}}%
\pgfpathlineto{\pgfqpoint{5.490123in}{3.765848in}}%
\pgfpathlineto{\pgfqpoint{5.494928in}{3.752601in}}%
\pgfpathlineto{\pgfqpoint{5.499734in}{3.760834in}}%
\pgfpathlineto{\pgfqpoint{5.502937in}{3.760864in}}%
\pgfpathlineto{\pgfqpoint{5.506141in}{3.756533in}}%
\pgfpathlineto{\pgfqpoint{5.510946in}{3.743340in}}%
\pgfpathlineto{\pgfqpoint{5.515751in}{3.751540in}}%
\pgfpathlineto{\pgfqpoint{5.518955in}{3.751568in}}%
\pgfpathlineto{\pgfqpoint{5.522158in}{3.747252in}}%
\pgfpathlineto{\pgfqpoint{5.526964in}{3.734114in}}%
\pgfpathlineto{\pgfqpoint{5.531769in}{3.742281in}}%
\pgfpathlineto{\pgfqpoint{5.534973in}{3.742308in}}%
\pgfpathlineto{\pgfqpoint{5.538176in}{3.738007in}}%
\pgfpathlineto{\pgfqpoint{5.542981in}{3.724922in}}%
\pgfpathlineto{\pgfqpoint{5.547787in}{3.733056in}}%
\pgfpathlineto{\pgfqpoint{5.550990in}{3.733082in}}%
\pgfpathlineto{\pgfqpoint{5.554194in}{3.728796in}}%
\pgfpathlineto{\pgfqpoint{5.558999in}{3.715764in}}%
\pgfpathlineto{\pgfqpoint{5.563804in}{3.723867in}}%
\pgfpathlineto{\pgfqpoint{5.567008in}{3.723891in}}%
\pgfpathlineto{\pgfqpoint{5.570212in}{3.719619in}}%
\pgfpathlineto{\pgfqpoint{5.575017in}{3.706641in}}%
\pgfpathlineto{\pgfqpoint{5.579822in}{3.714711in}}%
\pgfpathlineto{\pgfqpoint{5.583026in}{3.714733in}}%
\pgfpathlineto{\pgfqpoint{5.586229in}{3.710476in}}%
\pgfpathlineto{\pgfqpoint{5.591035in}{3.697552in}}%
\pgfpathlineto{\pgfqpoint{5.595840in}{3.705589in}}%
\pgfpathlineto{\pgfqpoint{5.599044in}{3.705611in}}%
\pgfpathlineto{\pgfqpoint{5.602247in}{3.701368in}}%
\pgfpathlineto{\pgfqpoint{5.607052in}{3.688496in}}%
\pgfpathlineto{\pgfqpoint{5.611858in}{3.696502in}}%
\pgfpathlineto{\pgfqpoint{5.615061in}{3.696522in}}%
\pgfpathlineto{\pgfqpoint{5.618265in}{3.692294in}}%
\pgfpathlineto{\pgfqpoint{5.623070in}{3.679475in}}%
\pgfpathlineto{\pgfqpoint{5.627875in}{3.687449in}}%
\pgfpathlineto{\pgfqpoint{5.631079in}{3.687467in}}%
\pgfpathlineto{\pgfqpoint{5.634283in}{3.683254in}}%
\pgfpathlineto{\pgfqpoint{5.639088in}{3.670487in}}%
\pgfpathlineto{\pgfqpoint{5.643893in}{3.678429in}}%
\pgfpathlineto{\pgfqpoint{5.647097in}{3.678446in}}%
\pgfpathlineto{\pgfqpoint{5.650300in}{3.674247in}}%
\pgfpathlineto{\pgfqpoint{5.655106in}{3.661533in}}%
\pgfpathlineto{\pgfqpoint{5.659911in}{3.669444in}}%
\pgfpathlineto{\pgfqpoint{5.663114in}{3.669459in}}%
\pgfpathlineto{\pgfqpoint{5.666318in}{3.665274in}}%
\pgfpathlineto{\pgfqpoint{5.671123in}{3.652612in}}%
\pgfpathlineto{\pgfqpoint{5.675929in}{3.660491in}}%
\pgfpathlineto{\pgfqpoint{5.679132in}{3.660506in}}%
\pgfpathlineto{\pgfqpoint{5.682336in}{3.656335in}}%
\pgfpathlineto{\pgfqpoint{5.687141in}{3.643725in}}%
\pgfpathlineto{\pgfqpoint{5.691946in}{3.651573in}}%
\pgfpathlineto{\pgfqpoint{5.695150in}{3.651586in}}%
\pgfpathlineto{\pgfqpoint{5.698353in}{3.647429in}}%
\pgfpathlineto{\pgfqpoint{5.703159in}{3.634871in}}%
\pgfpathlineto{\pgfqpoint{5.707964in}{3.642687in}}%
\pgfpathlineto{\pgfqpoint{5.711168in}{3.642699in}}%
\pgfpathlineto{\pgfqpoint{5.714371in}{3.638557in}}%
\pgfpathlineto{\pgfqpoint{5.719177in}{3.626050in}}%
\pgfpathlineto{\pgfqpoint{5.723982in}{3.633835in}}%
\pgfpathlineto{\pgfqpoint{5.727185in}{3.633845in}}%
\pgfpathlineto{\pgfqpoint{5.730389in}{3.629718in}}%
\pgfpathlineto{\pgfqpoint{5.735194in}{3.617262in}}%
\pgfpathlineto{\pgfqpoint{5.740000in}{3.625016in}}%
\pgfpathlineto{\pgfqpoint{5.743203in}{3.625025in}}%
\pgfpathlineto{\pgfqpoint{5.746407in}{3.620911in}}%
\pgfpathlineto{\pgfqpoint{5.751212in}{3.608507in}}%
\pgfpathlineto{\pgfqpoint{5.756017in}{3.616230in}}%
\pgfpathlineto{\pgfqpoint{5.759221in}{3.616237in}}%
\pgfpathlineto{\pgfqpoint{5.762424in}{3.612138in}}%
\pgfpathlineto{\pgfqpoint{5.767230in}{3.599784in}}%
\pgfpathlineto{\pgfqpoint{5.772035in}{3.607477in}}%
\pgfpathlineto{\pgfqpoint{5.775239in}{3.607483in}}%
\pgfpathlineto{\pgfqpoint{5.778442in}{3.603397in}}%
\pgfpathlineto{\pgfqpoint{5.783247in}{3.591095in}}%
\pgfpathlineto{\pgfqpoint{5.788053in}{3.598756in}}%
\pgfpathlineto{\pgfqpoint{5.791256in}{3.598761in}}%
\pgfpathlineto{\pgfqpoint{5.794460in}{3.594690in}}%
\pgfpathlineto{\pgfqpoint{5.799265in}{3.582437in}}%
\pgfpathlineto{\pgfqpoint{5.804071in}{3.590068in}}%
\pgfpathlineto{\pgfqpoint{5.807274in}{3.590072in}}%
\pgfpathlineto{\pgfqpoint{5.810478in}{3.586014in}}%
\pgfpathlineto{\pgfqpoint{5.815283in}{3.573812in}}%
\pgfpathlineto{\pgfqpoint{5.820088in}{3.581413in}}%
\pgfpathlineto{\pgfqpoint{5.823292in}{3.581415in}}%
\pgfpathlineto{\pgfqpoint{5.826495in}{3.577371in}}%
\pgfpathlineto{\pgfqpoint{5.831301in}{3.565219in}}%
\pgfpathlineto{\pgfqpoint{5.836106in}{3.572790in}}%
\pgfpathlineto{\pgfqpoint{5.839310in}{3.572791in}}%
\pgfpathlineto{\pgfqpoint{5.842513in}{3.568761in}}%
\pgfpathlineto{\pgfqpoint{5.847318in}{3.556659in}}%
\pgfpathlineto{\pgfqpoint{5.852124in}{3.564199in}}%
\pgfpathlineto{\pgfqpoint{5.855327in}{3.564198in}}%
\pgfpathlineto{\pgfqpoint{5.858531in}{3.560182in}}%
\pgfpathlineto{\pgfqpoint{5.863336in}{3.548130in}}%
\pgfpathlineto{\pgfqpoint{5.868141in}{3.555640in}}%
\pgfpathlineto{\pgfqpoint{5.871345in}{3.555638in}}%
\pgfpathlineto{\pgfqpoint{5.874549in}{3.551636in}}%
\pgfpathlineto{\pgfqpoint{5.879354in}{3.539633in}}%
\pgfpathlineto{\pgfqpoint{5.884159in}{3.547113in}}%
\pgfpathlineto{\pgfqpoint{5.887363in}{3.547110in}}%
\pgfpathlineto{\pgfqpoint{5.890566in}{3.543122in}}%
\pgfpathlineto{\pgfqpoint{5.895372in}{3.531168in}}%
\pgfpathlineto{\pgfqpoint{5.900177in}{3.538618in}}%
\pgfpathlineto{\pgfqpoint{5.903381in}{3.538614in}}%
\pgfpathlineto{\pgfqpoint{5.906584in}{3.534639in}}%
\pgfpathlineto{\pgfqpoint{5.911389in}{3.522734in}}%
\pgfpathlineto{\pgfqpoint{5.916195in}{3.530155in}}%
\pgfpathlineto{\pgfqpoint{5.919398in}{3.530150in}}%
\pgfpathlineto{\pgfqpoint{5.922602in}{3.526189in}}%
\pgfpathlineto{\pgfqpoint{5.927407in}{3.514333in}}%
\pgfpathlineto{\pgfqpoint{5.932212in}{3.521723in}}%
\pgfpathlineto{\pgfqpoint{5.935416in}{3.521717in}}%
\pgfpathlineto{\pgfqpoint{5.938620in}{3.517769in}}%
\pgfpathlineto{\pgfqpoint{5.943425in}{3.505962in}}%
\pgfpathlineto{\pgfqpoint{5.948230in}{3.513323in}}%
\pgfpathlineto{\pgfqpoint{5.951434in}{3.513316in}}%
\pgfpathlineto{\pgfqpoint{5.954637in}{3.509382in}}%
\pgfpathlineto{\pgfqpoint{5.959443in}{3.497623in}}%
\pgfpathlineto{\pgfqpoint{5.964248in}{3.504955in}}%
\pgfpathlineto{\pgfqpoint{5.967451in}{3.504946in}}%
\pgfpathlineto{\pgfqpoint{5.970655in}{3.501025in}}%
\pgfpathlineto{\pgfqpoint{5.975460in}{3.489315in}}%
\pgfpathlineto{\pgfqpoint{5.980266in}{3.496618in}}%
\pgfpathlineto{\pgfqpoint{5.983469in}{3.496607in}}%
\pgfpathlineto{\pgfqpoint{5.986673in}{3.492700in}}%
\pgfpathlineto{\pgfqpoint{5.991478in}{3.481038in}}%
\pgfpathlineto{\pgfqpoint{5.996283in}{3.488311in}}%
\pgfpathlineto{\pgfqpoint{5.999487in}{3.488300in}}%
\pgfpathlineto{\pgfqpoint{6.002690in}{3.484406in}}%
\pgfpathlineto{\pgfqpoint{6.007496in}{3.472792in}}%
\pgfpathlineto{\pgfqpoint{6.012301in}{3.480036in}}%
\pgfpathlineto{\pgfqpoint{6.015505in}{3.480024in}}%
\pgfpathlineto{\pgfqpoint{6.018708in}{3.476143in}}%
\pgfpathlineto{\pgfqpoint{6.023514in}{3.464576in}}%
\pgfpathlineto{\pgfqpoint{6.028319in}{3.471792in}}%
\pgfpathlineto{\pgfqpoint{6.031522in}{3.471778in}}%
\pgfpathlineto{\pgfqpoint{6.034726in}{3.467911in}}%
\pgfpathlineto{\pgfqpoint{6.039531in}{3.456392in}}%
\pgfpathlineto{\pgfqpoint{6.044337in}{3.463579in}}%
\pgfpathlineto{\pgfqpoint{6.047540in}{3.463564in}}%
\pgfpathlineto{\pgfqpoint{6.050744in}{3.459710in}}%
\pgfpathlineto{\pgfqpoint{6.055549in}{3.448238in}}%
\pgfpathlineto{\pgfqpoint{6.060354in}{3.455396in}}%
\pgfpathlineto{\pgfqpoint{6.063558in}{3.455380in}}%
\pgfpathlineto{\pgfqpoint{6.066761in}{3.451540in}}%
\pgfpathlineto{\pgfqpoint{6.071567in}{3.440114in}}%
\pgfpathlineto{\pgfqpoint{6.076372in}{3.447244in}}%
\pgfpathlineto{\pgfqpoint{6.079576in}{3.447227in}}%
\pgfpathlineto{\pgfqpoint{6.082779in}{3.443399in}}%
\pgfpathlineto{\pgfqpoint{6.087584in}{3.432021in}}%
\pgfpathlineto{\pgfqpoint{6.092390in}{3.439122in}}%
\pgfpathlineto{\pgfqpoint{6.095593in}{3.439104in}}%
\pgfpathlineto{\pgfqpoint{6.098797in}{3.435290in}}%
\pgfpathlineto{\pgfqpoint{6.103602in}{3.423958in}}%
\pgfpathlineto{\pgfqpoint{6.108408in}{3.431031in}}%
\pgfpathlineto{\pgfqpoint{6.111611in}{3.431012in}}%
\pgfpathlineto{\pgfqpoint{6.114815in}{3.427211in}}%
\pgfpathlineto{\pgfqpoint{6.119620in}{3.415925in}}%
\pgfpathlineto{\pgfqpoint{6.124425in}{3.422970in}}%
\pgfpathlineto{\pgfqpoint{6.127629in}{3.422950in}}%
\pgfpathlineto{\pgfqpoint{6.130832in}{3.419161in}}%
\pgfpathlineto{\pgfqpoint{6.135638in}{3.407923in}}%
\pgfpathlineto{\pgfqpoint{6.140443in}{3.414939in}}%
\pgfpathlineto{\pgfqpoint{6.143647in}{3.414918in}}%
\pgfpathlineto{\pgfqpoint{6.146850in}{3.411142in}}%
\pgfpathlineto{\pgfqpoint{6.151655in}{3.399950in}}%
\pgfpathlineto{\pgfqpoint{6.156461in}{3.406938in}}%
\pgfpathlineto{\pgfqpoint{6.159664in}{3.406916in}}%
\pgfpathlineto{\pgfqpoint{6.162868in}{3.403153in}}%
\pgfpathlineto{\pgfqpoint{6.167673in}{3.392007in}}%
\pgfpathlineto{\pgfqpoint{6.172478in}{3.398967in}}%
\pgfpathlineto{\pgfqpoint{6.175682in}{3.398944in}}%
\pgfpathlineto{\pgfqpoint{6.178886in}{3.395194in}}%
\pgfpathlineto{\pgfqpoint{6.183691in}{3.384093in}}%
\pgfpathlineto{\pgfqpoint{6.188496in}{3.391026in}}%
\pgfpathlineto{\pgfqpoint{6.191700in}{3.391001in}}%
\pgfpathlineto{\pgfqpoint{6.194903in}{3.387265in}}%
\pgfpathlineto{\pgfqpoint{6.199709in}{3.376209in}}%
\pgfpathlineto{\pgfqpoint{6.204514in}{3.383115in}}%
\pgfpathlineto{\pgfqpoint{6.207717in}{3.383089in}}%
\pgfpathlineto{\pgfqpoint{6.210921in}{3.379365in}}%
\pgfpathlineto{\pgfqpoint{6.215726in}{3.368355in}}%
\pgfpathlineto{\pgfqpoint{6.220532in}{3.375233in}}%
\pgfpathlineto{\pgfqpoint{6.223735in}{3.375206in}}%
\pgfpathlineto{\pgfqpoint{6.226939in}{3.371495in}}%
\pgfpathlineto{\pgfqpoint{6.231744in}{3.360530in}}%
\pgfpathlineto{\pgfqpoint{6.236549in}{3.367380in}}%
\pgfpathlineto{\pgfqpoint{6.239753in}{3.367353in}}%
\pgfpathlineto{\pgfqpoint{6.242957in}{3.363654in}}%
\pgfpathlineto{\pgfqpoint{6.247762in}{3.352734in}}%
\pgfpathlineto{\pgfqpoint{6.252567in}{3.359557in}}%
\pgfpathlineto{\pgfqpoint{6.255771in}{3.359528in}}%
\pgfpathlineto{\pgfqpoint{6.258974in}{3.355843in}}%
\pgfpathlineto{\pgfqpoint{6.263780in}{3.344968in}}%
\pgfpathlineto{\pgfqpoint{6.268585in}{3.351764in}}%
\pgfpathlineto{\pgfqpoint{6.271788in}{3.351734in}}%
\pgfpathlineto{\pgfqpoint{6.274992in}{3.348061in}}%
\pgfpathlineto{\pgfqpoint{6.279797in}{3.337230in}}%
\pgfpathlineto{\pgfqpoint{6.284603in}{3.343999in}}%
\pgfpathlineto{\pgfqpoint{6.287806in}{3.343968in}}%
\pgfpathlineto{\pgfqpoint{6.291010in}{3.340307in}}%
\pgfpathlineto{\pgfqpoint{6.295815in}{3.329521in}}%
\pgfpathlineto{\pgfqpoint{6.300620in}{3.336263in}}%
\pgfpathlineto{\pgfqpoint{6.303824in}{3.336231in}}%
\pgfpathlineto{\pgfqpoint{6.307027in}{3.332583in}}%
\pgfpathlineto{\pgfqpoint{6.311833in}{3.321842in}}%
\pgfpathlineto{\pgfqpoint{6.316638in}{3.328556in}}%
\pgfpathlineto{\pgfqpoint{6.319842in}{3.328523in}}%
\pgfpathlineto{\pgfqpoint{6.323045in}{3.324888in}}%
\pgfpathlineto{\pgfqpoint{6.327851in}{3.314190in}}%
\pgfpathlineto{\pgfqpoint{6.332656in}{3.320878in}}%
\pgfpathlineto{\pgfqpoint{6.335859in}{3.320844in}}%
\pgfpathlineto{\pgfqpoint{6.339063in}{3.317222in}}%
\pgfpathlineto{\pgfqpoint{6.343868in}{3.306568in}}%
\pgfpathlineto{\pgfqpoint{6.348674in}{3.313229in}}%
\pgfpathlineto{\pgfqpoint{6.351877in}{3.313194in}}%
\pgfpathlineto{\pgfqpoint{6.355081in}{3.309584in}}%
\pgfpathlineto{\pgfqpoint{6.359886in}{3.298974in}}%
\pgfpathlineto{\pgfqpoint{6.364691in}{3.305608in}}%
\pgfpathlineto{\pgfqpoint{6.367895in}{3.305572in}}%
\pgfpathlineto{\pgfqpoint{6.371098in}{3.301974in}}%
\pgfpathlineto{\pgfqpoint{6.375904in}{3.291408in}}%
\pgfpathlineto{\pgfqpoint{6.380709in}{3.298016in}}%
\pgfpathlineto{\pgfqpoint{6.383913in}{3.297979in}}%
\pgfpathlineto{\pgfqpoint{6.387116in}{3.294394in}}%
\pgfpathlineto{\pgfqpoint{6.391921in}{3.283871in}}%
\pgfpathlineto{\pgfqpoint{6.396727in}{3.290453in}}%
\pgfpathlineto{\pgfqpoint{6.399930in}{3.290415in}}%
\pgfpathlineto{\pgfqpoint{6.403134in}{3.286841in}}%
\pgfpathlineto{\pgfqpoint{6.407939in}{3.276361in}}%
\pgfpathlineto{\pgfqpoint{6.412745in}{3.282917in}}%
\pgfpathlineto{\pgfqpoint{6.415948in}{3.282878in}}%
\pgfpathlineto{\pgfqpoint{6.419152in}{3.279317in}}%
\pgfpathlineto{\pgfqpoint{6.423957in}{3.268880in}}%
\pgfpathlineto{\pgfqpoint{6.428762in}{3.275410in}}%
\pgfpathlineto{\pgfqpoint{6.431966in}{3.275370in}}%
\pgfpathlineto{\pgfqpoint{6.435169in}{3.271821in}}%
\pgfpathlineto{\pgfqpoint{6.439975in}{3.261427in}}%
\pgfpathlineto{\pgfqpoint{6.444780in}{3.267930in}}%
\pgfpathlineto{\pgfqpoint{6.447984in}{3.267889in}}%
\pgfpathlineto{\pgfqpoint{6.451187in}{3.264353in}}%
\pgfpathlineto{\pgfqpoint{6.455992in}{3.254002in}}%
\pgfpathlineto{\pgfqpoint{6.460798in}{3.260479in}}%
\pgfpathlineto{\pgfqpoint{6.464001in}{3.260437in}}%
\pgfpathlineto{\pgfqpoint{6.467205in}{3.256913in}}%
\pgfpathlineto{\pgfqpoint{6.472010in}{3.246604in}}%
\pgfpathlineto{\pgfqpoint{6.476815in}{3.253056in}}%
\pgfpathlineto{\pgfqpoint{6.480019in}{3.253013in}}%
\pgfpathlineto{\pgfqpoint{6.483223in}{3.249500in}}%
\pgfpathlineto{\pgfqpoint{6.488028in}{3.239234in}}%
\pgfpathlineto{\pgfqpoint{6.492833in}{3.245660in}}%
\pgfpathlineto{\pgfqpoint{6.496037in}{3.245616in}}%
\pgfpathlineto{\pgfqpoint{6.499240in}{3.242116in}}%
\pgfpathlineto{\pgfqpoint{6.504046in}{3.231891in}}%
\pgfpathlineto{\pgfqpoint{6.508851in}{3.238292in}}%
\pgfpathlineto{\pgfqpoint{6.512054in}{3.238247in}}%
\pgfpathlineto{\pgfqpoint{6.515258in}{3.234759in}}%
\pgfpathlineto{\pgfqpoint{6.520063in}{3.224576in}}%
\pgfpathlineto{\pgfqpoint{6.524869in}{3.230951in}}%
\pgfpathlineto{\pgfqpoint{6.528072in}{3.230906in}}%
\pgfpathlineto{\pgfqpoint{6.531276in}{3.227429in}}%
\pgfpathlineto{\pgfqpoint{6.536081in}{3.217289in}}%
\pgfpathlineto{\pgfqpoint{6.540886in}{3.223638in}}%
\pgfpathlineto{\pgfqpoint{6.544090in}{3.223592in}}%
\pgfpathlineto{\pgfqpoint{6.547293in}{3.220127in}}%
\pgfpathlineto{\pgfqpoint{6.552099in}{3.210028in}}%
\pgfpathlineto{\pgfqpoint{6.556904in}{3.216352in}}%
\pgfpathlineto{\pgfqpoint{6.560108in}{3.216305in}}%
\pgfpathlineto{\pgfqpoint{6.563311in}{3.212852in}}%
\pgfpathlineto{\pgfqpoint{6.568117in}{3.202795in}}%
\pgfpathlineto{\pgfqpoint{6.572922in}{3.209094in}}%
\pgfpathlineto{\pgfqpoint{6.576125in}{3.209046in}}%
\pgfpathlineto{\pgfqpoint{6.579329in}{3.205605in}}%
\pgfpathlineto{\pgfqpoint{6.584134in}{3.195589in}}%
\pgfpathlineto{\pgfqpoint{6.588940in}{3.201862in}}%
\pgfpathlineto{\pgfqpoint{6.592143in}{3.201813in}}%
\pgfpathlineto{\pgfqpoint{6.595347in}{3.198384in}}%
\pgfpathlineto{\pgfqpoint{6.600152in}{3.188409in}}%
\pgfpathlineto{\pgfqpoint{6.604957in}{3.194658in}}%
\pgfpathlineto{\pgfqpoint{6.608161in}{3.194608in}}%
\pgfpathlineto{\pgfqpoint{6.611364in}{3.191191in}}%
\pgfpathlineto{\pgfqpoint{6.616170in}{3.181257in}}%
\pgfpathlineto{\pgfqpoint{6.620975in}{3.187480in}}%
\pgfpathlineto{\pgfqpoint{6.624179in}{3.187430in}}%
\pgfpathlineto{\pgfqpoint{6.627382in}{3.184024in}}%
\pgfpathlineto{\pgfqpoint{6.632187in}{3.174131in}}%
\pgfpathlineto{\pgfqpoint{6.636993in}{3.180330in}}%
\pgfpathlineto{\pgfqpoint{6.640196in}{3.180278in}}%
\pgfpathlineto{\pgfqpoint{6.643400in}{3.176884in}}%
\pgfpathlineto{\pgfqpoint{6.648205in}{3.167032in}}%
\pgfpathlineto{\pgfqpoint{6.653011in}{3.173206in}}%
\pgfpathlineto{\pgfqpoint{6.656214in}{3.173153in}}%
\pgfpathlineto{\pgfqpoint{6.659418in}{3.169771in}}%
\pgfpathlineto{\pgfqpoint{6.664223in}{3.159959in}}%
\pgfpathlineto{\pgfqpoint{6.669028in}{3.166109in}}%
\pgfpathlineto{\pgfqpoint{6.672232in}{3.166055in}}%
\pgfpathlineto{\pgfqpoint{6.675435in}{3.162685in}}%
\pgfpathlineto{\pgfqpoint{6.680241in}{3.152913in}}%
\pgfpathlineto{\pgfqpoint{6.685046in}{3.159038in}}%
\pgfpathlineto{\pgfqpoint{6.688250in}{3.158984in}}%
\pgfpathlineto{\pgfqpoint{6.691453in}{3.155625in}}%
\pgfpathlineto{\pgfqpoint{6.696258in}{3.145893in}}%
\pgfpathlineto{\pgfqpoint{6.701064in}{3.151994in}}%
\pgfpathlineto{\pgfqpoint{6.704267in}{3.151939in}}%
\pgfpathlineto{\pgfqpoint{6.707471in}{3.148591in}}%
\pgfpathlineto{\pgfqpoint{6.712276in}{3.138900in}}%
\pgfpathlineto{\pgfqpoint{6.717081in}{3.144976in}}%
\pgfpathlineto{\pgfqpoint{6.720285in}{3.144920in}}%
\pgfpathlineto{\pgfqpoint{6.723489in}{3.141584in}}%
\pgfpathlineto{\pgfqpoint{6.728294in}{3.131932in}}%
\pgfpathlineto{\pgfqpoint{6.733099in}{3.137984in}}%
\pgfpathlineto{\pgfqpoint{6.736303in}{3.137927in}}%
\pgfpathlineto{\pgfqpoint{6.739506in}{3.134603in}}%
\pgfpathlineto{\pgfqpoint{6.744312in}{3.124991in}}%
\pgfpathlineto{\pgfqpoint{6.749117in}{3.131018in}}%
\pgfpathlineto{\pgfqpoint{6.752321in}{3.130961in}}%
\pgfpathlineto{\pgfqpoint{6.755524in}{3.127648in}}%
\pgfpathlineto{\pgfqpoint{6.760329in}{3.118075in}}%
\pgfpathlineto{\pgfqpoint{6.765135in}{3.124079in}}%
\pgfpathlineto{\pgfqpoint{6.768338in}{3.124021in}}%
\pgfpathlineto{\pgfqpoint{6.771542in}{3.120719in}}%
\pgfpathlineto{\pgfqpoint{6.776347in}{3.111186in}}%
\pgfpathlineto{\pgfqpoint{6.781152in}{3.117165in}}%
\pgfpathlineto{\pgfqpoint{6.784356in}{3.117106in}}%
\pgfpathlineto{\pgfqpoint{6.787560in}{3.113816in}}%
\pgfpathlineto{\pgfqpoint{6.792365in}{3.104322in}}%
\pgfpathlineto{\pgfqpoint{6.797170in}{3.110277in}}%
\pgfpathlineto{\pgfqpoint{6.800374in}{3.110218in}}%
\pgfpathlineto{\pgfqpoint{6.803577in}{3.106938in}}%
\pgfpathlineto{\pgfqpoint{6.808383in}{3.097484in}}%
\pgfpathlineto{\pgfqpoint{6.813188in}{3.103415in}}%
\pgfpathlineto{\pgfqpoint{6.816391in}{3.103355in}}%
\pgfpathlineto{\pgfqpoint{6.819595in}{3.100087in}}%
\pgfpathlineto{\pgfqpoint{6.824400in}{3.090671in}}%
\pgfpathlineto{\pgfqpoint{6.829206in}{3.096579in}}%
\pgfpathlineto{\pgfqpoint{6.832409in}{3.096518in}}%
\pgfpathlineto{\pgfqpoint{6.835613in}{3.093261in}}%
\pgfpathlineto{\pgfqpoint{6.840418in}{3.083884in}}%
\pgfpathlineto{\pgfqpoint{6.845223in}{3.089768in}}%
\pgfpathlineto{\pgfqpoint{6.848427in}{3.089706in}}%
\pgfpathlineto{\pgfqpoint{6.851630in}{3.086460in}}%
\pgfpathlineto{\pgfqpoint{6.856436in}{3.077122in}}%
\pgfpathlineto{\pgfqpoint{6.861241in}{3.082983in}}%
\pgfpathlineto{\pgfqpoint{6.864445in}{3.082920in}}%
\pgfpathlineto{\pgfqpoint{6.867648in}{3.079685in}}%
\pgfpathlineto{\pgfqpoint{6.872454in}{3.070385in}}%
\pgfpathlineto{\pgfqpoint{6.877259in}{3.076223in}}%
\pgfpathlineto{\pgfqpoint{6.880462in}{3.076159in}}%
\pgfpathlineto{\pgfqpoint{6.883666in}{3.072936in}}%
\pgfpathlineto{\pgfqpoint{6.888471in}{3.063674in}}%
\pgfpathlineto{\pgfqpoint{6.893277in}{3.069488in}}%
\pgfpathlineto{\pgfqpoint{6.896480in}{3.069424in}}%
\pgfpathlineto{\pgfqpoint{6.899684in}{3.066211in}}%
\pgfpathlineto{\pgfqpoint{6.904489in}{3.056987in}}%
\pgfpathlineto{\pgfqpoint{6.909294in}{3.062779in}}%
\pgfpathlineto{\pgfqpoint{6.912498in}{3.062713in}}%
\pgfpathlineto{\pgfqpoint{6.915701in}{3.059512in}}%
\pgfpathlineto{\pgfqpoint{6.918905in}{3.053185in}}%
\pgfpathlineto{\pgfqpoint{6.918905in}{3.053185in}}%
\pgfusepath{stroke}%
\end{pgfscope}%
\begin{pgfscope}%
\pgfsetrectcap%
\pgfsetmiterjoin%
\pgfsetlinewidth{0.803000pt}%
\definecolor{currentstroke}{rgb}{0.000000,0.000000,0.000000}%
\pgfsetstrokecolor{currentstroke}%
\pgfsetdash{}{0pt}%
\pgfpathmoveto{\pgfqpoint{1.875000in}{1.000000in}}%
\pgfpathlineto{\pgfqpoint{1.875000in}{7.040000in}}%
\pgfusepath{stroke}%
\end{pgfscope}%
\begin{pgfscope}%
\pgfsetrectcap%
\pgfsetmiterjoin%
\pgfsetlinewidth{0.803000pt}%
\definecolor{currentstroke}{rgb}{0.000000,0.000000,0.000000}%
\pgfsetstrokecolor{currentstroke}%
\pgfsetdash{}{0pt}%
\pgfpathmoveto{\pgfqpoint{7.159091in}{1.000000in}}%
\pgfpathlineto{\pgfqpoint{7.159091in}{7.040000in}}%
\pgfusepath{stroke}%
\end{pgfscope}%
\begin{pgfscope}%
\pgfsetrectcap%
\pgfsetmiterjoin%
\pgfsetlinewidth{0.803000pt}%
\definecolor{currentstroke}{rgb}{0.000000,0.000000,0.000000}%
\pgfsetstrokecolor{currentstroke}%
\pgfsetdash{}{0pt}%
\pgfpathmoveto{\pgfqpoint{1.875000in}{1.000000in}}%
\pgfpathlineto{\pgfqpoint{7.159091in}{1.000000in}}%
\pgfusepath{stroke}%
\end{pgfscope}%
\begin{pgfscope}%
\pgfsetrectcap%
\pgfsetmiterjoin%
\pgfsetlinewidth{0.803000pt}%
\definecolor{currentstroke}{rgb}{0.000000,0.000000,0.000000}%
\pgfsetstrokecolor{currentstroke}%
\pgfsetdash{}{0pt}%
\pgfpathmoveto{\pgfqpoint{1.875000in}{7.040000in}}%
\pgfpathlineto{\pgfqpoint{7.159091in}{7.040000in}}%
\pgfusepath{stroke}%
\end{pgfscope}%
\begin{pgfscope}%
\definecolor{textcolor}{rgb}{0.000000,0.000000,0.000000}%
\pgfsetstrokecolor{textcolor}%
\pgfsetfillcolor{textcolor}%
\pgftext[x=4.517045in,y=7.123333in,,base]{\color{textcolor}\sffamily\fontsize{20.000000}{24.000000}\selectfont a)}%
\end{pgfscope}%
\begin{pgfscope}%
\pgfsetbuttcap%
\pgfsetmiterjoin%
\definecolor{currentfill}{rgb}{1.000000,1.000000,1.000000}%
\pgfsetfillcolor{currentfill}%
\pgfsetfillopacity{0.800000}%
\pgfsetlinewidth{1.003750pt}%
\definecolor{currentstroke}{rgb}{0.800000,0.800000,0.800000}%
\pgfsetstrokecolor{currentstroke}%
\pgfsetstrokeopacity{0.800000}%
\pgfsetdash{}{0pt}%
\pgfpathmoveto{\pgfqpoint{5.038689in}{2.740632in}}%
\pgfpathlineto{\pgfqpoint{6.964646in}{2.740632in}}%
\pgfpathquadraticcurveto{\pgfqpoint{7.020202in}{2.740632in}}{\pgfqpoint{7.020202in}{2.796188in}}%
\pgfpathlineto{\pgfqpoint{7.020202in}{6.845556in}}%
\pgfpathquadraticcurveto{\pgfqpoint{7.020202in}{6.901111in}}{\pgfqpoint{6.964646in}{6.901111in}}%
\pgfpathlineto{\pgfqpoint{5.038689in}{6.901111in}}%
\pgfpathquadraticcurveto{\pgfqpoint{4.983134in}{6.901111in}}{\pgfqpoint{4.983134in}{6.845556in}}%
\pgfpathlineto{\pgfqpoint{4.983134in}{2.796188in}}%
\pgfpathquadraticcurveto{\pgfqpoint{4.983134in}{2.740632in}}{\pgfqpoint{5.038689in}{2.740632in}}%
\pgfpathlineto{\pgfqpoint{5.038689in}{2.740632in}}%
\pgfpathclose%
\pgfusepath{stroke,fill}%
\end{pgfscope}%
\begin{pgfscope}%
\pgfsetrectcap%
\pgfsetroundjoin%
\pgfsetlinewidth{1.505625pt}%
\definecolor{currentstroke}{rgb}{0.121569,0.466667,0.705882}%
\pgfsetstrokecolor{currentstroke}%
\pgfsetdash{}{0pt}%
\pgfpathmoveto{\pgfqpoint{5.094245in}{6.676176in}}%
\pgfpathlineto{\pgfqpoint{5.372022in}{6.676176in}}%
\pgfpathlineto{\pgfqpoint{5.649800in}{6.676176in}}%
\pgfusepath{stroke}%
\end{pgfscope}%
\begin{pgfscope}%
\definecolor{textcolor}{rgb}{0.000000,0.000000,0.000000}%
\pgfsetstrokecolor{textcolor}%
\pgfsetfillcolor{textcolor}%
\pgftext[x=5.872022in,y=6.578954in,left,base]{\color{textcolor}\sffamily\fontsize{20.000000}{24.000000}\selectfont \(\displaystyle \omega\) = 0.1}%
\end{pgfscope}%
\begin{pgfscope}%
\pgfsetrectcap%
\pgfsetroundjoin%
\pgfsetlinewidth{1.505625pt}%
\definecolor{currentstroke}{rgb}{1.000000,0.498039,0.054902}%
\pgfsetstrokecolor{currentstroke}%
\pgfsetdash{}{0pt}%
\pgfpathmoveto{\pgfqpoint{5.094245in}{6.268462in}}%
\pgfpathlineto{\pgfqpoint{5.372022in}{6.268462in}}%
\pgfpathlineto{\pgfqpoint{5.649800in}{6.268462in}}%
\pgfusepath{stroke}%
\end{pgfscope}%
\begin{pgfscope}%
\definecolor{textcolor}{rgb}{0.000000,0.000000,0.000000}%
\pgfsetstrokecolor{textcolor}%
\pgfsetfillcolor{textcolor}%
\pgftext[x=5.872022in,y=6.171239in,left,base]{\color{textcolor}\sffamily\fontsize{20.000000}{24.000000}\selectfont \(\displaystyle \omega\) = 0.3}%
\end{pgfscope}%
\begin{pgfscope}%
\pgfsetrectcap%
\pgfsetroundjoin%
\pgfsetlinewidth{1.505625pt}%
\definecolor{currentstroke}{rgb}{0.172549,0.627451,0.172549}%
\pgfsetstrokecolor{currentstroke}%
\pgfsetdash{}{0pt}%
\pgfpathmoveto{\pgfqpoint{5.094245in}{5.860747in}}%
\pgfpathlineto{\pgfqpoint{5.372022in}{5.860747in}}%
\pgfpathlineto{\pgfqpoint{5.649800in}{5.860747in}}%
\pgfusepath{stroke}%
\end{pgfscope}%
\begin{pgfscope}%
\definecolor{textcolor}{rgb}{0.000000,0.000000,0.000000}%
\pgfsetstrokecolor{textcolor}%
\pgfsetfillcolor{textcolor}%
\pgftext[x=5.872022in,y=5.763525in,left,base]{\color{textcolor}\sffamily\fontsize{20.000000}{24.000000}\selectfont \(\displaystyle \omega\) = 0.5}%
\end{pgfscope}%
\begin{pgfscope}%
\pgfsetrectcap%
\pgfsetroundjoin%
\pgfsetlinewidth{1.505625pt}%
\definecolor{currentstroke}{rgb}{0.839216,0.152941,0.156863}%
\pgfsetstrokecolor{currentstroke}%
\pgfsetdash{}{0pt}%
\pgfpathmoveto{\pgfqpoint{5.094245in}{5.453032in}}%
\pgfpathlineto{\pgfqpoint{5.372022in}{5.453032in}}%
\pgfpathlineto{\pgfqpoint{5.649800in}{5.453032in}}%
\pgfusepath{stroke}%
\end{pgfscope}%
\begin{pgfscope}%
\definecolor{textcolor}{rgb}{0.000000,0.000000,0.000000}%
\pgfsetstrokecolor{textcolor}%
\pgfsetfillcolor{textcolor}%
\pgftext[x=5.872022in,y=5.355810in,left,base]{\color{textcolor}\sffamily\fontsize{20.000000}{24.000000}\selectfont \(\displaystyle \omega\) = 0.7}%
\end{pgfscope}%
\begin{pgfscope}%
\pgfsetrectcap%
\pgfsetroundjoin%
\pgfsetlinewidth{1.505625pt}%
\definecolor{currentstroke}{rgb}{0.580392,0.403922,0.741176}%
\pgfsetstrokecolor{currentstroke}%
\pgfsetdash{}{0pt}%
\pgfpathmoveto{\pgfqpoint{5.094245in}{5.045318in}}%
\pgfpathlineto{\pgfqpoint{5.372022in}{5.045318in}}%
\pgfpathlineto{\pgfqpoint{5.649800in}{5.045318in}}%
\pgfusepath{stroke}%
\end{pgfscope}%
\begin{pgfscope}%
\definecolor{textcolor}{rgb}{0.000000,0.000000,0.000000}%
\pgfsetstrokecolor{textcolor}%
\pgfsetfillcolor{textcolor}%
\pgftext[x=5.872022in,y=4.948096in,left,base]{\color{textcolor}\sffamily\fontsize{20.000000}{24.000000}\selectfont \(\displaystyle \omega\) = 0.9}%
\end{pgfscope}%
\begin{pgfscope}%
\pgfsetrectcap%
\pgfsetroundjoin%
\pgfsetlinewidth{1.505625pt}%
\definecolor{currentstroke}{rgb}{0.549020,0.337255,0.294118}%
\pgfsetstrokecolor{currentstroke}%
\pgfsetdash{}{0pt}%
\pgfpathmoveto{\pgfqpoint{5.094245in}{4.637603in}}%
\pgfpathlineto{\pgfqpoint{5.372022in}{4.637603in}}%
\pgfpathlineto{\pgfqpoint{5.649800in}{4.637603in}}%
\pgfusepath{stroke}%
\end{pgfscope}%
\begin{pgfscope}%
\definecolor{textcolor}{rgb}{0.000000,0.000000,0.000000}%
\pgfsetstrokecolor{textcolor}%
\pgfsetfillcolor{textcolor}%
\pgftext[x=5.872022in,y=4.540381in,left,base]{\color{textcolor}\sffamily\fontsize{20.000000}{24.000000}\selectfont \(\displaystyle \omega\) = 1.1}%
\end{pgfscope}%
\begin{pgfscope}%
\pgfsetrectcap%
\pgfsetroundjoin%
\pgfsetlinewidth{1.505625pt}%
\definecolor{currentstroke}{rgb}{0.890196,0.466667,0.760784}%
\pgfsetstrokecolor{currentstroke}%
\pgfsetdash{}{0pt}%
\pgfpathmoveto{\pgfqpoint{5.094245in}{4.229889in}}%
\pgfpathlineto{\pgfqpoint{5.372022in}{4.229889in}}%
\pgfpathlineto{\pgfqpoint{5.649800in}{4.229889in}}%
\pgfusepath{stroke}%
\end{pgfscope}%
\begin{pgfscope}%
\definecolor{textcolor}{rgb}{0.000000,0.000000,0.000000}%
\pgfsetstrokecolor{textcolor}%
\pgfsetfillcolor{textcolor}%
\pgftext[x=5.872022in,y=4.132667in,left,base]{\color{textcolor}\sffamily\fontsize{20.000000}{24.000000}\selectfont \(\displaystyle \omega\) = 1.3}%
\end{pgfscope}%
\begin{pgfscope}%
\pgfsetrectcap%
\pgfsetroundjoin%
\pgfsetlinewidth{1.505625pt}%
\definecolor{currentstroke}{rgb}{0.498039,0.498039,0.498039}%
\pgfsetstrokecolor{currentstroke}%
\pgfsetdash{}{0pt}%
\pgfpathmoveto{\pgfqpoint{5.094245in}{3.822174in}}%
\pgfpathlineto{\pgfqpoint{5.372022in}{3.822174in}}%
\pgfpathlineto{\pgfqpoint{5.649800in}{3.822174in}}%
\pgfusepath{stroke}%
\end{pgfscope}%
\begin{pgfscope}%
\definecolor{textcolor}{rgb}{0.000000,0.000000,0.000000}%
\pgfsetstrokecolor{textcolor}%
\pgfsetfillcolor{textcolor}%
\pgftext[x=5.872022in,y=3.724952in,left,base]{\color{textcolor}\sffamily\fontsize{20.000000}{24.000000}\selectfont \(\displaystyle \omega\) = 1.5}%
\end{pgfscope}%
\begin{pgfscope}%
\pgfsetrectcap%
\pgfsetroundjoin%
\pgfsetlinewidth{1.505625pt}%
\definecolor{currentstroke}{rgb}{0.737255,0.741176,0.133333}%
\pgfsetstrokecolor{currentstroke}%
\pgfsetdash{}{0pt}%
\pgfpathmoveto{\pgfqpoint{5.094245in}{3.414460in}}%
\pgfpathlineto{\pgfqpoint{5.372022in}{3.414460in}}%
\pgfpathlineto{\pgfqpoint{5.649800in}{3.414460in}}%
\pgfusepath{stroke}%
\end{pgfscope}%
\begin{pgfscope}%
\definecolor{textcolor}{rgb}{0.000000,0.000000,0.000000}%
\pgfsetstrokecolor{textcolor}%
\pgfsetfillcolor{textcolor}%
\pgftext[x=5.872022in,y=3.317237in,left,base]{\color{textcolor}\sffamily\fontsize{20.000000}{24.000000}\selectfont \(\displaystyle \omega\) = 1.7}%
\end{pgfscope}%
\begin{pgfscope}%
\pgfsetrectcap%
\pgfsetroundjoin%
\pgfsetlinewidth{1.505625pt}%
\definecolor{currentstroke}{rgb}{0.090196,0.745098,0.811765}%
\pgfsetstrokecolor{currentstroke}%
\pgfsetdash{}{0pt}%
\pgfpathmoveto{\pgfqpoint{5.094245in}{3.006745in}}%
\pgfpathlineto{\pgfqpoint{5.372022in}{3.006745in}}%
\pgfpathlineto{\pgfqpoint{5.649800in}{3.006745in}}%
\pgfusepath{stroke}%
\end{pgfscope}%
\begin{pgfscope}%
\definecolor{textcolor}{rgb}{0.000000,0.000000,0.000000}%
\pgfsetstrokecolor{textcolor}%
\pgfsetfillcolor{textcolor}%
\pgftext[x=5.872022in,y=2.909523in,left,base]{\color{textcolor}\sffamily\fontsize{20.000000}{24.000000}\selectfont \(\displaystyle \omega\) = 1.9}%
\end{pgfscope}%
\begin{pgfscope}%
\pgfsetbuttcap%
\pgfsetmiterjoin%
\definecolor{currentfill}{rgb}{1.000000,1.000000,1.000000}%
\pgfsetfillcolor{currentfill}%
\pgfsetlinewidth{0.000000pt}%
\definecolor{currentstroke}{rgb}{0.000000,0.000000,0.000000}%
\pgfsetstrokecolor{currentstroke}%
\pgfsetstrokeopacity{0.000000}%
\pgfsetdash{}{0pt}%
\pgfpathmoveto{\pgfqpoint{8.215909in}{1.000000in}}%
\pgfpathlineto{\pgfqpoint{13.500000in}{1.000000in}}%
\pgfpathlineto{\pgfqpoint{13.500000in}{7.040000in}}%
\pgfpathlineto{\pgfqpoint{8.215909in}{7.040000in}}%
\pgfpathlineto{\pgfqpoint{8.215909in}{1.000000in}}%
\pgfpathclose%
\pgfusepath{fill}%
\end{pgfscope}%
\begin{pgfscope}%
\pgfpathrectangle{\pgfqpoint{8.215909in}{1.000000in}}{\pgfqpoint{5.284091in}{6.040000in}}%
\pgfusepath{clip}%
\pgfsetrectcap%
\pgfsetroundjoin%
\pgfsetlinewidth{0.803000pt}%
\definecolor{currentstroke}{rgb}{0.690196,0.690196,0.690196}%
\pgfsetstrokecolor{currentstroke}%
\pgfsetdash{}{0pt}%
\pgfpathmoveto{\pgfqpoint{8.456095in}{1.000000in}}%
\pgfpathlineto{\pgfqpoint{8.456095in}{7.040000in}}%
\pgfusepath{stroke}%
\end{pgfscope}%
\begin{pgfscope}%
\pgfsetbuttcap%
\pgfsetroundjoin%
\definecolor{currentfill}{rgb}{0.000000,0.000000,0.000000}%
\pgfsetfillcolor{currentfill}%
\pgfsetlinewidth{0.803000pt}%
\definecolor{currentstroke}{rgb}{0.000000,0.000000,0.000000}%
\pgfsetstrokecolor{currentstroke}%
\pgfsetdash{}{0pt}%
\pgfsys@defobject{currentmarker}{\pgfqpoint{0.000000in}{-0.048611in}}{\pgfqpoint{0.000000in}{0.000000in}}{%
\pgfpathmoveto{\pgfqpoint{0.000000in}{0.000000in}}%
\pgfpathlineto{\pgfqpoint{0.000000in}{-0.048611in}}%
\pgfusepath{stroke,fill}%
}%
\begin{pgfscope}%
\pgfsys@transformshift{8.456095in}{1.000000in}%
\pgfsys@useobject{currentmarker}{}%
\end{pgfscope}%
\end{pgfscope}%
\begin{pgfscope}%
\definecolor{textcolor}{rgb}{0.000000,0.000000,0.000000}%
\pgfsetstrokecolor{textcolor}%
\pgfsetfillcolor{textcolor}%
\pgftext[x=8.456095in,y=0.902778in,,top]{\color{textcolor}\sffamily\fontsize{16.000000}{19.200000}\selectfont 0}%
\end{pgfscope}%
\begin{pgfscope}%
\pgfpathrectangle{\pgfqpoint{8.215909in}{1.000000in}}{\pgfqpoint{5.284091in}{6.040000in}}%
\pgfusepath{clip}%
\pgfsetrectcap%
\pgfsetroundjoin%
\pgfsetlinewidth{0.803000pt}%
\definecolor{currentstroke}{rgb}{0.690196,0.690196,0.690196}%
\pgfsetstrokecolor{currentstroke}%
\pgfsetdash{}{0pt}%
\pgfpathmoveto{\pgfqpoint{9.256982in}{1.000000in}}%
\pgfpathlineto{\pgfqpoint{9.256982in}{7.040000in}}%
\pgfusepath{stroke}%
\end{pgfscope}%
\begin{pgfscope}%
\pgfsetbuttcap%
\pgfsetroundjoin%
\definecolor{currentfill}{rgb}{0.000000,0.000000,0.000000}%
\pgfsetfillcolor{currentfill}%
\pgfsetlinewidth{0.803000pt}%
\definecolor{currentstroke}{rgb}{0.000000,0.000000,0.000000}%
\pgfsetstrokecolor{currentstroke}%
\pgfsetdash{}{0pt}%
\pgfsys@defobject{currentmarker}{\pgfqpoint{0.000000in}{-0.048611in}}{\pgfqpoint{0.000000in}{0.000000in}}{%
\pgfpathmoveto{\pgfqpoint{0.000000in}{0.000000in}}%
\pgfpathlineto{\pgfqpoint{0.000000in}{-0.048611in}}%
\pgfusepath{stroke,fill}%
}%
\begin{pgfscope}%
\pgfsys@transformshift{9.256982in}{1.000000in}%
\pgfsys@useobject{currentmarker}{}%
\end{pgfscope}%
\end{pgfscope}%
\begin{pgfscope}%
\definecolor{textcolor}{rgb}{0.000000,0.000000,0.000000}%
\pgfsetstrokecolor{textcolor}%
\pgfsetfillcolor{textcolor}%
\pgftext[x=9.256982in,y=0.902778in,,top]{\color{textcolor}\sffamily\fontsize{16.000000}{19.200000}\selectfont 500}%
\end{pgfscope}%
\begin{pgfscope}%
\pgfpathrectangle{\pgfqpoint{8.215909in}{1.000000in}}{\pgfqpoint{5.284091in}{6.040000in}}%
\pgfusepath{clip}%
\pgfsetrectcap%
\pgfsetroundjoin%
\pgfsetlinewidth{0.803000pt}%
\definecolor{currentstroke}{rgb}{0.690196,0.690196,0.690196}%
\pgfsetstrokecolor{currentstroke}%
\pgfsetdash{}{0pt}%
\pgfpathmoveto{\pgfqpoint{10.057869in}{1.000000in}}%
\pgfpathlineto{\pgfqpoint{10.057869in}{7.040000in}}%
\pgfusepath{stroke}%
\end{pgfscope}%
\begin{pgfscope}%
\pgfsetbuttcap%
\pgfsetroundjoin%
\definecolor{currentfill}{rgb}{0.000000,0.000000,0.000000}%
\pgfsetfillcolor{currentfill}%
\pgfsetlinewidth{0.803000pt}%
\definecolor{currentstroke}{rgb}{0.000000,0.000000,0.000000}%
\pgfsetstrokecolor{currentstroke}%
\pgfsetdash{}{0pt}%
\pgfsys@defobject{currentmarker}{\pgfqpoint{0.000000in}{-0.048611in}}{\pgfqpoint{0.000000in}{0.000000in}}{%
\pgfpathmoveto{\pgfqpoint{0.000000in}{0.000000in}}%
\pgfpathlineto{\pgfqpoint{0.000000in}{-0.048611in}}%
\pgfusepath{stroke,fill}%
}%
\begin{pgfscope}%
\pgfsys@transformshift{10.057869in}{1.000000in}%
\pgfsys@useobject{currentmarker}{}%
\end{pgfscope}%
\end{pgfscope}%
\begin{pgfscope}%
\definecolor{textcolor}{rgb}{0.000000,0.000000,0.000000}%
\pgfsetstrokecolor{textcolor}%
\pgfsetfillcolor{textcolor}%
\pgftext[x=10.057869in,y=0.902778in,,top]{\color{textcolor}\sffamily\fontsize{16.000000}{19.200000}\selectfont 1000}%
\end{pgfscope}%
\begin{pgfscope}%
\pgfpathrectangle{\pgfqpoint{8.215909in}{1.000000in}}{\pgfqpoint{5.284091in}{6.040000in}}%
\pgfusepath{clip}%
\pgfsetrectcap%
\pgfsetroundjoin%
\pgfsetlinewidth{0.803000pt}%
\definecolor{currentstroke}{rgb}{0.690196,0.690196,0.690196}%
\pgfsetstrokecolor{currentstroke}%
\pgfsetdash{}{0pt}%
\pgfpathmoveto{\pgfqpoint{10.858755in}{1.000000in}}%
\pgfpathlineto{\pgfqpoint{10.858755in}{7.040000in}}%
\pgfusepath{stroke}%
\end{pgfscope}%
\begin{pgfscope}%
\pgfsetbuttcap%
\pgfsetroundjoin%
\definecolor{currentfill}{rgb}{0.000000,0.000000,0.000000}%
\pgfsetfillcolor{currentfill}%
\pgfsetlinewidth{0.803000pt}%
\definecolor{currentstroke}{rgb}{0.000000,0.000000,0.000000}%
\pgfsetstrokecolor{currentstroke}%
\pgfsetdash{}{0pt}%
\pgfsys@defobject{currentmarker}{\pgfqpoint{0.000000in}{-0.048611in}}{\pgfqpoint{0.000000in}{0.000000in}}{%
\pgfpathmoveto{\pgfqpoint{0.000000in}{0.000000in}}%
\pgfpathlineto{\pgfqpoint{0.000000in}{-0.048611in}}%
\pgfusepath{stroke,fill}%
}%
\begin{pgfscope}%
\pgfsys@transformshift{10.858755in}{1.000000in}%
\pgfsys@useobject{currentmarker}{}%
\end{pgfscope}%
\end{pgfscope}%
\begin{pgfscope}%
\definecolor{textcolor}{rgb}{0.000000,0.000000,0.000000}%
\pgfsetstrokecolor{textcolor}%
\pgfsetfillcolor{textcolor}%
\pgftext[x=10.858755in,y=0.902778in,,top]{\color{textcolor}\sffamily\fontsize{16.000000}{19.200000}\selectfont 1500}%
\end{pgfscope}%
\begin{pgfscope}%
\pgfpathrectangle{\pgfqpoint{8.215909in}{1.000000in}}{\pgfqpoint{5.284091in}{6.040000in}}%
\pgfusepath{clip}%
\pgfsetrectcap%
\pgfsetroundjoin%
\pgfsetlinewidth{0.803000pt}%
\definecolor{currentstroke}{rgb}{0.690196,0.690196,0.690196}%
\pgfsetstrokecolor{currentstroke}%
\pgfsetdash{}{0pt}%
\pgfpathmoveto{\pgfqpoint{11.659642in}{1.000000in}}%
\pgfpathlineto{\pgfqpoint{11.659642in}{7.040000in}}%
\pgfusepath{stroke}%
\end{pgfscope}%
\begin{pgfscope}%
\pgfsetbuttcap%
\pgfsetroundjoin%
\definecolor{currentfill}{rgb}{0.000000,0.000000,0.000000}%
\pgfsetfillcolor{currentfill}%
\pgfsetlinewidth{0.803000pt}%
\definecolor{currentstroke}{rgb}{0.000000,0.000000,0.000000}%
\pgfsetstrokecolor{currentstroke}%
\pgfsetdash{}{0pt}%
\pgfsys@defobject{currentmarker}{\pgfqpoint{0.000000in}{-0.048611in}}{\pgfqpoint{0.000000in}{0.000000in}}{%
\pgfpathmoveto{\pgfqpoint{0.000000in}{0.000000in}}%
\pgfpathlineto{\pgfqpoint{0.000000in}{-0.048611in}}%
\pgfusepath{stroke,fill}%
}%
\begin{pgfscope}%
\pgfsys@transformshift{11.659642in}{1.000000in}%
\pgfsys@useobject{currentmarker}{}%
\end{pgfscope}%
\end{pgfscope}%
\begin{pgfscope}%
\definecolor{textcolor}{rgb}{0.000000,0.000000,0.000000}%
\pgfsetstrokecolor{textcolor}%
\pgfsetfillcolor{textcolor}%
\pgftext[x=11.659642in,y=0.902778in,,top]{\color{textcolor}\sffamily\fontsize{16.000000}{19.200000}\selectfont 2000}%
\end{pgfscope}%
\begin{pgfscope}%
\pgfpathrectangle{\pgfqpoint{8.215909in}{1.000000in}}{\pgfqpoint{5.284091in}{6.040000in}}%
\pgfusepath{clip}%
\pgfsetrectcap%
\pgfsetroundjoin%
\pgfsetlinewidth{0.803000pt}%
\definecolor{currentstroke}{rgb}{0.690196,0.690196,0.690196}%
\pgfsetstrokecolor{currentstroke}%
\pgfsetdash{}{0pt}%
\pgfpathmoveto{\pgfqpoint{12.460529in}{1.000000in}}%
\pgfpathlineto{\pgfqpoint{12.460529in}{7.040000in}}%
\pgfusepath{stroke}%
\end{pgfscope}%
\begin{pgfscope}%
\pgfsetbuttcap%
\pgfsetroundjoin%
\definecolor{currentfill}{rgb}{0.000000,0.000000,0.000000}%
\pgfsetfillcolor{currentfill}%
\pgfsetlinewidth{0.803000pt}%
\definecolor{currentstroke}{rgb}{0.000000,0.000000,0.000000}%
\pgfsetstrokecolor{currentstroke}%
\pgfsetdash{}{0pt}%
\pgfsys@defobject{currentmarker}{\pgfqpoint{0.000000in}{-0.048611in}}{\pgfqpoint{0.000000in}{0.000000in}}{%
\pgfpathmoveto{\pgfqpoint{0.000000in}{0.000000in}}%
\pgfpathlineto{\pgfqpoint{0.000000in}{-0.048611in}}%
\pgfusepath{stroke,fill}%
}%
\begin{pgfscope}%
\pgfsys@transformshift{12.460529in}{1.000000in}%
\pgfsys@useobject{currentmarker}{}%
\end{pgfscope}%
\end{pgfscope}%
\begin{pgfscope}%
\definecolor{textcolor}{rgb}{0.000000,0.000000,0.000000}%
\pgfsetstrokecolor{textcolor}%
\pgfsetfillcolor{textcolor}%
\pgftext[x=12.460529in,y=0.902778in,,top]{\color{textcolor}\sffamily\fontsize{16.000000}{19.200000}\selectfont 2500}%
\end{pgfscope}%
\begin{pgfscope}%
\pgfpathrectangle{\pgfqpoint{8.215909in}{1.000000in}}{\pgfqpoint{5.284091in}{6.040000in}}%
\pgfusepath{clip}%
\pgfsetrectcap%
\pgfsetroundjoin%
\pgfsetlinewidth{0.803000pt}%
\definecolor{currentstroke}{rgb}{0.690196,0.690196,0.690196}%
\pgfsetstrokecolor{currentstroke}%
\pgfsetdash{}{0pt}%
\pgfpathmoveto{\pgfqpoint{13.261416in}{1.000000in}}%
\pgfpathlineto{\pgfqpoint{13.261416in}{7.040000in}}%
\pgfusepath{stroke}%
\end{pgfscope}%
\begin{pgfscope}%
\pgfsetbuttcap%
\pgfsetroundjoin%
\definecolor{currentfill}{rgb}{0.000000,0.000000,0.000000}%
\pgfsetfillcolor{currentfill}%
\pgfsetlinewidth{0.803000pt}%
\definecolor{currentstroke}{rgb}{0.000000,0.000000,0.000000}%
\pgfsetstrokecolor{currentstroke}%
\pgfsetdash{}{0pt}%
\pgfsys@defobject{currentmarker}{\pgfqpoint{0.000000in}{-0.048611in}}{\pgfqpoint{0.000000in}{0.000000in}}{%
\pgfpathmoveto{\pgfqpoint{0.000000in}{0.000000in}}%
\pgfpathlineto{\pgfqpoint{0.000000in}{-0.048611in}}%
\pgfusepath{stroke,fill}%
}%
\begin{pgfscope}%
\pgfsys@transformshift{13.261416in}{1.000000in}%
\pgfsys@useobject{currentmarker}{}%
\end{pgfscope}%
\end{pgfscope}%
\begin{pgfscope}%
\definecolor{textcolor}{rgb}{0.000000,0.000000,0.000000}%
\pgfsetstrokecolor{textcolor}%
\pgfsetfillcolor{textcolor}%
\pgftext[x=13.261416in,y=0.902778in,,top]{\color{textcolor}\sffamily\fontsize{16.000000}{19.200000}\selectfont 3000}%
\end{pgfscope}%
\begin{pgfscope}%
\definecolor{textcolor}{rgb}{0.000000,0.000000,0.000000}%
\pgfsetstrokecolor{textcolor}%
\pgfsetfillcolor{textcolor}%
\pgftext[x=10.857955in,y=0.632162in,,top]{\color{textcolor}\sffamily\fontsize{20.000000}{24.000000}\selectfont \(\displaystyle t\)}%
\end{pgfscope}%
\begin{pgfscope}%
\pgfpathrectangle{\pgfqpoint{8.215909in}{1.000000in}}{\pgfqpoint{5.284091in}{6.040000in}}%
\pgfusepath{clip}%
\pgfsetrectcap%
\pgfsetroundjoin%
\pgfsetlinewidth{0.803000pt}%
\definecolor{currentstroke}{rgb}{0.690196,0.690196,0.690196}%
\pgfsetstrokecolor{currentstroke}%
\pgfsetdash{}{0pt}%
\pgfpathmoveto{\pgfqpoint{8.215909in}{1.274545in}}%
\pgfpathlineto{\pgfqpoint{13.500000in}{1.274545in}}%
\pgfusepath{stroke}%
\end{pgfscope}%
\begin{pgfscope}%
\pgfsetbuttcap%
\pgfsetroundjoin%
\definecolor{currentfill}{rgb}{0.000000,0.000000,0.000000}%
\pgfsetfillcolor{currentfill}%
\pgfsetlinewidth{0.803000pt}%
\definecolor{currentstroke}{rgb}{0.000000,0.000000,0.000000}%
\pgfsetstrokecolor{currentstroke}%
\pgfsetdash{}{0pt}%
\pgfsys@defobject{currentmarker}{\pgfqpoint{-0.048611in}{0.000000in}}{\pgfqpoint{-0.000000in}{0.000000in}}{%
\pgfpathmoveto{\pgfqpoint{-0.000000in}{0.000000in}}%
\pgfpathlineto{\pgfqpoint{-0.048611in}{0.000000in}}%
\pgfusepath{stroke,fill}%
}%
\begin{pgfscope}%
\pgfsys@transformshift{8.215909in}{1.274545in}%
\pgfsys@useobject{currentmarker}{}%
\end{pgfscope}%
\end{pgfscope}%
\begin{pgfscope}%
\definecolor{textcolor}{rgb}{0.000000,0.000000,0.000000}%
\pgfsetstrokecolor{textcolor}%
\pgfsetfillcolor{textcolor}%
\pgftext[x=7.765280in, y=1.190127in, left, base]{\color{textcolor}\sffamily\fontsize{16.000000}{19.200000}\selectfont 0.0}%
\end{pgfscope}%
\begin{pgfscope}%
\pgfpathrectangle{\pgfqpoint{8.215909in}{1.000000in}}{\pgfqpoint{5.284091in}{6.040000in}}%
\pgfusepath{clip}%
\pgfsetrectcap%
\pgfsetroundjoin%
\pgfsetlinewidth{0.803000pt}%
\definecolor{currentstroke}{rgb}{0.690196,0.690196,0.690196}%
\pgfsetstrokecolor{currentstroke}%
\pgfsetdash{}{0pt}%
\pgfpathmoveto{\pgfqpoint{8.215909in}{2.372727in}}%
\pgfpathlineto{\pgfqpoint{13.500000in}{2.372727in}}%
\pgfusepath{stroke}%
\end{pgfscope}%
\begin{pgfscope}%
\pgfsetbuttcap%
\pgfsetroundjoin%
\definecolor{currentfill}{rgb}{0.000000,0.000000,0.000000}%
\pgfsetfillcolor{currentfill}%
\pgfsetlinewidth{0.803000pt}%
\definecolor{currentstroke}{rgb}{0.000000,0.000000,0.000000}%
\pgfsetstrokecolor{currentstroke}%
\pgfsetdash{}{0pt}%
\pgfsys@defobject{currentmarker}{\pgfqpoint{-0.048611in}{0.000000in}}{\pgfqpoint{-0.000000in}{0.000000in}}{%
\pgfpathmoveto{\pgfqpoint{-0.000000in}{0.000000in}}%
\pgfpathlineto{\pgfqpoint{-0.048611in}{0.000000in}}%
\pgfusepath{stroke,fill}%
}%
\begin{pgfscope}%
\pgfsys@transformshift{8.215909in}{2.372727in}%
\pgfsys@useobject{currentmarker}{}%
\end{pgfscope}%
\end{pgfscope}%
\begin{pgfscope}%
\definecolor{textcolor}{rgb}{0.000000,0.000000,0.000000}%
\pgfsetstrokecolor{textcolor}%
\pgfsetfillcolor{textcolor}%
\pgftext[x=7.765280in, y=2.288309in, left, base]{\color{textcolor}\sffamily\fontsize{16.000000}{19.200000}\selectfont 0.2}%
\end{pgfscope}%
\begin{pgfscope}%
\pgfpathrectangle{\pgfqpoint{8.215909in}{1.000000in}}{\pgfqpoint{5.284091in}{6.040000in}}%
\pgfusepath{clip}%
\pgfsetrectcap%
\pgfsetroundjoin%
\pgfsetlinewidth{0.803000pt}%
\definecolor{currentstroke}{rgb}{0.690196,0.690196,0.690196}%
\pgfsetstrokecolor{currentstroke}%
\pgfsetdash{}{0pt}%
\pgfpathmoveto{\pgfqpoint{8.215909in}{3.470909in}}%
\pgfpathlineto{\pgfqpoint{13.500000in}{3.470909in}}%
\pgfusepath{stroke}%
\end{pgfscope}%
\begin{pgfscope}%
\pgfsetbuttcap%
\pgfsetroundjoin%
\definecolor{currentfill}{rgb}{0.000000,0.000000,0.000000}%
\pgfsetfillcolor{currentfill}%
\pgfsetlinewidth{0.803000pt}%
\definecolor{currentstroke}{rgb}{0.000000,0.000000,0.000000}%
\pgfsetstrokecolor{currentstroke}%
\pgfsetdash{}{0pt}%
\pgfsys@defobject{currentmarker}{\pgfqpoint{-0.048611in}{0.000000in}}{\pgfqpoint{-0.000000in}{0.000000in}}{%
\pgfpathmoveto{\pgfqpoint{-0.000000in}{0.000000in}}%
\pgfpathlineto{\pgfqpoint{-0.048611in}{0.000000in}}%
\pgfusepath{stroke,fill}%
}%
\begin{pgfscope}%
\pgfsys@transformshift{8.215909in}{3.470909in}%
\pgfsys@useobject{currentmarker}{}%
\end{pgfscope}%
\end{pgfscope}%
\begin{pgfscope}%
\definecolor{textcolor}{rgb}{0.000000,0.000000,0.000000}%
\pgfsetstrokecolor{textcolor}%
\pgfsetfillcolor{textcolor}%
\pgftext[x=7.765280in, y=3.386491in, left, base]{\color{textcolor}\sffamily\fontsize{16.000000}{19.200000}\selectfont 0.4}%
\end{pgfscope}%
\begin{pgfscope}%
\pgfpathrectangle{\pgfqpoint{8.215909in}{1.000000in}}{\pgfqpoint{5.284091in}{6.040000in}}%
\pgfusepath{clip}%
\pgfsetrectcap%
\pgfsetroundjoin%
\pgfsetlinewidth{0.803000pt}%
\definecolor{currentstroke}{rgb}{0.690196,0.690196,0.690196}%
\pgfsetstrokecolor{currentstroke}%
\pgfsetdash{}{0pt}%
\pgfpathmoveto{\pgfqpoint{8.215909in}{4.569091in}}%
\pgfpathlineto{\pgfqpoint{13.500000in}{4.569091in}}%
\pgfusepath{stroke}%
\end{pgfscope}%
\begin{pgfscope}%
\pgfsetbuttcap%
\pgfsetroundjoin%
\definecolor{currentfill}{rgb}{0.000000,0.000000,0.000000}%
\pgfsetfillcolor{currentfill}%
\pgfsetlinewidth{0.803000pt}%
\definecolor{currentstroke}{rgb}{0.000000,0.000000,0.000000}%
\pgfsetstrokecolor{currentstroke}%
\pgfsetdash{}{0pt}%
\pgfsys@defobject{currentmarker}{\pgfqpoint{-0.048611in}{0.000000in}}{\pgfqpoint{-0.000000in}{0.000000in}}{%
\pgfpathmoveto{\pgfqpoint{-0.000000in}{0.000000in}}%
\pgfpathlineto{\pgfqpoint{-0.048611in}{0.000000in}}%
\pgfusepath{stroke,fill}%
}%
\begin{pgfscope}%
\pgfsys@transformshift{8.215909in}{4.569091in}%
\pgfsys@useobject{currentmarker}{}%
\end{pgfscope}%
\end{pgfscope}%
\begin{pgfscope}%
\definecolor{textcolor}{rgb}{0.000000,0.000000,0.000000}%
\pgfsetstrokecolor{textcolor}%
\pgfsetfillcolor{textcolor}%
\pgftext[x=7.765280in, y=4.484673in, left, base]{\color{textcolor}\sffamily\fontsize{16.000000}{19.200000}\selectfont 0.6}%
\end{pgfscope}%
\begin{pgfscope}%
\pgfpathrectangle{\pgfqpoint{8.215909in}{1.000000in}}{\pgfqpoint{5.284091in}{6.040000in}}%
\pgfusepath{clip}%
\pgfsetrectcap%
\pgfsetroundjoin%
\pgfsetlinewidth{0.803000pt}%
\definecolor{currentstroke}{rgb}{0.690196,0.690196,0.690196}%
\pgfsetstrokecolor{currentstroke}%
\pgfsetdash{}{0pt}%
\pgfpathmoveto{\pgfqpoint{8.215909in}{5.667273in}}%
\pgfpathlineto{\pgfqpoint{13.500000in}{5.667273in}}%
\pgfusepath{stroke}%
\end{pgfscope}%
\begin{pgfscope}%
\pgfsetbuttcap%
\pgfsetroundjoin%
\definecolor{currentfill}{rgb}{0.000000,0.000000,0.000000}%
\pgfsetfillcolor{currentfill}%
\pgfsetlinewidth{0.803000pt}%
\definecolor{currentstroke}{rgb}{0.000000,0.000000,0.000000}%
\pgfsetstrokecolor{currentstroke}%
\pgfsetdash{}{0pt}%
\pgfsys@defobject{currentmarker}{\pgfqpoint{-0.048611in}{0.000000in}}{\pgfqpoint{-0.000000in}{0.000000in}}{%
\pgfpathmoveto{\pgfqpoint{-0.000000in}{0.000000in}}%
\pgfpathlineto{\pgfqpoint{-0.048611in}{0.000000in}}%
\pgfusepath{stroke,fill}%
}%
\begin{pgfscope}%
\pgfsys@transformshift{8.215909in}{5.667273in}%
\pgfsys@useobject{currentmarker}{}%
\end{pgfscope}%
\end{pgfscope}%
\begin{pgfscope}%
\definecolor{textcolor}{rgb}{0.000000,0.000000,0.000000}%
\pgfsetstrokecolor{textcolor}%
\pgfsetfillcolor{textcolor}%
\pgftext[x=7.765280in, y=5.582854in, left, base]{\color{textcolor}\sffamily\fontsize{16.000000}{19.200000}\selectfont 0.8}%
\end{pgfscope}%
\begin{pgfscope}%
\pgfpathrectangle{\pgfqpoint{8.215909in}{1.000000in}}{\pgfqpoint{5.284091in}{6.040000in}}%
\pgfusepath{clip}%
\pgfsetrectcap%
\pgfsetroundjoin%
\pgfsetlinewidth{0.803000pt}%
\definecolor{currentstroke}{rgb}{0.690196,0.690196,0.690196}%
\pgfsetstrokecolor{currentstroke}%
\pgfsetdash{}{0pt}%
\pgfpathmoveto{\pgfqpoint{8.215909in}{6.765455in}}%
\pgfpathlineto{\pgfqpoint{13.500000in}{6.765455in}}%
\pgfusepath{stroke}%
\end{pgfscope}%
\begin{pgfscope}%
\pgfsetbuttcap%
\pgfsetroundjoin%
\definecolor{currentfill}{rgb}{0.000000,0.000000,0.000000}%
\pgfsetfillcolor{currentfill}%
\pgfsetlinewidth{0.803000pt}%
\definecolor{currentstroke}{rgb}{0.000000,0.000000,0.000000}%
\pgfsetstrokecolor{currentstroke}%
\pgfsetdash{}{0pt}%
\pgfsys@defobject{currentmarker}{\pgfqpoint{-0.048611in}{0.000000in}}{\pgfqpoint{-0.000000in}{0.000000in}}{%
\pgfpathmoveto{\pgfqpoint{-0.000000in}{0.000000in}}%
\pgfpathlineto{\pgfqpoint{-0.048611in}{0.000000in}}%
\pgfusepath{stroke,fill}%
}%
\begin{pgfscope}%
\pgfsys@transformshift{8.215909in}{6.765455in}%
\pgfsys@useobject{currentmarker}{}%
\end{pgfscope}%
\end{pgfscope}%
\begin{pgfscope}%
\definecolor{textcolor}{rgb}{0.000000,0.000000,0.000000}%
\pgfsetstrokecolor{textcolor}%
\pgfsetfillcolor{textcolor}%
\pgftext[x=7.765280in, y=6.681036in, left, base]{\color{textcolor}\sffamily\fontsize{16.000000}{19.200000}\selectfont 1.0}%
\end{pgfscope}%
\begin{pgfscope}%
\definecolor{textcolor}{rgb}{0.000000,0.000000,0.000000}%
\pgfsetstrokecolor{textcolor}%
\pgfsetfillcolor{textcolor}%
\pgftext[x=7.709724in,y=4.020000in,,bottom,rotate=90.000000]{\color{textcolor}\sffamily\fontsize{20.000000}{24.000000}\selectfont \(\displaystyle a(t)/a(0)\)}%
\end{pgfscope}%
\begin{pgfscope}%
\pgfpathrectangle{\pgfqpoint{8.215909in}{1.000000in}}{\pgfqpoint{5.284091in}{6.040000in}}%
\pgfusepath{clip}%
\pgfsetrectcap%
\pgfsetroundjoin%
\pgfsetlinewidth{1.505625pt}%
\definecolor{currentstroke}{rgb}{0.121569,0.466667,0.705882}%
\pgfsetstrokecolor{currentstroke}%
\pgfsetdash{}{0pt}%
\pgfpathmoveto{\pgfqpoint{8.456095in}{6.765455in}}%
\pgfpathlineto{\pgfqpoint{8.462502in}{4.424763in}}%
\pgfpathlineto{\pgfqpoint{8.468909in}{3.081873in}}%
\pgfpathlineto{\pgfqpoint{8.475316in}{2.311437in}}%
\pgfpathlineto{\pgfqpoint{8.481723in}{1.869426in}}%
\pgfpathlineto{\pgfqpoint{8.488131in}{1.615837in}}%
\pgfpathlineto{\pgfqpoint{8.494538in}{1.470350in}}%
\pgfpathlineto{\pgfqpoint{8.500945in}{1.386881in}}%
\pgfpathlineto{\pgfqpoint{8.507352in}{1.338994in}}%
\pgfpathlineto{\pgfqpoint{8.513759in}{1.311521in}}%
\pgfpathlineto{\pgfqpoint{8.520166in}{1.295759in}}%
\pgfpathlineto{\pgfqpoint{8.526573in}{1.286716in}}%
\pgfpathlineto{\pgfqpoint{8.534582in}{1.280622in}}%
\pgfpathlineto{\pgfqpoint{8.544193in}{1.277186in}}%
\pgfpathlineto{\pgfqpoint{8.560210in}{1.275204in}}%
\pgfpathlineto{\pgfqpoint{8.600255in}{1.274566in}}%
\pgfpathlineto{\pgfqpoint{9.545301in}{1.274545in}}%
\pgfpathlineto{\pgfqpoint{13.259814in}{1.274545in}}%
\pgfpathlineto{\pgfqpoint{13.259814in}{1.274545in}}%
\pgfusepath{stroke}%
\end{pgfscope}%
\begin{pgfscope}%
\pgfpathrectangle{\pgfqpoint{8.215909in}{1.000000in}}{\pgfqpoint{5.284091in}{6.040000in}}%
\pgfusepath{clip}%
\pgfsetrectcap%
\pgfsetroundjoin%
\pgfsetlinewidth{1.505625pt}%
\definecolor{currentstroke}{rgb}{1.000000,0.498039,0.054902}%
\pgfsetstrokecolor{currentstroke}%
\pgfsetdash{}{0pt}%
\pgfpathmoveto{\pgfqpoint{8.456095in}{6.765455in}}%
\pgfpathlineto{\pgfqpoint{8.467307in}{5.383210in}}%
\pgfpathlineto{\pgfqpoint{8.478520in}{4.348922in}}%
\pgfpathlineto{\pgfqpoint{8.489732in}{3.574999in}}%
\pgfpathlineto{\pgfqpoint{8.500945in}{2.995899in}}%
\pgfpathlineto{\pgfqpoint{8.512157in}{2.562577in}}%
\pgfpathlineto{\pgfqpoint{8.523370in}{2.238336in}}%
\pgfpathlineto{\pgfqpoint{8.534582in}{1.995718in}}%
\pgfpathlineto{\pgfqpoint{8.545794in}{1.814175in}}%
\pgfpathlineto{\pgfqpoint{8.557007in}{1.678332in}}%
\pgfpathlineto{\pgfqpoint{8.568219in}{1.576686in}}%
\pgfpathlineto{\pgfqpoint{8.579432in}{1.500627in}}%
\pgfpathlineto{\pgfqpoint{8.590644in}{1.443715in}}%
\pgfpathlineto{\pgfqpoint{8.601856in}{1.401129in}}%
\pgfpathlineto{\pgfqpoint{8.613069in}{1.369264in}}%
\pgfpathlineto{\pgfqpoint{8.624281in}{1.345420in}}%
\pgfpathlineto{\pgfqpoint{8.637095in}{1.325426in}}%
\pgfpathlineto{\pgfqpoint{8.649910in}{1.311073in}}%
\pgfpathlineto{\pgfqpoint{8.664326in}{1.299704in}}%
\pgfpathlineto{\pgfqpoint{8.680343in}{1.291171in}}%
\pgfpathlineto{\pgfqpoint{8.699565in}{1.284658in}}%
\pgfpathlineto{\pgfqpoint{8.723591in}{1.279978in}}%
\pgfpathlineto{\pgfqpoint{8.757228in}{1.276821in}}%
\pgfpathlineto{\pgfqpoint{8.813291in}{1.275079in}}%
\pgfpathlineto{\pgfqpoint{8.965459in}{1.274556in}}%
\pgfpathlineto{\pgfqpoint{13.259814in}{1.274545in}}%
\pgfpathlineto{\pgfqpoint{13.259814in}{1.274545in}}%
\pgfusepath{stroke}%
\end{pgfscope}%
\begin{pgfscope}%
\pgfpathrectangle{\pgfqpoint{8.215909in}{1.000000in}}{\pgfqpoint{5.284091in}{6.040000in}}%
\pgfusepath{clip}%
\pgfsetrectcap%
\pgfsetroundjoin%
\pgfsetlinewidth{1.505625pt}%
\definecolor{currentstroke}{rgb}{0.172549,0.627451,0.172549}%
\pgfsetstrokecolor{currentstroke}%
\pgfsetdash{}{0pt}%
\pgfpathmoveto{\pgfqpoint{8.456095in}{6.765455in}}%
\pgfpathlineto{\pgfqpoint{8.472113in}{5.684081in}}%
\pgfpathlineto{\pgfqpoint{8.488131in}{4.815671in}}%
\pgfpathlineto{\pgfqpoint{8.504148in}{4.118286in}}%
\pgfpathlineto{\pgfqpoint{8.520166in}{3.558243in}}%
\pgfpathlineto{\pgfqpoint{8.536184in}{3.108494in}}%
\pgfpathlineto{\pgfqpoint{8.552201in}{2.747318in}}%
\pgfpathlineto{\pgfqpoint{8.568219in}{2.457272in}}%
\pgfpathlineto{\pgfqpoint{8.584237in}{2.224347in}}%
\pgfpathlineto{\pgfqpoint{8.600255in}{2.037294in}}%
\pgfpathlineto{\pgfqpoint{8.616272in}{1.887079in}}%
\pgfpathlineto{\pgfqpoint{8.632290in}{1.766447in}}%
\pgfpathlineto{\pgfqpoint{8.648308in}{1.669573in}}%
\pgfpathlineto{\pgfqpoint{8.664326in}{1.591776in}}%
\pgfpathlineto{\pgfqpoint{8.680343in}{1.529301in}}%
\pgfpathlineto{\pgfqpoint{8.696361in}{1.479130in}}%
\pgfpathlineto{\pgfqpoint{8.712379in}{1.438839in}}%
\pgfpathlineto{\pgfqpoint{8.728397in}{1.406483in}}%
\pgfpathlineto{\pgfqpoint{8.744414in}{1.380500in}}%
\pgfpathlineto{\pgfqpoint{8.762034in}{1.357787in}}%
\pgfpathlineto{\pgfqpoint{8.779653in}{1.339944in}}%
\pgfpathlineto{\pgfqpoint{8.798875in}{1.324810in}}%
\pgfpathlineto{\pgfqpoint{8.819698in}{1.312341in}}%
\pgfpathlineto{\pgfqpoint{8.843724in}{1.301745in}}%
\pgfpathlineto{\pgfqpoint{8.870954in}{1.293280in}}%
\pgfpathlineto{\pgfqpoint{8.902990in}{1.286627in}}%
\pgfpathlineto{\pgfqpoint{8.944636in}{1.281376in}}%
\pgfpathlineto{\pgfqpoint{9.000698in}{1.277716in}}%
\pgfpathlineto{\pgfqpoint{9.087194in}{1.275515in}}%
\pgfpathlineto{\pgfqpoint{9.271398in}{1.274623in}}%
\pgfpathlineto{\pgfqpoint{10.807499in}{1.274545in}}%
\pgfpathlineto{\pgfqpoint{13.259814in}{1.274545in}}%
\pgfpathlineto{\pgfqpoint{13.259814in}{1.274545in}}%
\pgfusepath{stroke}%
\end{pgfscope}%
\begin{pgfscope}%
\pgfpathrectangle{\pgfqpoint{8.215909in}{1.000000in}}{\pgfqpoint{5.284091in}{6.040000in}}%
\pgfusepath{clip}%
\pgfsetrectcap%
\pgfsetroundjoin%
\pgfsetlinewidth{1.505625pt}%
\definecolor{currentstroke}{rgb}{0.839216,0.152941,0.156863}%
\pgfsetstrokecolor{currentstroke}%
\pgfsetdash{}{0pt}%
\pgfpathmoveto{\pgfqpoint{8.456095in}{6.765455in}}%
\pgfpathlineto{\pgfqpoint{8.475316in}{5.939915in}}%
\pgfpathlineto{\pgfqpoint{8.494538in}{5.238492in}}%
\pgfpathlineto{\pgfqpoint{8.513759in}{4.642526in}}%
\pgfpathlineto{\pgfqpoint{8.532980in}{4.136162in}}%
\pgfpathlineto{\pgfqpoint{8.552201in}{3.705927in}}%
\pgfpathlineto{\pgfqpoint{8.571423in}{3.340377in}}%
\pgfpathlineto{\pgfqpoint{8.590644in}{3.029787in}}%
\pgfpathlineto{\pgfqpoint{8.609865in}{2.765892in}}%
\pgfpathlineto{\pgfqpoint{8.629087in}{2.541673in}}%
\pgfpathlineto{\pgfqpoint{8.648308in}{2.351165in}}%
\pgfpathlineto{\pgfqpoint{8.667529in}{2.189299in}}%
\pgfpathlineto{\pgfqpoint{8.686750in}{2.051769in}}%
\pgfpathlineto{\pgfqpoint{8.705972in}{1.934916in}}%
\pgfpathlineto{\pgfqpoint{8.725193in}{1.835631in}}%
\pgfpathlineto{\pgfqpoint{8.744414in}{1.751274in}}%
\pgfpathlineto{\pgfqpoint{8.763636in}{1.679599in}}%
\pgfpathlineto{\pgfqpoint{8.784459in}{1.614060in}}%
\pgfpathlineto{\pgfqpoint{8.805282in}{1.559125in}}%
\pgfpathlineto{\pgfqpoint{8.826105in}{1.513079in}}%
\pgfpathlineto{\pgfqpoint{8.846928in}{1.474483in}}%
\pgfpathlineto{\pgfqpoint{8.867751in}{1.442132in}}%
\pgfpathlineto{\pgfqpoint{8.890176in}{1.413121in}}%
\pgfpathlineto{\pgfqpoint{8.912601in}{1.389133in}}%
\pgfpathlineto{\pgfqpoint{8.936627in}{1.368019in}}%
\pgfpathlineto{\pgfqpoint{8.962255in}{1.349767in}}%
\pgfpathlineto{\pgfqpoint{8.989486in}{1.334263in}}%
\pgfpathlineto{\pgfqpoint{9.019919in}{1.320685in}}%
\pgfpathlineto{\pgfqpoint{9.053557in}{1.309239in}}%
\pgfpathlineto{\pgfqpoint{9.091999in}{1.299591in}}%
\pgfpathlineto{\pgfqpoint{9.136849in}{1.291670in}}%
\pgfpathlineto{\pgfqpoint{9.191309in}{1.285338in}}%
\pgfpathlineto{\pgfqpoint{9.260185in}{1.280565in}}%
\pgfpathlineto{\pgfqpoint{9.356292in}{1.277211in}}%
\pgfpathlineto{\pgfqpoint{9.511664in}{1.275260in}}%
\pgfpathlineto{\pgfqpoint{9.881674in}{1.274576in}}%
\pgfpathlineto{\pgfqpoint{13.259814in}{1.274545in}}%
\pgfpathlineto{\pgfqpoint{13.259814in}{1.274545in}}%
\pgfusepath{stroke}%
\end{pgfscope}%
\begin{pgfscope}%
\pgfpathrectangle{\pgfqpoint{8.215909in}{1.000000in}}{\pgfqpoint{5.284091in}{6.040000in}}%
\pgfusepath{clip}%
\pgfsetrectcap%
\pgfsetroundjoin%
\pgfsetlinewidth{1.505625pt}%
\definecolor{currentstroke}{rgb}{0.580392,0.403922,0.741176}%
\pgfsetstrokecolor{currentstroke}%
\pgfsetdash{}{0pt}%
\pgfpathmoveto{\pgfqpoint{8.456095in}{6.765455in}}%
\pgfpathlineto{\pgfqpoint{8.480122in}{6.076688in}}%
\pgfpathlineto{\pgfqpoint{8.504148in}{5.474319in}}%
\pgfpathlineto{\pgfqpoint{8.528175in}{4.947509in}}%
\pgfpathlineto{\pgfqpoint{8.552201in}{4.486781in}}%
\pgfpathlineto{\pgfqpoint{8.576228in}{4.083846in}}%
\pgfpathlineto{\pgfqpoint{8.600255in}{3.731454in}}%
\pgfpathlineto{\pgfqpoint{8.624281in}{3.423265in}}%
\pgfpathlineto{\pgfqpoint{8.648308in}{3.153735in}}%
\pgfpathlineto{\pgfqpoint{8.672334in}{2.918014in}}%
\pgfpathlineto{\pgfqpoint{8.696361in}{2.711861in}}%
\pgfpathlineto{\pgfqpoint{8.720388in}{2.531568in}}%
\pgfpathlineto{\pgfqpoint{8.744414in}{2.373890in}}%
\pgfpathlineto{\pgfqpoint{8.768441in}{2.235991in}}%
\pgfpathlineto{\pgfqpoint{8.792467in}{2.115389in}}%
\pgfpathlineto{\pgfqpoint{8.816494in}{2.009916in}}%
\pgfpathlineto{\pgfqpoint{8.840521in}{1.917673in}}%
\pgfpathlineto{\pgfqpoint{8.864547in}{1.837000in}}%
\pgfpathlineto{\pgfqpoint{8.888574in}{1.766447in}}%
\pgfpathlineto{\pgfqpoint{8.914202in}{1.700917in}}%
\pgfpathlineto{\pgfqpoint{8.939831in}{1.644117in}}%
\pgfpathlineto{\pgfqpoint{8.965459in}{1.594884in}}%
\pgfpathlineto{\pgfqpoint{8.991087in}{1.552209in}}%
\pgfpathlineto{\pgfqpoint{9.016716in}{1.515220in}}%
\pgfpathlineto{\pgfqpoint{9.043946in}{1.481302in}}%
\pgfpathlineto{\pgfqpoint{9.071176in}{1.452164in}}%
\pgfpathlineto{\pgfqpoint{9.100008in}{1.425775in}}%
\pgfpathlineto{\pgfqpoint{9.130442in}{1.402162in}}%
\pgfpathlineto{\pgfqpoint{9.162477in}{1.381277in}}%
\pgfpathlineto{\pgfqpoint{9.196114in}{1.363016in}}%
\pgfpathlineto{\pgfqpoint{9.232955in}{1.346581in}}%
\pgfpathlineto{\pgfqpoint{9.273000in}{1.332160in}}%
\pgfpathlineto{\pgfqpoint{9.317849in}{1.319407in}}%
\pgfpathlineto{\pgfqpoint{9.367504in}{1.308553in}}%
\pgfpathlineto{\pgfqpoint{9.425168in}{1.299199in}}%
\pgfpathlineto{\pgfqpoint{9.492443in}{1.291484in}}%
\pgfpathlineto{\pgfqpoint{9.575735in}{1.285189in}}%
\pgfpathlineto{\pgfqpoint{9.681452in}{1.280447in}}%
\pgfpathlineto{\pgfqpoint{9.828815in}{1.277139in}}%
\pgfpathlineto{\pgfqpoint{10.067479in}{1.275231in}}%
\pgfpathlineto{\pgfqpoint{10.645720in}{1.274573in}}%
\pgfpathlineto{\pgfqpoint{13.259814in}{1.274545in}}%
\pgfpathlineto{\pgfqpoint{13.259814in}{1.274545in}}%
\pgfusepath{stroke}%
\end{pgfscope}%
\begin{pgfscope}%
\pgfpathrectangle{\pgfqpoint{8.215909in}{1.000000in}}{\pgfqpoint{5.284091in}{6.040000in}}%
\pgfusepath{clip}%
\pgfsetrectcap%
\pgfsetroundjoin%
\pgfsetlinewidth{1.505625pt}%
\definecolor{currentstroke}{rgb}{0.549020,0.337255,0.294118}%
\pgfsetstrokecolor{currentstroke}%
\pgfsetdash{}{0pt}%
\pgfpathmoveto{\pgfqpoint{8.456095in}{6.765455in}}%
\pgfpathlineto{\pgfqpoint{8.484927in}{6.204972in}}%
\pgfpathlineto{\pgfqpoint{8.513759in}{5.701700in}}%
\pgfpathlineto{\pgfqpoint{8.542591in}{5.249799in}}%
\pgfpathlineto{\pgfqpoint{8.571423in}{4.844026in}}%
\pgfpathlineto{\pgfqpoint{8.600255in}{4.479673in}}%
\pgfpathlineto{\pgfqpoint{8.629087in}{4.152510in}}%
\pgfpathlineto{\pgfqpoint{8.657919in}{3.858743in}}%
\pgfpathlineto{\pgfqpoint{8.686750in}{3.594962in}}%
\pgfpathlineto{\pgfqpoint{8.715582in}{3.358106in}}%
\pgfpathlineto{\pgfqpoint{8.744414in}{3.145427in}}%
\pgfpathlineto{\pgfqpoint{8.773246in}{2.954457in}}%
\pgfpathlineto{\pgfqpoint{8.802078in}{2.782981in}}%
\pgfpathlineto{\pgfqpoint{8.830910in}{2.629008in}}%
\pgfpathlineto{\pgfqpoint{8.859742in}{2.490751in}}%
\pgfpathlineto{\pgfqpoint{8.890176in}{2.360095in}}%
\pgfpathlineto{\pgfqpoint{8.920609in}{2.243474in}}%
\pgfpathlineto{\pgfqpoint{8.951043in}{2.139383in}}%
\pgfpathlineto{\pgfqpoint{8.981477in}{2.046474in}}%
\pgfpathlineto{\pgfqpoint{9.011910in}{1.963546in}}%
\pgfpathlineto{\pgfqpoint{9.042344in}{1.889526in}}%
\pgfpathlineto{\pgfqpoint{9.072778in}{1.823459in}}%
\pgfpathlineto{\pgfqpoint{9.103212in}{1.764490in}}%
\pgfpathlineto{\pgfqpoint{9.135247in}{1.709247in}}%
\pgfpathlineto{\pgfqpoint{9.167283in}{1.660233in}}%
\pgfpathlineto{\pgfqpoint{9.199318in}{1.616746in}}%
\pgfpathlineto{\pgfqpoint{9.232955in}{1.576351in}}%
\pgfpathlineto{\pgfqpoint{9.266592in}{1.540725in}}%
\pgfpathlineto{\pgfqpoint{9.301831in}{1.507904in}}%
\pgfpathlineto{\pgfqpoint{9.338672in}{1.477910in}}%
\pgfpathlineto{\pgfqpoint{9.377115in}{1.450714in}}%
\pgfpathlineto{\pgfqpoint{9.417159in}{1.426245in}}%
\pgfpathlineto{\pgfqpoint{9.458805in}{1.404396in}}%
\pgfpathlineto{\pgfqpoint{9.503655in}{1.384371in}}%
\pgfpathlineto{\pgfqpoint{9.551708in}{1.366331in}}%
\pgfpathlineto{\pgfqpoint{9.604567in}{1.349889in}}%
\pgfpathlineto{\pgfqpoint{9.662231in}{1.335293in}}%
\pgfpathlineto{\pgfqpoint{9.726302in}{1.322366in}}%
\pgfpathlineto{\pgfqpoint{9.798381in}{1.311081in}}%
\pgfpathlineto{\pgfqpoint{9.880072in}{1.301475in}}%
\pgfpathlineto{\pgfqpoint{9.976178in}{1.293354in}}%
\pgfpathlineto{\pgfqpoint{10.093108in}{1.286700in}}%
\pgfpathlineto{\pgfqpoint{10.240471in}{1.281556in}}%
\pgfpathlineto{\pgfqpoint{10.440693in}{1.277864in}}%
\pgfpathlineto{\pgfqpoint{10.748233in}{1.275598in}}%
\pgfpathlineto{\pgfqpoint{11.385739in}{1.274643in}}%
\pgfpathlineto{\pgfqpoint{13.259814in}{1.274546in}}%
\pgfpathlineto{\pgfqpoint{13.259814in}{1.274546in}}%
\pgfusepath{stroke}%
\end{pgfscope}%
\begin{pgfscope}%
\pgfpathrectangle{\pgfqpoint{8.215909in}{1.000000in}}{\pgfqpoint{5.284091in}{6.040000in}}%
\pgfusepath{clip}%
\pgfsetrectcap%
\pgfsetroundjoin%
\pgfsetlinewidth{1.505625pt}%
\definecolor{currentstroke}{rgb}{0.890196,0.466667,0.760784}%
\pgfsetstrokecolor{currentstroke}%
\pgfsetdash{}{0pt}%
\pgfpathmoveto{\pgfqpoint{8.456095in}{6.765455in}}%
\pgfpathlineto{\pgfqpoint{8.491334in}{6.309924in}}%
\pgfpathlineto{\pgfqpoint{8.526573in}{5.892185in}}%
\pgfpathlineto{\pgfqpoint{8.561812in}{5.509102in}}%
\pgfpathlineto{\pgfqpoint{8.598653in}{5.142543in}}%
\pgfpathlineto{\pgfqpoint{8.635494in}{4.807715in}}%
\pgfpathlineto{\pgfqpoint{8.672334in}{4.501870in}}%
\pgfpathlineto{\pgfqpoint{8.709175in}{4.222501in}}%
\pgfpathlineto{\pgfqpoint{8.746016in}{3.967315in}}%
\pgfpathlineto{\pgfqpoint{8.782857in}{3.734219in}}%
\pgfpathlineto{\pgfqpoint{8.819698in}{3.521300in}}%
\pgfpathlineto{\pgfqpoint{8.856538in}{3.326813in}}%
\pgfpathlineto{\pgfqpoint{8.893379in}{3.149161in}}%
\pgfpathlineto{\pgfqpoint{8.930220in}{2.986887in}}%
\pgfpathlineto{\pgfqpoint{8.967061in}{2.838661in}}%
\pgfpathlineto{\pgfqpoint{9.003902in}{2.703265in}}%
\pgfpathlineto{\pgfqpoint{9.040742in}{2.579590in}}%
\pgfpathlineto{\pgfqpoint{9.077583in}{2.466620in}}%
\pgfpathlineto{\pgfqpoint{9.114424in}{2.363430in}}%
\pgfpathlineto{\pgfqpoint{9.151265in}{2.269172in}}%
\pgfpathlineto{\pgfqpoint{9.188106in}{2.183073in}}%
\pgfpathlineto{\pgfqpoint{9.226548in}{2.101167in}}%
\pgfpathlineto{\pgfqpoint{9.264991in}{2.026645in}}%
\pgfpathlineto{\pgfqpoint{9.303433in}{1.958842in}}%
\pgfpathlineto{\pgfqpoint{9.341876in}{1.897151in}}%
\pgfpathlineto{\pgfqpoint{9.381920in}{1.838796in}}%
\pgfpathlineto{\pgfqpoint{9.421965in}{1.785910in}}%
\pgfpathlineto{\pgfqpoint{9.462009in}{1.737981in}}%
\pgfpathlineto{\pgfqpoint{9.503655in}{1.692895in}}%
\pgfpathlineto{\pgfqpoint{9.546903in}{1.650711in}}%
\pgfpathlineto{\pgfqpoint{9.590151in}{1.612780in}}%
\pgfpathlineto{\pgfqpoint{9.635000in}{1.577480in}}%
\pgfpathlineto{\pgfqpoint{9.681452in}{1.544797in}}%
\pgfpathlineto{\pgfqpoint{9.729505in}{1.514694in}}%
\pgfpathlineto{\pgfqpoint{9.780762in}{1.486270in}}%
\pgfpathlineto{\pgfqpoint{9.833620in}{1.460477in}}%
\pgfpathlineto{\pgfqpoint{9.889682in}{1.436546in}}%
\pgfpathlineto{\pgfqpoint{9.948948in}{1.414588in}}%
\pgfpathlineto{\pgfqpoint{10.013019in}{1.394185in}}%
\pgfpathlineto{\pgfqpoint{10.081895in}{1.375554in}}%
\pgfpathlineto{\pgfqpoint{10.155577in}{1.358824in}}%
\pgfpathlineto{\pgfqpoint{10.237267in}{1.343494in}}%
\pgfpathlineto{\pgfqpoint{10.326967in}{1.329853in}}%
\pgfpathlineto{\pgfqpoint{10.427878in}{1.317705in}}%
\pgfpathlineto{\pgfqpoint{10.541604in}{1.307181in}}%
\pgfpathlineto{\pgfqpoint{10.674551in}{1.298085in}}%
\pgfpathlineto{\pgfqpoint{10.831525in}{1.290550in}}%
\pgfpathlineto{\pgfqpoint{11.025340in}{1.284485in}}%
\pgfpathlineto{\pgfqpoint{11.275217in}{1.279924in}}%
\pgfpathlineto{\pgfqpoint{11.627607in}{1.276808in}}%
\pgfpathlineto{\pgfqpoint{12.217059in}{1.275077in}}%
\pgfpathlineto{\pgfqpoint{13.259814in}{1.274586in}}%
\pgfpathlineto{\pgfqpoint{13.259814in}{1.274586in}}%
\pgfusepath{stroke}%
\end{pgfscope}%
\begin{pgfscope}%
\pgfpathrectangle{\pgfqpoint{8.215909in}{1.000000in}}{\pgfqpoint{5.284091in}{6.040000in}}%
\pgfusepath{clip}%
\pgfsetrectcap%
\pgfsetroundjoin%
\pgfsetlinewidth{1.505625pt}%
\definecolor{currentstroke}{rgb}{0.498039,0.498039,0.498039}%
\pgfsetstrokecolor{currentstroke}%
\pgfsetdash{}{0pt}%
\pgfpathmoveto{\pgfqpoint{8.456095in}{6.765455in}}%
\pgfpathlineto{\pgfqpoint{8.502546in}{6.390800in}}%
\pgfpathlineto{\pgfqpoint{8.548998in}{6.041708in}}%
\pgfpathlineto{\pgfqpoint{8.595449in}{5.716436in}}%
\pgfpathlineto{\pgfqpoint{8.641901in}{5.413358in}}%
\pgfpathlineto{\pgfqpoint{8.688352in}{5.130959in}}%
\pgfpathlineto{\pgfqpoint{8.734804in}{4.867829in}}%
\pgfpathlineto{\pgfqpoint{8.781255in}{4.622653in}}%
\pgfpathlineto{\pgfqpoint{8.827707in}{4.394205in}}%
\pgfpathlineto{\pgfqpoint{8.874158in}{4.181345in}}%
\pgfpathlineto{\pgfqpoint{8.920609in}{3.983009in}}%
\pgfpathlineto{\pgfqpoint{8.967061in}{3.798205in}}%
\pgfpathlineto{\pgfqpoint{9.013512in}{3.626011in}}%
\pgfpathlineto{\pgfqpoint{9.059964in}{3.465567in}}%
\pgfpathlineto{\pgfqpoint{9.106415in}{3.316069in}}%
\pgfpathlineto{\pgfqpoint{9.152867in}{3.176772in}}%
\pgfpathlineto{\pgfqpoint{9.199318in}{3.046980in}}%
\pgfpathlineto{\pgfqpoint{9.245769in}{2.926043in}}%
\pgfpathlineto{\pgfqpoint{9.293823in}{2.809613in}}%
\pgfpathlineto{\pgfqpoint{9.341876in}{2.701391in}}%
\pgfpathlineto{\pgfqpoint{9.389929in}{2.600799in}}%
\pgfpathlineto{\pgfqpoint{9.437982in}{2.507299in}}%
\pgfpathlineto{\pgfqpoint{9.486035in}{2.420390in}}%
\pgfpathlineto{\pgfqpoint{9.535690in}{2.337016in}}%
\pgfpathlineto{\pgfqpoint{9.585345in}{2.259708in}}%
\pgfpathlineto{\pgfqpoint{9.635000in}{2.188026in}}%
\pgfpathlineto{\pgfqpoint{9.686257in}{2.119497in}}%
\pgfpathlineto{\pgfqpoint{9.737514in}{2.056110in}}%
\pgfpathlineto{\pgfqpoint{9.790372in}{1.995718in}}%
\pgfpathlineto{\pgfqpoint{9.843231in}{1.939993in}}%
\pgfpathlineto{\pgfqpoint{9.897691in}{1.887079in}}%
\pgfpathlineto{\pgfqpoint{9.953753in}{1.837000in}}%
\pgfpathlineto{\pgfqpoint{10.009815in}{1.791016in}}%
\pgfpathlineto{\pgfqpoint{10.067479in}{1.747637in}}%
\pgfpathlineto{\pgfqpoint{10.126745in}{1.706846in}}%
\pgfpathlineto{\pgfqpoint{10.189214in}{1.667652in}}%
\pgfpathlineto{\pgfqpoint{10.253285in}{1.631141in}}%
\pgfpathlineto{\pgfqpoint{10.318958in}{1.597234in}}%
\pgfpathlineto{\pgfqpoint{10.387834in}{1.565132in}}%
\pgfpathlineto{\pgfqpoint{10.459914in}{1.534950in}}%
\pgfpathlineto{\pgfqpoint{10.535197in}{1.506769in}}%
\pgfpathlineto{\pgfqpoint{10.615286in}{1.480129in}}%
\pgfpathlineto{\pgfqpoint{10.700180in}{1.455220in}}%
\pgfpathlineto{\pgfqpoint{10.789879in}{1.432172in}}%
\pgfpathlineto{\pgfqpoint{10.885986in}{1.410730in}}%
\pgfpathlineto{\pgfqpoint{10.988499in}{1.391064in}}%
\pgfpathlineto{\pgfqpoint{11.099021in}{1.373030in}}%
\pgfpathlineto{\pgfqpoint{11.219154in}{1.356579in}}%
\pgfpathlineto{\pgfqpoint{11.352102in}{1.341557in}}%
\pgfpathlineto{\pgfqpoint{11.499465in}{1.328098in}}%
\pgfpathlineto{\pgfqpoint{11.664448in}{1.316210in}}%
\pgfpathlineto{\pgfqpoint{11.851855in}{1.305874in}}%
\pgfpathlineto{\pgfqpoint{12.069696in}{1.297036in}}%
\pgfpathlineto{\pgfqpoint{12.329184in}{1.289700in}}%
\pgfpathlineto{\pgfqpoint{12.649538in}{1.283854in}}%
\pgfpathlineto{\pgfqpoint{13.069203in}{1.279461in}}%
\pgfpathlineto{\pgfqpoint{13.259814in}{1.278224in}}%
\pgfpathlineto{\pgfqpoint{13.259814in}{1.278224in}}%
\pgfusepath{stroke}%
\end{pgfscope}%
\begin{pgfscope}%
\pgfpathrectangle{\pgfqpoint{8.215909in}{1.000000in}}{\pgfqpoint{5.284091in}{6.040000in}}%
\pgfusepath{clip}%
\pgfsetrectcap%
\pgfsetroundjoin%
\pgfsetlinewidth{1.505625pt}%
\definecolor{currentstroke}{rgb}{0.737255,0.741176,0.133333}%
\pgfsetstrokecolor{currentstroke}%
\pgfsetdash{}{0pt}%
\pgfpathmoveto{\pgfqpoint{8.456095in}{6.765455in}}%
\pgfpathlineto{\pgfqpoint{8.518564in}{6.496011in}}%
\pgfpathlineto{\pgfqpoint{8.582635in}{6.233388in}}%
\pgfpathlineto{\pgfqpoint{8.646706in}{5.983974in}}%
\pgfpathlineto{\pgfqpoint{8.710777in}{5.747105in}}%
\pgfpathlineto{\pgfqpoint{8.774848in}{5.522150in}}%
\pgfpathlineto{\pgfqpoint{8.838919in}{5.308509in}}%
\pgfpathlineto{\pgfqpoint{8.902990in}{5.105613in}}%
\pgfpathlineto{\pgfqpoint{8.967061in}{4.912923in}}%
\pgfpathlineto{\pgfqpoint{9.031132in}{4.729924in}}%
\pgfpathlineto{\pgfqpoint{9.095203in}{4.556129in}}%
\pgfpathlineto{\pgfqpoint{9.159274in}{4.391076in}}%
\pgfpathlineto{\pgfqpoint{9.223345in}{4.234325in}}%
\pgfpathlineto{\pgfqpoint{9.289017in}{4.081833in}}%
\pgfpathlineto{\pgfqpoint{9.354690in}{3.937198in}}%
\pgfpathlineto{\pgfqpoint{9.420363in}{3.800015in}}%
\pgfpathlineto{\pgfqpoint{9.486035in}{3.669899in}}%
\pgfpathlineto{\pgfqpoint{9.551708in}{3.546488in}}%
\pgfpathlineto{\pgfqpoint{9.618983in}{3.426656in}}%
\pgfpathlineto{\pgfqpoint{9.686257in}{3.313145in}}%
\pgfpathlineto{\pgfqpoint{9.753532in}{3.205621in}}%
\pgfpathlineto{\pgfqpoint{9.822408in}{3.101409in}}%
\pgfpathlineto{\pgfqpoint{9.891284in}{3.002822in}}%
\pgfpathlineto{\pgfqpoint{9.960160in}{2.909554in}}%
\pgfpathlineto{\pgfqpoint{10.030638in}{2.819326in}}%
\pgfpathlineto{\pgfqpoint{10.101117in}{2.734077in}}%
\pgfpathlineto{\pgfqpoint{10.173196in}{2.651755in}}%
\pgfpathlineto{\pgfqpoint{10.245276in}{2.574075in}}%
\pgfpathlineto{\pgfqpoint{10.318958in}{2.499196in}}%
\pgfpathlineto{\pgfqpoint{10.394241in}{2.427144in}}%
\pgfpathlineto{\pgfqpoint{10.469524in}{2.359331in}}%
\pgfpathlineto{\pgfqpoint{10.546410in}{2.294191in}}%
\pgfpathlineto{\pgfqpoint{10.624896in}{2.231728in}}%
\pgfpathlineto{\pgfqpoint{10.704985in}{2.171932in}}%
\pgfpathlineto{\pgfqpoint{10.786676in}{2.114787in}}%
\pgfpathlineto{\pgfqpoint{10.869968in}{2.060266in}}%
\pgfpathlineto{\pgfqpoint{10.954862in}{2.008336in}}%
\pgfpathlineto{\pgfqpoint{11.042959in}{1.958072in}}%
\pgfpathlineto{\pgfqpoint{11.132659in}{1.910431in}}%
\pgfpathlineto{\pgfqpoint{11.225562in}{1.864585in}}%
\pgfpathlineto{\pgfqpoint{11.321668in}{1.820634in}}%
\pgfpathlineto{\pgfqpoint{11.420978in}{1.778654in}}%
\pgfpathlineto{\pgfqpoint{11.523491in}{1.738702in}}%
\pgfpathlineto{\pgfqpoint{11.629209in}{1.700816in}}%
\pgfpathlineto{\pgfqpoint{11.739731in}{1.664509in}}%
\pgfpathlineto{\pgfqpoint{11.855059in}{1.629917in}}%
\pgfpathlineto{\pgfqpoint{11.975192in}{1.597142in}}%
\pgfpathlineto{\pgfqpoint{12.100130in}{1.566259in}}%
\pgfpathlineto{\pgfqpoint{12.231475in}{1.536974in}}%
\pgfpathlineto{\pgfqpoint{12.370830in}{1.509111in}}%
\pgfpathlineto{\pgfqpoint{12.516591in}{1.483128in}}%
\pgfpathlineto{\pgfqpoint{12.671963in}{1.458592in}}%
\pgfpathlineto{\pgfqpoint{12.836946in}{1.435690in}}%
\pgfpathlineto{\pgfqpoint{13.013141in}{1.414370in}}%
\pgfpathlineto{\pgfqpoint{13.203752in}{1.394470in}}%
\pgfpathlineto{\pgfqpoint{13.259814in}{1.389175in}}%
\pgfpathlineto{\pgfqpoint{13.259814in}{1.389175in}}%
\pgfusepath{stroke}%
\end{pgfscope}%
\begin{pgfscope}%
\pgfpathrectangle{\pgfqpoint{8.215909in}{1.000000in}}{\pgfqpoint{5.284091in}{6.040000in}}%
\pgfusepath{clip}%
\pgfsetrectcap%
\pgfsetroundjoin%
\pgfsetlinewidth{1.505625pt}%
\definecolor{currentstroke}{rgb}{0.090196,0.745098,0.811765}%
\pgfsetstrokecolor{currentstroke}%
\pgfsetdash{}{0pt}%
\pgfpathmoveto{\pgfqpoint{8.456095in}{6.765455in}}%
\pgfpathlineto{\pgfqpoint{8.584237in}{6.599006in}}%
\pgfpathlineto{\pgfqpoint{8.712379in}{6.437603in}}%
\pgfpathlineto{\pgfqpoint{8.842122in}{6.279167in}}%
\pgfpathlineto{\pgfqpoint{8.971866in}{6.125593in}}%
\pgfpathlineto{\pgfqpoint{9.103212in}{5.974923in}}%
\pgfpathlineto{\pgfqpoint{9.234557in}{5.828932in}}%
\pgfpathlineto{\pgfqpoint{9.367504in}{5.685778in}}%
\pgfpathlineto{\pgfqpoint{9.500451in}{5.547123in}}%
\pgfpathlineto{\pgfqpoint{9.635000in}{5.411235in}}%
\pgfpathlineto{\pgfqpoint{9.771151in}{5.278128in}}%
\pgfpathlineto{\pgfqpoint{9.907302in}{5.149304in}}%
\pgfpathlineto{\pgfqpoint{10.045054in}{5.023182in}}%
\pgfpathlineto{\pgfqpoint{10.184409in}{4.899770in}}%
\pgfpathlineto{\pgfqpoint{10.325365in}{4.779073in}}%
\pgfpathlineto{\pgfqpoint{10.467923in}{4.661091in}}%
\pgfpathlineto{\pgfqpoint{10.612082in}{4.545821in}}%
\pgfpathlineto{\pgfqpoint{10.757844in}{4.433260in}}%
\pgfpathlineto{\pgfqpoint{10.905207in}{4.323398in}}%
\pgfpathlineto{\pgfqpoint{11.054172in}{4.216225in}}%
\pgfpathlineto{\pgfqpoint{11.204739in}{4.111728in}}%
\pgfpathlineto{\pgfqpoint{11.356907in}{4.009890in}}%
\pgfpathlineto{\pgfqpoint{11.510677in}{3.910693in}}%
\pgfpathlineto{\pgfqpoint{11.666049in}{3.814116in}}%
\pgfpathlineto{\pgfqpoint{11.824625in}{3.719195in}}%
\pgfpathlineto{\pgfqpoint{11.984802in}{3.626916in}}%
\pgfpathlineto{\pgfqpoint{12.148183in}{3.536380in}}%
\pgfpathlineto{\pgfqpoint{12.313166in}{3.448491in}}%
\pgfpathlineto{\pgfqpoint{12.481352in}{3.362410in}}%
\pgfpathlineto{\pgfqpoint{12.651140in}{3.278967in}}%
\pgfpathlineto{\pgfqpoint{12.824132in}{3.197377in}}%
\pgfpathlineto{\pgfqpoint{13.000327in}{3.117690in}}%
\pgfpathlineto{\pgfqpoint{13.179725in}{3.039947in}}%
\pgfpathlineto{\pgfqpoint{13.259814in}{3.006307in}}%
\pgfpathlineto{\pgfqpoint{13.259814in}{3.006307in}}%
\pgfusepath{stroke}%
\end{pgfscope}%
\begin{pgfscope}%
\pgfsetrectcap%
\pgfsetmiterjoin%
\pgfsetlinewidth{0.803000pt}%
\definecolor{currentstroke}{rgb}{0.000000,0.000000,0.000000}%
\pgfsetstrokecolor{currentstroke}%
\pgfsetdash{}{0pt}%
\pgfpathmoveto{\pgfqpoint{8.215909in}{1.000000in}}%
\pgfpathlineto{\pgfqpoint{8.215909in}{7.040000in}}%
\pgfusepath{stroke}%
\end{pgfscope}%
\begin{pgfscope}%
\pgfsetrectcap%
\pgfsetmiterjoin%
\pgfsetlinewidth{0.803000pt}%
\definecolor{currentstroke}{rgb}{0.000000,0.000000,0.000000}%
\pgfsetstrokecolor{currentstroke}%
\pgfsetdash{}{0pt}%
\pgfpathmoveto{\pgfqpoint{13.500000in}{1.000000in}}%
\pgfpathlineto{\pgfqpoint{13.500000in}{7.040000in}}%
\pgfusepath{stroke}%
\end{pgfscope}%
\begin{pgfscope}%
\pgfsetrectcap%
\pgfsetmiterjoin%
\pgfsetlinewidth{0.803000pt}%
\definecolor{currentstroke}{rgb}{0.000000,0.000000,0.000000}%
\pgfsetstrokecolor{currentstroke}%
\pgfsetdash{}{0pt}%
\pgfpathmoveto{\pgfqpoint{8.215909in}{1.000000in}}%
\pgfpathlineto{\pgfqpoint{13.500000in}{1.000000in}}%
\pgfusepath{stroke}%
\end{pgfscope}%
\begin{pgfscope}%
\pgfsetrectcap%
\pgfsetmiterjoin%
\pgfsetlinewidth{0.803000pt}%
\definecolor{currentstroke}{rgb}{0.000000,0.000000,0.000000}%
\pgfsetstrokecolor{currentstroke}%
\pgfsetdash{}{0pt}%
\pgfpathmoveto{\pgfqpoint{8.215909in}{7.040000in}}%
\pgfpathlineto{\pgfqpoint{13.500000in}{7.040000in}}%
\pgfusepath{stroke}%
\end{pgfscope}%
\begin{pgfscope}%
\definecolor{textcolor}{rgb}{0.000000,0.000000,0.000000}%
\pgfsetstrokecolor{textcolor}%
\pgfsetfillcolor{textcolor}%
\pgftext[x=10.857955in,y=7.123333in,,base]{\color{textcolor}\sffamily\fontsize{20.000000}{24.000000}\selectfont b)}%
\end{pgfscope}%
\begin{pgfscope}%
\pgfsetbuttcap%
\pgfsetmiterjoin%
\definecolor{currentfill}{rgb}{1.000000,1.000000,1.000000}%
\pgfsetfillcolor{currentfill}%
\pgfsetfillopacity{0.800000}%
\pgfsetlinewidth{1.003750pt}%
\definecolor{currentstroke}{rgb}{0.800000,0.800000,0.800000}%
\pgfsetstrokecolor{currentstroke}%
\pgfsetstrokeopacity{0.800000}%
\pgfsetdash{}{0pt}%
\pgfpathmoveto{\pgfqpoint{11.379598in}{2.740632in}}%
\pgfpathlineto{\pgfqpoint{13.305556in}{2.740632in}}%
\pgfpathquadraticcurveto{\pgfqpoint{13.361111in}{2.740632in}}{\pgfqpoint{13.361111in}{2.796188in}}%
\pgfpathlineto{\pgfqpoint{13.361111in}{6.845556in}}%
\pgfpathquadraticcurveto{\pgfqpoint{13.361111in}{6.901111in}}{\pgfqpoint{13.305556in}{6.901111in}}%
\pgfpathlineto{\pgfqpoint{11.379598in}{6.901111in}}%
\pgfpathquadraticcurveto{\pgfqpoint{11.324043in}{6.901111in}}{\pgfqpoint{11.324043in}{6.845556in}}%
\pgfpathlineto{\pgfqpoint{11.324043in}{2.796188in}}%
\pgfpathquadraticcurveto{\pgfqpoint{11.324043in}{2.740632in}}{\pgfqpoint{11.379598in}{2.740632in}}%
\pgfpathlineto{\pgfqpoint{11.379598in}{2.740632in}}%
\pgfpathclose%
\pgfusepath{stroke,fill}%
\end{pgfscope}%
\begin{pgfscope}%
\pgfsetrectcap%
\pgfsetroundjoin%
\pgfsetlinewidth{1.505625pt}%
\definecolor{currentstroke}{rgb}{0.121569,0.466667,0.705882}%
\pgfsetstrokecolor{currentstroke}%
\pgfsetdash{}{0pt}%
\pgfpathmoveto{\pgfqpoint{11.435154in}{6.676176in}}%
\pgfpathlineto{\pgfqpoint{11.712932in}{6.676176in}}%
\pgfpathlineto{\pgfqpoint{11.990709in}{6.676176in}}%
\pgfusepath{stroke}%
\end{pgfscope}%
\begin{pgfscope}%
\definecolor{textcolor}{rgb}{0.000000,0.000000,0.000000}%
\pgfsetstrokecolor{textcolor}%
\pgfsetfillcolor{textcolor}%
\pgftext[x=12.212932in,y=6.578954in,left,base]{\color{textcolor}\sffamily\fontsize{20.000000}{24.000000}\selectfont \(\displaystyle \omega\) = 0.1}%
\end{pgfscope}%
\begin{pgfscope}%
\pgfsetrectcap%
\pgfsetroundjoin%
\pgfsetlinewidth{1.505625pt}%
\definecolor{currentstroke}{rgb}{1.000000,0.498039,0.054902}%
\pgfsetstrokecolor{currentstroke}%
\pgfsetdash{}{0pt}%
\pgfpathmoveto{\pgfqpoint{11.435154in}{6.268462in}}%
\pgfpathlineto{\pgfqpoint{11.712932in}{6.268462in}}%
\pgfpathlineto{\pgfqpoint{11.990709in}{6.268462in}}%
\pgfusepath{stroke}%
\end{pgfscope}%
\begin{pgfscope}%
\definecolor{textcolor}{rgb}{0.000000,0.000000,0.000000}%
\pgfsetstrokecolor{textcolor}%
\pgfsetfillcolor{textcolor}%
\pgftext[x=12.212932in,y=6.171239in,left,base]{\color{textcolor}\sffamily\fontsize{20.000000}{24.000000}\selectfont \(\displaystyle \omega\) = 0.3}%
\end{pgfscope}%
\begin{pgfscope}%
\pgfsetrectcap%
\pgfsetroundjoin%
\pgfsetlinewidth{1.505625pt}%
\definecolor{currentstroke}{rgb}{0.172549,0.627451,0.172549}%
\pgfsetstrokecolor{currentstroke}%
\pgfsetdash{}{0pt}%
\pgfpathmoveto{\pgfqpoint{11.435154in}{5.860747in}}%
\pgfpathlineto{\pgfqpoint{11.712932in}{5.860747in}}%
\pgfpathlineto{\pgfqpoint{11.990709in}{5.860747in}}%
\pgfusepath{stroke}%
\end{pgfscope}%
\begin{pgfscope}%
\definecolor{textcolor}{rgb}{0.000000,0.000000,0.000000}%
\pgfsetstrokecolor{textcolor}%
\pgfsetfillcolor{textcolor}%
\pgftext[x=12.212932in,y=5.763525in,left,base]{\color{textcolor}\sffamily\fontsize{20.000000}{24.000000}\selectfont \(\displaystyle \omega\) = 0.5}%
\end{pgfscope}%
\begin{pgfscope}%
\pgfsetrectcap%
\pgfsetroundjoin%
\pgfsetlinewidth{1.505625pt}%
\definecolor{currentstroke}{rgb}{0.839216,0.152941,0.156863}%
\pgfsetstrokecolor{currentstroke}%
\pgfsetdash{}{0pt}%
\pgfpathmoveto{\pgfqpoint{11.435154in}{5.453032in}}%
\pgfpathlineto{\pgfqpoint{11.712932in}{5.453032in}}%
\pgfpathlineto{\pgfqpoint{11.990709in}{5.453032in}}%
\pgfusepath{stroke}%
\end{pgfscope}%
\begin{pgfscope}%
\definecolor{textcolor}{rgb}{0.000000,0.000000,0.000000}%
\pgfsetstrokecolor{textcolor}%
\pgfsetfillcolor{textcolor}%
\pgftext[x=12.212932in,y=5.355810in,left,base]{\color{textcolor}\sffamily\fontsize{20.000000}{24.000000}\selectfont \(\displaystyle \omega\) = 0.7}%
\end{pgfscope}%
\begin{pgfscope}%
\pgfsetrectcap%
\pgfsetroundjoin%
\pgfsetlinewidth{1.505625pt}%
\definecolor{currentstroke}{rgb}{0.580392,0.403922,0.741176}%
\pgfsetstrokecolor{currentstroke}%
\pgfsetdash{}{0pt}%
\pgfpathmoveto{\pgfqpoint{11.435154in}{5.045318in}}%
\pgfpathlineto{\pgfqpoint{11.712932in}{5.045318in}}%
\pgfpathlineto{\pgfqpoint{11.990709in}{5.045318in}}%
\pgfusepath{stroke}%
\end{pgfscope}%
\begin{pgfscope}%
\definecolor{textcolor}{rgb}{0.000000,0.000000,0.000000}%
\pgfsetstrokecolor{textcolor}%
\pgfsetfillcolor{textcolor}%
\pgftext[x=12.212932in,y=4.948096in,left,base]{\color{textcolor}\sffamily\fontsize{20.000000}{24.000000}\selectfont \(\displaystyle \omega\) = 0.9}%
\end{pgfscope}%
\begin{pgfscope}%
\pgfsetrectcap%
\pgfsetroundjoin%
\pgfsetlinewidth{1.505625pt}%
\definecolor{currentstroke}{rgb}{0.549020,0.337255,0.294118}%
\pgfsetstrokecolor{currentstroke}%
\pgfsetdash{}{0pt}%
\pgfpathmoveto{\pgfqpoint{11.435154in}{4.637603in}}%
\pgfpathlineto{\pgfqpoint{11.712932in}{4.637603in}}%
\pgfpathlineto{\pgfqpoint{11.990709in}{4.637603in}}%
\pgfusepath{stroke}%
\end{pgfscope}%
\begin{pgfscope}%
\definecolor{textcolor}{rgb}{0.000000,0.000000,0.000000}%
\pgfsetstrokecolor{textcolor}%
\pgfsetfillcolor{textcolor}%
\pgftext[x=12.212932in,y=4.540381in,left,base]{\color{textcolor}\sffamily\fontsize{20.000000}{24.000000}\selectfont \(\displaystyle \omega\) = 1.1}%
\end{pgfscope}%
\begin{pgfscope}%
\pgfsetrectcap%
\pgfsetroundjoin%
\pgfsetlinewidth{1.505625pt}%
\definecolor{currentstroke}{rgb}{0.890196,0.466667,0.760784}%
\pgfsetstrokecolor{currentstroke}%
\pgfsetdash{}{0pt}%
\pgfpathmoveto{\pgfqpoint{11.435154in}{4.229889in}}%
\pgfpathlineto{\pgfqpoint{11.712932in}{4.229889in}}%
\pgfpathlineto{\pgfqpoint{11.990709in}{4.229889in}}%
\pgfusepath{stroke}%
\end{pgfscope}%
\begin{pgfscope}%
\definecolor{textcolor}{rgb}{0.000000,0.000000,0.000000}%
\pgfsetstrokecolor{textcolor}%
\pgfsetfillcolor{textcolor}%
\pgftext[x=12.212932in,y=4.132667in,left,base]{\color{textcolor}\sffamily\fontsize{20.000000}{24.000000}\selectfont \(\displaystyle \omega\) = 1.3}%
\end{pgfscope}%
\begin{pgfscope}%
\pgfsetrectcap%
\pgfsetroundjoin%
\pgfsetlinewidth{1.505625pt}%
\definecolor{currentstroke}{rgb}{0.498039,0.498039,0.498039}%
\pgfsetstrokecolor{currentstroke}%
\pgfsetdash{}{0pt}%
\pgfpathmoveto{\pgfqpoint{11.435154in}{3.822174in}}%
\pgfpathlineto{\pgfqpoint{11.712932in}{3.822174in}}%
\pgfpathlineto{\pgfqpoint{11.990709in}{3.822174in}}%
\pgfusepath{stroke}%
\end{pgfscope}%
\begin{pgfscope}%
\definecolor{textcolor}{rgb}{0.000000,0.000000,0.000000}%
\pgfsetstrokecolor{textcolor}%
\pgfsetfillcolor{textcolor}%
\pgftext[x=12.212932in,y=3.724952in,left,base]{\color{textcolor}\sffamily\fontsize{20.000000}{24.000000}\selectfont \(\displaystyle \omega\) = 1.5}%
\end{pgfscope}%
\begin{pgfscope}%
\pgfsetrectcap%
\pgfsetroundjoin%
\pgfsetlinewidth{1.505625pt}%
\definecolor{currentstroke}{rgb}{0.737255,0.741176,0.133333}%
\pgfsetstrokecolor{currentstroke}%
\pgfsetdash{}{0pt}%
\pgfpathmoveto{\pgfqpoint{11.435154in}{3.414460in}}%
\pgfpathlineto{\pgfqpoint{11.712932in}{3.414460in}}%
\pgfpathlineto{\pgfqpoint{11.990709in}{3.414460in}}%
\pgfusepath{stroke}%
\end{pgfscope}%
\begin{pgfscope}%
\definecolor{textcolor}{rgb}{0.000000,0.000000,0.000000}%
\pgfsetstrokecolor{textcolor}%
\pgfsetfillcolor{textcolor}%
\pgftext[x=12.212932in,y=3.317237in,left,base]{\color{textcolor}\sffamily\fontsize{20.000000}{24.000000}\selectfont \(\displaystyle \omega\) = 1.7}%
\end{pgfscope}%
\begin{pgfscope}%
\pgfsetrectcap%
\pgfsetroundjoin%
\pgfsetlinewidth{1.505625pt}%
\definecolor{currentstroke}{rgb}{0.090196,0.745098,0.811765}%
\pgfsetstrokecolor{currentstroke}%
\pgfsetdash{}{0pt}%
\pgfpathmoveto{\pgfqpoint{11.435154in}{3.006745in}}%
\pgfpathlineto{\pgfqpoint{11.712932in}{3.006745in}}%
\pgfpathlineto{\pgfqpoint{11.990709in}{3.006745in}}%
\pgfusepath{stroke}%
\end{pgfscope}%
\begin{pgfscope}%
\definecolor{textcolor}{rgb}{0.000000,0.000000,0.000000}%
\pgfsetstrokecolor{textcolor}%
\pgfsetfillcolor{textcolor}%
\pgftext[x=12.212932in,y=2.909523in,left,base]{\color{textcolor}\sffamily\fontsize{20.000000}{24.000000}\selectfont \(\displaystyle \omega\) = 1.9}%
\end{pgfscope}%
\end{pgfpicture}%
\makeatother%
\endgroup%
}
% \vspace*{-10mm}
\caption[Normalized density decay]{The density decay for $0.1 \leq \omega \leq 1.9 $ and $T=3,000$. a) is the numerical decay and b) is the theoretical decay.}
\label{fig:m3-2-norm-vel}
\end{figure}
Both graphs in figure \ref{fig:m3-2-norm-vel} use the same axes, time on the x axis and the normalized maximum velocity values along the y dimension (height dimension) on the y axis.
It illustrates that the higher the collision frequency the faster the system converges towards an equilibrium state where $a)$ is the numerical, velocity decay and $b)$ is the theoretical viscosity decay.
For all values of $\omega$, except for $\omega=0.1$ and $\omega=0.3$, the numerical and the theoretical decay are very similar
In addition, figure \ref{fig:m3-2-vel} also shows that increases in the collision frequency $\omega$ have a disproportional bigger effect on the time it takes to reach the equilibrium state. However, the effect gets lower the bigger the lattice size.


\paragraph{Kinematic viscosity scaling with $\omega$}
The kinematic viscosity is (theoretically) only dependent on $\omega$. 
Plotting the theoretical and the numerical kinematic viscosities next to each other, as done in figure \ref{fig:m3-3-kinematic-viscocity}, should then reflect the model performs according to theory. 
\begin{figure}[ht]
\centering
\resizebox{0.8\columnwidth}{!}{\large%% Creator: Matplotlib, PGF backend
%%
%% To include the figure in your LaTeX document, write
%%   \input{<filename>.pgf}
%%
%% Make sure the required packages are loaded in your preamble
%%   \usepackage{pgf}
%%
%% Also ensure that all the required font packages are loaded; for instance,
%% the lmodern package is sometimes necessary when using math font.
%%   \usepackage{lmodern}
%%
%% Figures using additional raster images can only be included by \input if
%% they are in the same directory as the main LaTeX file. For loading figures
%% from other directories you can use the `import` package
%%   \usepackage{import}
%%
%% and then include the figures with
%%   \import{<path to file>}{<filename>.pgf}
%%
%% Matplotlib used the following preamble
%%   \usepackage{fontspec}
%%   \setmainfont{DejaVuSerif.ttf}[Path=\detokenize{/home/joe/miniconda3/envs/high/lib/python3.9/site-packages/matplotlib/mpl-data/fonts/ttf/}]
%%   \setsansfont{DejaVuSans.ttf}[Path=\detokenize{/home/joe/miniconda3/envs/high/lib/python3.9/site-packages/matplotlib/mpl-data/fonts/ttf/}]
%%   \setmonofont{DejaVuSansMono.ttf}[Path=\detokenize{/home/joe/miniconda3/envs/high/lib/python3.9/site-packages/matplotlib/mpl-data/fonts/ttf/}]
%%
\begingroup%
\makeatletter%
\begin{pgfpicture}%
\pgfpathrectangle{\pgfpointorigin}{\pgfqpoint{8.000000in}{5.000000in}}%
\pgfusepath{use as bounding box, clip}%
\begin{pgfscope}%
\pgfsetbuttcap%
\pgfsetmiterjoin%
\pgfsetlinewidth{0.000000pt}%
\definecolor{currentstroke}{rgb}{1.000000,1.000000,1.000000}%
\pgfsetstrokecolor{currentstroke}%
\pgfsetstrokeopacity{0.000000}%
\pgfsetdash{}{0pt}%
\pgfpathmoveto{\pgfqpoint{0.000000in}{0.000000in}}%
\pgfpathlineto{\pgfqpoint{8.000000in}{0.000000in}}%
\pgfpathlineto{\pgfqpoint{8.000000in}{5.000000in}}%
\pgfpathlineto{\pgfqpoint{0.000000in}{5.000000in}}%
\pgfpathlineto{\pgfqpoint{0.000000in}{0.000000in}}%
\pgfpathclose%
\pgfusepath{}%
\end{pgfscope}%
\begin{pgfscope}%
\pgfsetbuttcap%
\pgfsetmiterjoin%
\definecolor{currentfill}{rgb}{1.000000,1.000000,1.000000}%
\pgfsetfillcolor{currentfill}%
\pgfsetlinewidth{0.000000pt}%
\definecolor{currentstroke}{rgb}{0.000000,0.000000,0.000000}%
\pgfsetstrokecolor{currentstroke}%
\pgfsetstrokeopacity{0.000000}%
\pgfsetdash{}{0pt}%
\pgfpathmoveto{\pgfqpoint{1.000000in}{0.625000in}}%
\pgfpathlineto{\pgfqpoint{7.200000in}{0.625000in}}%
\pgfpathlineto{\pgfqpoint{7.200000in}{4.400000in}}%
\pgfpathlineto{\pgfqpoint{1.000000in}{4.400000in}}%
\pgfpathlineto{\pgfqpoint{1.000000in}{0.625000in}}%
\pgfpathclose%
\pgfusepath{fill}%
\end{pgfscope}%
\begin{pgfscope}%
\pgfpathrectangle{\pgfqpoint{1.000000in}{0.625000in}}{\pgfqpoint{6.200000in}{3.775000in}}%
\pgfusepath{clip}%
\pgfsetrectcap%
\pgfsetroundjoin%
\pgfsetlinewidth{0.803000pt}%
\definecolor{currentstroke}{rgb}{0.690196,0.690196,0.690196}%
\pgfsetstrokecolor{currentstroke}%
\pgfsetdash{}{0pt}%
\pgfpathmoveto{\pgfqpoint{1.751515in}{0.625000in}}%
\pgfpathlineto{\pgfqpoint{1.751515in}{4.400000in}}%
\pgfusepath{stroke}%
\end{pgfscope}%
\begin{pgfscope}%
\pgfsetbuttcap%
\pgfsetroundjoin%
\definecolor{currentfill}{rgb}{0.000000,0.000000,0.000000}%
\pgfsetfillcolor{currentfill}%
\pgfsetlinewidth{0.803000pt}%
\definecolor{currentstroke}{rgb}{0.000000,0.000000,0.000000}%
\pgfsetstrokecolor{currentstroke}%
\pgfsetdash{}{0pt}%
\pgfsys@defobject{currentmarker}{\pgfqpoint{0.000000in}{-0.048611in}}{\pgfqpoint{0.000000in}{0.000000in}}{%
\pgfpathmoveto{\pgfqpoint{0.000000in}{0.000000in}}%
\pgfpathlineto{\pgfqpoint{0.000000in}{-0.048611in}}%
\pgfusepath{stroke,fill}%
}%
\begin{pgfscope}%
\pgfsys@transformshift{1.751515in}{0.625000in}%
\pgfsys@useobject{currentmarker}{}%
\end{pgfscope}%
\end{pgfscope}%
\begin{pgfscope}%
\definecolor{textcolor}{rgb}{0.000000,0.000000,0.000000}%
\pgfsetstrokecolor{textcolor}%
\pgfsetfillcolor{textcolor}%
\pgftext[x=1.751515in,y=0.527778in,,top]{\color{textcolor}\sffamily\fontsize{16.000000}{19.200000}\selectfont 0.25}%
\end{pgfscope}%
\begin{pgfscope}%
\pgfpathrectangle{\pgfqpoint{1.000000in}{0.625000in}}{\pgfqpoint{6.200000in}{3.775000in}}%
\pgfusepath{clip}%
\pgfsetrectcap%
\pgfsetroundjoin%
\pgfsetlinewidth{0.803000pt}%
\definecolor{currentstroke}{rgb}{0.690196,0.690196,0.690196}%
\pgfsetstrokecolor{currentstroke}%
\pgfsetdash{}{0pt}%
\pgfpathmoveto{\pgfqpoint{2.534343in}{0.625000in}}%
\pgfpathlineto{\pgfqpoint{2.534343in}{4.400000in}}%
\pgfusepath{stroke}%
\end{pgfscope}%
\begin{pgfscope}%
\pgfsetbuttcap%
\pgfsetroundjoin%
\definecolor{currentfill}{rgb}{0.000000,0.000000,0.000000}%
\pgfsetfillcolor{currentfill}%
\pgfsetlinewidth{0.803000pt}%
\definecolor{currentstroke}{rgb}{0.000000,0.000000,0.000000}%
\pgfsetstrokecolor{currentstroke}%
\pgfsetdash{}{0pt}%
\pgfsys@defobject{currentmarker}{\pgfqpoint{0.000000in}{-0.048611in}}{\pgfqpoint{0.000000in}{0.000000in}}{%
\pgfpathmoveto{\pgfqpoint{0.000000in}{0.000000in}}%
\pgfpathlineto{\pgfqpoint{0.000000in}{-0.048611in}}%
\pgfusepath{stroke,fill}%
}%
\begin{pgfscope}%
\pgfsys@transformshift{2.534343in}{0.625000in}%
\pgfsys@useobject{currentmarker}{}%
\end{pgfscope}%
\end{pgfscope}%
\begin{pgfscope}%
\definecolor{textcolor}{rgb}{0.000000,0.000000,0.000000}%
\pgfsetstrokecolor{textcolor}%
\pgfsetfillcolor{textcolor}%
\pgftext[x=2.534343in,y=0.527778in,,top]{\color{textcolor}\sffamily\fontsize{16.000000}{19.200000}\selectfont 0.50}%
\end{pgfscope}%
\begin{pgfscope}%
\pgfpathrectangle{\pgfqpoint{1.000000in}{0.625000in}}{\pgfqpoint{6.200000in}{3.775000in}}%
\pgfusepath{clip}%
\pgfsetrectcap%
\pgfsetroundjoin%
\pgfsetlinewidth{0.803000pt}%
\definecolor{currentstroke}{rgb}{0.690196,0.690196,0.690196}%
\pgfsetstrokecolor{currentstroke}%
\pgfsetdash{}{0pt}%
\pgfpathmoveto{\pgfqpoint{3.317172in}{0.625000in}}%
\pgfpathlineto{\pgfqpoint{3.317172in}{4.400000in}}%
\pgfusepath{stroke}%
\end{pgfscope}%
\begin{pgfscope}%
\pgfsetbuttcap%
\pgfsetroundjoin%
\definecolor{currentfill}{rgb}{0.000000,0.000000,0.000000}%
\pgfsetfillcolor{currentfill}%
\pgfsetlinewidth{0.803000pt}%
\definecolor{currentstroke}{rgb}{0.000000,0.000000,0.000000}%
\pgfsetstrokecolor{currentstroke}%
\pgfsetdash{}{0pt}%
\pgfsys@defobject{currentmarker}{\pgfqpoint{0.000000in}{-0.048611in}}{\pgfqpoint{0.000000in}{0.000000in}}{%
\pgfpathmoveto{\pgfqpoint{0.000000in}{0.000000in}}%
\pgfpathlineto{\pgfqpoint{0.000000in}{-0.048611in}}%
\pgfusepath{stroke,fill}%
}%
\begin{pgfscope}%
\pgfsys@transformshift{3.317172in}{0.625000in}%
\pgfsys@useobject{currentmarker}{}%
\end{pgfscope}%
\end{pgfscope}%
\begin{pgfscope}%
\definecolor{textcolor}{rgb}{0.000000,0.000000,0.000000}%
\pgfsetstrokecolor{textcolor}%
\pgfsetfillcolor{textcolor}%
\pgftext[x=3.317172in,y=0.527778in,,top]{\color{textcolor}\sffamily\fontsize{16.000000}{19.200000}\selectfont 0.75}%
\end{pgfscope}%
\begin{pgfscope}%
\pgfpathrectangle{\pgfqpoint{1.000000in}{0.625000in}}{\pgfqpoint{6.200000in}{3.775000in}}%
\pgfusepath{clip}%
\pgfsetrectcap%
\pgfsetroundjoin%
\pgfsetlinewidth{0.803000pt}%
\definecolor{currentstroke}{rgb}{0.690196,0.690196,0.690196}%
\pgfsetstrokecolor{currentstroke}%
\pgfsetdash{}{0pt}%
\pgfpathmoveto{\pgfqpoint{4.100000in}{0.625000in}}%
\pgfpathlineto{\pgfqpoint{4.100000in}{4.400000in}}%
\pgfusepath{stroke}%
\end{pgfscope}%
\begin{pgfscope}%
\pgfsetbuttcap%
\pgfsetroundjoin%
\definecolor{currentfill}{rgb}{0.000000,0.000000,0.000000}%
\pgfsetfillcolor{currentfill}%
\pgfsetlinewidth{0.803000pt}%
\definecolor{currentstroke}{rgb}{0.000000,0.000000,0.000000}%
\pgfsetstrokecolor{currentstroke}%
\pgfsetdash{}{0pt}%
\pgfsys@defobject{currentmarker}{\pgfqpoint{0.000000in}{-0.048611in}}{\pgfqpoint{0.000000in}{0.000000in}}{%
\pgfpathmoveto{\pgfqpoint{0.000000in}{0.000000in}}%
\pgfpathlineto{\pgfqpoint{0.000000in}{-0.048611in}}%
\pgfusepath{stroke,fill}%
}%
\begin{pgfscope}%
\pgfsys@transformshift{4.100000in}{0.625000in}%
\pgfsys@useobject{currentmarker}{}%
\end{pgfscope}%
\end{pgfscope}%
\begin{pgfscope}%
\definecolor{textcolor}{rgb}{0.000000,0.000000,0.000000}%
\pgfsetstrokecolor{textcolor}%
\pgfsetfillcolor{textcolor}%
\pgftext[x=4.100000in,y=0.527778in,,top]{\color{textcolor}\sffamily\fontsize{16.000000}{19.200000}\selectfont 1.00}%
\end{pgfscope}%
\begin{pgfscope}%
\pgfpathrectangle{\pgfqpoint{1.000000in}{0.625000in}}{\pgfqpoint{6.200000in}{3.775000in}}%
\pgfusepath{clip}%
\pgfsetrectcap%
\pgfsetroundjoin%
\pgfsetlinewidth{0.803000pt}%
\definecolor{currentstroke}{rgb}{0.690196,0.690196,0.690196}%
\pgfsetstrokecolor{currentstroke}%
\pgfsetdash{}{0pt}%
\pgfpathmoveto{\pgfqpoint{4.882828in}{0.625000in}}%
\pgfpathlineto{\pgfqpoint{4.882828in}{4.400000in}}%
\pgfusepath{stroke}%
\end{pgfscope}%
\begin{pgfscope}%
\pgfsetbuttcap%
\pgfsetroundjoin%
\definecolor{currentfill}{rgb}{0.000000,0.000000,0.000000}%
\pgfsetfillcolor{currentfill}%
\pgfsetlinewidth{0.803000pt}%
\definecolor{currentstroke}{rgb}{0.000000,0.000000,0.000000}%
\pgfsetstrokecolor{currentstroke}%
\pgfsetdash{}{0pt}%
\pgfsys@defobject{currentmarker}{\pgfqpoint{0.000000in}{-0.048611in}}{\pgfqpoint{0.000000in}{0.000000in}}{%
\pgfpathmoveto{\pgfqpoint{0.000000in}{0.000000in}}%
\pgfpathlineto{\pgfqpoint{0.000000in}{-0.048611in}}%
\pgfusepath{stroke,fill}%
}%
\begin{pgfscope}%
\pgfsys@transformshift{4.882828in}{0.625000in}%
\pgfsys@useobject{currentmarker}{}%
\end{pgfscope}%
\end{pgfscope}%
\begin{pgfscope}%
\definecolor{textcolor}{rgb}{0.000000,0.000000,0.000000}%
\pgfsetstrokecolor{textcolor}%
\pgfsetfillcolor{textcolor}%
\pgftext[x=4.882828in,y=0.527778in,,top]{\color{textcolor}\sffamily\fontsize{16.000000}{19.200000}\selectfont 1.25}%
\end{pgfscope}%
\begin{pgfscope}%
\pgfpathrectangle{\pgfqpoint{1.000000in}{0.625000in}}{\pgfqpoint{6.200000in}{3.775000in}}%
\pgfusepath{clip}%
\pgfsetrectcap%
\pgfsetroundjoin%
\pgfsetlinewidth{0.803000pt}%
\definecolor{currentstroke}{rgb}{0.690196,0.690196,0.690196}%
\pgfsetstrokecolor{currentstroke}%
\pgfsetdash{}{0pt}%
\pgfpathmoveto{\pgfqpoint{5.665657in}{0.625000in}}%
\pgfpathlineto{\pgfqpoint{5.665657in}{4.400000in}}%
\pgfusepath{stroke}%
\end{pgfscope}%
\begin{pgfscope}%
\pgfsetbuttcap%
\pgfsetroundjoin%
\definecolor{currentfill}{rgb}{0.000000,0.000000,0.000000}%
\pgfsetfillcolor{currentfill}%
\pgfsetlinewidth{0.803000pt}%
\definecolor{currentstroke}{rgb}{0.000000,0.000000,0.000000}%
\pgfsetstrokecolor{currentstroke}%
\pgfsetdash{}{0pt}%
\pgfsys@defobject{currentmarker}{\pgfqpoint{0.000000in}{-0.048611in}}{\pgfqpoint{0.000000in}{0.000000in}}{%
\pgfpathmoveto{\pgfqpoint{0.000000in}{0.000000in}}%
\pgfpathlineto{\pgfqpoint{0.000000in}{-0.048611in}}%
\pgfusepath{stroke,fill}%
}%
\begin{pgfscope}%
\pgfsys@transformshift{5.665657in}{0.625000in}%
\pgfsys@useobject{currentmarker}{}%
\end{pgfscope}%
\end{pgfscope}%
\begin{pgfscope}%
\definecolor{textcolor}{rgb}{0.000000,0.000000,0.000000}%
\pgfsetstrokecolor{textcolor}%
\pgfsetfillcolor{textcolor}%
\pgftext[x=5.665657in,y=0.527778in,,top]{\color{textcolor}\sffamily\fontsize{16.000000}{19.200000}\selectfont 1.50}%
\end{pgfscope}%
\begin{pgfscope}%
\pgfpathrectangle{\pgfqpoint{1.000000in}{0.625000in}}{\pgfqpoint{6.200000in}{3.775000in}}%
\pgfusepath{clip}%
\pgfsetrectcap%
\pgfsetroundjoin%
\pgfsetlinewidth{0.803000pt}%
\definecolor{currentstroke}{rgb}{0.690196,0.690196,0.690196}%
\pgfsetstrokecolor{currentstroke}%
\pgfsetdash{}{0pt}%
\pgfpathmoveto{\pgfqpoint{6.448485in}{0.625000in}}%
\pgfpathlineto{\pgfqpoint{6.448485in}{4.400000in}}%
\pgfusepath{stroke}%
\end{pgfscope}%
\begin{pgfscope}%
\pgfsetbuttcap%
\pgfsetroundjoin%
\definecolor{currentfill}{rgb}{0.000000,0.000000,0.000000}%
\pgfsetfillcolor{currentfill}%
\pgfsetlinewidth{0.803000pt}%
\definecolor{currentstroke}{rgb}{0.000000,0.000000,0.000000}%
\pgfsetstrokecolor{currentstroke}%
\pgfsetdash{}{0pt}%
\pgfsys@defobject{currentmarker}{\pgfqpoint{0.000000in}{-0.048611in}}{\pgfqpoint{0.000000in}{0.000000in}}{%
\pgfpathmoveto{\pgfqpoint{0.000000in}{0.000000in}}%
\pgfpathlineto{\pgfqpoint{0.000000in}{-0.048611in}}%
\pgfusepath{stroke,fill}%
}%
\begin{pgfscope}%
\pgfsys@transformshift{6.448485in}{0.625000in}%
\pgfsys@useobject{currentmarker}{}%
\end{pgfscope}%
\end{pgfscope}%
\begin{pgfscope}%
\definecolor{textcolor}{rgb}{0.000000,0.000000,0.000000}%
\pgfsetstrokecolor{textcolor}%
\pgfsetfillcolor{textcolor}%
\pgftext[x=6.448485in,y=0.527778in,,top]{\color{textcolor}\sffamily\fontsize{16.000000}{19.200000}\selectfont 1.75}%
\end{pgfscope}%
\begin{pgfscope}%
\definecolor{textcolor}{rgb}{0.000000,0.000000,0.000000}%
\pgfsetstrokecolor{textcolor}%
\pgfsetfillcolor{textcolor}%
\pgftext[x=4.100000in,y=0.257162in,,top]{\color{textcolor}\sffamily\fontsize{20.000000}{24.000000}\selectfont \(\displaystyle \omega\)}%
\end{pgfscope}%
\begin{pgfscope}%
\pgfpathrectangle{\pgfqpoint{1.000000in}{0.625000in}}{\pgfqpoint{6.200000in}{3.775000in}}%
\pgfusepath{clip}%
\pgfsetrectcap%
\pgfsetroundjoin%
\pgfsetlinewidth{0.803000pt}%
\definecolor{currentstroke}{rgb}{0.690196,0.690196,0.690196}%
\pgfsetstrokecolor{currentstroke}%
\pgfsetdash{}{0pt}%
\pgfpathmoveto{\pgfqpoint{1.000000in}{0.787058in}}%
\pgfpathlineto{\pgfqpoint{7.200000in}{0.787058in}}%
\pgfusepath{stroke}%
\end{pgfscope}%
\begin{pgfscope}%
\pgfsetbuttcap%
\pgfsetroundjoin%
\definecolor{currentfill}{rgb}{0.000000,0.000000,0.000000}%
\pgfsetfillcolor{currentfill}%
\pgfsetlinewidth{0.803000pt}%
\definecolor{currentstroke}{rgb}{0.000000,0.000000,0.000000}%
\pgfsetstrokecolor{currentstroke}%
\pgfsetdash{}{0pt}%
\pgfsys@defobject{currentmarker}{\pgfqpoint{-0.048611in}{0.000000in}}{\pgfqpoint{-0.000000in}{0.000000in}}{%
\pgfpathmoveto{\pgfqpoint{-0.000000in}{0.000000in}}%
\pgfpathlineto{\pgfqpoint{-0.048611in}{0.000000in}}%
\pgfusepath{stroke,fill}%
}%
\begin{pgfscope}%
\pgfsys@transformshift{1.000000in}{0.787058in}%
\pgfsys@useobject{currentmarker}{}%
\end{pgfscope}%
\end{pgfscope}%
\begin{pgfscope}%
\definecolor{textcolor}{rgb}{0.000000,0.000000,0.000000}%
\pgfsetstrokecolor{textcolor}%
\pgfsetfillcolor{textcolor}%
\pgftext[x=0.549371in, y=0.702640in, left, base]{\color{textcolor}\sffamily\fontsize{16.000000}{19.200000}\selectfont 0.0}%
\end{pgfscope}%
\begin{pgfscope}%
\pgfpathrectangle{\pgfqpoint{1.000000in}{0.625000in}}{\pgfqpoint{6.200000in}{3.775000in}}%
\pgfusepath{clip}%
\pgfsetrectcap%
\pgfsetroundjoin%
\pgfsetlinewidth{0.803000pt}%
\definecolor{currentstroke}{rgb}{0.690196,0.690196,0.690196}%
\pgfsetstrokecolor{currentstroke}%
\pgfsetdash{}{0pt}%
\pgfpathmoveto{\pgfqpoint{1.000000in}{1.330429in}}%
\pgfpathlineto{\pgfqpoint{7.200000in}{1.330429in}}%
\pgfusepath{stroke}%
\end{pgfscope}%
\begin{pgfscope}%
\pgfsetbuttcap%
\pgfsetroundjoin%
\definecolor{currentfill}{rgb}{0.000000,0.000000,0.000000}%
\pgfsetfillcolor{currentfill}%
\pgfsetlinewidth{0.803000pt}%
\definecolor{currentstroke}{rgb}{0.000000,0.000000,0.000000}%
\pgfsetstrokecolor{currentstroke}%
\pgfsetdash{}{0pt}%
\pgfsys@defobject{currentmarker}{\pgfqpoint{-0.048611in}{0.000000in}}{\pgfqpoint{-0.000000in}{0.000000in}}{%
\pgfpathmoveto{\pgfqpoint{-0.000000in}{0.000000in}}%
\pgfpathlineto{\pgfqpoint{-0.048611in}{0.000000in}}%
\pgfusepath{stroke,fill}%
}%
\begin{pgfscope}%
\pgfsys@transformshift{1.000000in}{1.330429in}%
\pgfsys@useobject{currentmarker}{}%
\end{pgfscope}%
\end{pgfscope}%
\begin{pgfscope}%
\definecolor{textcolor}{rgb}{0.000000,0.000000,0.000000}%
\pgfsetstrokecolor{textcolor}%
\pgfsetfillcolor{textcolor}%
\pgftext[x=0.549371in, y=1.246011in, left, base]{\color{textcolor}\sffamily\fontsize{16.000000}{19.200000}\selectfont 0.5}%
\end{pgfscope}%
\begin{pgfscope}%
\pgfpathrectangle{\pgfqpoint{1.000000in}{0.625000in}}{\pgfqpoint{6.200000in}{3.775000in}}%
\pgfusepath{clip}%
\pgfsetrectcap%
\pgfsetroundjoin%
\pgfsetlinewidth{0.803000pt}%
\definecolor{currentstroke}{rgb}{0.690196,0.690196,0.690196}%
\pgfsetstrokecolor{currentstroke}%
\pgfsetdash{}{0pt}%
\pgfpathmoveto{\pgfqpoint{1.000000in}{1.873801in}}%
\pgfpathlineto{\pgfqpoint{7.200000in}{1.873801in}}%
\pgfusepath{stroke}%
\end{pgfscope}%
\begin{pgfscope}%
\pgfsetbuttcap%
\pgfsetroundjoin%
\definecolor{currentfill}{rgb}{0.000000,0.000000,0.000000}%
\pgfsetfillcolor{currentfill}%
\pgfsetlinewidth{0.803000pt}%
\definecolor{currentstroke}{rgb}{0.000000,0.000000,0.000000}%
\pgfsetstrokecolor{currentstroke}%
\pgfsetdash{}{0pt}%
\pgfsys@defobject{currentmarker}{\pgfqpoint{-0.048611in}{0.000000in}}{\pgfqpoint{-0.000000in}{0.000000in}}{%
\pgfpathmoveto{\pgfqpoint{-0.000000in}{0.000000in}}%
\pgfpathlineto{\pgfqpoint{-0.048611in}{0.000000in}}%
\pgfusepath{stroke,fill}%
}%
\begin{pgfscope}%
\pgfsys@transformshift{1.000000in}{1.873801in}%
\pgfsys@useobject{currentmarker}{}%
\end{pgfscope}%
\end{pgfscope}%
\begin{pgfscope}%
\definecolor{textcolor}{rgb}{0.000000,0.000000,0.000000}%
\pgfsetstrokecolor{textcolor}%
\pgfsetfillcolor{textcolor}%
\pgftext[x=0.549371in, y=1.789382in, left, base]{\color{textcolor}\sffamily\fontsize{16.000000}{19.200000}\selectfont 1.0}%
\end{pgfscope}%
\begin{pgfscope}%
\pgfpathrectangle{\pgfqpoint{1.000000in}{0.625000in}}{\pgfqpoint{6.200000in}{3.775000in}}%
\pgfusepath{clip}%
\pgfsetrectcap%
\pgfsetroundjoin%
\pgfsetlinewidth{0.803000pt}%
\definecolor{currentstroke}{rgb}{0.690196,0.690196,0.690196}%
\pgfsetstrokecolor{currentstroke}%
\pgfsetdash{}{0pt}%
\pgfpathmoveto{\pgfqpoint{1.000000in}{2.417172in}}%
\pgfpathlineto{\pgfqpoint{7.200000in}{2.417172in}}%
\pgfusepath{stroke}%
\end{pgfscope}%
\begin{pgfscope}%
\pgfsetbuttcap%
\pgfsetroundjoin%
\definecolor{currentfill}{rgb}{0.000000,0.000000,0.000000}%
\pgfsetfillcolor{currentfill}%
\pgfsetlinewidth{0.803000pt}%
\definecolor{currentstroke}{rgb}{0.000000,0.000000,0.000000}%
\pgfsetstrokecolor{currentstroke}%
\pgfsetdash{}{0pt}%
\pgfsys@defobject{currentmarker}{\pgfqpoint{-0.048611in}{0.000000in}}{\pgfqpoint{-0.000000in}{0.000000in}}{%
\pgfpathmoveto{\pgfqpoint{-0.000000in}{0.000000in}}%
\pgfpathlineto{\pgfqpoint{-0.048611in}{0.000000in}}%
\pgfusepath{stroke,fill}%
}%
\begin{pgfscope}%
\pgfsys@transformshift{1.000000in}{2.417172in}%
\pgfsys@useobject{currentmarker}{}%
\end{pgfscope}%
\end{pgfscope}%
\begin{pgfscope}%
\definecolor{textcolor}{rgb}{0.000000,0.000000,0.000000}%
\pgfsetstrokecolor{textcolor}%
\pgfsetfillcolor{textcolor}%
\pgftext[x=0.549371in, y=2.332753in, left, base]{\color{textcolor}\sffamily\fontsize{16.000000}{19.200000}\selectfont 1.5}%
\end{pgfscope}%
\begin{pgfscope}%
\pgfpathrectangle{\pgfqpoint{1.000000in}{0.625000in}}{\pgfqpoint{6.200000in}{3.775000in}}%
\pgfusepath{clip}%
\pgfsetrectcap%
\pgfsetroundjoin%
\pgfsetlinewidth{0.803000pt}%
\definecolor{currentstroke}{rgb}{0.690196,0.690196,0.690196}%
\pgfsetstrokecolor{currentstroke}%
\pgfsetdash{}{0pt}%
\pgfpathmoveto{\pgfqpoint{1.000000in}{2.960543in}}%
\pgfpathlineto{\pgfqpoint{7.200000in}{2.960543in}}%
\pgfusepath{stroke}%
\end{pgfscope}%
\begin{pgfscope}%
\pgfsetbuttcap%
\pgfsetroundjoin%
\definecolor{currentfill}{rgb}{0.000000,0.000000,0.000000}%
\pgfsetfillcolor{currentfill}%
\pgfsetlinewidth{0.803000pt}%
\definecolor{currentstroke}{rgb}{0.000000,0.000000,0.000000}%
\pgfsetstrokecolor{currentstroke}%
\pgfsetdash{}{0pt}%
\pgfsys@defobject{currentmarker}{\pgfqpoint{-0.048611in}{0.000000in}}{\pgfqpoint{-0.000000in}{0.000000in}}{%
\pgfpathmoveto{\pgfqpoint{-0.000000in}{0.000000in}}%
\pgfpathlineto{\pgfqpoint{-0.048611in}{0.000000in}}%
\pgfusepath{stroke,fill}%
}%
\begin{pgfscope}%
\pgfsys@transformshift{1.000000in}{2.960543in}%
\pgfsys@useobject{currentmarker}{}%
\end{pgfscope}%
\end{pgfscope}%
\begin{pgfscope}%
\definecolor{textcolor}{rgb}{0.000000,0.000000,0.000000}%
\pgfsetstrokecolor{textcolor}%
\pgfsetfillcolor{textcolor}%
\pgftext[x=0.549371in, y=2.876125in, left, base]{\color{textcolor}\sffamily\fontsize{16.000000}{19.200000}\selectfont 2.0}%
\end{pgfscope}%
\begin{pgfscope}%
\pgfpathrectangle{\pgfqpoint{1.000000in}{0.625000in}}{\pgfqpoint{6.200000in}{3.775000in}}%
\pgfusepath{clip}%
\pgfsetrectcap%
\pgfsetroundjoin%
\pgfsetlinewidth{0.803000pt}%
\definecolor{currentstroke}{rgb}{0.690196,0.690196,0.690196}%
\pgfsetstrokecolor{currentstroke}%
\pgfsetdash{}{0pt}%
\pgfpathmoveto{\pgfqpoint{1.000000in}{3.503914in}}%
\pgfpathlineto{\pgfqpoint{7.200000in}{3.503914in}}%
\pgfusepath{stroke}%
\end{pgfscope}%
\begin{pgfscope}%
\pgfsetbuttcap%
\pgfsetroundjoin%
\definecolor{currentfill}{rgb}{0.000000,0.000000,0.000000}%
\pgfsetfillcolor{currentfill}%
\pgfsetlinewidth{0.803000pt}%
\definecolor{currentstroke}{rgb}{0.000000,0.000000,0.000000}%
\pgfsetstrokecolor{currentstroke}%
\pgfsetdash{}{0pt}%
\pgfsys@defobject{currentmarker}{\pgfqpoint{-0.048611in}{0.000000in}}{\pgfqpoint{-0.000000in}{0.000000in}}{%
\pgfpathmoveto{\pgfqpoint{-0.000000in}{0.000000in}}%
\pgfpathlineto{\pgfqpoint{-0.048611in}{0.000000in}}%
\pgfusepath{stroke,fill}%
}%
\begin{pgfscope}%
\pgfsys@transformshift{1.000000in}{3.503914in}%
\pgfsys@useobject{currentmarker}{}%
\end{pgfscope}%
\end{pgfscope}%
\begin{pgfscope}%
\definecolor{textcolor}{rgb}{0.000000,0.000000,0.000000}%
\pgfsetstrokecolor{textcolor}%
\pgfsetfillcolor{textcolor}%
\pgftext[x=0.549371in, y=3.419496in, left, base]{\color{textcolor}\sffamily\fontsize{16.000000}{19.200000}\selectfont 2.5}%
\end{pgfscope}%
\begin{pgfscope}%
\pgfpathrectangle{\pgfqpoint{1.000000in}{0.625000in}}{\pgfqpoint{6.200000in}{3.775000in}}%
\pgfusepath{clip}%
\pgfsetrectcap%
\pgfsetroundjoin%
\pgfsetlinewidth{0.803000pt}%
\definecolor{currentstroke}{rgb}{0.690196,0.690196,0.690196}%
\pgfsetstrokecolor{currentstroke}%
\pgfsetdash{}{0pt}%
\pgfpathmoveto{\pgfqpoint{1.000000in}{4.047285in}}%
\pgfpathlineto{\pgfqpoint{7.200000in}{4.047285in}}%
\pgfusepath{stroke}%
\end{pgfscope}%
\begin{pgfscope}%
\pgfsetbuttcap%
\pgfsetroundjoin%
\definecolor{currentfill}{rgb}{0.000000,0.000000,0.000000}%
\pgfsetfillcolor{currentfill}%
\pgfsetlinewidth{0.803000pt}%
\definecolor{currentstroke}{rgb}{0.000000,0.000000,0.000000}%
\pgfsetstrokecolor{currentstroke}%
\pgfsetdash{}{0pt}%
\pgfsys@defobject{currentmarker}{\pgfqpoint{-0.048611in}{0.000000in}}{\pgfqpoint{-0.000000in}{0.000000in}}{%
\pgfpathmoveto{\pgfqpoint{-0.000000in}{0.000000in}}%
\pgfpathlineto{\pgfqpoint{-0.048611in}{0.000000in}}%
\pgfusepath{stroke,fill}%
}%
\begin{pgfscope}%
\pgfsys@transformshift{1.000000in}{4.047285in}%
\pgfsys@useobject{currentmarker}{}%
\end{pgfscope}%
\end{pgfscope}%
\begin{pgfscope}%
\definecolor{textcolor}{rgb}{0.000000,0.000000,0.000000}%
\pgfsetstrokecolor{textcolor}%
\pgfsetfillcolor{textcolor}%
\pgftext[x=0.549371in, y=3.962867in, left, base]{\color{textcolor}\sffamily\fontsize{16.000000}{19.200000}\selectfont 3.0}%
\end{pgfscope}%
\begin{pgfscope}%
\definecolor{textcolor}{rgb}{0.000000,0.000000,0.000000}%
\pgfsetstrokecolor{textcolor}%
\pgfsetfillcolor{textcolor}%
\pgftext[x=0.125001in, y=2.512500in, left, base,rotate=90.000000]{\color{textcolor}\sffamily\fontsize{20.000000}{24.000000}\selectfont \(\displaystyle \)}%
\end{pgfscope}%
\begin{pgfscope}%
\definecolor{textcolor}{rgb}{0.000000,0.000000,0.000000}%
\pgfsetstrokecolor{textcolor}%
\pgfsetfillcolor{textcolor}%
\pgftext[x=0.436035in, y=2.424474in, left, base,rotate=90.000000]{\color{textcolor}\sffamily\fontsize{20.000000}{24.000000}\selectfont u\(\displaystyle \)}%
\end{pgfscope}%
\begin{pgfscope}%
\pgfpathrectangle{\pgfqpoint{1.000000in}{0.625000in}}{\pgfqpoint{6.200000in}{3.775000in}}%
\pgfusepath{clip}%
\pgfsetbuttcap%
\pgfsetroundjoin%
\pgfsetlinewidth{1.505625pt}%
\definecolor{currentstroke}{rgb}{1.000000,0.000000,0.000000}%
\pgfsetstrokecolor{currentstroke}%
\pgfsetdash{{5.550000pt}{2.400000pt}}{0.000000pt}%
\pgfpathmoveto{\pgfqpoint{1.281818in}{4.228409in}}%
\pgfpathlineto{\pgfqpoint{1.908081in}{1.813426in}}%
\pgfpathlineto{\pgfqpoint{2.534343in}{1.330429in}}%
\pgfpathlineto{\pgfqpoint{3.160606in}{1.123431in}}%
\pgfpathlineto{\pgfqpoint{3.786869in}{1.008432in}}%
\pgfpathlineto{\pgfqpoint{4.413131in}{0.935250in}}%
\pgfpathlineto{\pgfqpoint{5.039394in}{0.884586in}}%
\pgfpathlineto{\pgfqpoint{5.665657in}{0.847433in}}%
\pgfpathlineto{\pgfqpoint{6.291919in}{0.819021in}}%
\pgfpathlineto{\pgfqpoint{6.918182in}{0.796591in}}%
\pgfusepath{stroke}%
\end{pgfscope}%
\begin{pgfscope}%
\pgfpathrectangle{\pgfqpoint{1.000000in}{0.625000in}}{\pgfqpoint{6.200000in}{3.775000in}}%
\pgfusepath{clip}%
\pgfsetbuttcap%
\pgfsetroundjoin%
\definecolor{currentfill}{rgb}{1.000000,0.000000,0.000000}%
\pgfsetfillcolor{currentfill}%
\pgfsetlinewidth{1.003750pt}%
\definecolor{currentstroke}{rgb}{1.000000,0.000000,0.000000}%
\pgfsetstrokecolor{currentstroke}%
\pgfsetdash{}{0pt}%
\pgfsys@defobject{currentmarker}{\pgfqpoint{-0.041667in}{-0.041667in}}{\pgfqpoint{0.041667in}{0.041667in}}{%
\pgfpathmoveto{\pgfqpoint{0.000000in}{-0.041667in}}%
\pgfpathcurveto{\pgfqpoint{0.011050in}{-0.041667in}}{\pgfqpoint{0.021649in}{-0.037276in}}{\pgfqpoint{0.029463in}{-0.029463in}}%
\pgfpathcurveto{\pgfqpoint{0.037276in}{-0.021649in}}{\pgfqpoint{0.041667in}{-0.011050in}}{\pgfqpoint{0.041667in}{0.000000in}}%
\pgfpathcurveto{\pgfqpoint{0.041667in}{0.011050in}}{\pgfqpoint{0.037276in}{0.021649in}}{\pgfqpoint{0.029463in}{0.029463in}}%
\pgfpathcurveto{\pgfqpoint{0.021649in}{0.037276in}}{\pgfqpoint{0.011050in}{0.041667in}}{\pgfqpoint{0.000000in}{0.041667in}}%
\pgfpathcurveto{\pgfqpoint{-0.011050in}{0.041667in}}{\pgfqpoint{-0.021649in}{0.037276in}}{\pgfqpoint{-0.029463in}{0.029463in}}%
\pgfpathcurveto{\pgfqpoint{-0.037276in}{0.021649in}}{\pgfqpoint{-0.041667in}{0.011050in}}{\pgfqpoint{-0.041667in}{0.000000in}}%
\pgfpathcurveto{\pgfqpoint{-0.041667in}{-0.011050in}}{\pgfqpoint{-0.037276in}{-0.021649in}}{\pgfqpoint{-0.029463in}{-0.029463in}}%
\pgfpathcurveto{\pgfqpoint{-0.021649in}{-0.037276in}}{\pgfqpoint{-0.011050in}{-0.041667in}}{\pgfqpoint{0.000000in}{-0.041667in}}%
\pgfpathlineto{\pgfqpoint{0.000000in}{-0.041667in}}%
\pgfpathclose%
\pgfusepath{stroke,fill}%
}%
\begin{pgfscope}%
\pgfsys@transformshift{1.281818in}{4.228409in}%
\pgfsys@useobject{currentmarker}{}%
\end{pgfscope}%
\begin{pgfscope}%
\pgfsys@transformshift{1.908081in}{1.813426in}%
\pgfsys@useobject{currentmarker}{}%
\end{pgfscope}%
\begin{pgfscope}%
\pgfsys@transformshift{2.534343in}{1.330429in}%
\pgfsys@useobject{currentmarker}{}%
\end{pgfscope}%
\begin{pgfscope}%
\pgfsys@transformshift{3.160606in}{1.123431in}%
\pgfsys@useobject{currentmarker}{}%
\end{pgfscope}%
\begin{pgfscope}%
\pgfsys@transformshift{3.786869in}{1.008432in}%
\pgfsys@useobject{currentmarker}{}%
\end{pgfscope}%
\begin{pgfscope}%
\pgfsys@transformshift{4.413131in}{0.935250in}%
\pgfsys@useobject{currentmarker}{}%
\end{pgfscope}%
\begin{pgfscope}%
\pgfsys@transformshift{5.039394in}{0.884586in}%
\pgfsys@useobject{currentmarker}{}%
\end{pgfscope}%
\begin{pgfscope}%
\pgfsys@transformshift{5.665657in}{0.847433in}%
\pgfsys@useobject{currentmarker}{}%
\end{pgfscope}%
\begin{pgfscope}%
\pgfsys@transformshift{6.291919in}{0.819021in}%
\pgfsys@useobject{currentmarker}{}%
\end{pgfscope}%
\begin{pgfscope}%
\pgfsys@transformshift{6.918182in}{0.796591in}%
\pgfsys@useobject{currentmarker}{}%
\end{pgfscope}%
\end{pgfscope}%
\begin{pgfscope}%
\pgfpathrectangle{\pgfqpoint{1.000000in}{0.625000in}}{\pgfqpoint{6.200000in}{3.775000in}}%
\pgfusepath{clip}%
\pgfsetbuttcap%
\pgfsetroundjoin%
\pgfsetlinewidth{1.505625pt}%
\definecolor{currentstroke}{rgb}{0.000000,0.000000,1.000000}%
\pgfsetstrokecolor{currentstroke}%
\pgfsetdash{{5.550000pt}{2.400000pt}}{0.000000pt}%
\pgfpathmoveto{\pgfqpoint{1.281818in}{1.789179in}}%
\pgfpathlineto{\pgfqpoint{1.908081in}{1.622232in}}%
\pgfpathlineto{\pgfqpoint{2.534343in}{1.300182in}}%
\pgfpathlineto{\pgfqpoint{3.160606in}{1.117490in}}%
\pgfpathlineto{\pgfqpoint{3.786869in}{1.007633in}}%
\pgfpathlineto{\pgfqpoint{4.413131in}{0.935609in}}%
\pgfpathlineto{\pgfqpoint{5.039394in}{0.885095in}}%
\pgfpathlineto{\pgfqpoint{5.665657in}{0.847827in}}%
\pgfpathlineto{\pgfqpoint{6.291919in}{0.819250in}}%
\pgfpathlineto{\pgfqpoint{6.918182in}{0.796684in}}%
\pgfusepath{stroke}%
\end{pgfscope}%
\begin{pgfscope}%
\pgfpathrectangle{\pgfqpoint{1.000000in}{0.625000in}}{\pgfqpoint{6.200000in}{3.775000in}}%
\pgfusepath{clip}%
\pgfsetbuttcap%
\pgfsetroundjoin%
\definecolor{currentfill}{rgb}{0.000000,0.000000,1.000000}%
\pgfsetfillcolor{currentfill}%
\pgfsetlinewidth{1.003750pt}%
\definecolor{currentstroke}{rgb}{0.000000,0.000000,1.000000}%
\pgfsetstrokecolor{currentstroke}%
\pgfsetdash{}{0pt}%
\pgfsys@defobject{currentmarker}{\pgfqpoint{-0.041667in}{-0.041667in}}{\pgfqpoint{0.041667in}{0.041667in}}{%
\pgfpathmoveto{\pgfqpoint{0.000000in}{-0.041667in}}%
\pgfpathcurveto{\pgfqpoint{0.011050in}{-0.041667in}}{\pgfqpoint{0.021649in}{-0.037276in}}{\pgfqpoint{0.029463in}{-0.029463in}}%
\pgfpathcurveto{\pgfqpoint{0.037276in}{-0.021649in}}{\pgfqpoint{0.041667in}{-0.011050in}}{\pgfqpoint{0.041667in}{0.000000in}}%
\pgfpathcurveto{\pgfqpoint{0.041667in}{0.011050in}}{\pgfqpoint{0.037276in}{0.021649in}}{\pgfqpoint{0.029463in}{0.029463in}}%
\pgfpathcurveto{\pgfqpoint{0.021649in}{0.037276in}}{\pgfqpoint{0.011050in}{0.041667in}}{\pgfqpoint{0.000000in}{0.041667in}}%
\pgfpathcurveto{\pgfqpoint{-0.011050in}{0.041667in}}{\pgfqpoint{-0.021649in}{0.037276in}}{\pgfqpoint{-0.029463in}{0.029463in}}%
\pgfpathcurveto{\pgfqpoint{-0.037276in}{0.021649in}}{\pgfqpoint{-0.041667in}{0.011050in}}{\pgfqpoint{-0.041667in}{0.000000in}}%
\pgfpathcurveto{\pgfqpoint{-0.041667in}{-0.011050in}}{\pgfqpoint{-0.037276in}{-0.021649in}}{\pgfqpoint{-0.029463in}{-0.029463in}}%
\pgfpathcurveto{\pgfqpoint{-0.021649in}{-0.037276in}}{\pgfqpoint{-0.011050in}{-0.041667in}}{\pgfqpoint{0.000000in}{-0.041667in}}%
\pgfpathlineto{\pgfqpoint{0.000000in}{-0.041667in}}%
\pgfpathclose%
\pgfusepath{stroke,fill}%
}%
\begin{pgfscope}%
\pgfsys@transformshift{1.281818in}{1.789179in}%
\pgfsys@useobject{currentmarker}{}%
\end{pgfscope}%
\begin{pgfscope}%
\pgfsys@transformshift{1.908081in}{1.622232in}%
\pgfsys@useobject{currentmarker}{}%
\end{pgfscope}%
\begin{pgfscope}%
\pgfsys@transformshift{2.534343in}{1.300182in}%
\pgfsys@useobject{currentmarker}{}%
\end{pgfscope}%
\begin{pgfscope}%
\pgfsys@transformshift{3.160606in}{1.117490in}%
\pgfsys@useobject{currentmarker}{}%
\end{pgfscope}%
\begin{pgfscope}%
\pgfsys@transformshift{3.786869in}{1.007633in}%
\pgfsys@useobject{currentmarker}{}%
\end{pgfscope}%
\begin{pgfscope}%
\pgfsys@transformshift{4.413131in}{0.935609in}%
\pgfsys@useobject{currentmarker}{}%
\end{pgfscope}%
\begin{pgfscope}%
\pgfsys@transformshift{5.039394in}{0.885095in}%
\pgfsys@useobject{currentmarker}{}%
\end{pgfscope}%
\begin{pgfscope}%
\pgfsys@transformshift{5.665657in}{0.847827in}%
\pgfsys@useobject{currentmarker}{}%
\end{pgfscope}%
\begin{pgfscope}%
\pgfsys@transformshift{6.291919in}{0.819250in}%
\pgfsys@useobject{currentmarker}{}%
\end{pgfscope}%
\begin{pgfscope}%
\pgfsys@transformshift{6.918182in}{0.796684in}%
\pgfsys@useobject{currentmarker}{}%
\end{pgfscope}%
\end{pgfscope}%
\begin{pgfscope}%
\pgfsetrectcap%
\pgfsetmiterjoin%
\pgfsetlinewidth{0.803000pt}%
\definecolor{currentstroke}{rgb}{0.000000,0.000000,0.000000}%
\pgfsetstrokecolor{currentstroke}%
\pgfsetdash{}{0pt}%
\pgfpathmoveto{\pgfqpoint{1.000000in}{0.625000in}}%
\pgfpathlineto{\pgfqpoint{1.000000in}{4.400000in}}%
\pgfusepath{stroke}%
\end{pgfscope}%
\begin{pgfscope}%
\pgfsetrectcap%
\pgfsetmiterjoin%
\pgfsetlinewidth{0.803000pt}%
\definecolor{currentstroke}{rgb}{0.000000,0.000000,0.000000}%
\pgfsetstrokecolor{currentstroke}%
\pgfsetdash{}{0pt}%
\pgfpathmoveto{\pgfqpoint{7.200000in}{0.625000in}}%
\pgfpathlineto{\pgfqpoint{7.200000in}{4.400000in}}%
\pgfusepath{stroke}%
\end{pgfscope}%
\begin{pgfscope}%
\pgfsetrectcap%
\pgfsetmiterjoin%
\pgfsetlinewidth{0.803000pt}%
\definecolor{currentstroke}{rgb}{0.000000,0.000000,0.000000}%
\pgfsetstrokecolor{currentstroke}%
\pgfsetdash{}{0pt}%
\pgfpathmoveto{\pgfqpoint{1.000000in}{0.625000in}}%
\pgfpathlineto{\pgfqpoint{7.200000in}{0.625000in}}%
\pgfusepath{stroke}%
\end{pgfscope}%
\begin{pgfscope}%
\pgfsetrectcap%
\pgfsetmiterjoin%
\pgfsetlinewidth{0.803000pt}%
\definecolor{currentstroke}{rgb}{0.000000,0.000000,0.000000}%
\pgfsetstrokecolor{currentstroke}%
\pgfsetdash{}{0pt}%
\pgfpathmoveto{\pgfqpoint{1.000000in}{4.400000in}}%
\pgfpathlineto{\pgfqpoint{7.200000in}{4.400000in}}%
\pgfusepath{stroke}%
\end{pgfscope}%
\begin{pgfscope}%
\pgfsetbuttcap%
\pgfsetmiterjoin%
\definecolor{currentfill}{rgb}{1.000000,1.000000,1.000000}%
\pgfsetfillcolor{currentfill}%
\pgfsetfillopacity{0.800000}%
\pgfsetlinewidth{1.003750pt}%
\definecolor{currentstroke}{rgb}{0.800000,0.800000,0.800000}%
\pgfsetstrokecolor{currentstroke}%
\pgfsetstrokeopacity{0.800000}%
\pgfsetdash{}{0pt}%
\pgfpathmoveto{\pgfqpoint{3.036019in}{3.362349in}}%
\pgfpathlineto{\pgfqpoint{7.005556in}{3.362349in}}%
\pgfpathquadraticcurveto{\pgfqpoint{7.061111in}{3.362349in}}{\pgfqpoint{7.061111in}{3.417904in}}%
\pgfpathlineto{\pgfqpoint{7.061111in}{4.205556in}}%
\pgfpathquadraticcurveto{\pgfqpoint{7.061111in}{4.261111in}}{\pgfqpoint{7.005556in}{4.261111in}}%
\pgfpathlineto{\pgfqpoint{3.036019in}{4.261111in}}%
\pgfpathquadraticcurveto{\pgfqpoint{2.980463in}{4.261111in}}{\pgfqpoint{2.980463in}{4.205556in}}%
\pgfpathlineto{\pgfqpoint{2.980463in}{3.417904in}}%
\pgfpathquadraticcurveto{\pgfqpoint{2.980463in}{3.362349in}}{\pgfqpoint{3.036019in}{3.362349in}}%
\pgfpathlineto{\pgfqpoint{3.036019in}{3.362349in}}%
\pgfpathclose%
\pgfusepath{stroke,fill}%
\end{pgfscope}%
\begin{pgfscope}%
\pgfsetbuttcap%
\pgfsetroundjoin%
\pgfsetlinewidth{1.505625pt}%
\definecolor{currentstroke}{rgb}{1.000000,0.000000,0.000000}%
\pgfsetstrokecolor{currentstroke}%
\pgfsetdash{{5.550000pt}{2.400000pt}}{0.000000pt}%
\pgfpathmoveto{\pgfqpoint{3.091574in}{4.036176in}}%
\pgfpathlineto{\pgfqpoint{3.369352in}{4.036176in}}%
\pgfpathlineto{\pgfqpoint{3.647130in}{4.036176in}}%
\pgfusepath{stroke}%
\end{pgfscope}%
\begin{pgfscope}%
\pgfsetbuttcap%
\pgfsetroundjoin%
\definecolor{currentfill}{rgb}{1.000000,0.000000,0.000000}%
\pgfsetfillcolor{currentfill}%
\pgfsetlinewidth{1.003750pt}%
\definecolor{currentstroke}{rgb}{1.000000,0.000000,0.000000}%
\pgfsetstrokecolor{currentstroke}%
\pgfsetdash{}{0pt}%
\pgfsys@defobject{currentmarker}{\pgfqpoint{-0.041667in}{-0.041667in}}{\pgfqpoint{0.041667in}{0.041667in}}{%
\pgfpathmoveto{\pgfqpoint{0.000000in}{-0.041667in}}%
\pgfpathcurveto{\pgfqpoint{0.011050in}{-0.041667in}}{\pgfqpoint{0.021649in}{-0.037276in}}{\pgfqpoint{0.029463in}{-0.029463in}}%
\pgfpathcurveto{\pgfqpoint{0.037276in}{-0.021649in}}{\pgfqpoint{0.041667in}{-0.011050in}}{\pgfqpoint{0.041667in}{0.000000in}}%
\pgfpathcurveto{\pgfqpoint{0.041667in}{0.011050in}}{\pgfqpoint{0.037276in}{0.021649in}}{\pgfqpoint{0.029463in}{0.029463in}}%
\pgfpathcurveto{\pgfqpoint{0.021649in}{0.037276in}}{\pgfqpoint{0.011050in}{0.041667in}}{\pgfqpoint{0.000000in}{0.041667in}}%
\pgfpathcurveto{\pgfqpoint{-0.011050in}{0.041667in}}{\pgfqpoint{-0.021649in}{0.037276in}}{\pgfqpoint{-0.029463in}{0.029463in}}%
\pgfpathcurveto{\pgfqpoint{-0.037276in}{0.021649in}}{\pgfqpoint{-0.041667in}{0.011050in}}{\pgfqpoint{-0.041667in}{0.000000in}}%
\pgfpathcurveto{\pgfqpoint{-0.041667in}{-0.011050in}}{\pgfqpoint{-0.037276in}{-0.021649in}}{\pgfqpoint{-0.029463in}{-0.029463in}}%
\pgfpathcurveto{\pgfqpoint{-0.021649in}{-0.037276in}}{\pgfqpoint{-0.011050in}{-0.041667in}}{\pgfqpoint{0.000000in}{-0.041667in}}%
\pgfpathlineto{\pgfqpoint{0.000000in}{-0.041667in}}%
\pgfpathclose%
\pgfusepath{stroke,fill}%
}%
\begin{pgfscope}%
\pgfsys@transformshift{3.369352in}{4.036176in}%
\pgfsys@useobject{currentmarker}{}%
\end{pgfscope}%
\end{pgfscope}%
\begin{pgfscope}%
\definecolor{textcolor}{rgb}{0.000000,0.000000,0.000000}%
\pgfsetstrokecolor{textcolor}%
\pgfsetfillcolor{textcolor}%
\pgftext[x=3.869352in,y=3.938954in,left,base]{\color{textcolor}\sffamily\fontsize{20.000000}{24.000000}\selectfont Theoretical viscocities}%
\end{pgfscope}%
\begin{pgfscope}%
\pgfsetbuttcap%
\pgfsetroundjoin%
\pgfsetlinewidth{1.505625pt}%
\definecolor{currentstroke}{rgb}{0.000000,0.000000,1.000000}%
\pgfsetstrokecolor{currentstroke}%
\pgfsetdash{{5.550000pt}{2.400000pt}}{0.000000pt}%
\pgfpathmoveto{\pgfqpoint{3.091574in}{3.628462in}}%
\pgfpathlineto{\pgfqpoint{3.369352in}{3.628462in}}%
\pgfpathlineto{\pgfqpoint{3.647130in}{3.628462in}}%
\pgfusepath{stroke}%
\end{pgfscope}%
\begin{pgfscope}%
\pgfsetbuttcap%
\pgfsetroundjoin%
\definecolor{currentfill}{rgb}{0.000000,0.000000,1.000000}%
\pgfsetfillcolor{currentfill}%
\pgfsetlinewidth{1.003750pt}%
\definecolor{currentstroke}{rgb}{0.000000,0.000000,1.000000}%
\pgfsetstrokecolor{currentstroke}%
\pgfsetdash{}{0pt}%
\pgfsys@defobject{currentmarker}{\pgfqpoint{-0.041667in}{-0.041667in}}{\pgfqpoint{0.041667in}{0.041667in}}{%
\pgfpathmoveto{\pgfqpoint{0.000000in}{-0.041667in}}%
\pgfpathcurveto{\pgfqpoint{0.011050in}{-0.041667in}}{\pgfqpoint{0.021649in}{-0.037276in}}{\pgfqpoint{0.029463in}{-0.029463in}}%
\pgfpathcurveto{\pgfqpoint{0.037276in}{-0.021649in}}{\pgfqpoint{0.041667in}{-0.011050in}}{\pgfqpoint{0.041667in}{0.000000in}}%
\pgfpathcurveto{\pgfqpoint{0.041667in}{0.011050in}}{\pgfqpoint{0.037276in}{0.021649in}}{\pgfqpoint{0.029463in}{0.029463in}}%
\pgfpathcurveto{\pgfqpoint{0.021649in}{0.037276in}}{\pgfqpoint{0.011050in}{0.041667in}}{\pgfqpoint{0.000000in}{0.041667in}}%
\pgfpathcurveto{\pgfqpoint{-0.011050in}{0.041667in}}{\pgfqpoint{-0.021649in}{0.037276in}}{\pgfqpoint{-0.029463in}{0.029463in}}%
\pgfpathcurveto{\pgfqpoint{-0.037276in}{0.021649in}}{\pgfqpoint{-0.041667in}{0.011050in}}{\pgfqpoint{-0.041667in}{0.000000in}}%
\pgfpathcurveto{\pgfqpoint{-0.041667in}{-0.011050in}}{\pgfqpoint{-0.037276in}{-0.021649in}}{\pgfqpoint{-0.029463in}{-0.029463in}}%
\pgfpathcurveto{\pgfqpoint{-0.021649in}{-0.037276in}}{\pgfqpoint{-0.011050in}{-0.041667in}}{\pgfqpoint{0.000000in}{-0.041667in}}%
\pgfpathlineto{\pgfqpoint{0.000000in}{-0.041667in}}%
\pgfpathclose%
\pgfusepath{stroke,fill}%
}%
\begin{pgfscope}%
\pgfsys@transformshift{3.369352in}{3.628462in}%
\pgfsys@useobject{currentmarker}{}%
\end{pgfscope}%
\end{pgfscope}%
\begin{pgfscope}%
\definecolor{textcolor}{rgb}{0.000000,0.000000,0.000000}%
\pgfsetstrokecolor{textcolor}%
\pgfsetfillcolor{textcolor}%
\pgftext[x=3.869352in,y=3.531239in,left,base]{\color{textcolor}\sffamily\fontsize{20.000000}{24.000000}\selectfont Numerical viscocities}%
\end{pgfscope}%
\end{pgfpicture}%
\makeatother%
\endgroup%
}
% \vspace*{-10mm}
\caption[Kinematic viscosity]{The numerical and theoretical kinematic viscosity for $0.1 \leq \omega \leq 1.9 $. For low $\omega$ the kinematic viscocities diverge.}
\label{fig:m3-3-kinematic-viscocity}
\end{figure}
For low values of $\omega$ the theoretical and numerical values of $\nu$ diverge quite significantly, while for values $\geq\omega$ the values are almost identical.

The reason for the divergence is the relative low physical space which leads to inaccuracies for high viscocities. Increasing the physical space to a $300\times300$ lattice alleviates the issue as shown in figure \ref{fig:m3-3-update}.
\begin{figure}[ht]
\centering
\resizebox{\columnwidth}{!}{\large%% Creator: Matplotlib, PGF backend
%%
%% To include the figure in your LaTeX document, write
%%   \input{<filename>.pgf}
%%
%% Make sure the required packages are loaded in your preamble
%%   \usepackage{pgf}
%%
%% Also ensure that all the required font packages are loaded; for instance,
%% the lmodern package is sometimes necessary when using math font.
%%   \usepackage{lmodern}
%%
%% Figures using additional raster images can only be included by \input if
%% they are in the same directory as the main LaTeX file. For loading figures
%% from other directories you can use the `import` package
%%   \usepackage{import}
%%
%% and then include the figures with
%%   \import{<path to file>}{<filename>.pgf}
%%
%% Matplotlib used the following preamble
%%   \usepackage{fontspec}
%%   \setmainfont{DejaVuSerif.ttf}[Path=\detokenize{/home/joe/miniconda3/envs/high/lib/python3.9/site-packages/matplotlib/mpl-data/fonts/ttf/}]
%%   \setsansfont{DejaVuSans.ttf}[Path=\detokenize{/home/joe/miniconda3/envs/high/lib/python3.9/site-packages/matplotlib/mpl-data/fonts/ttf/}]
%%   \setmonofont{DejaVuSansMono.ttf}[Path=\detokenize{/home/joe/miniconda3/envs/high/lib/python3.9/site-packages/matplotlib/mpl-data/fonts/ttf/}]
%%
\begingroup%
\makeatletter%
\begin{pgfpicture}%
\pgfpathrectangle{\pgfpointorigin}{\pgfqpoint{14.000000in}{5.000000in}}%
\pgfusepath{use as bounding box, clip}%
\begin{pgfscope}%
\pgfsetbuttcap%
\pgfsetmiterjoin%
\pgfsetlinewidth{0.000000pt}%
\definecolor{currentstroke}{rgb}{1.000000,1.000000,1.000000}%
\pgfsetstrokecolor{currentstroke}%
\pgfsetstrokeopacity{0.000000}%
\pgfsetdash{}{0pt}%
\pgfpathmoveto{\pgfqpoint{0.000000in}{0.000000in}}%
\pgfpathlineto{\pgfqpoint{14.000000in}{0.000000in}}%
\pgfpathlineto{\pgfqpoint{14.000000in}{5.000000in}}%
\pgfpathlineto{\pgfqpoint{0.000000in}{5.000000in}}%
\pgfpathlineto{\pgfqpoint{0.000000in}{0.000000in}}%
\pgfpathclose%
\pgfusepath{}%
\end{pgfscope}%
\begin{pgfscope}%
\pgfsetbuttcap%
\pgfsetmiterjoin%
\definecolor{currentfill}{rgb}{1.000000,1.000000,1.000000}%
\pgfsetfillcolor{currentfill}%
\pgfsetlinewidth{0.000000pt}%
\definecolor{currentstroke}{rgb}{0.000000,0.000000,0.000000}%
\pgfsetstrokecolor{currentstroke}%
\pgfsetstrokeopacity{0.000000}%
\pgfsetdash{}{0pt}%
\pgfpathmoveto{\pgfqpoint{1.750000in}{0.625000in}}%
\pgfpathlineto{\pgfqpoint{6.681818in}{0.625000in}}%
\pgfpathlineto{\pgfqpoint{6.681818in}{4.400000in}}%
\pgfpathlineto{\pgfqpoint{1.750000in}{4.400000in}}%
\pgfpathlineto{\pgfqpoint{1.750000in}{0.625000in}}%
\pgfpathclose%
\pgfusepath{fill}%
\end{pgfscope}%
\begin{pgfscope}%
\pgfsetbuttcap%
\pgfsetroundjoin%
\definecolor{currentfill}{rgb}{0.000000,0.000000,0.000000}%
\pgfsetfillcolor{currentfill}%
\pgfsetlinewidth{0.803000pt}%
\definecolor{currentstroke}{rgb}{0.000000,0.000000,0.000000}%
\pgfsetstrokecolor{currentstroke}%
\pgfsetdash{}{0pt}%
\pgfsys@defobject{currentmarker}{\pgfqpoint{0.000000in}{-0.048611in}}{\pgfqpoint{0.000000in}{0.000000in}}{%
\pgfpathmoveto{\pgfqpoint{0.000000in}{0.000000in}}%
\pgfpathlineto{\pgfqpoint{0.000000in}{-0.048611in}}%
\pgfusepath{stroke,fill}%
}%
\begin{pgfscope}%
\pgfsys@transformshift{1.974174in}{0.625000in}%
\pgfsys@useobject{currentmarker}{}%
\end{pgfscope}%
\end{pgfscope}%
\begin{pgfscope}%
\definecolor{textcolor}{rgb}{0.000000,0.000000,0.000000}%
\pgfsetstrokecolor{textcolor}%
\pgfsetfillcolor{textcolor}%
\pgftext[x=1.974174in,y=0.527778in,,top]{\color{textcolor}\sffamily\fontsize{16.000000}{19.200000}\selectfont 0}%
\end{pgfscope}%
\begin{pgfscope}%
\pgfsetbuttcap%
\pgfsetroundjoin%
\definecolor{currentfill}{rgb}{0.000000,0.000000,0.000000}%
\pgfsetfillcolor{currentfill}%
\pgfsetlinewidth{0.803000pt}%
\definecolor{currentstroke}{rgb}{0.000000,0.000000,0.000000}%
\pgfsetstrokecolor{currentstroke}%
\pgfsetdash{}{0pt}%
\pgfsys@defobject{currentmarker}{\pgfqpoint{0.000000in}{-0.048611in}}{\pgfqpoint{0.000000in}{0.000000in}}{%
\pgfpathmoveto{\pgfqpoint{0.000000in}{0.000000in}}%
\pgfpathlineto{\pgfqpoint{0.000000in}{-0.048611in}}%
\pgfusepath{stroke,fill}%
}%
\begin{pgfscope}%
\pgfsys@transformshift{2.721668in}{0.625000in}%
\pgfsys@useobject{currentmarker}{}%
\end{pgfscope}%
\end{pgfscope}%
\begin{pgfscope}%
\definecolor{textcolor}{rgb}{0.000000,0.000000,0.000000}%
\pgfsetstrokecolor{textcolor}%
\pgfsetfillcolor{textcolor}%
\pgftext[x=2.721668in,y=0.527778in,,top]{\color{textcolor}\sffamily\fontsize{16.000000}{19.200000}\selectfont 500}%
\end{pgfscope}%
\begin{pgfscope}%
\pgfsetbuttcap%
\pgfsetroundjoin%
\definecolor{currentfill}{rgb}{0.000000,0.000000,0.000000}%
\pgfsetfillcolor{currentfill}%
\pgfsetlinewidth{0.803000pt}%
\definecolor{currentstroke}{rgb}{0.000000,0.000000,0.000000}%
\pgfsetstrokecolor{currentstroke}%
\pgfsetdash{}{0pt}%
\pgfsys@defobject{currentmarker}{\pgfqpoint{0.000000in}{-0.048611in}}{\pgfqpoint{0.000000in}{0.000000in}}{%
\pgfpathmoveto{\pgfqpoint{0.000000in}{0.000000in}}%
\pgfpathlineto{\pgfqpoint{0.000000in}{-0.048611in}}%
\pgfusepath{stroke,fill}%
}%
\begin{pgfscope}%
\pgfsys@transformshift{3.469162in}{0.625000in}%
\pgfsys@useobject{currentmarker}{}%
\end{pgfscope}%
\end{pgfscope}%
\begin{pgfscope}%
\definecolor{textcolor}{rgb}{0.000000,0.000000,0.000000}%
\pgfsetstrokecolor{textcolor}%
\pgfsetfillcolor{textcolor}%
\pgftext[x=3.469162in,y=0.527778in,,top]{\color{textcolor}\sffamily\fontsize{16.000000}{19.200000}\selectfont 1000}%
\end{pgfscope}%
\begin{pgfscope}%
\pgfsetbuttcap%
\pgfsetroundjoin%
\definecolor{currentfill}{rgb}{0.000000,0.000000,0.000000}%
\pgfsetfillcolor{currentfill}%
\pgfsetlinewidth{0.803000pt}%
\definecolor{currentstroke}{rgb}{0.000000,0.000000,0.000000}%
\pgfsetstrokecolor{currentstroke}%
\pgfsetdash{}{0pt}%
\pgfsys@defobject{currentmarker}{\pgfqpoint{0.000000in}{-0.048611in}}{\pgfqpoint{0.000000in}{0.000000in}}{%
\pgfpathmoveto{\pgfqpoint{0.000000in}{0.000000in}}%
\pgfpathlineto{\pgfqpoint{0.000000in}{-0.048611in}}%
\pgfusepath{stroke,fill}%
}%
\begin{pgfscope}%
\pgfsys@transformshift{4.216657in}{0.625000in}%
\pgfsys@useobject{currentmarker}{}%
\end{pgfscope}%
\end{pgfscope}%
\begin{pgfscope}%
\definecolor{textcolor}{rgb}{0.000000,0.000000,0.000000}%
\pgfsetstrokecolor{textcolor}%
\pgfsetfillcolor{textcolor}%
\pgftext[x=4.216657in,y=0.527778in,,top]{\color{textcolor}\sffamily\fontsize{16.000000}{19.200000}\selectfont 1500}%
\end{pgfscope}%
\begin{pgfscope}%
\pgfsetbuttcap%
\pgfsetroundjoin%
\definecolor{currentfill}{rgb}{0.000000,0.000000,0.000000}%
\pgfsetfillcolor{currentfill}%
\pgfsetlinewidth{0.803000pt}%
\definecolor{currentstroke}{rgb}{0.000000,0.000000,0.000000}%
\pgfsetstrokecolor{currentstroke}%
\pgfsetdash{}{0pt}%
\pgfsys@defobject{currentmarker}{\pgfqpoint{0.000000in}{-0.048611in}}{\pgfqpoint{0.000000in}{0.000000in}}{%
\pgfpathmoveto{\pgfqpoint{0.000000in}{0.000000in}}%
\pgfpathlineto{\pgfqpoint{0.000000in}{-0.048611in}}%
\pgfusepath{stroke,fill}%
}%
\begin{pgfscope}%
\pgfsys@transformshift{4.964151in}{0.625000in}%
\pgfsys@useobject{currentmarker}{}%
\end{pgfscope}%
\end{pgfscope}%
\begin{pgfscope}%
\definecolor{textcolor}{rgb}{0.000000,0.000000,0.000000}%
\pgfsetstrokecolor{textcolor}%
\pgfsetfillcolor{textcolor}%
\pgftext[x=4.964151in,y=0.527778in,,top]{\color{textcolor}\sffamily\fontsize{16.000000}{19.200000}\selectfont 2000}%
\end{pgfscope}%
\begin{pgfscope}%
\pgfsetbuttcap%
\pgfsetroundjoin%
\definecolor{currentfill}{rgb}{0.000000,0.000000,0.000000}%
\pgfsetfillcolor{currentfill}%
\pgfsetlinewidth{0.803000pt}%
\definecolor{currentstroke}{rgb}{0.000000,0.000000,0.000000}%
\pgfsetstrokecolor{currentstroke}%
\pgfsetdash{}{0pt}%
\pgfsys@defobject{currentmarker}{\pgfqpoint{0.000000in}{-0.048611in}}{\pgfqpoint{0.000000in}{0.000000in}}{%
\pgfpathmoveto{\pgfqpoint{0.000000in}{0.000000in}}%
\pgfpathlineto{\pgfqpoint{0.000000in}{-0.048611in}}%
\pgfusepath{stroke,fill}%
}%
\begin{pgfscope}%
\pgfsys@transformshift{5.711645in}{0.625000in}%
\pgfsys@useobject{currentmarker}{}%
\end{pgfscope}%
\end{pgfscope}%
\begin{pgfscope}%
\definecolor{textcolor}{rgb}{0.000000,0.000000,0.000000}%
\pgfsetstrokecolor{textcolor}%
\pgfsetfillcolor{textcolor}%
\pgftext[x=5.711645in,y=0.527778in,,top]{\color{textcolor}\sffamily\fontsize{16.000000}{19.200000}\selectfont 2500}%
\end{pgfscope}%
\begin{pgfscope}%
\pgfsetbuttcap%
\pgfsetroundjoin%
\definecolor{currentfill}{rgb}{0.000000,0.000000,0.000000}%
\pgfsetfillcolor{currentfill}%
\pgfsetlinewidth{0.803000pt}%
\definecolor{currentstroke}{rgb}{0.000000,0.000000,0.000000}%
\pgfsetstrokecolor{currentstroke}%
\pgfsetdash{}{0pt}%
\pgfsys@defobject{currentmarker}{\pgfqpoint{0.000000in}{-0.048611in}}{\pgfqpoint{0.000000in}{0.000000in}}{%
\pgfpathmoveto{\pgfqpoint{0.000000in}{0.000000in}}%
\pgfpathlineto{\pgfqpoint{0.000000in}{-0.048611in}}%
\pgfusepath{stroke,fill}%
}%
\begin{pgfscope}%
\pgfsys@transformshift{6.459140in}{0.625000in}%
\pgfsys@useobject{currentmarker}{}%
\end{pgfscope}%
\end{pgfscope}%
\begin{pgfscope}%
\definecolor{textcolor}{rgb}{0.000000,0.000000,0.000000}%
\pgfsetstrokecolor{textcolor}%
\pgfsetfillcolor{textcolor}%
\pgftext[x=6.459140in,y=0.527778in,,top]{\color{textcolor}\sffamily\fontsize{16.000000}{19.200000}\selectfont 3000}%
\end{pgfscope}%
\begin{pgfscope}%
\definecolor{textcolor}{rgb}{0.000000,0.000000,0.000000}%
\pgfsetstrokecolor{textcolor}%
\pgfsetfillcolor{textcolor}%
\pgftext[x=4.215909in,y=0.257162in,,top]{\color{textcolor}\sffamily\fontsize{20.000000}{24.000000}\selectfont \(\displaystyle t\)}%
\end{pgfscope}%
\begin{pgfscope}%
\pgfsetbuttcap%
\pgfsetroundjoin%
\definecolor{currentfill}{rgb}{0.000000,0.000000,0.000000}%
\pgfsetfillcolor{currentfill}%
\pgfsetlinewidth{0.803000pt}%
\definecolor{currentstroke}{rgb}{0.000000,0.000000,0.000000}%
\pgfsetstrokecolor{currentstroke}%
\pgfsetdash{}{0pt}%
\pgfsys@defobject{currentmarker}{\pgfqpoint{-0.048611in}{0.000000in}}{\pgfqpoint{-0.000000in}{0.000000in}}{%
\pgfpathmoveto{\pgfqpoint{-0.000000in}{0.000000in}}%
\pgfpathlineto{\pgfqpoint{-0.048611in}{0.000000in}}%
\pgfusepath{stroke,fill}%
}%
\begin{pgfscope}%
\pgfsys@transformshift{1.750000in}{0.742498in}%
\pgfsys@useobject{currentmarker}{}%
\end{pgfscope}%
\end{pgfscope}%
\begin{pgfscope}%
\definecolor{textcolor}{rgb}{0.000000,0.000000,0.000000}%
\pgfsetstrokecolor{textcolor}%
\pgfsetfillcolor{textcolor}%
\pgftext[x=1.299371in, y=0.658079in, left, base]{\color{textcolor}\sffamily\fontsize{16.000000}{19.200000}\selectfont 0.0}%
\end{pgfscope}%
\begin{pgfscope}%
\pgfsetbuttcap%
\pgfsetroundjoin%
\definecolor{currentfill}{rgb}{0.000000,0.000000,0.000000}%
\pgfsetfillcolor{currentfill}%
\pgfsetlinewidth{0.803000pt}%
\definecolor{currentstroke}{rgb}{0.000000,0.000000,0.000000}%
\pgfsetstrokecolor{currentstroke}%
\pgfsetdash{}{0pt}%
\pgfsys@defobject{currentmarker}{\pgfqpoint{-0.048611in}{0.000000in}}{\pgfqpoint{-0.000000in}{0.000000in}}{%
\pgfpathmoveto{\pgfqpoint{-0.000000in}{0.000000in}}%
\pgfpathlineto{\pgfqpoint{-0.048611in}{0.000000in}}%
\pgfusepath{stroke,fill}%
}%
\begin{pgfscope}%
\pgfsys@transformshift{1.750000in}{1.439680in}%
\pgfsys@useobject{currentmarker}{}%
\end{pgfscope}%
\end{pgfscope}%
\begin{pgfscope}%
\definecolor{textcolor}{rgb}{0.000000,0.000000,0.000000}%
\pgfsetstrokecolor{textcolor}%
\pgfsetfillcolor{textcolor}%
\pgftext[x=1.299371in, y=1.355262in, left, base]{\color{textcolor}\sffamily\fontsize{16.000000}{19.200000}\selectfont 0.2}%
\end{pgfscope}%
\begin{pgfscope}%
\pgfsetbuttcap%
\pgfsetroundjoin%
\definecolor{currentfill}{rgb}{0.000000,0.000000,0.000000}%
\pgfsetfillcolor{currentfill}%
\pgfsetlinewidth{0.803000pt}%
\definecolor{currentstroke}{rgb}{0.000000,0.000000,0.000000}%
\pgfsetstrokecolor{currentstroke}%
\pgfsetdash{}{0pt}%
\pgfsys@defobject{currentmarker}{\pgfqpoint{-0.048611in}{0.000000in}}{\pgfqpoint{-0.000000in}{0.000000in}}{%
\pgfpathmoveto{\pgfqpoint{-0.000000in}{0.000000in}}%
\pgfpathlineto{\pgfqpoint{-0.048611in}{0.000000in}}%
\pgfusepath{stroke,fill}%
}%
\begin{pgfscope}%
\pgfsys@transformshift{1.750000in}{2.136862in}%
\pgfsys@useobject{currentmarker}{}%
\end{pgfscope}%
\end{pgfscope}%
\begin{pgfscope}%
\definecolor{textcolor}{rgb}{0.000000,0.000000,0.000000}%
\pgfsetstrokecolor{textcolor}%
\pgfsetfillcolor{textcolor}%
\pgftext[x=1.299371in, y=2.052444in, left, base]{\color{textcolor}\sffamily\fontsize{16.000000}{19.200000}\selectfont 0.4}%
\end{pgfscope}%
\begin{pgfscope}%
\pgfsetbuttcap%
\pgfsetroundjoin%
\definecolor{currentfill}{rgb}{0.000000,0.000000,0.000000}%
\pgfsetfillcolor{currentfill}%
\pgfsetlinewidth{0.803000pt}%
\definecolor{currentstroke}{rgb}{0.000000,0.000000,0.000000}%
\pgfsetstrokecolor{currentstroke}%
\pgfsetdash{}{0pt}%
\pgfsys@defobject{currentmarker}{\pgfqpoint{-0.048611in}{0.000000in}}{\pgfqpoint{-0.000000in}{0.000000in}}{%
\pgfpathmoveto{\pgfqpoint{-0.000000in}{0.000000in}}%
\pgfpathlineto{\pgfqpoint{-0.048611in}{0.000000in}}%
\pgfusepath{stroke,fill}%
}%
\begin{pgfscope}%
\pgfsys@transformshift{1.750000in}{2.834045in}%
\pgfsys@useobject{currentmarker}{}%
\end{pgfscope}%
\end{pgfscope}%
\begin{pgfscope}%
\definecolor{textcolor}{rgb}{0.000000,0.000000,0.000000}%
\pgfsetstrokecolor{textcolor}%
\pgfsetfillcolor{textcolor}%
\pgftext[x=1.299371in, y=2.749626in, left, base]{\color{textcolor}\sffamily\fontsize{16.000000}{19.200000}\selectfont 0.6}%
\end{pgfscope}%
\begin{pgfscope}%
\pgfsetbuttcap%
\pgfsetroundjoin%
\definecolor{currentfill}{rgb}{0.000000,0.000000,0.000000}%
\pgfsetfillcolor{currentfill}%
\pgfsetlinewidth{0.803000pt}%
\definecolor{currentstroke}{rgb}{0.000000,0.000000,0.000000}%
\pgfsetstrokecolor{currentstroke}%
\pgfsetdash{}{0pt}%
\pgfsys@defobject{currentmarker}{\pgfqpoint{-0.048611in}{0.000000in}}{\pgfqpoint{-0.000000in}{0.000000in}}{%
\pgfpathmoveto{\pgfqpoint{-0.000000in}{0.000000in}}%
\pgfpathlineto{\pgfqpoint{-0.048611in}{0.000000in}}%
\pgfusepath{stroke,fill}%
}%
\begin{pgfscope}%
\pgfsys@transformshift{1.750000in}{3.531227in}%
\pgfsys@useobject{currentmarker}{}%
\end{pgfscope}%
\end{pgfscope}%
\begin{pgfscope}%
\definecolor{textcolor}{rgb}{0.000000,0.000000,0.000000}%
\pgfsetstrokecolor{textcolor}%
\pgfsetfillcolor{textcolor}%
\pgftext[x=1.299371in, y=3.446808in, left, base]{\color{textcolor}\sffamily\fontsize{16.000000}{19.200000}\selectfont 0.8}%
\end{pgfscope}%
\begin{pgfscope}%
\pgfsetbuttcap%
\pgfsetroundjoin%
\definecolor{currentfill}{rgb}{0.000000,0.000000,0.000000}%
\pgfsetfillcolor{currentfill}%
\pgfsetlinewidth{0.803000pt}%
\definecolor{currentstroke}{rgb}{0.000000,0.000000,0.000000}%
\pgfsetstrokecolor{currentstroke}%
\pgfsetdash{}{0pt}%
\pgfsys@defobject{currentmarker}{\pgfqpoint{-0.048611in}{0.000000in}}{\pgfqpoint{-0.000000in}{0.000000in}}{%
\pgfpathmoveto{\pgfqpoint{-0.000000in}{0.000000in}}%
\pgfpathlineto{\pgfqpoint{-0.048611in}{0.000000in}}%
\pgfusepath{stroke,fill}%
}%
\begin{pgfscope}%
\pgfsys@transformshift{1.750000in}{4.228409in}%
\pgfsys@useobject{currentmarker}{}%
\end{pgfscope}%
\end{pgfscope}%
\begin{pgfscope}%
\definecolor{textcolor}{rgb}{0.000000,0.000000,0.000000}%
\pgfsetstrokecolor{textcolor}%
\pgfsetfillcolor{textcolor}%
\pgftext[x=1.299371in, y=4.143991in, left, base]{\color{textcolor}\sffamily\fontsize{16.000000}{19.200000}\selectfont 1.0}%
\end{pgfscope}%
\begin{pgfscope}%
\definecolor{textcolor}{rgb}{0.000000,0.000000,0.000000}%
\pgfsetstrokecolor{textcolor}%
\pgfsetfillcolor{textcolor}%
\pgftext[x=1.243815in,y=2.512500in,,bottom,rotate=90.000000]{\color{textcolor}\sffamily\fontsize{20.000000}{24.000000}\selectfont \(\displaystyle a(t)/a(0)\)}%
\end{pgfscope}%
\begin{pgfscope}%
\pgfpathrectangle{\pgfqpoint{1.750000in}{0.625000in}}{\pgfqpoint{4.931818in}{3.775000in}}%
\pgfusepath{clip}%
\pgfsetrectcap%
\pgfsetroundjoin%
\pgfsetlinewidth{1.505625pt}%
\definecolor{currentstroke}{rgb}{0.121569,0.466667,0.705882}%
\pgfsetstrokecolor{currentstroke}%
\pgfsetdash{}{0pt}%
\pgfpathmoveto{\pgfqpoint{1.974174in}{4.228409in}}%
\pgfpathlineto{\pgfqpoint{2.033973in}{4.040007in}}%
\pgfpathlineto{\pgfqpoint{2.093773in}{3.861787in}}%
\pgfpathlineto{\pgfqpoint{2.153572in}{3.693199in}}%
\pgfpathlineto{\pgfqpoint{2.213372in}{3.533723in}}%
\pgfpathlineto{\pgfqpoint{2.273171in}{3.382866in}}%
\pgfpathlineto{\pgfqpoint{2.332971in}{3.240163in}}%
\pgfpathlineto{\pgfqpoint{2.394265in}{3.101892in}}%
\pgfpathlineto{\pgfqpoint{2.455560in}{2.971276in}}%
\pgfpathlineto{\pgfqpoint{2.516854in}{2.847891in}}%
\pgfpathlineto{\pgfqpoint{2.578149in}{2.731337in}}%
\pgfpathlineto{\pgfqpoint{2.640939in}{2.618627in}}%
\pgfpathlineto{\pgfqpoint{2.703728in}{2.512305in}}%
\pgfpathlineto{\pgfqpoint{2.766518in}{2.412008in}}%
\pgfpathlineto{\pgfqpoint{2.829307in}{2.317395in}}%
\pgfpathlineto{\pgfqpoint{2.893592in}{2.226081in}}%
\pgfpathlineto{\pgfqpoint{2.957876in}{2.140062in}}%
\pgfpathlineto{\pgfqpoint{3.023656in}{2.057203in}}%
\pgfpathlineto{\pgfqpoint{3.089435in}{1.979257in}}%
\pgfpathlineto{\pgfqpoint{3.156710in}{1.904317in}}%
\pgfpathlineto{\pgfqpoint{3.223984in}{1.833917in}}%
\pgfpathlineto{\pgfqpoint{3.292754in}{1.766361in}}%
\pgfpathlineto{\pgfqpoint{3.363018in}{1.701652in}}%
\pgfpathlineto{\pgfqpoint{3.434778in}{1.639786in}}%
\pgfpathlineto{\pgfqpoint{3.508032in}{1.580746in}}%
\pgfpathlineto{\pgfqpoint{3.582781in}{1.524503in}}%
\pgfpathlineto{\pgfqpoint{3.659026in}{1.471021in}}%
\pgfpathlineto{\pgfqpoint{3.736765in}{1.420255in}}%
\pgfpathlineto{\pgfqpoint{3.816000in}{1.372151in}}%
\pgfpathlineto{\pgfqpoint{3.898224in}{1.325838in}}%
\pgfpathlineto{\pgfqpoint{3.983438in}{1.281432in}}%
\pgfpathlineto{\pgfqpoint{4.070148in}{1.239716in}}%
\pgfpathlineto{\pgfqpoint{4.159847in}{1.199956in}}%
\pgfpathlineto{\pgfqpoint{4.254031in}{1.161625in}}%
\pgfpathlineto{\pgfqpoint{4.351206in}{1.125441in}}%
\pgfpathlineto{\pgfqpoint{4.452865in}{1.090926in}}%
\pgfpathlineto{\pgfqpoint{4.559009in}{1.058203in}}%
\pgfpathlineto{\pgfqpoint{4.669638in}{1.027363in}}%
\pgfpathlineto{\pgfqpoint{4.786247in}{0.998113in}}%
\pgfpathlineto{\pgfqpoint{4.908836in}{0.970594in}}%
\pgfpathlineto{\pgfqpoint{5.037405in}{0.944911in}}%
\pgfpathlineto{\pgfqpoint{5.174944in}{0.920628in}}%
\pgfpathlineto{\pgfqpoint{5.321453in}{0.897958in}}%
\pgfpathlineto{\pgfqpoint{5.476932in}{0.877047in}}%
\pgfpathlineto{\pgfqpoint{5.644371in}{0.857661in}}%
\pgfpathlineto{\pgfqpoint{5.826759in}{0.839709in}}%
\pgfpathlineto{\pgfqpoint{6.025593in}{0.823311in}}%
\pgfpathlineto{\pgfqpoint{6.243861in}{0.808477in}}%
\pgfpathlineto{\pgfqpoint{6.457645in}{0.796591in}}%
\pgfpathlineto{\pgfqpoint{6.457645in}{0.796591in}}%
\pgfusepath{stroke}%
\end{pgfscope}%
\begin{pgfscope}%
\pgfpathrectangle{\pgfqpoint{1.750000in}{0.625000in}}{\pgfqpoint{4.931818in}{3.775000in}}%
\pgfusepath{clip}%
\pgfsetrectcap%
\pgfsetroundjoin%
\pgfsetlinewidth{1.505625pt}%
\definecolor{currentstroke}{rgb}{1.000000,0.498039,0.054902}%
\pgfsetstrokecolor{currentstroke}%
\pgfsetdash{}{0pt}%
\pgfpathmoveto{\pgfqpoint{1.974174in}{4.228409in}}%
\pgfpathlineto{\pgfqpoint{1.978659in}{4.226315in}}%
\pgfpathlineto{\pgfqpoint{1.984638in}{4.218462in}}%
\pgfpathlineto{\pgfqpoint{1.993608in}{4.199809in}}%
\pgfpathlineto{\pgfqpoint{2.008558in}{4.159672in}}%
\pgfpathlineto{\pgfqpoint{2.051913in}{4.029542in}}%
\pgfpathlineto{\pgfqpoint{2.114702in}{3.845776in}}%
\pgfpathlineto{\pgfqpoint{2.174502in}{3.680256in}}%
\pgfpathlineto{\pgfqpoint{2.234302in}{3.523565in}}%
\pgfpathlineto{\pgfqpoint{2.295596in}{3.371626in}}%
\pgfpathlineto{\pgfqpoint{2.356891in}{3.227987in}}%
\pgfpathlineto{\pgfqpoint{2.418185in}{3.092197in}}%
\pgfpathlineto{\pgfqpoint{2.479480in}{2.963825in}}%
\pgfpathlineto{\pgfqpoint{2.540774in}{2.842466in}}%
\pgfpathlineto{\pgfqpoint{2.603564in}{2.725019in}}%
\pgfpathlineto{\pgfqpoint{2.666353in}{2.614141in}}%
\pgfpathlineto{\pgfqpoint{2.729143in}{2.509464in}}%
\pgfpathlineto{\pgfqpoint{2.791932in}{2.410641in}}%
\pgfpathlineto{\pgfqpoint{2.856217in}{2.315189in}}%
\pgfpathlineto{\pgfqpoint{2.920501in}{2.225199in}}%
\pgfpathlineto{\pgfqpoint{2.986281in}{2.138443in}}%
\pgfpathlineto{\pgfqpoint{3.052060in}{2.056764in}}%
\pgfpathlineto{\pgfqpoint{3.119335in}{1.978170in}}%
\pgfpathlineto{\pgfqpoint{3.186609in}{1.904275in}}%
\pgfpathlineto{\pgfqpoint{3.255379in}{1.833304in}}%
\pgfpathlineto{\pgfqpoint{3.325643in}{1.765266in}}%
\pgfpathlineto{\pgfqpoint{3.395908in}{1.701472in}}%
\pgfpathlineto{\pgfqpoint{3.467667in}{1.640425in}}%
\pgfpathlineto{\pgfqpoint{3.540922in}{1.582114in}}%
\pgfpathlineto{\pgfqpoint{3.615671in}{1.526514in}}%
\pgfpathlineto{\pgfqpoint{3.691916in}{1.473593in}}%
\pgfpathlineto{\pgfqpoint{3.771150in}{1.422379in}}%
\pgfpathlineto{\pgfqpoint{3.851879in}{1.373886in}}%
\pgfpathlineto{\pgfqpoint{3.935599in}{1.327248in}}%
\pgfpathlineto{\pgfqpoint{4.020813in}{1.283313in}}%
\pgfpathlineto{\pgfqpoint{4.109017in}{1.241310in}}%
\pgfpathlineto{\pgfqpoint{4.200212in}{1.201310in}}%
\pgfpathlineto{\pgfqpoint{4.294396in}{1.163363in}}%
\pgfpathlineto{\pgfqpoint{4.393065in}{1.126970in}}%
\pgfpathlineto{\pgfqpoint{4.494724in}{1.092763in}}%
\pgfpathlineto{\pgfqpoint{4.600869in}{1.060291in}}%
\pgfpathlineto{\pgfqpoint{4.712993in}{1.029252in}}%
\pgfpathlineto{\pgfqpoint{4.829602in}{1.000184in}}%
\pgfpathlineto{\pgfqpoint{4.953686in}{0.972481in}}%
\pgfpathlineto{\pgfqpoint{5.083750in}{0.946635in}}%
\pgfpathlineto{\pgfqpoint{5.221289in}{0.922456in}}%
\pgfpathlineto{\pgfqpoint{5.367798in}{0.899842in}}%
\pgfpathlineto{\pgfqpoint{5.524772in}{0.878756in}}%
\pgfpathlineto{\pgfqpoint{5.693705in}{0.859210in}}%
\pgfpathlineto{\pgfqpoint{5.876094in}{0.841242in}}%
\pgfpathlineto{\pgfqpoint{6.074928in}{0.824790in}}%
\pgfpathlineto{\pgfqpoint{6.293196in}{0.809869in}}%
\pgfpathlineto{\pgfqpoint{6.457645in}{0.800442in}}%
\pgfpathlineto{\pgfqpoint{6.457645in}{0.800442in}}%
\pgfusepath{stroke}%
\end{pgfscope}%
\begin{pgfscope}%
\pgfpathrectangle{\pgfqpoint{1.750000in}{0.625000in}}{\pgfqpoint{4.931818in}{3.775000in}}%
\pgfusepath{clip}%
\pgfsetrectcap%
\pgfsetroundjoin%
\pgfsetlinewidth{1.505625pt}%
\definecolor{currentstroke}{rgb}{0.172549,0.627451,0.172549}%
\pgfsetstrokecolor{currentstroke}%
\pgfsetdash{}{0pt}%
\pgfpathmoveto{\pgfqpoint{1.974174in}{4.228409in}}%
\pgfpathlineto{\pgfqpoint{2.101248in}{4.107793in}}%
\pgfpathlineto{\pgfqpoint{2.229817in}{3.990005in}}%
\pgfpathlineto{\pgfqpoint{2.359881in}{3.875041in}}%
\pgfpathlineto{\pgfqpoint{2.491440in}{3.762896in}}%
\pgfpathlineto{\pgfqpoint{2.624494in}{3.653559in}}%
\pgfpathlineto{\pgfqpoint{2.759043in}{3.547019in}}%
\pgfpathlineto{\pgfqpoint{2.895087in}{3.443258in}}%
\pgfpathlineto{\pgfqpoint{3.032626in}{3.342259in}}%
\pgfpathlineto{\pgfqpoint{3.171659in}{3.244000in}}%
\pgfpathlineto{\pgfqpoint{3.312188in}{3.148458in}}%
\pgfpathlineto{\pgfqpoint{3.454212in}{3.055607in}}%
\pgfpathlineto{\pgfqpoint{3.599226in}{2.964497in}}%
\pgfpathlineto{\pgfqpoint{3.745735in}{2.876092in}}%
\pgfpathlineto{\pgfqpoint{3.895234in}{2.789508in}}%
\pgfpathlineto{\pgfqpoint{4.046228in}{2.705623in}}%
\pgfpathlineto{\pgfqpoint{4.200212in}{2.623617in}}%
\pgfpathlineto{\pgfqpoint{4.357186in}{2.543544in}}%
\pgfpathlineto{\pgfqpoint{4.515654in}{2.466165in}}%
\pgfpathlineto{\pgfqpoint{4.677113in}{2.390744in}}%
\pgfpathlineto{\pgfqpoint{4.841562in}{2.317318in}}%
\pgfpathlineto{\pgfqpoint{5.009001in}{2.245917in}}%
\pgfpathlineto{\pgfqpoint{5.180924in}{2.175970in}}%
\pgfpathlineto{\pgfqpoint{5.355838in}{2.108145in}}%
\pgfpathlineto{\pgfqpoint{5.535237in}{2.041914in}}%
\pgfpathlineto{\pgfqpoint{5.717625in}{1.977871in}}%
\pgfpathlineto{\pgfqpoint{5.904499in}{1.915525in}}%
\pgfpathlineto{\pgfqpoint{6.095857in}{1.854943in}}%
\pgfpathlineto{\pgfqpoint{6.291701in}{1.796179in}}%
\pgfpathlineto{\pgfqpoint{6.457645in}{1.748822in}}%
\pgfpathlineto{\pgfqpoint{6.457645in}{1.748822in}}%
\pgfusepath{stroke}%
\end{pgfscope}%
\begin{pgfscope}%
\pgfpathrectangle{\pgfqpoint{1.750000in}{0.625000in}}{\pgfqpoint{4.931818in}{3.775000in}}%
\pgfusepath{clip}%
\pgfsetrectcap%
\pgfsetroundjoin%
\pgfsetlinewidth{1.505625pt}%
\definecolor{currentstroke}{rgb}{0.839216,0.152941,0.156863}%
\pgfsetstrokecolor{currentstroke}%
\pgfsetdash{}{0pt}%
\pgfpathmoveto{\pgfqpoint{1.974174in}{4.228409in}}%
\pgfpathlineto{\pgfqpoint{1.978659in}{4.226682in}}%
\pgfpathlineto{\pgfqpoint{1.989123in}{4.217839in}}%
\pgfpathlineto{\pgfqpoint{2.223837in}{3.999334in}}%
\pgfpathlineto{\pgfqpoint{2.353901in}{3.884169in}}%
\pgfpathlineto{\pgfqpoint{2.485460in}{3.771823in}}%
\pgfpathlineto{\pgfqpoint{2.618514in}{3.662286in}}%
\pgfpathlineto{\pgfqpoint{2.753063in}{3.555545in}}%
\pgfpathlineto{\pgfqpoint{2.889107in}{3.451586in}}%
\pgfpathlineto{\pgfqpoint{3.026646in}{3.350388in}}%
\pgfpathlineto{\pgfqpoint{3.165680in}{3.251933in}}%
\pgfpathlineto{\pgfqpoint{3.306208in}{3.156195in}}%
\pgfpathlineto{\pgfqpoint{3.448232in}{3.063149in}}%
\pgfpathlineto{\pgfqpoint{3.593246in}{2.971845in}}%
\pgfpathlineto{\pgfqpoint{3.739755in}{2.883246in}}%
\pgfpathlineto{\pgfqpoint{3.889254in}{2.796468in}}%
\pgfpathlineto{\pgfqpoint{4.040248in}{2.712393in}}%
\pgfpathlineto{\pgfqpoint{4.194232in}{2.630196in}}%
\pgfpathlineto{\pgfqpoint{4.351206in}{2.549932in}}%
\pgfpathlineto{\pgfqpoint{4.509674in}{2.472365in}}%
\pgfpathlineto{\pgfqpoint{4.671133in}{2.396757in}}%
\pgfpathlineto{\pgfqpoint{4.835582in}{2.323145in}}%
\pgfpathlineto{\pgfqpoint{5.003021in}{2.251560in}}%
\pgfpathlineto{\pgfqpoint{5.174944in}{2.181429in}}%
\pgfpathlineto{\pgfqpoint{5.349858in}{2.113422in}}%
\pgfpathlineto{\pgfqpoint{5.529257in}{2.047008in}}%
\pgfpathlineto{\pgfqpoint{5.711645in}{1.982786in}}%
\pgfpathlineto{\pgfqpoint{5.898519in}{1.920262in}}%
\pgfpathlineto{\pgfqpoint{6.089877in}{1.859502in}}%
\pgfpathlineto{\pgfqpoint{6.285721in}{1.800562in}}%
\pgfpathlineto{\pgfqpoint{6.457645in}{1.751391in}}%
\pgfpathlineto{\pgfqpoint{6.457645in}{1.751391in}}%
\pgfusepath{stroke}%
\end{pgfscope}%
\begin{pgfscope}%
\pgfsetrectcap%
\pgfsetmiterjoin%
\pgfsetlinewidth{0.803000pt}%
\definecolor{currentstroke}{rgb}{0.000000,0.000000,0.000000}%
\pgfsetstrokecolor{currentstroke}%
\pgfsetdash{}{0pt}%
\pgfpathmoveto{\pgfqpoint{1.750000in}{0.625000in}}%
\pgfpathlineto{\pgfqpoint{1.750000in}{4.400000in}}%
\pgfusepath{stroke}%
\end{pgfscope}%
\begin{pgfscope}%
\pgfsetrectcap%
\pgfsetmiterjoin%
\pgfsetlinewidth{0.803000pt}%
\definecolor{currentstroke}{rgb}{0.000000,0.000000,0.000000}%
\pgfsetstrokecolor{currentstroke}%
\pgfsetdash{}{0pt}%
\pgfpathmoveto{\pgfqpoint{6.681818in}{0.625000in}}%
\pgfpathlineto{\pgfqpoint{6.681818in}{4.400000in}}%
\pgfusepath{stroke}%
\end{pgfscope}%
\begin{pgfscope}%
\pgfsetrectcap%
\pgfsetmiterjoin%
\pgfsetlinewidth{0.803000pt}%
\definecolor{currentstroke}{rgb}{0.000000,0.000000,0.000000}%
\pgfsetstrokecolor{currentstroke}%
\pgfsetdash{}{0pt}%
\pgfpathmoveto{\pgfqpoint{1.750000in}{0.625000in}}%
\pgfpathlineto{\pgfqpoint{6.681818in}{0.625000in}}%
\pgfusepath{stroke}%
\end{pgfscope}%
\begin{pgfscope}%
\pgfsetrectcap%
\pgfsetmiterjoin%
\pgfsetlinewidth{0.803000pt}%
\definecolor{currentstroke}{rgb}{0.000000,0.000000,0.000000}%
\pgfsetstrokecolor{currentstroke}%
\pgfsetdash{}{0pt}%
\pgfpathmoveto{\pgfqpoint{1.750000in}{4.400000in}}%
\pgfpathlineto{\pgfqpoint{6.681818in}{4.400000in}}%
\pgfusepath{stroke}%
\end{pgfscope}%
\begin{pgfscope}%
\definecolor{textcolor}{rgb}{0.000000,0.000000,0.000000}%
\pgfsetstrokecolor{textcolor}%
\pgfsetfillcolor{textcolor}%
\pgftext[x=4.215909in,y=4.483333in,,base]{\color{textcolor}\sffamily\fontsize{20.000000}{24.000000}\selectfont b)}%
\end{pgfscope}%
\begin{pgfscope}%
\pgfsetbuttcap%
\pgfsetmiterjoin%
\definecolor{currentfill}{rgb}{1.000000,1.000000,1.000000}%
\pgfsetfillcolor{currentfill}%
\pgfsetfillopacity{0.800000}%
\pgfsetlinewidth{1.003750pt}%
\definecolor{currentstroke}{rgb}{0.800000,0.800000,0.800000}%
\pgfsetstrokecolor{currentstroke}%
\pgfsetstrokeopacity{0.800000}%
\pgfsetdash{}{0pt}%
\pgfpathmoveto{\pgfqpoint{3.847305in}{2.546919in}}%
\pgfpathlineto{\pgfqpoint{6.487374in}{2.546919in}}%
\pgfpathquadraticcurveto{\pgfqpoint{6.542929in}{2.546919in}}{\pgfqpoint{6.542929in}{2.602475in}}%
\pgfpathlineto{\pgfqpoint{6.542929in}{4.205556in}}%
\pgfpathquadraticcurveto{\pgfqpoint{6.542929in}{4.261111in}}{\pgfqpoint{6.487374in}{4.261111in}}%
\pgfpathlineto{\pgfqpoint{3.847305in}{4.261111in}}%
\pgfpathquadraticcurveto{\pgfqpoint{3.791749in}{4.261111in}}{\pgfqpoint{3.791749in}{4.205556in}}%
\pgfpathlineto{\pgfqpoint{3.791749in}{2.602475in}}%
\pgfpathquadraticcurveto{\pgfqpoint{3.791749in}{2.546919in}}{\pgfqpoint{3.847305in}{2.546919in}}%
\pgfpathlineto{\pgfqpoint{3.847305in}{2.546919in}}%
\pgfpathclose%
\pgfusepath{stroke,fill}%
\end{pgfscope}%
\begin{pgfscope}%
\pgfsetrectcap%
\pgfsetroundjoin%
\pgfsetlinewidth{1.505625pt}%
\definecolor{currentstroke}{rgb}{0.121569,0.466667,0.705882}%
\pgfsetstrokecolor{currentstroke}%
\pgfsetdash{}{0pt}%
\pgfpathmoveto{\pgfqpoint{3.902861in}{4.036176in}}%
\pgfpathlineto{\pgfqpoint{4.180638in}{4.036176in}}%
\pgfpathlineto{\pgfqpoint{4.458416in}{4.036176in}}%
\pgfusepath{stroke}%
\end{pgfscope}%
\begin{pgfscope}%
\definecolor{textcolor}{rgb}{0.000000,0.000000,0.000000}%
\pgfsetstrokecolor{textcolor}%
\pgfsetfillcolor{textcolor}%
\pgftext[x=4.680638in,y=3.938954in,left,base]{\color{textcolor}\sffamily\fontsize{20.000000}{24.000000}\selectfont theo \(\displaystyle \omega\) = 0.1}%
\end{pgfscope}%
\begin{pgfscope}%
\pgfsetrectcap%
\pgfsetroundjoin%
\pgfsetlinewidth{1.505625pt}%
\definecolor{currentstroke}{rgb}{1.000000,0.498039,0.054902}%
\pgfsetstrokecolor{currentstroke}%
\pgfsetdash{}{0pt}%
\pgfpathmoveto{\pgfqpoint{3.902861in}{3.628462in}}%
\pgfpathlineto{\pgfqpoint{4.180638in}{3.628462in}}%
\pgfpathlineto{\pgfqpoint{4.458416in}{3.628462in}}%
\pgfusepath{stroke}%
\end{pgfscope}%
\begin{pgfscope}%
\definecolor{textcolor}{rgb}{0.000000,0.000000,0.000000}%
\pgfsetstrokecolor{textcolor}%
\pgfsetfillcolor{textcolor}%
\pgftext[x=4.680638in,y=3.531239in,left,base]{\color{textcolor}\sffamily\fontsize{20.000000}{24.000000}\selectfont num \(\displaystyle \omega\) = 0.1}%
\end{pgfscope}%
\begin{pgfscope}%
\pgfsetrectcap%
\pgfsetroundjoin%
\pgfsetlinewidth{1.505625pt}%
\definecolor{currentstroke}{rgb}{0.172549,0.627451,0.172549}%
\pgfsetstrokecolor{currentstroke}%
\pgfsetdash{}{0pt}%
\pgfpathmoveto{\pgfqpoint{3.902861in}{3.220747in}}%
\pgfpathlineto{\pgfqpoint{4.180638in}{3.220747in}}%
\pgfpathlineto{\pgfqpoint{4.458416in}{3.220747in}}%
\pgfusepath{stroke}%
\end{pgfscope}%
\begin{pgfscope}%
\definecolor{textcolor}{rgb}{0.000000,0.000000,0.000000}%
\pgfsetstrokecolor{textcolor}%
\pgfsetfillcolor{textcolor}%
\pgftext[x=4.680638in,y=3.123525in,left,base]{\color{textcolor}\sffamily\fontsize{20.000000}{24.000000}\selectfont theo \(\displaystyle \omega\) = 0.3}%
\end{pgfscope}%
\begin{pgfscope}%
\pgfsetrectcap%
\pgfsetroundjoin%
\pgfsetlinewidth{1.505625pt}%
\definecolor{currentstroke}{rgb}{0.839216,0.152941,0.156863}%
\pgfsetstrokecolor{currentstroke}%
\pgfsetdash{}{0pt}%
\pgfpathmoveto{\pgfqpoint{3.902861in}{2.813032in}}%
\pgfpathlineto{\pgfqpoint{4.180638in}{2.813032in}}%
\pgfpathlineto{\pgfqpoint{4.458416in}{2.813032in}}%
\pgfusepath{stroke}%
\end{pgfscope}%
\begin{pgfscope}%
\definecolor{textcolor}{rgb}{0.000000,0.000000,0.000000}%
\pgfsetstrokecolor{textcolor}%
\pgfsetfillcolor{textcolor}%
\pgftext[x=4.680638in,y=2.715810in,left,base]{\color{textcolor}\sffamily\fontsize{20.000000}{24.000000}\selectfont num \(\displaystyle \omega\) = 0.3}%
\end{pgfscope}%
\begin{pgfscope}%
\pgfsetbuttcap%
\pgfsetmiterjoin%
\definecolor{currentfill}{rgb}{1.000000,1.000000,1.000000}%
\pgfsetfillcolor{currentfill}%
\pgfsetlinewidth{0.000000pt}%
\definecolor{currentstroke}{rgb}{0.000000,0.000000,0.000000}%
\pgfsetstrokecolor{currentstroke}%
\pgfsetstrokeopacity{0.000000}%
\pgfsetdash{}{0pt}%
\pgfpathmoveto{\pgfqpoint{7.668182in}{0.625000in}}%
\pgfpathlineto{\pgfqpoint{12.600000in}{0.625000in}}%
\pgfpathlineto{\pgfqpoint{12.600000in}{4.400000in}}%
\pgfpathlineto{\pgfqpoint{7.668182in}{4.400000in}}%
\pgfpathlineto{\pgfqpoint{7.668182in}{0.625000in}}%
\pgfpathclose%
\pgfusepath{fill}%
\end{pgfscope}%
\begin{pgfscope}%
\pgfpathrectangle{\pgfqpoint{7.668182in}{0.625000in}}{\pgfqpoint{4.931818in}{3.775000in}}%
\pgfusepath{clip}%
\pgfsetrectcap%
\pgfsetroundjoin%
\pgfsetlinewidth{0.803000pt}%
\definecolor{currentstroke}{rgb}{0.690196,0.690196,0.690196}%
\pgfsetstrokecolor{currentstroke}%
\pgfsetdash{}{0pt}%
\pgfpathmoveto{\pgfqpoint{8.888682in}{0.625000in}}%
\pgfpathlineto{\pgfqpoint{8.888682in}{4.400000in}}%
\pgfusepath{stroke}%
\end{pgfscope}%
\begin{pgfscope}%
\pgfsetbuttcap%
\pgfsetroundjoin%
\definecolor{currentfill}{rgb}{0.000000,0.000000,0.000000}%
\pgfsetfillcolor{currentfill}%
\pgfsetlinewidth{0.803000pt}%
\definecolor{currentstroke}{rgb}{0.000000,0.000000,0.000000}%
\pgfsetstrokecolor{currentstroke}%
\pgfsetdash{}{0pt}%
\pgfsys@defobject{currentmarker}{\pgfqpoint{0.000000in}{-0.048611in}}{\pgfqpoint{0.000000in}{0.000000in}}{%
\pgfpathmoveto{\pgfqpoint{0.000000in}{0.000000in}}%
\pgfpathlineto{\pgfqpoint{0.000000in}{-0.048611in}}%
\pgfusepath{stroke,fill}%
}%
\begin{pgfscope}%
\pgfsys@transformshift{8.888682in}{0.625000in}%
\pgfsys@useobject{currentmarker}{}%
\end{pgfscope}%
\end{pgfscope}%
\begin{pgfscope}%
\definecolor{textcolor}{rgb}{0.000000,0.000000,0.000000}%
\pgfsetstrokecolor{textcolor}%
\pgfsetfillcolor{textcolor}%
\pgftext[x=8.888682in,y=0.527778in,,top]{\color{textcolor}\sffamily\fontsize{16.000000}{19.200000}\selectfont 0.5}%
\end{pgfscope}%
\begin{pgfscope}%
\pgfpathrectangle{\pgfqpoint{7.668182in}{0.625000in}}{\pgfqpoint{4.931818in}{3.775000in}}%
\pgfusepath{clip}%
\pgfsetrectcap%
\pgfsetroundjoin%
\pgfsetlinewidth{0.803000pt}%
\definecolor{currentstroke}{rgb}{0.690196,0.690196,0.690196}%
\pgfsetstrokecolor{currentstroke}%
\pgfsetdash{}{0pt}%
\pgfpathmoveto{\pgfqpoint{10.134091in}{0.625000in}}%
\pgfpathlineto{\pgfqpoint{10.134091in}{4.400000in}}%
\pgfusepath{stroke}%
\end{pgfscope}%
\begin{pgfscope}%
\pgfsetbuttcap%
\pgfsetroundjoin%
\definecolor{currentfill}{rgb}{0.000000,0.000000,0.000000}%
\pgfsetfillcolor{currentfill}%
\pgfsetlinewidth{0.803000pt}%
\definecolor{currentstroke}{rgb}{0.000000,0.000000,0.000000}%
\pgfsetstrokecolor{currentstroke}%
\pgfsetdash{}{0pt}%
\pgfsys@defobject{currentmarker}{\pgfqpoint{0.000000in}{-0.048611in}}{\pgfqpoint{0.000000in}{0.000000in}}{%
\pgfpathmoveto{\pgfqpoint{0.000000in}{0.000000in}}%
\pgfpathlineto{\pgfqpoint{0.000000in}{-0.048611in}}%
\pgfusepath{stroke,fill}%
}%
\begin{pgfscope}%
\pgfsys@transformshift{10.134091in}{0.625000in}%
\pgfsys@useobject{currentmarker}{}%
\end{pgfscope}%
\end{pgfscope}%
\begin{pgfscope}%
\definecolor{textcolor}{rgb}{0.000000,0.000000,0.000000}%
\pgfsetstrokecolor{textcolor}%
\pgfsetfillcolor{textcolor}%
\pgftext[x=10.134091in,y=0.527778in,,top]{\color{textcolor}\sffamily\fontsize{16.000000}{19.200000}\selectfont 1.0}%
\end{pgfscope}%
\begin{pgfscope}%
\pgfpathrectangle{\pgfqpoint{7.668182in}{0.625000in}}{\pgfqpoint{4.931818in}{3.775000in}}%
\pgfusepath{clip}%
\pgfsetrectcap%
\pgfsetroundjoin%
\pgfsetlinewidth{0.803000pt}%
\definecolor{currentstroke}{rgb}{0.690196,0.690196,0.690196}%
\pgfsetstrokecolor{currentstroke}%
\pgfsetdash{}{0pt}%
\pgfpathmoveto{\pgfqpoint{11.379500in}{0.625000in}}%
\pgfpathlineto{\pgfqpoint{11.379500in}{4.400000in}}%
\pgfusepath{stroke}%
\end{pgfscope}%
\begin{pgfscope}%
\pgfsetbuttcap%
\pgfsetroundjoin%
\definecolor{currentfill}{rgb}{0.000000,0.000000,0.000000}%
\pgfsetfillcolor{currentfill}%
\pgfsetlinewidth{0.803000pt}%
\definecolor{currentstroke}{rgb}{0.000000,0.000000,0.000000}%
\pgfsetstrokecolor{currentstroke}%
\pgfsetdash{}{0pt}%
\pgfsys@defobject{currentmarker}{\pgfqpoint{0.000000in}{-0.048611in}}{\pgfqpoint{0.000000in}{0.000000in}}{%
\pgfpathmoveto{\pgfqpoint{0.000000in}{0.000000in}}%
\pgfpathlineto{\pgfqpoint{0.000000in}{-0.048611in}}%
\pgfusepath{stroke,fill}%
}%
\begin{pgfscope}%
\pgfsys@transformshift{11.379500in}{0.625000in}%
\pgfsys@useobject{currentmarker}{}%
\end{pgfscope}%
\end{pgfscope}%
\begin{pgfscope}%
\definecolor{textcolor}{rgb}{0.000000,0.000000,0.000000}%
\pgfsetstrokecolor{textcolor}%
\pgfsetfillcolor{textcolor}%
\pgftext[x=11.379500in,y=0.527778in,,top]{\color{textcolor}\sffamily\fontsize{16.000000}{19.200000}\selectfont 1.5}%
\end{pgfscope}%
\begin{pgfscope}%
\definecolor{textcolor}{rgb}{0.000000,0.000000,0.000000}%
\pgfsetstrokecolor{textcolor}%
\pgfsetfillcolor{textcolor}%
\pgftext[x=10.134091in,y=0.257162in,,top]{\color{textcolor}\sffamily\fontsize{20.000000}{24.000000}\selectfont \(\displaystyle \omega\)}%
\end{pgfscope}%
\begin{pgfscope}%
\pgfpathrectangle{\pgfqpoint{7.668182in}{0.625000in}}{\pgfqpoint{4.931818in}{3.775000in}}%
\pgfusepath{clip}%
\pgfsetrectcap%
\pgfsetroundjoin%
\pgfsetlinewidth{0.803000pt}%
\definecolor{currentstroke}{rgb}{0.690196,0.690196,0.690196}%
\pgfsetstrokecolor{currentstroke}%
\pgfsetdash{}{0pt}%
\pgfpathmoveto{\pgfqpoint{7.668182in}{0.787058in}}%
\pgfpathlineto{\pgfqpoint{12.600000in}{0.787058in}}%
\pgfusepath{stroke}%
\end{pgfscope}%
\begin{pgfscope}%
\pgfsetbuttcap%
\pgfsetroundjoin%
\definecolor{currentfill}{rgb}{0.000000,0.000000,0.000000}%
\pgfsetfillcolor{currentfill}%
\pgfsetlinewidth{0.803000pt}%
\definecolor{currentstroke}{rgb}{0.000000,0.000000,0.000000}%
\pgfsetstrokecolor{currentstroke}%
\pgfsetdash{}{0pt}%
\pgfsys@defobject{currentmarker}{\pgfqpoint{-0.048611in}{0.000000in}}{\pgfqpoint{-0.000000in}{0.000000in}}{%
\pgfpathmoveto{\pgfqpoint{-0.000000in}{0.000000in}}%
\pgfpathlineto{\pgfqpoint{-0.048611in}{0.000000in}}%
\pgfusepath{stroke,fill}%
}%
\begin{pgfscope}%
\pgfsys@transformshift{7.668182in}{0.787058in}%
\pgfsys@useobject{currentmarker}{}%
\end{pgfscope}%
\end{pgfscope}%
\begin{pgfscope}%
\definecolor{textcolor}{rgb}{0.000000,0.000000,0.000000}%
\pgfsetstrokecolor{textcolor}%
\pgfsetfillcolor{textcolor}%
\pgftext[x=7.217553in, y=0.702640in, left, base]{\color{textcolor}\sffamily\fontsize{16.000000}{19.200000}\selectfont 0.0}%
\end{pgfscope}%
\begin{pgfscope}%
\pgfpathrectangle{\pgfqpoint{7.668182in}{0.625000in}}{\pgfqpoint{4.931818in}{3.775000in}}%
\pgfusepath{clip}%
\pgfsetrectcap%
\pgfsetroundjoin%
\pgfsetlinewidth{0.803000pt}%
\definecolor{currentstroke}{rgb}{0.690196,0.690196,0.690196}%
\pgfsetstrokecolor{currentstroke}%
\pgfsetdash{}{0pt}%
\pgfpathmoveto{\pgfqpoint{7.668182in}{1.330429in}}%
\pgfpathlineto{\pgfqpoint{12.600000in}{1.330429in}}%
\pgfusepath{stroke}%
\end{pgfscope}%
\begin{pgfscope}%
\pgfsetbuttcap%
\pgfsetroundjoin%
\definecolor{currentfill}{rgb}{0.000000,0.000000,0.000000}%
\pgfsetfillcolor{currentfill}%
\pgfsetlinewidth{0.803000pt}%
\definecolor{currentstroke}{rgb}{0.000000,0.000000,0.000000}%
\pgfsetstrokecolor{currentstroke}%
\pgfsetdash{}{0pt}%
\pgfsys@defobject{currentmarker}{\pgfqpoint{-0.048611in}{0.000000in}}{\pgfqpoint{-0.000000in}{0.000000in}}{%
\pgfpathmoveto{\pgfqpoint{-0.000000in}{0.000000in}}%
\pgfpathlineto{\pgfqpoint{-0.048611in}{0.000000in}}%
\pgfusepath{stroke,fill}%
}%
\begin{pgfscope}%
\pgfsys@transformshift{7.668182in}{1.330429in}%
\pgfsys@useobject{currentmarker}{}%
\end{pgfscope}%
\end{pgfscope}%
\begin{pgfscope}%
\definecolor{textcolor}{rgb}{0.000000,0.000000,0.000000}%
\pgfsetstrokecolor{textcolor}%
\pgfsetfillcolor{textcolor}%
\pgftext[x=7.217553in, y=1.246011in, left, base]{\color{textcolor}\sffamily\fontsize{16.000000}{19.200000}\selectfont 0.5}%
\end{pgfscope}%
\begin{pgfscope}%
\pgfpathrectangle{\pgfqpoint{7.668182in}{0.625000in}}{\pgfqpoint{4.931818in}{3.775000in}}%
\pgfusepath{clip}%
\pgfsetrectcap%
\pgfsetroundjoin%
\pgfsetlinewidth{0.803000pt}%
\definecolor{currentstroke}{rgb}{0.690196,0.690196,0.690196}%
\pgfsetstrokecolor{currentstroke}%
\pgfsetdash{}{0pt}%
\pgfpathmoveto{\pgfqpoint{7.668182in}{1.873801in}}%
\pgfpathlineto{\pgfqpoint{12.600000in}{1.873801in}}%
\pgfusepath{stroke}%
\end{pgfscope}%
\begin{pgfscope}%
\pgfsetbuttcap%
\pgfsetroundjoin%
\definecolor{currentfill}{rgb}{0.000000,0.000000,0.000000}%
\pgfsetfillcolor{currentfill}%
\pgfsetlinewidth{0.803000pt}%
\definecolor{currentstroke}{rgb}{0.000000,0.000000,0.000000}%
\pgfsetstrokecolor{currentstroke}%
\pgfsetdash{}{0pt}%
\pgfsys@defobject{currentmarker}{\pgfqpoint{-0.048611in}{0.000000in}}{\pgfqpoint{-0.000000in}{0.000000in}}{%
\pgfpathmoveto{\pgfqpoint{-0.000000in}{0.000000in}}%
\pgfpathlineto{\pgfqpoint{-0.048611in}{0.000000in}}%
\pgfusepath{stroke,fill}%
}%
\begin{pgfscope}%
\pgfsys@transformshift{7.668182in}{1.873801in}%
\pgfsys@useobject{currentmarker}{}%
\end{pgfscope}%
\end{pgfscope}%
\begin{pgfscope}%
\definecolor{textcolor}{rgb}{0.000000,0.000000,0.000000}%
\pgfsetstrokecolor{textcolor}%
\pgfsetfillcolor{textcolor}%
\pgftext[x=7.217553in, y=1.789382in, left, base]{\color{textcolor}\sffamily\fontsize{16.000000}{19.200000}\selectfont 1.0}%
\end{pgfscope}%
\begin{pgfscope}%
\pgfpathrectangle{\pgfqpoint{7.668182in}{0.625000in}}{\pgfqpoint{4.931818in}{3.775000in}}%
\pgfusepath{clip}%
\pgfsetrectcap%
\pgfsetroundjoin%
\pgfsetlinewidth{0.803000pt}%
\definecolor{currentstroke}{rgb}{0.690196,0.690196,0.690196}%
\pgfsetstrokecolor{currentstroke}%
\pgfsetdash{}{0pt}%
\pgfpathmoveto{\pgfqpoint{7.668182in}{2.417172in}}%
\pgfpathlineto{\pgfqpoint{12.600000in}{2.417172in}}%
\pgfusepath{stroke}%
\end{pgfscope}%
\begin{pgfscope}%
\pgfsetbuttcap%
\pgfsetroundjoin%
\definecolor{currentfill}{rgb}{0.000000,0.000000,0.000000}%
\pgfsetfillcolor{currentfill}%
\pgfsetlinewidth{0.803000pt}%
\definecolor{currentstroke}{rgb}{0.000000,0.000000,0.000000}%
\pgfsetstrokecolor{currentstroke}%
\pgfsetdash{}{0pt}%
\pgfsys@defobject{currentmarker}{\pgfqpoint{-0.048611in}{0.000000in}}{\pgfqpoint{-0.000000in}{0.000000in}}{%
\pgfpathmoveto{\pgfqpoint{-0.000000in}{0.000000in}}%
\pgfpathlineto{\pgfqpoint{-0.048611in}{0.000000in}}%
\pgfusepath{stroke,fill}%
}%
\begin{pgfscope}%
\pgfsys@transformshift{7.668182in}{2.417172in}%
\pgfsys@useobject{currentmarker}{}%
\end{pgfscope}%
\end{pgfscope}%
\begin{pgfscope}%
\definecolor{textcolor}{rgb}{0.000000,0.000000,0.000000}%
\pgfsetstrokecolor{textcolor}%
\pgfsetfillcolor{textcolor}%
\pgftext[x=7.217553in, y=2.332753in, left, base]{\color{textcolor}\sffamily\fontsize{16.000000}{19.200000}\selectfont 1.5}%
\end{pgfscope}%
\begin{pgfscope}%
\pgfpathrectangle{\pgfqpoint{7.668182in}{0.625000in}}{\pgfqpoint{4.931818in}{3.775000in}}%
\pgfusepath{clip}%
\pgfsetrectcap%
\pgfsetroundjoin%
\pgfsetlinewidth{0.803000pt}%
\definecolor{currentstroke}{rgb}{0.690196,0.690196,0.690196}%
\pgfsetstrokecolor{currentstroke}%
\pgfsetdash{}{0pt}%
\pgfpathmoveto{\pgfqpoint{7.668182in}{2.960543in}}%
\pgfpathlineto{\pgfqpoint{12.600000in}{2.960543in}}%
\pgfusepath{stroke}%
\end{pgfscope}%
\begin{pgfscope}%
\pgfsetbuttcap%
\pgfsetroundjoin%
\definecolor{currentfill}{rgb}{0.000000,0.000000,0.000000}%
\pgfsetfillcolor{currentfill}%
\pgfsetlinewidth{0.803000pt}%
\definecolor{currentstroke}{rgb}{0.000000,0.000000,0.000000}%
\pgfsetstrokecolor{currentstroke}%
\pgfsetdash{}{0pt}%
\pgfsys@defobject{currentmarker}{\pgfqpoint{-0.048611in}{0.000000in}}{\pgfqpoint{-0.000000in}{0.000000in}}{%
\pgfpathmoveto{\pgfqpoint{-0.000000in}{0.000000in}}%
\pgfpathlineto{\pgfqpoint{-0.048611in}{0.000000in}}%
\pgfusepath{stroke,fill}%
}%
\begin{pgfscope}%
\pgfsys@transformshift{7.668182in}{2.960543in}%
\pgfsys@useobject{currentmarker}{}%
\end{pgfscope}%
\end{pgfscope}%
\begin{pgfscope}%
\definecolor{textcolor}{rgb}{0.000000,0.000000,0.000000}%
\pgfsetstrokecolor{textcolor}%
\pgfsetfillcolor{textcolor}%
\pgftext[x=7.217553in, y=2.876125in, left, base]{\color{textcolor}\sffamily\fontsize{16.000000}{19.200000}\selectfont 2.0}%
\end{pgfscope}%
\begin{pgfscope}%
\pgfpathrectangle{\pgfqpoint{7.668182in}{0.625000in}}{\pgfqpoint{4.931818in}{3.775000in}}%
\pgfusepath{clip}%
\pgfsetrectcap%
\pgfsetroundjoin%
\pgfsetlinewidth{0.803000pt}%
\definecolor{currentstroke}{rgb}{0.690196,0.690196,0.690196}%
\pgfsetstrokecolor{currentstroke}%
\pgfsetdash{}{0pt}%
\pgfpathmoveto{\pgfqpoint{7.668182in}{3.503914in}}%
\pgfpathlineto{\pgfqpoint{12.600000in}{3.503914in}}%
\pgfusepath{stroke}%
\end{pgfscope}%
\begin{pgfscope}%
\pgfsetbuttcap%
\pgfsetroundjoin%
\definecolor{currentfill}{rgb}{0.000000,0.000000,0.000000}%
\pgfsetfillcolor{currentfill}%
\pgfsetlinewidth{0.803000pt}%
\definecolor{currentstroke}{rgb}{0.000000,0.000000,0.000000}%
\pgfsetstrokecolor{currentstroke}%
\pgfsetdash{}{0pt}%
\pgfsys@defobject{currentmarker}{\pgfqpoint{-0.048611in}{0.000000in}}{\pgfqpoint{-0.000000in}{0.000000in}}{%
\pgfpathmoveto{\pgfqpoint{-0.000000in}{0.000000in}}%
\pgfpathlineto{\pgfqpoint{-0.048611in}{0.000000in}}%
\pgfusepath{stroke,fill}%
}%
\begin{pgfscope}%
\pgfsys@transformshift{7.668182in}{3.503914in}%
\pgfsys@useobject{currentmarker}{}%
\end{pgfscope}%
\end{pgfscope}%
\begin{pgfscope}%
\definecolor{textcolor}{rgb}{0.000000,0.000000,0.000000}%
\pgfsetstrokecolor{textcolor}%
\pgfsetfillcolor{textcolor}%
\pgftext[x=7.217553in, y=3.419496in, left, base]{\color{textcolor}\sffamily\fontsize{16.000000}{19.200000}\selectfont 2.5}%
\end{pgfscope}%
\begin{pgfscope}%
\pgfpathrectangle{\pgfqpoint{7.668182in}{0.625000in}}{\pgfqpoint{4.931818in}{3.775000in}}%
\pgfusepath{clip}%
\pgfsetrectcap%
\pgfsetroundjoin%
\pgfsetlinewidth{0.803000pt}%
\definecolor{currentstroke}{rgb}{0.690196,0.690196,0.690196}%
\pgfsetstrokecolor{currentstroke}%
\pgfsetdash{}{0pt}%
\pgfpathmoveto{\pgfqpoint{7.668182in}{4.047285in}}%
\pgfpathlineto{\pgfqpoint{12.600000in}{4.047285in}}%
\pgfusepath{stroke}%
\end{pgfscope}%
\begin{pgfscope}%
\pgfsetbuttcap%
\pgfsetroundjoin%
\definecolor{currentfill}{rgb}{0.000000,0.000000,0.000000}%
\pgfsetfillcolor{currentfill}%
\pgfsetlinewidth{0.803000pt}%
\definecolor{currentstroke}{rgb}{0.000000,0.000000,0.000000}%
\pgfsetstrokecolor{currentstroke}%
\pgfsetdash{}{0pt}%
\pgfsys@defobject{currentmarker}{\pgfqpoint{-0.048611in}{0.000000in}}{\pgfqpoint{-0.000000in}{0.000000in}}{%
\pgfpathmoveto{\pgfqpoint{-0.000000in}{0.000000in}}%
\pgfpathlineto{\pgfqpoint{-0.048611in}{0.000000in}}%
\pgfusepath{stroke,fill}%
}%
\begin{pgfscope}%
\pgfsys@transformshift{7.668182in}{4.047285in}%
\pgfsys@useobject{currentmarker}{}%
\end{pgfscope}%
\end{pgfscope}%
\begin{pgfscope}%
\definecolor{textcolor}{rgb}{0.000000,0.000000,0.000000}%
\pgfsetstrokecolor{textcolor}%
\pgfsetfillcolor{textcolor}%
\pgftext[x=7.217553in, y=3.962867in, left, base]{\color{textcolor}\sffamily\fontsize{16.000000}{19.200000}\selectfont 3.0}%
\end{pgfscope}%
\begin{pgfscope}%
\definecolor{textcolor}{rgb}{0.000000,0.000000,0.000000}%
\pgfsetstrokecolor{textcolor}%
\pgfsetfillcolor{textcolor}%
\pgftext[x=6.793182in, y=2.512500in, left, base,rotate=90.000000]{\color{textcolor}\sffamily\fontsize{20.000000}{24.000000}\selectfont \(\displaystyle \)}%
\end{pgfscope}%
\begin{pgfscope}%
\definecolor{textcolor}{rgb}{0.000000,0.000000,0.000000}%
\pgfsetstrokecolor{textcolor}%
\pgfsetfillcolor{textcolor}%
\pgftext[x=7.104217in, y=2.424474in, left, base,rotate=90.000000]{\color{textcolor}\sffamily\fontsize{20.000000}{24.000000}\selectfont u\(\displaystyle \)}%
\end{pgfscope}%
\begin{pgfscope}%
\pgfpathrectangle{\pgfqpoint{7.668182in}{0.625000in}}{\pgfqpoint{4.931818in}{3.775000in}}%
\pgfusepath{clip}%
\pgfsetbuttcap%
\pgfsetroundjoin%
\pgfsetlinewidth{1.505625pt}%
\definecolor{currentstroke}{rgb}{1.000000,0.000000,0.000000}%
\pgfsetstrokecolor{currentstroke}%
\pgfsetdash{{5.550000pt}{2.400000pt}}{0.000000pt}%
\pgfpathmoveto{\pgfqpoint{7.892355in}{4.228409in}}%
\pgfpathlineto{\pgfqpoint{8.390519in}{1.813426in}}%
\pgfpathlineto{\pgfqpoint{8.888682in}{1.330429in}}%
\pgfpathlineto{\pgfqpoint{9.386846in}{1.123431in}}%
\pgfpathlineto{\pgfqpoint{9.885009in}{1.008432in}}%
\pgfpathlineto{\pgfqpoint{10.383173in}{0.935250in}}%
\pgfpathlineto{\pgfqpoint{10.881336in}{0.884586in}}%
\pgfpathlineto{\pgfqpoint{11.379500in}{0.847433in}}%
\pgfpathlineto{\pgfqpoint{11.877663in}{0.819021in}}%
\pgfpathlineto{\pgfqpoint{12.375826in}{0.796591in}}%
\pgfusepath{stroke}%
\end{pgfscope}%
\begin{pgfscope}%
\pgfpathrectangle{\pgfqpoint{7.668182in}{0.625000in}}{\pgfqpoint{4.931818in}{3.775000in}}%
\pgfusepath{clip}%
\pgfsetbuttcap%
\pgfsetroundjoin%
\definecolor{currentfill}{rgb}{1.000000,0.000000,0.000000}%
\pgfsetfillcolor{currentfill}%
\pgfsetlinewidth{1.003750pt}%
\definecolor{currentstroke}{rgb}{1.000000,0.000000,0.000000}%
\pgfsetstrokecolor{currentstroke}%
\pgfsetdash{}{0pt}%
\pgfsys@defobject{currentmarker}{\pgfqpoint{-0.041667in}{-0.041667in}}{\pgfqpoint{0.041667in}{0.041667in}}{%
\pgfpathmoveto{\pgfqpoint{0.000000in}{-0.041667in}}%
\pgfpathcurveto{\pgfqpoint{0.011050in}{-0.041667in}}{\pgfqpoint{0.021649in}{-0.037276in}}{\pgfqpoint{0.029463in}{-0.029463in}}%
\pgfpathcurveto{\pgfqpoint{0.037276in}{-0.021649in}}{\pgfqpoint{0.041667in}{-0.011050in}}{\pgfqpoint{0.041667in}{0.000000in}}%
\pgfpathcurveto{\pgfqpoint{0.041667in}{0.011050in}}{\pgfqpoint{0.037276in}{0.021649in}}{\pgfqpoint{0.029463in}{0.029463in}}%
\pgfpathcurveto{\pgfqpoint{0.021649in}{0.037276in}}{\pgfqpoint{0.011050in}{0.041667in}}{\pgfqpoint{0.000000in}{0.041667in}}%
\pgfpathcurveto{\pgfqpoint{-0.011050in}{0.041667in}}{\pgfqpoint{-0.021649in}{0.037276in}}{\pgfqpoint{-0.029463in}{0.029463in}}%
\pgfpathcurveto{\pgfqpoint{-0.037276in}{0.021649in}}{\pgfqpoint{-0.041667in}{0.011050in}}{\pgfqpoint{-0.041667in}{0.000000in}}%
\pgfpathcurveto{\pgfqpoint{-0.041667in}{-0.011050in}}{\pgfqpoint{-0.037276in}{-0.021649in}}{\pgfqpoint{-0.029463in}{-0.029463in}}%
\pgfpathcurveto{\pgfqpoint{-0.021649in}{-0.037276in}}{\pgfqpoint{-0.011050in}{-0.041667in}}{\pgfqpoint{0.000000in}{-0.041667in}}%
\pgfpathlineto{\pgfqpoint{0.000000in}{-0.041667in}}%
\pgfpathclose%
\pgfusepath{stroke,fill}%
}%
\begin{pgfscope}%
\pgfsys@transformshift{7.892355in}{4.228409in}%
\pgfsys@useobject{currentmarker}{}%
\end{pgfscope}%
\begin{pgfscope}%
\pgfsys@transformshift{8.390519in}{1.813426in}%
\pgfsys@useobject{currentmarker}{}%
\end{pgfscope}%
\begin{pgfscope}%
\pgfsys@transformshift{8.888682in}{1.330429in}%
\pgfsys@useobject{currentmarker}{}%
\end{pgfscope}%
\begin{pgfscope}%
\pgfsys@transformshift{9.386846in}{1.123431in}%
\pgfsys@useobject{currentmarker}{}%
\end{pgfscope}%
\begin{pgfscope}%
\pgfsys@transformshift{9.885009in}{1.008432in}%
\pgfsys@useobject{currentmarker}{}%
\end{pgfscope}%
\begin{pgfscope}%
\pgfsys@transformshift{10.383173in}{0.935250in}%
\pgfsys@useobject{currentmarker}{}%
\end{pgfscope}%
\begin{pgfscope}%
\pgfsys@transformshift{10.881336in}{0.884586in}%
\pgfsys@useobject{currentmarker}{}%
\end{pgfscope}%
\begin{pgfscope}%
\pgfsys@transformshift{11.379500in}{0.847433in}%
\pgfsys@useobject{currentmarker}{}%
\end{pgfscope}%
\begin{pgfscope}%
\pgfsys@transformshift{11.877663in}{0.819021in}%
\pgfsys@useobject{currentmarker}{}%
\end{pgfscope}%
\begin{pgfscope}%
\pgfsys@transformshift{12.375826in}{0.796591in}%
\pgfsys@useobject{currentmarker}{}%
\end{pgfscope}%
\end{pgfscope}%
\begin{pgfscope}%
\pgfpathrectangle{\pgfqpoint{7.668182in}{0.625000in}}{\pgfqpoint{4.931818in}{3.775000in}}%
\pgfusepath{clip}%
\pgfsetbuttcap%
\pgfsetroundjoin%
\pgfsetlinewidth{1.505625pt}%
\definecolor{currentstroke}{rgb}{0.000000,0.000000,1.000000}%
\pgfsetstrokecolor{currentstroke}%
\pgfsetdash{{5.550000pt}{2.400000pt}}{0.000000pt}%
\pgfpathmoveto{\pgfqpoint{7.892355in}{4.139399in}}%
\pgfpathlineto{\pgfqpoint{8.390519in}{1.810434in}}%
\pgfusepath{stroke}%
\end{pgfscope}%
\begin{pgfscope}%
\pgfpathrectangle{\pgfqpoint{7.668182in}{0.625000in}}{\pgfqpoint{4.931818in}{3.775000in}}%
\pgfusepath{clip}%
\pgfsetbuttcap%
\pgfsetroundjoin%
\definecolor{currentfill}{rgb}{0.000000,0.000000,1.000000}%
\pgfsetfillcolor{currentfill}%
\pgfsetlinewidth{1.003750pt}%
\definecolor{currentstroke}{rgb}{0.000000,0.000000,1.000000}%
\pgfsetstrokecolor{currentstroke}%
\pgfsetdash{}{0pt}%
\pgfsys@defobject{currentmarker}{\pgfqpoint{-0.041667in}{-0.041667in}}{\pgfqpoint{0.041667in}{0.041667in}}{%
\pgfpathmoveto{\pgfqpoint{0.000000in}{-0.041667in}}%
\pgfpathcurveto{\pgfqpoint{0.011050in}{-0.041667in}}{\pgfqpoint{0.021649in}{-0.037276in}}{\pgfqpoint{0.029463in}{-0.029463in}}%
\pgfpathcurveto{\pgfqpoint{0.037276in}{-0.021649in}}{\pgfqpoint{0.041667in}{-0.011050in}}{\pgfqpoint{0.041667in}{0.000000in}}%
\pgfpathcurveto{\pgfqpoint{0.041667in}{0.011050in}}{\pgfqpoint{0.037276in}{0.021649in}}{\pgfqpoint{0.029463in}{0.029463in}}%
\pgfpathcurveto{\pgfqpoint{0.021649in}{0.037276in}}{\pgfqpoint{0.011050in}{0.041667in}}{\pgfqpoint{0.000000in}{0.041667in}}%
\pgfpathcurveto{\pgfqpoint{-0.011050in}{0.041667in}}{\pgfqpoint{-0.021649in}{0.037276in}}{\pgfqpoint{-0.029463in}{0.029463in}}%
\pgfpathcurveto{\pgfqpoint{-0.037276in}{0.021649in}}{\pgfqpoint{-0.041667in}{0.011050in}}{\pgfqpoint{-0.041667in}{0.000000in}}%
\pgfpathcurveto{\pgfqpoint{-0.041667in}{-0.011050in}}{\pgfqpoint{-0.037276in}{-0.021649in}}{\pgfqpoint{-0.029463in}{-0.029463in}}%
\pgfpathcurveto{\pgfqpoint{-0.021649in}{-0.037276in}}{\pgfqpoint{-0.011050in}{-0.041667in}}{\pgfqpoint{0.000000in}{-0.041667in}}%
\pgfpathlineto{\pgfqpoint{0.000000in}{-0.041667in}}%
\pgfpathclose%
\pgfusepath{stroke,fill}%
}%
\begin{pgfscope}%
\pgfsys@transformshift{7.892355in}{4.139399in}%
\pgfsys@useobject{currentmarker}{}%
\end{pgfscope}%
\begin{pgfscope}%
\pgfsys@transformshift{8.390519in}{1.810434in}%
\pgfsys@useobject{currentmarker}{}%
\end{pgfscope}%
\end{pgfscope}%
\begin{pgfscope}%
\pgfsetrectcap%
\pgfsetmiterjoin%
\pgfsetlinewidth{0.803000pt}%
\definecolor{currentstroke}{rgb}{0.000000,0.000000,0.000000}%
\pgfsetstrokecolor{currentstroke}%
\pgfsetdash{}{0pt}%
\pgfpathmoveto{\pgfqpoint{7.668182in}{0.625000in}}%
\pgfpathlineto{\pgfqpoint{7.668182in}{4.400000in}}%
\pgfusepath{stroke}%
\end{pgfscope}%
\begin{pgfscope}%
\pgfsetrectcap%
\pgfsetmiterjoin%
\pgfsetlinewidth{0.803000pt}%
\definecolor{currentstroke}{rgb}{0.000000,0.000000,0.000000}%
\pgfsetstrokecolor{currentstroke}%
\pgfsetdash{}{0pt}%
\pgfpathmoveto{\pgfqpoint{12.600000in}{0.625000in}}%
\pgfpathlineto{\pgfqpoint{12.600000in}{4.400000in}}%
\pgfusepath{stroke}%
\end{pgfscope}%
\begin{pgfscope}%
\pgfsetrectcap%
\pgfsetmiterjoin%
\pgfsetlinewidth{0.803000pt}%
\definecolor{currentstroke}{rgb}{0.000000,0.000000,0.000000}%
\pgfsetstrokecolor{currentstroke}%
\pgfsetdash{}{0pt}%
\pgfpathmoveto{\pgfqpoint{7.668182in}{0.625000in}}%
\pgfpathlineto{\pgfqpoint{12.600000in}{0.625000in}}%
\pgfusepath{stroke}%
\end{pgfscope}%
\begin{pgfscope}%
\pgfsetrectcap%
\pgfsetmiterjoin%
\pgfsetlinewidth{0.803000pt}%
\definecolor{currentstroke}{rgb}{0.000000,0.000000,0.000000}%
\pgfsetstrokecolor{currentstroke}%
\pgfsetdash{}{0pt}%
\pgfpathmoveto{\pgfqpoint{7.668182in}{4.400000in}}%
\pgfpathlineto{\pgfqpoint{12.600000in}{4.400000in}}%
\pgfusepath{stroke}%
\end{pgfscope}%
\begin{pgfscope}%
\pgfsetbuttcap%
\pgfsetmiterjoin%
\definecolor{currentfill}{rgb}{1.000000,1.000000,1.000000}%
\pgfsetfillcolor{currentfill}%
\pgfsetfillopacity{0.800000}%
\pgfsetlinewidth{1.003750pt}%
\definecolor{currentstroke}{rgb}{0.800000,0.800000,0.800000}%
\pgfsetstrokecolor{currentstroke}%
\pgfsetstrokeopacity{0.800000}%
\pgfsetdash{}{0pt}%
\pgfpathmoveto{\pgfqpoint{9.301226in}{3.362349in}}%
\pgfpathlineto{\pgfqpoint{12.405556in}{3.362349in}}%
\pgfpathquadraticcurveto{\pgfqpoint{12.461111in}{3.362349in}}{\pgfqpoint{12.461111in}{3.417904in}}%
\pgfpathlineto{\pgfqpoint{12.461111in}{4.205556in}}%
\pgfpathquadraticcurveto{\pgfqpoint{12.461111in}{4.261111in}}{\pgfqpoint{12.405556in}{4.261111in}}%
\pgfpathlineto{\pgfqpoint{9.301226in}{4.261111in}}%
\pgfpathquadraticcurveto{\pgfqpoint{9.245671in}{4.261111in}}{\pgfqpoint{9.245671in}{4.205556in}}%
\pgfpathlineto{\pgfqpoint{9.245671in}{3.417904in}}%
\pgfpathquadraticcurveto{\pgfqpoint{9.245671in}{3.362349in}}{\pgfqpoint{9.301226in}{3.362349in}}%
\pgfpathlineto{\pgfqpoint{9.301226in}{3.362349in}}%
\pgfpathclose%
\pgfusepath{stroke,fill}%
\end{pgfscope}%
\begin{pgfscope}%
\pgfsetbuttcap%
\pgfsetroundjoin%
\pgfsetlinewidth{1.505625pt}%
\definecolor{currentstroke}{rgb}{1.000000,0.000000,0.000000}%
\pgfsetstrokecolor{currentstroke}%
\pgfsetdash{{5.550000pt}{2.400000pt}}{0.000000pt}%
\pgfpathmoveto{\pgfqpoint{9.356782in}{4.036176in}}%
\pgfpathlineto{\pgfqpoint{9.634559in}{4.036176in}}%
\pgfpathlineto{\pgfqpoint{9.912337in}{4.036176in}}%
\pgfusepath{stroke}%
\end{pgfscope}%
\begin{pgfscope}%
\pgfsetbuttcap%
\pgfsetroundjoin%
\definecolor{currentfill}{rgb}{1.000000,0.000000,0.000000}%
\pgfsetfillcolor{currentfill}%
\pgfsetlinewidth{1.003750pt}%
\definecolor{currentstroke}{rgb}{1.000000,0.000000,0.000000}%
\pgfsetstrokecolor{currentstroke}%
\pgfsetdash{}{0pt}%
\pgfsys@defobject{currentmarker}{\pgfqpoint{-0.041667in}{-0.041667in}}{\pgfqpoint{0.041667in}{0.041667in}}{%
\pgfpathmoveto{\pgfqpoint{0.000000in}{-0.041667in}}%
\pgfpathcurveto{\pgfqpoint{0.011050in}{-0.041667in}}{\pgfqpoint{0.021649in}{-0.037276in}}{\pgfqpoint{0.029463in}{-0.029463in}}%
\pgfpathcurveto{\pgfqpoint{0.037276in}{-0.021649in}}{\pgfqpoint{0.041667in}{-0.011050in}}{\pgfqpoint{0.041667in}{0.000000in}}%
\pgfpathcurveto{\pgfqpoint{0.041667in}{0.011050in}}{\pgfqpoint{0.037276in}{0.021649in}}{\pgfqpoint{0.029463in}{0.029463in}}%
\pgfpathcurveto{\pgfqpoint{0.021649in}{0.037276in}}{\pgfqpoint{0.011050in}{0.041667in}}{\pgfqpoint{0.000000in}{0.041667in}}%
\pgfpathcurveto{\pgfqpoint{-0.011050in}{0.041667in}}{\pgfqpoint{-0.021649in}{0.037276in}}{\pgfqpoint{-0.029463in}{0.029463in}}%
\pgfpathcurveto{\pgfqpoint{-0.037276in}{0.021649in}}{\pgfqpoint{-0.041667in}{0.011050in}}{\pgfqpoint{-0.041667in}{0.000000in}}%
\pgfpathcurveto{\pgfqpoint{-0.041667in}{-0.011050in}}{\pgfqpoint{-0.037276in}{-0.021649in}}{\pgfqpoint{-0.029463in}{-0.029463in}}%
\pgfpathcurveto{\pgfqpoint{-0.021649in}{-0.037276in}}{\pgfqpoint{-0.011050in}{-0.041667in}}{\pgfqpoint{0.000000in}{-0.041667in}}%
\pgfpathlineto{\pgfqpoint{0.000000in}{-0.041667in}}%
\pgfpathclose%
\pgfusepath{stroke,fill}%
}%
\begin{pgfscope}%
\pgfsys@transformshift{9.634559in}{4.036176in}%
\pgfsys@useobject{currentmarker}{}%
\end{pgfscope}%
\end{pgfscope}%
\begin{pgfscope}%
\definecolor{textcolor}{rgb}{0.000000,0.000000,0.000000}%
\pgfsetstrokecolor{textcolor}%
\pgfsetfillcolor{textcolor}%
\pgftext[x=10.134559in,y=3.938954in,left,base]{\color{textcolor}\sffamily\fontsize{20.000000}{24.000000}\selectfont Theo viscocities}%
\end{pgfscope}%
\begin{pgfscope}%
\pgfsetbuttcap%
\pgfsetroundjoin%
\pgfsetlinewidth{1.505625pt}%
\definecolor{currentstroke}{rgb}{0.000000,0.000000,1.000000}%
\pgfsetstrokecolor{currentstroke}%
\pgfsetdash{{5.550000pt}{2.400000pt}}{0.000000pt}%
\pgfpathmoveto{\pgfqpoint{9.356782in}{3.628462in}}%
\pgfpathlineto{\pgfqpoint{9.634559in}{3.628462in}}%
\pgfpathlineto{\pgfqpoint{9.912337in}{3.628462in}}%
\pgfusepath{stroke}%
\end{pgfscope}%
\begin{pgfscope}%
\pgfsetbuttcap%
\pgfsetroundjoin%
\definecolor{currentfill}{rgb}{0.000000,0.000000,1.000000}%
\pgfsetfillcolor{currentfill}%
\pgfsetlinewidth{1.003750pt}%
\definecolor{currentstroke}{rgb}{0.000000,0.000000,1.000000}%
\pgfsetstrokecolor{currentstroke}%
\pgfsetdash{}{0pt}%
\pgfsys@defobject{currentmarker}{\pgfqpoint{-0.041667in}{-0.041667in}}{\pgfqpoint{0.041667in}{0.041667in}}{%
\pgfpathmoveto{\pgfqpoint{0.000000in}{-0.041667in}}%
\pgfpathcurveto{\pgfqpoint{0.011050in}{-0.041667in}}{\pgfqpoint{0.021649in}{-0.037276in}}{\pgfqpoint{0.029463in}{-0.029463in}}%
\pgfpathcurveto{\pgfqpoint{0.037276in}{-0.021649in}}{\pgfqpoint{0.041667in}{-0.011050in}}{\pgfqpoint{0.041667in}{0.000000in}}%
\pgfpathcurveto{\pgfqpoint{0.041667in}{0.011050in}}{\pgfqpoint{0.037276in}{0.021649in}}{\pgfqpoint{0.029463in}{0.029463in}}%
\pgfpathcurveto{\pgfqpoint{0.021649in}{0.037276in}}{\pgfqpoint{0.011050in}{0.041667in}}{\pgfqpoint{0.000000in}{0.041667in}}%
\pgfpathcurveto{\pgfqpoint{-0.011050in}{0.041667in}}{\pgfqpoint{-0.021649in}{0.037276in}}{\pgfqpoint{-0.029463in}{0.029463in}}%
\pgfpathcurveto{\pgfqpoint{-0.037276in}{0.021649in}}{\pgfqpoint{-0.041667in}{0.011050in}}{\pgfqpoint{-0.041667in}{0.000000in}}%
\pgfpathcurveto{\pgfqpoint{-0.041667in}{-0.011050in}}{\pgfqpoint{-0.037276in}{-0.021649in}}{\pgfqpoint{-0.029463in}{-0.029463in}}%
\pgfpathcurveto{\pgfqpoint{-0.021649in}{-0.037276in}}{\pgfqpoint{-0.011050in}{-0.041667in}}{\pgfqpoint{0.000000in}{-0.041667in}}%
\pgfpathlineto{\pgfqpoint{0.000000in}{-0.041667in}}%
\pgfpathclose%
\pgfusepath{stroke,fill}%
}%
\begin{pgfscope}%
\pgfsys@transformshift{9.634559in}{3.628462in}%
\pgfsys@useobject{currentmarker}{}%
\end{pgfscope}%
\end{pgfscope}%
\begin{pgfscope}%
\definecolor{textcolor}{rgb}{0.000000,0.000000,0.000000}%
\pgfsetstrokecolor{textcolor}%
\pgfsetfillcolor{textcolor}%
\pgftext[x=10.134559in,y=3.531239in,left,base]{\color{textcolor}\sffamily\fontsize{20.000000}{24.000000}\selectfont Num viscocities}%
\end{pgfscope}%
\end{pgfpicture}%
\makeatother%
\endgroup%
}
% \vspace*{-10mm}
\caption[Kinematic viscosity]{The numerical and theoretical velocity amplitude decay and respective kinematic viscosity for $\omega=[0.1, 0.3] $.}
\label{fig:m3-3-update}
\end{figure}

Initially, I wanted to only include the correct numerical results, but decided to include the numerical incorrect results and then show that in a bigger lattice the problem does not occur.
I found it interesting that the correctness of the results would be influenced by the collision frequency \textbf{and} the dimensions of the physical space. 
As $\omega$ only influences how much the equilibrium function is integrated in the collision term, I suspect that the increased rolling over the boundaries somehow causes the equilibrium function to be more imprecise. 
This is just one possible reason, but no definitive explanation.

\section{M4: Couette flow}
The Couette flow is a fluid flow in between two reflective boundaries, where one of the two boundaries is moving relative to the other. 
This introduces a constant perturbation which means the system can never reach an equilibrium state.
The Couette flow can be used to model the the Earth's mantle and atmosphere and results in a Couette solution, where the velocity in the moving boundary direction is almost constant and decreasing from the moving boundary to the (relative) static one.

To model such a behaviour I use the following parameters:
\begin{enumerate}
    \item $L_{x}, L_{y} = 100, 100$
    \item $\rho(\textbf{r}, 0) = 0$
    \item $\textbf{u}(\textbf{r}, 0) = 0$
    \item $T = 100000$
    \item $top\_lid\_vel = 1$
    \item $\omega = 1$
\end{enumerate}
The corresponding results in the github repo can be found  \href{https://github.com/jonas27/pylbm/blob/master/milestones/m4/m4.ipynb}{here}. Unfortunately I could not provide the raw \textit{velocities.npy} data to reconstruct the results as they are too big. 

Figure \ref{fig:m4-1-streamplot} shows the Couette flow results in a \textit{streamplot} and \textit{imshow}.
\begin{figure}[ht]
\centering
\includegraphics[width=\columnwidth]{milestones/final/img/m4-1-streamplot.png}
\vspace*{-4mm}
\caption[Couette flow ]{Streamplot of the Couette flow after $t=200$, $t=400$, $t=600$ showing the velocity in the x direction on the grid. At the right end of the figure is an imshow of the velocities at $t=100,000$.}
\label{fig:m4-1-streamplot}
\end{figure}
After $t=200$ more than half the grid has a directional flow. At $t=400$ it is almost $90\%$ and at $t=600$ all particles flow in the x direction. 
After $100,000$ time steps the velocity reaches a state where the fluid flow stays (almost) the same, but no equilibrium state is reached, i.e. $f^{eq}_{cxy}\neq f_{cxy}$. 
It is decreasing from the top to bottom, with $v_{x}(y=1)=1$ and $v_{x}(y=98)=0$. 
The first and last rows are dry nodes.

To visualize the velocities over time I plot a single column over time on the lattice on the left side of figure \ref{fig:m4-1-velocities-over-time}.
\begin{figure}[ht]
\centering
\resizebox{\columnwidth}{!}{\large%% Creator: Matplotlib, PGF backend
%%
%% To include the figure in your LaTeX document, write
%%   \input{<filename>.pgf}
%%
%% Make sure the required packages are loaded in your preamble
%%   \usepackage{pgf}
%%
%% Also ensure that all the required font packages are loaded; for instance,
%% the lmodern package is sometimes necessary when using math font.
%%   \usepackage{lmodern}
%%
%% Figures using additional raster images can only be included by \input if
%% they are in the same directory as the main LaTeX file. For loading figures
%% from other directories you can use the `import` package
%%   \usepackage{import}
%%
%% and then include the figures with
%%   \import{<path to file>}{<filename>.pgf}
%%
%% Matplotlib used the following preamble
%%   \usepackage{fontspec}
%%   \setmainfont{DejaVuSerif.ttf}[Path=\detokenize{/home/joe/miniconda3/envs/high/lib/python3.9/site-packages/matplotlib/mpl-data/fonts/ttf/}]
%%   \setsansfont{DejaVuSans.ttf}[Path=\detokenize{/home/joe/miniconda3/envs/high/lib/python3.9/site-packages/matplotlib/mpl-data/fonts/ttf/}]
%%   \setmonofont{DejaVuSansMono.ttf}[Path=\detokenize{/home/joe/miniconda3/envs/high/lib/python3.9/site-packages/matplotlib/mpl-data/fonts/ttf/}]
%%
\begingroup%
\makeatletter%
\begin{pgfpicture}%
\pgfpathrectangle{\pgfpointorigin}{\pgfqpoint{10.000000in}{5.000000in}}%
\pgfusepath{use as bounding box, clip}%
\begin{pgfscope}%
\pgfsetbuttcap%
\pgfsetmiterjoin%
\pgfsetlinewidth{0.000000pt}%
\definecolor{currentstroke}{rgb}{1.000000,1.000000,1.000000}%
\pgfsetstrokecolor{currentstroke}%
\pgfsetstrokeopacity{0.000000}%
\pgfsetdash{}{0pt}%
\pgfpathmoveto{\pgfqpoint{0.000000in}{0.000000in}}%
\pgfpathlineto{\pgfqpoint{10.000000in}{0.000000in}}%
\pgfpathlineto{\pgfqpoint{10.000000in}{5.000000in}}%
\pgfpathlineto{\pgfqpoint{0.000000in}{5.000000in}}%
\pgfpathlineto{\pgfqpoint{0.000000in}{0.000000in}}%
\pgfpathclose%
\pgfusepath{}%
\end{pgfscope}%
\begin{pgfscope}%
\pgfsetbuttcap%
\pgfsetmiterjoin%
\definecolor{currentfill}{rgb}{1.000000,1.000000,1.000000}%
\pgfsetfillcolor{currentfill}%
\pgfsetlinewidth{0.000000pt}%
\definecolor{currentstroke}{rgb}{0.000000,0.000000,0.000000}%
\pgfsetstrokecolor{currentstroke}%
\pgfsetstrokeopacity{0.000000}%
\pgfsetdash{}{0pt}%
\pgfpathmoveto{\pgfqpoint{1.250000in}{0.625000in}}%
\pgfpathlineto{\pgfqpoint{9.000000in}{0.625000in}}%
\pgfpathlineto{\pgfqpoint{9.000000in}{4.400000in}}%
\pgfpathlineto{\pgfqpoint{1.250000in}{4.400000in}}%
\pgfpathlineto{\pgfqpoint{1.250000in}{0.625000in}}%
\pgfpathclose%
\pgfusepath{fill}%
\end{pgfscope}%
\begin{pgfscope}%
\pgfsetbuttcap%
\pgfsetroundjoin%
\definecolor{currentfill}{rgb}{0.000000,0.000000,0.000000}%
\pgfsetfillcolor{currentfill}%
\pgfsetlinewidth{0.803000pt}%
\definecolor{currentstroke}{rgb}{0.000000,0.000000,0.000000}%
\pgfsetstrokecolor{currentstroke}%
\pgfsetdash{}{0pt}%
\pgfsys@defobject{currentmarker}{\pgfqpoint{0.000000in}{-0.048611in}}{\pgfqpoint{0.000000in}{0.000000in}}{%
\pgfpathmoveto{\pgfqpoint{0.000000in}{0.000000in}}%
\pgfpathlineto{\pgfqpoint{0.000000in}{-0.048611in}}%
\pgfusepath{stroke,fill}%
}%
\begin{pgfscope}%
\pgfsys@transformshift{1.602273in}{0.625000in}%
\pgfsys@useobject{currentmarker}{}%
\end{pgfscope}%
\end{pgfscope}%
\begin{pgfscope}%
\definecolor{textcolor}{rgb}{0.000000,0.000000,0.000000}%
\pgfsetstrokecolor{textcolor}%
\pgfsetfillcolor{textcolor}%
\pgftext[x=1.602273in,y=0.527778in,,top]{\color{textcolor}\sffamily\fontsize{16.000000}{19.200000}\selectfont 0.0}%
\end{pgfscope}%
\begin{pgfscope}%
\pgfsetbuttcap%
\pgfsetroundjoin%
\definecolor{currentfill}{rgb}{0.000000,0.000000,0.000000}%
\pgfsetfillcolor{currentfill}%
\pgfsetlinewidth{0.803000pt}%
\definecolor{currentstroke}{rgb}{0.000000,0.000000,0.000000}%
\pgfsetstrokecolor{currentstroke}%
\pgfsetdash{}{0pt}%
\pgfsys@defobject{currentmarker}{\pgfqpoint{0.000000in}{-0.048611in}}{\pgfqpoint{0.000000in}{0.000000in}}{%
\pgfpathmoveto{\pgfqpoint{0.000000in}{0.000000in}}%
\pgfpathlineto{\pgfqpoint{0.000000in}{-0.048611in}}%
\pgfusepath{stroke,fill}%
}%
\begin{pgfscope}%
\pgfsys@transformshift{3.019244in}{0.625000in}%
\pgfsys@useobject{currentmarker}{}%
\end{pgfscope}%
\end{pgfscope}%
\begin{pgfscope}%
\definecolor{textcolor}{rgb}{0.000000,0.000000,0.000000}%
\pgfsetstrokecolor{textcolor}%
\pgfsetfillcolor{textcolor}%
\pgftext[x=3.019244in,y=0.527778in,,top]{\color{textcolor}\sffamily\fontsize{16.000000}{19.200000}\selectfont 0.2}%
\end{pgfscope}%
\begin{pgfscope}%
\pgfsetbuttcap%
\pgfsetroundjoin%
\definecolor{currentfill}{rgb}{0.000000,0.000000,0.000000}%
\pgfsetfillcolor{currentfill}%
\pgfsetlinewidth{0.803000pt}%
\definecolor{currentstroke}{rgb}{0.000000,0.000000,0.000000}%
\pgfsetstrokecolor{currentstroke}%
\pgfsetdash{}{0pt}%
\pgfsys@defobject{currentmarker}{\pgfqpoint{0.000000in}{-0.048611in}}{\pgfqpoint{0.000000in}{0.000000in}}{%
\pgfpathmoveto{\pgfqpoint{0.000000in}{0.000000in}}%
\pgfpathlineto{\pgfqpoint{0.000000in}{-0.048611in}}%
\pgfusepath{stroke,fill}%
}%
\begin{pgfscope}%
\pgfsys@transformshift{4.436216in}{0.625000in}%
\pgfsys@useobject{currentmarker}{}%
\end{pgfscope}%
\end{pgfscope}%
\begin{pgfscope}%
\definecolor{textcolor}{rgb}{0.000000,0.000000,0.000000}%
\pgfsetstrokecolor{textcolor}%
\pgfsetfillcolor{textcolor}%
\pgftext[x=4.436216in,y=0.527778in,,top]{\color{textcolor}\sffamily\fontsize{16.000000}{19.200000}\selectfont 0.4}%
\end{pgfscope}%
\begin{pgfscope}%
\pgfsetbuttcap%
\pgfsetroundjoin%
\definecolor{currentfill}{rgb}{0.000000,0.000000,0.000000}%
\pgfsetfillcolor{currentfill}%
\pgfsetlinewidth{0.803000pt}%
\definecolor{currentstroke}{rgb}{0.000000,0.000000,0.000000}%
\pgfsetstrokecolor{currentstroke}%
\pgfsetdash{}{0pt}%
\pgfsys@defobject{currentmarker}{\pgfqpoint{0.000000in}{-0.048611in}}{\pgfqpoint{0.000000in}{0.000000in}}{%
\pgfpathmoveto{\pgfqpoint{0.000000in}{0.000000in}}%
\pgfpathlineto{\pgfqpoint{0.000000in}{-0.048611in}}%
\pgfusepath{stroke,fill}%
}%
\begin{pgfscope}%
\pgfsys@transformshift{5.853188in}{0.625000in}%
\pgfsys@useobject{currentmarker}{}%
\end{pgfscope}%
\end{pgfscope}%
\begin{pgfscope}%
\definecolor{textcolor}{rgb}{0.000000,0.000000,0.000000}%
\pgfsetstrokecolor{textcolor}%
\pgfsetfillcolor{textcolor}%
\pgftext[x=5.853188in,y=0.527778in,,top]{\color{textcolor}\sffamily\fontsize{16.000000}{19.200000}\selectfont 0.6}%
\end{pgfscope}%
\begin{pgfscope}%
\pgfsetbuttcap%
\pgfsetroundjoin%
\definecolor{currentfill}{rgb}{0.000000,0.000000,0.000000}%
\pgfsetfillcolor{currentfill}%
\pgfsetlinewidth{0.803000pt}%
\definecolor{currentstroke}{rgb}{0.000000,0.000000,0.000000}%
\pgfsetstrokecolor{currentstroke}%
\pgfsetdash{}{0pt}%
\pgfsys@defobject{currentmarker}{\pgfqpoint{0.000000in}{-0.048611in}}{\pgfqpoint{0.000000in}{0.000000in}}{%
\pgfpathmoveto{\pgfqpoint{0.000000in}{0.000000in}}%
\pgfpathlineto{\pgfqpoint{0.000000in}{-0.048611in}}%
\pgfusepath{stroke,fill}%
}%
\begin{pgfscope}%
\pgfsys@transformshift{7.270160in}{0.625000in}%
\pgfsys@useobject{currentmarker}{}%
\end{pgfscope}%
\end{pgfscope}%
\begin{pgfscope}%
\definecolor{textcolor}{rgb}{0.000000,0.000000,0.000000}%
\pgfsetstrokecolor{textcolor}%
\pgfsetfillcolor{textcolor}%
\pgftext[x=7.270160in,y=0.527778in,,top]{\color{textcolor}\sffamily\fontsize{16.000000}{19.200000}\selectfont 0.8}%
\end{pgfscope}%
\begin{pgfscope}%
\pgfsetbuttcap%
\pgfsetroundjoin%
\definecolor{currentfill}{rgb}{0.000000,0.000000,0.000000}%
\pgfsetfillcolor{currentfill}%
\pgfsetlinewidth{0.803000pt}%
\definecolor{currentstroke}{rgb}{0.000000,0.000000,0.000000}%
\pgfsetstrokecolor{currentstroke}%
\pgfsetdash{}{0pt}%
\pgfsys@defobject{currentmarker}{\pgfqpoint{0.000000in}{-0.048611in}}{\pgfqpoint{0.000000in}{0.000000in}}{%
\pgfpathmoveto{\pgfqpoint{0.000000in}{0.000000in}}%
\pgfpathlineto{\pgfqpoint{0.000000in}{-0.048611in}}%
\pgfusepath{stroke,fill}%
}%
\begin{pgfscope}%
\pgfsys@transformshift{8.687131in}{0.625000in}%
\pgfsys@useobject{currentmarker}{}%
\end{pgfscope}%
\end{pgfscope}%
\begin{pgfscope}%
\definecolor{textcolor}{rgb}{0.000000,0.000000,0.000000}%
\pgfsetstrokecolor{textcolor}%
\pgfsetfillcolor{textcolor}%
\pgftext[x=8.687131in,y=0.527778in,,top]{\color{textcolor}\sffamily\fontsize{16.000000}{19.200000}\selectfont 1.0}%
\end{pgfscope}%
\begin{pgfscope}%
\definecolor{textcolor}{rgb}{0.000000,0.000000,0.000000}%
\pgfsetstrokecolor{textcolor}%
\pgfsetfillcolor{textcolor}%
\pgftext[x=5.125000in,y=0.257162in,,top]{\color{textcolor}\sffamily\fontsize{20.000000}{24.000000}\selectfont \(\displaystyle v_{x}\)}%
\end{pgfscope}%
\begin{pgfscope}%
\pgfsetbuttcap%
\pgfsetroundjoin%
\definecolor{currentfill}{rgb}{0.000000,0.000000,0.000000}%
\pgfsetfillcolor{currentfill}%
\pgfsetlinewidth{0.803000pt}%
\definecolor{currentstroke}{rgb}{0.000000,0.000000,0.000000}%
\pgfsetstrokecolor{currentstroke}%
\pgfsetdash{}{0pt}%
\pgfsys@defobject{currentmarker}{\pgfqpoint{-0.048611in}{0.000000in}}{\pgfqpoint{-0.000000in}{0.000000in}}{%
\pgfpathmoveto{\pgfqpoint{-0.000000in}{0.000000in}}%
\pgfpathlineto{\pgfqpoint{-0.048611in}{0.000000in}}%
\pgfusepath{stroke,fill}%
}%
\begin{pgfscope}%
\pgfsys@transformshift{1.250000in}{0.761211in}%
\pgfsys@useobject{currentmarker}{}%
\end{pgfscope}%
\end{pgfscope}%
\begin{pgfscope}%
\definecolor{textcolor}{rgb}{0.000000,0.000000,0.000000}%
\pgfsetstrokecolor{textcolor}%
\pgfsetfillcolor{textcolor}%
\pgftext[x=1.011393in, y=0.676793in, left, base]{\color{textcolor}\sffamily\fontsize{16.000000}{19.200000}\selectfont 0}%
\end{pgfscope}%
\begin{pgfscope}%
\pgfsetbuttcap%
\pgfsetroundjoin%
\definecolor{currentfill}{rgb}{0.000000,0.000000,0.000000}%
\pgfsetfillcolor{currentfill}%
\pgfsetlinewidth{0.803000pt}%
\definecolor{currentstroke}{rgb}{0.000000,0.000000,0.000000}%
\pgfsetstrokecolor{currentstroke}%
\pgfsetdash{}{0pt}%
\pgfsys@defobject{currentmarker}{\pgfqpoint{-0.048611in}{0.000000in}}{\pgfqpoint{-0.000000in}{0.000000in}}{%
\pgfpathmoveto{\pgfqpoint{-0.000000in}{0.000000in}}%
\pgfpathlineto{\pgfqpoint{-0.048611in}{0.000000in}}%
\pgfusepath{stroke,fill}%
}%
\begin{pgfscope}%
\pgfsys@transformshift{1.250000in}{1.468803in}%
\pgfsys@useobject{currentmarker}{}%
\end{pgfscope}%
\end{pgfscope}%
\begin{pgfscope}%
\definecolor{textcolor}{rgb}{0.000000,0.000000,0.000000}%
\pgfsetstrokecolor{textcolor}%
\pgfsetfillcolor{textcolor}%
\pgftext[x=0.870009in, y=1.384384in, left, base]{\color{textcolor}\sffamily\fontsize{16.000000}{19.200000}\selectfont 20}%
\end{pgfscope}%
\begin{pgfscope}%
\pgfsetbuttcap%
\pgfsetroundjoin%
\definecolor{currentfill}{rgb}{0.000000,0.000000,0.000000}%
\pgfsetfillcolor{currentfill}%
\pgfsetlinewidth{0.803000pt}%
\definecolor{currentstroke}{rgb}{0.000000,0.000000,0.000000}%
\pgfsetstrokecolor{currentstroke}%
\pgfsetdash{}{0pt}%
\pgfsys@defobject{currentmarker}{\pgfqpoint{-0.048611in}{0.000000in}}{\pgfqpoint{-0.000000in}{0.000000in}}{%
\pgfpathmoveto{\pgfqpoint{-0.000000in}{0.000000in}}%
\pgfpathlineto{\pgfqpoint{-0.048611in}{0.000000in}}%
\pgfusepath{stroke,fill}%
}%
\begin{pgfscope}%
\pgfsys@transformshift{1.250000in}{2.176394in}%
\pgfsys@useobject{currentmarker}{}%
\end{pgfscope}%
\end{pgfscope}%
\begin{pgfscope}%
\definecolor{textcolor}{rgb}{0.000000,0.000000,0.000000}%
\pgfsetstrokecolor{textcolor}%
\pgfsetfillcolor{textcolor}%
\pgftext[x=0.870009in, y=2.091976in, left, base]{\color{textcolor}\sffamily\fontsize{16.000000}{19.200000}\selectfont 40}%
\end{pgfscope}%
\begin{pgfscope}%
\pgfsetbuttcap%
\pgfsetroundjoin%
\definecolor{currentfill}{rgb}{0.000000,0.000000,0.000000}%
\pgfsetfillcolor{currentfill}%
\pgfsetlinewidth{0.803000pt}%
\definecolor{currentstroke}{rgb}{0.000000,0.000000,0.000000}%
\pgfsetstrokecolor{currentstroke}%
\pgfsetdash{}{0pt}%
\pgfsys@defobject{currentmarker}{\pgfqpoint{-0.048611in}{0.000000in}}{\pgfqpoint{-0.000000in}{0.000000in}}{%
\pgfpathmoveto{\pgfqpoint{-0.000000in}{0.000000in}}%
\pgfpathlineto{\pgfqpoint{-0.048611in}{0.000000in}}%
\pgfusepath{stroke,fill}%
}%
\begin{pgfscope}%
\pgfsys@transformshift{1.250000in}{2.883985in}%
\pgfsys@useobject{currentmarker}{}%
\end{pgfscope}%
\end{pgfscope}%
\begin{pgfscope}%
\definecolor{textcolor}{rgb}{0.000000,0.000000,0.000000}%
\pgfsetstrokecolor{textcolor}%
\pgfsetfillcolor{textcolor}%
\pgftext[x=0.870009in, y=2.799567in, left, base]{\color{textcolor}\sffamily\fontsize{16.000000}{19.200000}\selectfont 60}%
\end{pgfscope}%
\begin{pgfscope}%
\pgfsetbuttcap%
\pgfsetroundjoin%
\definecolor{currentfill}{rgb}{0.000000,0.000000,0.000000}%
\pgfsetfillcolor{currentfill}%
\pgfsetlinewidth{0.803000pt}%
\definecolor{currentstroke}{rgb}{0.000000,0.000000,0.000000}%
\pgfsetstrokecolor{currentstroke}%
\pgfsetdash{}{0pt}%
\pgfsys@defobject{currentmarker}{\pgfqpoint{-0.048611in}{0.000000in}}{\pgfqpoint{-0.000000in}{0.000000in}}{%
\pgfpathmoveto{\pgfqpoint{-0.000000in}{0.000000in}}%
\pgfpathlineto{\pgfqpoint{-0.048611in}{0.000000in}}%
\pgfusepath{stroke,fill}%
}%
\begin{pgfscope}%
\pgfsys@transformshift{1.250000in}{3.591577in}%
\pgfsys@useobject{currentmarker}{}%
\end{pgfscope}%
\end{pgfscope}%
\begin{pgfscope}%
\definecolor{textcolor}{rgb}{0.000000,0.000000,0.000000}%
\pgfsetstrokecolor{textcolor}%
\pgfsetfillcolor{textcolor}%
\pgftext[x=0.870009in, y=3.507158in, left, base]{\color{textcolor}\sffamily\fontsize{16.000000}{19.200000}\selectfont 80}%
\end{pgfscope}%
\begin{pgfscope}%
\pgfsetbuttcap%
\pgfsetroundjoin%
\definecolor{currentfill}{rgb}{0.000000,0.000000,0.000000}%
\pgfsetfillcolor{currentfill}%
\pgfsetlinewidth{0.803000pt}%
\definecolor{currentstroke}{rgb}{0.000000,0.000000,0.000000}%
\pgfsetstrokecolor{currentstroke}%
\pgfsetdash{}{0pt}%
\pgfsys@defobject{currentmarker}{\pgfqpoint{-0.048611in}{0.000000in}}{\pgfqpoint{-0.000000in}{0.000000in}}{%
\pgfpathmoveto{\pgfqpoint{-0.000000in}{0.000000in}}%
\pgfpathlineto{\pgfqpoint{-0.048611in}{0.000000in}}%
\pgfusepath{stroke,fill}%
}%
\begin{pgfscope}%
\pgfsys@transformshift{1.250000in}{4.299168in}%
\pgfsys@useobject{currentmarker}{}%
\end{pgfscope}%
\end{pgfscope}%
\begin{pgfscope}%
\definecolor{textcolor}{rgb}{0.000000,0.000000,0.000000}%
\pgfsetstrokecolor{textcolor}%
\pgfsetfillcolor{textcolor}%
\pgftext[x=0.728624in, y=4.214750in, left, base]{\color{textcolor}\sffamily\fontsize{16.000000}{19.200000}\selectfont 100}%
\end{pgfscope}%
\begin{pgfscope}%
\definecolor{textcolor}{rgb}{0.000000,0.000000,0.000000}%
\pgfsetstrokecolor{textcolor}%
\pgfsetfillcolor{textcolor}%
\pgftext[x=0.673069in,y=2.512500in,,bottom,rotate=90.000000]{\color{textcolor}\sffamily\fontsize{20.000000}{24.000000}\selectfont y}%
\end{pgfscope}%
\begin{pgfscope}%
\pgfpathrectangle{\pgfqpoint{1.250000in}{0.625000in}}{\pgfqpoint{7.750000in}{3.775000in}}%
\pgfusepath{clip}%
\pgfsetrectcap%
\pgfsetroundjoin%
\pgfsetlinewidth{1.505625pt}%
\definecolor{currentstroke}{rgb}{0.121569,0.466667,0.705882}%
\pgfsetstrokecolor{currentstroke}%
\pgfsetdash{}{0pt}%
\pgfpathmoveto{\pgfqpoint{1.602273in}{0.796591in}}%
\pgfpathlineto{\pgfqpoint{1.602273in}{0.831970in}}%
\pgfpathlineto{\pgfqpoint{1.602273in}{0.867350in}}%
\pgfpathlineto{\pgfqpoint{1.602273in}{0.902730in}}%
\pgfpathlineto{\pgfqpoint{1.602273in}{0.938109in}}%
\pgfpathlineto{\pgfqpoint{1.602273in}{0.973489in}}%
\pgfpathlineto{\pgfqpoint{1.602273in}{1.008868in}}%
\pgfpathlineto{\pgfqpoint{1.602273in}{1.044248in}}%
\pgfpathlineto{\pgfqpoint{1.602273in}{1.079627in}}%
\pgfpathlineto{\pgfqpoint{1.602273in}{1.115007in}}%
\pgfpathlineto{\pgfqpoint{1.602273in}{1.150387in}}%
\pgfpathlineto{\pgfqpoint{1.602273in}{1.185766in}}%
\pgfpathlineto{\pgfqpoint{1.602273in}{1.221146in}}%
\pgfpathlineto{\pgfqpoint{1.602273in}{1.256525in}}%
\pgfpathlineto{\pgfqpoint{1.602273in}{1.291905in}}%
\pgfpathlineto{\pgfqpoint{1.602273in}{1.327284in}}%
\pgfpathlineto{\pgfqpoint{1.602273in}{1.362664in}}%
\pgfpathlineto{\pgfqpoint{1.602273in}{1.398044in}}%
\pgfpathlineto{\pgfqpoint{1.602273in}{1.433423in}}%
\pgfpathlineto{\pgfqpoint{1.602273in}{1.468803in}}%
\pgfpathlineto{\pgfqpoint{1.602273in}{1.504182in}}%
\pgfpathlineto{\pgfqpoint{1.602273in}{1.539562in}}%
\pgfpathlineto{\pgfqpoint{1.602273in}{1.574941in}}%
\pgfpathlineto{\pgfqpoint{1.602273in}{1.610321in}}%
\pgfpathlineto{\pgfqpoint{1.602273in}{1.645701in}}%
\pgfpathlineto{\pgfqpoint{1.602273in}{1.681080in}}%
\pgfpathlineto{\pgfqpoint{1.602273in}{1.716460in}}%
\pgfpathlineto{\pgfqpoint{1.602273in}{1.751839in}}%
\pgfpathlineto{\pgfqpoint{1.602273in}{1.787219in}}%
\pgfpathlineto{\pgfqpoint{1.602273in}{1.822598in}}%
\pgfpathlineto{\pgfqpoint{1.602273in}{1.857978in}}%
\pgfpathlineto{\pgfqpoint{1.602273in}{1.893358in}}%
\pgfpathlineto{\pgfqpoint{1.602273in}{1.928737in}}%
\pgfpathlineto{\pgfqpoint{1.602273in}{1.964117in}}%
\pgfpathlineto{\pgfqpoint{1.602273in}{1.999496in}}%
\pgfpathlineto{\pgfqpoint{1.602273in}{2.034876in}}%
\pgfpathlineto{\pgfqpoint{1.602273in}{2.070255in}}%
\pgfpathlineto{\pgfqpoint{1.602273in}{2.105635in}}%
\pgfpathlineto{\pgfqpoint{1.602273in}{2.141015in}}%
\pgfpathlineto{\pgfqpoint{1.602273in}{2.176394in}}%
\pgfpathlineto{\pgfqpoint{1.602273in}{2.211774in}}%
\pgfpathlineto{\pgfqpoint{1.602273in}{2.247153in}}%
\pgfpathlineto{\pgfqpoint{1.602273in}{2.282533in}}%
\pgfpathlineto{\pgfqpoint{1.602273in}{2.317912in}}%
\pgfpathlineto{\pgfqpoint{1.602273in}{2.353292in}}%
\pgfpathlineto{\pgfqpoint{1.602273in}{2.388672in}}%
\pgfpathlineto{\pgfqpoint{1.602273in}{2.424051in}}%
\pgfpathlineto{\pgfqpoint{1.602273in}{2.459431in}}%
\pgfpathlineto{\pgfqpoint{1.602273in}{2.494810in}}%
\pgfpathlineto{\pgfqpoint{1.602273in}{2.530190in}}%
\pgfpathlineto{\pgfqpoint{1.602273in}{2.565569in}}%
\pgfpathlineto{\pgfqpoint{1.602273in}{2.600949in}}%
\pgfpathlineto{\pgfqpoint{1.602273in}{2.636328in}}%
\pgfpathlineto{\pgfqpoint{1.602273in}{2.671708in}}%
\pgfpathlineto{\pgfqpoint{1.602273in}{2.707088in}}%
\pgfpathlineto{\pgfqpoint{1.602273in}{2.742467in}}%
\pgfpathlineto{\pgfqpoint{1.602273in}{2.777847in}}%
\pgfpathlineto{\pgfqpoint{1.602273in}{2.813226in}}%
\pgfpathlineto{\pgfqpoint{1.602273in}{2.848606in}}%
\pgfpathlineto{\pgfqpoint{1.602273in}{2.883985in}}%
\pgfpathlineto{\pgfqpoint{1.602273in}{2.919365in}}%
\pgfpathlineto{\pgfqpoint{1.602273in}{2.954745in}}%
\pgfpathlineto{\pgfqpoint{1.602273in}{2.990124in}}%
\pgfpathlineto{\pgfqpoint{1.602273in}{3.025504in}}%
\pgfpathlineto{\pgfqpoint{1.602273in}{3.060883in}}%
\pgfpathlineto{\pgfqpoint{1.602273in}{3.096263in}}%
\pgfpathlineto{\pgfqpoint{1.602273in}{3.131642in}}%
\pgfpathlineto{\pgfqpoint{1.602274in}{3.167022in}}%
\pgfpathlineto{\pgfqpoint{1.602275in}{3.202402in}}%
\pgfpathlineto{\pgfqpoint{1.602278in}{3.237781in}}%
\pgfpathlineto{\pgfqpoint{1.602286in}{3.273161in}}%
\pgfpathlineto{\pgfqpoint{1.602303in}{3.308540in}}%
\pgfpathlineto{\pgfqpoint{1.602342in}{3.343920in}}%
\pgfpathlineto{\pgfqpoint{1.602425in}{3.379299in}}%
\pgfpathlineto{\pgfqpoint{1.602598in}{3.414679in}}%
\pgfpathlineto{\pgfqpoint{1.602948in}{3.450059in}}%
\pgfpathlineto{\pgfqpoint{1.603636in}{3.485438in}}%
\pgfpathlineto{\pgfqpoint{1.604948in}{3.520818in}}%
\pgfpathlineto{\pgfqpoint{1.607375in}{3.556197in}}%
\pgfpathlineto{\pgfqpoint{1.611731in}{3.591577in}}%
\pgfpathlineto{\pgfqpoint{1.619320in}{3.626956in}}%
\pgfpathlineto{\pgfqpoint{1.632149in}{3.662336in}}%
\pgfpathlineto{\pgfqpoint{1.653196in}{3.697716in}}%
\pgfpathlineto{\pgfqpoint{1.686701in}{3.733095in}}%
\pgfpathlineto{\pgfqpoint{1.738466in}{3.768475in}}%
\pgfpathlineto{\pgfqpoint{1.816075in}{3.803854in}}%
\pgfpathlineto{\pgfqpoint{1.928993in}{3.839234in}}%
\pgfpathlineto{\pgfqpoint{2.088429in}{3.874613in}}%
\pgfpathlineto{\pgfqpoint{2.306896in}{3.909993in}}%
\pgfpathlineto{\pgfqpoint{2.597403in}{3.945373in}}%
\pgfpathlineto{\pgfqpoint{2.972291in}{3.980752in}}%
\pgfpathlineto{\pgfqpoint{3.441771in}{4.016132in}}%
\pgfpathlineto{\pgfqpoint{4.012338in}{4.051511in}}%
\pgfpathlineto{\pgfqpoint{4.685264in}{4.086891in}}%
\pgfpathlineto{\pgfqpoint{5.455458in}{4.122270in}}%
\pgfpathlineto{\pgfqpoint{6.310925in}{4.157650in}}%
\pgfpathlineto{\pgfqpoint{7.233025in}{4.193030in}}%
\pgfpathlineto{\pgfqpoint{8.197570in}{4.228409in}}%
\pgfusepath{stroke}%
\end{pgfscope}%
\begin{pgfscope}%
\pgfpathrectangle{\pgfqpoint{1.250000in}{0.625000in}}{\pgfqpoint{7.750000in}{3.775000in}}%
\pgfusepath{clip}%
\pgfsetrectcap%
\pgfsetroundjoin%
\pgfsetlinewidth{1.505625pt}%
\definecolor{currentstroke}{rgb}{1.000000,0.498039,0.054902}%
\pgfsetstrokecolor{currentstroke}%
\pgfsetdash{}{0pt}%
\pgfpathmoveto{\pgfqpoint{1.602497in}{0.796591in}}%
\pgfpathlineto{\pgfqpoint{1.602949in}{0.831970in}}%
\pgfpathlineto{\pgfqpoint{1.603414in}{0.867350in}}%
\pgfpathlineto{\pgfqpoint{1.603899in}{0.902730in}}%
\pgfpathlineto{\pgfqpoint{1.604413in}{0.938109in}}%
\pgfpathlineto{\pgfqpoint{1.604966in}{0.973489in}}%
\pgfpathlineto{\pgfqpoint{1.605566in}{1.008868in}}%
\pgfpathlineto{\pgfqpoint{1.606224in}{1.044248in}}%
\pgfpathlineto{\pgfqpoint{1.606950in}{1.079627in}}%
\pgfpathlineto{\pgfqpoint{1.607756in}{1.115007in}}%
\pgfpathlineto{\pgfqpoint{1.608653in}{1.150387in}}%
\pgfpathlineto{\pgfqpoint{1.609656in}{1.185766in}}%
\pgfpathlineto{\pgfqpoint{1.610779in}{1.221146in}}%
\pgfpathlineto{\pgfqpoint{1.612037in}{1.256525in}}%
\pgfpathlineto{\pgfqpoint{1.613448in}{1.291905in}}%
\pgfpathlineto{\pgfqpoint{1.615029in}{1.327284in}}%
\pgfpathlineto{\pgfqpoint{1.616803in}{1.362664in}}%
\pgfpathlineto{\pgfqpoint{1.618789in}{1.398044in}}%
\pgfpathlineto{\pgfqpoint{1.621014in}{1.433423in}}%
\pgfpathlineto{\pgfqpoint{1.623501in}{1.468803in}}%
\pgfpathlineto{\pgfqpoint{1.626281in}{1.504182in}}%
\pgfpathlineto{\pgfqpoint{1.629383in}{1.539562in}}%
\pgfpathlineto{\pgfqpoint{1.632842in}{1.574941in}}%
\pgfpathlineto{\pgfqpoint{1.636692in}{1.610321in}}%
\pgfpathlineto{\pgfqpoint{1.640973in}{1.645701in}}%
\pgfpathlineto{\pgfqpoint{1.645726in}{1.681080in}}%
\pgfpathlineto{\pgfqpoint{1.650997in}{1.716460in}}%
\pgfpathlineto{\pgfqpoint{1.656834in}{1.751839in}}%
\pgfpathlineto{\pgfqpoint{1.663289in}{1.787219in}}%
\pgfpathlineto{\pgfqpoint{1.670416in}{1.822598in}}%
\pgfpathlineto{\pgfqpoint{1.678276in}{1.857978in}}%
\pgfpathlineto{\pgfqpoint{1.686931in}{1.893358in}}%
\pgfpathlineto{\pgfqpoint{1.696448in}{1.928737in}}%
\pgfpathlineto{\pgfqpoint{1.706898in}{1.964117in}}%
\pgfpathlineto{\pgfqpoint{1.718357in}{1.999496in}}%
\pgfpathlineto{\pgfqpoint{1.730903in}{2.034876in}}%
\pgfpathlineto{\pgfqpoint{1.744621in}{2.070255in}}%
\pgfpathlineto{\pgfqpoint{1.759599in}{2.105635in}}%
\pgfpathlineto{\pgfqpoint{1.775928in}{2.141015in}}%
\pgfpathlineto{\pgfqpoint{1.793707in}{2.176394in}}%
\pgfpathlineto{\pgfqpoint{1.813034in}{2.211774in}}%
\pgfpathlineto{\pgfqpoint{1.834017in}{2.247153in}}%
\pgfpathlineto{\pgfqpoint{1.856763in}{2.282533in}}%
\pgfpathlineto{\pgfqpoint{1.881386in}{2.317912in}}%
\pgfpathlineto{\pgfqpoint{1.908002in}{2.353292in}}%
\pgfpathlineto{\pgfqpoint{1.936733in}{2.388672in}}%
\pgfpathlineto{\pgfqpoint{1.967701in}{2.424051in}}%
\pgfpathlineto{\pgfqpoint{2.001033in}{2.459431in}}%
\pgfpathlineto{\pgfqpoint{2.036859in}{2.494810in}}%
\pgfpathlineto{\pgfqpoint{2.075310in}{2.530190in}}%
\pgfpathlineto{\pgfqpoint{2.116519in}{2.565569in}}%
\pgfpathlineto{\pgfqpoint{2.160622in}{2.600949in}}%
\pgfpathlineto{\pgfqpoint{2.207755in}{2.636328in}}%
\pgfpathlineto{\pgfqpoint{2.258052in}{2.671708in}}%
\pgfpathlineto{\pgfqpoint{2.311651in}{2.707088in}}%
\pgfpathlineto{\pgfqpoint{2.368686in}{2.742467in}}%
\pgfpathlineto{\pgfqpoint{2.429292in}{2.777847in}}%
\pgfpathlineto{\pgfqpoint{2.493599in}{2.813226in}}%
\pgfpathlineto{\pgfqpoint{2.561736in}{2.848606in}}%
\pgfpathlineto{\pgfqpoint{2.633829in}{2.883985in}}%
\pgfpathlineto{\pgfqpoint{2.709998in}{2.919365in}}%
\pgfpathlineto{\pgfqpoint{2.790359in}{2.954745in}}%
\pgfpathlineto{\pgfqpoint{2.875021in}{2.990124in}}%
\pgfpathlineto{\pgfqpoint{2.964087in}{3.025504in}}%
\pgfpathlineto{\pgfqpoint{3.057654in}{3.060883in}}%
\pgfpathlineto{\pgfqpoint{3.155807in}{3.096263in}}%
\pgfpathlineto{\pgfqpoint{3.258624in}{3.131642in}}%
\pgfpathlineto{\pgfqpoint{3.366174in}{3.167022in}}%
\pgfpathlineto{\pgfqpoint{3.478514in}{3.202402in}}%
\pgfpathlineto{\pgfqpoint{3.595689in}{3.237781in}}%
\pgfpathlineto{\pgfqpoint{3.717733in}{3.273161in}}%
\pgfpathlineto{\pgfqpoint{3.844667in}{3.308540in}}%
\pgfpathlineto{\pgfqpoint{3.976499in}{3.343920in}}%
\pgfpathlineto{\pgfqpoint{4.113222in}{3.379299in}}%
\pgfpathlineto{\pgfqpoint{4.254814in}{3.414679in}}%
\pgfpathlineto{\pgfqpoint{4.401241in}{3.450059in}}%
\pgfpathlineto{\pgfqpoint{4.552451in}{3.485438in}}%
\pgfpathlineto{\pgfqpoint{4.708378in}{3.520818in}}%
\pgfpathlineto{\pgfqpoint{4.868938in}{3.556197in}}%
\pgfpathlineto{\pgfqpoint{5.034034in}{3.591577in}}%
\pgfpathlineto{\pgfqpoint{5.203551in}{3.626956in}}%
\pgfpathlineto{\pgfqpoint{5.377360in}{3.662336in}}%
\pgfpathlineto{\pgfqpoint{5.555314in}{3.697716in}}%
\pgfpathlineto{\pgfqpoint{5.737253in}{3.733095in}}%
\pgfpathlineto{\pgfqpoint{5.922999in}{3.768475in}}%
\pgfpathlineto{\pgfqpoint{6.112363in}{3.803854in}}%
\pgfpathlineto{\pgfqpoint{6.305139in}{3.839234in}}%
\pgfpathlineto{\pgfqpoint{6.501107in}{3.874613in}}%
\pgfpathlineto{\pgfqpoint{6.700038in}{3.909993in}}%
\pgfpathlineto{\pgfqpoint{6.901686in}{3.945373in}}%
\pgfpathlineto{\pgfqpoint{7.105798in}{3.980752in}}%
\pgfpathlineto{\pgfqpoint{7.312109in}{4.016132in}}%
\pgfpathlineto{\pgfqpoint{7.520345in}{4.051511in}}%
\pgfpathlineto{\pgfqpoint{7.730223in}{4.086891in}}%
\pgfpathlineto{\pgfqpoint{7.941454in}{4.122270in}}%
\pgfpathlineto{\pgfqpoint{8.153745in}{4.157650in}}%
\pgfpathlineto{\pgfqpoint{8.366795in}{4.193030in}}%
\pgfpathlineto{\pgfqpoint{8.580302in}{4.228409in}}%
\pgfusepath{stroke}%
\end{pgfscope}%
\begin{pgfscope}%
\pgfpathrectangle{\pgfqpoint{1.250000in}{0.625000in}}{\pgfqpoint{7.750000in}{3.775000in}}%
\pgfusepath{clip}%
\pgfsetrectcap%
\pgfsetroundjoin%
\pgfsetlinewidth{1.505625pt}%
\definecolor{currentstroke}{rgb}{0.172549,0.627451,0.172549}%
\pgfsetstrokecolor{currentstroke}%
\pgfsetdash{}{0pt}%
\pgfpathmoveto{\pgfqpoint{1.606828in}{0.796591in}}%
\pgfpathlineto{\pgfqpoint{1.615957in}{0.831970in}}%
\pgfpathlineto{\pgfqpoint{1.625148in}{0.867350in}}%
\pgfpathlineto{\pgfqpoint{1.634438in}{0.902730in}}%
\pgfpathlineto{\pgfqpoint{1.643870in}{0.938109in}}%
\pgfpathlineto{\pgfqpoint{1.653483in}{0.973489in}}%
\pgfpathlineto{\pgfqpoint{1.663319in}{1.008868in}}%
\pgfpathlineto{\pgfqpoint{1.673418in}{1.044248in}}%
\pgfpathlineto{\pgfqpoint{1.683822in}{1.079627in}}%
\pgfpathlineto{\pgfqpoint{1.694572in}{1.115007in}}%
\pgfpathlineto{\pgfqpoint{1.705711in}{1.150387in}}%
\pgfpathlineto{\pgfqpoint{1.717281in}{1.185766in}}%
\pgfpathlineto{\pgfqpoint{1.729325in}{1.221146in}}%
\pgfpathlineto{\pgfqpoint{1.741887in}{1.256525in}}%
\pgfpathlineto{\pgfqpoint{1.755010in}{1.291905in}}%
\pgfpathlineto{\pgfqpoint{1.768740in}{1.327284in}}%
\pgfpathlineto{\pgfqpoint{1.783121in}{1.362664in}}%
\pgfpathlineto{\pgfqpoint{1.798198in}{1.398044in}}%
\pgfpathlineto{\pgfqpoint{1.814019in}{1.433423in}}%
\pgfpathlineto{\pgfqpoint{1.830630in}{1.468803in}}%
\pgfpathlineto{\pgfqpoint{1.848079in}{1.504182in}}%
\pgfpathlineto{\pgfqpoint{1.866412in}{1.539562in}}%
\pgfpathlineto{\pgfqpoint{1.885680in}{1.574941in}}%
\pgfpathlineto{\pgfqpoint{1.905931in}{1.610321in}}%
\pgfpathlineto{\pgfqpoint{1.927215in}{1.645701in}}%
\pgfpathlineto{\pgfqpoint{1.949582in}{1.681080in}}%
\pgfpathlineto{\pgfqpoint{1.973082in}{1.716460in}}%
\pgfpathlineto{\pgfqpoint{1.997767in}{1.751839in}}%
\pgfpathlineto{\pgfqpoint{2.023688in}{1.787219in}}%
\pgfpathlineto{\pgfqpoint{2.050896in}{1.822598in}}%
\pgfpathlineto{\pgfqpoint{2.079445in}{1.857978in}}%
\pgfpathlineto{\pgfqpoint{2.109385in}{1.893358in}}%
\pgfpathlineto{\pgfqpoint{2.140769in}{1.928737in}}%
\pgfpathlineto{\pgfqpoint{2.173650in}{1.964117in}}%
\pgfpathlineto{\pgfqpoint{2.208080in}{1.999496in}}%
\pgfpathlineto{\pgfqpoint{2.244110in}{2.034876in}}%
\pgfpathlineto{\pgfqpoint{2.281794in}{2.070255in}}%
\pgfpathlineto{\pgfqpoint{2.321182in}{2.105635in}}%
\pgfpathlineto{\pgfqpoint{2.362326in}{2.141015in}}%
\pgfpathlineto{\pgfqpoint{2.405276in}{2.176394in}}%
\pgfpathlineto{\pgfqpoint{2.450082in}{2.211774in}}%
\pgfpathlineto{\pgfqpoint{2.496794in}{2.247153in}}%
\pgfpathlineto{\pgfqpoint{2.545459in}{2.282533in}}%
\pgfpathlineto{\pgfqpoint{2.596124in}{2.317912in}}%
\pgfpathlineto{\pgfqpoint{2.648837in}{2.353292in}}%
\pgfpathlineto{\pgfqpoint{2.703640in}{2.388672in}}%
\pgfpathlineto{\pgfqpoint{2.760577in}{2.424051in}}%
\pgfpathlineto{\pgfqpoint{2.819691in}{2.459431in}}%
\pgfpathlineto{\pgfqpoint{2.881019in}{2.494810in}}%
\pgfpathlineto{\pgfqpoint{2.944601in}{2.530190in}}%
\pgfpathlineto{\pgfqpoint{3.010472in}{2.565569in}}%
\pgfpathlineto{\pgfqpoint{3.078666in}{2.600949in}}%
\pgfpathlineto{\pgfqpoint{3.149214in}{2.636328in}}%
\pgfpathlineto{\pgfqpoint{3.222144in}{2.671708in}}%
\pgfpathlineto{\pgfqpoint{3.297483in}{2.707088in}}%
\pgfpathlineto{\pgfqpoint{3.375255in}{2.742467in}}%
\pgfpathlineto{\pgfqpoint{3.455478in}{2.777847in}}%
\pgfpathlineto{\pgfqpoint{3.538172in}{2.813226in}}%
\pgfpathlineto{\pgfqpoint{3.623349in}{2.848606in}}%
\pgfpathlineto{\pgfqpoint{3.711022in}{2.883985in}}%
\pgfpathlineto{\pgfqpoint{3.801196in}{2.919365in}}%
\pgfpathlineto{\pgfqpoint{3.893877in}{2.954745in}}%
\pgfpathlineto{\pgfqpoint{3.989065in}{2.990124in}}%
\pgfpathlineto{\pgfqpoint{4.086756in}{3.025504in}}%
\pgfpathlineto{\pgfqpoint{4.186942in}{3.060883in}}%
\pgfpathlineto{\pgfqpoint{4.289614in}{3.096263in}}%
\pgfpathlineto{\pgfqpoint{4.394755in}{3.131642in}}%
\pgfpathlineto{\pgfqpoint{4.502346in}{3.167022in}}%
\pgfpathlineto{\pgfqpoint{4.612365in}{3.202402in}}%
\pgfpathlineto{\pgfqpoint{4.724783in}{3.237781in}}%
\pgfpathlineto{\pgfqpoint{4.839569in}{3.273161in}}%
\pgfpathlineto{\pgfqpoint{4.956688in}{3.308540in}}%
\pgfpathlineto{\pgfqpoint{5.076100in}{3.343920in}}%
\pgfpathlineto{\pgfqpoint{5.197760in}{3.379299in}}%
\pgfpathlineto{\pgfqpoint{5.321620in}{3.414679in}}%
\pgfpathlineto{\pgfqpoint{5.447627in}{3.450059in}}%
\pgfpathlineto{\pgfqpoint{5.575726in}{3.485438in}}%
\pgfpathlineto{\pgfqpoint{5.705856in}{3.520818in}}%
\pgfpathlineto{\pgfqpoint{5.837953in}{3.556197in}}%
\pgfpathlineto{\pgfqpoint{5.971947in}{3.591577in}}%
\pgfpathlineto{\pgfqpoint{6.107767in}{3.626956in}}%
\pgfpathlineto{\pgfqpoint{6.245338in}{3.662336in}}%
\pgfpathlineto{\pgfqpoint{6.384579in}{3.697716in}}%
\pgfpathlineto{\pgfqpoint{6.525408in}{3.733095in}}%
\pgfpathlineto{\pgfqpoint{6.667740in}{3.768475in}}%
\pgfpathlineto{\pgfqpoint{6.811484in}{3.803854in}}%
\pgfpathlineto{\pgfqpoint{6.956549in}{3.839234in}}%
\pgfpathlineto{\pgfqpoint{7.102839in}{3.874613in}}%
\pgfpathlineto{\pgfqpoint{7.250258in}{3.909993in}}%
\pgfpathlineto{\pgfqpoint{7.398705in}{3.945373in}}%
\pgfpathlineto{\pgfqpoint{7.548078in}{3.980752in}}%
\pgfpathlineto{\pgfqpoint{7.698273in}{4.016132in}}%
\pgfpathlineto{\pgfqpoint{7.849184in}{4.051511in}}%
\pgfpathlineto{\pgfqpoint{8.000703in}{4.086891in}}%
\pgfpathlineto{\pgfqpoint{8.152723in}{4.122270in}}%
\pgfpathlineto{\pgfqpoint{8.305132in}{4.157650in}}%
\pgfpathlineto{\pgfqpoint{8.457820in}{4.193030in}}%
\pgfpathlineto{\pgfqpoint{8.610675in}{4.228409in}}%
\pgfusepath{stroke}%
\end{pgfscope}%
\begin{pgfscope}%
\pgfpathrectangle{\pgfqpoint{1.250000in}{0.625000in}}{\pgfqpoint{7.750000in}{3.775000in}}%
\pgfusepath{clip}%
\pgfsetrectcap%
\pgfsetroundjoin%
\pgfsetlinewidth{1.505625pt}%
\definecolor{currentstroke}{rgb}{0.839216,0.152941,0.156863}%
\pgfsetstrokecolor{currentstroke}%
\pgfsetdash{}{0pt}%
\pgfpathmoveto{\pgfqpoint{1.614090in}{0.796591in}}%
\pgfpathlineto{\pgfqpoint{1.637746in}{0.831970in}}%
\pgfpathlineto{\pgfqpoint{1.661468in}{0.867350in}}%
\pgfpathlineto{\pgfqpoint{1.685297in}{0.902730in}}%
\pgfpathlineto{\pgfqpoint{1.709278in}{0.938109in}}%
\pgfpathlineto{\pgfqpoint{1.733453in}{0.973489in}}%
\pgfpathlineto{\pgfqpoint{1.757866in}{1.008868in}}%
\pgfpathlineto{\pgfqpoint{1.782559in}{1.044248in}}%
\pgfpathlineto{\pgfqpoint{1.807576in}{1.079627in}}%
\pgfpathlineto{\pgfqpoint{1.832958in}{1.115007in}}%
\pgfpathlineto{\pgfqpoint{1.858750in}{1.150387in}}%
\pgfpathlineto{\pgfqpoint{1.884992in}{1.185766in}}%
\pgfpathlineto{\pgfqpoint{1.911727in}{1.221146in}}%
\pgfpathlineto{\pgfqpoint{1.938998in}{1.256525in}}%
\pgfpathlineto{\pgfqpoint{1.966845in}{1.291905in}}%
\pgfpathlineto{\pgfqpoint{1.995310in}{1.327284in}}%
\pgfpathlineto{\pgfqpoint{2.024435in}{1.362664in}}%
\pgfpathlineto{\pgfqpoint{2.054261in}{1.398044in}}%
\pgfpathlineto{\pgfqpoint{2.084826in}{1.433423in}}%
\pgfpathlineto{\pgfqpoint{2.116173in}{1.468803in}}%
\pgfpathlineto{\pgfqpoint{2.148340in}{1.504182in}}%
\pgfpathlineto{\pgfqpoint{2.181367in}{1.539562in}}%
\pgfpathlineto{\pgfqpoint{2.215293in}{1.574941in}}%
\pgfpathlineto{\pgfqpoint{2.250155in}{1.610321in}}%
\pgfpathlineto{\pgfqpoint{2.285992in}{1.645701in}}%
\pgfpathlineto{\pgfqpoint{2.322841in}{1.681080in}}%
\pgfpathlineto{\pgfqpoint{2.360738in}{1.716460in}}%
\pgfpathlineto{\pgfqpoint{2.399719in}{1.751839in}}%
\pgfpathlineto{\pgfqpoint{2.439820in}{1.787219in}}%
\pgfpathlineto{\pgfqpoint{2.481075in}{1.822598in}}%
\pgfpathlineto{\pgfqpoint{2.523517in}{1.857978in}}%
\pgfpathlineto{\pgfqpoint{2.567180in}{1.893358in}}%
\pgfpathlineto{\pgfqpoint{2.612096in}{1.928737in}}%
\pgfpathlineto{\pgfqpoint{2.658296in}{1.964117in}}%
\pgfpathlineto{\pgfqpoint{2.705810in}{1.999496in}}%
\pgfpathlineto{\pgfqpoint{2.754668in}{2.034876in}}%
\pgfpathlineto{\pgfqpoint{2.804897in}{2.070255in}}%
\pgfpathlineto{\pgfqpoint{2.856526in}{2.105635in}}%
\pgfpathlineto{\pgfqpoint{2.909579in}{2.141015in}}%
\pgfpathlineto{\pgfqpoint{2.964082in}{2.176394in}}%
\pgfpathlineto{\pgfqpoint{3.020058in}{2.211774in}}%
\pgfpathlineto{\pgfqpoint{3.077531in}{2.247153in}}%
\pgfpathlineto{\pgfqpoint{3.136520in}{2.282533in}}%
\pgfpathlineto{\pgfqpoint{3.197047in}{2.317912in}}%
\pgfpathlineto{\pgfqpoint{3.259128in}{2.353292in}}%
\pgfpathlineto{\pgfqpoint{3.322781in}{2.388672in}}%
\pgfpathlineto{\pgfqpoint{3.388022in}{2.424051in}}%
\pgfpathlineto{\pgfqpoint{3.454864in}{2.459431in}}%
\pgfpathlineto{\pgfqpoint{3.523320in}{2.494810in}}%
\pgfpathlineto{\pgfqpoint{3.593401in}{2.530190in}}%
\pgfpathlineto{\pgfqpoint{3.665115in}{2.565569in}}%
\pgfpathlineto{\pgfqpoint{3.738470in}{2.600949in}}%
\pgfpathlineto{\pgfqpoint{3.813472in}{2.636328in}}%
\pgfpathlineto{\pgfqpoint{3.890125in}{2.671708in}}%
\pgfpathlineto{\pgfqpoint{3.968430in}{2.707088in}}%
\pgfpathlineto{\pgfqpoint{4.048389in}{2.742467in}}%
\pgfpathlineto{\pgfqpoint{4.129999in}{2.777847in}}%
\pgfpathlineto{\pgfqpoint{4.213257in}{2.813226in}}%
\pgfpathlineto{\pgfqpoint{4.298159in}{2.848606in}}%
\pgfpathlineto{\pgfqpoint{4.384696in}{2.883985in}}%
\pgfpathlineto{\pgfqpoint{4.472861in}{2.919365in}}%
\pgfpathlineto{\pgfqpoint{4.562641in}{2.954745in}}%
\pgfpathlineto{\pgfqpoint{4.654023in}{2.990124in}}%
\pgfpathlineto{\pgfqpoint{4.746994in}{3.025504in}}%
\pgfpathlineto{\pgfqpoint{4.841535in}{3.060883in}}%
\pgfpathlineto{\pgfqpoint{4.937629in}{3.096263in}}%
\pgfpathlineto{\pgfqpoint{5.035254in}{3.131642in}}%
\pgfpathlineto{\pgfqpoint{5.134387in}{3.167022in}}%
\pgfpathlineto{\pgfqpoint{5.235005in}{3.202402in}}%
\pgfpathlineto{\pgfqpoint{5.337079in}{3.237781in}}%
\pgfpathlineto{\pgfqpoint{5.440582in}{3.273161in}}%
\pgfpathlineto{\pgfqpoint{5.545483in}{3.308540in}}%
\pgfpathlineto{\pgfqpoint{5.651749in}{3.343920in}}%
\pgfpathlineto{\pgfqpoint{5.759348in}{3.379299in}}%
\pgfpathlineto{\pgfqpoint{5.868242in}{3.414679in}}%
\pgfpathlineto{\pgfqpoint{5.978394in}{3.450059in}}%
\pgfpathlineto{\pgfqpoint{6.089764in}{3.485438in}}%
\pgfpathlineto{\pgfqpoint{6.202311in}{3.520818in}}%
\pgfpathlineto{\pgfqpoint{6.315993in}{3.556197in}}%
\pgfpathlineto{\pgfqpoint{6.430764in}{3.591577in}}%
\pgfpathlineto{\pgfqpoint{6.546579in}{3.626956in}}%
\pgfpathlineto{\pgfqpoint{6.663391in}{3.662336in}}%
\pgfpathlineto{\pgfqpoint{6.781151in}{3.697716in}}%
\pgfpathlineto{\pgfqpoint{6.899807in}{3.733095in}}%
\pgfpathlineto{\pgfqpoint{7.019310in}{3.768475in}}%
\pgfpathlineto{\pgfqpoint{7.139605in}{3.803854in}}%
\pgfpathlineto{\pgfqpoint{7.260639in}{3.839234in}}%
\pgfpathlineto{\pgfqpoint{7.382357in}{3.874613in}}%
\pgfpathlineto{\pgfqpoint{7.504704in}{3.909993in}}%
\pgfpathlineto{\pgfqpoint{7.627621in}{3.945373in}}%
\pgfpathlineto{\pgfqpoint{7.751051in}{3.980752in}}%
\pgfpathlineto{\pgfqpoint{7.874936in}{4.016132in}}%
\pgfpathlineto{\pgfqpoint{7.999217in}{4.051511in}}%
\pgfpathlineto{\pgfqpoint{8.123833in}{4.086891in}}%
\pgfpathlineto{\pgfqpoint{8.248724in}{4.122270in}}%
\pgfpathlineto{\pgfqpoint{8.373830in}{4.157650in}}%
\pgfpathlineto{\pgfqpoint{8.499089in}{4.193030in}}%
\pgfpathlineto{\pgfqpoint{8.624440in}{4.228409in}}%
\pgfusepath{stroke}%
\end{pgfscope}%
\begin{pgfscope}%
\pgfpathrectangle{\pgfqpoint{1.250000in}{0.625000in}}{\pgfqpoint{7.750000in}{3.775000in}}%
\pgfusepath{clip}%
\pgfsetrectcap%
\pgfsetroundjoin%
\pgfsetlinewidth{1.505625pt}%
\definecolor{currentstroke}{rgb}{0.580392,0.403922,0.741176}%
\pgfsetstrokecolor{currentstroke}%
\pgfsetdash{}{0pt}%
\pgfpathmoveto{\pgfqpoint{1.620646in}{0.796591in}}%
\pgfpathlineto{\pgfqpoint{1.657410in}{0.831970in}}%
\pgfpathlineto{\pgfqpoint{1.694227in}{0.867350in}}%
\pgfpathlineto{\pgfqpoint{1.731130in}{0.902730in}}%
\pgfpathlineto{\pgfqpoint{1.768155in}{0.938109in}}%
\pgfpathlineto{\pgfqpoint{1.805336in}{0.973489in}}%
\pgfpathlineto{\pgfqpoint{1.842707in}{1.008868in}}%
\pgfpathlineto{\pgfqpoint{1.880304in}{1.044248in}}%
\pgfpathlineto{\pgfqpoint{1.918159in}{1.079627in}}%
\pgfpathlineto{\pgfqpoint{1.956308in}{1.115007in}}%
\pgfpathlineto{\pgfqpoint{1.994783in}{1.150387in}}%
\pgfpathlineto{\pgfqpoint{2.033617in}{1.185766in}}%
\pgfpathlineto{\pgfqpoint{2.072845in}{1.221146in}}%
\pgfpathlineto{\pgfqpoint{2.112498in}{1.256525in}}%
\pgfpathlineto{\pgfqpoint{2.152610in}{1.291905in}}%
\pgfpathlineto{\pgfqpoint{2.193211in}{1.327284in}}%
\pgfpathlineto{\pgfqpoint{2.234334in}{1.362664in}}%
\pgfpathlineto{\pgfqpoint{2.276010in}{1.398044in}}%
\pgfpathlineto{\pgfqpoint{2.318269in}{1.433423in}}%
\pgfpathlineto{\pgfqpoint{2.361142in}{1.468803in}}%
\pgfpathlineto{\pgfqpoint{2.404658in}{1.504182in}}%
\pgfpathlineto{\pgfqpoint{2.448846in}{1.539562in}}%
\pgfpathlineto{\pgfqpoint{2.493735in}{1.574941in}}%
\pgfpathlineto{\pgfqpoint{2.539353in}{1.610321in}}%
\pgfpathlineto{\pgfqpoint{2.585726in}{1.645701in}}%
\pgfpathlineto{\pgfqpoint{2.632882in}{1.681080in}}%
\pgfpathlineto{\pgfqpoint{2.680846in}{1.716460in}}%
\pgfpathlineto{\pgfqpoint{2.729644in}{1.751839in}}%
\pgfpathlineto{\pgfqpoint{2.779300in}{1.787219in}}%
\pgfpathlineto{\pgfqpoint{2.829838in}{1.822598in}}%
\pgfpathlineto{\pgfqpoint{2.881279in}{1.857978in}}%
\pgfpathlineto{\pgfqpoint{2.933648in}{1.893358in}}%
\pgfpathlineto{\pgfqpoint{2.986964in}{1.928737in}}%
\pgfpathlineto{\pgfqpoint{3.041247in}{1.964117in}}%
\pgfpathlineto{\pgfqpoint{3.096518in}{1.999496in}}%
\pgfpathlineto{\pgfqpoint{3.152795in}{2.034876in}}%
\pgfpathlineto{\pgfqpoint{3.210096in}{2.070255in}}%
\pgfpathlineto{\pgfqpoint{3.268436in}{2.105635in}}%
\pgfpathlineto{\pgfqpoint{3.327832in}{2.141015in}}%
\pgfpathlineto{\pgfqpoint{3.388297in}{2.176394in}}%
\pgfpathlineto{\pgfqpoint{3.449847in}{2.211774in}}%
\pgfpathlineto{\pgfqpoint{3.512492in}{2.247153in}}%
\pgfpathlineto{\pgfqpoint{3.576245in}{2.282533in}}%
\pgfpathlineto{\pgfqpoint{3.641116in}{2.317912in}}%
\pgfpathlineto{\pgfqpoint{3.707113in}{2.353292in}}%
\pgfpathlineto{\pgfqpoint{3.774247in}{2.388672in}}%
\pgfpathlineto{\pgfqpoint{3.842522in}{2.424051in}}%
\pgfpathlineto{\pgfqpoint{3.911945in}{2.459431in}}%
\pgfpathlineto{\pgfqpoint{3.982521in}{2.494810in}}%
\pgfpathlineto{\pgfqpoint{4.054253in}{2.530190in}}%
\pgfpathlineto{\pgfqpoint{4.127143in}{2.565569in}}%
\pgfpathlineto{\pgfqpoint{4.201193in}{2.600949in}}%
\pgfpathlineto{\pgfqpoint{4.276403in}{2.636328in}}%
\pgfpathlineto{\pgfqpoint{4.352771in}{2.671708in}}%
\pgfpathlineto{\pgfqpoint{4.430294in}{2.707088in}}%
\pgfpathlineto{\pgfqpoint{4.508970in}{2.742467in}}%
\pgfpathlineto{\pgfqpoint{4.588792in}{2.777847in}}%
\pgfpathlineto{\pgfqpoint{4.669756in}{2.813226in}}%
\pgfpathlineto{\pgfqpoint{4.751853in}{2.848606in}}%
\pgfpathlineto{\pgfqpoint{4.835074in}{2.883985in}}%
\pgfpathlineto{\pgfqpoint{4.919411in}{2.919365in}}%
\pgfpathlineto{\pgfqpoint{5.004852in}{2.954745in}}%
\pgfpathlineto{\pgfqpoint{5.091384in}{2.990124in}}%
\pgfpathlineto{\pgfqpoint{5.178995in}{3.025504in}}%
\pgfpathlineto{\pgfqpoint{5.267669in}{3.060883in}}%
\pgfpathlineto{\pgfqpoint{5.357391in}{3.096263in}}%
\pgfpathlineto{\pgfqpoint{5.448144in}{3.131642in}}%
\pgfpathlineto{\pgfqpoint{5.539909in}{3.167022in}}%
\pgfpathlineto{\pgfqpoint{5.632668in}{3.202402in}}%
\pgfpathlineto{\pgfqpoint{5.726399in}{3.237781in}}%
\pgfpathlineto{\pgfqpoint{5.821082in}{3.273161in}}%
\pgfpathlineto{\pgfqpoint{5.916693in}{3.308540in}}%
\pgfpathlineto{\pgfqpoint{6.013210in}{3.343920in}}%
\pgfpathlineto{\pgfqpoint{6.110607in}{3.379299in}}%
\pgfpathlineto{\pgfqpoint{6.208858in}{3.414679in}}%
\pgfpathlineto{\pgfqpoint{6.307938in}{3.450059in}}%
\pgfpathlineto{\pgfqpoint{6.407817in}{3.485438in}}%
\pgfpathlineto{\pgfqpoint{6.508468in}{3.520818in}}%
\pgfpathlineto{\pgfqpoint{6.609861in}{3.556197in}}%
\pgfpathlineto{\pgfqpoint{6.711966in}{3.591577in}}%
\pgfpathlineto{\pgfqpoint{6.814751in}{3.626956in}}%
\pgfpathlineto{\pgfqpoint{6.918185in}{3.662336in}}%
\pgfpathlineto{\pgfqpoint{7.022234in}{3.697716in}}%
\pgfpathlineto{\pgfqpoint{7.126865in}{3.733095in}}%
\pgfpathlineto{\pgfqpoint{7.232044in}{3.768475in}}%
\pgfpathlineto{\pgfqpoint{7.337736in}{3.803854in}}%
\pgfpathlineto{\pgfqpoint{7.443905in}{3.839234in}}%
\pgfpathlineto{\pgfqpoint{7.550516in}{3.874613in}}%
\pgfpathlineto{\pgfqpoint{7.657531in}{3.909993in}}%
\pgfpathlineto{\pgfqpoint{7.764915in}{3.945373in}}%
\pgfpathlineto{\pgfqpoint{7.872628in}{3.980752in}}%
\pgfpathlineto{\pgfqpoint{7.980634in}{4.016132in}}%
\pgfpathlineto{\pgfqpoint{8.088894in}{4.051511in}}%
\pgfpathlineto{\pgfqpoint{8.197370in}{4.086891in}}%
\pgfpathlineto{\pgfqpoint{8.306022in}{4.122270in}}%
\pgfpathlineto{\pgfqpoint{8.414812in}{4.157650in}}%
\pgfpathlineto{\pgfqpoint{8.523700in}{4.193030in}}%
\pgfpathlineto{\pgfqpoint{8.632648in}{4.228409in}}%
\pgfusepath{stroke}%
\end{pgfscope}%
\begin{pgfscope}%
\pgfpathrectangle{\pgfqpoint{1.250000in}{0.625000in}}{\pgfqpoint{7.750000in}{3.775000in}}%
\pgfusepath{clip}%
\pgfsetrectcap%
\pgfsetroundjoin%
\pgfsetlinewidth{1.505625pt}%
\definecolor{currentstroke}{rgb}{0.549020,0.337255,0.294118}%
\pgfsetstrokecolor{currentstroke}%
\pgfsetdash{}{0pt}%
\pgfpathmoveto{\pgfqpoint{1.625673in}{0.796591in}}%
\pgfpathlineto{\pgfqpoint{1.672487in}{0.831970in}}%
\pgfpathlineto{\pgfqpoint{1.719340in}{0.867350in}}%
\pgfpathlineto{\pgfqpoint{1.766257in}{0.902730in}}%
\pgfpathlineto{\pgfqpoint{1.813264in}{0.938109in}}%
\pgfpathlineto{\pgfqpoint{1.860386in}{0.973489in}}%
\pgfpathlineto{\pgfqpoint{1.907649in}{1.008868in}}%
\pgfpathlineto{\pgfqpoint{1.955079in}{1.044248in}}%
\pgfpathlineto{\pgfqpoint{2.002700in}{1.079627in}}%
\pgfpathlineto{\pgfqpoint{2.050538in}{1.115007in}}%
\pgfpathlineto{\pgfqpoint{2.098617in}{1.150387in}}%
\pgfpathlineto{\pgfqpoint{2.146961in}{1.185766in}}%
\pgfpathlineto{\pgfqpoint{2.195595in}{1.221146in}}%
\pgfpathlineto{\pgfqpoint{2.244544in}{1.256525in}}%
\pgfpathlineto{\pgfqpoint{2.293830in}{1.291905in}}%
\pgfpathlineto{\pgfqpoint{2.343477in}{1.327284in}}%
\pgfpathlineto{\pgfqpoint{2.393508in}{1.362664in}}%
\pgfpathlineto{\pgfqpoint{2.443946in}{1.398044in}}%
\pgfpathlineto{\pgfqpoint{2.494813in}{1.433423in}}%
\pgfpathlineto{\pgfqpoint{2.546131in}{1.468803in}}%
\pgfpathlineto{\pgfqpoint{2.597921in}{1.504182in}}%
\pgfpathlineto{\pgfqpoint{2.650204in}{1.539562in}}%
\pgfpathlineto{\pgfqpoint{2.703000in}{1.574941in}}%
\pgfpathlineto{\pgfqpoint{2.756330in}{1.610321in}}%
\pgfpathlineto{\pgfqpoint{2.810214in}{1.645701in}}%
\pgfpathlineto{\pgfqpoint{2.864669in}{1.681080in}}%
\pgfpathlineto{\pgfqpoint{2.919714in}{1.716460in}}%
\pgfpathlineto{\pgfqpoint{2.975368in}{1.751839in}}%
\pgfpathlineto{\pgfqpoint{3.031647in}{1.787219in}}%
\pgfpathlineto{\pgfqpoint{3.088567in}{1.822598in}}%
\pgfpathlineto{\pgfqpoint{3.146145in}{1.857978in}}%
\pgfpathlineto{\pgfqpoint{3.204397in}{1.893358in}}%
\pgfpathlineto{\pgfqpoint{3.263335in}{1.928737in}}%
\pgfpathlineto{\pgfqpoint{3.322976in}{1.964117in}}%
\pgfpathlineto{\pgfqpoint{3.383331in}{1.999496in}}%
\pgfpathlineto{\pgfqpoint{3.444413in}{2.034876in}}%
\pgfpathlineto{\pgfqpoint{3.506234in}{2.070255in}}%
\pgfpathlineto{\pgfqpoint{3.568806in}{2.105635in}}%
\pgfpathlineto{\pgfqpoint{3.632137in}{2.141015in}}%
\pgfpathlineto{\pgfqpoint{3.696239in}{2.176394in}}%
\pgfpathlineto{\pgfqpoint{3.761120in}{2.211774in}}%
\pgfpathlineto{\pgfqpoint{3.826787in}{2.247153in}}%
\pgfpathlineto{\pgfqpoint{3.893248in}{2.282533in}}%
\pgfpathlineto{\pgfqpoint{3.960510in}{2.317912in}}%
\pgfpathlineto{\pgfqpoint{4.028577in}{2.353292in}}%
\pgfpathlineto{\pgfqpoint{4.097455in}{2.388672in}}%
\pgfpathlineto{\pgfqpoint{4.167147in}{2.424051in}}%
\pgfpathlineto{\pgfqpoint{4.237658in}{2.459431in}}%
\pgfpathlineto{\pgfqpoint{4.308988in}{2.494810in}}%
\pgfpathlineto{\pgfqpoint{4.381139in}{2.530190in}}%
\pgfpathlineto{\pgfqpoint{4.454113in}{2.565569in}}%
\pgfpathlineto{\pgfqpoint{4.527908in}{2.600949in}}%
\pgfpathlineto{\pgfqpoint{4.602524in}{2.636328in}}%
\pgfpathlineto{\pgfqpoint{4.677959in}{2.671708in}}%
\pgfpathlineto{\pgfqpoint{4.754209in}{2.707088in}}%
\pgfpathlineto{\pgfqpoint{4.831272in}{2.742467in}}%
\pgfpathlineto{\pgfqpoint{4.909142in}{2.777847in}}%
\pgfpathlineto{\pgfqpoint{4.987815in}{2.813226in}}%
\pgfpathlineto{\pgfqpoint{5.067284in}{2.848606in}}%
\pgfpathlineto{\pgfqpoint{5.147542in}{2.883985in}}%
\pgfpathlineto{\pgfqpoint{5.228581in}{2.919365in}}%
\pgfpathlineto{\pgfqpoint{5.310394in}{2.954745in}}%
\pgfpathlineto{\pgfqpoint{5.392969in}{2.990124in}}%
\pgfpathlineto{\pgfqpoint{5.476298in}{3.025504in}}%
\pgfpathlineto{\pgfqpoint{5.560368in}{3.060883in}}%
\pgfpathlineto{\pgfqpoint{5.645169in}{3.096263in}}%
\pgfpathlineto{\pgfqpoint{5.730687in}{3.131642in}}%
\pgfpathlineto{\pgfqpoint{5.816909in}{3.167022in}}%
\pgfpathlineto{\pgfqpoint{5.903821in}{3.202402in}}%
\pgfpathlineto{\pgfqpoint{5.991408in}{3.237781in}}%
\pgfpathlineto{\pgfqpoint{6.079654in}{3.273161in}}%
\pgfpathlineto{\pgfqpoint{6.168544in}{3.308540in}}%
\pgfpathlineto{\pgfqpoint{6.258060in}{3.343920in}}%
\pgfpathlineto{\pgfqpoint{6.348185in}{3.379299in}}%
\pgfpathlineto{\pgfqpoint{6.438900in}{3.414679in}}%
\pgfpathlineto{\pgfqpoint{6.530186in}{3.450059in}}%
\pgfpathlineto{\pgfqpoint{6.622024in}{3.485438in}}%
\pgfpathlineto{\pgfqpoint{6.714393in}{3.520818in}}%
\pgfpathlineto{\pgfqpoint{6.807274in}{3.556197in}}%
\pgfpathlineto{\pgfqpoint{6.900644in}{3.591577in}}%
\pgfpathlineto{\pgfqpoint{6.994481in}{3.626956in}}%
\pgfpathlineto{\pgfqpoint{7.088765in}{3.662336in}}%
\pgfpathlineto{\pgfqpoint{7.183470in}{3.697716in}}%
\pgfpathlineto{\pgfqpoint{7.278576in}{3.733095in}}%
\pgfpathlineto{\pgfqpoint{7.374057in}{3.768475in}}%
\pgfpathlineto{\pgfqpoint{7.469889in}{3.803854in}}%
\pgfpathlineto{\pgfqpoint{7.566049in}{3.839234in}}%
\pgfpathlineto{\pgfqpoint{7.662511in}{3.874613in}}%
\pgfpathlineto{\pgfqpoint{7.759251in}{3.909993in}}%
\pgfpathlineto{\pgfqpoint{7.856242in}{3.945373in}}%
\pgfpathlineto{\pgfqpoint{7.953458in}{3.980752in}}%
\pgfpathlineto{\pgfqpoint{8.050875in}{4.016132in}}%
\pgfpathlineto{\pgfqpoint{8.148466in}{4.051511in}}%
\pgfpathlineto{\pgfqpoint{8.246203in}{4.086891in}}%
\pgfpathlineto{\pgfqpoint{8.344062in}{4.122270in}}%
\pgfpathlineto{\pgfqpoint{8.442014in}{4.157650in}}%
\pgfpathlineto{\pgfqpoint{8.540034in}{4.193030in}}%
\pgfpathlineto{\pgfqpoint{8.638094in}{4.228409in}}%
\pgfusepath{stroke}%
\end{pgfscope}%
\begin{pgfscope}%
\pgfpathrectangle{\pgfqpoint{1.250000in}{0.625000in}}{\pgfqpoint{7.750000in}{3.775000in}}%
\pgfusepath{clip}%
\pgfsetrectcap%
\pgfsetroundjoin%
\pgfsetlinewidth{1.505625pt}%
\definecolor{currentstroke}{rgb}{0.890196,0.466667,0.760784}%
\pgfsetstrokecolor{currentstroke}%
\pgfsetdash{}{0pt}%
\pgfpathmoveto{\pgfqpoint{1.629338in}{0.796591in}}%
\pgfpathlineto{\pgfqpoint{1.683477in}{0.831970in}}%
\pgfpathlineto{\pgfqpoint{1.737644in}{0.867350in}}%
\pgfpathlineto{\pgfqpoint{1.791858in}{0.902730in}}%
\pgfpathlineto{\pgfqpoint{1.846136in}{0.938109in}}%
\pgfpathlineto{\pgfqpoint{1.900498in}{0.973489in}}%
\pgfpathlineto{\pgfqpoint{1.954961in}{1.008868in}}%
\pgfpathlineto{\pgfqpoint{2.009544in}{1.044248in}}%
\pgfpathlineto{\pgfqpoint{2.064265in}{1.079627in}}%
\pgfpathlineto{\pgfqpoint{2.119141in}{1.115007in}}%
\pgfpathlineto{\pgfqpoint{2.174191in}{1.150387in}}%
\pgfpathlineto{\pgfqpoint{2.229433in}{1.185766in}}%
\pgfpathlineto{\pgfqpoint{2.284883in}{1.221146in}}%
\pgfpathlineto{\pgfqpoint{2.340559in}{1.256525in}}%
\pgfpathlineto{\pgfqpoint{2.396478in}{1.291905in}}%
\pgfpathlineto{\pgfqpoint{2.452656in}{1.327284in}}%
\pgfpathlineto{\pgfqpoint{2.509111in}{1.362664in}}%
\pgfpathlineto{\pgfqpoint{2.565858in}{1.398044in}}%
\pgfpathlineto{\pgfqpoint{2.622914in}{1.433423in}}%
\pgfpathlineto{\pgfqpoint{2.680293in}{1.468803in}}%
\pgfpathlineto{\pgfqpoint{2.738012in}{1.504182in}}%
\pgfpathlineto{\pgfqpoint{2.796084in}{1.539562in}}%
\pgfpathlineto{\pgfqpoint{2.854526in}{1.574941in}}%
\pgfpathlineto{\pgfqpoint{2.913350in}{1.610321in}}%
\pgfpathlineto{\pgfqpoint{2.972572in}{1.645701in}}%
\pgfpathlineto{\pgfqpoint{3.032203in}{1.681080in}}%
\pgfpathlineto{\pgfqpoint{3.092258in}{1.716460in}}%
\pgfpathlineto{\pgfqpoint{3.152748in}{1.751839in}}%
\pgfpathlineto{\pgfqpoint{3.213687in}{1.787219in}}%
\pgfpathlineto{\pgfqpoint{3.275085in}{1.822598in}}%
\pgfpathlineto{\pgfqpoint{3.336954in}{1.857978in}}%
\pgfpathlineto{\pgfqpoint{3.399305in}{1.893358in}}%
\pgfpathlineto{\pgfqpoint{3.462147in}{1.928737in}}%
\pgfpathlineto{\pgfqpoint{3.525492in}{1.964117in}}%
\pgfpathlineto{\pgfqpoint{3.589347in}{1.999496in}}%
\pgfpathlineto{\pgfqpoint{3.653722in}{2.034876in}}%
\pgfpathlineto{\pgfqpoint{3.718624in}{2.070255in}}%
\pgfpathlineto{\pgfqpoint{3.784063in}{2.105635in}}%
\pgfpathlineto{\pgfqpoint{3.850044in}{2.141015in}}%
\pgfpathlineto{\pgfqpoint{3.916574in}{2.176394in}}%
\pgfpathlineto{\pgfqpoint{3.983659in}{2.211774in}}%
\pgfpathlineto{\pgfqpoint{4.051306in}{2.247153in}}%
\pgfpathlineto{\pgfqpoint{4.119517in}{2.282533in}}%
\pgfpathlineto{\pgfqpoint{4.188299in}{2.317912in}}%
\pgfpathlineto{\pgfqpoint{4.257654in}{2.353292in}}%
\pgfpathlineto{\pgfqpoint{4.327586in}{2.388672in}}%
\pgfpathlineto{\pgfqpoint{4.398097in}{2.424051in}}%
\pgfpathlineto{\pgfqpoint{4.469190in}{2.459431in}}%
\pgfpathlineto{\pgfqpoint{4.540864in}{2.494810in}}%
\pgfpathlineto{\pgfqpoint{4.613123in}{2.530190in}}%
\pgfpathlineto{\pgfqpoint{4.685964in}{2.565569in}}%
\pgfpathlineto{\pgfqpoint{4.759389in}{2.600949in}}%
\pgfpathlineto{\pgfqpoint{4.833395in}{2.636328in}}%
\pgfpathlineto{\pgfqpoint{4.907982in}{2.671708in}}%
\pgfpathlineto{\pgfqpoint{4.983147in}{2.707088in}}%
\pgfpathlineto{\pgfqpoint{5.058887in}{2.742467in}}%
\pgfpathlineto{\pgfqpoint{5.135199in}{2.777847in}}%
\pgfpathlineto{\pgfqpoint{5.212078in}{2.813226in}}%
\pgfpathlineto{\pgfqpoint{5.289521in}{2.848606in}}%
\pgfpathlineto{\pgfqpoint{5.367522in}{2.883985in}}%
\pgfpathlineto{\pgfqpoint{5.446074in}{2.919365in}}%
\pgfpathlineto{\pgfqpoint{5.525173in}{2.954745in}}%
\pgfpathlineto{\pgfqpoint{5.604810in}{2.990124in}}%
\pgfpathlineto{\pgfqpoint{5.684979in}{3.025504in}}%
\pgfpathlineto{\pgfqpoint{5.765671in}{3.060883in}}%
\pgfpathlineto{\pgfqpoint{5.846878in}{3.096263in}}%
\pgfpathlineto{\pgfqpoint{5.928591in}{3.131642in}}%
\pgfpathlineto{\pgfqpoint{6.010800in}{3.167022in}}%
\pgfpathlineto{\pgfqpoint{6.093495in}{3.202402in}}%
\pgfpathlineto{\pgfqpoint{6.176665in}{3.237781in}}%
\pgfpathlineto{\pgfqpoint{6.260299in}{3.273161in}}%
\pgfpathlineto{\pgfqpoint{6.344386in}{3.308540in}}%
\pgfpathlineto{\pgfqpoint{6.428913in}{3.343920in}}%
\pgfpathlineto{\pgfqpoint{6.513868in}{3.379299in}}%
\pgfpathlineto{\pgfqpoint{6.599238in}{3.414679in}}%
\pgfpathlineto{\pgfqpoint{6.685010in}{3.450059in}}%
\pgfpathlineto{\pgfqpoint{6.771169in}{3.485438in}}%
\pgfpathlineto{\pgfqpoint{6.857701in}{3.520818in}}%
\pgfpathlineto{\pgfqpoint{6.944592in}{3.556197in}}%
\pgfpathlineto{\pgfqpoint{7.031827in}{3.591577in}}%
\pgfpathlineto{\pgfqpoint{7.119390in}{3.626956in}}%
\pgfpathlineto{\pgfqpoint{7.207265in}{3.662336in}}%
\pgfpathlineto{\pgfqpoint{7.295437in}{3.697716in}}%
\pgfpathlineto{\pgfqpoint{7.383889in}{3.733095in}}%
\pgfpathlineto{\pgfqpoint{7.472604in}{3.768475in}}%
\pgfpathlineto{\pgfqpoint{7.561566in}{3.803854in}}%
\pgfpathlineto{\pgfqpoint{7.650757in}{3.839234in}}%
\pgfpathlineto{\pgfqpoint{7.740160in}{3.874613in}}%
\pgfpathlineto{\pgfqpoint{7.829757in}{3.909993in}}%
\pgfpathlineto{\pgfqpoint{7.919530in}{3.945373in}}%
\pgfpathlineto{\pgfqpoint{8.009462in}{3.980752in}}%
\pgfpathlineto{\pgfqpoint{8.099533in}{4.016132in}}%
\pgfpathlineto{\pgfqpoint{8.189726in}{4.051511in}}%
\pgfpathlineto{\pgfqpoint{8.280022in}{4.086891in}}%
\pgfpathlineto{\pgfqpoint{8.370403in}{4.122270in}}%
\pgfpathlineto{\pgfqpoint{8.460850in}{4.157650in}}%
\pgfpathlineto{\pgfqpoint{8.551344in}{4.193030in}}%
\pgfpathlineto{\pgfqpoint{8.641866in}{4.228409in}}%
\pgfusepath{stroke}%
\end{pgfscope}%
\begin{pgfscope}%
\pgfpathrectangle{\pgfqpoint{1.250000in}{0.625000in}}{\pgfqpoint{7.750000in}{3.775000in}}%
\pgfusepath{clip}%
\pgfsetrectcap%
\pgfsetroundjoin%
\pgfsetlinewidth{1.505625pt}%
\definecolor{currentstroke}{rgb}{0.498039,0.498039,0.498039}%
\pgfsetstrokecolor{currentstroke}%
\pgfsetdash{}{0pt}%
\pgfpathmoveto{\pgfqpoint{1.631964in}{0.796591in}}%
\pgfpathlineto{\pgfqpoint{1.691352in}{0.831970in}}%
\pgfpathlineto{\pgfqpoint{1.750761in}{0.867350in}}%
\pgfpathlineto{\pgfqpoint{1.810203in}{0.902730in}}%
\pgfpathlineto{\pgfqpoint{1.869690in}{0.938109in}}%
\pgfpathlineto{\pgfqpoint{1.929238in}{0.973489in}}%
\pgfpathlineto{\pgfqpoint{1.988857in}{1.008868in}}%
\pgfpathlineto{\pgfqpoint{2.048562in}{1.044248in}}%
\pgfpathlineto{\pgfqpoint{2.108366in}{1.079627in}}%
\pgfpathlineto{\pgfqpoint{2.168280in}{1.115007in}}%
\pgfpathlineto{\pgfqpoint{2.228319in}{1.150387in}}%
\pgfpathlineto{\pgfqpoint{2.288494in}{1.185766in}}%
\pgfpathlineto{\pgfqpoint{2.348818in}{1.221146in}}%
\pgfpathlineto{\pgfqpoint{2.409303in}{1.256525in}}%
\pgfpathlineto{\pgfqpoint{2.469961in}{1.291905in}}%
\pgfpathlineto{\pgfqpoint{2.530804in}{1.327284in}}%
\pgfpathlineto{\pgfqpoint{2.591845in}{1.362664in}}%
\pgfpathlineto{\pgfqpoint{2.653094in}{1.398044in}}%
\pgfpathlineto{\pgfqpoint{2.714562in}{1.433423in}}%
\pgfpathlineto{\pgfqpoint{2.776262in}{1.468803in}}%
\pgfpathlineto{\pgfqpoint{2.838203in}{1.504182in}}%
\pgfpathlineto{\pgfqpoint{2.900397in}{1.539562in}}%
\pgfpathlineto{\pgfqpoint{2.962854in}{1.574941in}}%
\pgfpathlineto{\pgfqpoint{3.025584in}{1.610321in}}%
\pgfpathlineto{\pgfqpoint{3.088597in}{1.645701in}}%
\pgfpathlineto{\pgfqpoint{3.151901in}{1.681080in}}%
\pgfpathlineto{\pgfqpoint{3.215508in}{1.716460in}}%
\pgfpathlineto{\pgfqpoint{3.279424in}{1.751839in}}%
\pgfpathlineto{\pgfqpoint{3.343660in}{1.787219in}}%
\pgfpathlineto{\pgfqpoint{3.408224in}{1.822598in}}%
\pgfpathlineto{\pgfqpoint{3.473122in}{1.857978in}}%
\pgfpathlineto{\pgfqpoint{3.538364in}{1.893358in}}%
\pgfpathlineto{\pgfqpoint{3.603956in}{1.928737in}}%
\pgfpathlineto{\pgfqpoint{3.669904in}{1.964117in}}%
\pgfpathlineto{\pgfqpoint{3.736217in}{1.999496in}}%
\pgfpathlineto{\pgfqpoint{3.802899in}{2.034876in}}%
\pgfpathlineto{\pgfqpoint{3.869957in}{2.070255in}}%
\pgfpathlineto{\pgfqpoint{3.937396in}{2.105635in}}%
\pgfpathlineto{\pgfqpoint{4.005220in}{2.141015in}}%
\pgfpathlineto{\pgfqpoint{4.073436in}{2.176394in}}%
\pgfpathlineto{\pgfqpoint{4.142046in}{2.211774in}}%
\pgfpathlineto{\pgfqpoint{4.211055in}{2.247153in}}%
\pgfpathlineto{\pgfqpoint{4.280466in}{2.282533in}}%
\pgfpathlineto{\pgfqpoint{4.350281in}{2.317912in}}%
\pgfpathlineto{\pgfqpoint{4.420505in}{2.353292in}}%
\pgfpathlineto{\pgfqpoint{4.491138in}{2.388672in}}%
\pgfpathlineto{\pgfqpoint{4.562183in}{2.424051in}}%
\pgfpathlineto{\pgfqpoint{4.633640in}{2.459431in}}%
\pgfpathlineto{\pgfqpoint{4.705511in}{2.494810in}}%
\pgfpathlineto{\pgfqpoint{4.777796in}{2.530190in}}%
\pgfpathlineto{\pgfqpoint{4.850496in}{2.565569in}}%
\pgfpathlineto{\pgfqpoint{4.923609in}{2.600949in}}%
\pgfpathlineto{\pgfqpoint{4.997135in}{2.636328in}}%
\pgfpathlineto{\pgfqpoint{5.071073in}{2.671708in}}%
\pgfpathlineto{\pgfqpoint{5.145421in}{2.707088in}}%
\pgfpathlineto{\pgfqpoint{5.220177in}{2.742467in}}%
\pgfpathlineto{\pgfqpoint{5.295339in}{2.777847in}}%
\pgfpathlineto{\pgfqpoint{5.370903in}{2.813226in}}%
\pgfpathlineto{\pgfqpoint{5.446866in}{2.848606in}}%
\pgfpathlineto{\pgfqpoint{5.523225in}{2.883985in}}%
\pgfpathlineto{\pgfqpoint{5.599975in}{2.919365in}}%
\pgfpathlineto{\pgfqpoint{5.677112in}{2.954745in}}%
\pgfpathlineto{\pgfqpoint{5.754631in}{2.990124in}}%
\pgfpathlineto{\pgfqpoint{5.832526in}{3.025504in}}%
\pgfpathlineto{\pgfqpoint{5.910792in}{3.060883in}}%
\pgfpathlineto{\pgfqpoint{5.989422in}{3.096263in}}%
\pgfpathlineto{\pgfqpoint{6.068411in}{3.131642in}}%
\pgfpathlineto{\pgfqpoint{6.147751in}{3.167022in}}%
\pgfpathlineto{\pgfqpoint{6.227435in}{3.202402in}}%
\pgfpathlineto{\pgfqpoint{6.307456in}{3.237781in}}%
\pgfpathlineto{\pgfqpoint{6.387804in}{3.273161in}}%
\pgfpathlineto{\pgfqpoint{6.468474in}{3.308540in}}%
\pgfpathlineto{\pgfqpoint{6.549455in}{3.343920in}}%
\pgfpathlineto{\pgfqpoint{6.630738in}{3.379299in}}%
\pgfpathlineto{\pgfqpoint{6.712315in}{3.414679in}}%
\pgfpathlineto{\pgfqpoint{6.794176in}{3.450059in}}%
\pgfpathlineto{\pgfqpoint{6.876311in}{3.485438in}}%
\pgfpathlineto{\pgfqpoint{6.958710in}{3.520818in}}%
\pgfpathlineto{\pgfqpoint{7.041363in}{3.556197in}}%
\pgfpathlineto{\pgfqpoint{7.124259in}{3.591577in}}%
\pgfpathlineto{\pgfqpoint{7.207386in}{3.626956in}}%
\pgfpathlineto{\pgfqpoint{7.290735in}{3.662336in}}%
\pgfpathlineto{\pgfqpoint{7.374293in}{3.697716in}}%
\pgfpathlineto{\pgfqpoint{7.458049in}{3.733095in}}%
\pgfpathlineto{\pgfqpoint{7.541991in}{3.768475in}}%
\pgfpathlineto{\pgfqpoint{7.626108in}{3.803854in}}%
\pgfpathlineto{\pgfqpoint{7.710386in}{3.839234in}}%
\pgfpathlineto{\pgfqpoint{7.794814in}{3.874613in}}%
\pgfpathlineto{\pgfqpoint{7.879379in}{3.909993in}}%
\pgfpathlineto{\pgfqpoint{7.964069in}{3.945373in}}%
\pgfpathlineto{\pgfqpoint{8.048871in}{3.980752in}}%
\pgfpathlineto{\pgfqpoint{8.133771in}{4.016132in}}%
\pgfpathlineto{\pgfqpoint{8.218758in}{4.051511in}}%
\pgfpathlineto{\pgfqpoint{8.303817in}{4.086891in}}%
\pgfpathlineto{\pgfqpoint{8.388936in}{4.122270in}}%
\pgfpathlineto{\pgfqpoint{8.474101in}{4.157650in}}%
\pgfpathlineto{\pgfqpoint{8.559300in}{4.193030in}}%
\pgfpathlineto{\pgfqpoint{8.644519in}{4.228409in}}%
\pgfusepath{stroke}%
\end{pgfscope}%
\begin{pgfscope}%
\pgfpathrectangle{\pgfqpoint{1.250000in}{0.625000in}}{\pgfqpoint{7.750000in}{3.775000in}}%
\pgfusepath{clip}%
\pgfsetrectcap%
\pgfsetroundjoin%
\pgfsetlinewidth{1.505625pt}%
\definecolor{currentstroke}{rgb}{0.737255,0.741176,0.133333}%
\pgfsetstrokecolor{currentstroke}%
\pgfsetdash{}{0pt}%
\pgfpathmoveto{\pgfqpoint{1.633834in}{0.796591in}}%
\pgfpathlineto{\pgfqpoint{1.696962in}{0.831970in}}%
\pgfpathlineto{\pgfqpoint{1.760104in}{0.867350in}}%
\pgfpathlineto{\pgfqpoint{1.823269in}{0.902730in}}%
\pgfpathlineto{\pgfqpoint{1.886467in}{0.938109in}}%
\pgfpathlineto{\pgfqpoint{1.949708in}{0.973489in}}%
\pgfpathlineto{\pgfqpoint{2.013000in}{1.008868in}}%
\pgfpathlineto{\pgfqpoint{2.076353in}{1.044248in}}%
\pgfpathlineto{\pgfqpoint{2.139775in}{1.079627in}}%
\pgfpathlineto{\pgfqpoint{2.203277in}{1.115007in}}%
\pgfpathlineto{\pgfqpoint{2.266867in}{1.150387in}}%
\pgfpathlineto{\pgfqpoint{2.330554in}{1.185766in}}%
\pgfpathlineto{\pgfqpoint{2.394347in}{1.221146in}}%
\pgfpathlineto{\pgfqpoint{2.458254in}{1.256525in}}%
\pgfpathlineto{\pgfqpoint{2.522285in}{1.291905in}}%
\pgfpathlineto{\pgfqpoint{2.586447in}{1.327284in}}%
\pgfpathlineto{\pgfqpoint{2.650750in}{1.362664in}}%
\pgfpathlineto{\pgfqpoint{2.715200in}{1.398044in}}%
\pgfpathlineto{\pgfqpoint{2.779807in}{1.433423in}}%
\pgfpathlineto{\pgfqpoint{2.844578in}{1.468803in}}%
\pgfpathlineto{\pgfqpoint{2.909522in}{1.504182in}}%
\pgfpathlineto{\pgfqpoint{2.974644in}{1.539562in}}%
\pgfpathlineto{\pgfqpoint{3.039954in}{1.574941in}}%
\pgfpathlineto{\pgfqpoint{3.105457in}{1.610321in}}%
\pgfpathlineto{\pgfqpoint{3.171162in}{1.645701in}}%
\pgfpathlineto{\pgfqpoint{3.237074in}{1.681080in}}%
\pgfpathlineto{\pgfqpoint{3.303201in}{1.716460in}}%
\pgfpathlineto{\pgfqpoint{3.369548in}{1.751839in}}%
\pgfpathlineto{\pgfqpoint{3.436122in}{1.787219in}}%
\pgfpathlineto{\pgfqpoint{3.502929in}{1.822598in}}%
\pgfpathlineto{\pgfqpoint{3.569974in}{1.857978in}}%
\pgfpathlineto{\pgfqpoint{3.637263in}{1.893358in}}%
\pgfpathlineto{\pgfqpoint{3.704801in}{1.928737in}}%
\pgfpathlineto{\pgfqpoint{3.772593in}{1.964117in}}%
\pgfpathlineto{\pgfqpoint{3.840642in}{1.999496in}}%
\pgfpathlineto{\pgfqpoint{3.908955in}{2.034876in}}%
\pgfpathlineto{\pgfqpoint{3.977535in}{2.070255in}}%
\pgfpathlineto{\pgfqpoint{4.046385in}{2.105635in}}%
\pgfpathlineto{\pgfqpoint{4.115509in}{2.141015in}}%
\pgfpathlineto{\pgfqpoint{4.184911in}{2.176394in}}%
\pgfpathlineto{\pgfqpoint{4.254593in}{2.211774in}}%
\pgfpathlineto{\pgfqpoint{4.324559in}{2.247153in}}%
\pgfpathlineto{\pgfqpoint{4.394810in}{2.282533in}}%
\pgfpathlineto{\pgfqpoint{4.465348in}{2.317912in}}%
\pgfpathlineto{\pgfqpoint{4.536176in}{2.353292in}}%
\pgfpathlineto{\pgfqpoint{4.607296in}{2.388672in}}%
\pgfpathlineto{\pgfqpoint{4.678707in}{2.424051in}}%
\pgfpathlineto{\pgfqpoint{4.750411in}{2.459431in}}%
\pgfpathlineto{\pgfqpoint{4.822409in}{2.494810in}}%
\pgfpathlineto{\pgfqpoint{4.894701in}{2.530190in}}%
\pgfpathlineto{\pgfqpoint{4.967288in}{2.565569in}}%
\pgfpathlineto{\pgfqpoint{5.040168in}{2.600949in}}%
\pgfpathlineto{\pgfqpoint{5.113341in}{2.636328in}}%
\pgfpathlineto{\pgfqpoint{5.186806in}{2.671708in}}%
\pgfpathlineto{\pgfqpoint{5.260562in}{2.707088in}}%
\pgfpathlineto{\pgfqpoint{5.334608in}{2.742467in}}%
\pgfpathlineto{\pgfqpoint{5.408942in}{2.777847in}}%
\pgfpathlineto{\pgfqpoint{5.483561in}{2.813226in}}%
\pgfpathlineto{\pgfqpoint{5.558464in}{2.848606in}}%
\pgfpathlineto{\pgfqpoint{5.633648in}{2.883985in}}%
\pgfpathlineto{\pgfqpoint{5.709109in}{2.919365in}}%
\pgfpathlineto{\pgfqpoint{5.784844in}{2.954745in}}%
\pgfpathlineto{\pgfqpoint{5.860850in}{2.990124in}}%
\pgfpathlineto{\pgfqpoint{5.937124in}{3.025504in}}%
\pgfpathlineto{\pgfqpoint{6.013660in}{3.060883in}}%
\pgfpathlineto{\pgfqpoint{6.090455in}{3.096263in}}%
\pgfpathlineto{\pgfqpoint{6.167504in}{3.131642in}}%
\pgfpathlineto{\pgfqpoint{6.244802in}{3.167022in}}%
\pgfpathlineto{\pgfqpoint{6.322344in}{3.202402in}}%
\pgfpathlineto{\pgfqpoint{6.400125in}{3.237781in}}%
\pgfpathlineto{\pgfqpoint{6.478139in}{3.273161in}}%
\pgfpathlineto{\pgfqpoint{6.556380in}{3.308540in}}%
\pgfpathlineto{\pgfqpoint{6.634841in}{3.343920in}}%
\pgfpathlineto{\pgfqpoint{6.713518in}{3.379299in}}%
\pgfpathlineto{\pgfqpoint{6.792402in}{3.414679in}}%
\pgfpathlineto{\pgfqpoint{6.871488in}{3.450059in}}%
\pgfpathlineto{\pgfqpoint{6.950769in}{3.485438in}}%
\pgfpathlineto{\pgfqpoint{7.030236in}{3.520818in}}%
\pgfpathlineto{\pgfqpoint{7.109883in}{3.556197in}}%
\pgfpathlineto{\pgfqpoint{7.189702in}{3.591577in}}%
\pgfpathlineto{\pgfqpoint{7.269686in}{3.626956in}}%
\pgfpathlineto{\pgfqpoint{7.349827in}{3.662336in}}%
\pgfpathlineto{\pgfqpoint{7.430116in}{3.697716in}}%
\pgfpathlineto{\pgfqpoint{7.510545in}{3.733095in}}%
\pgfpathlineto{\pgfqpoint{7.591106in}{3.768475in}}%
\pgfpathlineto{\pgfqpoint{7.671791in}{3.803854in}}%
\pgfpathlineto{\pgfqpoint{7.752591in}{3.839234in}}%
\pgfpathlineto{\pgfqpoint{7.833496in}{3.874613in}}%
\pgfpathlineto{\pgfqpoint{7.914499in}{3.909993in}}%
\pgfpathlineto{\pgfqpoint{7.995590in}{3.945373in}}%
\pgfpathlineto{\pgfqpoint{8.076760in}{3.980752in}}%
\pgfpathlineto{\pgfqpoint{8.158001in}{4.016132in}}%
\pgfpathlineto{\pgfqpoint{8.239302in}{4.051511in}}%
\pgfpathlineto{\pgfqpoint{8.320655in}{4.086891in}}%
\pgfpathlineto{\pgfqpoint{8.402050in}{4.122270in}}%
\pgfpathlineto{\pgfqpoint{8.483478in}{4.157650in}}%
\pgfpathlineto{\pgfqpoint{8.564930in}{4.193030in}}%
\pgfpathlineto{\pgfqpoint{8.646396in}{4.228409in}}%
\pgfusepath{stroke}%
\end{pgfscope}%
\begin{pgfscope}%
\pgfpathrectangle{\pgfqpoint{1.250000in}{0.625000in}}{\pgfqpoint{7.750000in}{3.775000in}}%
\pgfusepath{clip}%
\pgfsetrectcap%
\pgfsetroundjoin%
\pgfsetlinewidth{1.505625pt}%
\definecolor{currentstroke}{rgb}{0.090196,0.745098,0.811765}%
\pgfsetstrokecolor{currentstroke}%
\pgfsetdash{}{0pt}%
\pgfpathmoveto{\pgfqpoint{1.635164in}{0.796591in}}%
\pgfpathlineto{\pgfqpoint{1.700949in}{0.831970in}}%
\pgfpathlineto{\pgfqpoint{1.766745in}{0.867350in}}%
\pgfpathlineto{\pgfqpoint{1.832557in}{0.902730in}}%
\pgfpathlineto{\pgfqpoint{1.898392in}{0.938109in}}%
\pgfpathlineto{\pgfqpoint{1.964258in}{0.973489in}}%
\pgfpathlineto{\pgfqpoint{2.030160in}{1.008868in}}%
\pgfpathlineto{\pgfqpoint{2.096105in}{1.044248in}}%
\pgfpathlineto{\pgfqpoint{2.162100in}{1.079627in}}%
\pgfpathlineto{\pgfqpoint{2.228151in}{1.115007in}}%
\pgfpathlineto{\pgfqpoint{2.294265in}{1.150387in}}%
\pgfpathlineto{\pgfqpoint{2.360448in}{1.185766in}}%
\pgfpathlineto{\pgfqpoint{2.426706in}{1.221146in}}%
\pgfpathlineto{\pgfqpoint{2.493045in}{1.256525in}}%
\pgfpathlineto{\pgfqpoint{2.559472in}{1.291905in}}%
\pgfpathlineto{\pgfqpoint{2.625992in}{1.327284in}}%
\pgfpathlineto{\pgfqpoint{2.692612in}{1.362664in}}%
\pgfpathlineto{\pgfqpoint{2.759337in}{1.398044in}}%
\pgfpathlineto{\pgfqpoint{2.826174in}{1.433423in}}%
\pgfpathlineto{\pgfqpoint{2.893127in}{1.468803in}}%
\pgfpathlineto{\pgfqpoint{2.960202in}{1.504182in}}%
\pgfpathlineto{\pgfqpoint{3.027404in}{1.539562in}}%
\pgfpathlineto{\pgfqpoint{3.094739in}{1.574941in}}%
\pgfpathlineto{\pgfqpoint{3.162212in}{1.610321in}}%
\pgfpathlineto{\pgfqpoint{3.229828in}{1.645701in}}%
\pgfpathlineto{\pgfqpoint{3.297592in}{1.681080in}}%
\pgfpathlineto{\pgfqpoint{3.365508in}{1.716460in}}%
\pgfpathlineto{\pgfqpoint{3.433580in}{1.751839in}}%
\pgfpathlineto{\pgfqpoint{3.501814in}{1.787219in}}%
\pgfpathlineto{\pgfqpoint{3.570212in}{1.822598in}}%
\pgfpathlineto{\pgfqpoint{3.638780in}{1.857978in}}%
\pgfpathlineto{\pgfqpoint{3.707522in}{1.893358in}}%
\pgfpathlineto{\pgfqpoint{3.776440in}{1.928737in}}%
\pgfpathlineto{\pgfqpoint{3.845538in}{1.964117in}}%
\pgfpathlineto{\pgfqpoint{3.914819in}{1.999496in}}%
\pgfpathlineto{\pgfqpoint{3.984287in}{2.034876in}}%
\pgfpathlineto{\pgfqpoint{4.053945in}{2.070255in}}%
\pgfpathlineto{\pgfqpoint{4.123795in}{2.105635in}}%
\pgfpathlineto{\pgfqpoint{4.193840in}{2.141015in}}%
\pgfpathlineto{\pgfqpoint{4.264081in}{2.176394in}}%
\pgfpathlineto{\pgfqpoint{4.334522in}{2.211774in}}%
\pgfpathlineto{\pgfqpoint{4.405164in}{2.247153in}}%
\pgfpathlineto{\pgfqpoint{4.476009in}{2.282533in}}%
\pgfpathlineto{\pgfqpoint{4.547058in}{2.317912in}}%
\pgfpathlineto{\pgfqpoint{4.618312in}{2.353292in}}%
\pgfpathlineto{\pgfqpoint{4.689773in}{2.388672in}}%
\pgfpathlineto{\pgfqpoint{4.761442in}{2.424051in}}%
\pgfpathlineto{\pgfqpoint{4.833318in}{2.459431in}}%
\pgfpathlineto{\pgfqpoint{4.905403in}{2.494810in}}%
\pgfpathlineto{\pgfqpoint{4.977697in}{2.530190in}}%
\pgfpathlineto{\pgfqpoint{5.050200in}{2.565569in}}%
\pgfpathlineto{\pgfqpoint{5.122911in}{2.600949in}}%
\pgfpathlineto{\pgfqpoint{5.195830in}{2.636328in}}%
\pgfpathlineto{\pgfqpoint{5.268957in}{2.671708in}}%
\pgfpathlineto{\pgfqpoint{5.342291in}{2.707088in}}%
\pgfpathlineto{\pgfqpoint{5.415830in}{2.742467in}}%
\pgfpathlineto{\pgfqpoint{5.489573in}{2.777847in}}%
\pgfpathlineto{\pgfqpoint{5.563519in}{2.813226in}}%
\pgfpathlineto{\pgfqpoint{5.637666in}{2.848606in}}%
\pgfpathlineto{\pgfqpoint{5.712012in}{2.883985in}}%
\pgfpathlineto{\pgfqpoint{5.786556in}{2.919365in}}%
\pgfpathlineto{\pgfqpoint{5.861294in}{2.954745in}}%
\pgfpathlineto{\pgfqpoint{5.936224in}{2.990124in}}%
\pgfpathlineto{\pgfqpoint{6.011344in}{3.025504in}}%
\pgfpathlineto{\pgfqpoint{6.086651in}{3.060883in}}%
\pgfpathlineto{\pgfqpoint{6.162141in}{3.096263in}}%
\pgfpathlineto{\pgfqpoint{6.237812in}{3.131642in}}%
\pgfpathlineto{\pgfqpoint{6.313659in}{3.167022in}}%
\pgfpathlineto{\pgfqpoint{6.389679in}{3.202402in}}%
\pgfpathlineto{\pgfqpoint{6.465869in}{3.237781in}}%
\pgfpathlineto{\pgfqpoint{6.542224in}{3.273161in}}%
\pgfpathlineto{\pgfqpoint{6.618741in}{3.308540in}}%
\pgfpathlineto{\pgfqpoint{6.695414in}{3.343920in}}%
\pgfpathlineto{\pgfqpoint{6.772239in}{3.379299in}}%
\pgfpathlineto{\pgfqpoint{6.849212in}{3.414679in}}%
\pgfpathlineto{\pgfqpoint{6.926328in}{3.450059in}}%
\pgfpathlineto{\pgfqpoint{7.003582in}{3.485438in}}%
\pgfpathlineto{\pgfqpoint{7.080969in}{3.520818in}}%
\pgfpathlineto{\pgfqpoint{7.158483in}{3.556197in}}%
\pgfpathlineto{\pgfqpoint{7.236119in}{3.591577in}}%
\pgfpathlineto{\pgfqpoint{7.313873in}{3.626956in}}%
\pgfpathlineto{\pgfqpoint{7.391737in}{3.662336in}}%
\pgfpathlineto{\pgfqpoint{7.469707in}{3.697716in}}%
\pgfpathlineto{\pgfqpoint{7.547776in}{3.733095in}}%
\pgfpathlineto{\pgfqpoint{7.625939in}{3.768475in}}%
\pgfpathlineto{\pgfqpoint{7.704189in}{3.803854in}}%
\pgfpathlineto{\pgfqpoint{7.782521in}{3.839234in}}%
\pgfpathlineto{\pgfqpoint{7.860928in}{3.874613in}}%
\pgfpathlineto{\pgfqpoint{7.939404in}{3.909993in}}%
\pgfpathlineto{\pgfqpoint{8.017943in}{3.945373in}}%
\pgfpathlineto{\pgfqpoint{8.096538in}{3.980752in}}%
\pgfpathlineto{\pgfqpoint{8.175183in}{4.016132in}}%
\pgfpathlineto{\pgfqpoint{8.253871in}{4.051511in}}%
\pgfpathlineto{\pgfqpoint{8.332595in}{4.086891in}}%
\pgfpathlineto{\pgfqpoint{8.411350in}{4.122270in}}%
\pgfpathlineto{\pgfqpoint{8.490128in}{4.157650in}}%
\pgfpathlineto{\pgfqpoint{8.568923in}{4.193030in}}%
\pgfpathlineto{\pgfqpoint{8.647727in}{4.228409in}}%
\pgfusepath{stroke}%
\end{pgfscope}%
\begin{pgfscope}%
\pgfsetrectcap%
\pgfsetmiterjoin%
\pgfsetlinewidth{0.803000pt}%
\definecolor{currentstroke}{rgb}{0.000000,0.000000,0.000000}%
\pgfsetstrokecolor{currentstroke}%
\pgfsetdash{}{0pt}%
\pgfpathmoveto{\pgfqpoint{1.250000in}{0.625000in}}%
\pgfpathlineto{\pgfqpoint{1.250000in}{4.400000in}}%
\pgfusepath{stroke}%
\end{pgfscope}%
\begin{pgfscope}%
\pgfsetrectcap%
\pgfsetmiterjoin%
\pgfsetlinewidth{0.803000pt}%
\definecolor{currentstroke}{rgb}{0.000000,0.000000,0.000000}%
\pgfsetstrokecolor{currentstroke}%
\pgfsetdash{}{0pt}%
\pgfpathmoveto{\pgfqpoint{9.000000in}{0.625000in}}%
\pgfpathlineto{\pgfqpoint{9.000000in}{4.400000in}}%
\pgfusepath{stroke}%
\end{pgfscope}%
\begin{pgfscope}%
\pgfsetrectcap%
\pgfsetmiterjoin%
\pgfsetlinewidth{0.803000pt}%
\definecolor{currentstroke}{rgb}{0.000000,0.000000,0.000000}%
\pgfsetstrokecolor{currentstroke}%
\pgfsetdash{}{0pt}%
\pgfpathmoveto{\pgfqpoint{1.250000in}{0.625000in}}%
\pgfpathlineto{\pgfqpoint{9.000000in}{0.625000in}}%
\pgfusepath{stroke}%
\end{pgfscope}%
\begin{pgfscope}%
\pgfsetrectcap%
\pgfsetmiterjoin%
\pgfsetlinewidth{0.803000pt}%
\definecolor{currentstroke}{rgb}{0.000000,0.000000,0.000000}%
\pgfsetstrokecolor{currentstroke}%
\pgfsetdash{}{0pt}%
\pgfpathmoveto{\pgfqpoint{1.250000in}{4.400000in}}%
\pgfpathlineto{\pgfqpoint{9.000000in}{4.400000in}}%
\pgfusepath{stroke}%
\end{pgfscope}%
\begin{pgfscope}%
\pgfsetbuttcap%
\pgfsetmiterjoin%
\definecolor{currentfill}{rgb}{1.000000,1.000000,1.000000}%
\pgfsetfillcolor{currentfill}%
\pgfsetfillopacity{0.800000}%
\pgfsetlinewidth{1.003750pt}%
\definecolor{currentstroke}{rgb}{0.800000,0.800000,0.800000}%
\pgfsetstrokecolor{currentstroke}%
\pgfsetstrokeopacity{0.800000}%
\pgfsetdash{}{0pt}%
\pgfpathmoveto{\pgfqpoint{7.153082in}{0.736111in}}%
\pgfpathlineto{\pgfqpoint{8.844444in}{0.736111in}}%
\pgfpathquadraticcurveto{\pgfqpoint{8.888889in}{0.736111in}}{\pgfqpoint{8.888889in}{0.780556in}}%
\pgfpathlineto{\pgfqpoint{8.888889in}{4.020050in}}%
\pgfpathquadraticcurveto{\pgfqpoint{8.888889in}{4.064494in}}{\pgfqpoint{8.844444in}{4.064494in}}%
\pgfpathlineto{\pgfqpoint{7.153082in}{4.064494in}}%
\pgfpathquadraticcurveto{\pgfqpoint{7.108637in}{4.064494in}}{\pgfqpoint{7.108637in}{4.020050in}}%
\pgfpathlineto{\pgfqpoint{7.108637in}{0.780556in}}%
\pgfpathquadraticcurveto{\pgfqpoint{7.108637in}{0.736111in}}{\pgfqpoint{7.153082in}{0.736111in}}%
\pgfpathlineto{\pgfqpoint{7.153082in}{0.736111in}}%
\pgfpathclose%
\pgfusepath{stroke,fill}%
\end{pgfscope}%
\begin{pgfscope}%
\pgfsetrectcap%
\pgfsetroundjoin%
\pgfsetlinewidth{1.505625pt}%
\definecolor{currentstroke}{rgb}{0.121569,0.466667,0.705882}%
\pgfsetstrokecolor{currentstroke}%
\pgfsetdash{}{0pt}%
\pgfpathmoveto{\pgfqpoint{7.197526in}{3.884547in}}%
\pgfpathlineto{\pgfqpoint{7.419748in}{3.884547in}}%
\pgfpathlineto{\pgfqpoint{7.641970in}{3.884547in}}%
\pgfusepath{stroke}%
\end{pgfscope}%
\begin{pgfscope}%
\definecolor{textcolor}{rgb}{0.000000,0.000000,0.000000}%
\pgfsetstrokecolor{textcolor}%
\pgfsetfillcolor{textcolor}%
\pgftext[x=7.819748in,y=3.806769in,left,base]{\color{textcolor}\sffamily\fontsize{16.000000}{19.200000}\selectfont t=100}%
\end{pgfscope}%
\begin{pgfscope}%
\pgfsetrectcap%
\pgfsetroundjoin%
\pgfsetlinewidth{1.505625pt}%
\definecolor{currentstroke}{rgb}{1.000000,0.498039,0.054902}%
\pgfsetstrokecolor{currentstroke}%
\pgfsetdash{}{0pt}%
\pgfpathmoveto{\pgfqpoint{7.197526in}{3.558375in}}%
\pgfpathlineto{\pgfqpoint{7.419748in}{3.558375in}}%
\pgfpathlineto{\pgfqpoint{7.641970in}{3.558375in}}%
\pgfusepath{stroke}%
\end{pgfscope}%
\begin{pgfscope}%
\definecolor{textcolor}{rgb}{0.000000,0.000000,0.000000}%
\pgfsetstrokecolor{textcolor}%
\pgfsetfillcolor{textcolor}%
\pgftext[x=7.819748in,y=3.480597in,left,base]{\color{textcolor}\sffamily\fontsize{16.000000}{19.200000}\selectfont t=2100}%
\end{pgfscope}%
\begin{pgfscope}%
\pgfsetrectcap%
\pgfsetroundjoin%
\pgfsetlinewidth{1.505625pt}%
\definecolor{currentstroke}{rgb}{0.172549,0.627451,0.172549}%
\pgfsetstrokecolor{currentstroke}%
\pgfsetdash{}{0pt}%
\pgfpathmoveto{\pgfqpoint{7.197526in}{3.232203in}}%
\pgfpathlineto{\pgfqpoint{7.419748in}{3.232203in}}%
\pgfpathlineto{\pgfqpoint{7.641970in}{3.232203in}}%
\pgfusepath{stroke}%
\end{pgfscope}%
\begin{pgfscope}%
\definecolor{textcolor}{rgb}{0.000000,0.000000,0.000000}%
\pgfsetstrokecolor{textcolor}%
\pgfsetfillcolor{textcolor}%
\pgftext[x=7.819748in,y=3.154425in,left,base]{\color{textcolor}\sffamily\fontsize{16.000000}{19.200000}\selectfont t=4100}%
\end{pgfscope}%
\begin{pgfscope}%
\pgfsetrectcap%
\pgfsetroundjoin%
\pgfsetlinewidth{1.505625pt}%
\definecolor{currentstroke}{rgb}{0.839216,0.152941,0.156863}%
\pgfsetstrokecolor{currentstroke}%
\pgfsetdash{}{0pt}%
\pgfpathmoveto{\pgfqpoint{7.197526in}{2.906032in}}%
\pgfpathlineto{\pgfqpoint{7.419748in}{2.906032in}}%
\pgfpathlineto{\pgfqpoint{7.641970in}{2.906032in}}%
\pgfusepath{stroke}%
\end{pgfscope}%
\begin{pgfscope}%
\definecolor{textcolor}{rgb}{0.000000,0.000000,0.000000}%
\pgfsetstrokecolor{textcolor}%
\pgfsetfillcolor{textcolor}%
\pgftext[x=7.819748in,y=2.828254in,left,base]{\color{textcolor}\sffamily\fontsize{16.000000}{19.200000}\selectfont t=6100}%
\end{pgfscope}%
\begin{pgfscope}%
\pgfsetrectcap%
\pgfsetroundjoin%
\pgfsetlinewidth{1.505625pt}%
\definecolor{currentstroke}{rgb}{0.580392,0.403922,0.741176}%
\pgfsetstrokecolor{currentstroke}%
\pgfsetdash{}{0pt}%
\pgfpathmoveto{\pgfqpoint{7.197526in}{2.579860in}}%
\pgfpathlineto{\pgfqpoint{7.419748in}{2.579860in}}%
\pgfpathlineto{\pgfqpoint{7.641970in}{2.579860in}}%
\pgfusepath{stroke}%
\end{pgfscope}%
\begin{pgfscope}%
\definecolor{textcolor}{rgb}{0.000000,0.000000,0.000000}%
\pgfsetstrokecolor{textcolor}%
\pgfsetfillcolor{textcolor}%
\pgftext[x=7.819748in,y=2.502082in,left,base]{\color{textcolor}\sffamily\fontsize{16.000000}{19.200000}\selectfont t=8100}%
\end{pgfscope}%
\begin{pgfscope}%
\pgfsetrectcap%
\pgfsetroundjoin%
\pgfsetlinewidth{1.505625pt}%
\definecolor{currentstroke}{rgb}{0.549020,0.337255,0.294118}%
\pgfsetstrokecolor{currentstroke}%
\pgfsetdash{}{0pt}%
\pgfpathmoveto{\pgfqpoint{7.197526in}{2.253688in}}%
\pgfpathlineto{\pgfqpoint{7.419748in}{2.253688in}}%
\pgfpathlineto{\pgfqpoint{7.641970in}{2.253688in}}%
\pgfusepath{stroke}%
\end{pgfscope}%
\begin{pgfscope}%
\definecolor{textcolor}{rgb}{0.000000,0.000000,0.000000}%
\pgfsetstrokecolor{textcolor}%
\pgfsetfillcolor{textcolor}%
\pgftext[x=7.819748in,y=2.175910in,left,base]{\color{textcolor}\sffamily\fontsize{16.000000}{19.200000}\selectfont t=10100}%
\end{pgfscope}%
\begin{pgfscope}%
\pgfsetrectcap%
\pgfsetroundjoin%
\pgfsetlinewidth{1.505625pt}%
\definecolor{currentstroke}{rgb}{0.890196,0.466667,0.760784}%
\pgfsetstrokecolor{currentstroke}%
\pgfsetdash{}{0pt}%
\pgfpathmoveto{\pgfqpoint{7.197526in}{1.927517in}}%
\pgfpathlineto{\pgfqpoint{7.419748in}{1.927517in}}%
\pgfpathlineto{\pgfqpoint{7.641970in}{1.927517in}}%
\pgfusepath{stroke}%
\end{pgfscope}%
\begin{pgfscope}%
\definecolor{textcolor}{rgb}{0.000000,0.000000,0.000000}%
\pgfsetstrokecolor{textcolor}%
\pgfsetfillcolor{textcolor}%
\pgftext[x=7.819748in,y=1.849739in,left,base]{\color{textcolor}\sffamily\fontsize{16.000000}{19.200000}\selectfont t=12100}%
\end{pgfscope}%
\begin{pgfscope}%
\pgfsetrectcap%
\pgfsetroundjoin%
\pgfsetlinewidth{1.505625pt}%
\definecolor{currentstroke}{rgb}{0.498039,0.498039,0.498039}%
\pgfsetstrokecolor{currentstroke}%
\pgfsetdash{}{0pt}%
\pgfpathmoveto{\pgfqpoint{7.197526in}{1.601345in}}%
\pgfpathlineto{\pgfqpoint{7.419748in}{1.601345in}}%
\pgfpathlineto{\pgfqpoint{7.641970in}{1.601345in}}%
\pgfusepath{stroke}%
\end{pgfscope}%
\begin{pgfscope}%
\definecolor{textcolor}{rgb}{0.000000,0.000000,0.000000}%
\pgfsetstrokecolor{textcolor}%
\pgfsetfillcolor{textcolor}%
\pgftext[x=7.819748in,y=1.523567in,left,base]{\color{textcolor}\sffamily\fontsize{16.000000}{19.200000}\selectfont t=14100}%
\end{pgfscope}%
\begin{pgfscope}%
\pgfsetrectcap%
\pgfsetroundjoin%
\pgfsetlinewidth{1.505625pt}%
\definecolor{currentstroke}{rgb}{0.737255,0.741176,0.133333}%
\pgfsetstrokecolor{currentstroke}%
\pgfsetdash{}{0pt}%
\pgfpathmoveto{\pgfqpoint{7.197526in}{1.275173in}}%
\pgfpathlineto{\pgfqpoint{7.419748in}{1.275173in}}%
\pgfpathlineto{\pgfqpoint{7.641970in}{1.275173in}}%
\pgfusepath{stroke}%
\end{pgfscope}%
\begin{pgfscope}%
\definecolor{textcolor}{rgb}{0.000000,0.000000,0.000000}%
\pgfsetstrokecolor{textcolor}%
\pgfsetfillcolor{textcolor}%
\pgftext[x=7.819748in,y=1.197395in,left,base]{\color{textcolor}\sffamily\fontsize{16.000000}{19.200000}\selectfont t=16100}%
\end{pgfscope}%
\begin{pgfscope}%
\pgfsetrectcap%
\pgfsetroundjoin%
\pgfsetlinewidth{1.505625pt}%
\definecolor{currentstroke}{rgb}{0.090196,0.745098,0.811765}%
\pgfsetstrokecolor{currentstroke}%
\pgfsetdash{}{0pt}%
\pgfpathmoveto{\pgfqpoint{7.197526in}{0.949002in}}%
\pgfpathlineto{\pgfqpoint{7.419748in}{0.949002in}}%
\pgfpathlineto{\pgfqpoint{7.641970in}{0.949002in}}%
\pgfusepath{stroke}%
\end{pgfscope}%
\begin{pgfscope}%
\definecolor{textcolor}{rgb}{0.000000,0.000000,0.000000}%
\pgfsetstrokecolor{textcolor}%
\pgfsetfillcolor{textcolor}%
\pgftext[x=7.819748in,y=0.871224in,left,base]{\color{textcolor}\sffamily\fontsize{16.000000}{19.200000}\selectfont t=18100}%
\end{pgfscope}%
\end{pgfpicture}%
\makeatother%
\endgroup%
}
\caption[Column velocities over time]{The left graph shows the velocities in $x$ direction over time. The right graph shows the numerical solution of $v_{x}$ at $t=100,000$ vs the cleaned solution.}
\label{fig:m4-1-velocities-over-time}
\end{figure}
At first the velocities are high only at the moving lid, but over time the velocities inside the lattice are increasing as well.
Finally, the velocities converge to a straight line where the velocity is constantly decreasing from the moving wall with velocity $v_{lid}=0.995$ to the opposite wall with velocity $v_{lid}=0.005$
However, figure \ref{fig:m4-1-velocities-over-time} only shows the wet nodes.
Plotting the dry nodes as well, as done on the right hand side of figure \ref{fig:m4-1-velocities-over-time}, shows that the dry nodes do not follow this pattern. 
The dry nodes are influenced from the dry nodes on the other side via the streaming and the velocity is applied to the wet nodes but leaks into the dry nodes.
Thus, the top dry nodes are too slow and the bottom dry nodes are too fast.
However, fitting a curve to the wet nodes and extending it on the dry nodes, as done in the orange line in figure \ref{fig:m4-1-velocities-over-time}, shows that the cleaned values behave as expected.

\section{M5: Poiseuille Flow}
The Poiseuille flow describes a pressure-induced fluid flow between two opposite and static boundaries. 
It is most commonly used to model the flow of a fluid through a pipe. 
Because the outer boundaries introduce friction, the velocity is expected to be highest in the middle of the pipe on lowest at the sides.
To model this flow I use the following parameters:
\begin{enumerate}
    \item $L_{x}, L_{y} = 100, 80$
    \item $\rho(\textbf{r}, 0) = 1$
    \item $\textbf{u}(\textbf{r}, 0) = 0$
    \item $T = 100,000$
    \item $omega = 0.5$
    \item $\rho_{in} = 1.01$
    \item $\rho_{out} = 0.99$
\end{enumerate}
The corresponding results in the github repo can be found  \href{https://github.com/jonas27/pylbm/blob/master/milestones/m5/m5.ipynb}{here}.

At the inlet a constant pressure is applied and propagated through the grid until it reaches the outlet. The resulting pressure difference between the in- and outlet is $0.02$. 
The graphical results are shown in figure \ref{fig:m5-1-vel-time}.
\begin{figure}[ht]
\centering
\includegraphics[width=\columnwidth]{milestones/final/img/m5-1-vel-time.png}
\vspace*{-4mm}
\caption[Poiseuille Velocities]{Poiseuille velocities visualized with streamplot at $t=50$ and $t=100$ and with imshow at $t=500$ and $t=10,000$ }
\label{fig:m5-1-vel-time}
\end{figure}
The two left grids visualize the velocities with \textit{streamplot} at $t=50$ and $t=100$ while the last two grids use \textit{imshow}. 

At $t=50$ and $t=100$ the velocities in the \textit{streamplot} flow in the correct direction and the effects of the reflective boundaries are clearly visible at the bottom and top of the pipe. While in the middle of the pipe the arrows point parallel to the boundaries, close to the boundaries turbulences caused by friction develop, which become less pronounced over time.
At $t=500$ and $t=10,000$ I use an \textit{imshow} to show the magnitudes of the velocities.
At $t=500$ they are still very small, between $0\leq\textbf{u}_{x}(x,y)\leq0.3$, but become larger and converge towards $0\leq\textbf{u}_{x}(x,y)\leq1.1$.

Finally, I compare the theoretical solution to the numerical solution at $t=100,000$ shown in figure \ref{fig:m5-1-num-theo}.
\begin{figure}[ht]
\centering
\resizebox{0.6\columnwidth}{!}{\large%% Creator: Matplotlib, PGF backend
%%
%% To include the figure in your LaTeX document, write
%%   \input{<filename>.pgf}
%%
%% Make sure the required packages are loaded in your preamble
%%   \usepackage{pgf}
%%
%% Also ensure that all the required font packages are loaded; for instance,
%% the lmodern package is sometimes necessary when using math font.
%%   \usepackage{lmodern}
%%
%% Figures using additional raster images can only be included by \input if
%% they are in the same directory as the main LaTeX file. For loading figures
%% from other directories you can use the `import` package
%%   \usepackage{import}
%%
%% and then include the figures with
%%   \import{<path to file>}{<filename>.pgf}
%%
%% Matplotlib used the following preamble
%%   \usepackage{fontspec}
%%   \setmainfont{DejaVuSerif.ttf}[Path=\detokenize{/home/joe/miniconda3/envs/high/lib/python3.9/site-packages/matplotlib/mpl-data/fonts/ttf/}]
%%   \setsansfont{DejaVuSans.ttf}[Path=\detokenize{/home/joe/miniconda3/envs/high/lib/python3.9/site-packages/matplotlib/mpl-data/fonts/ttf/}]
%%   \setmonofont{DejaVuSansMono.ttf}[Path=\detokenize{/home/joe/miniconda3/envs/high/lib/python3.9/site-packages/matplotlib/mpl-data/fonts/ttf/}]
%%
\begingroup%
\makeatletter%
\begin{pgfpicture}%
\pgfpathrectangle{\pgfpointorigin}{\pgfqpoint{10.000000in}{8.000000in}}%
\pgfusepath{use as bounding box, clip}%
\begin{pgfscope}%
\pgfsetbuttcap%
\pgfsetmiterjoin%
\pgfsetlinewidth{0.000000pt}%
\definecolor{currentstroke}{rgb}{1.000000,1.000000,1.000000}%
\pgfsetstrokecolor{currentstroke}%
\pgfsetstrokeopacity{0.000000}%
\pgfsetdash{}{0pt}%
\pgfpathmoveto{\pgfqpoint{0.000000in}{0.000000in}}%
\pgfpathlineto{\pgfqpoint{10.000000in}{0.000000in}}%
\pgfpathlineto{\pgfqpoint{10.000000in}{8.000000in}}%
\pgfpathlineto{\pgfqpoint{0.000000in}{8.000000in}}%
\pgfpathlineto{\pgfqpoint{0.000000in}{0.000000in}}%
\pgfpathclose%
\pgfusepath{}%
\end{pgfscope}%
\begin{pgfscope}%
\pgfsetbuttcap%
\pgfsetmiterjoin%
\definecolor{currentfill}{rgb}{1.000000,1.000000,1.000000}%
\pgfsetfillcolor{currentfill}%
\pgfsetlinewidth{0.000000pt}%
\definecolor{currentstroke}{rgb}{0.000000,0.000000,0.000000}%
\pgfsetstrokecolor{currentstroke}%
\pgfsetstrokeopacity{0.000000}%
\pgfsetdash{}{0pt}%
\pgfpathmoveto{\pgfqpoint{1.250000in}{1.000000in}}%
\pgfpathlineto{\pgfqpoint{9.000000in}{1.000000in}}%
\pgfpathlineto{\pgfqpoint{9.000000in}{7.040000in}}%
\pgfpathlineto{\pgfqpoint{1.250000in}{7.040000in}}%
\pgfpathlineto{\pgfqpoint{1.250000in}{1.000000in}}%
\pgfpathclose%
\pgfusepath{fill}%
\end{pgfscope}%
\begin{pgfscope}%
\pgfsetbuttcap%
\pgfsetroundjoin%
\definecolor{currentfill}{rgb}{0.000000,0.000000,0.000000}%
\pgfsetfillcolor{currentfill}%
\pgfsetlinewidth{0.803000pt}%
\definecolor{currentstroke}{rgb}{0.000000,0.000000,0.000000}%
\pgfsetstrokecolor{currentstroke}%
\pgfsetdash{}{0pt}%
\pgfsys@defobject{currentmarker}{\pgfqpoint{0.000000in}{-0.048611in}}{\pgfqpoint{0.000000in}{0.000000in}}{%
\pgfpathmoveto{\pgfqpoint{0.000000in}{0.000000in}}%
\pgfpathlineto{\pgfqpoint{0.000000in}{-0.048611in}}%
\pgfusepath{stroke,fill}%
}%
\begin{pgfscope}%
\pgfsys@transformshift{1.602273in}{1.000000in}%
\pgfsys@useobject{currentmarker}{}%
\end{pgfscope}%
\end{pgfscope}%
\begin{pgfscope}%
\definecolor{textcolor}{rgb}{0.000000,0.000000,0.000000}%
\pgfsetstrokecolor{textcolor}%
\pgfsetfillcolor{textcolor}%
\pgftext[x=1.602273in,y=0.902778in,,top]{\color{textcolor}\sffamily\fontsize{16.000000}{19.200000}\selectfont 0.00}%
\end{pgfscope}%
\begin{pgfscope}%
\pgfsetbuttcap%
\pgfsetroundjoin%
\definecolor{currentfill}{rgb}{0.000000,0.000000,0.000000}%
\pgfsetfillcolor{currentfill}%
\pgfsetlinewidth{0.803000pt}%
\definecolor{currentstroke}{rgb}{0.000000,0.000000,0.000000}%
\pgfsetstrokecolor{currentstroke}%
\pgfsetdash{}{0pt}%
\pgfsys@defobject{currentmarker}{\pgfqpoint{0.000000in}{-0.048611in}}{\pgfqpoint{0.000000in}{0.000000in}}{%
\pgfpathmoveto{\pgfqpoint{0.000000in}{0.000000in}}%
\pgfpathlineto{\pgfqpoint{0.000000in}{-0.048611in}}%
\pgfusepath{stroke,fill}%
}%
\begin{pgfscope}%
\pgfsys@transformshift{2.978012in}{1.000000in}%
\pgfsys@useobject{currentmarker}{}%
\end{pgfscope}%
\end{pgfscope}%
\begin{pgfscope}%
\definecolor{textcolor}{rgb}{0.000000,0.000000,0.000000}%
\pgfsetstrokecolor{textcolor}%
\pgfsetfillcolor{textcolor}%
\pgftext[x=2.978012in,y=0.902778in,,top]{\color{textcolor}\sffamily\fontsize{16.000000}{19.200000}\selectfont 0.02}%
\end{pgfscope}%
\begin{pgfscope}%
\pgfsetbuttcap%
\pgfsetroundjoin%
\definecolor{currentfill}{rgb}{0.000000,0.000000,0.000000}%
\pgfsetfillcolor{currentfill}%
\pgfsetlinewidth{0.803000pt}%
\definecolor{currentstroke}{rgb}{0.000000,0.000000,0.000000}%
\pgfsetstrokecolor{currentstroke}%
\pgfsetdash{}{0pt}%
\pgfsys@defobject{currentmarker}{\pgfqpoint{0.000000in}{-0.048611in}}{\pgfqpoint{0.000000in}{0.000000in}}{%
\pgfpathmoveto{\pgfqpoint{0.000000in}{0.000000in}}%
\pgfpathlineto{\pgfqpoint{0.000000in}{-0.048611in}}%
\pgfusepath{stroke,fill}%
}%
\begin{pgfscope}%
\pgfsys@transformshift{4.353752in}{1.000000in}%
\pgfsys@useobject{currentmarker}{}%
\end{pgfscope}%
\end{pgfscope}%
\begin{pgfscope}%
\definecolor{textcolor}{rgb}{0.000000,0.000000,0.000000}%
\pgfsetstrokecolor{textcolor}%
\pgfsetfillcolor{textcolor}%
\pgftext[x=4.353752in,y=0.902778in,,top]{\color{textcolor}\sffamily\fontsize{16.000000}{19.200000}\selectfont 0.04}%
\end{pgfscope}%
\begin{pgfscope}%
\pgfsetbuttcap%
\pgfsetroundjoin%
\definecolor{currentfill}{rgb}{0.000000,0.000000,0.000000}%
\pgfsetfillcolor{currentfill}%
\pgfsetlinewidth{0.803000pt}%
\definecolor{currentstroke}{rgb}{0.000000,0.000000,0.000000}%
\pgfsetstrokecolor{currentstroke}%
\pgfsetdash{}{0pt}%
\pgfsys@defobject{currentmarker}{\pgfqpoint{0.000000in}{-0.048611in}}{\pgfqpoint{0.000000in}{0.000000in}}{%
\pgfpathmoveto{\pgfqpoint{0.000000in}{0.000000in}}%
\pgfpathlineto{\pgfqpoint{0.000000in}{-0.048611in}}%
\pgfusepath{stroke,fill}%
}%
\begin{pgfscope}%
\pgfsys@transformshift{5.729492in}{1.000000in}%
\pgfsys@useobject{currentmarker}{}%
\end{pgfscope}%
\end{pgfscope}%
\begin{pgfscope}%
\definecolor{textcolor}{rgb}{0.000000,0.000000,0.000000}%
\pgfsetstrokecolor{textcolor}%
\pgfsetfillcolor{textcolor}%
\pgftext[x=5.729492in,y=0.902778in,,top]{\color{textcolor}\sffamily\fontsize{16.000000}{19.200000}\selectfont 0.06}%
\end{pgfscope}%
\begin{pgfscope}%
\pgfsetbuttcap%
\pgfsetroundjoin%
\definecolor{currentfill}{rgb}{0.000000,0.000000,0.000000}%
\pgfsetfillcolor{currentfill}%
\pgfsetlinewidth{0.803000pt}%
\definecolor{currentstroke}{rgb}{0.000000,0.000000,0.000000}%
\pgfsetstrokecolor{currentstroke}%
\pgfsetdash{}{0pt}%
\pgfsys@defobject{currentmarker}{\pgfqpoint{0.000000in}{-0.048611in}}{\pgfqpoint{0.000000in}{0.000000in}}{%
\pgfpathmoveto{\pgfqpoint{0.000000in}{0.000000in}}%
\pgfpathlineto{\pgfqpoint{0.000000in}{-0.048611in}}%
\pgfusepath{stroke,fill}%
}%
\begin{pgfscope}%
\pgfsys@transformshift{7.105231in}{1.000000in}%
\pgfsys@useobject{currentmarker}{}%
\end{pgfscope}%
\end{pgfscope}%
\begin{pgfscope}%
\definecolor{textcolor}{rgb}{0.000000,0.000000,0.000000}%
\pgfsetstrokecolor{textcolor}%
\pgfsetfillcolor{textcolor}%
\pgftext[x=7.105231in,y=0.902778in,,top]{\color{textcolor}\sffamily\fontsize{16.000000}{19.200000}\selectfont 0.08}%
\end{pgfscope}%
\begin{pgfscope}%
\pgfsetbuttcap%
\pgfsetroundjoin%
\definecolor{currentfill}{rgb}{0.000000,0.000000,0.000000}%
\pgfsetfillcolor{currentfill}%
\pgfsetlinewidth{0.803000pt}%
\definecolor{currentstroke}{rgb}{0.000000,0.000000,0.000000}%
\pgfsetstrokecolor{currentstroke}%
\pgfsetdash{}{0pt}%
\pgfsys@defobject{currentmarker}{\pgfqpoint{0.000000in}{-0.048611in}}{\pgfqpoint{0.000000in}{0.000000in}}{%
\pgfpathmoveto{\pgfqpoint{0.000000in}{0.000000in}}%
\pgfpathlineto{\pgfqpoint{0.000000in}{-0.048611in}}%
\pgfusepath{stroke,fill}%
}%
\begin{pgfscope}%
\pgfsys@transformshift{8.480971in}{1.000000in}%
\pgfsys@useobject{currentmarker}{}%
\end{pgfscope}%
\end{pgfscope}%
\begin{pgfscope}%
\definecolor{textcolor}{rgb}{0.000000,0.000000,0.000000}%
\pgfsetstrokecolor{textcolor}%
\pgfsetfillcolor{textcolor}%
\pgftext[x=8.480971in,y=0.902778in,,top]{\color{textcolor}\sffamily\fontsize{16.000000}{19.200000}\selectfont 0.10}%
\end{pgfscope}%
\begin{pgfscope}%
\definecolor{textcolor}{rgb}{0.000000,0.000000,0.000000}%
\pgfsetstrokecolor{textcolor}%
\pgfsetfillcolor{textcolor}%
\pgftext[x=5.125000in,y=0.632162in,,top]{\color{textcolor}\sffamily\fontsize{20.000000}{24.000000}\selectfont \(\displaystyle 	\textbf{u}_{x}\)}%
\end{pgfscope}%
\begin{pgfscope}%
\pgfsetbuttcap%
\pgfsetroundjoin%
\definecolor{currentfill}{rgb}{0.000000,0.000000,0.000000}%
\pgfsetfillcolor{currentfill}%
\pgfsetlinewidth{0.803000pt}%
\definecolor{currentstroke}{rgb}{0.000000,0.000000,0.000000}%
\pgfsetstrokecolor{currentstroke}%
\pgfsetdash{}{0pt}%
\pgfsys@defobject{currentmarker}{\pgfqpoint{-0.048611in}{0.000000in}}{\pgfqpoint{-0.000000in}{0.000000in}}{%
\pgfpathmoveto{\pgfqpoint{-0.000000in}{0.000000in}}%
\pgfpathlineto{\pgfqpoint{-0.048611in}{0.000000in}}%
\pgfusepath{stroke,fill}%
}%
\begin{pgfscope}%
\pgfsys@transformshift{1.250000in}{1.274545in}%
\pgfsys@useobject{currentmarker}{}%
\end{pgfscope}%
\end{pgfscope}%
\begin{pgfscope}%
\definecolor{textcolor}{rgb}{0.000000,0.000000,0.000000}%
\pgfsetstrokecolor{textcolor}%
\pgfsetfillcolor{textcolor}%
\pgftext[x=1.011393in, y=1.190127in, left, base]{\color{textcolor}\sffamily\fontsize{16.000000}{19.200000}\selectfont 0}%
\end{pgfscope}%
\begin{pgfscope}%
\pgfsetbuttcap%
\pgfsetroundjoin%
\definecolor{currentfill}{rgb}{0.000000,0.000000,0.000000}%
\pgfsetfillcolor{currentfill}%
\pgfsetlinewidth{0.803000pt}%
\definecolor{currentstroke}{rgb}{0.000000,0.000000,0.000000}%
\pgfsetstrokecolor{currentstroke}%
\pgfsetdash{}{0pt}%
\pgfsys@defobject{currentmarker}{\pgfqpoint{-0.048611in}{0.000000in}}{\pgfqpoint{-0.000000in}{0.000000in}}{%
\pgfpathmoveto{\pgfqpoint{-0.000000in}{0.000000in}}%
\pgfpathlineto{\pgfqpoint{-0.048611in}{0.000000in}}%
\pgfusepath{stroke,fill}%
}%
\begin{pgfscope}%
\pgfsys@transformshift{1.250000in}{1.978508in}%
\pgfsys@useobject{currentmarker}{}%
\end{pgfscope}%
\end{pgfscope}%
\begin{pgfscope}%
\definecolor{textcolor}{rgb}{0.000000,0.000000,0.000000}%
\pgfsetstrokecolor{textcolor}%
\pgfsetfillcolor{textcolor}%
\pgftext[x=0.870009in, y=1.894090in, left, base]{\color{textcolor}\sffamily\fontsize{16.000000}{19.200000}\selectfont 10}%
\end{pgfscope}%
\begin{pgfscope}%
\pgfsetbuttcap%
\pgfsetroundjoin%
\definecolor{currentfill}{rgb}{0.000000,0.000000,0.000000}%
\pgfsetfillcolor{currentfill}%
\pgfsetlinewidth{0.803000pt}%
\definecolor{currentstroke}{rgb}{0.000000,0.000000,0.000000}%
\pgfsetstrokecolor{currentstroke}%
\pgfsetdash{}{0pt}%
\pgfsys@defobject{currentmarker}{\pgfqpoint{-0.048611in}{0.000000in}}{\pgfqpoint{-0.000000in}{0.000000in}}{%
\pgfpathmoveto{\pgfqpoint{-0.000000in}{0.000000in}}%
\pgfpathlineto{\pgfqpoint{-0.048611in}{0.000000in}}%
\pgfusepath{stroke,fill}%
}%
\begin{pgfscope}%
\pgfsys@transformshift{1.250000in}{2.682471in}%
\pgfsys@useobject{currentmarker}{}%
\end{pgfscope}%
\end{pgfscope}%
\begin{pgfscope}%
\definecolor{textcolor}{rgb}{0.000000,0.000000,0.000000}%
\pgfsetstrokecolor{textcolor}%
\pgfsetfillcolor{textcolor}%
\pgftext[x=0.870009in, y=2.598052in, left, base]{\color{textcolor}\sffamily\fontsize{16.000000}{19.200000}\selectfont 20}%
\end{pgfscope}%
\begin{pgfscope}%
\pgfsetbuttcap%
\pgfsetroundjoin%
\definecolor{currentfill}{rgb}{0.000000,0.000000,0.000000}%
\pgfsetfillcolor{currentfill}%
\pgfsetlinewidth{0.803000pt}%
\definecolor{currentstroke}{rgb}{0.000000,0.000000,0.000000}%
\pgfsetstrokecolor{currentstroke}%
\pgfsetdash{}{0pt}%
\pgfsys@defobject{currentmarker}{\pgfqpoint{-0.048611in}{0.000000in}}{\pgfqpoint{-0.000000in}{0.000000in}}{%
\pgfpathmoveto{\pgfqpoint{-0.000000in}{0.000000in}}%
\pgfpathlineto{\pgfqpoint{-0.048611in}{0.000000in}}%
\pgfusepath{stroke,fill}%
}%
\begin{pgfscope}%
\pgfsys@transformshift{1.250000in}{3.386434in}%
\pgfsys@useobject{currentmarker}{}%
\end{pgfscope}%
\end{pgfscope}%
\begin{pgfscope}%
\definecolor{textcolor}{rgb}{0.000000,0.000000,0.000000}%
\pgfsetstrokecolor{textcolor}%
\pgfsetfillcolor{textcolor}%
\pgftext[x=0.870009in, y=3.302015in, left, base]{\color{textcolor}\sffamily\fontsize{16.000000}{19.200000}\selectfont 30}%
\end{pgfscope}%
\begin{pgfscope}%
\pgfsetbuttcap%
\pgfsetroundjoin%
\definecolor{currentfill}{rgb}{0.000000,0.000000,0.000000}%
\pgfsetfillcolor{currentfill}%
\pgfsetlinewidth{0.803000pt}%
\definecolor{currentstroke}{rgb}{0.000000,0.000000,0.000000}%
\pgfsetstrokecolor{currentstroke}%
\pgfsetdash{}{0pt}%
\pgfsys@defobject{currentmarker}{\pgfqpoint{-0.048611in}{0.000000in}}{\pgfqpoint{-0.000000in}{0.000000in}}{%
\pgfpathmoveto{\pgfqpoint{-0.000000in}{0.000000in}}%
\pgfpathlineto{\pgfqpoint{-0.048611in}{0.000000in}}%
\pgfusepath{stroke,fill}%
}%
\begin{pgfscope}%
\pgfsys@transformshift{1.250000in}{4.090396in}%
\pgfsys@useobject{currentmarker}{}%
\end{pgfscope}%
\end{pgfscope}%
\begin{pgfscope}%
\definecolor{textcolor}{rgb}{0.000000,0.000000,0.000000}%
\pgfsetstrokecolor{textcolor}%
\pgfsetfillcolor{textcolor}%
\pgftext[x=0.870009in, y=4.005978in, left, base]{\color{textcolor}\sffamily\fontsize{16.000000}{19.200000}\selectfont 40}%
\end{pgfscope}%
\begin{pgfscope}%
\pgfsetbuttcap%
\pgfsetroundjoin%
\definecolor{currentfill}{rgb}{0.000000,0.000000,0.000000}%
\pgfsetfillcolor{currentfill}%
\pgfsetlinewidth{0.803000pt}%
\definecolor{currentstroke}{rgb}{0.000000,0.000000,0.000000}%
\pgfsetstrokecolor{currentstroke}%
\pgfsetdash{}{0pt}%
\pgfsys@defobject{currentmarker}{\pgfqpoint{-0.048611in}{0.000000in}}{\pgfqpoint{-0.000000in}{0.000000in}}{%
\pgfpathmoveto{\pgfqpoint{-0.000000in}{0.000000in}}%
\pgfpathlineto{\pgfqpoint{-0.048611in}{0.000000in}}%
\pgfusepath{stroke,fill}%
}%
\begin{pgfscope}%
\pgfsys@transformshift{1.250000in}{4.794359in}%
\pgfsys@useobject{currentmarker}{}%
\end{pgfscope}%
\end{pgfscope}%
\begin{pgfscope}%
\definecolor{textcolor}{rgb}{0.000000,0.000000,0.000000}%
\pgfsetstrokecolor{textcolor}%
\pgfsetfillcolor{textcolor}%
\pgftext[x=0.870009in, y=4.709941in, left, base]{\color{textcolor}\sffamily\fontsize{16.000000}{19.200000}\selectfont 50}%
\end{pgfscope}%
\begin{pgfscope}%
\pgfsetbuttcap%
\pgfsetroundjoin%
\definecolor{currentfill}{rgb}{0.000000,0.000000,0.000000}%
\pgfsetfillcolor{currentfill}%
\pgfsetlinewidth{0.803000pt}%
\definecolor{currentstroke}{rgb}{0.000000,0.000000,0.000000}%
\pgfsetstrokecolor{currentstroke}%
\pgfsetdash{}{0pt}%
\pgfsys@defobject{currentmarker}{\pgfqpoint{-0.048611in}{0.000000in}}{\pgfqpoint{-0.000000in}{0.000000in}}{%
\pgfpathmoveto{\pgfqpoint{-0.000000in}{0.000000in}}%
\pgfpathlineto{\pgfqpoint{-0.048611in}{0.000000in}}%
\pgfusepath{stroke,fill}%
}%
\begin{pgfscope}%
\pgfsys@transformshift{1.250000in}{5.498322in}%
\pgfsys@useobject{currentmarker}{}%
\end{pgfscope}%
\end{pgfscope}%
\begin{pgfscope}%
\definecolor{textcolor}{rgb}{0.000000,0.000000,0.000000}%
\pgfsetstrokecolor{textcolor}%
\pgfsetfillcolor{textcolor}%
\pgftext[x=0.870009in, y=5.413903in, left, base]{\color{textcolor}\sffamily\fontsize{16.000000}{19.200000}\selectfont 60}%
\end{pgfscope}%
\begin{pgfscope}%
\pgfsetbuttcap%
\pgfsetroundjoin%
\definecolor{currentfill}{rgb}{0.000000,0.000000,0.000000}%
\pgfsetfillcolor{currentfill}%
\pgfsetlinewidth{0.803000pt}%
\definecolor{currentstroke}{rgb}{0.000000,0.000000,0.000000}%
\pgfsetstrokecolor{currentstroke}%
\pgfsetdash{}{0pt}%
\pgfsys@defobject{currentmarker}{\pgfqpoint{-0.048611in}{0.000000in}}{\pgfqpoint{-0.000000in}{0.000000in}}{%
\pgfpathmoveto{\pgfqpoint{-0.000000in}{0.000000in}}%
\pgfpathlineto{\pgfqpoint{-0.048611in}{0.000000in}}%
\pgfusepath{stroke,fill}%
}%
\begin{pgfscope}%
\pgfsys@transformshift{1.250000in}{6.202284in}%
\pgfsys@useobject{currentmarker}{}%
\end{pgfscope}%
\end{pgfscope}%
\begin{pgfscope}%
\definecolor{textcolor}{rgb}{0.000000,0.000000,0.000000}%
\pgfsetstrokecolor{textcolor}%
\pgfsetfillcolor{textcolor}%
\pgftext[x=0.870009in, y=6.117866in, left, base]{\color{textcolor}\sffamily\fontsize{16.000000}{19.200000}\selectfont 70}%
\end{pgfscope}%
\begin{pgfscope}%
\pgfsetbuttcap%
\pgfsetroundjoin%
\definecolor{currentfill}{rgb}{0.000000,0.000000,0.000000}%
\pgfsetfillcolor{currentfill}%
\pgfsetlinewidth{0.803000pt}%
\definecolor{currentstroke}{rgb}{0.000000,0.000000,0.000000}%
\pgfsetstrokecolor{currentstroke}%
\pgfsetdash{}{0pt}%
\pgfsys@defobject{currentmarker}{\pgfqpoint{-0.048611in}{0.000000in}}{\pgfqpoint{-0.000000in}{0.000000in}}{%
\pgfpathmoveto{\pgfqpoint{-0.000000in}{0.000000in}}%
\pgfpathlineto{\pgfqpoint{-0.048611in}{0.000000in}}%
\pgfusepath{stroke,fill}%
}%
\begin{pgfscope}%
\pgfsys@transformshift{1.250000in}{6.906247in}%
\pgfsys@useobject{currentmarker}{}%
\end{pgfscope}%
\end{pgfscope}%
\begin{pgfscope}%
\definecolor{textcolor}{rgb}{0.000000,0.000000,0.000000}%
\pgfsetstrokecolor{textcolor}%
\pgfsetfillcolor{textcolor}%
\pgftext[x=0.870009in, y=6.821829in, left, base]{\color{textcolor}\sffamily\fontsize{16.000000}{19.200000}\selectfont 80}%
\end{pgfscope}%
\begin{pgfscope}%
\definecolor{textcolor}{rgb}{0.000000,0.000000,0.000000}%
\pgfsetstrokecolor{textcolor}%
\pgfsetfillcolor{textcolor}%
\pgftext[x=0.814453in,y=4.020000in,,bottom,rotate=90.000000]{\color{textcolor}\sffamily\fontsize{20.000000}{24.000000}\selectfont y}%
\end{pgfscope}%
\begin{pgfscope}%
\pgfpathrectangle{\pgfqpoint{1.250000in}{1.000000in}}{\pgfqpoint{7.750000in}{6.040000in}}%
\pgfusepath{clip}%
\pgfsetrectcap%
\pgfsetroundjoin%
\pgfsetlinewidth{1.505625pt}%
\definecolor{currentstroke}{rgb}{1.000000,0.000000,0.000000}%
\pgfsetstrokecolor{currentstroke}%
\pgfsetdash{}{0pt}%
\pgfpathmoveto{\pgfqpoint{1.602273in}{1.274545in}}%
\pgfpathlineto{\pgfqpoint{1.958946in}{1.344942in}}%
\pgfpathlineto{\pgfqpoint{2.306355in}{1.415338in}}%
\pgfpathlineto{\pgfqpoint{2.644500in}{1.485734in}}%
\pgfpathlineto{\pgfqpoint{2.973380in}{1.556131in}}%
\pgfpathlineto{\pgfqpoint{3.292997in}{1.626527in}}%
\pgfpathlineto{\pgfqpoint{3.603349in}{1.696923in}}%
\pgfpathlineto{\pgfqpoint{3.904436in}{1.767319in}}%
\pgfpathlineto{\pgfqpoint{4.196260in}{1.837716in}}%
\pgfpathlineto{\pgfqpoint{4.478819in}{1.908112in}}%
\pgfpathlineto{\pgfqpoint{4.752114in}{1.978508in}}%
\pgfpathlineto{\pgfqpoint{5.016145in}{2.048904in}}%
\pgfpathlineto{\pgfqpoint{5.270912in}{2.119301in}}%
\pgfpathlineto{\pgfqpoint{5.516414in}{2.189697in}}%
\pgfpathlineto{\pgfqpoint{5.752652in}{2.260093in}}%
\pgfpathlineto{\pgfqpoint{5.979626in}{2.330490in}}%
\pgfpathlineto{\pgfqpoint{6.197336in}{2.400886in}}%
\pgfpathlineto{\pgfqpoint{6.405781in}{2.471282in}}%
\pgfpathlineto{\pgfqpoint{6.604962in}{2.541678in}}%
\pgfpathlineto{\pgfqpoint{6.794879in}{2.612075in}}%
\pgfpathlineto{\pgfqpoint{6.975532in}{2.682471in}}%
\pgfpathlineto{\pgfqpoint{7.146920in}{2.752867in}}%
\pgfpathlineto{\pgfqpoint{7.309045in}{2.823263in}}%
\pgfpathlineto{\pgfqpoint{7.461905in}{2.893660in}}%
\pgfpathlineto{\pgfqpoint{7.605500in}{2.964056in}}%
\pgfpathlineto{\pgfqpoint{7.739832in}{3.034452in}}%
\pgfpathlineto{\pgfqpoint{7.864899in}{3.104848in}}%
\pgfpathlineto{\pgfqpoint{7.980702in}{3.175245in}}%
\pgfpathlineto{\pgfqpoint{8.087241in}{3.245641in}}%
\pgfpathlineto{\pgfqpoint{8.184515in}{3.316037in}}%
\pgfpathlineto{\pgfqpoint{8.272526in}{3.386434in}}%
\pgfpathlineto{\pgfqpoint{8.351272in}{3.456830in}}%
\pgfpathlineto{\pgfqpoint{8.420753in}{3.527226in}}%
\pgfpathlineto{\pgfqpoint{8.480971in}{3.597622in}}%
\pgfpathlineto{\pgfqpoint{8.531924in}{3.668019in}}%
\pgfpathlineto{\pgfqpoint{8.573613in}{3.738415in}}%
\pgfpathlineto{\pgfqpoint{8.606038in}{3.808811in}}%
\pgfpathlineto{\pgfqpoint{8.629199in}{3.879207in}}%
\pgfpathlineto{\pgfqpoint{8.643095in}{3.949604in}}%
\pgfpathlineto{\pgfqpoint{8.647727in}{4.020000in}}%
\pgfpathlineto{\pgfqpoint{8.643095in}{4.090396in}}%
\pgfpathlineto{\pgfqpoint{8.629199in}{4.160793in}}%
\pgfpathlineto{\pgfqpoint{8.606038in}{4.231189in}}%
\pgfpathlineto{\pgfqpoint{8.573613in}{4.301585in}}%
\pgfpathlineto{\pgfqpoint{8.531924in}{4.371981in}}%
\pgfpathlineto{\pgfqpoint{8.480971in}{4.442378in}}%
\pgfpathlineto{\pgfqpoint{8.420753in}{4.512774in}}%
\pgfpathlineto{\pgfqpoint{8.351272in}{4.583170in}}%
\pgfpathlineto{\pgfqpoint{8.272526in}{4.653566in}}%
\pgfpathlineto{\pgfqpoint{8.184515in}{4.723963in}}%
\pgfpathlineto{\pgfqpoint{8.087241in}{4.794359in}}%
\pgfpathlineto{\pgfqpoint{7.980702in}{4.864755in}}%
\pgfpathlineto{\pgfqpoint{7.864899in}{4.935152in}}%
\pgfpathlineto{\pgfqpoint{7.739832in}{5.005548in}}%
\pgfpathlineto{\pgfqpoint{7.605500in}{5.075944in}}%
\pgfpathlineto{\pgfqpoint{7.461905in}{5.146340in}}%
\pgfpathlineto{\pgfqpoint{7.309045in}{5.216737in}}%
\pgfpathlineto{\pgfqpoint{7.146920in}{5.287133in}}%
\pgfpathlineto{\pgfqpoint{6.975532in}{5.357529in}}%
\pgfpathlineto{\pgfqpoint{6.794879in}{5.427925in}}%
\pgfpathlineto{\pgfqpoint{6.604962in}{5.498322in}}%
\pgfpathlineto{\pgfqpoint{6.405781in}{5.568718in}}%
\pgfpathlineto{\pgfqpoint{6.197336in}{5.639114in}}%
\pgfpathlineto{\pgfqpoint{5.979626in}{5.709510in}}%
\pgfpathlineto{\pgfqpoint{5.752652in}{5.779907in}}%
\pgfpathlineto{\pgfqpoint{5.516414in}{5.850303in}}%
\pgfpathlineto{\pgfqpoint{5.270912in}{5.920699in}}%
\pgfpathlineto{\pgfqpoint{5.016145in}{5.991096in}}%
\pgfpathlineto{\pgfqpoint{4.752114in}{6.061492in}}%
\pgfpathlineto{\pgfqpoint{4.478819in}{6.131888in}}%
\pgfpathlineto{\pgfqpoint{4.196260in}{6.202284in}}%
\pgfpathlineto{\pgfqpoint{3.904436in}{6.272681in}}%
\pgfpathlineto{\pgfqpoint{3.603349in}{6.343077in}}%
\pgfpathlineto{\pgfqpoint{3.292997in}{6.413473in}}%
\pgfpathlineto{\pgfqpoint{2.973380in}{6.483869in}}%
\pgfpathlineto{\pgfqpoint{2.644500in}{6.554266in}}%
\pgfpathlineto{\pgfqpoint{2.306355in}{6.624662in}}%
\pgfpathlineto{\pgfqpoint{1.958946in}{6.695058in}}%
\pgfpathlineto{\pgfqpoint{1.602273in}{6.765455in}}%
\pgfusepath{stroke}%
\end{pgfscope}%
\begin{pgfscope}%
\pgfpathrectangle{\pgfqpoint{1.250000in}{1.000000in}}{\pgfqpoint{7.750000in}{6.040000in}}%
\pgfusepath{clip}%
\pgfsetrectcap%
\pgfsetroundjoin%
\pgfsetlinewidth{1.505625pt}%
\definecolor{currentstroke}{rgb}{0.000000,0.000000,1.000000}%
\pgfsetstrokecolor{currentstroke}%
\pgfsetdash{}{0pt}%
\pgfpathmoveto{\pgfqpoint{2.345366in}{1.344942in}}%
\pgfpathlineto{\pgfqpoint{2.673257in}{1.415338in}}%
\pgfpathlineto{\pgfqpoint{2.992393in}{1.485734in}}%
\pgfpathlineto{\pgfqpoint{3.302781in}{1.556131in}}%
\pgfpathlineto{\pgfqpoint{3.604419in}{1.626527in}}%
\pgfpathlineto{\pgfqpoint{3.897305in}{1.696923in}}%
\pgfpathlineto{\pgfqpoint{4.181434in}{1.767319in}}%
\pgfpathlineto{\pgfqpoint{4.456799in}{1.837716in}}%
\pgfpathlineto{\pgfqpoint{4.723393in}{1.908112in}}%
\pgfpathlineto{\pgfqpoint{4.981209in}{1.978508in}}%
\pgfpathlineto{\pgfqpoint{5.230237in}{2.048904in}}%
\pgfpathlineto{\pgfqpoint{5.470471in}{2.119301in}}%
\pgfpathlineto{\pgfqpoint{5.701905in}{2.189697in}}%
\pgfpathlineto{\pgfqpoint{5.924534in}{2.260093in}}%
\pgfpathlineto{\pgfqpoint{6.138354in}{2.330490in}}%
\pgfpathlineto{\pgfqpoint{6.343363in}{2.400886in}}%
\pgfpathlineto{\pgfqpoint{6.539563in}{2.471282in}}%
\pgfpathlineto{\pgfqpoint{6.726954in}{2.541678in}}%
\pgfpathlineto{\pgfqpoint{6.905541in}{2.612075in}}%
\pgfpathlineto{\pgfqpoint{7.075330in}{2.682471in}}%
\pgfpathlineto{\pgfqpoint{7.236327in}{2.752867in}}%
\pgfpathlineto{\pgfqpoint{7.388539in}{2.823263in}}%
\pgfpathlineto{\pgfqpoint{7.531978in}{2.893660in}}%
\pgfpathlineto{\pgfqpoint{7.666652in}{2.964056in}}%
\pgfpathlineto{\pgfqpoint{7.792572in}{3.034452in}}%
\pgfpathlineto{\pgfqpoint{7.909750in}{3.104848in}}%
\pgfpathlineto{\pgfqpoint{8.018196in}{3.175245in}}%
\pgfpathlineto{\pgfqpoint{8.117921in}{3.245641in}}%
\pgfpathlineto{\pgfqpoint{8.208935in}{3.316037in}}%
\pgfpathlineto{\pgfqpoint{8.291247in}{3.386434in}}%
\pgfpathlineto{\pgfqpoint{8.364866in}{3.456830in}}%
\pgfpathlineto{\pgfqpoint{8.429800in}{3.527226in}}%
\pgfpathlineto{\pgfqpoint{8.486056in}{3.597622in}}%
\pgfpathlineto{\pgfqpoint{8.533638in}{3.668019in}}%
\pgfpathlineto{\pgfqpoint{8.572552in}{3.738415in}}%
\pgfpathlineto{\pgfqpoint{8.602800in}{3.808811in}}%
\pgfpathlineto{\pgfqpoint{8.624383in}{3.879207in}}%
\pgfpathlineto{\pgfqpoint{8.637302in}{3.949604in}}%
\pgfpathlineto{\pgfqpoint{8.641555in}{4.020000in}}%
\pgfpathlineto{\pgfqpoint{8.637141in}{4.090396in}}%
\pgfpathlineto{\pgfqpoint{8.624053in}{4.160793in}}%
\pgfpathlineto{\pgfqpoint{8.602288in}{4.231189in}}%
\pgfpathlineto{\pgfqpoint{8.571837in}{4.301585in}}%
\pgfpathlineto{\pgfqpoint{8.532692in}{4.371981in}}%
\pgfpathlineto{\pgfqpoint{8.484843in}{4.442378in}}%
\pgfpathlineto{\pgfqpoint{8.428279in}{4.512774in}}%
\pgfpathlineto{\pgfqpoint{8.362989in}{4.583170in}}%
\pgfpathlineto{\pgfqpoint{8.288959in}{4.653566in}}%
\pgfpathlineto{\pgfqpoint{8.206175in}{4.723963in}}%
\pgfpathlineto{\pgfqpoint{8.114624in}{4.794359in}}%
\pgfpathlineto{\pgfqpoint{8.014290in}{4.864755in}}%
\pgfpathlineto{\pgfqpoint{7.905161in}{4.935152in}}%
\pgfpathlineto{\pgfqpoint{7.787222in}{5.005548in}}%
\pgfpathlineto{\pgfqpoint{7.660461in}{5.075944in}}%
\pgfpathlineto{\pgfqpoint{7.524868in}{5.146340in}}%
\pgfpathlineto{\pgfqpoint{7.380435in}{5.216737in}}%
\pgfpathlineto{\pgfqpoint{7.227157in}{5.287133in}}%
\pgfpathlineto{\pgfqpoint{7.065031in}{5.357529in}}%
\pgfpathlineto{\pgfqpoint{6.894066in}{5.427925in}}%
\pgfpathlineto{\pgfqpoint{6.714271in}{5.498322in}}%
\pgfpathlineto{\pgfqpoint{6.525664in}{5.568718in}}%
\pgfpathlineto{\pgfqpoint{6.328267in}{5.639114in}}%
\pgfpathlineto{\pgfqpoint{6.122113in}{5.709510in}}%
\pgfpathlineto{\pgfqpoint{5.907242in}{5.779907in}}%
\pgfpathlineto{\pgfqpoint{5.683699in}{5.850303in}}%
\pgfpathlineto{\pgfqpoint{5.451541in}{5.920699in}}%
\pgfpathlineto{\pgfqpoint{5.210826in}{5.991096in}}%
\pgfpathlineto{\pgfqpoint{4.961623in}{6.061492in}}%
\pgfpathlineto{\pgfqpoint{4.704001in}{6.131888in}}%
\pgfpathlineto{\pgfqpoint{4.438033in}{6.202284in}}%
\pgfpathlineto{\pgfqpoint{4.163792in}{6.272681in}}%
\pgfpathlineto{\pgfqpoint{3.881349in}{6.343077in}}%
\pgfpathlineto{\pgfqpoint{3.590767in}{6.413473in}}%
\pgfpathlineto{\pgfqpoint{3.292105in}{6.483869in}}%
\pgfpathlineto{\pgfqpoint{2.985405in}{6.554266in}}%
\pgfpathlineto{\pgfqpoint{2.670700in}{6.624662in}}%
\pgfpathlineto{\pgfqpoint{2.347990in}{6.695058in}}%
\pgfusepath{stroke}%
\end{pgfscope}%
\begin{pgfscope}%
\pgfsetrectcap%
\pgfsetmiterjoin%
\pgfsetlinewidth{0.803000pt}%
\definecolor{currentstroke}{rgb}{0.000000,0.000000,0.000000}%
\pgfsetstrokecolor{currentstroke}%
\pgfsetdash{}{0pt}%
\pgfpathmoveto{\pgfqpoint{1.250000in}{1.000000in}}%
\pgfpathlineto{\pgfqpoint{1.250000in}{7.040000in}}%
\pgfusepath{stroke}%
\end{pgfscope}%
\begin{pgfscope}%
\pgfsetrectcap%
\pgfsetmiterjoin%
\pgfsetlinewidth{0.803000pt}%
\definecolor{currentstroke}{rgb}{0.000000,0.000000,0.000000}%
\pgfsetstrokecolor{currentstroke}%
\pgfsetdash{}{0pt}%
\pgfpathmoveto{\pgfqpoint{9.000000in}{1.000000in}}%
\pgfpathlineto{\pgfqpoint{9.000000in}{7.040000in}}%
\pgfusepath{stroke}%
\end{pgfscope}%
\begin{pgfscope}%
\pgfsetrectcap%
\pgfsetmiterjoin%
\pgfsetlinewidth{0.803000pt}%
\definecolor{currentstroke}{rgb}{0.000000,0.000000,0.000000}%
\pgfsetstrokecolor{currentstroke}%
\pgfsetdash{}{0pt}%
\pgfpathmoveto{\pgfqpoint{1.250000in}{1.000000in}}%
\pgfpathlineto{\pgfqpoint{9.000000in}{1.000000in}}%
\pgfusepath{stroke}%
\end{pgfscope}%
\begin{pgfscope}%
\pgfsetrectcap%
\pgfsetmiterjoin%
\pgfsetlinewidth{0.803000pt}%
\definecolor{currentstroke}{rgb}{0.000000,0.000000,0.000000}%
\pgfsetstrokecolor{currentstroke}%
\pgfsetdash{}{0pt}%
\pgfpathmoveto{\pgfqpoint{1.250000in}{7.040000in}}%
\pgfpathlineto{\pgfqpoint{9.000000in}{7.040000in}}%
\pgfusepath{stroke}%
\end{pgfscope}%
\begin{pgfscope}%
\pgfsetbuttcap%
\pgfsetmiterjoin%
\definecolor{currentfill}{rgb}{1.000000,1.000000,1.000000}%
\pgfsetfillcolor{currentfill}%
\pgfsetfillopacity{0.800000}%
\pgfsetlinewidth{1.003750pt}%
\definecolor{currentstroke}{rgb}{0.800000,0.800000,0.800000}%
\pgfsetstrokecolor{currentstroke}%
\pgfsetstrokeopacity{0.800000}%
\pgfsetdash{}{0pt}%
\pgfpathmoveto{\pgfqpoint{6.060742in}{6.209879in}}%
\pgfpathlineto{\pgfqpoint{8.844444in}{6.209879in}}%
\pgfpathquadraticcurveto{\pgfqpoint{8.888889in}{6.209879in}}{\pgfqpoint{8.888889in}{6.254323in}}%
\pgfpathlineto{\pgfqpoint{8.888889in}{6.884444in}}%
\pgfpathquadraticcurveto{\pgfqpoint{8.888889in}{6.928889in}}{\pgfqpoint{8.844444in}{6.928889in}}%
\pgfpathlineto{\pgfqpoint{6.060742in}{6.928889in}}%
\pgfpathquadraticcurveto{\pgfqpoint{6.016298in}{6.928889in}}{\pgfqpoint{6.016298in}{6.884444in}}%
\pgfpathlineto{\pgfqpoint{6.016298in}{6.254323in}}%
\pgfpathquadraticcurveto{\pgfqpoint{6.016298in}{6.209879in}}{\pgfqpoint{6.060742in}{6.209879in}}%
\pgfpathlineto{\pgfqpoint{6.060742in}{6.209879in}}%
\pgfpathclose%
\pgfusepath{stroke,fill}%
\end{pgfscope}%
\begin{pgfscope}%
\pgfsetrectcap%
\pgfsetroundjoin%
\pgfsetlinewidth{1.505625pt}%
\definecolor{currentstroke}{rgb}{1.000000,0.000000,0.000000}%
\pgfsetstrokecolor{currentstroke}%
\pgfsetdash{}{0pt}%
\pgfpathmoveto{\pgfqpoint{6.105187in}{6.748941in}}%
\pgfpathlineto{\pgfqpoint{6.327409in}{6.748941in}}%
\pgfpathlineto{\pgfqpoint{6.549631in}{6.748941in}}%
\pgfusepath{stroke}%
\end{pgfscope}%
\begin{pgfscope}%
\definecolor{textcolor}{rgb}{0.000000,0.000000,0.000000}%
\pgfsetstrokecolor{textcolor}%
\pgfsetfillcolor{textcolor}%
\pgftext[x=6.727409in,y=6.671163in,left,base]{\color{textcolor}\sffamily\fontsize{16.000000}{19.200000}\selectfont Analytical velocity}%
\end{pgfscope}%
\begin{pgfscope}%
\pgfsetrectcap%
\pgfsetroundjoin%
\pgfsetlinewidth{1.505625pt}%
\definecolor{currentstroke}{rgb}{0.000000,0.000000,1.000000}%
\pgfsetstrokecolor{currentstroke}%
\pgfsetdash{}{0pt}%
\pgfpathmoveto{\pgfqpoint{6.105187in}{6.422769in}}%
\pgfpathlineto{\pgfqpoint{6.327409in}{6.422769in}}%
\pgfpathlineto{\pgfqpoint{6.549631in}{6.422769in}}%
\pgfusepath{stroke}%
\end{pgfscope}%
\begin{pgfscope}%
\definecolor{textcolor}{rgb}{0.000000,0.000000,0.000000}%
\pgfsetstrokecolor{textcolor}%
\pgfsetfillcolor{textcolor}%
\pgftext[x=6.727409in,y=6.344992in,left,base]{\color{textcolor}\sffamily\fontsize{16.000000}{19.200000}\selectfont Numerical velocity}%
\end{pgfscope}%
\end{pgfpicture}%
\makeatother%
\endgroup%
}
\caption[Poiseuille numerical vs analytical]{Poiseuille velocities comparing the numerical (blue) and theoretical (red) solution at $t=100,000$.}
\label{fig:m5-1-num-theo}
\end{figure}
The theoretical solution, red line, has similar velocities at the boundaries but significantly higher velocities at the boundaries (the exact difference is $0.00397$).
A possible explanation could be the velocity reduction because of the collisions.

To test this hypothesis I first calculated the velocities for $L_x=L_y=30$ and $L_x=L_y=300$ as shown in the first two graphs of figure \ref{fig:m5-1-num-theo-extended}.
\begin{figure}[ht]
\centering
\resizebox{\columnwidth}{!}{\large%% Creator: Matplotlib, PGF backend
%%
%% To include the figure in your LaTeX document, write
%%   \input{<filename>.pgf}
%%
%% Make sure the required packages are loaded in your preamble
%%   \usepackage{pgf}
%%
%% Also ensure that all the required font packages are loaded; for instance,
%% the lmodern package is sometimes necessary when using math font.
%%   \usepackage{lmodern}
%%
%% Figures using additional raster images can only be included by \input if
%% they are in the same directory as the main LaTeX file. For loading figures
%% from other directories you can use the `import` package
%%   \usepackage{import}
%%
%% and then include the figures with
%%   \import{<path to file>}{<filename>.pgf}
%%
%% Matplotlib used the following preamble
%%   \usepackage{fontspec}
%%   \setmainfont{DejaVuSerif.ttf}[Path=\detokenize{/home/joe/miniconda3/envs/high/lib/python3.9/site-packages/matplotlib/mpl-data/fonts/ttf/}]
%%   \setsansfont{DejaVuSans.ttf}[Path=\detokenize{/home/joe/miniconda3/envs/high/lib/python3.9/site-packages/matplotlib/mpl-data/fonts/ttf/}]
%%   \setmonofont{DejaVuSansMono.ttf}[Path=\detokenize{/home/joe/miniconda3/envs/high/lib/python3.9/site-packages/matplotlib/mpl-data/fonts/ttf/}]
%%
\begingroup%
\makeatletter%
\begin{pgfpicture}%
\pgfpathrectangle{\pgfpointorigin}{\pgfqpoint{20.000000in}{5.000000in}}%
\pgfusepath{use as bounding box, clip}%
\begin{pgfscope}%
\pgfsetbuttcap%
\pgfsetmiterjoin%
\pgfsetlinewidth{0.000000pt}%
\definecolor{currentstroke}{rgb}{1.000000,1.000000,1.000000}%
\pgfsetstrokecolor{currentstroke}%
\pgfsetstrokeopacity{0.000000}%
\pgfsetdash{}{0pt}%
\pgfpathmoveto{\pgfqpoint{0.000000in}{0.000000in}}%
\pgfpathlineto{\pgfqpoint{20.000000in}{0.000000in}}%
\pgfpathlineto{\pgfqpoint{20.000000in}{5.000000in}}%
\pgfpathlineto{\pgfqpoint{0.000000in}{5.000000in}}%
\pgfpathlineto{\pgfqpoint{0.000000in}{0.000000in}}%
\pgfpathclose%
\pgfusepath{}%
\end{pgfscope}%
\begin{pgfscope}%
\pgfsetbuttcap%
\pgfsetmiterjoin%
\definecolor{currentfill}{rgb}{1.000000,1.000000,1.000000}%
\pgfsetfillcolor{currentfill}%
\pgfsetlinewidth{0.000000pt}%
\definecolor{currentstroke}{rgb}{0.000000,0.000000,0.000000}%
\pgfsetstrokecolor{currentstroke}%
\pgfsetstrokeopacity{0.000000}%
\pgfsetdash{}{0pt}%
\pgfpathmoveto{\pgfqpoint{2.500000in}{0.625000in}}%
\pgfpathlineto{\pgfqpoint{7.058824in}{0.625000in}}%
\pgfpathlineto{\pgfqpoint{7.058824in}{4.400000in}}%
\pgfpathlineto{\pgfqpoint{2.500000in}{4.400000in}}%
\pgfpathlineto{\pgfqpoint{2.500000in}{0.625000in}}%
\pgfpathclose%
\pgfusepath{fill}%
\end{pgfscope}%
\begin{pgfscope}%
\pgfsetbuttcap%
\pgfsetroundjoin%
\definecolor{currentfill}{rgb}{0.000000,0.000000,0.000000}%
\pgfsetfillcolor{currentfill}%
\pgfsetlinewidth{0.803000pt}%
\definecolor{currentstroke}{rgb}{0.000000,0.000000,0.000000}%
\pgfsetstrokecolor{currentstroke}%
\pgfsetdash{}{0pt}%
\pgfsys@defobject{currentmarker}{\pgfqpoint{0.000000in}{-0.048611in}}{\pgfqpoint{0.000000in}{0.000000in}}{%
\pgfpathmoveto{\pgfqpoint{0.000000in}{0.000000in}}%
\pgfpathlineto{\pgfqpoint{0.000000in}{-0.048611in}}%
\pgfusepath{stroke,fill}%
}%
\begin{pgfscope}%
\pgfsys@transformshift{2.707219in}{0.625000in}%
\pgfsys@useobject{currentmarker}{}%
\end{pgfscope}%
\end{pgfscope}%
\begin{pgfscope}%
\definecolor{textcolor}{rgb}{0.000000,0.000000,0.000000}%
\pgfsetstrokecolor{textcolor}%
\pgfsetfillcolor{textcolor}%
\pgftext[x=2.707219in,y=0.527778in,,top]{\color{textcolor}\sffamily\fontsize{16.000000}{19.200000}\selectfont 0.00}%
\end{pgfscope}%
\begin{pgfscope}%
\pgfsetbuttcap%
\pgfsetroundjoin%
\definecolor{currentfill}{rgb}{0.000000,0.000000,0.000000}%
\pgfsetfillcolor{currentfill}%
\pgfsetlinewidth{0.803000pt}%
\definecolor{currentstroke}{rgb}{0.000000,0.000000,0.000000}%
\pgfsetstrokecolor{currentstroke}%
\pgfsetdash{}{0pt}%
\pgfsys@defobject{currentmarker}{\pgfqpoint{0.000000in}{-0.048611in}}{\pgfqpoint{0.000000in}{0.000000in}}{%
\pgfpathmoveto{\pgfqpoint{0.000000in}{0.000000in}}%
\pgfpathlineto{\pgfqpoint{0.000000in}{-0.048611in}}%
\pgfusepath{stroke,fill}%
}%
\begin{pgfscope}%
\pgfsys@transformshift{3.627392in}{0.625000in}%
\pgfsys@useobject{currentmarker}{}%
\end{pgfscope}%
\end{pgfscope}%
\begin{pgfscope}%
\definecolor{textcolor}{rgb}{0.000000,0.000000,0.000000}%
\pgfsetstrokecolor{textcolor}%
\pgfsetfillcolor{textcolor}%
\pgftext[x=3.627392in,y=0.527778in,,top]{\color{textcolor}\sffamily\fontsize{16.000000}{19.200000}\selectfont 0.01}%
\end{pgfscope}%
\begin{pgfscope}%
\pgfsetbuttcap%
\pgfsetroundjoin%
\definecolor{currentfill}{rgb}{0.000000,0.000000,0.000000}%
\pgfsetfillcolor{currentfill}%
\pgfsetlinewidth{0.803000pt}%
\definecolor{currentstroke}{rgb}{0.000000,0.000000,0.000000}%
\pgfsetstrokecolor{currentstroke}%
\pgfsetdash{}{0pt}%
\pgfsys@defobject{currentmarker}{\pgfqpoint{0.000000in}{-0.048611in}}{\pgfqpoint{0.000000in}{0.000000in}}{%
\pgfpathmoveto{\pgfqpoint{0.000000in}{0.000000in}}%
\pgfpathlineto{\pgfqpoint{0.000000in}{-0.048611in}}%
\pgfusepath{stroke,fill}%
}%
\begin{pgfscope}%
\pgfsys@transformshift{4.547565in}{0.625000in}%
\pgfsys@useobject{currentmarker}{}%
\end{pgfscope}%
\end{pgfscope}%
\begin{pgfscope}%
\definecolor{textcolor}{rgb}{0.000000,0.000000,0.000000}%
\pgfsetstrokecolor{textcolor}%
\pgfsetfillcolor{textcolor}%
\pgftext[x=4.547565in,y=0.527778in,,top]{\color{textcolor}\sffamily\fontsize{16.000000}{19.200000}\selectfont 0.02}%
\end{pgfscope}%
\begin{pgfscope}%
\pgfsetbuttcap%
\pgfsetroundjoin%
\definecolor{currentfill}{rgb}{0.000000,0.000000,0.000000}%
\pgfsetfillcolor{currentfill}%
\pgfsetlinewidth{0.803000pt}%
\definecolor{currentstroke}{rgb}{0.000000,0.000000,0.000000}%
\pgfsetstrokecolor{currentstroke}%
\pgfsetdash{}{0pt}%
\pgfsys@defobject{currentmarker}{\pgfqpoint{0.000000in}{-0.048611in}}{\pgfqpoint{0.000000in}{0.000000in}}{%
\pgfpathmoveto{\pgfqpoint{0.000000in}{0.000000in}}%
\pgfpathlineto{\pgfqpoint{0.000000in}{-0.048611in}}%
\pgfusepath{stroke,fill}%
}%
\begin{pgfscope}%
\pgfsys@transformshift{5.467738in}{0.625000in}%
\pgfsys@useobject{currentmarker}{}%
\end{pgfscope}%
\end{pgfscope}%
\begin{pgfscope}%
\definecolor{textcolor}{rgb}{0.000000,0.000000,0.000000}%
\pgfsetstrokecolor{textcolor}%
\pgfsetfillcolor{textcolor}%
\pgftext[x=5.467738in,y=0.527778in,,top]{\color{textcolor}\sffamily\fontsize{16.000000}{19.200000}\selectfont 0.03}%
\end{pgfscope}%
\begin{pgfscope}%
\pgfsetbuttcap%
\pgfsetroundjoin%
\definecolor{currentfill}{rgb}{0.000000,0.000000,0.000000}%
\pgfsetfillcolor{currentfill}%
\pgfsetlinewidth{0.803000pt}%
\definecolor{currentstroke}{rgb}{0.000000,0.000000,0.000000}%
\pgfsetstrokecolor{currentstroke}%
\pgfsetdash{}{0pt}%
\pgfsys@defobject{currentmarker}{\pgfqpoint{0.000000in}{-0.048611in}}{\pgfqpoint{0.000000in}{0.000000in}}{%
\pgfpathmoveto{\pgfqpoint{0.000000in}{0.000000in}}%
\pgfpathlineto{\pgfqpoint{0.000000in}{-0.048611in}}%
\pgfusepath{stroke,fill}%
}%
\begin{pgfscope}%
\pgfsys@transformshift{6.387910in}{0.625000in}%
\pgfsys@useobject{currentmarker}{}%
\end{pgfscope}%
\end{pgfscope}%
\begin{pgfscope}%
\definecolor{textcolor}{rgb}{0.000000,0.000000,0.000000}%
\pgfsetstrokecolor{textcolor}%
\pgfsetfillcolor{textcolor}%
\pgftext[x=6.387910in,y=0.527778in,,top]{\color{textcolor}\sffamily\fontsize{16.000000}{19.200000}\selectfont 0.04}%
\end{pgfscope}%
\begin{pgfscope}%
\definecolor{textcolor}{rgb}{0.000000,0.000000,0.000000}%
\pgfsetstrokecolor{textcolor}%
\pgfsetfillcolor{textcolor}%
\pgftext[x=4.779412in,y=0.257162in,,top]{\color{textcolor}\sffamily\fontsize{20.000000}{24.000000}\selectfont \(\displaystyle u_{x}\)}%
\end{pgfscope}%
\begin{pgfscope}%
\pgfsetbuttcap%
\pgfsetroundjoin%
\definecolor{currentfill}{rgb}{0.000000,0.000000,0.000000}%
\pgfsetfillcolor{currentfill}%
\pgfsetlinewidth{0.803000pt}%
\definecolor{currentstroke}{rgb}{0.000000,0.000000,0.000000}%
\pgfsetstrokecolor{currentstroke}%
\pgfsetdash{}{0pt}%
\pgfsys@defobject{currentmarker}{\pgfqpoint{-0.048611in}{0.000000in}}{\pgfqpoint{-0.000000in}{0.000000in}}{%
\pgfpathmoveto{\pgfqpoint{-0.000000in}{0.000000in}}%
\pgfpathlineto{\pgfqpoint{-0.048611in}{0.000000in}}%
\pgfusepath{stroke,fill}%
}%
\begin{pgfscope}%
\pgfsys@transformshift{2.500000in}{0.796591in}%
\pgfsys@useobject{currentmarker}{}%
\end{pgfscope}%
\end{pgfscope}%
\begin{pgfscope}%
\definecolor{textcolor}{rgb}{0.000000,0.000000,0.000000}%
\pgfsetstrokecolor{textcolor}%
\pgfsetfillcolor{textcolor}%
\pgftext[x=2.261393in, y=0.712173in, left, base]{\color{textcolor}\sffamily\fontsize{16.000000}{19.200000}\selectfont 0}%
\end{pgfscope}%
\begin{pgfscope}%
\pgfsetbuttcap%
\pgfsetroundjoin%
\definecolor{currentfill}{rgb}{0.000000,0.000000,0.000000}%
\pgfsetfillcolor{currentfill}%
\pgfsetlinewidth{0.803000pt}%
\definecolor{currentstroke}{rgb}{0.000000,0.000000,0.000000}%
\pgfsetstrokecolor{currentstroke}%
\pgfsetdash{}{0pt}%
\pgfsys@defobject{currentmarker}{\pgfqpoint{-0.048611in}{0.000000in}}{\pgfqpoint{-0.000000in}{0.000000in}}{%
\pgfpathmoveto{\pgfqpoint{-0.000000in}{0.000000in}}%
\pgfpathlineto{\pgfqpoint{-0.048611in}{0.000000in}}%
\pgfusepath{stroke,fill}%
}%
\begin{pgfscope}%
\pgfsys@transformshift{2.500000in}{1.409416in}%
\pgfsys@useobject{currentmarker}{}%
\end{pgfscope}%
\end{pgfscope}%
\begin{pgfscope}%
\definecolor{textcolor}{rgb}{0.000000,0.000000,0.000000}%
\pgfsetstrokecolor{textcolor}%
\pgfsetfillcolor{textcolor}%
\pgftext[x=2.261393in, y=1.324997in, left, base]{\color{textcolor}\sffamily\fontsize{16.000000}{19.200000}\selectfont 5}%
\end{pgfscope}%
\begin{pgfscope}%
\pgfsetbuttcap%
\pgfsetroundjoin%
\definecolor{currentfill}{rgb}{0.000000,0.000000,0.000000}%
\pgfsetfillcolor{currentfill}%
\pgfsetlinewidth{0.803000pt}%
\definecolor{currentstroke}{rgb}{0.000000,0.000000,0.000000}%
\pgfsetstrokecolor{currentstroke}%
\pgfsetdash{}{0pt}%
\pgfsys@defobject{currentmarker}{\pgfqpoint{-0.048611in}{0.000000in}}{\pgfqpoint{-0.000000in}{0.000000in}}{%
\pgfpathmoveto{\pgfqpoint{-0.000000in}{0.000000in}}%
\pgfpathlineto{\pgfqpoint{-0.048611in}{0.000000in}}%
\pgfusepath{stroke,fill}%
}%
\begin{pgfscope}%
\pgfsys@transformshift{2.500000in}{2.022240in}%
\pgfsys@useobject{currentmarker}{}%
\end{pgfscope}%
\end{pgfscope}%
\begin{pgfscope}%
\definecolor{textcolor}{rgb}{0.000000,0.000000,0.000000}%
\pgfsetstrokecolor{textcolor}%
\pgfsetfillcolor{textcolor}%
\pgftext[x=2.120009in, y=1.937822in, left, base]{\color{textcolor}\sffamily\fontsize{16.000000}{19.200000}\selectfont 10}%
\end{pgfscope}%
\begin{pgfscope}%
\pgfsetbuttcap%
\pgfsetroundjoin%
\definecolor{currentfill}{rgb}{0.000000,0.000000,0.000000}%
\pgfsetfillcolor{currentfill}%
\pgfsetlinewidth{0.803000pt}%
\definecolor{currentstroke}{rgb}{0.000000,0.000000,0.000000}%
\pgfsetstrokecolor{currentstroke}%
\pgfsetdash{}{0pt}%
\pgfsys@defobject{currentmarker}{\pgfqpoint{-0.048611in}{0.000000in}}{\pgfqpoint{-0.000000in}{0.000000in}}{%
\pgfpathmoveto{\pgfqpoint{-0.000000in}{0.000000in}}%
\pgfpathlineto{\pgfqpoint{-0.048611in}{0.000000in}}%
\pgfusepath{stroke,fill}%
}%
\begin{pgfscope}%
\pgfsys@transformshift{2.500000in}{2.635065in}%
\pgfsys@useobject{currentmarker}{}%
\end{pgfscope}%
\end{pgfscope}%
\begin{pgfscope}%
\definecolor{textcolor}{rgb}{0.000000,0.000000,0.000000}%
\pgfsetstrokecolor{textcolor}%
\pgfsetfillcolor{textcolor}%
\pgftext[x=2.120009in, y=2.550647in, left, base]{\color{textcolor}\sffamily\fontsize{16.000000}{19.200000}\selectfont 15}%
\end{pgfscope}%
\begin{pgfscope}%
\pgfsetbuttcap%
\pgfsetroundjoin%
\definecolor{currentfill}{rgb}{0.000000,0.000000,0.000000}%
\pgfsetfillcolor{currentfill}%
\pgfsetlinewidth{0.803000pt}%
\definecolor{currentstroke}{rgb}{0.000000,0.000000,0.000000}%
\pgfsetstrokecolor{currentstroke}%
\pgfsetdash{}{0pt}%
\pgfsys@defobject{currentmarker}{\pgfqpoint{-0.048611in}{0.000000in}}{\pgfqpoint{-0.000000in}{0.000000in}}{%
\pgfpathmoveto{\pgfqpoint{-0.000000in}{0.000000in}}%
\pgfpathlineto{\pgfqpoint{-0.048611in}{0.000000in}}%
\pgfusepath{stroke,fill}%
}%
\begin{pgfscope}%
\pgfsys@transformshift{2.500000in}{3.247890in}%
\pgfsys@useobject{currentmarker}{}%
\end{pgfscope}%
\end{pgfscope}%
\begin{pgfscope}%
\definecolor{textcolor}{rgb}{0.000000,0.000000,0.000000}%
\pgfsetstrokecolor{textcolor}%
\pgfsetfillcolor{textcolor}%
\pgftext[x=2.120009in, y=3.163471in, left, base]{\color{textcolor}\sffamily\fontsize{16.000000}{19.200000}\selectfont 20}%
\end{pgfscope}%
\begin{pgfscope}%
\pgfsetbuttcap%
\pgfsetroundjoin%
\definecolor{currentfill}{rgb}{0.000000,0.000000,0.000000}%
\pgfsetfillcolor{currentfill}%
\pgfsetlinewidth{0.803000pt}%
\definecolor{currentstroke}{rgb}{0.000000,0.000000,0.000000}%
\pgfsetstrokecolor{currentstroke}%
\pgfsetdash{}{0pt}%
\pgfsys@defobject{currentmarker}{\pgfqpoint{-0.048611in}{0.000000in}}{\pgfqpoint{-0.000000in}{0.000000in}}{%
\pgfpathmoveto{\pgfqpoint{-0.000000in}{0.000000in}}%
\pgfpathlineto{\pgfqpoint{-0.048611in}{0.000000in}}%
\pgfusepath{stroke,fill}%
}%
\begin{pgfscope}%
\pgfsys@transformshift{2.500000in}{3.860714in}%
\pgfsys@useobject{currentmarker}{}%
\end{pgfscope}%
\end{pgfscope}%
\begin{pgfscope}%
\definecolor{textcolor}{rgb}{0.000000,0.000000,0.000000}%
\pgfsetstrokecolor{textcolor}%
\pgfsetfillcolor{textcolor}%
\pgftext[x=2.120009in, y=3.776296in, left, base]{\color{textcolor}\sffamily\fontsize{16.000000}{19.200000}\selectfont 25}%
\end{pgfscope}%
\begin{pgfscope}%
\definecolor{textcolor}{rgb}{0.000000,0.000000,0.000000}%
\pgfsetstrokecolor{textcolor}%
\pgfsetfillcolor{textcolor}%
\pgftext[x=2.064453in,y=2.512500in,,bottom,rotate=90.000000]{\color{textcolor}\sffamily\fontsize{20.000000}{24.000000}\selectfont y}%
\end{pgfscope}%
\begin{pgfscope}%
\pgfpathrectangle{\pgfqpoint{2.500000in}{0.625000in}}{\pgfqpoint{4.558824in}{3.775000in}}%
\pgfusepath{clip}%
\pgfsetrectcap%
\pgfsetroundjoin%
\pgfsetlinewidth{1.505625pt}%
\definecolor{currentstroke}{rgb}{1.000000,0.000000,0.000000}%
\pgfsetstrokecolor{currentstroke}%
\pgfsetdash{}{0pt}%
\pgfpathmoveto{\pgfqpoint{2.707219in}{0.796591in}}%
\pgfpathlineto{\pgfqpoint{3.259323in}{0.919156in}}%
\pgfpathlineto{\pgfqpoint{3.770530in}{1.041721in}}%
\pgfpathlineto{\pgfqpoint{4.240841in}{1.164286in}}%
\pgfpathlineto{\pgfqpoint{4.670255in}{1.286851in}}%
\pgfpathlineto{\pgfqpoint{5.058772in}{1.409416in}}%
\pgfpathlineto{\pgfqpoint{5.406393in}{1.531981in}}%
\pgfpathlineto{\pgfqpoint{5.713117in}{1.654545in}}%
\pgfpathlineto{\pgfqpoint{5.978945in}{1.777110in}}%
\pgfpathlineto{\pgfqpoint{6.203876in}{1.899675in}}%
\pgfpathlineto{\pgfqpoint{6.387910in}{2.022240in}}%
\pgfpathlineto{\pgfqpoint{6.531048in}{2.144805in}}%
\pgfpathlineto{\pgfqpoint{6.633290in}{2.267370in}}%
\pgfpathlineto{\pgfqpoint{6.694635in}{2.389935in}}%
\pgfpathlineto{\pgfqpoint{6.715083in}{2.512500in}}%
\pgfpathlineto{\pgfqpoint{6.694635in}{2.635065in}}%
\pgfpathlineto{\pgfqpoint{6.633290in}{2.757630in}}%
\pgfpathlineto{\pgfqpoint{6.531048in}{2.880195in}}%
\pgfpathlineto{\pgfqpoint{6.387910in}{3.002760in}}%
\pgfpathlineto{\pgfqpoint{6.203876in}{3.125325in}}%
\pgfpathlineto{\pgfqpoint{5.978945in}{3.247890in}}%
\pgfpathlineto{\pgfqpoint{5.713117in}{3.370455in}}%
\pgfpathlineto{\pgfqpoint{5.406393in}{3.493019in}}%
\pgfpathlineto{\pgfqpoint{5.058772in}{3.615584in}}%
\pgfpathlineto{\pgfqpoint{4.670255in}{3.738149in}}%
\pgfpathlineto{\pgfqpoint{4.240841in}{3.860714in}}%
\pgfpathlineto{\pgfqpoint{3.770530in}{3.983279in}}%
\pgfpathlineto{\pgfqpoint{3.259323in}{4.105844in}}%
\pgfpathlineto{\pgfqpoint{2.707219in}{4.228409in}}%
\pgfusepath{stroke}%
\end{pgfscope}%
\begin{pgfscope}%
\pgfpathrectangle{\pgfqpoint{2.500000in}{0.625000in}}{\pgfqpoint{4.558824in}{3.775000in}}%
\pgfusepath{clip}%
\pgfsetrectcap%
\pgfsetroundjoin%
\pgfsetlinewidth{1.505625pt}%
\definecolor{currentstroke}{rgb}{0.000000,0.000000,1.000000}%
\pgfsetstrokecolor{currentstroke}%
\pgfsetdash{}{0pt}%
\pgfpathmoveto{\pgfqpoint{3.042478in}{0.857873in}}%
\pgfpathlineto{\pgfqpoint{3.587376in}{0.980438in}}%
\pgfpathlineto{\pgfqpoint{4.090081in}{1.103003in}}%
\pgfpathlineto{\pgfqpoint{4.550746in}{1.225568in}}%
\pgfpathlineto{\pgfqpoint{4.969404in}{1.348133in}}%
\pgfpathlineto{\pgfqpoint{5.346084in}{1.470698in}}%
\pgfpathlineto{\pgfqpoint{5.680816in}{1.593263in}}%
\pgfpathlineto{\pgfqpoint{5.973629in}{1.715828in}}%
\pgfpathlineto{\pgfqpoint{6.224552in}{1.838393in}}%
\pgfpathlineto{\pgfqpoint{6.433611in}{1.960958in}}%
\pgfpathlineto{\pgfqpoint{6.600829in}{2.083523in}}%
\pgfpathlineto{\pgfqpoint{6.726225in}{2.206088in}}%
\pgfpathlineto{\pgfqpoint{6.809815in}{2.328653in}}%
\pgfpathlineto{\pgfqpoint{6.851604in}{2.451218in}}%
\pgfpathlineto{\pgfqpoint{6.851604in}{2.573782in}}%
\pgfpathlineto{\pgfqpoint{6.809815in}{2.696347in}}%
\pgfpathlineto{\pgfqpoint{6.726225in}{2.818912in}}%
\pgfpathlineto{\pgfqpoint{6.600829in}{2.941477in}}%
\pgfpathlineto{\pgfqpoint{6.433611in}{3.064042in}}%
\pgfpathlineto{\pgfqpoint{6.224552in}{3.186607in}}%
\pgfpathlineto{\pgfqpoint{5.973629in}{3.309172in}}%
\pgfpathlineto{\pgfqpoint{5.680816in}{3.431737in}}%
\pgfpathlineto{\pgfqpoint{5.346084in}{3.554302in}}%
\pgfpathlineto{\pgfqpoint{4.969404in}{3.676867in}}%
\pgfpathlineto{\pgfqpoint{4.550746in}{3.799432in}}%
\pgfpathlineto{\pgfqpoint{4.090081in}{3.921997in}}%
\pgfpathlineto{\pgfqpoint{3.587376in}{4.044562in}}%
\pgfpathlineto{\pgfqpoint{3.042478in}{4.167127in}}%
\pgfusepath{stroke}%
\end{pgfscope}%
\begin{pgfscope}%
\pgfsetrectcap%
\pgfsetmiterjoin%
\pgfsetlinewidth{0.803000pt}%
\definecolor{currentstroke}{rgb}{0.000000,0.000000,0.000000}%
\pgfsetstrokecolor{currentstroke}%
\pgfsetdash{}{0pt}%
\pgfpathmoveto{\pgfqpoint{2.500000in}{0.625000in}}%
\pgfpathlineto{\pgfqpoint{2.500000in}{4.400000in}}%
\pgfusepath{stroke}%
\end{pgfscope}%
\begin{pgfscope}%
\pgfsetrectcap%
\pgfsetmiterjoin%
\pgfsetlinewidth{0.803000pt}%
\definecolor{currentstroke}{rgb}{0.000000,0.000000,0.000000}%
\pgfsetstrokecolor{currentstroke}%
\pgfsetdash{}{0pt}%
\pgfpathmoveto{\pgfqpoint{7.058824in}{0.625000in}}%
\pgfpathlineto{\pgfqpoint{7.058824in}{4.400000in}}%
\pgfusepath{stroke}%
\end{pgfscope}%
\begin{pgfscope}%
\pgfsetrectcap%
\pgfsetmiterjoin%
\pgfsetlinewidth{0.803000pt}%
\definecolor{currentstroke}{rgb}{0.000000,0.000000,0.000000}%
\pgfsetstrokecolor{currentstroke}%
\pgfsetdash{}{0pt}%
\pgfpathmoveto{\pgfqpoint{2.500000in}{0.625000in}}%
\pgfpathlineto{\pgfqpoint{7.058824in}{0.625000in}}%
\pgfusepath{stroke}%
\end{pgfscope}%
\begin{pgfscope}%
\pgfsetrectcap%
\pgfsetmiterjoin%
\pgfsetlinewidth{0.803000pt}%
\definecolor{currentstroke}{rgb}{0.000000,0.000000,0.000000}%
\pgfsetstrokecolor{currentstroke}%
\pgfsetdash{}{0pt}%
\pgfpathmoveto{\pgfqpoint{2.500000in}{4.400000in}}%
\pgfpathlineto{\pgfqpoint{7.058824in}{4.400000in}}%
\pgfusepath{stroke}%
\end{pgfscope}%
\begin{pgfscope}%
\definecolor{textcolor}{rgb}{0.000000,0.000000,0.000000}%
\pgfsetstrokecolor{textcolor}%
\pgfsetfillcolor{textcolor}%
\pgftext[x=4.779412in,y=4.483333in,,base]{\color{textcolor}\sffamily\fontsize{20.000000}{24.000000}\selectfont \(\displaystyle L_x=\)30 and \(\displaystyle L_y=\)30}%
\end{pgfscope}%
\begin{pgfscope}%
\pgfsetbuttcap%
\pgfsetmiterjoin%
\definecolor{currentfill}{rgb}{1.000000,1.000000,1.000000}%
\pgfsetfillcolor{currentfill}%
\pgfsetfillopacity{0.800000}%
\pgfsetlinewidth{1.003750pt}%
\definecolor{currentstroke}{rgb}{0.800000,0.800000,0.800000}%
\pgfsetstrokecolor{currentstroke}%
\pgfsetstrokeopacity{0.800000}%
\pgfsetdash{}{0pt}%
\pgfpathmoveto{\pgfqpoint{2.655556in}{2.152995in}}%
\pgfpathlineto{\pgfqpoint{5.037782in}{2.152995in}}%
\pgfpathquadraticcurveto{\pgfqpoint{5.082227in}{2.152995in}}{\pgfqpoint{5.082227in}{2.197439in}}%
\pgfpathlineto{\pgfqpoint{5.082227in}{2.827561in}}%
\pgfpathquadraticcurveto{\pgfqpoint{5.082227in}{2.872005in}}{\pgfqpoint{5.037782in}{2.872005in}}%
\pgfpathlineto{\pgfqpoint{2.655556in}{2.872005in}}%
\pgfpathquadraticcurveto{\pgfqpoint{2.611111in}{2.872005in}}{\pgfqpoint{2.611111in}{2.827561in}}%
\pgfpathlineto{\pgfqpoint{2.611111in}{2.197439in}}%
\pgfpathquadraticcurveto{\pgfqpoint{2.611111in}{2.152995in}}{\pgfqpoint{2.655556in}{2.152995in}}%
\pgfpathlineto{\pgfqpoint{2.655556in}{2.152995in}}%
\pgfpathclose%
\pgfusepath{stroke,fill}%
\end{pgfscope}%
\begin{pgfscope}%
\pgfsetrectcap%
\pgfsetroundjoin%
\pgfsetlinewidth{1.505625pt}%
\definecolor{currentstroke}{rgb}{1.000000,0.000000,0.000000}%
\pgfsetstrokecolor{currentstroke}%
\pgfsetdash{}{0pt}%
\pgfpathmoveto{\pgfqpoint{2.700000in}{2.692057in}}%
\pgfpathlineto{\pgfqpoint{2.922222in}{2.692057in}}%
\pgfpathlineto{\pgfqpoint{3.144444in}{2.692057in}}%
\pgfusepath{stroke}%
\end{pgfscope}%
\begin{pgfscope}%
\definecolor{textcolor}{rgb}{0.000000,0.000000,0.000000}%
\pgfsetstrokecolor{textcolor}%
\pgfsetfillcolor{textcolor}%
\pgftext[x=3.322222in,y=2.614279in,left,base]{\color{textcolor}\sffamily\fontsize{16.000000}{19.200000}\selectfont Theo velocities}%
\end{pgfscope}%
\begin{pgfscope}%
\pgfsetrectcap%
\pgfsetroundjoin%
\pgfsetlinewidth{1.505625pt}%
\definecolor{currentstroke}{rgb}{0.000000,0.000000,1.000000}%
\pgfsetstrokecolor{currentstroke}%
\pgfsetdash{}{0pt}%
\pgfpathmoveto{\pgfqpoint{2.700000in}{2.365885in}}%
\pgfpathlineto{\pgfqpoint{2.922222in}{2.365885in}}%
\pgfpathlineto{\pgfqpoint{3.144444in}{2.365885in}}%
\pgfusepath{stroke}%
\end{pgfscope}%
\begin{pgfscope}%
\definecolor{textcolor}{rgb}{0.000000,0.000000,0.000000}%
\pgfsetstrokecolor{textcolor}%
\pgfsetfillcolor{textcolor}%
\pgftext[x=3.322222in,y=2.288108in,left,base]{\color{textcolor}\sffamily\fontsize{16.000000}{19.200000}\selectfont Num velocities}%
\end{pgfscope}%
\begin{pgfscope}%
\pgfsetbuttcap%
\pgfsetmiterjoin%
\definecolor{currentfill}{rgb}{1.000000,1.000000,1.000000}%
\pgfsetfillcolor{currentfill}%
\pgfsetlinewidth{0.000000pt}%
\definecolor{currentstroke}{rgb}{0.000000,0.000000,0.000000}%
\pgfsetstrokecolor{currentstroke}%
\pgfsetstrokeopacity{0.000000}%
\pgfsetdash{}{0pt}%
\pgfpathmoveto{\pgfqpoint{7.970588in}{0.625000in}}%
\pgfpathlineto{\pgfqpoint{12.529412in}{0.625000in}}%
\pgfpathlineto{\pgfqpoint{12.529412in}{4.400000in}}%
\pgfpathlineto{\pgfqpoint{7.970588in}{4.400000in}}%
\pgfpathlineto{\pgfqpoint{7.970588in}{0.625000in}}%
\pgfpathclose%
\pgfusepath{fill}%
\end{pgfscope}%
\begin{pgfscope}%
\pgfsetbuttcap%
\pgfsetroundjoin%
\definecolor{currentfill}{rgb}{0.000000,0.000000,0.000000}%
\pgfsetfillcolor{currentfill}%
\pgfsetlinewidth{0.803000pt}%
\definecolor{currentstroke}{rgb}{0.000000,0.000000,0.000000}%
\pgfsetstrokecolor{currentstroke}%
\pgfsetdash{}{0pt}%
\pgfsys@defobject{currentmarker}{\pgfqpoint{0.000000in}{-0.048611in}}{\pgfqpoint{0.000000in}{0.000000in}}{%
\pgfpathmoveto{\pgfqpoint{0.000000in}{0.000000in}}%
\pgfpathlineto{\pgfqpoint{0.000000in}{-0.048611in}}%
\pgfusepath{stroke,fill}%
}%
\begin{pgfscope}%
\pgfsys@transformshift{8.177807in}{0.625000in}%
\pgfsys@useobject{currentmarker}{}%
\end{pgfscope}%
\end{pgfscope}%
\begin{pgfscope}%
\definecolor{textcolor}{rgb}{0.000000,0.000000,0.000000}%
\pgfsetstrokecolor{textcolor}%
\pgfsetfillcolor{textcolor}%
\pgftext[x=8.177807in,y=0.527778in,,top]{\color{textcolor}\sffamily\fontsize{16.000000}{19.200000}\selectfont 0.0}%
\end{pgfscope}%
\begin{pgfscope}%
\pgfsetbuttcap%
\pgfsetroundjoin%
\definecolor{currentfill}{rgb}{0.000000,0.000000,0.000000}%
\pgfsetfillcolor{currentfill}%
\pgfsetlinewidth{0.803000pt}%
\definecolor{currentstroke}{rgb}{0.000000,0.000000,0.000000}%
\pgfsetstrokecolor{currentstroke}%
\pgfsetdash{}{0pt}%
\pgfsys@defobject{currentmarker}{\pgfqpoint{0.000000in}{-0.048611in}}{\pgfqpoint{0.000000in}{0.000000in}}{%
\pgfpathmoveto{\pgfqpoint{0.000000in}{0.000000in}}%
\pgfpathlineto{\pgfqpoint{0.000000in}{-0.048611in}}%
\pgfusepath{stroke,fill}%
}%
\begin{pgfscope}%
\pgfsys@transformshift{9.017848in}{0.625000in}%
\pgfsys@useobject{currentmarker}{}%
\end{pgfscope}%
\end{pgfscope}%
\begin{pgfscope}%
\definecolor{textcolor}{rgb}{0.000000,0.000000,0.000000}%
\pgfsetstrokecolor{textcolor}%
\pgfsetfillcolor{textcolor}%
\pgftext[x=9.017848in,y=0.527778in,,top]{\color{textcolor}\sffamily\fontsize{16.000000}{19.200000}\selectfont 0.1}%
\end{pgfscope}%
\begin{pgfscope}%
\pgfsetbuttcap%
\pgfsetroundjoin%
\definecolor{currentfill}{rgb}{0.000000,0.000000,0.000000}%
\pgfsetfillcolor{currentfill}%
\pgfsetlinewidth{0.803000pt}%
\definecolor{currentstroke}{rgb}{0.000000,0.000000,0.000000}%
\pgfsetstrokecolor{currentstroke}%
\pgfsetdash{}{0pt}%
\pgfsys@defobject{currentmarker}{\pgfqpoint{0.000000in}{-0.048611in}}{\pgfqpoint{0.000000in}{0.000000in}}{%
\pgfpathmoveto{\pgfqpoint{0.000000in}{0.000000in}}%
\pgfpathlineto{\pgfqpoint{0.000000in}{-0.048611in}}%
\pgfusepath{stroke,fill}%
}%
\begin{pgfscope}%
\pgfsys@transformshift{9.857888in}{0.625000in}%
\pgfsys@useobject{currentmarker}{}%
\end{pgfscope}%
\end{pgfscope}%
\begin{pgfscope}%
\definecolor{textcolor}{rgb}{0.000000,0.000000,0.000000}%
\pgfsetstrokecolor{textcolor}%
\pgfsetfillcolor{textcolor}%
\pgftext[x=9.857888in,y=0.527778in,,top]{\color{textcolor}\sffamily\fontsize{16.000000}{19.200000}\selectfont 0.2}%
\end{pgfscope}%
\begin{pgfscope}%
\pgfsetbuttcap%
\pgfsetroundjoin%
\definecolor{currentfill}{rgb}{0.000000,0.000000,0.000000}%
\pgfsetfillcolor{currentfill}%
\pgfsetlinewidth{0.803000pt}%
\definecolor{currentstroke}{rgb}{0.000000,0.000000,0.000000}%
\pgfsetstrokecolor{currentstroke}%
\pgfsetdash{}{0pt}%
\pgfsys@defobject{currentmarker}{\pgfqpoint{0.000000in}{-0.048611in}}{\pgfqpoint{0.000000in}{0.000000in}}{%
\pgfpathmoveto{\pgfqpoint{0.000000in}{0.000000in}}%
\pgfpathlineto{\pgfqpoint{0.000000in}{-0.048611in}}%
\pgfusepath{stroke,fill}%
}%
\begin{pgfscope}%
\pgfsys@transformshift{10.697928in}{0.625000in}%
\pgfsys@useobject{currentmarker}{}%
\end{pgfscope}%
\end{pgfscope}%
\begin{pgfscope}%
\definecolor{textcolor}{rgb}{0.000000,0.000000,0.000000}%
\pgfsetstrokecolor{textcolor}%
\pgfsetfillcolor{textcolor}%
\pgftext[x=10.697928in,y=0.527778in,,top]{\color{textcolor}\sffamily\fontsize{16.000000}{19.200000}\selectfont 0.3}%
\end{pgfscope}%
\begin{pgfscope}%
\pgfsetbuttcap%
\pgfsetroundjoin%
\definecolor{currentfill}{rgb}{0.000000,0.000000,0.000000}%
\pgfsetfillcolor{currentfill}%
\pgfsetlinewidth{0.803000pt}%
\definecolor{currentstroke}{rgb}{0.000000,0.000000,0.000000}%
\pgfsetstrokecolor{currentstroke}%
\pgfsetdash{}{0pt}%
\pgfsys@defobject{currentmarker}{\pgfqpoint{0.000000in}{-0.048611in}}{\pgfqpoint{0.000000in}{0.000000in}}{%
\pgfpathmoveto{\pgfqpoint{0.000000in}{0.000000in}}%
\pgfpathlineto{\pgfqpoint{0.000000in}{-0.048611in}}%
\pgfusepath{stroke,fill}%
}%
\begin{pgfscope}%
\pgfsys@transformshift{11.537968in}{0.625000in}%
\pgfsys@useobject{currentmarker}{}%
\end{pgfscope}%
\end{pgfscope}%
\begin{pgfscope}%
\definecolor{textcolor}{rgb}{0.000000,0.000000,0.000000}%
\pgfsetstrokecolor{textcolor}%
\pgfsetfillcolor{textcolor}%
\pgftext[x=11.537968in,y=0.527778in,,top]{\color{textcolor}\sffamily\fontsize{16.000000}{19.200000}\selectfont 0.4}%
\end{pgfscope}%
\begin{pgfscope}%
\pgfsetbuttcap%
\pgfsetroundjoin%
\definecolor{currentfill}{rgb}{0.000000,0.000000,0.000000}%
\pgfsetfillcolor{currentfill}%
\pgfsetlinewidth{0.803000pt}%
\definecolor{currentstroke}{rgb}{0.000000,0.000000,0.000000}%
\pgfsetstrokecolor{currentstroke}%
\pgfsetdash{}{0pt}%
\pgfsys@defobject{currentmarker}{\pgfqpoint{0.000000in}{-0.048611in}}{\pgfqpoint{0.000000in}{0.000000in}}{%
\pgfpathmoveto{\pgfqpoint{0.000000in}{0.000000in}}%
\pgfpathlineto{\pgfqpoint{0.000000in}{-0.048611in}}%
\pgfusepath{stroke,fill}%
}%
\begin{pgfscope}%
\pgfsys@transformshift{12.378009in}{0.625000in}%
\pgfsys@useobject{currentmarker}{}%
\end{pgfscope}%
\end{pgfscope}%
\begin{pgfscope}%
\definecolor{textcolor}{rgb}{0.000000,0.000000,0.000000}%
\pgfsetstrokecolor{textcolor}%
\pgfsetfillcolor{textcolor}%
\pgftext[x=12.378009in,y=0.527778in,,top]{\color{textcolor}\sffamily\fontsize{16.000000}{19.200000}\selectfont 0.5}%
\end{pgfscope}%
\begin{pgfscope}%
\definecolor{textcolor}{rgb}{0.000000,0.000000,0.000000}%
\pgfsetstrokecolor{textcolor}%
\pgfsetfillcolor{textcolor}%
\pgftext[x=10.250000in,y=0.257162in,,top]{\color{textcolor}\sffamily\fontsize{20.000000}{24.000000}\selectfont \(\displaystyle u_{x}\)}%
\end{pgfscope}%
\begin{pgfscope}%
\pgfsetbuttcap%
\pgfsetroundjoin%
\definecolor{currentfill}{rgb}{0.000000,0.000000,0.000000}%
\pgfsetfillcolor{currentfill}%
\pgfsetlinewidth{0.803000pt}%
\definecolor{currentstroke}{rgb}{0.000000,0.000000,0.000000}%
\pgfsetstrokecolor{currentstroke}%
\pgfsetdash{}{0pt}%
\pgfsys@defobject{currentmarker}{\pgfqpoint{-0.048611in}{0.000000in}}{\pgfqpoint{-0.000000in}{0.000000in}}{%
\pgfpathmoveto{\pgfqpoint{-0.000000in}{0.000000in}}%
\pgfpathlineto{\pgfqpoint{-0.048611in}{0.000000in}}%
\pgfusepath{stroke,fill}%
}%
\begin{pgfscope}%
\pgfsys@transformshift{7.970588in}{0.796591in}%
\pgfsys@useobject{currentmarker}{}%
\end{pgfscope}%
\end{pgfscope}%
\begin{pgfscope}%
\definecolor{textcolor}{rgb}{0.000000,0.000000,0.000000}%
\pgfsetstrokecolor{textcolor}%
\pgfsetfillcolor{textcolor}%
\pgftext[x=7.731981in, y=0.712173in, left, base]{\color{textcolor}\sffamily\fontsize{16.000000}{19.200000}\selectfont 0}%
\end{pgfscope}%
\begin{pgfscope}%
\pgfsetbuttcap%
\pgfsetroundjoin%
\definecolor{currentfill}{rgb}{0.000000,0.000000,0.000000}%
\pgfsetfillcolor{currentfill}%
\pgfsetlinewidth{0.803000pt}%
\definecolor{currentstroke}{rgb}{0.000000,0.000000,0.000000}%
\pgfsetstrokecolor{currentstroke}%
\pgfsetdash{}{0pt}%
\pgfsys@defobject{currentmarker}{\pgfqpoint{-0.048611in}{0.000000in}}{\pgfqpoint{-0.000000in}{0.000000in}}{%
\pgfpathmoveto{\pgfqpoint{-0.000000in}{0.000000in}}%
\pgfpathlineto{\pgfqpoint{-0.048611in}{0.000000in}}%
\pgfusepath{stroke,fill}%
}%
\begin{pgfscope}%
\pgfsys@transformshift{7.970588in}{1.372399in}%
\pgfsys@useobject{currentmarker}{}%
\end{pgfscope}%
\end{pgfscope}%
\begin{pgfscope}%
\definecolor{textcolor}{rgb}{0.000000,0.000000,0.000000}%
\pgfsetstrokecolor{textcolor}%
\pgfsetfillcolor{textcolor}%
\pgftext[x=7.590597in, y=1.287981in, left, base]{\color{textcolor}\sffamily\fontsize{16.000000}{19.200000}\selectfont 50}%
\end{pgfscope}%
\begin{pgfscope}%
\pgfsetbuttcap%
\pgfsetroundjoin%
\definecolor{currentfill}{rgb}{0.000000,0.000000,0.000000}%
\pgfsetfillcolor{currentfill}%
\pgfsetlinewidth{0.803000pt}%
\definecolor{currentstroke}{rgb}{0.000000,0.000000,0.000000}%
\pgfsetstrokecolor{currentstroke}%
\pgfsetdash{}{0pt}%
\pgfsys@defobject{currentmarker}{\pgfqpoint{-0.048611in}{0.000000in}}{\pgfqpoint{-0.000000in}{0.000000in}}{%
\pgfpathmoveto{\pgfqpoint{-0.000000in}{0.000000in}}%
\pgfpathlineto{\pgfqpoint{-0.048611in}{0.000000in}}%
\pgfusepath{stroke,fill}%
}%
\begin{pgfscope}%
\pgfsys@transformshift{7.970588in}{1.948208in}%
\pgfsys@useobject{currentmarker}{}%
\end{pgfscope}%
\end{pgfscope}%
\begin{pgfscope}%
\definecolor{textcolor}{rgb}{0.000000,0.000000,0.000000}%
\pgfsetstrokecolor{textcolor}%
\pgfsetfillcolor{textcolor}%
\pgftext[x=7.449212in, y=1.863789in, left, base]{\color{textcolor}\sffamily\fontsize{16.000000}{19.200000}\selectfont 100}%
\end{pgfscope}%
\begin{pgfscope}%
\pgfsetbuttcap%
\pgfsetroundjoin%
\definecolor{currentfill}{rgb}{0.000000,0.000000,0.000000}%
\pgfsetfillcolor{currentfill}%
\pgfsetlinewidth{0.803000pt}%
\definecolor{currentstroke}{rgb}{0.000000,0.000000,0.000000}%
\pgfsetstrokecolor{currentstroke}%
\pgfsetdash{}{0pt}%
\pgfsys@defobject{currentmarker}{\pgfqpoint{-0.048611in}{0.000000in}}{\pgfqpoint{-0.000000in}{0.000000in}}{%
\pgfpathmoveto{\pgfqpoint{-0.000000in}{0.000000in}}%
\pgfpathlineto{\pgfqpoint{-0.048611in}{0.000000in}}%
\pgfusepath{stroke,fill}%
}%
\begin{pgfscope}%
\pgfsys@transformshift{7.970588in}{2.524016in}%
\pgfsys@useobject{currentmarker}{}%
\end{pgfscope}%
\end{pgfscope}%
\begin{pgfscope}%
\definecolor{textcolor}{rgb}{0.000000,0.000000,0.000000}%
\pgfsetstrokecolor{textcolor}%
\pgfsetfillcolor{textcolor}%
\pgftext[x=7.449212in, y=2.439598in, left, base]{\color{textcolor}\sffamily\fontsize{16.000000}{19.200000}\selectfont 150}%
\end{pgfscope}%
\begin{pgfscope}%
\pgfsetbuttcap%
\pgfsetroundjoin%
\definecolor{currentfill}{rgb}{0.000000,0.000000,0.000000}%
\pgfsetfillcolor{currentfill}%
\pgfsetlinewidth{0.803000pt}%
\definecolor{currentstroke}{rgb}{0.000000,0.000000,0.000000}%
\pgfsetstrokecolor{currentstroke}%
\pgfsetdash{}{0pt}%
\pgfsys@defobject{currentmarker}{\pgfqpoint{-0.048611in}{0.000000in}}{\pgfqpoint{-0.000000in}{0.000000in}}{%
\pgfpathmoveto{\pgfqpoint{-0.000000in}{0.000000in}}%
\pgfpathlineto{\pgfqpoint{-0.048611in}{0.000000in}}%
\pgfusepath{stroke,fill}%
}%
\begin{pgfscope}%
\pgfsys@transformshift{7.970588in}{3.099825in}%
\pgfsys@useobject{currentmarker}{}%
\end{pgfscope}%
\end{pgfscope}%
\begin{pgfscope}%
\definecolor{textcolor}{rgb}{0.000000,0.000000,0.000000}%
\pgfsetstrokecolor{textcolor}%
\pgfsetfillcolor{textcolor}%
\pgftext[x=7.449212in, y=3.015406in, left, base]{\color{textcolor}\sffamily\fontsize{16.000000}{19.200000}\selectfont 200}%
\end{pgfscope}%
\begin{pgfscope}%
\pgfsetbuttcap%
\pgfsetroundjoin%
\definecolor{currentfill}{rgb}{0.000000,0.000000,0.000000}%
\pgfsetfillcolor{currentfill}%
\pgfsetlinewidth{0.803000pt}%
\definecolor{currentstroke}{rgb}{0.000000,0.000000,0.000000}%
\pgfsetstrokecolor{currentstroke}%
\pgfsetdash{}{0pt}%
\pgfsys@defobject{currentmarker}{\pgfqpoint{-0.048611in}{0.000000in}}{\pgfqpoint{-0.000000in}{0.000000in}}{%
\pgfpathmoveto{\pgfqpoint{-0.000000in}{0.000000in}}%
\pgfpathlineto{\pgfqpoint{-0.048611in}{0.000000in}}%
\pgfusepath{stroke,fill}%
}%
\begin{pgfscope}%
\pgfsys@transformshift{7.970588in}{3.675633in}%
\pgfsys@useobject{currentmarker}{}%
\end{pgfscope}%
\end{pgfscope}%
\begin{pgfscope}%
\definecolor{textcolor}{rgb}{0.000000,0.000000,0.000000}%
\pgfsetstrokecolor{textcolor}%
\pgfsetfillcolor{textcolor}%
\pgftext[x=7.449212in, y=3.591215in, left, base]{\color{textcolor}\sffamily\fontsize{16.000000}{19.200000}\selectfont 250}%
\end{pgfscope}%
\begin{pgfscope}%
\pgfsetbuttcap%
\pgfsetroundjoin%
\definecolor{currentfill}{rgb}{0.000000,0.000000,0.000000}%
\pgfsetfillcolor{currentfill}%
\pgfsetlinewidth{0.803000pt}%
\definecolor{currentstroke}{rgb}{0.000000,0.000000,0.000000}%
\pgfsetstrokecolor{currentstroke}%
\pgfsetdash{}{0pt}%
\pgfsys@defobject{currentmarker}{\pgfqpoint{-0.048611in}{0.000000in}}{\pgfqpoint{-0.000000in}{0.000000in}}{%
\pgfpathmoveto{\pgfqpoint{-0.000000in}{0.000000in}}%
\pgfpathlineto{\pgfqpoint{-0.048611in}{0.000000in}}%
\pgfusepath{stroke,fill}%
}%
\begin{pgfscope}%
\pgfsys@transformshift{7.970588in}{4.251441in}%
\pgfsys@useobject{currentmarker}{}%
\end{pgfscope}%
\end{pgfscope}%
\begin{pgfscope}%
\definecolor{textcolor}{rgb}{0.000000,0.000000,0.000000}%
\pgfsetstrokecolor{textcolor}%
\pgfsetfillcolor{textcolor}%
\pgftext[x=7.449212in, y=4.167023in, left, base]{\color{textcolor}\sffamily\fontsize{16.000000}{19.200000}\selectfont 300}%
\end{pgfscope}%
\begin{pgfscope}%
\definecolor{textcolor}{rgb}{0.000000,0.000000,0.000000}%
\pgfsetstrokecolor{textcolor}%
\pgfsetfillcolor{textcolor}%
\pgftext[x=7.393657in,y=2.512500in,,bottom,rotate=90.000000]{\color{textcolor}\sffamily\fontsize{20.000000}{24.000000}\selectfont y}%
\end{pgfscope}%
\begin{pgfscope}%
\pgfpathrectangle{\pgfqpoint{7.970588in}{0.625000in}}{\pgfqpoint{4.558824in}{3.775000in}}%
\pgfusepath{clip}%
\pgfsetrectcap%
\pgfsetroundjoin%
\pgfsetlinewidth{1.505625pt}%
\definecolor{currentstroke}{rgb}{1.000000,0.000000,0.000000}%
\pgfsetstrokecolor{currentstroke}%
\pgfsetdash{}{0pt}%
\pgfpathmoveto{\pgfqpoint{8.177807in}{0.796591in}}%
\pgfpathlineto{\pgfqpoint{8.504863in}{0.865688in}}%
\pgfpathlineto{\pgfqpoint{8.818478in}{0.934785in}}%
\pgfpathlineto{\pgfqpoint{9.118653in}{1.003882in}}%
\pgfpathlineto{\pgfqpoint{9.405386in}{1.072979in}}%
\pgfpathlineto{\pgfqpoint{9.678679in}{1.142076in}}%
\pgfpathlineto{\pgfqpoint{9.938532in}{1.211173in}}%
\pgfpathlineto{\pgfqpoint{10.184944in}{1.280270in}}%
\pgfpathlineto{\pgfqpoint{10.380020in}{1.337851in}}%
\pgfpathlineto{\pgfqpoint{10.565762in}{1.395432in}}%
\pgfpathlineto{\pgfqpoint{10.742170in}{1.453013in}}%
\pgfpathlineto{\pgfqpoint{10.909245in}{1.510593in}}%
\pgfpathlineto{\pgfqpoint{11.066986in}{1.568174in}}%
\pgfpathlineto{\pgfqpoint{11.215393in}{1.625755in}}%
\pgfpathlineto{\pgfqpoint{11.354466in}{1.683336in}}%
\pgfpathlineto{\pgfqpoint{11.484206in}{1.740917in}}%
\pgfpathlineto{\pgfqpoint{11.581277in}{1.786981in}}%
\pgfpathlineto{\pgfqpoint{11.672375in}{1.833046in}}%
\pgfpathlineto{\pgfqpoint{11.757499in}{1.879111in}}%
\pgfpathlineto{\pgfqpoint{11.836649in}{1.925175in}}%
\pgfpathlineto{\pgfqpoint{11.909826in}{1.971240in}}%
\pgfpathlineto{\pgfqpoint{11.977029in}{2.017305in}}%
\pgfpathlineto{\pgfqpoint{12.038259in}{2.063369in}}%
\pgfpathlineto{\pgfqpoint{12.093515in}{2.109434in}}%
\pgfpathlineto{\pgfqpoint{12.142797in}{2.155499in}}%
\pgfpathlineto{\pgfqpoint{12.175839in}{2.190047in}}%
\pgfpathlineto{\pgfqpoint{12.205520in}{2.224596in}}%
\pgfpathlineto{\pgfqpoint{12.231842in}{2.259144in}}%
\pgfpathlineto{\pgfqpoint{12.254803in}{2.293693in}}%
\pgfpathlineto{\pgfqpoint{12.274404in}{2.328241in}}%
\pgfpathlineto{\pgfqpoint{12.290644in}{2.362790in}}%
\pgfpathlineto{\pgfqpoint{12.303525in}{2.397338in}}%
\pgfpathlineto{\pgfqpoint{12.313045in}{2.431887in}}%
\pgfpathlineto{\pgfqpoint{12.319206in}{2.466435in}}%
\pgfpathlineto{\pgfqpoint{12.322006in}{2.500984in}}%
\pgfpathlineto{\pgfqpoint{12.321446in}{2.535532in}}%
\pgfpathlineto{\pgfqpoint{12.317526in}{2.570081in}}%
\pgfpathlineto{\pgfqpoint{12.310245in}{2.604629in}}%
\pgfpathlineto{\pgfqpoint{12.299605in}{2.639178in}}%
\pgfpathlineto{\pgfqpoint{12.285604in}{2.673726in}}%
\pgfpathlineto{\pgfqpoint{12.268243in}{2.708275in}}%
\pgfpathlineto{\pgfqpoint{12.247522in}{2.742823in}}%
\pgfpathlineto{\pgfqpoint{12.223441in}{2.777372in}}%
\pgfpathlineto{\pgfqpoint{12.196000in}{2.811920in}}%
\pgfpathlineto{\pgfqpoint{12.165198in}{2.846469in}}%
\pgfpathlineto{\pgfqpoint{12.131037in}{2.881017in}}%
\pgfpathlineto{\pgfqpoint{12.080261in}{2.927082in}}%
\pgfpathlineto{\pgfqpoint{12.023512in}{2.973147in}}%
\pgfpathlineto{\pgfqpoint{11.960789in}{3.019211in}}%
\pgfpathlineto{\pgfqpoint{11.892092in}{3.065276in}}%
\pgfpathlineto{\pgfqpoint{11.817422in}{3.111341in}}%
\pgfpathlineto{\pgfqpoint{11.736778in}{3.157405in}}%
\pgfpathlineto{\pgfqpoint{11.650160in}{3.203470in}}%
\pgfpathlineto{\pgfqpoint{11.557569in}{3.249535in}}%
\pgfpathlineto{\pgfqpoint{11.459005in}{3.295599in}}%
\pgfpathlineto{\pgfqpoint{11.327398in}{3.353180in}}%
\pgfpathlineto{\pgfqpoint{11.186458in}{3.410761in}}%
\pgfpathlineto{\pgfqpoint{11.036184in}{3.468342in}}%
\pgfpathlineto{\pgfqpoint{10.876577in}{3.525923in}}%
\pgfpathlineto{\pgfqpoint{10.707635in}{3.583504in}}%
\pgfpathlineto{\pgfqpoint{10.529360in}{3.641085in}}%
\pgfpathlineto{\pgfqpoint{10.341751in}{3.698665in}}%
\pgfpathlineto{\pgfqpoint{10.144808in}{3.756246in}}%
\pgfpathlineto{\pgfqpoint{9.896156in}{3.825343in}}%
\pgfpathlineto{\pgfqpoint{9.634064in}{3.894440in}}%
\pgfpathlineto{\pgfqpoint{9.358531in}{3.963537in}}%
\pgfpathlineto{\pgfqpoint{9.069557in}{4.032634in}}%
\pgfpathlineto{\pgfqpoint{8.767142in}{4.101731in}}%
\pgfpathlineto{\pgfqpoint{8.451287in}{4.170828in}}%
\pgfpathlineto{\pgfqpoint{8.177807in}{4.228409in}}%
\pgfpathlineto{\pgfqpoint{8.177807in}{4.228409in}}%
\pgfusepath{stroke}%
\end{pgfscope}%
\begin{pgfscope}%
\pgfpathrectangle{\pgfqpoint{7.970588in}{0.625000in}}{\pgfqpoint{4.558824in}{3.775000in}}%
\pgfusepath{clip}%
\pgfsetrectcap%
\pgfsetroundjoin%
\pgfsetlinewidth{1.505625pt}%
\definecolor{currentstroke}{rgb}{0.000000,0.000000,1.000000}%
\pgfsetstrokecolor{currentstroke}%
\pgfsetdash{}{0pt}%
\pgfpathmoveto{\pgfqpoint{8.198377in}{0.802349in}}%
\pgfpathlineto{\pgfqpoint{8.434188in}{0.871446in}}%
\pgfpathlineto{\pgfqpoint{8.657986in}{0.940543in}}%
\pgfpathlineto{\pgfqpoint{8.869839in}{1.009640in}}%
\pgfpathlineto{\pgfqpoint{9.037341in}{1.067221in}}%
\pgfpathlineto{\pgfqpoint{9.196731in}{1.124802in}}%
\pgfpathlineto{\pgfqpoint{9.348134in}{1.182383in}}%
\pgfpathlineto{\pgfqpoint{9.491704in}{1.239963in}}%
\pgfpathlineto{\pgfqpoint{9.627616in}{1.297544in}}%
\pgfpathlineto{\pgfqpoint{9.756059in}{1.355125in}}%
\pgfpathlineto{\pgfqpoint{9.877229in}{1.412706in}}%
\pgfpathlineto{\pgfqpoint{9.991321in}{1.470287in}}%
\pgfpathlineto{\pgfqpoint{10.098518in}{1.527868in}}%
\pgfpathlineto{\pgfqpoint{10.198988in}{1.585448in}}%
\pgfpathlineto{\pgfqpoint{10.292882in}{1.643029in}}%
\pgfpathlineto{\pgfqpoint{10.380328in}{1.700610in}}%
\pgfpathlineto{\pgfqpoint{10.461435in}{1.758191in}}%
\pgfpathlineto{\pgfqpoint{10.536296in}{1.815772in}}%
\pgfpathlineto{\pgfqpoint{10.591741in}{1.861836in}}%
\pgfpathlineto{\pgfqpoint{10.643273in}{1.907901in}}%
\pgfpathlineto{\pgfqpoint{10.690922in}{1.953966in}}%
\pgfpathlineto{\pgfqpoint{10.734714in}{2.000031in}}%
\pgfpathlineto{\pgfqpoint{10.774673in}{2.046095in}}%
\pgfpathlineto{\pgfqpoint{10.810817in}{2.092160in}}%
\pgfpathlineto{\pgfqpoint{10.843165in}{2.138225in}}%
\pgfpathlineto{\pgfqpoint{10.871730in}{2.184289in}}%
\pgfpathlineto{\pgfqpoint{10.896526in}{2.230354in}}%
\pgfpathlineto{\pgfqpoint{10.917562in}{2.276419in}}%
\pgfpathlineto{\pgfqpoint{10.934847in}{2.322483in}}%
\pgfpathlineto{\pgfqpoint{10.948389in}{2.368548in}}%
\pgfpathlineto{\pgfqpoint{10.958190in}{2.414613in}}%
\pgfpathlineto{\pgfqpoint{10.964257in}{2.460677in}}%
\pgfpathlineto{\pgfqpoint{10.966590in}{2.506742in}}%
\pgfpathlineto{\pgfqpoint{10.965190in}{2.552807in}}%
\pgfpathlineto{\pgfqpoint{10.960057in}{2.598871in}}%
\pgfpathlineto{\pgfqpoint{10.951189in}{2.644936in}}%
\pgfpathlineto{\pgfqpoint{10.938584in}{2.691001in}}%
\pgfpathlineto{\pgfqpoint{10.922235in}{2.737065in}}%
\pgfpathlineto{\pgfqpoint{10.902137in}{2.783130in}}%
\pgfpathlineto{\pgfqpoint{10.878282in}{2.829195in}}%
\pgfpathlineto{\pgfqpoint{10.850660in}{2.875259in}}%
\pgfpathlineto{\pgfqpoint{10.819259in}{2.921324in}}%
\pgfpathlineto{\pgfqpoint{10.784066in}{2.967389in}}%
\pgfpathlineto{\pgfqpoint{10.745063in}{3.013453in}}%
\pgfpathlineto{\pgfqpoint{10.702231in}{3.059518in}}%
\pgfpathlineto{\pgfqpoint{10.655548in}{3.105583in}}%
\pgfpathlineto{\pgfqpoint{10.604989in}{3.151647in}}%
\pgfpathlineto{\pgfqpoint{10.550526in}{3.197712in}}%
\pgfpathlineto{\pgfqpoint{10.492124in}{3.243777in}}%
\pgfpathlineto{\pgfqpoint{10.413525in}{3.301358in}}%
\pgfpathlineto{\pgfqpoint{10.328627in}{3.358938in}}%
\pgfpathlineto{\pgfqpoint{10.237327in}{3.416519in}}%
\pgfpathlineto{\pgfqpoint{10.139504in}{3.474100in}}%
\pgfpathlineto{\pgfqpoint{10.035017in}{3.531681in}}%
\pgfpathlineto{\pgfqpoint{9.923704in}{3.589262in}}%
\pgfpathlineto{\pgfqpoint{9.805388in}{3.646843in}}%
\pgfpathlineto{\pgfqpoint{9.679878in}{3.704423in}}%
\pgfpathlineto{\pgfqpoint{9.546977in}{3.762004in}}%
\pgfpathlineto{\pgfqpoint{9.406492in}{3.819585in}}%
\pgfpathlineto{\pgfqpoint{9.258242in}{3.877166in}}%
\pgfpathlineto{\pgfqpoint{9.102064in}{3.934747in}}%
\pgfpathlineto{\pgfqpoint{8.937820in}{3.992328in}}%
\pgfpathlineto{\pgfqpoint{8.765400in}{4.049908in}}%
\pgfpathlineto{\pgfqpoint{8.547586in}{4.119005in}}%
\pgfpathlineto{\pgfqpoint{8.317786in}{4.188103in}}%
\pgfpathlineto{\pgfqpoint{8.198377in}{4.222651in}}%
\pgfpathlineto{\pgfqpoint{8.198377in}{4.222651in}}%
\pgfusepath{stroke}%
\end{pgfscope}%
\begin{pgfscope}%
\pgfsetrectcap%
\pgfsetmiterjoin%
\pgfsetlinewidth{0.803000pt}%
\definecolor{currentstroke}{rgb}{0.000000,0.000000,0.000000}%
\pgfsetstrokecolor{currentstroke}%
\pgfsetdash{}{0pt}%
\pgfpathmoveto{\pgfqpoint{7.970588in}{0.625000in}}%
\pgfpathlineto{\pgfqpoint{7.970588in}{4.400000in}}%
\pgfusepath{stroke}%
\end{pgfscope}%
\begin{pgfscope}%
\pgfsetrectcap%
\pgfsetmiterjoin%
\pgfsetlinewidth{0.803000pt}%
\definecolor{currentstroke}{rgb}{0.000000,0.000000,0.000000}%
\pgfsetstrokecolor{currentstroke}%
\pgfsetdash{}{0pt}%
\pgfpathmoveto{\pgfqpoint{12.529412in}{0.625000in}}%
\pgfpathlineto{\pgfqpoint{12.529412in}{4.400000in}}%
\pgfusepath{stroke}%
\end{pgfscope}%
\begin{pgfscope}%
\pgfsetrectcap%
\pgfsetmiterjoin%
\pgfsetlinewidth{0.803000pt}%
\definecolor{currentstroke}{rgb}{0.000000,0.000000,0.000000}%
\pgfsetstrokecolor{currentstroke}%
\pgfsetdash{}{0pt}%
\pgfpathmoveto{\pgfqpoint{7.970588in}{0.625000in}}%
\pgfpathlineto{\pgfqpoint{12.529412in}{0.625000in}}%
\pgfusepath{stroke}%
\end{pgfscope}%
\begin{pgfscope}%
\pgfsetrectcap%
\pgfsetmiterjoin%
\pgfsetlinewidth{0.803000pt}%
\definecolor{currentstroke}{rgb}{0.000000,0.000000,0.000000}%
\pgfsetstrokecolor{currentstroke}%
\pgfsetdash{}{0pt}%
\pgfpathmoveto{\pgfqpoint{7.970588in}{4.400000in}}%
\pgfpathlineto{\pgfqpoint{12.529412in}{4.400000in}}%
\pgfusepath{stroke}%
\end{pgfscope}%
\begin{pgfscope}%
\definecolor{textcolor}{rgb}{0.000000,0.000000,0.000000}%
\pgfsetstrokecolor{textcolor}%
\pgfsetfillcolor{textcolor}%
\pgftext[x=10.250000in,y=4.483333in,,base]{\color{textcolor}\sffamily\fontsize{20.000000}{24.000000}\selectfont \(\displaystyle L_x=300\) and \(\displaystyle L_y=300\)}%
\end{pgfscope}%
\begin{pgfscope}%
\pgfsetbuttcap%
\pgfsetmiterjoin%
\definecolor{currentfill}{rgb}{1.000000,1.000000,1.000000}%
\pgfsetfillcolor{currentfill}%
\pgfsetfillopacity{0.800000}%
\pgfsetlinewidth{1.003750pt}%
\definecolor{currentstroke}{rgb}{0.800000,0.800000,0.800000}%
\pgfsetstrokecolor{currentstroke}%
\pgfsetstrokeopacity{0.800000}%
\pgfsetdash{}{0pt}%
\pgfpathmoveto{\pgfqpoint{8.126144in}{2.152995in}}%
\pgfpathlineto{\pgfqpoint{10.508370in}{2.152995in}}%
\pgfpathquadraticcurveto{\pgfqpoint{10.552815in}{2.152995in}}{\pgfqpoint{10.552815in}{2.197439in}}%
\pgfpathlineto{\pgfqpoint{10.552815in}{2.827561in}}%
\pgfpathquadraticcurveto{\pgfqpoint{10.552815in}{2.872005in}}{\pgfqpoint{10.508370in}{2.872005in}}%
\pgfpathlineto{\pgfqpoint{8.126144in}{2.872005in}}%
\pgfpathquadraticcurveto{\pgfqpoint{8.081699in}{2.872005in}}{\pgfqpoint{8.081699in}{2.827561in}}%
\pgfpathlineto{\pgfqpoint{8.081699in}{2.197439in}}%
\pgfpathquadraticcurveto{\pgfqpoint{8.081699in}{2.152995in}}{\pgfqpoint{8.126144in}{2.152995in}}%
\pgfpathlineto{\pgfqpoint{8.126144in}{2.152995in}}%
\pgfpathclose%
\pgfusepath{stroke,fill}%
\end{pgfscope}%
\begin{pgfscope}%
\pgfsetrectcap%
\pgfsetroundjoin%
\pgfsetlinewidth{1.505625pt}%
\definecolor{currentstroke}{rgb}{1.000000,0.000000,0.000000}%
\pgfsetstrokecolor{currentstroke}%
\pgfsetdash{}{0pt}%
\pgfpathmoveto{\pgfqpoint{8.170588in}{2.692057in}}%
\pgfpathlineto{\pgfqpoint{8.392810in}{2.692057in}}%
\pgfpathlineto{\pgfqpoint{8.615033in}{2.692057in}}%
\pgfusepath{stroke}%
\end{pgfscope}%
\begin{pgfscope}%
\definecolor{textcolor}{rgb}{0.000000,0.000000,0.000000}%
\pgfsetstrokecolor{textcolor}%
\pgfsetfillcolor{textcolor}%
\pgftext[x=8.792810in,y=2.614279in,left,base]{\color{textcolor}\sffamily\fontsize{16.000000}{19.200000}\selectfont Theo velocities}%
\end{pgfscope}%
\begin{pgfscope}%
\pgfsetrectcap%
\pgfsetroundjoin%
\pgfsetlinewidth{1.505625pt}%
\definecolor{currentstroke}{rgb}{0.000000,0.000000,1.000000}%
\pgfsetstrokecolor{currentstroke}%
\pgfsetdash{}{0pt}%
\pgfpathmoveto{\pgfqpoint{8.170588in}{2.365885in}}%
\pgfpathlineto{\pgfqpoint{8.392810in}{2.365885in}}%
\pgfpathlineto{\pgfqpoint{8.615033in}{2.365885in}}%
\pgfusepath{stroke}%
\end{pgfscope}%
\begin{pgfscope}%
\definecolor{textcolor}{rgb}{0.000000,0.000000,0.000000}%
\pgfsetstrokecolor{textcolor}%
\pgfsetfillcolor{textcolor}%
\pgftext[x=8.792810in,y=2.288108in,left,base]{\color{textcolor}\sffamily\fontsize{16.000000}{19.200000}\selectfont Num velocities}%
\end{pgfscope}%
\begin{pgfscope}%
\pgfsetbuttcap%
\pgfsetmiterjoin%
\definecolor{currentfill}{rgb}{1.000000,1.000000,1.000000}%
\pgfsetfillcolor{currentfill}%
\pgfsetlinewidth{0.000000pt}%
\definecolor{currentstroke}{rgb}{0.000000,0.000000,0.000000}%
\pgfsetstrokecolor{currentstroke}%
\pgfsetstrokeopacity{0.000000}%
\pgfsetdash{}{0pt}%
\pgfpathmoveto{\pgfqpoint{13.441176in}{0.625000in}}%
\pgfpathlineto{\pgfqpoint{18.000000in}{0.625000in}}%
\pgfpathlineto{\pgfqpoint{18.000000in}{4.400000in}}%
\pgfpathlineto{\pgfqpoint{13.441176in}{4.400000in}}%
\pgfpathlineto{\pgfqpoint{13.441176in}{0.625000in}}%
\pgfpathclose%
\pgfusepath{fill}%
\end{pgfscope}%
\begin{pgfscope}%
\pgfsetbuttcap%
\pgfsetroundjoin%
\definecolor{currentfill}{rgb}{0.000000,0.000000,0.000000}%
\pgfsetfillcolor{currentfill}%
\pgfsetlinewidth{0.803000pt}%
\definecolor{currentstroke}{rgb}{0.000000,0.000000,0.000000}%
\pgfsetstrokecolor{currentstroke}%
\pgfsetdash{}{0pt}%
\pgfsys@defobject{currentmarker}{\pgfqpoint{0.000000in}{-0.048611in}}{\pgfqpoint{0.000000in}{0.000000in}}{%
\pgfpathmoveto{\pgfqpoint{0.000000in}{0.000000in}}%
\pgfpathlineto{\pgfqpoint{0.000000in}{-0.048611in}}%
\pgfusepath{stroke,fill}%
}%
\begin{pgfscope}%
\pgfsys@transformshift{13.648396in}{0.625000in}%
\pgfsys@useobject{currentmarker}{}%
\end{pgfscope}%
\end{pgfscope}%
\begin{pgfscope}%
\definecolor{textcolor}{rgb}{0.000000,0.000000,0.000000}%
\pgfsetstrokecolor{textcolor}%
\pgfsetfillcolor{textcolor}%
\pgftext[x=13.648396in,y=0.527778in,,top]{\color{textcolor}\sffamily\fontsize{16.000000}{19.200000}\selectfont 0.00}%
\end{pgfscope}%
\begin{pgfscope}%
\pgfsetbuttcap%
\pgfsetroundjoin%
\definecolor{currentfill}{rgb}{0.000000,0.000000,0.000000}%
\pgfsetfillcolor{currentfill}%
\pgfsetlinewidth{0.803000pt}%
\definecolor{currentstroke}{rgb}{0.000000,0.000000,0.000000}%
\pgfsetstrokecolor{currentstroke}%
\pgfsetdash{}{0pt}%
\pgfsys@defobject{currentmarker}{\pgfqpoint{0.000000in}{-0.048611in}}{\pgfqpoint{0.000000in}{0.000000in}}{%
\pgfpathmoveto{\pgfqpoint{0.000000in}{0.000000in}}%
\pgfpathlineto{\pgfqpoint{0.000000in}{-0.048611in}}%
\pgfusepath{stroke,fill}%
}%
\begin{pgfscope}%
\pgfsys@transformshift{14.727663in}{0.625000in}%
\pgfsys@useobject{currentmarker}{}%
\end{pgfscope}%
\end{pgfscope}%
\begin{pgfscope}%
\definecolor{textcolor}{rgb}{0.000000,0.000000,0.000000}%
\pgfsetstrokecolor{textcolor}%
\pgfsetfillcolor{textcolor}%
\pgftext[x=14.727663in,y=0.527778in,,top]{\color{textcolor}\sffamily\fontsize{16.000000}{19.200000}\selectfont 0.02}%
\end{pgfscope}%
\begin{pgfscope}%
\pgfsetbuttcap%
\pgfsetroundjoin%
\definecolor{currentfill}{rgb}{0.000000,0.000000,0.000000}%
\pgfsetfillcolor{currentfill}%
\pgfsetlinewidth{0.803000pt}%
\definecolor{currentstroke}{rgb}{0.000000,0.000000,0.000000}%
\pgfsetstrokecolor{currentstroke}%
\pgfsetdash{}{0pt}%
\pgfsys@defobject{currentmarker}{\pgfqpoint{0.000000in}{-0.048611in}}{\pgfqpoint{0.000000in}{0.000000in}}{%
\pgfpathmoveto{\pgfqpoint{0.000000in}{0.000000in}}%
\pgfpathlineto{\pgfqpoint{0.000000in}{-0.048611in}}%
\pgfusepath{stroke,fill}%
}%
\begin{pgfscope}%
\pgfsys@transformshift{15.806930in}{0.625000in}%
\pgfsys@useobject{currentmarker}{}%
\end{pgfscope}%
\end{pgfscope}%
\begin{pgfscope}%
\definecolor{textcolor}{rgb}{0.000000,0.000000,0.000000}%
\pgfsetstrokecolor{textcolor}%
\pgfsetfillcolor{textcolor}%
\pgftext[x=15.806930in,y=0.527778in,,top]{\color{textcolor}\sffamily\fontsize{16.000000}{19.200000}\selectfont 0.04}%
\end{pgfscope}%
\begin{pgfscope}%
\pgfsetbuttcap%
\pgfsetroundjoin%
\definecolor{currentfill}{rgb}{0.000000,0.000000,0.000000}%
\pgfsetfillcolor{currentfill}%
\pgfsetlinewidth{0.803000pt}%
\definecolor{currentstroke}{rgb}{0.000000,0.000000,0.000000}%
\pgfsetstrokecolor{currentstroke}%
\pgfsetdash{}{0pt}%
\pgfsys@defobject{currentmarker}{\pgfqpoint{0.000000in}{-0.048611in}}{\pgfqpoint{0.000000in}{0.000000in}}{%
\pgfpathmoveto{\pgfqpoint{0.000000in}{0.000000in}}%
\pgfpathlineto{\pgfqpoint{0.000000in}{-0.048611in}}%
\pgfusepath{stroke,fill}%
}%
\begin{pgfscope}%
\pgfsys@transformshift{16.886197in}{0.625000in}%
\pgfsys@useobject{currentmarker}{}%
\end{pgfscope}%
\end{pgfscope}%
\begin{pgfscope}%
\definecolor{textcolor}{rgb}{0.000000,0.000000,0.000000}%
\pgfsetstrokecolor{textcolor}%
\pgfsetfillcolor{textcolor}%
\pgftext[x=16.886197in,y=0.527778in,,top]{\color{textcolor}\sffamily\fontsize{16.000000}{19.200000}\selectfont 0.06}%
\end{pgfscope}%
\begin{pgfscope}%
\pgfsetbuttcap%
\pgfsetroundjoin%
\definecolor{currentfill}{rgb}{0.000000,0.000000,0.000000}%
\pgfsetfillcolor{currentfill}%
\pgfsetlinewidth{0.803000pt}%
\definecolor{currentstroke}{rgb}{0.000000,0.000000,0.000000}%
\pgfsetstrokecolor{currentstroke}%
\pgfsetdash{}{0pt}%
\pgfsys@defobject{currentmarker}{\pgfqpoint{0.000000in}{-0.048611in}}{\pgfqpoint{0.000000in}{0.000000in}}{%
\pgfpathmoveto{\pgfqpoint{0.000000in}{0.000000in}}%
\pgfpathlineto{\pgfqpoint{0.000000in}{-0.048611in}}%
\pgfusepath{stroke,fill}%
}%
\begin{pgfscope}%
\pgfsys@transformshift{17.965463in}{0.625000in}%
\pgfsys@useobject{currentmarker}{}%
\end{pgfscope}%
\end{pgfscope}%
\begin{pgfscope}%
\definecolor{textcolor}{rgb}{0.000000,0.000000,0.000000}%
\pgfsetstrokecolor{textcolor}%
\pgfsetfillcolor{textcolor}%
\pgftext[x=17.965463in,y=0.527778in,,top]{\color{textcolor}\sffamily\fontsize{16.000000}{19.200000}\selectfont 0.08}%
\end{pgfscope}%
\begin{pgfscope}%
\definecolor{textcolor}{rgb}{0.000000,0.000000,0.000000}%
\pgfsetstrokecolor{textcolor}%
\pgfsetfillcolor{textcolor}%
\pgftext[x=15.720588in,y=0.257162in,,top]{\color{textcolor}\sffamily\fontsize{20.000000}{24.000000}\selectfont \(\displaystyle u_{x}\)}%
\end{pgfscope}%
\begin{pgfscope}%
\pgfsetbuttcap%
\pgfsetroundjoin%
\definecolor{currentfill}{rgb}{0.000000,0.000000,0.000000}%
\pgfsetfillcolor{currentfill}%
\pgfsetlinewidth{0.803000pt}%
\definecolor{currentstroke}{rgb}{0.000000,0.000000,0.000000}%
\pgfsetstrokecolor{currentstroke}%
\pgfsetdash{}{0pt}%
\pgfsys@defobject{currentmarker}{\pgfqpoint{-0.048611in}{0.000000in}}{\pgfqpoint{-0.000000in}{0.000000in}}{%
\pgfpathmoveto{\pgfqpoint{-0.000000in}{0.000000in}}%
\pgfpathlineto{\pgfqpoint{-0.048611in}{0.000000in}}%
\pgfusepath{stroke,fill}%
}%
\begin{pgfscope}%
\pgfsys@transformshift{13.441176in}{0.796591in}%
\pgfsys@useobject{currentmarker}{}%
\end{pgfscope}%
\end{pgfscope}%
\begin{pgfscope}%
\definecolor{textcolor}{rgb}{0.000000,0.000000,0.000000}%
\pgfsetstrokecolor{textcolor}%
\pgfsetfillcolor{textcolor}%
\pgftext[x=13.202570in, y=0.712173in, left, base]{\color{textcolor}\sffamily\fontsize{16.000000}{19.200000}\selectfont 0}%
\end{pgfscope}%
\begin{pgfscope}%
\pgfsetbuttcap%
\pgfsetroundjoin%
\definecolor{currentfill}{rgb}{0.000000,0.000000,0.000000}%
\pgfsetfillcolor{currentfill}%
\pgfsetlinewidth{0.803000pt}%
\definecolor{currentstroke}{rgb}{0.000000,0.000000,0.000000}%
\pgfsetstrokecolor{currentstroke}%
\pgfsetdash{}{0pt}%
\pgfsys@defobject{currentmarker}{\pgfqpoint{-0.048611in}{0.000000in}}{\pgfqpoint{-0.000000in}{0.000000in}}{%
\pgfpathmoveto{\pgfqpoint{-0.000000in}{0.000000in}}%
\pgfpathlineto{\pgfqpoint{-0.048611in}{0.000000in}}%
\pgfusepath{stroke,fill}%
}%
\begin{pgfscope}%
\pgfsys@transformshift{13.441176in}{1.511553in}%
\pgfsys@useobject{currentmarker}{}%
\end{pgfscope}%
\end{pgfscope}%
\begin{pgfscope}%
\definecolor{textcolor}{rgb}{0.000000,0.000000,0.000000}%
\pgfsetstrokecolor{textcolor}%
\pgfsetfillcolor{textcolor}%
\pgftext[x=13.061185in, y=1.427135in, left, base]{\color{textcolor}\sffamily\fontsize{16.000000}{19.200000}\selectfont 10}%
\end{pgfscope}%
\begin{pgfscope}%
\pgfsetbuttcap%
\pgfsetroundjoin%
\definecolor{currentfill}{rgb}{0.000000,0.000000,0.000000}%
\pgfsetfillcolor{currentfill}%
\pgfsetlinewidth{0.803000pt}%
\definecolor{currentstroke}{rgb}{0.000000,0.000000,0.000000}%
\pgfsetstrokecolor{currentstroke}%
\pgfsetdash{}{0pt}%
\pgfsys@defobject{currentmarker}{\pgfqpoint{-0.048611in}{0.000000in}}{\pgfqpoint{-0.000000in}{0.000000in}}{%
\pgfpathmoveto{\pgfqpoint{-0.000000in}{0.000000in}}%
\pgfpathlineto{\pgfqpoint{-0.048611in}{0.000000in}}%
\pgfusepath{stroke,fill}%
}%
\begin{pgfscope}%
\pgfsys@transformshift{13.441176in}{2.226515in}%
\pgfsys@useobject{currentmarker}{}%
\end{pgfscope}%
\end{pgfscope}%
\begin{pgfscope}%
\definecolor{textcolor}{rgb}{0.000000,0.000000,0.000000}%
\pgfsetstrokecolor{textcolor}%
\pgfsetfillcolor{textcolor}%
\pgftext[x=13.061185in, y=2.142097in, left, base]{\color{textcolor}\sffamily\fontsize{16.000000}{19.200000}\selectfont 20}%
\end{pgfscope}%
\begin{pgfscope}%
\pgfsetbuttcap%
\pgfsetroundjoin%
\definecolor{currentfill}{rgb}{0.000000,0.000000,0.000000}%
\pgfsetfillcolor{currentfill}%
\pgfsetlinewidth{0.803000pt}%
\definecolor{currentstroke}{rgb}{0.000000,0.000000,0.000000}%
\pgfsetstrokecolor{currentstroke}%
\pgfsetdash{}{0pt}%
\pgfsys@defobject{currentmarker}{\pgfqpoint{-0.048611in}{0.000000in}}{\pgfqpoint{-0.000000in}{0.000000in}}{%
\pgfpathmoveto{\pgfqpoint{-0.000000in}{0.000000in}}%
\pgfpathlineto{\pgfqpoint{-0.048611in}{0.000000in}}%
\pgfusepath{stroke,fill}%
}%
\begin{pgfscope}%
\pgfsys@transformshift{13.441176in}{2.941477in}%
\pgfsys@useobject{currentmarker}{}%
\end{pgfscope}%
\end{pgfscope}%
\begin{pgfscope}%
\definecolor{textcolor}{rgb}{0.000000,0.000000,0.000000}%
\pgfsetstrokecolor{textcolor}%
\pgfsetfillcolor{textcolor}%
\pgftext[x=13.061185in, y=2.857059in, left, base]{\color{textcolor}\sffamily\fontsize{16.000000}{19.200000}\selectfont 30}%
\end{pgfscope}%
\begin{pgfscope}%
\pgfsetbuttcap%
\pgfsetroundjoin%
\definecolor{currentfill}{rgb}{0.000000,0.000000,0.000000}%
\pgfsetfillcolor{currentfill}%
\pgfsetlinewidth{0.803000pt}%
\definecolor{currentstroke}{rgb}{0.000000,0.000000,0.000000}%
\pgfsetstrokecolor{currentstroke}%
\pgfsetdash{}{0pt}%
\pgfsys@defobject{currentmarker}{\pgfqpoint{-0.048611in}{0.000000in}}{\pgfqpoint{-0.000000in}{0.000000in}}{%
\pgfpathmoveto{\pgfqpoint{-0.000000in}{0.000000in}}%
\pgfpathlineto{\pgfqpoint{-0.048611in}{0.000000in}}%
\pgfusepath{stroke,fill}%
}%
\begin{pgfscope}%
\pgfsys@transformshift{13.441176in}{3.656439in}%
\pgfsys@useobject{currentmarker}{}%
\end{pgfscope}%
\end{pgfscope}%
\begin{pgfscope}%
\definecolor{textcolor}{rgb}{0.000000,0.000000,0.000000}%
\pgfsetstrokecolor{textcolor}%
\pgfsetfillcolor{textcolor}%
\pgftext[x=13.061185in, y=3.572021in, left, base]{\color{textcolor}\sffamily\fontsize{16.000000}{19.200000}\selectfont 40}%
\end{pgfscope}%
\begin{pgfscope}%
\pgfsetbuttcap%
\pgfsetroundjoin%
\definecolor{currentfill}{rgb}{0.000000,0.000000,0.000000}%
\pgfsetfillcolor{currentfill}%
\pgfsetlinewidth{0.803000pt}%
\definecolor{currentstroke}{rgb}{0.000000,0.000000,0.000000}%
\pgfsetstrokecolor{currentstroke}%
\pgfsetdash{}{0pt}%
\pgfsys@defobject{currentmarker}{\pgfqpoint{-0.048611in}{0.000000in}}{\pgfqpoint{-0.000000in}{0.000000in}}{%
\pgfpathmoveto{\pgfqpoint{-0.000000in}{0.000000in}}%
\pgfpathlineto{\pgfqpoint{-0.048611in}{0.000000in}}%
\pgfusepath{stroke,fill}%
}%
\begin{pgfscope}%
\pgfsys@transformshift{13.441176in}{4.371402in}%
\pgfsys@useobject{currentmarker}{}%
\end{pgfscope}%
\end{pgfscope}%
\begin{pgfscope}%
\definecolor{textcolor}{rgb}{0.000000,0.000000,0.000000}%
\pgfsetstrokecolor{textcolor}%
\pgfsetfillcolor{textcolor}%
\pgftext[x=13.061185in, y=4.286983in, left, base]{\color{textcolor}\sffamily\fontsize{16.000000}{19.200000}\selectfont 50}%
\end{pgfscope}%
\begin{pgfscope}%
\definecolor{textcolor}{rgb}{0.000000,0.000000,0.000000}%
\pgfsetstrokecolor{textcolor}%
\pgfsetfillcolor{textcolor}%
\pgftext[x=13.005630in,y=2.512500in,,bottom,rotate=90.000000]{\color{textcolor}\sffamily\fontsize{20.000000}{24.000000}\selectfont y}%
\end{pgfscope}%
\begin{pgfscope}%
\pgfpathrectangle{\pgfqpoint{13.441176in}{0.625000in}}{\pgfqpoint{4.558824in}{3.775000in}}%
\pgfusepath{clip}%
\pgfsetrectcap%
\pgfsetroundjoin%
\pgfsetlinewidth{1.505625pt}%
\definecolor{currentstroke}{rgb}{1.000000,0.000000,0.000000}%
\pgfsetstrokecolor{currentstroke}%
\pgfsetdash{}{0pt}%
\pgfpathmoveto{\pgfqpoint{13.648396in}{0.796591in}}%
\pgfpathlineto{\pgfqpoint{13.986566in}{0.868087in}}%
\pgfpathlineto{\pgfqpoint{14.310346in}{0.939583in}}%
\pgfpathlineto{\pgfqpoint{14.619736in}{1.011080in}}%
\pgfpathlineto{\pgfqpoint{14.914736in}{1.082576in}}%
\pgfpathlineto{\pgfqpoint{15.195345in}{1.154072in}}%
\pgfpathlineto{\pgfqpoint{15.461564in}{1.225568in}}%
\pgfpathlineto{\pgfqpoint{15.713393in}{1.297064in}}%
\pgfpathlineto{\pgfqpoint{15.950832in}{1.368561in}}%
\pgfpathlineto{\pgfqpoint{16.173880in}{1.440057in}}%
\pgfpathlineto{\pgfqpoint{16.382539in}{1.511553in}}%
\pgfpathlineto{\pgfqpoint{16.576807in}{1.583049in}}%
\pgfpathlineto{\pgfqpoint{16.756684in}{1.654545in}}%
\pgfpathlineto{\pgfqpoint{16.922172in}{1.726042in}}%
\pgfpathlineto{\pgfqpoint{17.073269in}{1.797538in}}%
\pgfpathlineto{\pgfqpoint{17.209977in}{1.869034in}}%
\pgfpathlineto{\pgfqpoint{17.332294in}{1.940530in}}%
\pgfpathlineto{\pgfqpoint{17.440220in}{2.012027in}}%
\pgfpathlineto{\pgfqpoint{17.533757in}{2.083523in}}%
\pgfpathlineto{\pgfqpoint{17.612903in}{2.155019in}}%
\pgfpathlineto{\pgfqpoint{17.677659in}{2.226515in}}%
\pgfpathlineto{\pgfqpoint{17.728025in}{2.298011in}}%
\pgfpathlineto{\pgfqpoint{17.764000in}{2.369508in}}%
\pgfpathlineto{\pgfqpoint{17.785586in}{2.441004in}}%
\pgfpathlineto{\pgfqpoint{17.792781in}{2.512500in}}%
\pgfpathlineto{\pgfqpoint{17.785586in}{2.583996in}}%
\pgfpathlineto{\pgfqpoint{17.764000in}{2.655492in}}%
\pgfpathlineto{\pgfqpoint{17.728025in}{2.726989in}}%
\pgfpathlineto{\pgfqpoint{17.677659in}{2.798485in}}%
\pgfpathlineto{\pgfqpoint{17.612903in}{2.869981in}}%
\pgfpathlineto{\pgfqpoint{17.533757in}{2.941477in}}%
\pgfpathlineto{\pgfqpoint{17.440220in}{3.012973in}}%
\pgfpathlineto{\pgfqpoint{17.332294in}{3.084470in}}%
\pgfpathlineto{\pgfqpoint{17.209977in}{3.155966in}}%
\pgfpathlineto{\pgfqpoint{17.073269in}{3.227462in}}%
\pgfpathlineto{\pgfqpoint{16.922172in}{3.298958in}}%
\pgfpathlineto{\pgfqpoint{16.756684in}{3.370455in}}%
\pgfpathlineto{\pgfqpoint{16.576807in}{3.441951in}}%
\pgfpathlineto{\pgfqpoint{16.382539in}{3.513447in}}%
\pgfpathlineto{\pgfqpoint{16.173880in}{3.584943in}}%
\pgfpathlineto{\pgfqpoint{15.950832in}{3.656439in}}%
\pgfpathlineto{\pgfqpoint{15.713393in}{3.727936in}}%
\pgfpathlineto{\pgfqpoint{15.461564in}{3.799432in}}%
\pgfpathlineto{\pgfqpoint{15.195345in}{3.870928in}}%
\pgfpathlineto{\pgfqpoint{14.914736in}{3.942424in}}%
\pgfpathlineto{\pgfqpoint{14.619736in}{4.013920in}}%
\pgfpathlineto{\pgfqpoint{14.310346in}{4.085417in}}%
\pgfpathlineto{\pgfqpoint{13.986566in}{4.156913in}}%
\pgfpathlineto{\pgfqpoint{13.648396in}{4.228409in}}%
\pgfusepath{stroke}%
\end{pgfscope}%
\begin{pgfscope}%
\pgfpathrectangle{\pgfqpoint{13.441176in}{0.625000in}}{\pgfqpoint{4.558824in}{3.775000in}}%
\pgfusepath{clip}%
\pgfsetrectcap%
\pgfsetroundjoin%
\pgfsetlinewidth{1.505625pt}%
\definecolor{currentstroke}{rgb}{0.000000,0.000000,1.000000}%
\pgfsetstrokecolor{currentstroke}%
\pgfsetdash{}{0pt}%
\pgfpathmoveto{\pgfqpoint{13.834862in}{0.832339in}}%
\pgfpathlineto{\pgfqpoint{14.164374in}{0.903835in}}%
\pgfpathlineto{\pgfqpoint{14.479400in}{0.975331in}}%
\pgfpathlineto{\pgfqpoint{14.779951in}{1.046828in}}%
\pgfpathlineto{\pgfqpoint{15.066034in}{1.118324in}}%
\pgfpathlineto{\pgfqpoint{15.337662in}{1.189820in}}%
\pgfpathlineto{\pgfqpoint{15.594848in}{1.261316in}}%
\pgfpathlineto{\pgfqpoint{15.837607in}{1.332812in}}%
\pgfpathlineto{\pgfqpoint{16.065957in}{1.404309in}}%
\pgfpathlineto{\pgfqpoint{16.279916in}{1.475805in}}%
\pgfpathlineto{\pgfqpoint{16.479502in}{1.547301in}}%
\pgfpathlineto{\pgfqpoint{16.664735in}{1.618797in}}%
\pgfpathlineto{\pgfqpoint{16.835633in}{1.690294in}}%
\pgfpathlineto{\pgfqpoint{16.992215in}{1.761790in}}%
\pgfpathlineto{\pgfqpoint{17.134499in}{1.833286in}}%
\pgfpathlineto{\pgfqpoint{17.262502in}{1.904782in}}%
\pgfpathlineto{\pgfqpoint{17.376240in}{1.976278in}}%
\pgfpathlineto{\pgfqpoint{17.475728in}{2.047775in}}%
\pgfpathlineto{\pgfqpoint{17.560977in}{2.119271in}}%
\pgfpathlineto{\pgfqpoint{17.632001in}{2.190767in}}%
\pgfpathlineto{\pgfqpoint{17.688808in}{2.262263in}}%
\pgfpathlineto{\pgfqpoint{17.731406in}{2.333759in}}%
\pgfpathlineto{\pgfqpoint{17.759802in}{2.405256in}}%
\pgfpathlineto{\pgfqpoint{17.773999in}{2.476752in}}%
\pgfpathlineto{\pgfqpoint{17.773999in}{2.548248in}}%
\pgfpathlineto{\pgfqpoint{17.759802in}{2.619744in}}%
\pgfpathlineto{\pgfqpoint{17.731406in}{2.691241in}}%
\pgfpathlineto{\pgfqpoint{17.688808in}{2.762737in}}%
\pgfpathlineto{\pgfqpoint{17.632001in}{2.834233in}}%
\pgfpathlineto{\pgfqpoint{17.560977in}{2.905729in}}%
\pgfpathlineto{\pgfqpoint{17.475728in}{2.977225in}}%
\pgfpathlineto{\pgfqpoint{17.376240in}{3.048722in}}%
\pgfpathlineto{\pgfqpoint{17.262502in}{3.120218in}}%
\pgfpathlineto{\pgfqpoint{17.134499in}{3.191714in}}%
\pgfpathlineto{\pgfqpoint{16.992215in}{3.263210in}}%
\pgfpathlineto{\pgfqpoint{16.835633in}{3.334706in}}%
\pgfpathlineto{\pgfqpoint{16.664735in}{3.406203in}}%
\pgfpathlineto{\pgfqpoint{16.479502in}{3.477699in}}%
\pgfpathlineto{\pgfqpoint{16.279916in}{3.549195in}}%
\pgfpathlineto{\pgfqpoint{16.065957in}{3.620691in}}%
\pgfpathlineto{\pgfqpoint{15.837607in}{3.692188in}}%
\pgfpathlineto{\pgfqpoint{15.594848in}{3.763684in}}%
\pgfpathlineto{\pgfqpoint{15.337662in}{3.835180in}}%
\pgfpathlineto{\pgfqpoint{15.066034in}{3.906676in}}%
\pgfpathlineto{\pgfqpoint{14.779951in}{3.978172in}}%
\pgfpathlineto{\pgfqpoint{14.479400in}{4.049669in}}%
\pgfpathlineto{\pgfqpoint{14.164374in}{4.121165in}}%
\pgfpathlineto{\pgfqpoint{13.834862in}{4.192661in}}%
\pgfusepath{stroke}%
\end{pgfscope}%
\begin{pgfscope}%
\pgfsetrectcap%
\pgfsetmiterjoin%
\pgfsetlinewidth{0.803000pt}%
\definecolor{currentstroke}{rgb}{0.000000,0.000000,0.000000}%
\pgfsetstrokecolor{currentstroke}%
\pgfsetdash{}{0pt}%
\pgfpathmoveto{\pgfqpoint{13.441176in}{0.625000in}}%
\pgfpathlineto{\pgfqpoint{13.441176in}{4.400000in}}%
\pgfusepath{stroke}%
\end{pgfscope}%
\begin{pgfscope}%
\pgfsetrectcap%
\pgfsetmiterjoin%
\pgfsetlinewidth{0.803000pt}%
\definecolor{currentstroke}{rgb}{0.000000,0.000000,0.000000}%
\pgfsetstrokecolor{currentstroke}%
\pgfsetdash{}{0pt}%
\pgfpathmoveto{\pgfqpoint{18.000000in}{0.625000in}}%
\pgfpathlineto{\pgfqpoint{18.000000in}{4.400000in}}%
\pgfusepath{stroke}%
\end{pgfscope}%
\begin{pgfscope}%
\pgfsetrectcap%
\pgfsetmiterjoin%
\pgfsetlinewidth{0.803000pt}%
\definecolor{currentstroke}{rgb}{0.000000,0.000000,0.000000}%
\pgfsetstrokecolor{currentstroke}%
\pgfsetdash{}{0pt}%
\pgfpathmoveto{\pgfqpoint{13.441176in}{0.625000in}}%
\pgfpathlineto{\pgfqpoint{18.000000in}{0.625000in}}%
\pgfusepath{stroke}%
\end{pgfscope}%
\begin{pgfscope}%
\pgfsetrectcap%
\pgfsetmiterjoin%
\pgfsetlinewidth{0.803000pt}%
\definecolor{currentstroke}{rgb}{0.000000,0.000000,0.000000}%
\pgfsetstrokecolor{currentstroke}%
\pgfsetdash{}{0pt}%
\pgfpathmoveto{\pgfqpoint{13.441176in}{4.400000in}}%
\pgfpathlineto{\pgfqpoint{18.000000in}{4.400000in}}%
\pgfusepath{stroke}%
\end{pgfscope}%
\begin{pgfscope}%
\definecolor{textcolor}{rgb}{0.000000,0.000000,0.000000}%
\pgfsetstrokecolor{textcolor}%
\pgfsetfillcolor{textcolor}%
\pgftext[x=15.720588in,y=4.483333in,,base]{\color{textcolor}\sffamily\fontsize{20.000000}{24.000000}\selectfont \(\displaystyle L_x=50\) and \(\displaystyle L_y=50\)}%
\end{pgfscope}%
\begin{pgfscope}%
\pgfsetbuttcap%
\pgfsetmiterjoin%
\definecolor{currentfill}{rgb}{1.000000,1.000000,1.000000}%
\pgfsetfillcolor{currentfill}%
\pgfsetfillopacity{0.800000}%
\pgfsetlinewidth{1.003750pt}%
\definecolor{currentstroke}{rgb}{0.800000,0.800000,0.800000}%
\pgfsetstrokecolor{currentstroke}%
\pgfsetstrokeopacity{0.800000}%
\pgfsetdash{}{0pt}%
\pgfpathmoveto{\pgfqpoint{13.596732in}{2.152995in}}%
\pgfpathlineto{\pgfqpoint{15.978959in}{2.152995in}}%
\pgfpathquadraticcurveto{\pgfqpoint{16.023403in}{2.152995in}}{\pgfqpoint{16.023403in}{2.197439in}}%
\pgfpathlineto{\pgfqpoint{16.023403in}{2.827561in}}%
\pgfpathquadraticcurveto{\pgfqpoint{16.023403in}{2.872005in}}{\pgfqpoint{15.978959in}{2.872005in}}%
\pgfpathlineto{\pgfqpoint{13.596732in}{2.872005in}}%
\pgfpathquadraticcurveto{\pgfqpoint{13.552288in}{2.872005in}}{\pgfqpoint{13.552288in}{2.827561in}}%
\pgfpathlineto{\pgfqpoint{13.552288in}{2.197439in}}%
\pgfpathquadraticcurveto{\pgfqpoint{13.552288in}{2.152995in}}{\pgfqpoint{13.596732in}{2.152995in}}%
\pgfpathlineto{\pgfqpoint{13.596732in}{2.152995in}}%
\pgfpathclose%
\pgfusepath{stroke,fill}%
\end{pgfscope}%
\begin{pgfscope}%
\pgfsetrectcap%
\pgfsetroundjoin%
\pgfsetlinewidth{1.505625pt}%
\definecolor{currentstroke}{rgb}{1.000000,0.000000,0.000000}%
\pgfsetstrokecolor{currentstroke}%
\pgfsetdash{}{0pt}%
\pgfpathmoveto{\pgfqpoint{13.641176in}{2.692057in}}%
\pgfpathlineto{\pgfqpoint{13.863399in}{2.692057in}}%
\pgfpathlineto{\pgfqpoint{14.085621in}{2.692057in}}%
\pgfusepath{stroke}%
\end{pgfscope}%
\begin{pgfscope}%
\definecolor{textcolor}{rgb}{0.000000,0.000000,0.000000}%
\pgfsetstrokecolor{textcolor}%
\pgfsetfillcolor{textcolor}%
\pgftext[x=14.263399in,y=2.614279in,left,base]{\color{textcolor}\sffamily\fontsize{16.000000}{19.200000}\selectfont Theo velocities}%
\end{pgfscope}%
\begin{pgfscope}%
\pgfsetrectcap%
\pgfsetroundjoin%
\pgfsetlinewidth{1.505625pt}%
\definecolor{currentstroke}{rgb}{0.000000,0.000000,1.000000}%
\pgfsetstrokecolor{currentstroke}%
\pgfsetdash{}{0pt}%
\pgfpathmoveto{\pgfqpoint{13.641176in}{2.365885in}}%
\pgfpathlineto{\pgfqpoint{13.863399in}{2.365885in}}%
\pgfpathlineto{\pgfqpoint{14.085621in}{2.365885in}}%
\pgfusepath{stroke}%
\end{pgfscope}%
\begin{pgfscope}%
\definecolor{textcolor}{rgb}{0.000000,0.000000,0.000000}%
\pgfsetstrokecolor{textcolor}%
\pgfsetfillcolor{textcolor}%
\pgftext[x=14.263399in,y=2.288108in,left,base]{\color{textcolor}\sffamily\fontsize{16.000000}{19.200000}\selectfont Num velocities}%
\end{pgfscope}%
\end{pgfpicture}%
\makeatother%
\endgroup%
}
\caption[Poiseuille analysis extended]{Poiseuille velocities comparing the numerical (blue) and theoretical (red) solution at $t=100,000$ for different pipe sizes. The maximal difference following the order of the graphs is: $-0.00148$, $0.161374$ and $0.000348$, respectively.}
\label{fig:m5-1-num-theo-extended}
\end{figure}
Interestingly, the difference in the theoretical and the numerical velocities increases with the size of the grid or pipe. In a small pipe, the numerical solution is actually higher than the theoretical, whereas at $L=300$ the difference between expected and actual velocity is $0.161$. 
I then tried to find the numerical solution most similar to the theoretical solution, which c.p. is around $L=50$ (right graph in figure \ref{fig:m5-1-num-theo-extended} with a difference of maximal $0.000348$.


\section{M6: Sliding Lid}
The sliding lid problem has been developed to benchmark the efficieny of algorithms simulating the flow of viscous (mostly) incompressible fluid flow~\cite{bruneau20062-lid-cavity}.
It has been investigated by many authors starting in the early 80ties and a Reynolds number of $1,000$ has been widely adopted to make the results comparable~\cite{bruneau20062-lid-cavity}.

In the lecture the the following parameters were recommended:
\begin{enumerate}
    \item $L_x = L_y = 300$
    \item $\rho(\textbf{r},0) = 1.0$
    \item $\textbf{u}(\textbf{r},0) = 0.0$
    \item $T = 100000$
    \item $\omega = 1.7$
    \item $top-vel = 0.1$
\end{enumerate}
The corresponding results in the github repo can be found  \href{https://github.com/jonas27/pylbm/blob/master/milestones/m6/m6.ipynb}{here}.
These parameters result in a Reynolds number of $1,020$, which could be close enough to be comparable with similar research.

Running this configuration results in a lid driven cavity  after $T=100,000$ shown in figure \ref{fig:m6-1}.
The most striking feature in the graphs is the moving cavity. Because of the moving lid, it forms in the top right corner and steadily moves to the center of the graph. 
Cavities also form in the bottom left and right corners, albeit a lot smaller ones. 

\clearpage
\begin{figure}[ht]
\centering
% \begin{frame}{}
\textbf{\large a)}
% \animategraphics[loop, width=\columnwidth]{10}{ani/ani-}{0}{119}
\animategraphics[loop,autoplay, width=\columnwidth]{10}{ani/ani-}{0}{119}
\vspace*{2mm}

\textbf{\large b)}
\includegraphics[width=\columnwidth]{milestones/final/img/m6-1-vels.png}
\caption[Lid-driven cavity]{Lid-driven cavity shown as \textit{streamplot}. a) is an animation of the lid-driven cavity for time steps of $1,000$ viewable in some pdf readers (tested in Adobe and Okular) and can be downloaded \href{https://raw.githubusercontent.com/jonas27/pylbm/master/milestones/m6/m6.gif}{here}. b) shows the same solution but as static graphs in time steps of $10,000$. }
\label{fig:m6-1}
\end{figure}
\clearpage



\section{M7: Parallelization}
Parallelization is the concept of running multiple processes at the same time and thereby enables high performance computing on multiple nodes and CPUs. 
But its important to understand that parallelization is only useful with efficient concurrency. 
Concurrency is a program's ability to run a program at different time points and/or on different processes and is the foundation of parallelization.
For example, a program running the same problem multiple times, using only the fastest result and discarding the rest is using parallelization but very inefficiently.
On the other hand, splitting up the calculations in multiple smaller parts, executing them in different processes and combining them in the end can decrease the run time significantly.
To make use of parallelization we use the Message Passing Interface (MPI) framework using the openMPI~\cite{gabriel04:_open_mpi} implementation which has an effective implementation of concurrency.

\say{Performance evaluation is an effective and inexpensive method for assessing research results}~\cite{beyer2019-benchmark}.
It is critical in areas like high-performance computing, but can often be influenced by outside factors.  
To measure the implementation performance I let the code run on the BwUniCluster for the following configurations:
\begin{enumerate}
    \item 2 Nodes with $[2, 8, 18, 32^{*},]$ CPUs
    \item 4 Nodes with $[1, 4, 9, 16^{*}, 25, 36]$ CPUs
    \item Slurm default memory configuration
\end{enumerate}
This results in a total of eight overlapping configurations and two configurations only possible for 4 nodes. Each total number of CPUs is present twice, once on two and once on four nodes. 
This allows me to compare the performance between setups with more nodes but less CPUs per node and less nodes but more CPUs per node.
The configurations marked with $*$ do not result in a perfect geometric splitting of the LBM grid but are included to provide one more datapoint to compare the performance between 2 and 4 nodes.
Running on one, six or eight nodes was not possible on the BwUniCluster.

To recreate the results, run the following command from listing \ref{m7run} from inside the root directory of the repo.
\begin{lstlisting}[language=bash, label=m7run, caption=Run lbm with extra arguments.]
  $ python3 pylbm/lbm.py -d --x_dim=300 --y_dim=300  \
            --velocity=0.0 --time=100000 --density=1.0 \
            --omega=1.7 --lid_vel=0.1
\end{lstlisting}
\textit{-d} starts in debug mode and will print the current configurations and print after 1,000 time steps.
Alternatively, to recreate the results for all CPU configurations one can simply run the bash file \textit{schedule.sh} under \url{https://github.com/jonas27/pylbm/blob/master/milestones/m7/schedule.sh}.
This will start a slurm job for each configuration.

Finally, GPUs have recently gained popularity~\cite{boyer2013gpu-pop} for two main purposes: training artificial neural networks (ANN) and mining crypto currency. 
Both of these rely on heavy number crunching and GPUs are well suited for that purpose without going further into the details of GPUs\footnote{For the interested reader this article provides more explanation \url{https://bit.ly/3zV04Ik}}
\footnote{TPUs are designed around tensor cores which allow for faster matrix multiplication and are even more specialized then GPUs.}.
I rewrote the code using Pytorch~\cite{pytorch} and benchmarked the application on a \textit{Tesla V100 SXM2} with 32GB of memory and 125 Tensor TFLOPS cores.
I use double precision floating-point (FP64) as is the case in the numpy implementation.
But differences in how the GPU uses mathematical approximations could lead to tiny differences in the results~\cite{precision-nvidia}.
The Pytorch implementation can be found \href{https://github.com/jonas27/pylbm/blob/master/pylbm/torchlbm/lbm.py}{here}.


The results for the parallization results are shown in figure \ref{fig:m7}.
The MLUPS score is calculated by the number of grid points multiplied by the number of time steps and divided by the runtime.
Contrary to my expectations using more nodes was superior to less nodes with the same number of total CPUs. 
Less nodes would also mean less communication outside the node which should theoretically speed up the computation. The contrary was the case.

It is also surprising that the GPU performed as well as 4 nodes and 16 CPUs per node, so 64 CPUs in total. 
Both run for 87 seconds and achieved an MLUPS of $103448275$.
The GPU I used has 125 TPUs and uses advanced algorithm to make matrix computation even faster. This means comparing the results between CPU and GPU is very hard and needs deep knowledge of both architectures, the underlying network stack and the frameworks.
\begin{figure}[ht]
\centering
\textbf{\large a)} \\
\resizebox{0.7\columnwidth}{!}{\large%% Creator: Matplotlib, PGF backend
%%
%% To include the figure in your LaTeX document, write
%%   \input{<filename>.pgf}
%%
%% Make sure the required packages are loaded in your preamble
%%   \usepackage{pgf}
%%
%% Also ensure that all the required font packages are loaded; for instance,
%% the lmodern package is sometimes necessary when using math font.
%%   \usepackage{lmodern}
%%
%% Figures using additional raster images can only be included by \input if
%% they are in the same directory as the main LaTeX file. For loading figures
%% from other directories you can use the `import` package
%%   \usepackage{import}
%%
%% and then include the figures with
%%   \import{<path to file>}{<filename>.pgf}
%%
%% Matplotlib used the following preamble
%%   \usepackage{fontspec}
%%   \setmainfont{DejaVuSerif.ttf}[Path=\detokenize{/home/joe/miniconda3/envs/high/lib/python3.9/site-packages/matplotlib/mpl-data/fonts/ttf/}]
%%   \setsansfont{DejaVuSans.ttf}[Path=\detokenize{/home/joe/miniconda3/envs/high/lib/python3.9/site-packages/matplotlib/mpl-data/fonts/ttf/}]
%%   \setmonofont{DejaVuSansMono.ttf}[Path=\detokenize{/home/joe/miniconda3/envs/high/lib/python3.9/site-packages/matplotlib/mpl-data/fonts/ttf/}]
%%
\begingroup%
\makeatletter%
\begin{pgfpicture}%
\pgfpathrectangle{\pgfpointorigin}{\pgfqpoint{10.000000in}{3.000000in}}%
\pgfusepath{use as bounding box, clip}%
\begin{pgfscope}%
\pgfsetbuttcap%
\pgfsetmiterjoin%
\pgfsetlinewidth{0.000000pt}%
\definecolor{currentstroke}{rgb}{1.000000,1.000000,1.000000}%
\pgfsetstrokecolor{currentstroke}%
\pgfsetstrokeopacity{0.000000}%
\pgfsetdash{}{0pt}%
\pgfpathmoveto{\pgfqpoint{0.000000in}{0.000000in}}%
\pgfpathlineto{\pgfqpoint{10.000000in}{0.000000in}}%
\pgfpathlineto{\pgfqpoint{10.000000in}{3.000000in}}%
\pgfpathlineto{\pgfqpoint{0.000000in}{3.000000in}}%
\pgfpathlineto{\pgfqpoint{0.000000in}{0.000000in}}%
\pgfpathclose%
\pgfusepath{}%
\end{pgfscope}%
\begin{pgfscope}%
\pgfsetbuttcap%
\pgfsetmiterjoin%
\definecolor{currentfill}{rgb}{1.000000,1.000000,1.000000}%
\pgfsetfillcolor{currentfill}%
\pgfsetlinewidth{0.000000pt}%
\definecolor{currentstroke}{rgb}{0.000000,0.000000,0.000000}%
\pgfsetstrokecolor{currentstroke}%
\pgfsetstrokeopacity{0.000000}%
\pgfsetdash{}{0pt}%
\pgfpathmoveto{\pgfqpoint{1.250000in}{0.375000in}}%
\pgfpathlineto{\pgfqpoint{9.000000in}{0.375000in}}%
\pgfpathlineto{\pgfqpoint{9.000000in}{2.640000in}}%
\pgfpathlineto{\pgfqpoint{1.250000in}{2.640000in}}%
\pgfpathlineto{\pgfqpoint{1.250000in}{0.375000in}}%
\pgfpathclose%
\pgfusepath{fill}%
\end{pgfscope}%
\begin{pgfscope}%
\pgfsetbuttcap%
\pgfsetroundjoin%
\definecolor{currentfill}{rgb}{0.000000,0.000000,0.000000}%
\pgfsetfillcolor{currentfill}%
\pgfsetlinewidth{0.803000pt}%
\definecolor{currentstroke}{rgb}{0.000000,0.000000,0.000000}%
\pgfsetstrokecolor{currentstroke}%
\pgfsetdash{}{0pt}%
\pgfsys@defobject{currentmarker}{\pgfqpoint{0.000000in}{-0.048611in}}{\pgfqpoint{0.000000in}{0.000000in}}{%
\pgfpathmoveto{\pgfqpoint{0.000000in}{0.000000in}}%
\pgfpathlineto{\pgfqpoint{0.000000in}{-0.048611in}}%
\pgfusepath{stroke,fill}%
}%
\begin{pgfscope}%
\pgfsys@transformshift{2.295268in}{0.375000in}%
\pgfsys@useobject{currentmarker}{}%
\end{pgfscope}%
\end{pgfscope}%
\begin{pgfscope}%
\definecolor{textcolor}{rgb}{0.000000,0.000000,0.000000}%
\pgfsetstrokecolor{textcolor}%
\pgfsetfillcolor{textcolor}%
\pgftext[x=2.295268in,y=0.277778in,,top]{\color{textcolor}\sffamily\fontsize{16.000000}{19.200000}\selectfont 10}%
\end{pgfscope}%
\begin{pgfscope}%
\pgfsetbuttcap%
\pgfsetroundjoin%
\definecolor{currentfill}{rgb}{0.000000,0.000000,0.000000}%
\pgfsetfillcolor{currentfill}%
\pgfsetlinewidth{0.803000pt}%
\definecolor{currentstroke}{rgb}{0.000000,0.000000,0.000000}%
\pgfsetstrokecolor{currentstroke}%
\pgfsetdash{}{0pt}%
\pgfsys@defobject{currentmarker}{\pgfqpoint{0.000000in}{-0.048611in}}{\pgfqpoint{0.000000in}{0.000000in}}{%
\pgfpathmoveto{\pgfqpoint{0.000000in}{0.000000in}}%
\pgfpathlineto{\pgfqpoint{0.000000in}{-0.048611in}}%
\pgfusepath{stroke,fill}%
}%
\begin{pgfscope}%
\pgfsys@transformshift{3.450261in}{0.375000in}%
\pgfsys@useobject{currentmarker}{}%
\end{pgfscope}%
\end{pgfscope}%
\begin{pgfscope}%
\definecolor{textcolor}{rgb}{0.000000,0.000000,0.000000}%
\pgfsetstrokecolor{textcolor}%
\pgfsetfillcolor{textcolor}%
\pgftext[x=3.450261in,y=0.277778in,,top]{\color{textcolor}\sffamily\fontsize{16.000000}{19.200000}\selectfont 20}%
\end{pgfscope}%
\begin{pgfscope}%
\pgfsetbuttcap%
\pgfsetroundjoin%
\definecolor{currentfill}{rgb}{0.000000,0.000000,0.000000}%
\pgfsetfillcolor{currentfill}%
\pgfsetlinewidth{0.803000pt}%
\definecolor{currentstroke}{rgb}{0.000000,0.000000,0.000000}%
\pgfsetstrokecolor{currentstroke}%
\pgfsetdash{}{0pt}%
\pgfsys@defobject{currentmarker}{\pgfqpoint{0.000000in}{-0.048611in}}{\pgfqpoint{0.000000in}{0.000000in}}{%
\pgfpathmoveto{\pgfqpoint{0.000000in}{0.000000in}}%
\pgfpathlineto{\pgfqpoint{0.000000in}{-0.048611in}}%
\pgfusepath{stroke,fill}%
}%
\begin{pgfscope}%
\pgfsys@transformshift{4.605253in}{0.375000in}%
\pgfsys@useobject{currentmarker}{}%
\end{pgfscope}%
\end{pgfscope}%
\begin{pgfscope}%
\definecolor{textcolor}{rgb}{0.000000,0.000000,0.000000}%
\pgfsetstrokecolor{textcolor}%
\pgfsetfillcolor{textcolor}%
\pgftext[x=4.605253in,y=0.277778in,,top]{\color{textcolor}\sffamily\fontsize{16.000000}{19.200000}\selectfont 30}%
\end{pgfscope}%
\begin{pgfscope}%
\pgfsetbuttcap%
\pgfsetroundjoin%
\definecolor{currentfill}{rgb}{0.000000,0.000000,0.000000}%
\pgfsetfillcolor{currentfill}%
\pgfsetlinewidth{0.803000pt}%
\definecolor{currentstroke}{rgb}{0.000000,0.000000,0.000000}%
\pgfsetstrokecolor{currentstroke}%
\pgfsetdash{}{0pt}%
\pgfsys@defobject{currentmarker}{\pgfqpoint{0.000000in}{-0.048611in}}{\pgfqpoint{0.000000in}{0.000000in}}{%
\pgfpathmoveto{\pgfqpoint{0.000000in}{0.000000in}}%
\pgfpathlineto{\pgfqpoint{0.000000in}{-0.048611in}}%
\pgfusepath{stroke,fill}%
}%
\begin{pgfscope}%
\pgfsys@transformshift{5.760246in}{0.375000in}%
\pgfsys@useobject{currentmarker}{}%
\end{pgfscope}%
\end{pgfscope}%
\begin{pgfscope}%
\definecolor{textcolor}{rgb}{0.000000,0.000000,0.000000}%
\pgfsetstrokecolor{textcolor}%
\pgfsetfillcolor{textcolor}%
\pgftext[x=5.760246in,y=0.277778in,,top]{\color{textcolor}\sffamily\fontsize{16.000000}{19.200000}\selectfont 40}%
\end{pgfscope}%
\begin{pgfscope}%
\pgfsetbuttcap%
\pgfsetroundjoin%
\definecolor{currentfill}{rgb}{0.000000,0.000000,0.000000}%
\pgfsetfillcolor{currentfill}%
\pgfsetlinewidth{0.803000pt}%
\definecolor{currentstroke}{rgb}{0.000000,0.000000,0.000000}%
\pgfsetstrokecolor{currentstroke}%
\pgfsetdash{}{0pt}%
\pgfsys@defobject{currentmarker}{\pgfqpoint{0.000000in}{-0.048611in}}{\pgfqpoint{0.000000in}{0.000000in}}{%
\pgfpathmoveto{\pgfqpoint{0.000000in}{0.000000in}}%
\pgfpathlineto{\pgfqpoint{0.000000in}{-0.048611in}}%
\pgfusepath{stroke,fill}%
}%
\begin{pgfscope}%
\pgfsys@transformshift{6.915238in}{0.375000in}%
\pgfsys@useobject{currentmarker}{}%
\end{pgfscope}%
\end{pgfscope}%
\begin{pgfscope}%
\definecolor{textcolor}{rgb}{0.000000,0.000000,0.000000}%
\pgfsetstrokecolor{textcolor}%
\pgfsetfillcolor{textcolor}%
\pgftext[x=6.915238in,y=0.277778in,,top]{\color{textcolor}\sffamily\fontsize{16.000000}{19.200000}\selectfont 50}%
\end{pgfscope}%
\begin{pgfscope}%
\pgfsetbuttcap%
\pgfsetroundjoin%
\definecolor{currentfill}{rgb}{0.000000,0.000000,0.000000}%
\pgfsetfillcolor{currentfill}%
\pgfsetlinewidth{0.803000pt}%
\definecolor{currentstroke}{rgb}{0.000000,0.000000,0.000000}%
\pgfsetstrokecolor{currentstroke}%
\pgfsetdash{}{0pt}%
\pgfsys@defobject{currentmarker}{\pgfqpoint{0.000000in}{-0.048611in}}{\pgfqpoint{0.000000in}{0.000000in}}{%
\pgfpathmoveto{\pgfqpoint{0.000000in}{0.000000in}}%
\pgfpathlineto{\pgfqpoint{0.000000in}{-0.048611in}}%
\pgfusepath{stroke,fill}%
}%
\begin{pgfscope}%
\pgfsys@transformshift{8.070231in}{0.375000in}%
\pgfsys@useobject{currentmarker}{}%
\end{pgfscope}%
\end{pgfscope}%
\begin{pgfscope}%
\definecolor{textcolor}{rgb}{0.000000,0.000000,0.000000}%
\pgfsetstrokecolor{textcolor}%
\pgfsetfillcolor{textcolor}%
\pgftext[x=8.070231in,y=0.277778in,,top]{\color{textcolor}\sffamily\fontsize{16.000000}{19.200000}\selectfont 60}%
\end{pgfscope}%
\begin{pgfscope}%
\definecolor{textcolor}{rgb}{0.000000,0.000000,0.000000}%
\pgfsetstrokecolor{textcolor}%
\pgfsetfillcolor{textcolor}%
\pgftext[x=5.125000in,y=0.007162in,,top]{\color{textcolor}\sffamily\fontsize{20.000000}{24.000000}\selectfont cpus}%
\end{pgfscope}%
\begin{pgfscope}%
\pgfsetbuttcap%
\pgfsetroundjoin%
\definecolor{currentfill}{rgb}{0.000000,0.000000,0.000000}%
\pgfsetfillcolor{currentfill}%
\pgfsetlinewidth{0.803000pt}%
\definecolor{currentstroke}{rgb}{0.000000,0.000000,0.000000}%
\pgfsetstrokecolor{currentstroke}%
\pgfsetdash{}{0pt}%
\pgfsys@defobject{currentmarker}{\pgfqpoint{-0.048611in}{0.000000in}}{\pgfqpoint{-0.000000in}{0.000000in}}{%
\pgfpathmoveto{\pgfqpoint{-0.000000in}{0.000000in}}%
\pgfpathlineto{\pgfqpoint{-0.048611in}{0.000000in}}%
\pgfusepath{stroke,fill}%
}%
\begin{pgfscope}%
\pgfsys@transformshift{1.250000in}{0.567781in}%
\pgfsys@useobject{currentmarker}{}%
\end{pgfscope}%
\end{pgfscope}%
\begin{pgfscope}%
\definecolor{textcolor}{rgb}{0.000000,0.000000,0.000000}%
\pgfsetstrokecolor{textcolor}%
\pgfsetfillcolor{textcolor}%
\pgftext[x=0.728624in, y=0.483362in, left, base]{\color{textcolor}\sffamily\fontsize{16.000000}{19.200000}\selectfont 100}%
\end{pgfscope}%
\begin{pgfscope}%
\pgfsetbuttcap%
\pgfsetroundjoin%
\definecolor{currentfill}{rgb}{0.000000,0.000000,0.000000}%
\pgfsetfillcolor{currentfill}%
\pgfsetlinewidth{0.803000pt}%
\definecolor{currentstroke}{rgb}{0.000000,0.000000,0.000000}%
\pgfsetstrokecolor{currentstroke}%
\pgfsetdash{}{0pt}%
\pgfsys@defobject{currentmarker}{\pgfqpoint{-0.048611in}{0.000000in}}{\pgfqpoint{-0.000000in}{0.000000in}}{%
\pgfpathmoveto{\pgfqpoint{-0.000000in}{0.000000in}}%
\pgfpathlineto{\pgfqpoint{-0.048611in}{0.000000in}}%
\pgfusepath{stroke,fill}%
}%
\begin{pgfscope}%
\pgfsys@transformshift{1.250000in}{1.258751in}%
\pgfsys@useobject{currentmarker}{}%
\end{pgfscope}%
\end{pgfscope}%
\begin{pgfscope}%
\definecolor{textcolor}{rgb}{0.000000,0.000000,0.000000}%
\pgfsetstrokecolor{textcolor}%
\pgfsetfillcolor{textcolor}%
\pgftext[x=0.728624in, y=1.174332in, left, base]{\color{textcolor}\sffamily\fontsize{16.000000}{19.200000}\selectfont 200}%
\end{pgfscope}%
\begin{pgfscope}%
\pgfsetbuttcap%
\pgfsetroundjoin%
\definecolor{currentfill}{rgb}{0.000000,0.000000,0.000000}%
\pgfsetfillcolor{currentfill}%
\pgfsetlinewidth{0.803000pt}%
\definecolor{currentstroke}{rgb}{0.000000,0.000000,0.000000}%
\pgfsetstrokecolor{currentstroke}%
\pgfsetdash{}{0pt}%
\pgfsys@defobject{currentmarker}{\pgfqpoint{-0.048611in}{0.000000in}}{\pgfqpoint{-0.000000in}{0.000000in}}{%
\pgfpathmoveto{\pgfqpoint{-0.000000in}{0.000000in}}%
\pgfpathlineto{\pgfqpoint{-0.048611in}{0.000000in}}%
\pgfusepath{stroke,fill}%
}%
\begin{pgfscope}%
\pgfsys@transformshift{1.250000in}{1.949721in}%
\pgfsys@useobject{currentmarker}{}%
\end{pgfscope}%
\end{pgfscope}%
\begin{pgfscope}%
\definecolor{textcolor}{rgb}{0.000000,0.000000,0.000000}%
\pgfsetstrokecolor{textcolor}%
\pgfsetfillcolor{textcolor}%
\pgftext[x=0.728624in, y=1.865302in, left, base]{\color{textcolor}\sffamily\fontsize{16.000000}{19.200000}\selectfont 300}%
\end{pgfscope}%
\begin{pgfscope}%
\definecolor{textcolor}{rgb}{0.000000,0.000000,0.000000}%
\pgfsetstrokecolor{textcolor}%
\pgfsetfillcolor{textcolor}%
\pgftext[x=0.673069in,y=1.507500in,,bottom,rotate=90.000000]{\color{textcolor}\sffamily\fontsize{20.000000}{24.000000}\selectfont time in sec}%
\end{pgfscope}%
\begin{pgfscope}%
\pgfpathrectangle{\pgfqpoint{1.250000in}{0.375000in}}{\pgfqpoint{7.750000in}{2.265000in}}%
\pgfusepath{clip}%
\pgfsetbuttcap%
\pgfsetroundjoin%
\pgfsetlinewidth{1.505625pt}%
\definecolor{currentstroke}{rgb}{0.121569,0.466667,0.705882}%
\pgfsetstrokecolor{currentstroke}%
\pgfsetdash{{5.550000pt}{2.400000pt}}{0.000000pt}%
\pgfpathmoveto{\pgfqpoint{1.602273in}{2.537045in}}%
\pgfpathlineto{\pgfqpoint{2.988264in}{0.989272in}}%
\pgfpathlineto{\pgfqpoint{5.298249in}{0.643787in}}%
\pgfpathlineto{\pgfqpoint{8.532228in}{0.498684in}}%
\pgfusepath{stroke}%
\end{pgfscope}%
\begin{pgfscope}%
\pgfpathrectangle{\pgfqpoint{1.250000in}{0.375000in}}{\pgfqpoint{7.750000in}{2.265000in}}%
\pgfusepath{clip}%
\pgfsetbuttcap%
\pgfsetroundjoin%
\definecolor{currentfill}{rgb}{0.121569,0.466667,0.705882}%
\pgfsetfillcolor{currentfill}%
\pgfsetlinewidth{1.003750pt}%
\definecolor{currentstroke}{rgb}{0.121569,0.466667,0.705882}%
\pgfsetstrokecolor{currentstroke}%
\pgfsetdash{}{0pt}%
\pgfsys@defobject{currentmarker}{\pgfqpoint{-0.041667in}{-0.041667in}}{\pgfqpoint{0.041667in}{0.041667in}}{%
\pgfpathmoveto{\pgfqpoint{0.000000in}{-0.041667in}}%
\pgfpathcurveto{\pgfqpoint{0.011050in}{-0.041667in}}{\pgfqpoint{0.021649in}{-0.037276in}}{\pgfqpoint{0.029463in}{-0.029463in}}%
\pgfpathcurveto{\pgfqpoint{0.037276in}{-0.021649in}}{\pgfqpoint{0.041667in}{-0.011050in}}{\pgfqpoint{0.041667in}{0.000000in}}%
\pgfpathcurveto{\pgfqpoint{0.041667in}{0.011050in}}{\pgfqpoint{0.037276in}{0.021649in}}{\pgfqpoint{0.029463in}{0.029463in}}%
\pgfpathcurveto{\pgfqpoint{0.021649in}{0.037276in}}{\pgfqpoint{0.011050in}{0.041667in}}{\pgfqpoint{0.000000in}{0.041667in}}%
\pgfpathcurveto{\pgfqpoint{-0.011050in}{0.041667in}}{\pgfqpoint{-0.021649in}{0.037276in}}{\pgfqpoint{-0.029463in}{0.029463in}}%
\pgfpathcurveto{\pgfqpoint{-0.037276in}{0.021649in}}{\pgfqpoint{-0.041667in}{0.011050in}}{\pgfqpoint{-0.041667in}{0.000000in}}%
\pgfpathcurveto{\pgfqpoint{-0.041667in}{-0.011050in}}{\pgfqpoint{-0.037276in}{-0.021649in}}{\pgfqpoint{-0.029463in}{-0.029463in}}%
\pgfpathcurveto{\pgfqpoint{-0.021649in}{-0.037276in}}{\pgfqpoint{-0.011050in}{-0.041667in}}{\pgfqpoint{0.000000in}{-0.041667in}}%
\pgfpathlineto{\pgfqpoint{0.000000in}{-0.041667in}}%
\pgfpathclose%
\pgfusepath{stroke,fill}%
}%
\begin{pgfscope}%
\pgfsys@transformshift{1.602273in}{2.537045in}%
\pgfsys@useobject{currentmarker}{}%
\end{pgfscope}%
\begin{pgfscope}%
\pgfsys@transformshift{2.988264in}{0.989272in}%
\pgfsys@useobject{currentmarker}{}%
\end{pgfscope}%
\begin{pgfscope}%
\pgfsys@transformshift{5.298249in}{0.643787in}%
\pgfsys@useobject{currentmarker}{}%
\end{pgfscope}%
\begin{pgfscope}%
\pgfsys@transformshift{8.532228in}{0.498684in}%
\pgfsys@useobject{currentmarker}{}%
\end{pgfscope}%
\end{pgfscope}%
\begin{pgfscope}%
\pgfpathrectangle{\pgfqpoint{1.250000in}{0.375000in}}{\pgfqpoint{7.750000in}{2.265000in}}%
\pgfusepath{clip}%
\pgfsetbuttcap%
\pgfsetroundjoin%
\pgfsetlinewidth{1.505625pt}%
\definecolor{currentstroke}{rgb}{1.000000,0.498039,0.054902}%
\pgfsetstrokecolor{currentstroke}%
\pgfsetdash{{5.550000pt}{2.400000pt}}{0.000000pt}%
\pgfpathmoveto{\pgfqpoint{1.602273in}{2.081005in}}%
\pgfpathlineto{\pgfqpoint{2.988264in}{0.761252in}}%
\pgfpathlineto{\pgfqpoint{5.298249in}{0.574690in}}%
\pgfpathlineto{\pgfqpoint{8.532228in}{0.477955in}}%
\pgfusepath{stroke}%
\end{pgfscope}%
\begin{pgfscope}%
\pgfpathrectangle{\pgfqpoint{1.250000in}{0.375000in}}{\pgfqpoint{7.750000in}{2.265000in}}%
\pgfusepath{clip}%
\pgfsetbuttcap%
\pgfsetroundjoin%
\definecolor{currentfill}{rgb}{1.000000,0.498039,0.054902}%
\pgfsetfillcolor{currentfill}%
\pgfsetlinewidth{1.003750pt}%
\definecolor{currentstroke}{rgb}{1.000000,0.498039,0.054902}%
\pgfsetstrokecolor{currentstroke}%
\pgfsetdash{}{0pt}%
\pgfsys@defobject{currentmarker}{\pgfqpoint{-0.041667in}{-0.041667in}}{\pgfqpoint{0.041667in}{0.041667in}}{%
\pgfpathmoveto{\pgfqpoint{0.000000in}{-0.041667in}}%
\pgfpathcurveto{\pgfqpoint{0.011050in}{-0.041667in}}{\pgfqpoint{0.021649in}{-0.037276in}}{\pgfqpoint{0.029463in}{-0.029463in}}%
\pgfpathcurveto{\pgfqpoint{0.037276in}{-0.021649in}}{\pgfqpoint{0.041667in}{-0.011050in}}{\pgfqpoint{0.041667in}{0.000000in}}%
\pgfpathcurveto{\pgfqpoint{0.041667in}{0.011050in}}{\pgfqpoint{0.037276in}{0.021649in}}{\pgfqpoint{0.029463in}{0.029463in}}%
\pgfpathcurveto{\pgfqpoint{0.021649in}{0.037276in}}{\pgfqpoint{0.011050in}{0.041667in}}{\pgfqpoint{0.000000in}{0.041667in}}%
\pgfpathcurveto{\pgfqpoint{-0.011050in}{0.041667in}}{\pgfqpoint{-0.021649in}{0.037276in}}{\pgfqpoint{-0.029463in}{0.029463in}}%
\pgfpathcurveto{\pgfqpoint{-0.037276in}{0.021649in}}{\pgfqpoint{-0.041667in}{0.011050in}}{\pgfqpoint{-0.041667in}{0.000000in}}%
\pgfpathcurveto{\pgfqpoint{-0.041667in}{-0.011050in}}{\pgfqpoint{-0.037276in}{-0.021649in}}{\pgfqpoint{-0.029463in}{-0.029463in}}%
\pgfpathcurveto{\pgfqpoint{-0.021649in}{-0.037276in}}{\pgfqpoint{-0.011050in}{-0.041667in}}{\pgfqpoint{0.000000in}{-0.041667in}}%
\pgfpathlineto{\pgfqpoint{0.000000in}{-0.041667in}}%
\pgfpathclose%
\pgfusepath{stroke,fill}%
}%
\begin{pgfscope}%
\pgfsys@transformshift{1.602273in}{2.081005in}%
\pgfsys@useobject{currentmarker}{}%
\end{pgfscope}%
\begin{pgfscope}%
\pgfsys@transformshift{2.988264in}{0.761252in}%
\pgfsys@useobject{currentmarker}{}%
\end{pgfscope}%
\begin{pgfscope}%
\pgfsys@transformshift{5.298249in}{0.574690in}%
\pgfsys@useobject{currentmarker}{}%
\end{pgfscope}%
\begin{pgfscope}%
\pgfsys@transformshift{8.532228in}{0.477955in}%
\pgfsys@useobject{currentmarker}{}%
\end{pgfscope}%
\end{pgfscope}%
\begin{pgfscope}%
\pgfpathrectangle{\pgfqpoint{1.250000in}{0.375000in}}{\pgfqpoint{7.750000in}{2.265000in}}%
\pgfusepath{clip}%
\pgfsetbuttcap%
\pgfsetroundjoin%
\definecolor{currentfill}{rgb}{1.000000,0.000000,0.000000}%
\pgfsetfillcolor{currentfill}%
\pgfsetlinewidth{1.003750pt}%
\definecolor{currentstroke}{rgb}{1.000000,0.000000,0.000000}%
\pgfsetstrokecolor{currentstroke}%
\pgfsetdash{}{0pt}%
\pgfsys@defobject{currentmarker}{\pgfqpoint{-0.041667in}{-0.041667in}}{\pgfqpoint{0.041667in}{0.041667in}}{%
\pgfpathmoveto{\pgfqpoint{0.000000in}{-0.041667in}}%
\pgfpathcurveto{\pgfqpoint{0.011050in}{-0.041667in}}{\pgfqpoint{0.021649in}{-0.037276in}}{\pgfqpoint{0.029463in}{-0.029463in}}%
\pgfpathcurveto{\pgfqpoint{0.037276in}{-0.021649in}}{\pgfqpoint{0.041667in}{-0.011050in}}{\pgfqpoint{0.041667in}{0.000000in}}%
\pgfpathcurveto{\pgfqpoint{0.041667in}{0.011050in}}{\pgfqpoint{0.037276in}{0.021649in}}{\pgfqpoint{0.029463in}{0.029463in}}%
\pgfpathcurveto{\pgfqpoint{0.021649in}{0.037276in}}{\pgfqpoint{0.011050in}{0.041667in}}{\pgfqpoint{0.000000in}{0.041667in}}%
\pgfpathcurveto{\pgfqpoint{-0.011050in}{0.041667in}}{\pgfqpoint{-0.021649in}{0.037276in}}{\pgfqpoint{-0.029463in}{0.029463in}}%
\pgfpathcurveto{\pgfqpoint{-0.037276in}{0.021649in}}{\pgfqpoint{-0.041667in}{0.011050in}}{\pgfqpoint{-0.041667in}{0.000000in}}%
\pgfpathcurveto{\pgfqpoint{-0.041667in}{-0.011050in}}{\pgfqpoint{-0.037276in}{-0.021649in}}{\pgfqpoint{-0.029463in}{-0.029463in}}%
\pgfpathcurveto{\pgfqpoint{-0.021649in}{-0.037276in}}{\pgfqpoint{-0.011050in}{-0.041667in}}{\pgfqpoint{0.000000in}{-0.041667in}}%
\pgfpathlineto{\pgfqpoint{0.000000in}{-0.041667in}}%
\pgfpathclose%
\pgfusepath{stroke,fill}%
}%
\begin{pgfscope}%
\pgfsys@transformshift{8.647727in}{0.479495in}%
\pgfsys@useobject{currentmarker}{}%
\end{pgfscope}%
\end{pgfscope}%
\begin{pgfscope}%
\pgfsetrectcap%
\pgfsetmiterjoin%
\pgfsetlinewidth{0.803000pt}%
\definecolor{currentstroke}{rgb}{0.000000,0.000000,0.000000}%
\pgfsetstrokecolor{currentstroke}%
\pgfsetdash{}{0pt}%
\pgfpathmoveto{\pgfqpoint{1.250000in}{0.375000in}}%
\pgfpathlineto{\pgfqpoint{1.250000in}{2.640000in}}%
\pgfusepath{stroke}%
\end{pgfscope}%
\begin{pgfscope}%
\pgfsetrectcap%
\pgfsetmiterjoin%
\pgfsetlinewidth{0.803000pt}%
\definecolor{currentstroke}{rgb}{0.000000,0.000000,0.000000}%
\pgfsetstrokecolor{currentstroke}%
\pgfsetdash{}{0pt}%
\pgfpathmoveto{\pgfqpoint{9.000000in}{0.375000in}}%
\pgfpathlineto{\pgfqpoint{9.000000in}{2.640000in}}%
\pgfusepath{stroke}%
\end{pgfscope}%
\begin{pgfscope}%
\pgfsetrectcap%
\pgfsetmiterjoin%
\pgfsetlinewidth{0.803000pt}%
\definecolor{currentstroke}{rgb}{0.000000,0.000000,0.000000}%
\pgfsetstrokecolor{currentstroke}%
\pgfsetdash{}{0pt}%
\pgfpathmoveto{\pgfqpoint{1.250000in}{0.375000in}}%
\pgfpathlineto{\pgfqpoint{9.000000in}{0.375000in}}%
\pgfusepath{stroke}%
\end{pgfscope}%
\begin{pgfscope}%
\pgfsetrectcap%
\pgfsetmiterjoin%
\pgfsetlinewidth{0.803000pt}%
\definecolor{currentstroke}{rgb}{0.000000,0.000000,0.000000}%
\pgfsetstrokecolor{currentstroke}%
\pgfsetdash{}{0pt}%
\pgfpathmoveto{\pgfqpoint{1.250000in}{2.640000in}}%
\pgfpathlineto{\pgfqpoint{9.000000in}{2.640000in}}%
\pgfusepath{stroke}%
\end{pgfscope}%
\begin{pgfscope}%
\definecolor{textcolor}{rgb}{0.000000,0.000000,0.000000}%
\pgfsetstrokecolor{textcolor}%
\pgfsetfillcolor{textcolor}%
\pgftext[x=5.125000in,y=2.723333in,,base]{\color{textcolor}\sffamily\fontsize{20.000000}{24.000000}\selectfont Time with respect to number of cpus}%
\end{pgfscope}%
\begin{pgfscope}%
\pgfsetbuttcap%
\pgfsetmiterjoin%
\definecolor{currentfill}{rgb}{1.000000,1.000000,1.000000}%
\pgfsetfillcolor{currentfill}%
\pgfsetfillopacity{0.800000}%
\pgfsetlinewidth{1.003750pt}%
\definecolor{currentstroke}{rgb}{0.800000,0.800000,0.800000}%
\pgfsetstrokecolor{currentstroke}%
\pgfsetstrokeopacity{0.800000}%
\pgfsetdash{}{0pt}%
\pgfpathmoveto{\pgfqpoint{5.563021in}{1.483707in}}%
\pgfpathlineto{\pgfqpoint{8.844444in}{1.483707in}}%
\pgfpathquadraticcurveto{\pgfqpoint{8.888889in}{1.483707in}}{\pgfqpoint{8.888889in}{1.528152in}}%
\pgfpathlineto{\pgfqpoint{8.888889in}{2.484444in}}%
\pgfpathquadraticcurveto{\pgfqpoint{8.888889in}{2.528889in}}{\pgfqpoint{8.844444in}{2.528889in}}%
\pgfpathlineto{\pgfqpoint{5.563021in}{2.528889in}}%
\pgfpathquadraticcurveto{\pgfqpoint{5.518576in}{2.528889in}}{\pgfqpoint{5.518576in}{2.484444in}}%
\pgfpathlineto{\pgfqpoint{5.518576in}{1.528152in}}%
\pgfpathquadraticcurveto{\pgfqpoint{5.518576in}{1.483707in}}{\pgfqpoint{5.563021in}{1.483707in}}%
\pgfpathlineto{\pgfqpoint{5.563021in}{1.483707in}}%
\pgfpathclose%
\pgfusepath{stroke,fill}%
\end{pgfscope}%
\begin{pgfscope}%
\pgfsetbuttcap%
\pgfsetroundjoin%
\pgfsetlinewidth{1.505625pt}%
\definecolor{currentstroke}{rgb}{0.121569,0.466667,0.705882}%
\pgfsetstrokecolor{currentstroke}%
\pgfsetdash{{5.550000pt}{2.400000pt}}{0.000000pt}%
\pgfpathmoveto{\pgfqpoint{5.607465in}{2.348941in}}%
\pgfpathlineto{\pgfqpoint{5.829687in}{2.348941in}}%
\pgfpathlineto{\pgfqpoint{6.051910in}{2.348941in}}%
\pgfusepath{stroke}%
\end{pgfscope}%
\begin{pgfscope}%
\pgfsetbuttcap%
\pgfsetroundjoin%
\definecolor{currentfill}{rgb}{0.121569,0.466667,0.705882}%
\pgfsetfillcolor{currentfill}%
\pgfsetlinewidth{1.003750pt}%
\definecolor{currentstroke}{rgb}{0.121569,0.466667,0.705882}%
\pgfsetstrokecolor{currentstroke}%
\pgfsetdash{}{0pt}%
\pgfsys@defobject{currentmarker}{\pgfqpoint{-0.041667in}{-0.041667in}}{\pgfqpoint{0.041667in}{0.041667in}}{%
\pgfpathmoveto{\pgfqpoint{0.000000in}{-0.041667in}}%
\pgfpathcurveto{\pgfqpoint{0.011050in}{-0.041667in}}{\pgfqpoint{0.021649in}{-0.037276in}}{\pgfqpoint{0.029463in}{-0.029463in}}%
\pgfpathcurveto{\pgfqpoint{0.037276in}{-0.021649in}}{\pgfqpoint{0.041667in}{-0.011050in}}{\pgfqpoint{0.041667in}{0.000000in}}%
\pgfpathcurveto{\pgfqpoint{0.041667in}{0.011050in}}{\pgfqpoint{0.037276in}{0.021649in}}{\pgfqpoint{0.029463in}{0.029463in}}%
\pgfpathcurveto{\pgfqpoint{0.021649in}{0.037276in}}{\pgfqpoint{0.011050in}{0.041667in}}{\pgfqpoint{0.000000in}{0.041667in}}%
\pgfpathcurveto{\pgfqpoint{-0.011050in}{0.041667in}}{\pgfqpoint{-0.021649in}{0.037276in}}{\pgfqpoint{-0.029463in}{0.029463in}}%
\pgfpathcurveto{\pgfqpoint{-0.037276in}{0.021649in}}{\pgfqpoint{-0.041667in}{0.011050in}}{\pgfqpoint{-0.041667in}{0.000000in}}%
\pgfpathcurveto{\pgfqpoint{-0.041667in}{-0.011050in}}{\pgfqpoint{-0.037276in}{-0.021649in}}{\pgfqpoint{-0.029463in}{-0.029463in}}%
\pgfpathcurveto{\pgfqpoint{-0.021649in}{-0.037276in}}{\pgfqpoint{-0.011050in}{-0.041667in}}{\pgfqpoint{0.000000in}{-0.041667in}}%
\pgfpathlineto{\pgfqpoint{0.000000in}{-0.041667in}}%
\pgfpathclose%
\pgfusepath{stroke,fill}%
}%
\begin{pgfscope}%
\pgfsys@transformshift{5.829687in}{2.348941in}%
\pgfsys@useobject{currentmarker}{}%
\end{pgfscope}%
\end{pgfscope}%
\begin{pgfscope}%
\definecolor{textcolor}{rgb}{0.000000,0.000000,0.000000}%
\pgfsetstrokecolor{textcolor}%
\pgfsetfillcolor{textcolor}%
\pgftext[x=6.229687in,y=2.271163in,left,base]{\color{textcolor}\sffamily\fontsize{16.000000}{19.200000}\selectfont nodes=2}%
\end{pgfscope}%
\begin{pgfscope}%
\pgfsetbuttcap%
\pgfsetroundjoin%
\pgfsetlinewidth{1.505625pt}%
\definecolor{currentstroke}{rgb}{1.000000,0.498039,0.054902}%
\pgfsetstrokecolor{currentstroke}%
\pgfsetdash{{5.550000pt}{2.400000pt}}{0.000000pt}%
\pgfpathmoveto{\pgfqpoint{5.607465in}{2.022769in}}%
\pgfpathlineto{\pgfqpoint{5.829687in}{2.022769in}}%
\pgfpathlineto{\pgfqpoint{6.051910in}{2.022769in}}%
\pgfusepath{stroke}%
\end{pgfscope}%
\begin{pgfscope}%
\pgfsetbuttcap%
\pgfsetroundjoin%
\definecolor{currentfill}{rgb}{1.000000,0.498039,0.054902}%
\pgfsetfillcolor{currentfill}%
\pgfsetlinewidth{1.003750pt}%
\definecolor{currentstroke}{rgb}{1.000000,0.498039,0.054902}%
\pgfsetstrokecolor{currentstroke}%
\pgfsetdash{}{0pt}%
\pgfsys@defobject{currentmarker}{\pgfqpoint{-0.041667in}{-0.041667in}}{\pgfqpoint{0.041667in}{0.041667in}}{%
\pgfpathmoveto{\pgfqpoint{0.000000in}{-0.041667in}}%
\pgfpathcurveto{\pgfqpoint{0.011050in}{-0.041667in}}{\pgfqpoint{0.021649in}{-0.037276in}}{\pgfqpoint{0.029463in}{-0.029463in}}%
\pgfpathcurveto{\pgfqpoint{0.037276in}{-0.021649in}}{\pgfqpoint{0.041667in}{-0.011050in}}{\pgfqpoint{0.041667in}{0.000000in}}%
\pgfpathcurveto{\pgfqpoint{0.041667in}{0.011050in}}{\pgfqpoint{0.037276in}{0.021649in}}{\pgfqpoint{0.029463in}{0.029463in}}%
\pgfpathcurveto{\pgfqpoint{0.021649in}{0.037276in}}{\pgfqpoint{0.011050in}{0.041667in}}{\pgfqpoint{0.000000in}{0.041667in}}%
\pgfpathcurveto{\pgfqpoint{-0.011050in}{0.041667in}}{\pgfqpoint{-0.021649in}{0.037276in}}{\pgfqpoint{-0.029463in}{0.029463in}}%
\pgfpathcurveto{\pgfqpoint{-0.037276in}{0.021649in}}{\pgfqpoint{-0.041667in}{0.011050in}}{\pgfqpoint{-0.041667in}{0.000000in}}%
\pgfpathcurveto{\pgfqpoint{-0.041667in}{-0.011050in}}{\pgfqpoint{-0.037276in}{-0.021649in}}{\pgfqpoint{-0.029463in}{-0.029463in}}%
\pgfpathcurveto{\pgfqpoint{-0.021649in}{-0.037276in}}{\pgfqpoint{-0.011050in}{-0.041667in}}{\pgfqpoint{0.000000in}{-0.041667in}}%
\pgfpathlineto{\pgfqpoint{0.000000in}{-0.041667in}}%
\pgfpathclose%
\pgfusepath{stroke,fill}%
}%
\begin{pgfscope}%
\pgfsys@transformshift{5.829687in}{2.022769in}%
\pgfsys@useobject{currentmarker}{}%
\end{pgfscope}%
\end{pgfscope}%
\begin{pgfscope}%
\definecolor{textcolor}{rgb}{0.000000,0.000000,0.000000}%
\pgfsetstrokecolor{textcolor}%
\pgfsetfillcolor{textcolor}%
\pgftext[x=6.229687in,y=1.944992in,left,base]{\color{textcolor}\sffamily\fontsize{16.000000}{19.200000}\selectfont nodes=4}%
\end{pgfscope}%
\begin{pgfscope}%
\pgfsetbuttcap%
\pgfsetroundjoin%
\definecolor{currentfill}{rgb}{1.000000,0.000000,0.000000}%
\pgfsetfillcolor{currentfill}%
\pgfsetlinewidth{1.003750pt}%
\definecolor{currentstroke}{rgb}{1.000000,0.000000,0.000000}%
\pgfsetstrokecolor{currentstroke}%
\pgfsetdash{}{0pt}%
\pgfsys@defobject{currentmarker}{\pgfqpoint{-0.041667in}{-0.041667in}}{\pgfqpoint{0.041667in}{0.041667in}}{%
\pgfpathmoveto{\pgfqpoint{0.000000in}{-0.041667in}}%
\pgfpathcurveto{\pgfqpoint{0.011050in}{-0.041667in}}{\pgfqpoint{0.021649in}{-0.037276in}}{\pgfqpoint{0.029463in}{-0.029463in}}%
\pgfpathcurveto{\pgfqpoint{0.037276in}{-0.021649in}}{\pgfqpoint{0.041667in}{-0.011050in}}{\pgfqpoint{0.041667in}{0.000000in}}%
\pgfpathcurveto{\pgfqpoint{0.041667in}{0.011050in}}{\pgfqpoint{0.037276in}{0.021649in}}{\pgfqpoint{0.029463in}{0.029463in}}%
\pgfpathcurveto{\pgfqpoint{0.021649in}{0.037276in}}{\pgfqpoint{0.011050in}{0.041667in}}{\pgfqpoint{0.000000in}{0.041667in}}%
\pgfpathcurveto{\pgfqpoint{-0.011050in}{0.041667in}}{\pgfqpoint{-0.021649in}{0.037276in}}{\pgfqpoint{-0.029463in}{0.029463in}}%
\pgfpathcurveto{\pgfqpoint{-0.037276in}{0.021649in}}{\pgfqpoint{-0.041667in}{0.011050in}}{\pgfqpoint{-0.041667in}{0.000000in}}%
\pgfpathcurveto{\pgfqpoint{-0.041667in}{-0.011050in}}{\pgfqpoint{-0.037276in}{-0.021649in}}{\pgfqpoint{-0.029463in}{-0.029463in}}%
\pgfpathcurveto{\pgfqpoint{-0.021649in}{-0.037276in}}{\pgfqpoint{-0.011050in}{-0.041667in}}{\pgfqpoint{0.000000in}{-0.041667in}}%
\pgfpathlineto{\pgfqpoint{0.000000in}{-0.041667in}}%
\pgfpathclose%
\pgfusepath{stroke,fill}%
}%
\begin{pgfscope}%
\pgfsys@transformshift{5.829687in}{1.696598in}%
\pgfsys@useobject{currentmarker}{}%
\end{pgfscope}%
\end{pgfscope}%
\begin{pgfscope}%
\definecolor{textcolor}{rgb}{0.000000,0.000000,0.000000}%
\pgfsetstrokecolor{textcolor}%
\pgfsetfillcolor{textcolor}%
\pgftext[x=6.229687in,y=1.618820in,left,base]{\color{textcolor}\sffamily\fontsize{16.000000}{19.200000}\selectfont Tesla V100 SXM2 32GB}%
\end{pgfscope}%
\end{pgfpicture}%
\makeatother%
\endgroup%
}
\vspace*{2mm}

\textbf{\large b)} \\
\resizebox{0.7\columnwidth}{!}{\large%% Creator: Matplotlib, PGF backend
%%
%% To include the figure in your LaTeX document, write
%%   \input{<filename>.pgf}
%%
%% Make sure the required packages are loaded in your preamble
%%   \usepackage{pgf}
%%
%% Also ensure that all the required font packages are loaded; for instance,
%% the lmodern package is sometimes necessary when using math font.
%%   \usepackage{lmodern}
%%
%% Figures using additional raster images can only be included by \input if
%% they are in the same directory as the main LaTeX file. For loading figures
%% from other directories you can use the `import` package
%%   \usepackage{import}
%%
%% and then include the figures with
%%   \import{<path to file>}{<filename>.pgf}
%%
%% Matplotlib used the following preamble
%%   \usepackage{fontspec}
%%   \setmainfont{DejaVuSerif.ttf}[Path=\detokenize{/home/joe/.local/lib/python3.8/site-packages/matplotlib/mpl-data/fonts/ttf/}]
%%   \setsansfont{DejaVuSans.ttf}[Path=\detokenize{/home/joe/.local/lib/python3.8/site-packages/matplotlib/mpl-data/fonts/ttf/}]
%%   \setmonofont{DejaVuSansMono.ttf}[Path=\detokenize{/home/joe/.local/lib/python3.8/site-packages/matplotlib/mpl-data/fonts/ttf/}]
%%
\begingroup%
\makeatletter%
\begin{pgfpicture}%
\pgfpathrectangle{\pgfpointorigin}{\pgfqpoint{10.000000in}{5.000000in}}%
\pgfusepath{use as bounding box, clip}%
\begin{pgfscope}%
\pgfsetbuttcap%
\pgfsetmiterjoin%
\pgfsetlinewidth{0.000000pt}%
\definecolor{currentstroke}{rgb}{1.000000,1.000000,1.000000}%
\pgfsetstrokecolor{currentstroke}%
\pgfsetstrokeopacity{0.000000}%
\pgfsetdash{}{0pt}%
\pgfpathmoveto{\pgfqpoint{0.000000in}{0.000000in}}%
\pgfpathlineto{\pgfqpoint{10.000000in}{0.000000in}}%
\pgfpathlineto{\pgfqpoint{10.000000in}{5.000000in}}%
\pgfpathlineto{\pgfqpoint{0.000000in}{5.000000in}}%
\pgfpathlineto{\pgfqpoint{0.000000in}{0.000000in}}%
\pgfpathclose%
\pgfusepath{}%
\end{pgfscope}%
\begin{pgfscope}%
\pgfsetbuttcap%
\pgfsetmiterjoin%
\definecolor{currentfill}{rgb}{1.000000,1.000000,1.000000}%
\pgfsetfillcolor{currentfill}%
\pgfsetlinewidth{0.000000pt}%
\definecolor{currentstroke}{rgb}{0.000000,0.000000,0.000000}%
\pgfsetstrokecolor{currentstroke}%
\pgfsetstrokeopacity{0.000000}%
\pgfsetdash{}{0pt}%
\pgfpathmoveto{\pgfqpoint{1.250000in}{0.625000in}}%
\pgfpathlineto{\pgfqpoint{9.000000in}{0.625000in}}%
\pgfpathlineto{\pgfqpoint{9.000000in}{4.400000in}}%
\pgfpathlineto{\pgfqpoint{1.250000in}{4.400000in}}%
\pgfpathlineto{\pgfqpoint{1.250000in}{0.625000in}}%
\pgfpathclose%
\pgfusepath{fill}%
\end{pgfscope}%
\begin{pgfscope}%
\pgfsetbuttcap%
\pgfsetroundjoin%
\definecolor{currentfill}{rgb}{0.000000,0.000000,0.000000}%
\pgfsetfillcolor{currentfill}%
\pgfsetlinewidth{0.803000pt}%
\definecolor{currentstroke}{rgb}{0.000000,0.000000,0.000000}%
\pgfsetstrokecolor{currentstroke}%
\pgfsetdash{}{0pt}%
\pgfsys@defobject{currentmarker}{\pgfqpoint{0.000000in}{-0.048611in}}{\pgfqpoint{0.000000in}{0.000000in}}{%
\pgfpathmoveto{\pgfqpoint{0.000000in}{0.000000in}}%
\pgfpathlineto{\pgfqpoint{0.000000in}{-0.048611in}}%
\pgfusepath{stroke,fill}%
}%
\begin{pgfscope}%
\pgfsys@transformshift{1.400974in}{0.625000in}%
\pgfsys@useobject{currentmarker}{}%
\end{pgfscope}%
\end{pgfscope}%
\begin{pgfscope}%
\definecolor{textcolor}{rgb}{0.000000,0.000000,0.000000}%
\pgfsetstrokecolor{textcolor}%
\pgfsetfillcolor{textcolor}%
\pgftext[x=1.400974in,y=0.527778in,,top]{\color{textcolor}\sffamily\fontsize{16.000000}{19.200000}\selectfont 0}%
\end{pgfscope}%
\begin{pgfscope}%
\pgfsetbuttcap%
\pgfsetroundjoin%
\definecolor{currentfill}{rgb}{0.000000,0.000000,0.000000}%
\pgfsetfillcolor{currentfill}%
\pgfsetlinewidth{0.803000pt}%
\definecolor{currentstroke}{rgb}{0.000000,0.000000,0.000000}%
\pgfsetstrokecolor{currentstroke}%
\pgfsetdash{}{0pt}%
\pgfsys@defobject{currentmarker}{\pgfqpoint{0.000000in}{-0.048611in}}{\pgfqpoint{0.000000in}{0.000000in}}{%
\pgfpathmoveto{\pgfqpoint{0.000000in}{0.000000in}}%
\pgfpathlineto{\pgfqpoint{0.000000in}{-0.048611in}}%
\pgfusepath{stroke,fill}%
}%
\begin{pgfscope}%
\pgfsys@transformshift{2.407468in}{0.625000in}%
\pgfsys@useobject{currentmarker}{}%
\end{pgfscope}%
\end{pgfscope}%
\begin{pgfscope}%
\definecolor{textcolor}{rgb}{0.000000,0.000000,0.000000}%
\pgfsetstrokecolor{textcolor}%
\pgfsetfillcolor{textcolor}%
\pgftext[x=2.407468in,y=0.527778in,,top]{\color{textcolor}\sffamily\fontsize{16.000000}{19.200000}\selectfont 20}%
\end{pgfscope}%
\begin{pgfscope}%
\pgfsetbuttcap%
\pgfsetroundjoin%
\definecolor{currentfill}{rgb}{0.000000,0.000000,0.000000}%
\pgfsetfillcolor{currentfill}%
\pgfsetlinewidth{0.803000pt}%
\definecolor{currentstroke}{rgb}{0.000000,0.000000,0.000000}%
\pgfsetstrokecolor{currentstroke}%
\pgfsetdash{}{0pt}%
\pgfsys@defobject{currentmarker}{\pgfqpoint{0.000000in}{-0.048611in}}{\pgfqpoint{0.000000in}{0.000000in}}{%
\pgfpathmoveto{\pgfqpoint{0.000000in}{0.000000in}}%
\pgfpathlineto{\pgfqpoint{0.000000in}{-0.048611in}}%
\pgfusepath{stroke,fill}%
}%
\begin{pgfscope}%
\pgfsys@transformshift{3.413961in}{0.625000in}%
\pgfsys@useobject{currentmarker}{}%
\end{pgfscope}%
\end{pgfscope}%
\begin{pgfscope}%
\definecolor{textcolor}{rgb}{0.000000,0.000000,0.000000}%
\pgfsetstrokecolor{textcolor}%
\pgfsetfillcolor{textcolor}%
\pgftext[x=3.413961in,y=0.527778in,,top]{\color{textcolor}\sffamily\fontsize{16.000000}{19.200000}\selectfont 40}%
\end{pgfscope}%
\begin{pgfscope}%
\pgfsetbuttcap%
\pgfsetroundjoin%
\definecolor{currentfill}{rgb}{0.000000,0.000000,0.000000}%
\pgfsetfillcolor{currentfill}%
\pgfsetlinewidth{0.803000pt}%
\definecolor{currentstroke}{rgb}{0.000000,0.000000,0.000000}%
\pgfsetstrokecolor{currentstroke}%
\pgfsetdash{}{0pt}%
\pgfsys@defobject{currentmarker}{\pgfqpoint{0.000000in}{-0.048611in}}{\pgfqpoint{0.000000in}{0.000000in}}{%
\pgfpathmoveto{\pgfqpoint{0.000000in}{0.000000in}}%
\pgfpathlineto{\pgfqpoint{0.000000in}{-0.048611in}}%
\pgfusepath{stroke,fill}%
}%
\begin{pgfscope}%
\pgfsys@transformshift{4.420455in}{0.625000in}%
\pgfsys@useobject{currentmarker}{}%
\end{pgfscope}%
\end{pgfscope}%
\begin{pgfscope}%
\definecolor{textcolor}{rgb}{0.000000,0.000000,0.000000}%
\pgfsetstrokecolor{textcolor}%
\pgfsetfillcolor{textcolor}%
\pgftext[x=4.420455in,y=0.527778in,,top]{\color{textcolor}\sffamily\fontsize{16.000000}{19.200000}\selectfont 60}%
\end{pgfscope}%
\begin{pgfscope}%
\pgfsetbuttcap%
\pgfsetroundjoin%
\definecolor{currentfill}{rgb}{0.000000,0.000000,0.000000}%
\pgfsetfillcolor{currentfill}%
\pgfsetlinewidth{0.803000pt}%
\definecolor{currentstroke}{rgb}{0.000000,0.000000,0.000000}%
\pgfsetstrokecolor{currentstroke}%
\pgfsetdash{}{0pt}%
\pgfsys@defobject{currentmarker}{\pgfqpoint{0.000000in}{-0.048611in}}{\pgfqpoint{0.000000in}{0.000000in}}{%
\pgfpathmoveto{\pgfqpoint{0.000000in}{0.000000in}}%
\pgfpathlineto{\pgfqpoint{0.000000in}{-0.048611in}}%
\pgfusepath{stroke,fill}%
}%
\begin{pgfscope}%
\pgfsys@transformshift{5.426948in}{0.625000in}%
\pgfsys@useobject{currentmarker}{}%
\end{pgfscope}%
\end{pgfscope}%
\begin{pgfscope}%
\definecolor{textcolor}{rgb}{0.000000,0.000000,0.000000}%
\pgfsetstrokecolor{textcolor}%
\pgfsetfillcolor{textcolor}%
\pgftext[x=5.426948in,y=0.527778in,,top]{\color{textcolor}\sffamily\fontsize{16.000000}{19.200000}\selectfont 80}%
\end{pgfscope}%
\begin{pgfscope}%
\pgfsetbuttcap%
\pgfsetroundjoin%
\definecolor{currentfill}{rgb}{0.000000,0.000000,0.000000}%
\pgfsetfillcolor{currentfill}%
\pgfsetlinewidth{0.803000pt}%
\definecolor{currentstroke}{rgb}{0.000000,0.000000,0.000000}%
\pgfsetstrokecolor{currentstroke}%
\pgfsetdash{}{0pt}%
\pgfsys@defobject{currentmarker}{\pgfqpoint{0.000000in}{-0.048611in}}{\pgfqpoint{0.000000in}{0.000000in}}{%
\pgfpathmoveto{\pgfqpoint{0.000000in}{0.000000in}}%
\pgfpathlineto{\pgfqpoint{0.000000in}{-0.048611in}}%
\pgfusepath{stroke,fill}%
}%
\begin{pgfscope}%
\pgfsys@transformshift{6.433442in}{0.625000in}%
\pgfsys@useobject{currentmarker}{}%
\end{pgfscope}%
\end{pgfscope}%
\begin{pgfscope}%
\definecolor{textcolor}{rgb}{0.000000,0.000000,0.000000}%
\pgfsetstrokecolor{textcolor}%
\pgfsetfillcolor{textcolor}%
\pgftext[x=6.433442in,y=0.527778in,,top]{\color{textcolor}\sffamily\fontsize{16.000000}{19.200000}\selectfont 100}%
\end{pgfscope}%
\begin{pgfscope}%
\pgfsetbuttcap%
\pgfsetroundjoin%
\definecolor{currentfill}{rgb}{0.000000,0.000000,0.000000}%
\pgfsetfillcolor{currentfill}%
\pgfsetlinewidth{0.803000pt}%
\definecolor{currentstroke}{rgb}{0.000000,0.000000,0.000000}%
\pgfsetstrokecolor{currentstroke}%
\pgfsetdash{}{0pt}%
\pgfsys@defobject{currentmarker}{\pgfqpoint{0.000000in}{-0.048611in}}{\pgfqpoint{0.000000in}{0.000000in}}{%
\pgfpathmoveto{\pgfqpoint{0.000000in}{0.000000in}}%
\pgfpathlineto{\pgfqpoint{0.000000in}{-0.048611in}}%
\pgfusepath{stroke,fill}%
}%
\begin{pgfscope}%
\pgfsys@transformshift{7.439935in}{0.625000in}%
\pgfsys@useobject{currentmarker}{}%
\end{pgfscope}%
\end{pgfscope}%
\begin{pgfscope}%
\definecolor{textcolor}{rgb}{0.000000,0.000000,0.000000}%
\pgfsetstrokecolor{textcolor}%
\pgfsetfillcolor{textcolor}%
\pgftext[x=7.439935in,y=0.527778in,,top]{\color{textcolor}\sffamily\fontsize{16.000000}{19.200000}\selectfont 120}%
\end{pgfscope}%
\begin{pgfscope}%
\pgfsetbuttcap%
\pgfsetroundjoin%
\definecolor{currentfill}{rgb}{0.000000,0.000000,0.000000}%
\pgfsetfillcolor{currentfill}%
\pgfsetlinewidth{0.803000pt}%
\definecolor{currentstroke}{rgb}{0.000000,0.000000,0.000000}%
\pgfsetstrokecolor{currentstroke}%
\pgfsetdash{}{0pt}%
\pgfsys@defobject{currentmarker}{\pgfqpoint{0.000000in}{-0.048611in}}{\pgfqpoint{0.000000in}{0.000000in}}{%
\pgfpathmoveto{\pgfqpoint{0.000000in}{0.000000in}}%
\pgfpathlineto{\pgfqpoint{0.000000in}{-0.048611in}}%
\pgfusepath{stroke,fill}%
}%
\begin{pgfscope}%
\pgfsys@transformshift{8.446429in}{0.625000in}%
\pgfsys@useobject{currentmarker}{}%
\end{pgfscope}%
\end{pgfscope}%
\begin{pgfscope}%
\definecolor{textcolor}{rgb}{0.000000,0.000000,0.000000}%
\pgfsetstrokecolor{textcolor}%
\pgfsetfillcolor{textcolor}%
\pgftext[x=8.446429in,y=0.527778in,,top]{\color{textcolor}\sffamily\fontsize{16.000000}{19.200000}\selectfont 140}%
\end{pgfscope}%
\begin{pgfscope}%
\definecolor{textcolor}{rgb}{0.000000,0.000000,0.000000}%
\pgfsetstrokecolor{textcolor}%
\pgfsetfillcolor{textcolor}%
\pgftext[x=5.125000in,y=0.257162in,,top]{\color{textcolor}\sffamily\fontsize{20.000000}{24.000000}\selectfont CPUs}%
\end{pgfscope}%
\begin{pgfscope}%
\pgfsetbuttcap%
\pgfsetroundjoin%
\definecolor{currentfill}{rgb}{0.000000,0.000000,0.000000}%
\pgfsetfillcolor{currentfill}%
\pgfsetlinewidth{0.803000pt}%
\definecolor{currentstroke}{rgb}{0.000000,0.000000,0.000000}%
\pgfsetstrokecolor{currentstroke}%
\pgfsetdash{}{0pt}%
\pgfsys@defobject{currentmarker}{\pgfqpoint{-0.048611in}{0.000000in}}{\pgfqpoint{-0.000000in}{0.000000in}}{%
\pgfpathmoveto{\pgfqpoint{-0.000000in}{0.000000in}}%
\pgfpathlineto{\pgfqpoint{-0.048611in}{0.000000in}}%
\pgfusepath{stroke,fill}%
}%
\begin{pgfscope}%
\pgfsys@transformshift{1.250000in}{0.672664in}%
\pgfsys@useobject{currentmarker}{}%
\end{pgfscope}%
\end{pgfscope}%
\begin{pgfscope}%
\definecolor{textcolor}{rgb}{0.000000,0.000000,0.000000}%
\pgfsetstrokecolor{textcolor}%
\pgfsetfillcolor{textcolor}%
\pgftext[x=0.799371in, y=0.588246in, left, base]{\color{textcolor}\sffamily\fontsize{16.000000}{19.200000}\selectfont 0.2}%
\end{pgfscope}%
\begin{pgfscope}%
\pgfsetbuttcap%
\pgfsetroundjoin%
\definecolor{currentfill}{rgb}{0.000000,0.000000,0.000000}%
\pgfsetfillcolor{currentfill}%
\pgfsetlinewidth{0.803000pt}%
\definecolor{currentstroke}{rgb}{0.000000,0.000000,0.000000}%
\pgfsetstrokecolor{currentstroke}%
\pgfsetdash{}{0pt}%
\pgfsys@defobject{currentmarker}{\pgfqpoint{-0.048611in}{0.000000in}}{\pgfqpoint{-0.000000in}{0.000000in}}{%
\pgfpathmoveto{\pgfqpoint{-0.000000in}{0.000000in}}%
\pgfpathlineto{\pgfqpoint{-0.048611in}{0.000000in}}%
\pgfusepath{stroke,fill}%
}%
\begin{pgfscope}%
\pgfsys@transformshift{1.250000in}{1.406692in}%
\pgfsys@useobject{currentmarker}{}%
\end{pgfscope}%
\end{pgfscope}%
\begin{pgfscope}%
\definecolor{textcolor}{rgb}{0.000000,0.000000,0.000000}%
\pgfsetstrokecolor{textcolor}%
\pgfsetfillcolor{textcolor}%
\pgftext[x=0.799371in, y=1.322274in, left, base]{\color{textcolor}\sffamily\fontsize{16.000000}{19.200000}\selectfont 0.4}%
\end{pgfscope}%
\begin{pgfscope}%
\pgfsetbuttcap%
\pgfsetroundjoin%
\definecolor{currentfill}{rgb}{0.000000,0.000000,0.000000}%
\pgfsetfillcolor{currentfill}%
\pgfsetlinewidth{0.803000pt}%
\definecolor{currentstroke}{rgb}{0.000000,0.000000,0.000000}%
\pgfsetstrokecolor{currentstroke}%
\pgfsetdash{}{0pt}%
\pgfsys@defobject{currentmarker}{\pgfqpoint{-0.048611in}{0.000000in}}{\pgfqpoint{-0.000000in}{0.000000in}}{%
\pgfpathmoveto{\pgfqpoint{-0.000000in}{0.000000in}}%
\pgfpathlineto{\pgfqpoint{-0.048611in}{0.000000in}}%
\pgfusepath{stroke,fill}%
}%
\begin{pgfscope}%
\pgfsys@transformshift{1.250000in}{2.140720in}%
\pgfsys@useobject{currentmarker}{}%
\end{pgfscope}%
\end{pgfscope}%
\begin{pgfscope}%
\definecolor{textcolor}{rgb}{0.000000,0.000000,0.000000}%
\pgfsetstrokecolor{textcolor}%
\pgfsetfillcolor{textcolor}%
\pgftext[x=0.799371in, y=2.056301in, left, base]{\color{textcolor}\sffamily\fontsize{16.000000}{19.200000}\selectfont 0.6}%
\end{pgfscope}%
\begin{pgfscope}%
\pgfsetbuttcap%
\pgfsetroundjoin%
\definecolor{currentfill}{rgb}{0.000000,0.000000,0.000000}%
\pgfsetfillcolor{currentfill}%
\pgfsetlinewidth{0.803000pt}%
\definecolor{currentstroke}{rgb}{0.000000,0.000000,0.000000}%
\pgfsetstrokecolor{currentstroke}%
\pgfsetdash{}{0pt}%
\pgfsys@defobject{currentmarker}{\pgfqpoint{-0.048611in}{0.000000in}}{\pgfqpoint{-0.000000in}{0.000000in}}{%
\pgfpathmoveto{\pgfqpoint{-0.000000in}{0.000000in}}%
\pgfpathlineto{\pgfqpoint{-0.048611in}{0.000000in}}%
\pgfusepath{stroke,fill}%
}%
\begin{pgfscope}%
\pgfsys@transformshift{1.250000in}{2.874747in}%
\pgfsys@useobject{currentmarker}{}%
\end{pgfscope}%
\end{pgfscope}%
\begin{pgfscope}%
\definecolor{textcolor}{rgb}{0.000000,0.000000,0.000000}%
\pgfsetstrokecolor{textcolor}%
\pgfsetfillcolor{textcolor}%
\pgftext[x=0.799371in, y=2.790329in, left, base]{\color{textcolor}\sffamily\fontsize{16.000000}{19.200000}\selectfont 0.8}%
\end{pgfscope}%
\begin{pgfscope}%
\pgfsetbuttcap%
\pgfsetroundjoin%
\definecolor{currentfill}{rgb}{0.000000,0.000000,0.000000}%
\pgfsetfillcolor{currentfill}%
\pgfsetlinewidth{0.803000pt}%
\definecolor{currentstroke}{rgb}{0.000000,0.000000,0.000000}%
\pgfsetstrokecolor{currentstroke}%
\pgfsetdash{}{0pt}%
\pgfsys@defobject{currentmarker}{\pgfqpoint{-0.048611in}{0.000000in}}{\pgfqpoint{-0.000000in}{0.000000in}}{%
\pgfpathmoveto{\pgfqpoint{-0.000000in}{0.000000in}}%
\pgfpathlineto{\pgfqpoint{-0.048611in}{0.000000in}}%
\pgfusepath{stroke,fill}%
}%
\begin{pgfscope}%
\pgfsys@transformshift{1.250000in}{3.608775in}%
\pgfsys@useobject{currentmarker}{}%
\end{pgfscope}%
\end{pgfscope}%
\begin{pgfscope}%
\definecolor{textcolor}{rgb}{0.000000,0.000000,0.000000}%
\pgfsetstrokecolor{textcolor}%
\pgfsetfillcolor{textcolor}%
\pgftext[x=0.799371in, y=3.524357in, left, base]{\color{textcolor}\sffamily\fontsize{16.000000}{19.200000}\selectfont 1.0}%
\end{pgfscope}%
\begin{pgfscope}%
\pgfsetbuttcap%
\pgfsetroundjoin%
\definecolor{currentfill}{rgb}{0.000000,0.000000,0.000000}%
\pgfsetfillcolor{currentfill}%
\pgfsetlinewidth{0.803000pt}%
\definecolor{currentstroke}{rgb}{0.000000,0.000000,0.000000}%
\pgfsetstrokecolor{currentstroke}%
\pgfsetdash{}{0pt}%
\pgfsys@defobject{currentmarker}{\pgfqpoint{-0.048611in}{0.000000in}}{\pgfqpoint{-0.000000in}{0.000000in}}{%
\pgfpathmoveto{\pgfqpoint{-0.000000in}{0.000000in}}%
\pgfpathlineto{\pgfqpoint{-0.048611in}{0.000000in}}%
\pgfusepath{stroke,fill}%
}%
\begin{pgfscope}%
\pgfsys@transformshift{1.250000in}{4.342803in}%
\pgfsys@useobject{currentmarker}{}%
\end{pgfscope}%
\end{pgfscope}%
\begin{pgfscope}%
\definecolor{textcolor}{rgb}{0.000000,0.000000,0.000000}%
\pgfsetstrokecolor{textcolor}%
\pgfsetfillcolor{textcolor}%
\pgftext[x=0.799371in, y=4.258385in, left, base]{\color{textcolor}\sffamily\fontsize{16.000000}{19.200000}\selectfont 1.2}%
\end{pgfscope}%
\begin{pgfscope}%
\definecolor{textcolor}{rgb}{0.000000,0.000000,0.000000}%
\pgfsetstrokecolor{textcolor}%
\pgfsetfillcolor{textcolor}%
\pgftext[x=0.743815in,y=2.512500in,,bottom,rotate=90.000000]{\color{textcolor}\sffamily\fontsize{20.000000}{24.000000}\selectfont MLUPS}%
\end{pgfscope}%
\begin{pgfscope}%
\definecolor{textcolor}{rgb}{0.000000,0.000000,0.000000}%
\pgfsetstrokecolor{textcolor}%
\pgfsetfillcolor{textcolor}%
\pgftext[x=1.250000in,y=4.441667in,left,base]{\color{textcolor}\sffamily\fontsize{16.000000}{19.200000}\selectfont 1e8}%
\end{pgfscope}%
\begin{pgfscope}%
\pgfpathrectangle{\pgfqpoint{1.250000in}{0.625000in}}{\pgfqpoint{7.750000in}{3.775000in}}%
\pgfusepath{clip}%
\pgfsetbuttcap%
\pgfsetroundjoin%
\pgfsetlinewidth{1.505625pt}%
\definecolor{currentstroke}{rgb}{0.121569,0.466667,0.705882}%
\pgfsetstrokecolor{currentstroke}%
\pgfsetdash{{5.550000pt}{2.400000pt}}{0.000000pt}%
\pgfpathmoveto{\pgfqpoint{1.602273in}{0.796591in}}%
\pgfpathlineto{\pgfqpoint{2.206169in}{1.990267in}}%
\pgfpathlineto{\pgfqpoint{3.212662in}{2.914425in}}%
\pgfpathlineto{\pgfqpoint{4.621753in}{3.608775in}}%
\pgfusepath{stroke}%
\end{pgfscope}%
\begin{pgfscope}%
\pgfpathrectangle{\pgfqpoint{1.250000in}{0.625000in}}{\pgfqpoint{7.750000in}{3.775000in}}%
\pgfusepath{clip}%
\pgfsetbuttcap%
\pgfsetroundjoin%
\definecolor{currentfill}{rgb}{0.121569,0.466667,0.705882}%
\pgfsetfillcolor{currentfill}%
\pgfsetlinewidth{1.003750pt}%
\definecolor{currentstroke}{rgb}{0.121569,0.466667,0.705882}%
\pgfsetstrokecolor{currentstroke}%
\pgfsetdash{}{0pt}%
\pgfsys@defobject{currentmarker}{\pgfqpoint{-0.041667in}{-0.041667in}}{\pgfqpoint{0.041667in}{0.041667in}}{%
\pgfpathmoveto{\pgfqpoint{0.000000in}{-0.041667in}}%
\pgfpathcurveto{\pgfqpoint{0.011050in}{-0.041667in}}{\pgfqpoint{0.021649in}{-0.037276in}}{\pgfqpoint{0.029463in}{-0.029463in}}%
\pgfpathcurveto{\pgfqpoint{0.037276in}{-0.021649in}}{\pgfqpoint{0.041667in}{-0.011050in}}{\pgfqpoint{0.041667in}{0.000000in}}%
\pgfpathcurveto{\pgfqpoint{0.041667in}{0.011050in}}{\pgfqpoint{0.037276in}{0.021649in}}{\pgfqpoint{0.029463in}{0.029463in}}%
\pgfpathcurveto{\pgfqpoint{0.021649in}{0.037276in}}{\pgfqpoint{0.011050in}{0.041667in}}{\pgfqpoint{0.000000in}{0.041667in}}%
\pgfpathcurveto{\pgfqpoint{-0.011050in}{0.041667in}}{\pgfqpoint{-0.021649in}{0.037276in}}{\pgfqpoint{-0.029463in}{0.029463in}}%
\pgfpathcurveto{\pgfqpoint{-0.037276in}{0.021649in}}{\pgfqpoint{-0.041667in}{0.011050in}}{\pgfqpoint{-0.041667in}{0.000000in}}%
\pgfpathcurveto{\pgfqpoint{-0.041667in}{-0.011050in}}{\pgfqpoint{-0.037276in}{-0.021649in}}{\pgfqpoint{-0.029463in}{-0.029463in}}%
\pgfpathcurveto{\pgfqpoint{-0.021649in}{-0.037276in}}{\pgfqpoint{-0.011050in}{-0.041667in}}{\pgfqpoint{0.000000in}{-0.041667in}}%
\pgfpathlineto{\pgfqpoint{0.000000in}{-0.041667in}}%
\pgfpathclose%
\pgfusepath{stroke,fill}%
}%
\begin{pgfscope}%
\pgfsys@transformshift{1.602273in}{0.796591in}%
\pgfsys@useobject{currentmarker}{}%
\end{pgfscope}%
\begin{pgfscope}%
\pgfsys@transformshift{2.206169in}{1.990267in}%
\pgfsys@useobject{currentmarker}{}%
\end{pgfscope}%
\begin{pgfscope}%
\pgfsys@transformshift{3.212662in}{2.914425in}%
\pgfsys@useobject{currentmarker}{}%
\end{pgfscope}%
\begin{pgfscope}%
\pgfsys@transformshift{4.621753in}{3.608775in}%
\pgfsys@useobject{currentmarker}{}%
\end{pgfscope}%
\end{pgfscope}%
\begin{pgfscope}%
\pgfpathrectangle{\pgfqpoint{1.250000in}{0.625000in}}{\pgfqpoint{7.750000in}{3.775000in}}%
\pgfusepath{clip}%
\pgfsetbuttcap%
\pgfsetroundjoin%
\pgfsetlinewidth{1.505625pt}%
\definecolor{currentstroke}{rgb}{1.000000,0.498039,0.054902}%
\pgfsetstrokecolor{currentstroke}%
\pgfsetdash{{5.550000pt}{2.400000pt}}{0.000000pt}%
\pgfpathmoveto{\pgfqpoint{1.602273in}{0.974099in}}%
\pgfpathlineto{\pgfqpoint{2.206169in}{2.519203in}}%
\pgfpathlineto{\pgfqpoint{3.212662in}{3.209057in}}%
\pgfpathlineto{\pgfqpoint{4.621753in}{3.735332in}}%
\pgfpathlineto{\pgfqpoint{6.433442in}{4.016568in}}%
\pgfpathlineto{\pgfqpoint{8.647727in}{4.228409in}}%
\pgfusepath{stroke}%
\end{pgfscope}%
\begin{pgfscope}%
\pgfpathrectangle{\pgfqpoint{1.250000in}{0.625000in}}{\pgfqpoint{7.750000in}{3.775000in}}%
\pgfusepath{clip}%
\pgfsetbuttcap%
\pgfsetroundjoin%
\definecolor{currentfill}{rgb}{1.000000,0.498039,0.054902}%
\pgfsetfillcolor{currentfill}%
\pgfsetlinewidth{1.003750pt}%
\definecolor{currentstroke}{rgb}{1.000000,0.498039,0.054902}%
\pgfsetstrokecolor{currentstroke}%
\pgfsetdash{}{0pt}%
\pgfsys@defobject{currentmarker}{\pgfqpoint{-0.041667in}{-0.041667in}}{\pgfqpoint{0.041667in}{0.041667in}}{%
\pgfpathmoveto{\pgfqpoint{0.000000in}{-0.041667in}}%
\pgfpathcurveto{\pgfqpoint{0.011050in}{-0.041667in}}{\pgfqpoint{0.021649in}{-0.037276in}}{\pgfqpoint{0.029463in}{-0.029463in}}%
\pgfpathcurveto{\pgfqpoint{0.037276in}{-0.021649in}}{\pgfqpoint{0.041667in}{-0.011050in}}{\pgfqpoint{0.041667in}{0.000000in}}%
\pgfpathcurveto{\pgfqpoint{0.041667in}{0.011050in}}{\pgfqpoint{0.037276in}{0.021649in}}{\pgfqpoint{0.029463in}{0.029463in}}%
\pgfpathcurveto{\pgfqpoint{0.021649in}{0.037276in}}{\pgfqpoint{0.011050in}{0.041667in}}{\pgfqpoint{0.000000in}{0.041667in}}%
\pgfpathcurveto{\pgfqpoint{-0.011050in}{0.041667in}}{\pgfqpoint{-0.021649in}{0.037276in}}{\pgfqpoint{-0.029463in}{0.029463in}}%
\pgfpathcurveto{\pgfqpoint{-0.037276in}{0.021649in}}{\pgfqpoint{-0.041667in}{0.011050in}}{\pgfqpoint{-0.041667in}{0.000000in}}%
\pgfpathcurveto{\pgfqpoint{-0.041667in}{-0.011050in}}{\pgfqpoint{-0.037276in}{-0.021649in}}{\pgfqpoint{-0.029463in}{-0.029463in}}%
\pgfpathcurveto{\pgfqpoint{-0.021649in}{-0.037276in}}{\pgfqpoint{-0.011050in}{-0.041667in}}{\pgfqpoint{0.000000in}{-0.041667in}}%
\pgfpathlineto{\pgfqpoint{0.000000in}{-0.041667in}}%
\pgfpathclose%
\pgfusepath{stroke,fill}%
}%
\begin{pgfscope}%
\pgfsys@transformshift{1.602273in}{0.974099in}%
\pgfsys@useobject{currentmarker}{}%
\end{pgfscope}%
\begin{pgfscope}%
\pgfsys@transformshift{2.206169in}{2.519203in}%
\pgfsys@useobject{currentmarker}{}%
\end{pgfscope}%
\begin{pgfscope}%
\pgfsys@transformshift{3.212662in}{3.209057in}%
\pgfsys@useobject{currentmarker}{}%
\end{pgfscope}%
\begin{pgfscope}%
\pgfsys@transformshift{4.621753in}{3.735332in}%
\pgfsys@useobject{currentmarker}{}%
\end{pgfscope}%
\begin{pgfscope}%
\pgfsys@transformshift{6.433442in}{4.016568in}%
\pgfsys@useobject{currentmarker}{}%
\end{pgfscope}%
\begin{pgfscope}%
\pgfsys@transformshift{8.647727in}{4.228409in}%
\pgfsys@useobject{currentmarker}{}%
\end{pgfscope}%
\end{pgfscope}%
\begin{pgfscope}%
\pgfpathrectangle{\pgfqpoint{1.250000in}{0.625000in}}{\pgfqpoint{7.750000in}{3.775000in}}%
\pgfusepath{clip}%
\pgfsetbuttcap%
\pgfsetroundjoin%
\definecolor{currentfill}{rgb}{1.000000,0.000000,0.000000}%
\pgfsetfillcolor{currentfill}%
\pgfsetlinewidth{1.003750pt}%
\definecolor{currentstroke}{rgb}{1.000000,0.000000,0.000000}%
\pgfsetstrokecolor{currentstroke}%
\pgfsetdash{}{0pt}%
\pgfsys@defobject{currentmarker}{\pgfqpoint{-0.041667in}{-0.041667in}}{\pgfqpoint{0.041667in}{0.041667in}}{%
\pgfpathmoveto{\pgfqpoint{0.000000in}{-0.041667in}}%
\pgfpathcurveto{\pgfqpoint{0.011050in}{-0.041667in}}{\pgfqpoint{0.021649in}{-0.037276in}}{\pgfqpoint{0.029463in}{-0.029463in}}%
\pgfpathcurveto{\pgfqpoint{0.037276in}{-0.021649in}}{\pgfqpoint{0.041667in}{-0.011050in}}{\pgfqpoint{0.041667in}{0.000000in}}%
\pgfpathcurveto{\pgfqpoint{0.041667in}{0.011050in}}{\pgfqpoint{0.037276in}{0.021649in}}{\pgfqpoint{0.029463in}{0.029463in}}%
\pgfpathcurveto{\pgfqpoint{0.021649in}{0.037276in}}{\pgfqpoint{0.011050in}{0.041667in}}{\pgfqpoint{0.000000in}{0.041667in}}%
\pgfpathcurveto{\pgfqpoint{-0.011050in}{0.041667in}}{\pgfqpoint{-0.021649in}{0.037276in}}{\pgfqpoint{-0.029463in}{0.029463in}}%
\pgfpathcurveto{\pgfqpoint{-0.037276in}{0.021649in}}{\pgfqpoint{-0.041667in}{0.011050in}}{\pgfqpoint{-0.041667in}{0.000000in}}%
\pgfpathcurveto{\pgfqpoint{-0.041667in}{-0.011050in}}{\pgfqpoint{-0.037276in}{-0.021649in}}{\pgfqpoint{-0.029463in}{-0.029463in}}%
\pgfpathcurveto{\pgfqpoint{-0.021649in}{-0.037276in}}{\pgfqpoint{-0.011050in}{-0.041667in}}{\pgfqpoint{0.000000in}{-0.041667in}}%
\pgfpathlineto{\pgfqpoint{0.000000in}{-0.041667in}}%
\pgfpathclose%
\pgfusepath{stroke,fill}%
}%
\begin{pgfscope}%
\pgfsys@transformshift{7.691558in}{3.725625in}%
\pgfsys@useobject{currentmarker}{}%
\end{pgfscope}%
\end{pgfscope}%
\begin{pgfscope}%
\pgfsetrectcap%
\pgfsetmiterjoin%
\pgfsetlinewidth{0.803000pt}%
\definecolor{currentstroke}{rgb}{0.000000,0.000000,0.000000}%
\pgfsetstrokecolor{currentstroke}%
\pgfsetdash{}{0pt}%
\pgfpathmoveto{\pgfqpoint{1.250000in}{0.625000in}}%
\pgfpathlineto{\pgfqpoint{1.250000in}{4.400000in}}%
\pgfusepath{stroke}%
\end{pgfscope}%
\begin{pgfscope}%
\pgfsetrectcap%
\pgfsetmiterjoin%
\pgfsetlinewidth{0.803000pt}%
\definecolor{currentstroke}{rgb}{0.000000,0.000000,0.000000}%
\pgfsetstrokecolor{currentstroke}%
\pgfsetdash{}{0pt}%
\pgfpathmoveto{\pgfqpoint{9.000000in}{0.625000in}}%
\pgfpathlineto{\pgfqpoint{9.000000in}{4.400000in}}%
\pgfusepath{stroke}%
\end{pgfscope}%
\begin{pgfscope}%
\pgfsetrectcap%
\pgfsetmiterjoin%
\pgfsetlinewidth{0.803000pt}%
\definecolor{currentstroke}{rgb}{0.000000,0.000000,0.000000}%
\pgfsetstrokecolor{currentstroke}%
\pgfsetdash{}{0pt}%
\pgfpathmoveto{\pgfqpoint{1.250000in}{0.625000in}}%
\pgfpathlineto{\pgfqpoint{9.000000in}{0.625000in}}%
\pgfusepath{stroke}%
\end{pgfscope}%
\begin{pgfscope}%
\pgfsetrectcap%
\pgfsetmiterjoin%
\pgfsetlinewidth{0.803000pt}%
\definecolor{currentstroke}{rgb}{0.000000,0.000000,0.000000}%
\pgfsetstrokecolor{currentstroke}%
\pgfsetdash{}{0pt}%
\pgfpathmoveto{\pgfqpoint{1.250000in}{4.400000in}}%
\pgfpathlineto{\pgfqpoint{9.000000in}{4.400000in}}%
\pgfusepath{stroke}%
\end{pgfscope}%
\begin{pgfscope}%
\definecolor{textcolor}{rgb}{0.000000,0.000000,0.000000}%
\pgfsetstrokecolor{textcolor}%
\pgfsetfillcolor{textcolor}%
\pgftext[x=5.125000in,y=4.483333in,,base]{\color{textcolor}\sffamily\fontsize{20.000000}{24.000000}\selectfont MLUPS with respect to time}%
\end{pgfscope}%
\begin{pgfscope}%
\pgfsetbuttcap%
\pgfsetmiterjoin%
\definecolor{currentfill}{rgb}{1.000000,1.000000,1.000000}%
\pgfsetfillcolor{currentfill}%
\pgfsetfillopacity{0.800000}%
\pgfsetlinewidth{1.003750pt}%
\definecolor{currentstroke}{rgb}{0.800000,0.800000,0.800000}%
\pgfsetstrokecolor{currentstroke}%
\pgfsetstrokeopacity{0.800000}%
\pgfsetdash{}{0pt}%
\pgfpathmoveto{\pgfqpoint{5.563021in}{0.736111in}}%
\pgfpathlineto{\pgfqpoint{8.844444in}{0.736111in}}%
\pgfpathquadraticcurveto{\pgfqpoint{8.888889in}{0.736111in}}{\pgfqpoint{8.888889in}{0.780556in}}%
\pgfpathlineto{\pgfqpoint{8.888889in}{1.736848in}}%
\pgfpathquadraticcurveto{\pgfqpoint{8.888889in}{1.781293in}}{\pgfqpoint{8.844444in}{1.781293in}}%
\pgfpathlineto{\pgfqpoint{5.563021in}{1.781293in}}%
\pgfpathquadraticcurveto{\pgfqpoint{5.518576in}{1.781293in}}{\pgfqpoint{5.518576in}{1.736848in}}%
\pgfpathlineto{\pgfqpoint{5.518576in}{0.780556in}}%
\pgfpathquadraticcurveto{\pgfqpoint{5.518576in}{0.736111in}}{\pgfqpoint{5.563021in}{0.736111in}}%
\pgfpathlineto{\pgfqpoint{5.563021in}{0.736111in}}%
\pgfpathclose%
\pgfusepath{stroke,fill}%
\end{pgfscope}%
\begin{pgfscope}%
\pgfsetbuttcap%
\pgfsetroundjoin%
\pgfsetlinewidth{1.505625pt}%
\definecolor{currentstroke}{rgb}{0.121569,0.466667,0.705882}%
\pgfsetstrokecolor{currentstroke}%
\pgfsetdash{{5.550000pt}{2.400000pt}}{0.000000pt}%
\pgfpathmoveto{\pgfqpoint{5.607465in}{1.601345in}}%
\pgfpathlineto{\pgfqpoint{5.829687in}{1.601345in}}%
\pgfpathlineto{\pgfqpoint{6.051910in}{1.601345in}}%
\pgfusepath{stroke}%
\end{pgfscope}%
\begin{pgfscope}%
\pgfsetbuttcap%
\pgfsetroundjoin%
\definecolor{currentfill}{rgb}{0.121569,0.466667,0.705882}%
\pgfsetfillcolor{currentfill}%
\pgfsetlinewidth{1.003750pt}%
\definecolor{currentstroke}{rgb}{0.121569,0.466667,0.705882}%
\pgfsetstrokecolor{currentstroke}%
\pgfsetdash{}{0pt}%
\pgfsys@defobject{currentmarker}{\pgfqpoint{-0.041667in}{-0.041667in}}{\pgfqpoint{0.041667in}{0.041667in}}{%
\pgfpathmoveto{\pgfqpoint{0.000000in}{-0.041667in}}%
\pgfpathcurveto{\pgfqpoint{0.011050in}{-0.041667in}}{\pgfqpoint{0.021649in}{-0.037276in}}{\pgfqpoint{0.029463in}{-0.029463in}}%
\pgfpathcurveto{\pgfqpoint{0.037276in}{-0.021649in}}{\pgfqpoint{0.041667in}{-0.011050in}}{\pgfqpoint{0.041667in}{0.000000in}}%
\pgfpathcurveto{\pgfqpoint{0.041667in}{0.011050in}}{\pgfqpoint{0.037276in}{0.021649in}}{\pgfqpoint{0.029463in}{0.029463in}}%
\pgfpathcurveto{\pgfqpoint{0.021649in}{0.037276in}}{\pgfqpoint{0.011050in}{0.041667in}}{\pgfqpoint{0.000000in}{0.041667in}}%
\pgfpathcurveto{\pgfqpoint{-0.011050in}{0.041667in}}{\pgfqpoint{-0.021649in}{0.037276in}}{\pgfqpoint{-0.029463in}{0.029463in}}%
\pgfpathcurveto{\pgfqpoint{-0.037276in}{0.021649in}}{\pgfqpoint{-0.041667in}{0.011050in}}{\pgfqpoint{-0.041667in}{0.000000in}}%
\pgfpathcurveto{\pgfqpoint{-0.041667in}{-0.011050in}}{\pgfqpoint{-0.037276in}{-0.021649in}}{\pgfqpoint{-0.029463in}{-0.029463in}}%
\pgfpathcurveto{\pgfqpoint{-0.021649in}{-0.037276in}}{\pgfqpoint{-0.011050in}{-0.041667in}}{\pgfqpoint{0.000000in}{-0.041667in}}%
\pgfpathlineto{\pgfqpoint{0.000000in}{-0.041667in}}%
\pgfpathclose%
\pgfusepath{stroke,fill}%
}%
\begin{pgfscope}%
\pgfsys@transformshift{5.829687in}{1.601345in}%
\pgfsys@useobject{currentmarker}{}%
\end{pgfscope}%
\end{pgfscope}%
\begin{pgfscope}%
\definecolor{textcolor}{rgb}{0.000000,0.000000,0.000000}%
\pgfsetstrokecolor{textcolor}%
\pgfsetfillcolor{textcolor}%
\pgftext[x=6.229687in,y=1.523567in,left,base]{\color{textcolor}\sffamily\fontsize{16.000000}{19.200000}\selectfont nodes=2}%
\end{pgfscope}%
\begin{pgfscope}%
\pgfsetbuttcap%
\pgfsetroundjoin%
\pgfsetlinewidth{1.505625pt}%
\definecolor{currentstroke}{rgb}{1.000000,0.498039,0.054902}%
\pgfsetstrokecolor{currentstroke}%
\pgfsetdash{{5.550000pt}{2.400000pt}}{0.000000pt}%
\pgfpathmoveto{\pgfqpoint{5.607465in}{1.275173in}}%
\pgfpathlineto{\pgfqpoint{5.829687in}{1.275173in}}%
\pgfpathlineto{\pgfqpoint{6.051910in}{1.275173in}}%
\pgfusepath{stroke}%
\end{pgfscope}%
\begin{pgfscope}%
\pgfsetbuttcap%
\pgfsetroundjoin%
\definecolor{currentfill}{rgb}{1.000000,0.498039,0.054902}%
\pgfsetfillcolor{currentfill}%
\pgfsetlinewidth{1.003750pt}%
\definecolor{currentstroke}{rgb}{1.000000,0.498039,0.054902}%
\pgfsetstrokecolor{currentstroke}%
\pgfsetdash{}{0pt}%
\pgfsys@defobject{currentmarker}{\pgfqpoint{-0.041667in}{-0.041667in}}{\pgfqpoint{0.041667in}{0.041667in}}{%
\pgfpathmoveto{\pgfqpoint{0.000000in}{-0.041667in}}%
\pgfpathcurveto{\pgfqpoint{0.011050in}{-0.041667in}}{\pgfqpoint{0.021649in}{-0.037276in}}{\pgfqpoint{0.029463in}{-0.029463in}}%
\pgfpathcurveto{\pgfqpoint{0.037276in}{-0.021649in}}{\pgfqpoint{0.041667in}{-0.011050in}}{\pgfqpoint{0.041667in}{0.000000in}}%
\pgfpathcurveto{\pgfqpoint{0.041667in}{0.011050in}}{\pgfqpoint{0.037276in}{0.021649in}}{\pgfqpoint{0.029463in}{0.029463in}}%
\pgfpathcurveto{\pgfqpoint{0.021649in}{0.037276in}}{\pgfqpoint{0.011050in}{0.041667in}}{\pgfqpoint{0.000000in}{0.041667in}}%
\pgfpathcurveto{\pgfqpoint{-0.011050in}{0.041667in}}{\pgfqpoint{-0.021649in}{0.037276in}}{\pgfqpoint{-0.029463in}{0.029463in}}%
\pgfpathcurveto{\pgfqpoint{-0.037276in}{0.021649in}}{\pgfqpoint{-0.041667in}{0.011050in}}{\pgfqpoint{-0.041667in}{0.000000in}}%
\pgfpathcurveto{\pgfqpoint{-0.041667in}{-0.011050in}}{\pgfqpoint{-0.037276in}{-0.021649in}}{\pgfqpoint{-0.029463in}{-0.029463in}}%
\pgfpathcurveto{\pgfqpoint{-0.021649in}{-0.037276in}}{\pgfqpoint{-0.011050in}{-0.041667in}}{\pgfqpoint{0.000000in}{-0.041667in}}%
\pgfpathlineto{\pgfqpoint{0.000000in}{-0.041667in}}%
\pgfpathclose%
\pgfusepath{stroke,fill}%
}%
\begin{pgfscope}%
\pgfsys@transformshift{5.829687in}{1.275173in}%
\pgfsys@useobject{currentmarker}{}%
\end{pgfscope}%
\end{pgfscope}%
\begin{pgfscope}%
\definecolor{textcolor}{rgb}{0.000000,0.000000,0.000000}%
\pgfsetstrokecolor{textcolor}%
\pgfsetfillcolor{textcolor}%
\pgftext[x=6.229687in,y=1.197395in,left,base]{\color{textcolor}\sffamily\fontsize{16.000000}{19.200000}\selectfont nodes=4}%
\end{pgfscope}%
\begin{pgfscope}%
\pgfsetbuttcap%
\pgfsetroundjoin%
\definecolor{currentfill}{rgb}{1.000000,0.000000,0.000000}%
\pgfsetfillcolor{currentfill}%
\pgfsetlinewidth{1.003750pt}%
\definecolor{currentstroke}{rgb}{1.000000,0.000000,0.000000}%
\pgfsetstrokecolor{currentstroke}%
\pgfsetdash{}{0pt}%
\pgfsys@defobject{currentmarker}{\pgfqpoint{-0.041667in}{-0.041667in}}{\pgfqpoint{0.041667in}{0.041667in}}{%
\pgfpathmoveto{\pgfqpoint{0.000000in}{-0.041667in}}%
\pgfpathcurveto{\pgfqpoint{0.011050in}{-0.041667in}}{\pgfqpoint{0.021649in}{-0.037276in}}{\pgfqpoint{0.029463in}{-0.029463in}}%
\pgfpathcurveto{\pgfqpoint{0.037276in}{-0.021649in}}{\pgfqpoint{0.041667in}{-0.011050in}}{\pgfqpoint{0.041667in}{0.000000in}}%
\pgfpathcurveto{\pgfqpoint{0.041667in}{0.011050in}}{\pgfqpoint{0.037276in}{0.021649in}}{\pgfqpoint{0.029463in}{0.029463in}}%
\pgfpathcurveto{\pgfqpoint{0.021649in}{0.037276in}}{\pgfqpoint{0.011050in}{0.041667in}}{\pgfqpoint{0.000000in}{0.041667in}}%
\pgfpathcurveto{\pgfqpoint{-0.011050in}{0.041667in}}{\pgfqpoint{-0.021649in}{0.037276in}}{\pgfqpoint{-0.029463in}{0.029463in}}%
\pgfpathcurveto{\pgfqpoint{-0.037276in}{0.021649in}}{\pgfqpoint{-0.041667in}{0.011050in}}{\pgfqpoint{-0.041667in}{0.000000in}}%
\pgfpathcurveto{\pgfqpoint{-0.041667in}{-0.011050in}}{\pgfqpoint{-0.037276in}{-0.021649in}}{\pgfqpoint{-0.029463in}{-0.029463in}}%
\pgfpathcurveto{\pgfqpoint{-0.021649in}{-0.037276in}}{\pgfqpoint{-0.011050in}{-0.041667in}}{\pgfqpoint{0.000000in}{-0.041667in}}%
\pgfpathlineto{\pgfqpoint{0.000000in}{-0.041667in}}%
\pgfpathclose%
\pgfusepath{stroke,fill}%
}%
\begin{pgfscope}%
\pgfsys@transformshift{5.829687in}{0.949002in}%
\pgfsys@useobject{currentmarker}{}%
\end{pgfscope}%
\end{pgfscope}%
\begin{pgfscope}%
\definecolor{textcolor}{rgb}{0.000000,0.000000,0.000000}%
\pgfsetstrokecolor{textcolor}%
\pgfsetfillcolor{textcolor}%
\pgftext[x=6.229687in,y=0.871224in,left,base]{\color{textcolor}\sffamily\fontsize{16.000000}{19.200000}\selectfont Tesla V100 SXM2 32GB}%
\end{pgfscope}%
\end{pgfpicture}%
\makeatother%
\endgroup%
}
\caption[Performance per configuration.]{The performance per configuration. The blue line is the performance for 2 nodes, the orange line is the performance for 4 nodes and the red dot is the performance for the GPU. The GPU is marked at 125 CPUs but has instead 125 TPUs.
a) shows the performance in seconds whereas b) shows the performance in MLUPS.}
\label{fig:m7}
\end{figure}



\chapter{Conclusion}
The LBM is an easy to use method to model fluid and gas flows. In this report I first explained the evolution and theoretical background of the LBM. 
Afterwards, I showed and explained important implementations like the streaming and collision functions.
Finally, all relevant milestones were achieved and analysed.

For the future, using MPI on GPUs and TPUs should make the computation even faster.
Using CuPy could make the integration with BwUniCluster quite easy and it would be super cool to try. Unfortunately I was not able to do it, but it could speed up the computation quite significantly.
However, this is given that MPI on GPUs behaves similarly to CPU, which is probably not the case, as GPUs pack hundreds of cores and are not designed to run in parallel. 
This is changing making the future more interesting than ever.



